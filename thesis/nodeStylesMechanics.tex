%\usepackage{tikz,pgf}
\usetikzlibrary{calc}


% fast grunn = {x}{y}{l}{r}
% Tegner en linje med korte streker nedfra. 
% x, y er startpunkter; l er lengde, r er rotasjon
% f.eks \node[fast grunn = {0}{0}{10}{60}]{};
\tikzset{
	fast grunn/.style n args={4}{%
		append after command={\pgfextra{%
			\begin{scope}[shift={(#1,#2)}]
				% Hovedlinje ==================================================
				\coordinate (A) at  (0,0); % Venstre kant
				\coordinate (B) at  (#3,0); % Høyre kant

				\draw[] let \p{A}=(A), \p{B}=(B) in
				{[rotate)=#4] (\p{A}) -- (\p{B})};

				% Korte streker ===============================================
				\foreach \i in {0, .25, ...,#3}{
					\pgfmathparse{\i}
					\draw[] let \p{I}=($ (A) + (\pgfmathresult, 0) $), \p{II} = ($ (A) + (\pgfmathresult, 0) - (.15,.15) $) in
					{[rotate=#4] (\p{I}) to (\p{II})}
				} 
			\end{scope}
			}
		},
	}
}
%
%% Feste
%% feste = {x}{y}{r}
%% Tegne et fast, rotasjonsfritt feste ved x,y rotert til r grader
%% Eksempel: \node[feste={0}{0}{60}] {}
\tikzset{
	feste/.style n args={3}{%
		append after command={\pgfextra{%
			\begin{scope}[shift={(#1,#2)}]
				% Hovedlinje =================================================
				\coordinate (A) at (-.6,0);	% Venstre kant
				\coordinate (B) at (.6,0);	% Høyre kant

				\draw[] let \p{A}=(A), \p(B)=(B) in 
				{[rotate=#3] (\p{A}) -- (\p{B})};

				% Korte streker ==============================================
				\foreach \i in {0,...,12} {
					\pgfmathparse{\i/10}
					\draw[] let \p{I}=($ (A) + (pgfmathresult, 0) $), \p{II}=( $ (A) + (\pgfmathresult, 0) - (.15,.15) $) in
					{[rotate=#3] (\p{I}) to (\p{II})};
				}
			\end{scope}
			},
		}
	}
}

% Fast ledd
% ledd fast={x}{y}{r}
% tegner et ledd der øverste sirkel er sentrert i x, y og rotert r grader
% Eksempel : \node[ledd fast={0}{0}{90}] {};
\tikzset{
	ledd fast/.style n args={3}{%
		append after command={\pgfextra{%
			\begin{scope}[shift={(#1,#2)}]
				% Hovedtriangel =====================================================
				\coordinate (A) at (0,0); % leddpunkt
				\coordinate (B) at (-.5,-.75); % nedre hjørne (venstre)
				\coordinate (C) at (.5,-.75);	% Nedre hjørne (høyre)

				\draw[] let \p{A}=(A),\p{B}=(B),\p{C}=(C) in
				{[rotate=#3] (\p{A}) -- (\p{B}) -- (\p{C}) -- cycle};

				% Lengre bunnlinje ==================================================
				\coordinate (BB) at (-.6,-.75);
				\coordinate (CC) at (.6,-.75);

				\draw[] let \p{BB}=(BB), \p{CC}=(CC) in
				{[rotate=#3] (\p{BB}) -- (\p{CC})};

				% Liten sirkel ======================================================
				\draw[fill=white] (A) circle (.1);
				% Korte streker =====================================================
				\foreach \i in {0,...,12} {
					\pgfmathparse{\i/10}
					\draw[] let \p{I}=($ (BB) + (\pgfmathresult, 0) $), \p{II}=($ (BB) + (\pgfmathresult, 0) - (.15,.15) $) in
					{[rotate=#3] (\p{I}) to (\p{II})};
				}
				\end{scope}
			}
		},
	}
}

% Skyveledd
% ledd skyve={x}{y}{r}
% Tegner et ledd der øverste sirkel er sentrert i x,y og rotert r grader
% Eksempel: \node[ledd skyve{3}{0}{0}] {};
\tikzset{
	ledd skyve/.style n args = {3}{%
		append after command={\pgfextra{%
		\begin{scope}[shift={(#1,#2)}]
			% Hovedtriangel ======================================================
			\coordinate (A) at (0,0); % Leddpunkt
			\coordinate (B) at (-.5,-.75); % Nedre hjørne (venstre)
			\coordinate (C) at (.5,-.75); % Nedre hørne (høyre)

			\draw[] let \p{A}=(A),\p{B}=(B),\p{C}=(C) in
			{[rotate=#3] (\p{A}) -- (\p{B}) -- (\p{C}) -- cycle};

			% Lengre bunnlinje ===================================================
			\coordinate (BB) at (-.6,-.82);
			\coordinate (CC) at (.6,-.82);

			\draw[] let \p{BB}=(BB),\p{CC}=(CC) in 
			{[rotate=#3] (\p{BB}) -- (\p{CC})};

			% Liten sirkel =======================================================
			\draw[fill=white] (A) circle (.1);

			% Korte streker ======================================================
			\foreach \i in {0,...,12} {
				\pgfmathparse{\i/10}
				\draw[] let \p{I}=($ (BB) + (\pgfmathresult, 0) $), \p{II}=($ (BB) + (\pgfmathresult, 0) - (.15,.15) $) in
				{[rotate=#3] (\p{I}) to (\p{II})};
			}
			\end{scope}
			}
		},
	}
}

%\tikzset{
%    name/.append style={
%        /tikz/name path=qrr-node-#1
%    },
%    arrow length/.code={
%        \pgfmathsetlength\qrrArrowLength{#1}
%    },
%    m/.style={
%        arrow length=6*(.5pt+.25\pgflinewidth),
%        to path={
%            \pgfextra{
%                \path[name path=qrr-\the\qrrArrowLineCounter-path] (\tikztostart) -- (\tikztotarget);
%                \path[name intersections={of=qrr-\the\qrrArrowLineCounter-path and qrr-node-\tikztostart}] (intersection-1) coordinate (qrr-\the\qrrArrowLineCounter-start);
%                \path[name intersections={of=qrr-\the\qrrArrowLineCounter-path and qrr-node-\tikztotarget}] (intersection-1) coordinate (qrr-\the\qrrArrowLineCounter-target);
%                \path (\tikztostart) -- ($(qrr-\the\qrrArrowLineCounter-target)!\the\qrrArrowLength!(qrr-\the\qrrArrowLineCounter-start)$) \tikztonodes;
%                \global\advance\qrrArrowLineCounter by 1\relax
%            }
%            (\tikztostart) -- (\tikztotarget)
%        }
%    },
%}
