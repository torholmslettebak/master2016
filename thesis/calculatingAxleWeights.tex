% !TEX root = main.tex
\section{Calculate the axle weights}
\label{section:calculating_axle_weights}
The system setup described in section \ref{system_setup}, gives three different locations for measuring strain and so thus three diffferent influence lines generated by the BWIM program. When calculating the axle weights corresponding to each train, this will give three different estimates of the axle weights. However, since data of actual axle weights of the trains are unknown, estimating correctness of the BWIM can not be done through comparing known and calculated axle weights.


Figures \ref{averaged_filtered_infl_lines} and \ref{averaged_infl_lines} are the influence lines used to calculate the axle weights in the tables \ref{table:axleWeights_filteredStrains_trains_all_sensors} and \ref{table:all_trains_all sensors}. By studying the different influence lines, it is clear that the visible differences between the two variants are minimal. This can also be seen in their respected tables. The influence lines produced by filtered strain seems to produce a influence line of slightly lower magnitude, which when used to calculate axle weights results in slightly different values.
Sensor location clearly distinguishes the influence lines:
\begin{easylist}[itemize]
	& The influence line from sensor 1, the middle section sensor, appear to be a mixture of the influence lines from the other sensors.
	&& sensor 3 is closer to the end of the bridge's first section, and this can clearly be seen through the negative influence after the first 5 meters of the bridge.
	&& sensor 2 has lesser negative magnitude after the first 5 meters. The entry effects of the averaged influence lines also appears to be least significant for this sensor location.
	& sensor 3 is influenced by a larger section of the bridge.
\end{easylist}
% \begin{localsize}{10}
% \begin{table}[h]
\begin{sidewaystable}[htpb]
	% \tabcolsep=2pt
  % \begin{adjustbox}{center, angle=90}
	\resizebox{\linewidth}{!}{%
    \begin{tabularx}{\textheight}{ |l|X|X|X|X|X|X|X|X|X|X|X|X|X|X|X| }
      \toprule
			& \multicolumn{15}{ |c| }{trains and their axle weights for sensors} \\
      \toprule
			& \multicolumn{5}{ |c| }{sensor 1} & \multicolumn{5}{ |c| }{sensor 2} & \multicolumn{5}{ |c| }{sensor 3} \\
      \toprule
      axle & train 3 & train 4 & train 5 & train 6 & train 8 & train 3 & train 4 & train 5 & train 6 & train 8 & train 3 & train 4 & train 5 & train 6 & train 8 \\
      \hline
      1 &   8563   &   10689   &     8578   &    11156   &    10617    &    9006  &     11278    &    9570   &    11195   &    11233   &     8341   &     8763    &    8402   &    9111    &     8532  \\
      \hline
      2 &   9343   &   10379   &     9170   &    10237   &    10284    &    7814  &      8628    &    7491   &     8176   &     8295   &     9581   &     9820    &    9440   &   10945    &    10043  \\
      \hline
      3 &   8709   &   10294   &     8817   &    11353   &    10353    &    9521  &     11446    &    9983   &    11940   &    11668   &     8837   &     8563    &    9203   &    8752    &     8320  \\
      \hline
      4 &   9057   &    9868   &     8451   &    10400   &    10285    &    8214  &      8586    &    7351   &     8336   &     8697   &     9073   &     9626    &    8547   &   10479    &    10262  \\
      \hline
      sum car & 35672   &   41230   &    35016   &    43146   &    41539    &   34555  &     39938    &   34395   &    39647   &    39893   &    35832   &    36772    &   35592   &   39287    &    37157  \\
      \hline
      5 & 13392   &   15615   &    13546   &    15879   &    14865    &   14904  &     18402    &   15379   &    17434   &    17489   &    14064   &    14462    &   14660   &   13515    &    13475  \\
      \hline
      6 & 14581   &   14893   &    13859   &    15985   &    16313    &   13059  &     13336    &   11674   &    13391   &    14079   &    16116   &    15509    &   15121   &   18038    &    17548  \\
      \hline
      7 & 11303   &   15097   &    11479   &    15656   &    14380    &   13238  &     17679    &   13678   &    17332   &    17278   &    12374   &    13561    &   12595   &   13615    &    12636  \\
      \hline
      8 & 14184   &   12962   &    13616   &    13549   &    14350    &   12496  &     10913    &   11196   &    10910   &    12026   &    14788   &    13792    &   13933   &   16003    &    15501  \\
      \hline
      sum loc & 53460   &   58567   &    52500   &    61069   &    59908    &   53697  &     60330    &   51927   &    59067   &    60872   &    57342   &    57324    &   56309   &   61171    &    59160  \\
      \bottomrule
      sum tot & 89132 & 99797 & 87516 & 104215 & 101447 & 88252 & 100268 & 86322 & 98714 & 100765 & 93174 & 94096 & 91901 & 100458 & 96317  \\
      \bottomrule
    \end{tabularx}}
	% \end{adjustbox}
	\caption{Table of axle weights for averaged influence lines, all trains}
	\label{table:all_trains_all sensors}
% \end{table}
\end{sidewaystable}
% \end{localsize}

The axle weights in table \ref{table:axleWeights_filteredStrains_trains_all_sensors} is calculated using the influence lines from figure \ref{averaged_filtered_infl_lines} . These
\begin{sidewaystable}[htpb]
	% \tabcolsep=2pt
	% \begin{adjustbox}{center, angle=90}
	\resizebox{\linewidth}{!}{%
	\begin{tabularx}{\textheight}{ |l|X|X|X|X|X|X|X|X|X|X|X|X|X|X|X| }
		\toprule
		& \multicolumn{15}{ |c| }{trains and their axle weights for sensors} \\
		\toprule
		& \multicolumn{5}{ |c| }{sensor 1} & \multicolumn{5}{ |c| }{sensor 2} & \multicolumn{5}{ |c| }{sensor 3} \\
		\toprule
		axle & train 3 & train 4 & train 5 & train 6 & train 8 & train 3 & train 4 & train 5 & train 6 & train 8 & train 3 & train 4 & train 5 & train 6 & train 8 \\
		\hline
		1 &  8819   &    10971   &     8837   &    11301   &    10858   &     9788   &    12060   &    10353   &    11531   &    11859    &    8042    &    8436    &    8093   &     8916 &	 8241 \\
		\hline
		2 &  9106   &    10086   &     8932   &    10052   &    10046   &     6847   &     7536   &     6423   &     7435   &     7367    &    9835    &   10100    &    9715   &    11083 &	10263 \\
		\hline
		3 &  8940   &    10522   &     9055   &    11488   &    10561   &    10343   &    12185   &    10765   &    12282   &    12338    &    8625    &    8317    &    8994   &     8561 &	 8081 \\
		\hline
		4 &  8822   &     9620   &     8207   &    10252   &    10066   &     7231   &     7577   &     6313   &     7563   &     7740    &    9376    &    9928    &    8867   &    10687 &	10522 \\
		\hline
		sum car & 35687   &    41199   &    35031   &    43093   &    41531   &    34209   &    39358   &    33854   &    38811   &    39304    &   35878    &   36781    &   35669   &    39247 &	37107 \\
		\hline
		5 & 13772   &    16046   &    13938   &    16095   &    15283   &    16131   &    19592   &    16561   &    17926   &    18582    &   13572    &   13946    &   14169   &    13138 &	12950 \\
		\hline
		6 & 14255   &    14494   &    13517   &    15729   &    15949   &    11548   &    11597   &    10066   &    12196   &    12487    &   16510    &   15931    &   15561   &    18235 &	17885 \\
		\hline
		7 & 11696   &    15513   &    11883   &    15874   &    14771   &    14504   &    18803   &    14944   &    17770   &    18271    &   11909    &   13092    &   12127   &    13310 &	12179 \\
		\hline
		8 & 13866   &    12567   &    13280   &    13332   &    13990   &    11051   &     9210   &     9625   &     9748   &    10496    &   15178    &   14230    &   14359   &    16217 &	15850 \\
		\hline
		Sum loc & 53589   &    58620   &    52618   &    61030   &    59993   &    53234   &    59202   &    51196   &    57640   &    59836    &   57169    &   57199    &   56216   &    60900 &	58864 \\
		\hline
		Sum tot & 89276   &    99819   &    87649   &   104123   &   101524   &    87443   &    98560   &    85050   &    96451   &    99140    &   93047    &   93980    &   91885   &   100147 &	95971 \\
		\bottomrule
	\end{tabularx}}
	% \end{adjustbox}
	\caption{Table of axle weights for averaged influence lines, where strains have been filtered, all trains}
	\label{table:axleWeights_filteredStrains_trains_all_sensors}
\end{sidewaystable}
The axle weights calculated for the minimal influence lines is similar to what is shown in the tables \ref{table:axleWeights_filteredStrains_trains_all_sensors} and \ref{table:all_trains_all sensors}. A shorter influence line may still contain dynamic effects, but likely less than longer influence lines. Less because of .....
\begin{sidewaystable}[htpb]
	% \tabcolsep=2pt
	% \begin{adjustbox}{center, angle=90}
	\resizebox{\linewidth}{!}{%
	\begin{tabularx}{\textheight}{ |l|X|X|X|X|X|X|X|X|X|X|X|X|X|X|X| }
		\toprule
		& \multicolumn{15}{ |c| }{trains and their axle weights for sensors} \\
		\toprule
		& \multicolumn{5}{ |c| }{sensor 1} & \multicolumn{5}{ |c| }{sensor 2} & \multicolumn{5}{ |c| }{sensor 3} \\
		\toprule
		axle & train 3 & train 4 & train 5 & train 6 & train 8 & train 3 & train 4 & train 5 & train 6 & train 8 & train 3 & train 4 & train 5 & train 6 & train 8 \\
		\hline
			1 &  8797   &    10984   &     8905   &    11393   &    10746   &     9139    &   11473    &    9802    &   11373   &    11272   &     8458   &     9126   &     8626   &     9632	& 8771 \\
			\hline
			2 &  9431   &    10467   &     9162   &    10726   &    10627   &     7961    &    8553    &    7503    &    8949   &     8582   &     9677   &     9765   &     9427   &    11362 & 10232 \\
			\hline
			3 &  8860   &    10476   &     9080   &    11255   &    10311   &     9479    &   11454    &   10045    &   11886   &    11452   &     8855   &     8918   &     9344   &     9046 & 8444 \\
			\hline
			4 &  8956   &     9718   &     8217   &    10956   &    10475   &     8324    &    8475    &    7317    &    9240   &     9024   &     8845   &     9098   &     8134   &    10820 & 10192 \\
			\hline
			sum car & 36044   &    41645   &    35364   &    44330   &    42159   &    34903    &   39955    &   34667    &   41448   &    40330   &    35835   &    36907   &    35531   & 40860 & 37639 \\
			\hline
			5 & 13598   &    15795   &    13884   &    16239   &    14856   &    15072    &   18646    &   15682    &   17677   &    17494   &    14189   &    14869   &    14916   &    14155 & 13712 \\
			\hline
			6 & 14769   &    15069   &    13863   &    16527   &    16865   &    13304    &   13285    &   11703    &   14454   &    14578   &    16245   &    15406   &    15067   &    18496 & 17837 \\
			\hline
			7 & 10980   &    14773   &    11322   &    15297   &    13793   &    12888    &   17424    &   13476    &   17120   &    16756   &    11988   &    13622   &    12372   &    13889 & 12354 \\
			\hline
			8 & 13984   &    12654   &    13245   &    13941   &    14430   &    12477    &   10492    &   10988    &   11852   &    12197   &    14577   &    13214   &    13550   &    16187 & 15341 \\
			\hline
			sum loc & 53331   &    58291   &    52314   &    62004   &    59944   &    53741    &   59847    &   51849    &   61103   &    61025   &    56999   &    57111   &    55905   &   62727 & 59244 \\
			\hline
			sum tot & 89375   &    99936   &    87678   &   106334   &   102103   &    88644    &   99802    &   86516    &  102551   &   101355   &    92834   &    94018   &    91436   &   103587 & 96883 \\
			\hline
		\end{tabularx}}
	% \end{adjustbox}
	\caption{Table of axle weights for minimal averaged influence lines}
	\label{table:axleWeights_for_minimalInfl}
\end{sidewaystable}
% \sout{As table \ref{table:axleWeights_elongated} shows, there clearly is some error in the calculated axle weights. Especially sensor 2 and 3 which generally gives very low estimates. This trend hold for minimal and extended influence lines as well which means that the sensors have not been calibrated. By looking at the same measurement for a specific axle for one sensor and comparing with the same calculated axle weight for another sensor it is possible to calculate the ratio between them. If this ratio also holds for the other axles, the relationship between sensor 1 and 2 is almost constant which in the authors opinion shows the uncalibrated nature of the sensors.
% If at least one trains axle weights were known, it would be possible to scale the sensor readings to the show the correct values. This would in theory make the table above show the correct results.
% }
\subsection{Accuracy of axle weights}

As seen in tables \ref{table:axleWeights_filteredStrains_trains_all_sensors} and \ref{table:all_trains_all sensors}, there are differences between the calculated axle weights for each sensor. The values for the different axles should be relatively similar for each calculation, but for some of the signals it is clear that values vary with up to 2000 kg which unlikely is explainable by passenger distribution in the train. The axle weights of a bogie should be fairly equal, which the tables are not showing. A more reasonable explanation for these differences from axle to axle could be the placement of the influence line representing the axle, as has been discussed in \ref{section:using_influence_lines}. These are errors which may have one or more reasons.

\begin{table}[htpb]
	\begin{adjustbox}{center}
		\begin{tabularx}{\textwidth}{ |X"X|X|X|X|X| }
			\hline
			\multicolumn{6}{ |c| }{Gross weight train}\\
			\hline
			& train 3 & train 4 & train 5 & train 6 & train 8  \\
			\hline
			sensor 1 &  89132 &	99797	& 87516	& 104215 & 101447  \\
			\hline
			sensor 2 & 88252 & 100268 & 86322 &	98714 &	100765   \\
			\hline
			ratio: & 0.99013	& 1.00472	& 0.98636	& 0.94722 & 0.99328 \\
			\thickhline
			sensor 1 &  89132 &	99797	& 87516	& 104215 & 101447  \\
			\hline
			sensor 3 & 93174 & 94096 & 91901 &	100458 &	96317   \\
			\hline
			ratio:  & 1.04535	& 0.94287	& 1.05011	& 0.96395	& 0.94943	 \\
			\thickhline
			sensor 2 & 88252 & 100268 & 86322 &	98714 &	100765  \\
			\hline
			sensor 3 & 93174 & 94096 & 91901 &	100458 &	96317   \\
			\hline
			ratio: & 1.05577& 0.93845	& 1.06463	& 1.01767	& 0.95586  \\
			\thickhline
		\end{tabularx}
	\end{adjustbox}
	\caption{Ratio table showing the ratio between gross train weight for the different sensors, using values from table \ref{table:all_trains_all sensors}}
	\label{table:gross_ratio}
\end{table}
\begin{table}[htpb]
	\begin{adjustbox}{center}
		\begin{tabularx}{\textwidth}{ |X|X|X|X|X|X| }
			\hline
			\multicolumn{6}{ |c| }{Gross weight train}  \\
			\hline
			& train 3 & train 4 & train 5 & train 6 & train 8  \\
			\thickhline
			sensor 1 & 89375   &   99936   &   87678   &  106334   &   102103  \\
			\hline
			sensor 2 & 88644   &   99802   &   86516   &  102551   &   101355  \\
			\hline
			ratio: & 0.99182 &	0.99866 &	0.98675 &	0.96442 &	0.99267  \\
			\thickhline
			sensor 1 & 89375   &   99936   &   87678   &  106334   &   102103  \\
			\hline
			sensor 3 & 92834   &   94018   &	 91436   &	103587	 &   96883   \\
			\hline
			ratio: & 1.03870 &	0.94078	& 1.04286	& 0.97417	& 0.94888	 \\
			\thickhline
			sensor 2 & 88644 &	99802 &	86516 &	102551	& 101355  \\
			\hline
			sensor 3 & 92834 &	94018 &	91436 &	103587	& 96883  \\
			\hline
			ratio:   & 1.04727	& 0.94205 &	1.05687 &	1.01010 &	0.95588  \\
			\thickhline
		\end{tabularx}
	\end{adjustbox}
	\caption{Ratio table showing the ratio between gross train weight for the different sensors, using values from table \ref{table:axleWeights_for_minimalInfl}}
	\label{table:gross_ratio_minimal}
\end{table}

\begin{table}[h]
	\begin{adjustbox}{center}
		\begin{tabularx}{\textwidth}{ |X|X|X|X|X|X|X| }
			\hline
			\multicolumn{6}{ |c| }{Gross weight train, from filtered signal}  \\
			\hline
			& train 3 & train 4 & train 5 & train 6 & train 8  \\
			\thickhline
			sensor 1 & 89276 & 99819 & 87649 & 104123 & 101524  \\
			\hline
			sensor 2 & 87443 & 98560 & 85050 & 96451 & 99140  \\
			\hline
			ratio:   & 0.97947 & 0.98739 & 0.97035 & 0.92632 & 0.97652 \\
			\thickhline
			sensor 1 & 89276 & 99819 & 87649 & 104123 & 101524  \\
			\hline
			sensor 3 & 93047 & 93980 & 91885 & 100147 & 95971  \\
			\hline
			ratio:   & 1.0422 & 0.94150 & 1.0483 & 0.96181 & 0.94530 \\
			\thickhline
			sensor 2 & 87443 & 98560 & 85050 & 96451 & 99140  \\
			\hline
			sensor 3 & 93047 & 93980 & 91885 & 100147 & 95971  \\
			\hline
			ratio:   & 1.0641 & 0.95353 & 1.0804 & 1.0383 & 0.96804  \\
			\thickhline
		\end{tabularx}
	\end{adjustbox}
	\caption{Ratio table showing the ratio between gross train weight for the different sensors, using values from table \ref{table:axleWeights_filteredStrains_trains_all_sensors}}
	\label{table:gross_ratio_filtered_signal}
\end{table}
The ratio tables, \ref{table:gross_ratio_filtered_signal}, \ref{table:gross_ratio_minimal} and \ref{table:gross_ratio}, highlight the differences and similarities between the influence lines. The ratio between axle weights for the different versions of the influence line differ little from each other, all ratios are within 10 \%  showing that the calculated gross train weights are reasonably constant from sensor to sensor. The tables show that the difference between minimal and standard length influence line are small, while the influence lines calculated from filtered signals have the highest values of difference.
The tables also indicate that sensor location is of significance. Sensor 1 and sensor to produce the most consistent ratio values, where calculated gross vehicle weight is higher for every train compared with the same values from sensor 2. The same can not be said for the comparison of sensow 1 and 3 as well as sensor 2 and 3, where the ratio values vary from over 1 to under one for different trains. These differences can be seen by studying the influence lines for the different sensors. The influence lines for sensor 1 and 2 are visually similar while the influence line for sensor 3 have the lowest peak value but appears to have a wider zone of influence meaning that the sensor is affected more from the other sections of the bridge.

Likley sources of error in calculated axle weights:
\newline
\begin{easylist}[itemize]
	& Wrongly determined train velocity - resulting in incorrect influence lines, and error in alignment of influence lines with strain signal.
	& Peak detected by placement algorithm is wrong.
	& Averaged influence line does not represent the strain signal, that is the axle weights used to calculate the influence lines was not correct and resulted in too high or low magnitude of the influence lines peak.
		& the sensors may not be correctly calibrated, resulting in differences in axle weights from sensor to sensor. This can be controlled by calculating the ration between the same axles for different sensors.
\end{easylist}
