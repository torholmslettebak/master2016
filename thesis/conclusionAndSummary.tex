% !TEX root = main.tex
\chapter{Conclusion and summary}
This thesis is based on a developed BWIM program.

% \section{Problems}
% \begin{easylist}[itemize]
% & Big problem with identifying exactly when train enters and leaves the bridge. This results in guesswork when placing influence line in a coordinate system. Where does the bridge begin and end in the influence line.. The only definite certainty seems to be placing the index of the maximum magnitude of the influence line in the correct position according to the measuring sensor's location.
% & This could be problematic when using the found influence lines
% & These problems have been reduced, now the biggest problem is placing the peak of the influence line as well as possible. Possibly performing a smoothing and then finding position of peak could give a better estimate of sensorloc at influence line.. currently the max value of influence line is placed at sensorloc.
% \end{easylist}


\section{How does the matrix method perform}
The plots showing recreated strains, figure \ref{fig:recreated_strains} and appendix figures \ref{fig:recreated_strains_train4} to \ref{fig:recreated_strains_train8}  show the good accuracy of the matrix method. Given accurate values of train velocity and axle distances the individual influence lines are able to almost exactly recreate the signals. This means that the matrix method by itself is a superior tool, but that it requires high levels of accuracy from the rest of a the BWIM system.

The matrix method runtime depends on the signal length and number of train axles. Relying on the symmetry of the matrices it is possible to only form half the matrix and use the transpose operation to form the full a matrix, which saves computational time.

The problem with the matrix method is that it is subjected to dynamic effects from the bridge. This can be solved by having several calibration runs in both directions and average the resulting influence lines. A train moving at very low velocity would likely also help minimize those dynamic effects and together with several calibration runs this would likely eliminate or make the the dynamic effects negligible.
\section{The placement algorithm}
Section \ref{section:using_influence_lines} shows how the influence lines have been placed according to strain signal. This method has not been controlled properly, but seems to result in satisying accuracy. It does not require any special input from user, and is likely reliable. That is it will not increase the error of future calculation with much.

This method can easily be improved, instead of using only the first peak of the strain signal to align the influence lines all the peaks could be used to identify a even safer placement. Possibly an optimization routine, much like the one used in this thesis to find the trains velocities, could be used to find the best possible placement of the influence lines. One way of performing this optimization of alignment could be:

\begin{easylist}[enumerate]
& Use the placement found by the existing method as an initial guess of placement.
& For each iteration of optimization, the axle weight is calculated and used to recreate the strain signal.
& A mean square error or similar function, is calculated
& The alignment of influence lines which minimizes this error will likely be a good or perfect placement.
\end{easylist}

This way of optimizing placement of influence lines could be used in a similar manner to find the axle distances of the train. That way detection of peaks corresponding to axles would not be necessary, also this method likely would be less subjected to, or even independent of, noise.
However such an algorithm could identify solutions with minimal errors but which do not represent the actual train or signal.

\section{Axle detection}
Jernbaneverket controls the flow of train traffic in Norway, and given live locations of trains the BWIM algorithm might not need to find axle distances. Instead a query of existing systems could provide axle distances for the pasing train. This could in theory eliminate a source of error in BWIM algorithm. Detecting axle peaks correctly likely requires an optimization type algorithm for a bridge of this type. There might also exist other methods of getting good estimates of axle distances, but the method of doing this through peaks in a strain signal is too susceptible to noise and dynamics. The peak method appears to be a good way to identy train bogies, and maybe even the center position of the bogies, which may be acceptible for some BWIM systems.

\section{General summary of the master thesis}
In section \ref{researchObjective} there was a list of the goals of this thesis and project. The developement of the a simple BWIM program can be called a success as result have been verified to work, the system is capable of reading and using any sort of strain signal and identifying peaks corresponding to specific axles or at least bogie. The speed of the trains have been identified using a brute force method, also other methods of finding vehicle speed have been developed and tested with the theoretical beam model. The BWIM system has been implemented with the matrix method enabling the calculation influence lines for a strain signal where the properties of the passing train is known. The influence lines have been the main focus of this master thesis, and efforts went into developing alternatives to the linear matrix method. This was done through optimization, and was successful for the theoretical strain signals produced through the beam model, but more complex bridge structures the optimization proved more subjected to the initial guess of the influence line. With optimization there might also exist more than one satisfying solution satisfying tolerance limits for the routhine. When strain signals from Leirelva was was used to test and further develope the Optimization routing it was found that it required special considerations compared to the matrix method which performed no matter the signal complexities.

When it came to aligning the identified influence lines with the equation system several solutions were tried and tested. Due to noise levels of the signal, this is a source of error and identifying individual peaks for alignment was found possible to do for each signal but no general method was successfully developed which was able to do this for every signal. Therefore a better solution proved to be identification of peaks produced by a bogie. This was done through smoothing the signal to the point where individual axles is not visible,



\section{The main challenges of BWIM}
Hva er hovedutfordringene med BWIM?

\begin{easylist}[itemize]
  & Noise leading to problems with:
  && Axle detection
  && Alignment of peaks to build system for calculating axle weights.
  && something ?
  & Having sufficient calibration vehicles to find a representative influence lines
  & Dynamic effects in influence lines

\end{easylist}
\section{Possible improvements, and suggestions of future work}
Hvordan skal man sette opp et nytt forsøk slik at  man kan verifisere og forbedre fremgangsmåten?
Har du noen forslag til videre arbeid?
\section{}
I konklusjonen (og tidligere i diskusjonen) kan du komme med erfaringene dine med implementering av BWIM algorithmen.
