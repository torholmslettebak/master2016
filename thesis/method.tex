\section{Method}

\subsection{Programming a BWIM system}
Describe shortly how the BWIM system have been programmed.
Keywords: 
\begin{itemize}
\item Beam bridge model
\item Producing a strain history through influence lines
\item Finding the speed of the train
\item Finding Axle distances
\item Solving system for axle weights
\end{itemize}

This master project began by learning how the BWIM-system works, which means programming it. To not make this a too big project this meant building a simple beam model of a bridge in Matlab, and simulate moving loads crossing it. Through the theoretical influence lines of the beam a strain history was then produced for the given set of moving axle loads. This strain history could then be used as a base to build the code for a BWIM system. 
\\
%\begin{figure}
\begin{tikzpicture}
\draw[thick] (0,0) to (5,0);
	\node[ledd fast={0}{0}{0}] (ledd) {};
	\node[ledd skyve={5}{0}{0}] (skyveledd) {};
	%\draw[->] (0,1) to {$\scriptstyle g$} (0,.1); 
	\node (a) at (0,1.5) {};
	\node (b) at (0,.1) {};
	\draw[-open triangle 90] (a) to node[pos=-.4] {$axle 2$} (b);
	\node (c) at (2.5,1.5) {};
	\node (d) at (2.5,.1) {};
	\draw[-open triangle 90] (c) to node[pos=-.4] {$axle 1$} (d);
	\node (e) at (2.5,1) {};
	\node (f) at (3.5,1) {};
	\draw[-open triangle 90] (e) to node[pos=1.2] {$v$} (f);
	\node (g) at (0,.5) {};
	\node (h) at (2.5,.5) {};
	\draw[open triangle 90-open triangle 90] (g) to node[above] {$axle spacing$} (h);
	%\draw (2.5,0) circle [radius=0.1] {sensor};
	%\node[draw,circle] (s) at (2.5,0){};
	\filldraw	
	(2.5,0) circle (2pt) node[align=left,   below] {strain sensor};
\end{tikzpicture}
\captionof{figure}{Beam model for initial BWIM}
%\caption{Beam model for initial BWIM}
%\end{figure}

\subsection{Finding influence lines}
Describe how influence lines have been found from the given strain history from Lerelva Bridge. 
Keywords:
\begin{itemize}
\item Matrix method
\item Optimization
\item Speed
\end{itemize}
\subsubsection{Matrix method}
Describe the matrix method.

\subsubsection{Optimization}
Describe how optimization can be used to find optimal influence lines for the bridge.

\subsection{System setup}
To test the BWIM-program on actual data, we Gunnstein, Daniel set up a BWIM-system to gather strain data from actual train passings. The subject bridge were Lerelva-Bridge in Trondheim, a typical Norwegian steel railway bridge. Three strain gauges, \SI{3}{\mm} \SI{120}{ohms} from HBM, were placed by the support towards Trondheim on the first section of the longitudinal stringer. The sensors were placed with \SI{1}{\m} spacing around the middle of the stringer section. These strain gauges were connected to a National Instruments compactDAQ with module NI 9235 which produced an continuous data flow to a standard laptop. A Kipor generator was brought for power.
% INSERT SYSTEM IMAGE HERE


\subsection{Testing}
Keywords:
\begin{itemize}
\item Comparing calculated strain with measured strain
\end{itemize}