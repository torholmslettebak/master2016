\section{Analysis}

This chapter will describe how the BWIM system performs. What works? Why? How? etc.
The main focus should perhaps be placed on identifying the pros and cons of the matrix method and optimization method. 

Should include:
\begin{itemize}
\item Compare theoretical and calculated influence lines. Also include influence lines found through Abaqus.
\item Check how influence lines found through matrix method and optimization reproduces the strain history
\item Test obtained influence line by running the bwim routine on the hitherto unused freight train. (Depends on getting info about the train). Also Do this test on the other trains.
\end{itemize}


\subsection{Problems}
\begin{itemize}
\item Big problem with identifying exactly when train enters and leaves the bridge. This results in guesswork when placing influence line in a coordinate system. Where does the bridge begin and end in the influence line.. The only definite certainty seems to be placing the index of the maximum magnitude of the influence line in the correct position according to the measuring sensor's location.
\item This could be problematic when using the found influence lines 
\end{itemize}