\section{Analysis}

This chapter will describe how the BWIM system performs. What works? Why? How? etc.
The main focus should perhaps be placed on identifying the pros and cons of the matrix method and optimization method. 

Should include:
\begin{itemize}
\item Compare theoretical and calculated influence lines. Also include influence lines found through Abaqus.
\item Check how influence lines found through matrix method and optimization reproduces the strain history
\item Test obtained influence line by running the bwim routine on the hitherto unused freight train. (Depends on getting info about the train). Also Do this test on the other trains.
\end{itemize}

\subsection{The matrix method}
The matrix method creates an influence line for a specific strain history given a known train with known axle weights and velocity. Thus if the strain signal were recreated given with given parameters, the signal would be a almost exact replica of the measured strain signal, where the differences should originate from sensor noise. The found influence line would however be for this specific train and the passing's dynamic effects on the bridge, which is likely to vary from train to train. Therefore an averaging of a sufficient number of calculated influence lines should reduce or eliminate the dynamic effects from the influence line.

The analysis of the matrix method is based on 4 different train passings, and 3 sensor readings on each passing. The trains in these measurements is of the same type (not entirely shure!!) but the exact weight is not known. The weight of each axle were approximated by distributing carriage and locomotive gross weight. Passengers in the passing trains were not accounted for, and may lead to some deviation from ideal results. 

TODO:
\begin{itemize}
\item Show the found influence lines for some sensors
\item discuss the plots
\item reproduce strain signal, and compare with measured signal
\item show averaged influence line, and perform the same tests
\item show interpolation of this averaged influence line
\item perform the same test with this interpolated influence line
\item the alternative should also be done, interpolate each found influence line and average them, then reproduce the strain signal, and find difference through comparison.
\end{itemize} 

\subsection{Optimized influence lines}
Perform the same procedures as for the matrix method

\subsection{Differences between the methods}
Compare the optimized influence lines and the matrix method influence lines. This should be done in a thorough manner.
\subsection{Problems}
\begin{itemize}
\item Big problem with identifying exactly when train enters and leaves the bridge. This results in guesswork when placing influence line in a coordinate system. Where does the bridge begin and end in the influence line.. The only definite certainty seems to be placing the index of the maximum magnitude of the influence line in the correct position according to the measuring sensor's location.
\item This could be problematic when using the found influence lines 
\item These problems have been reduced, now the biggest problem is placing the peak of the influence line as well as possible. Possibly performing a smoothing and then finding position of peak could give a better estimate of sensorloc at influence line.. currently the max value of influence line is placed at sensorloc.
\end{itemize}

% This file was created by matlab2tikz.
%
%The latest updates can be retrieved from
%  http://www.mathworks.com/matlabcentral/fileexchange/22022-matlab2tikz-matlab2tikz
%where you can also make suggestions and rate matlab2tikz.
%
\definecolor{mycolor1}{rgb}{0.00000,0.44700,0.74100}%
\definecolor{mycolor2}{rgb}{0.85000,0.32500,0.09800}%
%
\begin{tikzpicture}

\begin{axis}[%
width=0.951\figurewidth,
height=\figureheight,
at={(0\figurewidth,0\figureheight)},
scale only axis,
xmin=-60,
xmax=60,
ymin=-2e-09,
ymax=1.2e-08,
axis background/.style={fill=white},
title style={font=\bfseries},
title={Influencelines for 4 trains, middle sensor},
legend style={legend cell align=left,align=left,draw=white!15!black}
]
\addplot [color=mycolor1,solid,forget plot]
  table[row sep=crcr]{%
-47.80724609375	-2.4370529569693e-11\\
-47.786748046875	-6.76412695658214e-11\\
-47.76625	-7.04825924520687e-11\\
-47.745751953125	-3.83282736349824e-11\\
-47.72525390625	-2.76241443340147e-11\\
-47.704755859375	-1.00608535823748e-10\\
-47.6842578125	-7.85422568649416e-11\\
-47.663759765625	-1.65977371633769e-11\\
-47.64326171875	-7.25506485136516e-11\\
-47.622763671875	-4.50116083969703e-11\\
-47.602265625	-7.87973899513719e-11\\
-47.581767578125	3.47683545211701e-11\\
-47.56126953125	-8.58630388578688e-11\\
-47.540771484375	-1.32798980825877e-11\\
-47.5202734375	-1.16735181792551e-10\\
-47.499775390625	-1.39964370152938e-10\\
-47.47927734375	-1.36063957271998e-10\\
-47.458779296875	-8.77792361906882e-11\\
-47.43828125	-1.85467833701983e-10\\
-47.417783203125	-1.06862863344414e-10\\
-47.39728515625	-1.23553555612858e-10\\
-47.376787109375	-1.26641307290729e-10\\
-47.3562890625	-1.51587171796693e-10\\
-47.335791015625	-1.99949146627953e-10\\
-47.31529296875	-2.27479944788244e-10\\
-47.294794921875	-2.10526185116155e-10\\
-47.274296875	-2.23811979637212e-10\\
-47.253798828125	-1.4656932579172e-10\\
-47.23330078125	-1.88224221468316e-10\\
-47.212802734375	-8.07640215896905e-11\\
-47.1923046875	-9.29015807357997e-11\\
-47.171806640625	-1.92774932787596e-11\\
-47.15130859375	-1.83646915994822e-11\\
-47.130810546875	-5.19044743822193e-11\\
-47.1103125	-7.46409089755532e-11\\
-47.089814453125	-1.23853739657532e-11\\
-47.06931640625	-4.77479755196351e-11\\
-47.048818359375	-5.44521074571857e-11\\
-47.0283203125	-2.41290617575423e-11\\
-47.007822265625	-9.02498510558771e-11\\
-46.98732421875	7.14215183875685e-11\\
-46.966826171875	6.75498386012756e-13\\
-46.946328125	9.96514475628178e-11\\
-46.925830078125	1.09079908174467e-10\\
-46.90533203125	2.21934627700921e-10\\
-46.884833984375	1.4700887264218e-10\\
-46.8643359375	1.58715941912081e-10\\
-46.843837890625	1.71152555322278e-10\\
-46.82333984375	1.52359829651946e-10\\
-46.802841796875	1.32578405295047e-10\\
-46.78234375	1.50263623436428e-10\\
-46.761845703125	2.22992185326935e-10\\
-46.74134765625	2.70153391478226e-10\\
-46.720849609375	2.1469947899775e-10\\
-46.7003515625	2.81758095389842e-10\\
-46.679853515625	2.74383331773195e-10\\
-46.65935546875	2.7258605376745e-10\\
-46.638857421875	2.79496242337174e-10\\
-46.618359375	2.08346375006836e-10\\
-46.597861328125	2.7583966937815e-10\\
-46.57736328125	2.73914883309658e-10\\
-46.556865234375	3.05739015983726e-10\\
-46.5363671875	3.55846212504025e-10\\
-46.515869140625	4.39509362008014e-10\\
-46.49537109375	4.25758082752513e-10\\
-46.474873046875	4.40967185232545e-10\\
-46.454375	3.7216412457092e-10\\
-46.433876953125	3.34052926500275e-10\\
-46.41337890625	2.83803529930267e-10\\
-46.392880859375	2.52268957719453e-10\\
-46.3723828125	1.7181662383727e-10\\
-46.351884765625	1.28855479767504e-10\\
-46.33138671875	1.80750585188279e-10\\
-46.310888671875	1.51351669063087e-10\\
-46.290390625	2.45678150334622e-10\\
-46.269892578125	2.31982190848084e-10\\
-46.24939453125	2.44894166855282e-10\\
-46.228896484375	1.76051400084553e-10\\
-46.2083984375	1.52745484473482e-10\\
-46.187900390625	7.19769331901005e-11\\
-46.16740234375	5.40941980367601e-11\\
-46.146904296875	3.16486725849327e-11\\
-46.12640625	-5.6107500297088e-11\\
-46.105908203125	-1.09157776178958e-10\\
-46.08541015625	-8.34315606761191e-11\\
-46.064912109375	-1.75003704575612e-11\\
-46.0444140625	-5.39835180888065e-11\\
-46.023916015625	-4.34590954858783e-11\\
-46.00341796875	-5.83523501914444e-11\\
-45.982919921875	-1.29297113348399e-10\\
-45.962421875	-1.6676389368119e-10\\
-45.941923828125	-1.44182958271439e-10\\
-45.92142578125	-1.20439496912649e-10\\
-45.900927734375	-1.2892540286916e-10\\
-45.8804296875	-1.01554174236211e-10\\
-45.859931640625	-9.7617468481579e-11\\
-45.83943359375	-1.87665459629076e-10\\
-45.818935546875	-6.46840604248638e-11\\
-45.7984375	-6.46215433826738e-11\\
-45.777939453125	-1.78695048235656e-11\\
-45.75744140625	-8.00601697819869e-11\\
-45.736943359375	-1.41617331056308e-10\\
-45.7164453125	-1.65152815597712e-10\\
-45.695947265625	-2.07461317237453e-10\\
-45.67544921875	-2.01781728407862e-10\\
-45.654951171875	-1.09618071835181e-10\\
-45.634453125	-1.82726911212764e-10\\
-45.613955078125	-1.18001245795161e-10\\
-45.59345703125	-6.79153772060643e-11\\
-45.572958984375	-7.11757437741966e-12\\
-45.5524609375	-7.33873348863847e-11\\
-45.531962890625	-5.90016674212978e-11\\
-45.51146484375	-9.33080522069938e-11\\
-45.490966796875	-7.34553024616178e-11\\
-45.47046875	-1.510537059726e-10\\
-45.449970703125	-9.05887491202436e-11\\
-45.42947265625	-1.50284472306019e-10\\
-45.408974609375	-1.23772626704279e-10\\
-45.3884765625	-9.51403030751728e-11\\
-45.367978515625	-1.8622481453095e-10\\
-45.34748046875	-1.81072120112643e-10\\
-45.326982421875	-1.0541715518441e-10\\
-45.306484375	-2.22904102481793e-10\\
-45.285986328125	-1.59476250320706e-10\\
-45.26548828125	-1.56630002223745e-10\\
-45.244990234375	-1.71157241096898e-10\\
-45.2244921875	-1.66365097593583e-10\\
-45.203994140625	-1.64062780596584e-10\\
-45.18349609375	-1.56349182478499e-10\\
-45.162998046875	-1.39374729909613e-10\\
-45.1425	-1.89742318953461e-10\\
-45.122001953125	-1.68236631450443e-10\\
-45.10150390625	-9.65704705358004e-11\\
-45.081005859375	-1.91174123812457e-10\\
-45.0605078125	-2.46878396182673e-11\\
-45.040009765625	-1.22270187485889e-10\\
-45.01951171875	-3.87977299426144e-12\\
-44.999013671875	-1.73987193539183e-10\\
-44.978515625	-1.69700334430718e-10\\
-44.958017578125	-1.85088629460084e-10\\
-44.93751953125	-1.59468140802526e-10\\
-44.917021484375	-2.2042231028506e-10\\
-44.8965234375	-1.12995199104761e-10\\
-44.876025390625	-2.10789998431101e-10\\
-44.85552734375	-8.76504805329426e-11\\
-44.835029296875	-3.82187192024651e-11\\
-44.81453125	-3.05482864944895e-11\\
-44.794033203125	1.68000686776333e-11\\
-44.77353515625	-4.94165401719396e-11\\
-44.753037109375	-1.29589470351348e-11\\
-44.7325390625	-4.24035917476352e-11\\
-44.712041015625	-1.28176871179356e-10\\
-44.69154296875	-8.43421923102766e-11\\
-44.671044921875	-1.14937780696805e-10\\
-44.650546875	-3.0471041768557e-11\\
-44.630048828125	-6.98856382889671e-11\\
-44.60955078125	-5.70764510523242e-11\\
-44.589052734375	-1.12693122342654e-10\\
-44.5685546875	-9.93616506929688e-11\\
-44.548056640625	-9.44395414121417e-11\\
-44.52755859375	-9.72258941171357e-11\\
-44.507060546875	-1.46772625141571e-10\\
-44.4865625	-2.33699573668157e-10\\
-44.466064453125	-9.34003972203334e-11\\
-44.44556640625	-2.46039802393468e-10\\
-44.425068359375	4.27126755965698e-12\\
-44.4045703125	-1.63011305049391e-10\\
-44.384072265625	-7.34188324976548e-11\\
-44.36357421875	-1.59305427924389e-10\\
-44.343076171875	-1.77819533003367e-10\\
-44.322578125	-1.79714981509762e-10\\
-44.302080078125	-2.72134023003957e-10\\
-44.28158203125	-2.72938537888904e-10\\
-44.261083984375	-3.03388313413938e-10\\
-44.2405859375	-2.33206275092864e-10\\
-44.220087890625	-1.75288090943118e-10\\
-44.19958984375	-1.70488479663618e-10\\
-44.179091796875	-9.09146290915577e-11\\
-44.15859375	-1.03135323511149e-10\\
-44.138095703125	-1.22172661001088e-10\\
-44.11759765625	-1.56081986923806e-10\\
-44.097099609375	-1.98023407487492e-10\\
-44.0766015625	-2.39919008866866e-10\\
-44.056103515625	-3.18560589002802e-10\\
-44.03560546875	-3.02990842537391e-10\\
-44.015107421875	-2.78918187952127e-10\\
-43.994609375	-1.70697398629844e-10\\
-43.974111328125	-1.66534380971981e-10\\
-43.95361328125	-6.09628435884459e-11\\
-43.933115234375	-6.9637869126518e-11\\
-43.9126171875	-7.52410762374288e-12\\
-43.892119140625	-8.02192562018398e-11\\
-43.87162109375	-1.44315185760569e-10\\
-43.851123046875	-1.38981570686634e-10\\
-43.830625	-2.34835487599973e-10\\
-43.810126953125	-1.68042379145472e-10\\
-43.78962890625	-1.35855969321293e-10\\
-43.769130859375	5.7075090080041e-13\\
-43.7486328125	-1.8503470145512e-11\\
-43.728134765625	8.64467195996867e-11\\
-43.70763671875	1.00787202114885e-10\\
-43.687138671875	1.16908422623539e-10\\
-43.666640625	1.71506272163839e-10\\
-43.646142578125	1.32735743925704e-10\\
-43.62564453125	8.63688373143256e-11\\
-43.605146484375	2.57114693314531e-11\\
-43.5846484375	-2.73816032874047e-12\\
-43.564150390625	7.5490937808945e-11\\
-43.54365234375	1.32891752286755e-10\\
-43.523154296875	1.82533303417517e-10\\
-43.50265625	2.06770772389348e-10\\
-43.482158203125	2.81100634251441e-10\\
-43.46166015625	3.08297977862894e-10\\
-43.441162109375	2.78436386588062e-10\\
-43.4206640625	2.94568365950699e-10\\
-43.400166015625	1.94874511308606e-10\\
-43.37966796875	1.87450034299306e-10\\
-43.359169921875	2.00613085301587e-10\\
-43.338671875	2.06198649449924e-10\\
-43.318173828125	1.98306043892115e-10\\
-43.29767578125	2.16869062280161e-10\\
-43.277177734375	2.3088982364316e-10\\
-43.2566796875	2.5290633341769e-10\\
-43.236181640625	2.68816703719254e-10\\
-43.21568359375	2.65799174064096e-10\\
-43.195185546875	2.4028730048537e-10\\
-43.1746875	1.84173258358457e-10\\
-43.154189453125	1.66636956134302e-10\\
-43.13369140625	1.67277344626467e-10\\
-43.113193359375	6.1590886288495e-11\\
-43.0926953125	9.73327337426242e-11\\
-43.072197265625	8.15084703932969e-11\\
-43.05169921875	1.62606131777731e-10\\
-43.031201171875	1.08379035999678e-10\\
-43.010703125	1.14180815683521e-10\\
-42.990205078125	3.91864815561887e-11\\
-42.96970703125	1.46081002220508e-10\\
-42.949208984375	1.73944068168479e-11\\
-42.9287109375	4.5571925774103e-11\\
-42.908212890625	-3.76196153490876e-13\\
-42.88771484375	4.35629402214301e-11\\
-42.867216796875	5.67052060158369e-11\\
-42.84671875	4.13493123813221e-11\\
-42.826220703125	7.54871935615261e-12\\
-42.80572265625	9.33175441274131e-11\\
-42.785224609375	1.45681971101093e-11\\
-42.7647265625	1.20339088425569e-10\\
-42.744228515625	9.04768114592204e-11\\
-42.72373046875	8.89286618839739e-11\\
-42.703232421875	1.03925477447614e-10\\
-42.682734375	5.08391860636399e-11\\
-42.662236328125	6.543871447275e-11\\
-42.64173828125	7.53689990009357e-11\\
-42.621240234375	2.82063751043634e-11\\
-42.6007421875	5.58297575512128e-11\\
-42.580244140625	2.15739194839556e-12\\
-42.55974609375	1.2610919948276e-11\\
-42.539248046875	1.35420195272911e-10\\
-42.51875	3.39957741157105e-12\\
-42.498251953125	1.25784078763702e-10\\
-42.47775390625	9.80499872428313e-11\\
-42.457255859375	2.6957331152882e-11\\
-42.4367578125	7.23136684202976e-11\\
-42.416259765625	6.01680008155727e-11\\
-42.39576171875	1.39809933853907e-12\\
-42.375263671875	5.71143024505245e-11\\
-42.354765625	-3.43409202083263e-11\\
-42.334267578125	2.09896716523741e-11\\
-42.31376953125	-2.87462247597367e-11\\
-42.293271484375	-8.91518796747675e-13\\
-42.2727734375	-3.84275750202034e-11\\
-42.252275390625	-5.20943997980093e-11\\
-42.23177734375	-4.92540564323453e-11\\
-42.211279296875	-5.5706163563991e-11\\
-42.19078125	-5.35155124842642e-11\\
-42.170283203125	-2.83890736503841e-11\\
-42.14978515625	-1.03148114022763e-10\\
-42.129287109375	-1.08605589192765e-10\\
-42.1087890625	-1.0812902680934e-10\\
-42.088291015625	1.7722187529294e-11\\
-42.06779296875	-1.26119770102314e-12\\
-42.047294921875	-3.76705009100402e-11\\
-42.026796875	1.06926391923899e-11\\
-42.006298828125	-7.42925435364945e-11\\
-41.98580078125	-5.13703905510765e-11\\
-41.965302734375	-3.25046739199096e-11\\
-41.9448046875	-4.20190866368749e-12\\
-41.924306640625	-1.58658101088326e-10\\
-41.90380859375	4.75681217750274e-11\\
-41.883310546875	-6.6747184501766e-11\\
-41.8628125	8.46665029290415e-11\\
-41.842314453125	-2.35447447951239e-11\\
-41.82181640625	6.76499452743182e-11\\
-41.801318359375	8.30193779385428e-11\\
-41.7808203125	5.66400753425881e-11\\
-41.760322265625	3.33445886559793e-11\\
-41.73982421875	7.58547285317218e-11\\
-41.719326171875	-5.20811917818066e-13\\
-41.698828125	1.79813039143019e-11\\
-41.678330078125	-7.97218355350926e-11\\
-41.65783203125	-1.71498492881036e-12\\
-41.637333984375	4.30316222311074e-11\\
-41.6168359375	1.79853284036646e-11\\
-41.596337890625	1.02288976026185e-10\\
-41.57583984375	7.90099835714483e-11\\
-41.555341796875	1.09755366617674e-10\\
-41.53484375	3.57430270705521e-11\\
-41.514345703125	5.09797893862615e-11\\
-41.49384765625	6.41001100426885e-11\\
-41.473349609375	5.20916779305159e-11\\
-41.4528515625	-4.33839479396288e-11\\
-41.432353515625	6.1883057846072e-11\\
-41.41185546875	2.9930403188496e-11\\
-41.391357421875	8.75789863327408e-11\\
-41.370859375	1.03226518295996e-11\\
-41.350361328125	9.61375134575913e-11\\
-41.32986328125	2.75150975292078e-12\\
-41.309365234375	-1.24915572559449e-10\\
-41.2888671875	-4.17575866966838e-11\\
-41.268369140625	-2.02576675465637e-10\\
-41.24787109375	-1.50380253509055e-10\\
-41.227373046875	-1.89730257515546e-10\\
-41.206875	-7.97710461634385e-11\\
-41.186376953125	-7.08171131077364e-11\\
-41.16587890625	-4.25903933771973e-11\\
-41.145380859375	-7.80892761449904e-12\\
-41.1248828125	-4.46946000914062e-11\\
-41.104384765625	-1.25617467850902e-10\\
-41.08388671875	-1.66789954766832e-10\\
-41.063388671875	-2.47912598965613e-10\\
-41.042890625	-2.63493745114281e-10\\
-41.022392578125	-2.82896400312863e-10\\
-41.00189453125	-3.52791446911245e-10\\
-40.981396484375	-2.9450059268826e-10\\
-40.9608984375	-2.64647396812156e-10\\
-40.940400390625	-3.21536075939006e-10\\
-40.91990234375	-3.36802635411827e-10\\
-40.899404296875	-2.95711988841547e-10\\
-40.87890625	-3.41320973356839e-10\\
-40.858408203125	-3.55923144003857e-10\\
-40.83791015625	-3.25161034026948e-10\\
-40.817412109375	-3.70911263303417e-10\\
-40.7969140625	-4.07704965262258e-10\\
-40.776416015625	-4.32316836471408e-10\\
-40.75591796875	-4.10399255614962e-10\\
-40.735419921875	-3.62887128240707e-10\\
-40.714921875	-4.31972132421921e-10\\
-40.694423828125	-4.03745660512117e-10\\
-40.67392578125	-4.3969756701352e-10\\
-40.653427734375	-5.01210382019698e-10\\
-40.6329296875	-4.24424389015375e-10\\
-40.612431640625	-3.9206320245078e-10\\
-40.59193359375	-4.18081749102949e-10\\
-40.571435546875	-3.82183157403141e-10\\
-40.5509375	-4.03079842989857e-10\\
-40.530439453125	-3.78149580736955e-10\\
-40.50994140625	-4.11677449689774e-10\\
-40.489443359375	-3.03049809743092e-10\\
-40.4689453125	-3.75777156540765e-10\\
-40.448447265625	-2.75633593037605e-10\\
-40.42794921875	-2.95071297716757e-10\\
-40.407451171875	-2.12140161628082e-10\\
-40.386953125	-2.53772938188988e-10\\
-40.366455078125	-3.12261522334417e-10\\
-40.34595703125	-3.24855092604317e-10\\
-40.325458984375	-2.85532505901922e-10\\
-40.3049609375	-3.62693563660592e-10\\
-40.284462890625	-2.71095948331594e-10\\
-40.26396484375	-3.13059533106578e-10\\
-40.243466796875	-1.7424008191346e-10\\
-40.22296875	-3.26289586191517e-10\\
-40.202470703125	-2.32820826917072e-10\\
-40.18197265625	-2.36584242895271e-10\\
-40.161474609375	-2.5356067029202e-10\\
-40.1409765625	-2.64483111797473e-10\\
-40.120478515625	-1.95470718142148e-10\\
-40.09998046875	-3.32677866245061e-10\\
-40.079482421875	-2.53365671693719e-10\\
-40.058984375	-3.11851886815883e-10\\
-40.038486328125	-2.79258280182171e-10\\
-40.01798828125	-2.35231403978292e-10\\
-39.997490234375	-3.78926361986388e-10\\
-39.9769921875	-1.55229338754135e-10\\
-39.956494140625	-2.99302467522859e-10\\
-39.93599609375	-2.6519345638568e-10\\
-39.915498046875	-3.071736670704e-10\\
-39.895	-2.29104825832654e-10\\
-39.874501953125	-4.31940482921724e-10\\
-39.85400390625	-2.22436885320928e-10\\
-39.833505859375	-2.79517333366643e-10\\
-39.8130078125	-2.24130750732498e-10\\
-39.792509765625	-3.33032635995573e-10\\
-39.77201171875	-1.62850608037395e-10\\
-39.751513671875	-2.23820331046041e-10\\
-39.731015625	-2.00734533576307e-10\\
-39.710517578125	-1.44227870592136e-10\\
-39.69001953125	-2.06071801756154e-10\\
-39.669521484375	-1.95836249556777e-10\\
-39.6490234375	-1.18203839682936e-10\\
-39.628525390625	-2.40733771544018e-10\\
-39.60802734375	-1.17208049006203e-10\\
-39.587529296875	-1.3557372941635e-10\\
-39.56703125	-1.54825605041536e-10\\
-39.546533203125	-1.45010173380379e-10\\
-39.52603515625	-1.59993842461513e-10\\
-39.505537109375	-1.61673471374549e-10\\
-39.4850390625	-2.11800170570598e-10\\
-39.464541015625	-1.76435921487929e-10\\
-39.44404296875	-2.33714851042444e-10\\
-39.423544921875	-1.67677295735729e-10\\
-39.403046875	-2.63606005887887e-10\\
-39.382548828125	-1.09659661857093e-10\\
-39.36205078125	-2.4552997113978e-10\\
-39.341552734375	-1.05851944558909e-10\\
-39.3210546875	-2.69934157045365e-10\\
-39.300556640625	-1.40795064053165e-10\\
-39.28005859375	-2.51949845610364e-10\\
-39.259560546875	-1.87427295278074e-10\\
-39.2390625	-2.92081362933309e-10\\
-39.218564453125	-2.23825102088817e-10\\
-39.19806640625	-1.98119320707781e-10\\
-39.177568359375	-2.16523555187096e-10\\
-39.1570703125	-2.72650814014247e-10\\
-39.136572265625	-1.68844305463524e-10\\
-39.11607421875	-3.23355866927318e-10\\
-39.095576171875	-2.84826596388677e-10\\
-39.075078125	-2.17152369276604e-10\\
-39.054580078125	-3.2591998859373e-10\\
-39.03408203125	-4.15541508334375e-10\\
-39.013583984375	-3.76668961893971e-10\\
-38.9930859375	-3.30463414435605e-10\\
-38.972587890625	-3.53906602252982e-10\\
-38.95208984375	-4.26992944754936e-10\\
-38.931591796875	-3.29165363839538e-10\\
-38.91109375	-2.68291440969641e-10\\
-38.890595703125	-2.94868089891041e-10\\
-38.87009765625	-3.0338658642566e-10\\
-38.849599609375	-2.89148678046849e-10\\
-38.8291015625	-2.54350810334438e-10\\
-38.808603515625	-3.67395972880507e-10\\
-38.78810546875	-3.91992065445773e-10\\
-38.767607421875	-2.41001414707364e-10\\
-38.747109375	-2.77490744883141e-10\\
-38.726611328125	-1.88497229204881e-10\\
-38.70611328125	-1.45173420263548e-10\\
-38.685615234375	-1.333760998605e-10\\
-38.6651171875	-1.14081345542005e-10\\
-38.644619140625	-1.24482749625964e-10\\
-38.62412109375	-9.46822070124371e-11\\
-38.603623046875	-1.42529562991275e-10\\
-38.583125	-1.45058746679459e-10\\
-38.562626953125	-4.71287670045022e-11\\
-38.54212890625	-5.20207274483355e-11\\
-38.521630859375	4.09046061365687e-11\\
-38.5011328125	1.03973183844875e-10\\
-38.480634765625	4.8348337849198e-12\\
-38.46013671875	1.03130033143878e-11\\
-38.439638671875	3.43873234409072e-11\\
-38.419140625	3.82894236308859e-11\\
-38.398642578125	3.61975536991627e-11\\
-38.37814453125	4.62878889040457e-11\\
-38.357646484375	-6.41379367530794e-12\\
-38.3371484375	9.3921850926914e-11\\
-38.316650390625	4.52495240027909e-11\\
-38.29615234375	1.49009344884071e-10\\
-38.275654296875	1.30867716383913e-10\\
-38.25515625	1.12657448154001e-10\\
-38.234658203125	7.03260423884601e-11\\
-38.21416015625	1.7494580458364e-10\\
-38.193662109375	9.9627684440805e-11\\
-38.1731640625	1.73000321616672e-10\\
-38.152666015625	1.62412321808815e-10\\
-38.13216796875	2.00038026146195e-10\\
-38.111669921875	2.73716424475181e-10\\
-38.091171875	1.59200418596043e-10\\
-38.070673828125	2.00447051638324e-10\\
-38.05017578125	2.46203050656431e-10\\
-38.029677734375	1.95704788119271e-10\\
-38.0091796875	2.76695474769094e-10\\
-37.988681640625	2.18047552143221e-10\\
-37.96818359375	2.03887597507386e-10\\
-37.947685546875	1.91039321568452e-10\\
-37.9271875	2.01059618406936e-10\\
-37.906689453125	2.46010398766639e-10\\
-37.88619140625	1.40099572281501e-10\\
-37.865693359375	7.43503627724523e-11\\
-37.8451953125	8.19113790421749e-11\\
-37.824697265625	2.94607871421883e-11\\
-37.80419921875	1.51391714390138e-10\\
-37.783701171875	2.28626569909503e-11\\
-37.763203125	2.20947119880331e-10\\
-37.742705078125	5.89082437733604e-11\\
-37.72220703125	1.70222704926209e-10\\
-37.701708984375	6.43257177504482e-11\\
-37.6812109375	9.38790831845578e-11\\
-37.660712890625	-1.12665107289064e-11\\
-37.64021484375	4.79102567554143e-11\\
-37.619716796875	3.69467692633174e-11\\
-37.59921875	3.8325384757282e-12\\
-37.578720703125	6.72817370958142e-11\\
-37.55822265625	6.91338882450926e-11\\
-37.537724609375	1.13804584050746e-10\\
-37.5172265625	1.00262224977272e-10\\
-37.496728515625	1.06231211609237e-10\\
-37.47623046875	9.05038699209431e-11\\
-37.455732421875	8.62860422731344e-11\\
-37.435234375	-7.32388560086504e-11\\
-37.414736328125	7.18150019763922e-11\\
-37.39423828125	3.82488202463548e-11\\
-37.373740234375	1.82615888533702e-10\\
-37.3532421875	7.97574641086924e-11\\
-37.332744140625	2.44497498147003e-10\\
-37.31224609375	4.83904764405651e-11\\
-37.291748046875	1.85358296499498e-10\\
-37.27125	-5.84344410718485e-11\\
-37.250751953125	7.79092009841347e-11\\
-37.23025390625	-1.21296954765195e-10\\
-37.209755859375	-7.06471257683685e-11\\
-37.1892578125	-1.31013336558729e-10\\
-37.168759765625	-1.84557624472321e-10\\
-37.14826171875	-4.9320257971808e-11\\
-37.127763671875	-2.72558161937921e-11\\
-37.107265625	-3.57768744289603e-11\\
-37.086767578125	-5.82211246964638e-11\\
-37.06626953125	-2.63002285144141e-11\\
-37.045771484375	-8.85701180598897e-11\\
-37.0252734375	-1.23872643144257e-10\\
-37.004775390625	-9.43784941405879e-11\\
-36.98427734375	-1.10110662572698e-10\\
-36.963779296875	-1.19727470092674e-10\\
-36.94328125	-8.3883044125967e-11\\
-36.922783203125	2.12514487475159e-12\\
-36.90228515625	4.09425612094573e-11\\
-36.881787109375	2.36991738062023e-11\\
-36.8612890625	5.8443253141339e-11\\
-36.840791015625	-6.15344297602922e-11\\
-36.82029296875	1.37824706232309e-10\\
-36.799794921875	-1.01979772590287e-10\\
-36.779296875	3.2159058768948e-11\\
-36.758798828125	-1.23334065489439e-10\\
-36.73830078125	3.71828071017887e-11\\
-36.717802734375	1.44886861057736e-11\\
-36.6973046875	8.72153538970241e-12\\
-36.676806640625	1.0620092983637e-10\\
-36.65630859375	1.4972639776455e-10\\
-36.635810546875	6.49241859726128e-11\\
-36.6153125	2.24434346369325e-10\\
-36.594814453125	1.0084935212056e-10\\
-36.57431640625	1.3422009638113e-10\\
-36.553818359375	6.29185357731133e-11\\
-36.5333203125	1.4246829813351e-10\\
-36.512822265625	5.52756820168662e-11\\
-36.49232421875	1.48478601153349e-10\\
-36.471826171875	1.29929756013831e-10\\
-36.451328125	2.68652546861897e-10\\
-36.430830078125	1.63235664867162e-10\\
-36.41033203125	1.87650593474263e-10\\
-36.389833984375	1.82276762943648e-11\\
-36.3693359375	1.19655572552039e-10\\
-36.348837890625	2.14247363037773e-11\\
-36.32833984375	7.59518321805791e-11\\
-36.307841796875	3.42135354380228e-11\\
-36.28734375	-3.89194636686716e-11\\
-36.266845703125	3.4658850744576e-11\\
-36.24634765625	6.30639790727913e-11\\
-36.225849609375	8.57933086036211e-12\\
-36.2053515625	2.4597237955873e-11\\
-36.184853515625	-5.45482939957753e-11\\
-36.16435546875	-3.05470128904018e-11\\
-36.143857421875	-4.36426381846529e-12\\
-36.123359375	-1.21167789022453e-10\\
-36.102861328125	-7.4739903021147e-11\\
-36.08236328125	-1.78838283627487e-10\\
-36.061865234375	-1.31358210136683e-10\\
-36.0413671875	-1.3264847204389e-10\\
-36.020869140625	-1.85368525708284e-10\\
-36.00037109375	-1.9375636008274e-10\\
-35.979873046875	-1.89446413158316e-10\\
-35.959375	-3.13946865600642e-10\\
-35.938876953125	-2.10746681090873e-10\\
-35.91837890625	-3.04570619071644e-10\\
-35.897880859375	-2.60110681163022e-10\\
-35.8773828125	-2.4597744465921e-10\\
-35.856884765625	-2.27142467602337e-10\\
-35.83638671875	-2.72160203091535e-10\\
-35.815888671875	-1.76843962197602e-10\\
-35.795390625	-2.19242361712804e-10\\
-35.774892578125	-1.45054786593112e-10\\
-35.75439453125	-3.15904488040773e-10\\
-35.733896484375	-2.22447342167835e-10\\
-35.7133984375	-2.64372481762769e-10\\
-35.692900390625	-2.34569401662897e-10\\
-35.67240234375	-2.44030415911239e-10\\
-35.651904296875	-1.38751304165167e-10\\
-35.63140625	-2.41352971979466e-10\\
-35.610908203125	-2.82729133405883e-10\\
-35.59041015625	-1.97052494142853e-10\\
-35.569912109375	-2.65861880468054e-10\\
-35.5494140625	-2.35757909917689e-10\\
-35.528916015625	-2.9407938006626e-10\\
-35.50841796875	-3.68240998759433e-10\\
-35.487919921875	-2.87991961853032e-10\\
-35.467421875	-3.22129357013581e-10\\
-35.446923828125	-2.96561671131772e-10\\
-35.42642578125	-3.46221187438228e-10\\
-35.405927734375	-3.47725828853597e-10\\
-35.3854296875	-3.30027573591555e-10\\
-35.364931640625	-1.73731284123408e-10\\
-35.34443359375	-2.22448461898812e-10\\
-35.323935546875	-1.36968883853764e-10\\
-35.3034375	-2.27281404675594e-10\\
-35.282939453125	-1.36813509408626e-10\\
-35.26244140625	-2.77858960865132e-10\\
-35.241943359375	-8.80690603537006e-11\\
-35.2214453125	-2.65461955741103e-10\\
-35.200947265625	-1.68241019058186e-10\\
-35.18044921875	-3.22660094506424e-10\\
-35.159951171875	-2.29349846743533e-10\\
-35.139453125	-3.1998767157558e-10\\
-35.118955078125	-2.1322181320088e-10\\
-35.09845703125	-2.84919654364422e-10\\
-35.077958984375	-1.54620530829652e-10\\
-35.0574609375	-1.87558125755792e-10\\
-35.036962890625	-1.68784672592821e-10\\
-35.01646484375	-1.67637237930317e-10\\
-34.995966796875	-2.29276656316161e-10\\
-34.97546875	-2.05317415452383e-10\\
-34.954970703125	-2.08408669519618e-10\\
-34.93447265625	-1.56402916505215e-10\\
-34.913974609375	-2.72481233198379e-10\\
-34.8934765625	-1.90715752166322e-10\\
-34.872978515625	-2.64549420315457e-10\\
-34.85248046875	-1.10054999354712e-10\\
-34.831982421875	-1.89159896782515e-10\\
-34.811484375	-1.15919103174153e-10\\
-34.790986328125	-8.02725039962911e-11\\
-34.77048828125	3.35815758695839e-12\\
-34.749990234375	-3.80005475644437e-11\\
-34.7294921875	1.17638321855201e-11\\
-34.708994140625	1.05857647792475e-11\\
-34.68849609375	1.01696273460711e-10\\
-34.667998046875	-3.60314018279949e-11\\
-34.6475	4.6942006524457e-11\\
-34.627001953125	1.12810805637335e-10\\
-34.60650390625	9.01424204601564e-11\\
-34.586005859375	1.27445728900908e-10\\
-34.5655078125	7.1401911615486e-11\\
-34.545009765625	1.80761493287755e-10\\
-34.52451171875	1.82046135889398e-10\\
-34.504013671875	2.43821180068143e-10\\
-34.483515625	2.3291571031761e-10\\
-34.463017578125	1.91613471082961e-10\\
-34.44251953125	2.10283566342064e-10\\
-34.422021484375	1.87473827512556e-10\\
-34.4015234375	1.40351325728883e-10\\
-34.381025390625	2.38483463547964e-10\\
-34.36052734375	9.85793181454612e-11\\
-34.340029296875	2.2167474569825e-10\\
-34.31953125	8.21320997704577e-11\\
-34.299033203125	1.69674901260042e-10\\
-34.27853515625	1.31037196264715e-12\\
-34.258037109375	4.47829753485292e-11\\
-34.2375390625	-1.17343873801704e-10\\
-34.217041015625	-5.31599541020777e-11\\
-34.19654296875	-1.29266027056522e-10\\
-34.176044921875	-1.23763339429968e-11\\
-34.155546875	-4.45441571997538e-11\\
-34.135048828125	-1.43784132530117e-10\\
-34.11455078125	-1.12372965999298e-10\\
-34.094052734375	-1.9069720860477e-11\\
-34.0735546875	-1.91233040540657e-10\\
-34.053056640625	-1.43948441408393e-10\\
-34.03255859375	-3.0371164921538e-10\\
-34.012060546875	-2.33192413387574e-10\\
-33.9915625	-2.17612319505944e-10\\
-33.971064453125	-2.30030716728256e-10\\
-33.95056640625	-2.11091935515535e-10\\
-33.930068359375	-1.63547356198711e-10\\
-33.9095703125	-2.00035722989366e-10\\
-33.889072265625	1.79010192067565e-11\\
-33.86857421875	-8.44236816717882e-11\\
-33.848076171875	2.41548725372697e-11\\
-33.827578125	-1.45120604769899e-10\\
-33.807080078125	-9.38109569454367e-11\\
-33.78658203125	-8.66613773231265e-11\\
-33.766083984375	-1.29930986535239e-10\\
-33.7455859375	-3.3293992826245e-11\\
-33.725087890625	1.32564627487529e-11\\
-33.70458984375	1.15287461193382e-11\\
-33.684091796875	3.16895539212338e-11\\
-33.66359375	6.58842948530784e-11\\
-33.643095703125	7.20805593406007e-11\\
-33.62259765625	1.80442932618352e-11\\
-33.602099609375	-1.17266517527784e-10\\
-33.5816015625	-2.05172145649472e-11\\
-33.561103515625	1.50813802969261e-11\\
-33.54060546875	3.48016281098012e-11\\
-33.520107421875	1.1838611679433e-10\\
-33.499609375	1.92521177868276e-10\\
-33.479111328125	2.46483411518665e-10\\
-33.45861328125	2.49863887876155e-10\\
-33.438115234375	2.64599990303967e-10\\
-33.4176171875	3.92123099158468e-10\\
-33.397119140625	2.3335980525426e-10\\
-33.37662109375	2.58425101183077e-10\\
-33.356123046875	1.78073847927061e-10\\
-33.335625	2.25166699017857e-10\\
-33.315126953125	1.36589852197965e-10\\
-33.29462890625	1.51322890307234e-10\\
-33.274130859375	1.06982456971427e-10\\
-33.2536328125	2.13218561578295e-10\\
-33.233134765625	2.0188783719264e-10\\
-33.21263671875	2.84470907312474e-10\\
-33.192138671875	2.35904334382016e-10\\
-33.171640625	2.28917352151958e-10\\
-33.151142578125	1.68237623609119e-10\\
-33.13064453125	1.7669263054999e-10\\
-33.110146484375	1.69203201801327e-10\\
-33.0896484375	1.76351573602808e-10\\
-33.069150390625	1.51192198750019e-10\\
-33.04865234375	2.32551563648392e-10\\
-33.028154296875	2.18804460550957e-10\\
-33.00765625	2.47626317775721e-10\\
-32.987158203125	2.76949779597633e-10\\
-32.96666015625	2.86177238469747e-10\\
-32.946162109375	1.67783366967889e-10\\
-32.9256640625	1.85294298971709e-10\\
-32.905166015625	1.21145439813678e-10\\
-32.88466796875	2.04202406645452e-10\\
-32.864169921875	1.06394505446597e-10\\
-32.843671875	2.43256343494496e-10\\
-32.823173828125	1.94688989477171e-10\\
-32.80267578125	2.5065773713656e-10\\
-32.782177734375	1.50183773399719e-10\\
-32.7616796875	2.14929264756746e-10\\
-32.741181640625	1.25886972029346e-10\\
-32.72068359375	1.91857563362923e-10\\
-32.700185546875	9.39655711516629e-11\\
-32.6796875	2.21548225385032e-10\\
-32.659189453125	8.10567332998843e-11\\
-32.63869140625	1.97828269279171e-10\\
-32.618193359375	1.09768444992149e-10\\
-32.5976953125	2.83641474833084e-10\\
-32.577197265625	1.36833704563123e-10\\
-32.55669921875	2.66470369785791e-10\\
-32.536201171875	7.63067005612726e-11\\
-32.515703125	1.34500045611437e-10\\
-32.495205078125	8.53008970778196e-11\\
-32.47470703125	3.88868729878994e-11\\
-32.454208984375	8.16072469725124e-11\\
-32.4337109375	2.20365764847535e-11\\
-32.413212890625	7.80654326920425e-11\\
-32.39271484375	-3.11572156874166e-11\\
-32.372216796875	1.70934279069009e-10\\
-32.35171875	-2.85030334222833e-12\\
-32.331220703125	1.54271922417097e-10\\
-32.31072265625	-6.1201312581547e-11\\
-32.290224609375	2.3064791997877e-11\\
-32.2697265625	-1.28691780956708e-10\\
-32.249228515625	-7.44026541461509e-11\\
-32.22873046875	-2.33946413779661e-10\\
-32.208232421875	-2.00365224654425e-10\\
-32.187734375	-2.07012084400349e-10\\
-32.167236328125	-3.080650710617e-10\\
-32.14673828125	-3.29232931622168e-10\\
-32.126240234375	-3.05024166107629e-10\\
-32.1057421875	-3.18057421924294e-10\\
-32.085244140625	-3.62489640052125e-10\\
-32.06474609375	-2.9339045105764e-10\\
-32.044248046875	-4.23415611878774e-10\\
-32.02375	-4.23715313243359e-10\\
-32.003251953125	-5.06210503935127e-10\\
-31.98275390625	-5.30546865275643e-10\\
-31.962255859375	-5.47588901843845e-10\\
-31.9417578125	-5.81019816827194e-10\\
-31.921259765625	-5.94841189445573e-10\\
-31.90076171875	-5.59996909621928e-10\\
-31.880263671875	-6.43438408633815e-10\\
-31.859765625	-4.92462302341816e-10\\
-31.839267578125	-6.16928984391504e-10\\
-31.81876953125	-3.77264983205627e-10\\
-31.798271484375	-4.91722395618532e-10\\
-31.7777734375	-2.95471999425622e-10\\
-31.757275390625	-4.22805014321519e-10\\
-31.73677734375	-2.49312334246648e-10\\
-31.716279296875	-3.24782416030526e-10\\
-31.69578125	-1.72536215398118e-10\\
-31.675283203125	-3.59465194175287e-10\\
-31.65478515625	-2.52611592345658e-10\\
-31.634287109375	-2.94052545272821e-10\\
-31.6137890625	-1.99942513210538e-10\\
-31.593291015625	-2.78231513663996e-10\\
-31.57279296875	-1.37055977445848e-10\\
-31.552294921875	-1.90398891933571e-10\\
-31.531796875	-1.49192812745465e-10\\
-31.511298828125	-1.47733674267541e-10\\
-31.49080078125	-3.29329519324896e-11\\
-31.470302734375	-1.84050936607226e-10\\
-31.4498046875	-1.12259430589828e-10\\
-31.429306640625	-1.3416761171456e-10\\
-31.40880859375	-1.63093687549817e-10\\
-31.388310546875	-2.37206906105782e-10\\
-31.3678125	-1.20792052370815e-10\\
-31.347314453125	-3.08392290891962e-10\\
-31.32681640625	-1.56646901572916e-10\\
-31.306318359375	-2.017665093444e-10\\
-31.2858203125	-1.70729098008204e-10\\
-31.265322265625	-1.7784208648306e-10\\
-31.24482421875	-2.13179302761114e-10\\
-31.224326171875	-2.58112018997306e-10\\
-31.203828125	-2.86220653895347e-10\\
-31.183330078125	-3.92372042033445e-10\\
-31.16283203125	-3.67306136980365e-10\\
-31.142333984375	-3.33641411888999e-10\\
-31.1218359375	-2.60147882309022e-10\\
-31.101337890625	-2.40356523147208e-10\\
-31.08083984375	-2.9638084576793e-10\\
-31.060341796875	-2.4627556299923e-10\\
-31.03984375	-3.5973443637887e-10\\
-31.019345703125	-2.9912853213741e-10\\
-30.99884765625	-3.2280447813404e-10\\
-30.978349609375	-4.23604506056376e-10\\
-30.9578515625	-4.44236035864367e-10\\
-30.937353515625	-5.78690558472394e-10\\
-30.91685546875	-5.61386011418471e-10\\
-30.896357421875	-5.4310057456992e-10\\
-30.875859375	-6.41766084862422e-10\\
-30.855361328125	-4.65045147091396e-10\\
-30.83486328125	-4.44513415492343e-10\\
-30.814365234375	-3.92473271732625e-10\\
-30.7938671875	-4.09989779841481e-10\\
-30.773369140625	-3.74040894078563e-10\\
-30.75287109375	-4.53415752285516e-10\\
-30.732373046875	-3.66659183530578e-10\\
-30.711875	-5.33251310745355e-10\\
-30.691376953125	-5.39351115960253e-10\\
-30.67087890625	-6.01175376546948e-10\\
-30.650380859375	-5.97111200560554e-10\\
-30.6298828125	-5.47716573853273e-10\\
-30.609384765625	-5.23210297195809e-10\\
-30.58888671875	-5.1387919115259e-10\\
-30.568388671875	-4.52918471821396e-10\\
-30.547890625	-4.61064392564758e-10\\
-30.527392578125	-4.16204047868874e-10\\
-30.50689453125	-4.51602654039987e-10\\
-30.486396484375	-4.83309350291477e-10\\
-30.4658984375	-5.40480676181324e-10\\
-30.445400390625	-6.12402918024846e-10\\
-30.42490234375	-6.90565513615334e-10\\
-30.404404296875	-5.87917001825054e-10\\
-30.38390625	-6.85396890488229e-10\\
-30.363408203125	-6.11096519873681e-10\\
-30.34291015625	-6.60744014741394e-10\\
-30.322412109375	-5.41745360567078e-10\\
-30.3019140625	-6.4312655155399e-10\\
-30.281416015625	-4.71974244081539e-10\\
-30.26091796875	-4.78161569860451e-10\\
-30.240419921875	-5.08157623839939e-10\\
-30.219921875	-5.39859925764491e-10\\
-30.199423828125	-5.25976707357929e-10\\
-30.17892578125	-7.09746623814943e-10\\
-30.158427734375	-5.6600637846544e-10\\
-30.1379296875	-7.38647224694593e-10\\
-30.117431640625	-5.97451457357177e-10\\
-30.09693359375	-6.76122427412771e-10\\
-30.076435546875	-6.43207344825368e-10\\
-30.0559375	-6.3372422582573e-10\\
-30.035439453125	-4.99624313643329e-10\\
-30.01494140625	-6.94868661500702e-10\\
-29.994443359375	-6.15971068556171e-10\\
-29.9739453125	-6.70398013307509e-10\\
-29.953447265625	-8.21914119702505e-10\\
-29.93294921875	-7.10587769169449e-10\\
-29.912451171875	-8.22347206885736e-10\\
-29.891953125	-5.89459813338395e-10\\
-29.871455078125	-8.03939222241858e-10\\
-29.85095703125	-5.79338260989867e-10\\
-29.830458984375	-6.73121525878508e-10\\
-29.8099609375	-5.29587817419373e-10\\
-29.789462890625	-6.66625333128549e-10\\
-29.76896484375	-5.4309303765801e-10\\
-29.748466796875	-6.65176270958887e-10\\
-29.72796875	-5.08644124530757e-10\\
-29.707470703125	-5.76884309682577e-10\\
-29.68697265625	-5.04945763904092e-10\\
-29.666474609375	-4.81788006758792e-10\\
-29.6459765625	-3.63061058343081e-10\\
-29.625478515625	-3.25093328380624e-10\\
-29.60498046875	-1.72785702957005e-10\\
-29.584482421875	-1.31288840481608e-10\\
-29.563984375	-9.67179461755185e-11\\
-29.543486328125	-1.247037868058e-10\\
-29.52298828125	-1.65514573597487e-10\\
-29.502490234375	-1.74441011826804e-11\\
-29.4819921875	-2.62498089193332e-11\\
-29.461494140625	9.63971093953017e-11\\
-29.44099609375	5.5376279116422e-11\\
-29.420498046875	1.16970632131199e-10\\
-29.4	2.0367403910271e-10\\
-29.379501953125	2.14407497326199e-10\\
-29.35900390625	1.62892183562069e-10\\
-29.338505859375	2.6536843170802e-10\\
-29.3180078125	1.13864317534068e-10\\
-29.297509765625	2.37187254305088e-10\\
-29.27701171875	8.08348712407825e-11\\
-29.256513671875	1.59844583958681e-10\\
-29.236015625	1.09628545337197e-10\\
-29.215517578125	1.74825477734868e-10\\
-29.19501953125	1.05878692152165e-10\\
-29.174521484375	3.03531976236508e-10\\
-29.1540234375	1.0583125129292e-10\\
-29.133525390625	1.98919002915882e-10\\
-29.11302734375	4.00004880522346e-11\\
-29.092529296875	-6.0354765839924e-12\\
-29.07203125	-8.2711929948482e-11\\
-29.051533203125	-5.68329584199584e-11\\
-29.03103515625	-1.12767891878213e-10\\
-29.010537109375	-9.80605128115168e-11\\
-28.9900390625	-1.80665865413101e-10\\
-28.969541015625	-1.33484064084029e-10\\
-28.94904296875	-1.48348869323683e-10\\
-28.928544921875	-7.89883045250085e-11\\
-28.908046875	-1.38629759545831e-10\\
-28.887548828125	-1.57071259706493e-10\\
-28.86705078125	-2.30947743073985e-10\\
-28.846552734375	-1.3806723981106e-10\\
-28.8260546875	-2.91833799222182e-10\\
-28.805556640625	-1.20503155860705e-10\\
-28.78505859375	-2.05707527312887e-10\\
-28.764560546875	-3.95877339096758e-11\\
-28.7440625	-9.50352341733802e-11\\
-28.723564453125	-1.02308670982217e-10\\
-28.70306640625	-1.08361157550085e-10\\
-28.682568359375	1.30220776082681e-11\\
-28.6620703125	-3.6260552417094e-11\\
-28.641572265625	3.08225403563496e-11\\
-28.62107421875	7.25884321330978e-11\\
-28.600576171875	6.2390172502237e-11\\
-28.580078125	7.85144799853568e-11\\
-28.559580078125	-1.50227897129488e-11\\
-28.53908203125	4.41228276978628e-11\\
-28.518583984375	-5.36149673751203e-11\\
-28.4980859375	-1.82438886398164e-11\\
-28.477587890625	9.79966958209176e-12\\
-28.45708984375	1.63025208292691e-10\\
-28.436591796875	1.92612438206145e-10\\
-28.41609375	2.16774129319305e-10\\
-28.395595703125	3.28203802220659e-10\\
-28.37509765625	3.44320731897048e-10\\
-28.354599609375	2.10497358365116e-10\\
-28.3341015625	3.15752964519583e-10\\
-28.313603515625	1.75792115772212e-10\\
-28.29310546875	9.87682029592734e-11\\
-28.272607421875	4.28532093573102e-11\\
-28.252109375	5.68336427805593e-11\\
-28.231611328125	4.10728885136744e-11\\
-28.21111328125	3.09786633703119e-10\\
-28.190615234375	2.02135334886324e-10\\
-28.1701171875	3.68011409974786e-10\\
-28.149619140625	3.73556409909117e-10\\
-28.12912109375	3.68424759131756e-10\\
-28.108623046875	3.33497756916915e-10\\
-28.088125	3.41319863511737e-10\\
-28.067626953125	2.43350446703986e-10\\
-28.04712890625	1.93641100503012e-10\\
-28.026630859375	8.16713825613129e-11\\
-28.0061328125	2.4410280577344e-10\\
-27.985634765625	3.4303157711534e-10\\
-27.96513671875	3.96390785619121e-10\\
-27.944638671875	5.25598240259158e-10\\
-27.924140625	5.66915962461251e-10\\
-27.903642578125	3.87294225925351e-10\\
-27.88314453125	4.6432969258473e-10\\
-27.862646484375	3.77824068698351e-10\\
-27.8421484375	3.58822991543993e-10\\
-27.821650390625	2.44413461044094e-10\\
-27.80115234375	3.05624048168836e-10\\
-27.780654296875	3.14902747106008e-10\\
-27.76015625	3.87427289966688e-10\\
-27.739658203125	3.2546826547185e-10\\
-27.71916015625	3.84690905644356e-10\\
-27.698662109375	3.62692516407135e-10\\
-27.6781640625	4.95441701982259e-10\\
-27.657666015625	4.08646117788418e-10\\
-27.63716796875	4.93570992632799e-10\\
-27.616669921875	2.87585596638204e-10\\
-27.596171875	4.24473830303613e-10\\
-27.575673828125	3.14146641158112e-10\\
-27.55517578125	4.1680499664801e-10\\
-27.534677734375	3.31228159579364e-10\\
-27.5141796875	3.53864040336428e-10\\
-27.493681640625	3.80176515974391e-10\\
-27.47318359375	4.34554849712576e-10\\
-27.452685546875	5.00312676323061e-10\\
-27.4321875	4.31263093352121e-10\\
-27.411689453125	5.51775132350406e-10\\
-27.39119140625	3.82395165017077e-10\\
-27.370693359375	4.25972591410749e-10\\
-27.3501953125	3.11864148222009e-10\\
-27.329697265625	4.91986054395617e-10\\
-27.30919921875	2.34549310702439e-10\\
-27.288701171875	4.3040990331624e-10\\
-27.268203125	2.41458670941011e-10\\
-27.247705078125	3.80058041424955e-10\\
-27.22720703125	2.57497233398825e-10\\
-27.206708984375	3.15041460193023e-10\\
-27.1862109375	1.7861326121122e-10\\
-27.165712890625	2.22218780147746e-10\\
-27.14521484375	1.41275906659517e-10\\
-27.124716796875	5.23228013707418e-11\\
-27.10421875	5.50512914704979e-11\\
-27.083720703125	5.77374376296174e-13\\
-27.06322265625	-1.43505694365434e-10\\
-27.042724609375	-1.27880964916823e-10\\
-27.0222265625	-2.07298536785735e-10\\
-27.001728515625	-1.85180934364765e-10\\
-26.98123046875	-1.96571105588238e-10\\
-26.960732421875	-2.32933806927163e-10\\
-26.940234375	-2.27863748813019e-10\\
-26.919736328125	-3.64130847777441e-10\\
-26.89923828125	-4.20927804864042e-10\\
-26.878740234375	-5.29064101665872e-10\\
-26.8582421875	-5.4764460473296e-10\\
-26.837744140625	-6.55520232803382e-10\\
-26.81724609375	-4.74444588300888e-10\\
-26.796748046875	-6.22542857582175e-10\\
-26.77625	-3.60007566846691e-10\\
-26.755751953125	-4.67281823535064e-10\\
-26.73525390625	-2.26310585744288e-10\\
-26.714755859375	-3.89829618151502e-10\\
-26.6942578125	-1.77248013824427e-10\\
-26.673759765625	-3.53405854241646e-10\\
-26.65326171875	-2.95863910602774e-10\\
-26.632763671875	-5.23094188727627e-10\\
-26.612265625	-3.77154566377344e-10\\
-26.591767578125	-5.32332668129481e-10\\
-26.57126953125	-4.33783743315568e-10\\
-26.550771484375	-4.38126440193245e-10\\
-26.5302734375	-2.35454321730348e-10\\
-26.509775390625	-2.21387452532635e-10\\
-26.48927734375	-1.58635343314881e-10\\
-26.468779296875	-1.20443727840764e-10\\
-26.44828125	-2.95954455263834e-11\\
-26.427783203125	-1.69656910461369e-10\\
-26.40728515625	-1.49640448149113e-10\\
-26.386787109375	-7.71708789709581e-11\\
-26.3662890625	-1.97336397813894e-10\\
-26.345791015625	-1.98963610175169e-10\\
-26.32529296875	-1.60901641565085e-10\\
-26.304794921875	-1.55368358245734e-10\\
-26.284296875	-4.49082121857237e-11\\
-26.263798828125	-2.33367307264446e-10\\
-26.24330078125	-1.31445902528375e-11\\
-26.222802734375	-1.25387998546535e-10\\
-26.2023046875	-1.26628094086963e-10\\
-26.181806640625	-1.97587739069585e-10\\
-26.16130859375	-1.55093920548404e-10\\
-26.140810546875	-2.34958575223583e-10\\
-26.1203125	-2.44436527481101e-10\\
-26.099814453125	-2.11852166625429e-10\\
-26.07931640625	-1.85114070898164e-10\\
-26.058818359375	-1.53456012311276e-10\\
-26.0383203125	-1.93708471760093e-10\\
-26.017822265625	-1.73814999117194e-10\\
-25.99732421875	-1.73632647692185e-10\\
-25.976826171875	-1.90947052169896e-10\\
-25.956328125	-2.86382183801624e-10\\
-25.935830078125	-4.39261309617752e-10\\
-25.91533203125	-4.80917808632759e-10\\
-25.894833984375	-5.81911272481076e-10\\
-25.8743359375	-5.49797181590927e-10\\
-25.853837890625	-6.06338612467326e-10\\
-25.83333984375	-6.47662212101803e-10\\
-25.812841796875	-5.3473897173886e-10\\
-25.79234375	-5.7497744240266e-10\\
-25.771845703125	-5.63671676454558e-10\\
-25.75134765625	-6.16792576046283e-10\\
-25.730849609375	-5.68694401316387e-10\\
-25.7103515625	-7.21905238973247e-10\\
-25.689853515625	-5.81144045090619e-10\\
-25.66935546875	-7.96104755847731e-10\\
-25.648857421875	-6.80409016494201e-10\\
-25.628359375	-6.281117597573e-10\\
-25.607861328125	-7.65591341472489e-10\\
-25.58736328125	-6.52607505190414e-10\\
-25.566865234375	-6.95695847988676e-10\\
-25.5463671875	-7.23974665084892e-10\\
-25.525869140625	-5.91443870986604e-10\\
-25.50537109375	-6.9585384111774e-10\\
-25.484873046875	-6.38227837283628e-10\\
-25.464375	-7.26888388987412e-10\\
-25.443876953125	-8.37359129403392e-10\\
-25.42337890625	-8.76180885738613e-10\\
-25.402880859375	-8.29863655757029e-10\\
-25.3823828125	-8.70030285371951e-10\\
-25.361884765625	-7.14211369196517e-10\\
-25.34138671875	-7.75759186396994e-10\\
-25.320888671875	-6.13954280005421e-10\\
-25.300390625	-6.01735023921941e-10\\
-25.279892578125	-5.19907426235153e-10\\
-25.25939453125	-7.41643216034332e-10\\
-25.238896484375	-6.163894633408e-10\\
-25.2183984375	-8.58942832840959e-10\\
-25.197900390625	-7.86465433812064e-10\\
-25.17740234375	-7.79604706534194e-10\\
-25.156904296875	-7.49200659315873e-10\\
-25.13640625	-7.57767369138053e-10\\
-25.115908203125	-4.95482024229184e-10\\
-25.09541015625	-4.74136970004198e-10\\
-25.074912109375	-4.0612051086566e-10\\
-25.0544140625	-5.1036033110868e-10\\
-25.033916015625	-3.97520790569781e-10\\
-25.01341796875	-5.70809312798264e-10\\
-24.992919921875	-5.84543385213395e-10\\
-24.972421875	-6.90630947470849e-10\\
-24.951923828125	-5.75314441182549e-10\\
-24.93142578125	-6.33000686191618e-10\\
-24.910927734375	-7.45671598587692e-10\\
-24.8904296875	-5.69710022077678e-10\\
-24.869931640625	-4.97626568851995e-10\\
-24.84943359375	-3.46485845282947e-10\\
-24.828935546875	-4.37802139098733e-10\\
-24.8084375	-2.196528447614e-10\\
-24.787939453125	-3.5080767577524e-10\\
-24.76744140625	-1.62823692629751e-10\\
-24.746943359375	-3.36093661706288e-10\\
-24.7264453125	-1.40781420427297e-10\\
-24.705947265625	-3.35451746652701e-10\\
-24.68544921875	-2.01976077971631e-10\\
-24.664951171875	-2.8658045306103e-10\\
-24.644453125	-1.17007653543005e-10\\
-24.623955078125	-1.02237102083157e-10\\
-24.60345703125	6.49557993842538e-11\\
-24.582958984375	9.25344116448981e-11\\
-24.5624609375	2.68848960003724e-10\\
-24.541962890625	3.23359949727891e-10\\
-24.52146484375	3.75101311602231e-10\\
-24.500966796875	4.02466918064408e-10\\
-24.48046875	4.05680923281665e-10\\
-24.459970703125	3.89282341787149e-10\\
-24.43947265625	4.03887360594593e-10\\
-24.418974609375	5.08250497553532e-10\\
-24.3984765625	5.90721690758432e-10\\
-24.377978515625	6.93615558584352e-10\\
-24.35748046875	7.49983199602056e-10\\
-24.336982421875	8.36571099285043e-10\\
-24.316484375	7.38390269507193e-10\\
-24.295986328125	8.97847255473567e-10\\
-24.27548828125	6.82568806358473e-10\\
-24.254990234375	8.39956993424467e-10\\
-24.2344921875	6.96029006303225e-10\\
-24.213994140625	8.12854421338397e-10\\
-24.19349609375	6.53381135543258e-10\\
-24.172998046875	9.01569718591692e-10\\
-24.1525	7.61135736354042e-10\\
-24.132001953125	1.07220631823355e-09\\
-24.11150390625	8.89424069299068e-10\\
-24.091005859375	1.07553830183875e-09\\
-24.0705078125	9.22634108201927e-10\\
-24.050009765625	8.15926940368067e-10\\
-24.02951171875	6.81466226697769e-10\\
-24.009013671875	7.12056480407329e-10\\
-23.988515625	6.39895974410479e-10\\
-23.968017578125	5.47673817163947e-10\\
-23.94751953125	4.23565522260592e-10\\
-23.927021484375	5.58781535971904e-10\\
-23.9065234375	5.99504678636812e-10\\
-23.886025390625	4.98647427719558e-10\\
-23.86552734375	6.4025842441776e-10\\
-23.845029296875	5.54460385815773e-10\\
-23.82453125	5.94472031529167e-10\\
-23.804033203125	4.98904793856485e-10\\
-23.78353515625	4.6906992731281e-10\\
-23.763037109375	6.05446408680147e-10\\
-23.7425390625	4.22411529760703e-10\\
-23.722041015625	6.24112345735287e-10\\
-23.70154296875	4.65250925728573e-10\\
-23.681044921875	5.86474132095083e-10\\
-23.660546875	6.09340044751484e-10\\
-23.640048828125	6.38282756026834e-10\\
-23.61955078125	6.01771267576195e-10\\
-23.599052734375	7.23275644487007e-10\\
-23.5785546875	6.00459830798447e-10\\
-23.558056640625	6.01200120191929e-10\\
-23.53755859375	6.39634258712574e-10\\
-23.517060546875	6.16641418681654e-10\\
-23.4965625	6.20260813390564e-10\\
-23.476064453125	7.11897058309184e-10\\
-23.45556640625	6.80771428252782e-10\\
-23.435068359375	8.689237698187e-10\\
-23.4145703125	8.54639270151204e-10\\
-23.394072265625	1.00586372581473e-09\\
-23.37357421875	1.12235458302105e-09\\
-23.353076171875	1.18081058692038e-09\\
-23.332578125	1.15103786784728e-09\\
-23.312080078125	1.18180382462625e-09\\
-23.29158203125	1.35514223971539e-09\\
-23.271083984375	1.21656138907515e-09\\
-23.2505859375	1.32620198031609e-09\\
-23.230087890625	1.25796293080688e-09\\
-23.20958984375	1.41518035924179e-09\\
-23.189091796875	1.36451853439134e-09\\
-23.16859375	1.58173258582058e-09\\
-23.148095703125	1.43105758011692e-09\\
-23.12759765625	1.49659232148964e-09\\
-23.107099609375	1.53528826160855e-09\\
-23.0866015625	1.26347450727742e-09\\
-23.066103515625	1.38197828920228e-09\\
-23.04560546875	1.29087431556003e-09\\
-23.025107421875	1.07104133758634e-09\\
-23.004609375	1.18623168001135e-09\\
-22.984111328125	1.10963200352016e-09\\
-22.96361328125	1.20127925651423e-09\\
-22.943115234375	1.25157350541821e-09\\
-22.9226171875	1.2289648965199e-09\\
-22.902119140625	1.18179266364473e-09\\
-22.88162109375	1.16825755967319e-09\\
-22.861123046875	1.02561111921731e-09\\
-22.840625	1.02938820900195e-09\\
-22.820126953125	8.65947516388429e-10\\
-22.79962890625	6.99621434461682e-10\\
-22.779130859375	5.01661474421161e-10\\
-22.7586328125	7.10899764636225e-10\\
-22.738134765625	5.87545204790223e-10\\
-22.71763671875	7.71161740384632e-10\\
-22.697138671875	6.50538206859024e-10\\
-22.676640625	7.49067403675309e-10\\
-22.656142578125	5.71097464322099e-10\\
-22.63564453125	6.16731232834042e-10\\
-22.615146484375	3.48862774904525e-10\\
-22.5946484375	3.8587018992339e-10\\
-22.574150390625	2.00406985452312e-10\\
-22.55365234375	2.84330438040088e-10\\
-22.533154296875	3.30169467887574e-10\\
-22.51265625	4.23104418981395e-10\\
-22.492158203125	3.56658183593805e-10\\
-22.47166015625	5.73660872446654e-10\\
-22.451162109375	4.4560542169969e-10\\
-22.4306640625	3.61299493731497e-10\\
-22.410166015625	3.7954520594304e-10\\
-22.38966796875	-7.72121597462275e-13\\
-22.369169921875	1.11409378854849e-10\\
-22.348671875	-1.18975975132178e-10\\
-22.328173828125	1.55162563634658e-10\\
-22.30767578125	1.84209163975116e-11\\
-22.287177734375	2.87253706225241e-10\\
-22.2666796875	1.55776072565833e-10\\
-22.246181640625	5.95276015079038e-10\\
-22.22568359375	2.87163001982816e-10\\
-22.205185546875	5.48355410915711e-10\\
-22.1846875	2.8669463356524e-10\\
-22.164189453125	2.89264925495127e-10\\
-22.14369140625	8.66741824072746e-11\\
-22.123193359375	1.22413538672952e-10\\
-22.1026953125	-8.29972384102577e-11\\
-22.082197265625	2.17925189936788e-11\\
-22.06169921875	2.49450491684205e-11\\
-22.041201171875	8.19839081660249e-11\\
-22.020703125	4.30624543010896e-11\\
-22.000205078125	1.37487688265747e-10\\
-21.97970703125	2.01482231601402e-10\\
-21.959208984375	3.16826280669201e-10\\
-21.9387109375	2.0191419421561e-10\\
-21.918212890625	1.13424679144886e-10\\
-21.89771484375	9.81874366026459e-11\\
-21.877216796875	-4.72448477542682e-11\\
-21.85671875	-1.19295515792686e-12\\
-21.836220703125	-1.2099211403845e-10\\
-21.81572265625	-8.67374506599343e-12\\
-21.795224609375	-1.32687959484072e-10\\
-21.7747265625	2.15677436017398e-11\\
-21.754228515625	-1.83755497603332e-10\\
-21.73373046875	1.53950867737815e-11\\
-21.713232421875	-1.86436422606589e-10\\
-21.692734375	-1.27229729092893e-10\\
-21.672236328125	-3.56981026785038e-10\\
-21.65173828125	-3.09478736301412e-10\\
-21.631240234375	-5.45184946402622e-10\\
-21.6107421875	-5.26666527853287e-10\\
-21.590244140625	-5.50342675111284e-10\\
-21.56974609375	-4.33889185723126e-10\\
-21.549248046875	-4.6615689545714e-10\\
-21.52875	-2.96145547884649e-10\\
-21.508251953125	-2.42247840619857e-10\\
-21.48775390625	-2.25222455784122e-10\\
-21.467255859375	-1.00247355288585e-10\\
-21.4467578125	-2.10787564724742e-10\\
-21.426259765625	-2.49325887039018e-10\\
-21.40576171875	-3.85463366915419e-10\\
-21.385263671875	-2.27821991826604e-10\\
-21.364765625	-2.74808958973617e-10\\
-21.344267578125	-1.9481155938508e-10\\
-21.32376953125	-1.79518640501704e-10\\
-21.303271484375	3.9823643796645e-11\\
-21.2827734375	9.04506040912113e-11\\
-21.262275390625	1.36835796038916e-11\\
-21.24177734375	2.40335925203355e-10\\
-21.221279296875	6.22227275918517e-11\\
-21.20078125	2.37198488374233e-10\\
-21.180283203125	7.2029601524914e-11\\
-21.15978515625	2.2704092848714e-10\\
-21.139287109375	1.27016134387242e-10\\
-21.1187890625	1.99737891701949e-10\\
-21.098291015625	1.05251654811978e-10\\
-21.07779296875	2.3035446849781e-10\\
-21.057294921875	1.63181377449193e-10\\
-21.036796875	1.26966113477048e-10\\
-21.016298828125	7.2744447488526e-11\\
-20.99580078125	1.10048046440012e-10\\
-20.975302734375	-1.38588394323896e-10\\
-20.9548046875	-3.34480255599115e-11\\
-20.934306640625	-2.27677496344848e-10\\
-20.91380859375	-3.12296619345642e-10\\
-20.893310546875	-3.6307264498643e-10\\
-20.8728125	-4.19200264277851e-10\\
-20.852314453125	-3.6197440912305e-10\\
-20.83181640625	-4.18587524682248e-10\\
-20.811318359375	-3.517302431475e-10\\
-20.7908203125	-5.58304548453478e-10\\
-20.770322265625	-2.65679417859915e-10\\
-20.74982421875	-4.87301500600741e-10\\
-20.729326171875	-3.57185462631214e-10\\
-20.708828125	-4.96609841275806e-10\\
-20.688330078125	-4.17843888292335e-10\\
-20.66783203125	-6.6944697520069e-10\\
-20.647333984375	-6.52002437301958e-10\\
-20.6268359375	-6.94421688937813e-10\\
-20.606337890625	-7.59701191752857e-10\\
-20.58583984375	-5.21471139463442e-10\\
-20.565341796875	-6.09229244394647e-10\\
-20.54484375	-6.0065445950328e-10\\
-20.524345703125	-4.98478678685936e-10\\
-20.50384765625	-6.55352099659383e-10\\
-20.483349609375	-4.3326347648168e-10\\
-20.4628515625	-7.20257423005936e-10\\
-20.442353515625	-5.48260840205097e-10\\
-20.42185546875	-6.90733149684231e-10\\
-20.401357421875	-6.66707002106361e-10\\
-20.380859375	-6.4210072728086e-10\\
-20.360361328125	-4.75130708070498e-10\\
-20.33986328125	-5.64182912849417e-10\\
-20.319365234375	-4.56221630277498e-10\\
-20.2988671875	-4.38230521944811e-10\\
-20.278369140625	-3.76313789052569e-10\\
-20.25787109375	-6.15894393025245e-10\\
-20.237373046875	-4.58186172619224e-10\\
-20.216875	-4.28867003422198e-10\\
-20.196376953125	-4.88890854523705e-10\\
-20.17587890625	-5.19416236882324e-10\\
-20.155380859375	-3.10882172381421e-10\\
-20.1348828125	-2.59664998989221e-10\\
-20.114384765625	-3.55280649994759e-10\\
-20.09388671875	-4.60152115556908e-10\\
-20.073388671875	-3.62248666868541e-10\\
-20.052890625	-4.78855838942851e-10\\
-20.032392578125	-6.64825021707222e-10\\
-20.01189453125	-4.56193931024701e-10\\
-19.991396484375	-5.02265207883143e-10\\
-19.9708984375	-6.06972366124678e-10\\
-19.950400390625	-4.14176426313658e-10\\
-19.92990234375	-3.45602510438649e-10\\
-19.909404296875	-4.03824111780144e-10\\
-19.88890625	-7.83532069994276e-11\\
-19.868408203125	-4.52597455416171e-10\\
-19.84791015625	-1.53499505044952e-10\\
-19.827412109375	-6.21174081974969e-10\\
-19.8069140625	-4.42878189148011e-10\\
-19.786416015625	-5.96792713141844e-10\\
-19.76591796875	-4.88238419274496e-10\\
-19.745419921875	-7.78890567379086e-10\\
-19.724921875	-3.83511906695995e-10\\
-19.704423828125	-4.92260817213058e-10\\
-19.68392578125	-2.10821857785683e-10\\
-19.663427734375	-3.3849056073829e-10\\
-19.6429296875	-1.70557343701892e-10\\
-19.622431640625	-3.66844528630634e-10\\
-19.60193359375	-3.5344846852998e-10\\
-19.581435546875	-4.29405324263615e-10\\
-19.5609375	-4.41201127651882e-10\\
-19.540439453125	-6.19872782139403e-10\\
-19.51994140625	-4.11570802999976e-10\\
-19.499443359375	-4.30027741145103e-10\\
-19.4789453125	-2.76915719104557e-10\\
-19.458447265625	-1.34645267721626e-10\\
-19.43794921875	-4.52250595601102e-11\\
-19.417451171875	2.95592286850711e-12\\
-19.396953125	-1.22788226140315e-10\\
-19.376455078125	-6.39758648629968e-11\\
-19.35595703125	-2.01703437216598e-10\\
-19.335458984375	-2.32310997863061e-10\\
-19.3149609375	-3.35283568284366e-10\\
-19.294462890625	-2.75064202412187e-10\\
-19.27396484375	-3.27248188288331e-10\\
-19.253466796875	-2.59585903387369e-11\\
-19.23296875	-1.54686500337428e-10\\
-19.212470703125	6.74184371473618e-11\\
-19.19197265625	3.08756829432993e-11\\
-19.171474609375	2.11194312634073e-10\\
-19.1509765625	2.98109090541394e-11\\
-19.130478515625	2.18575056875762e-10\\
-19.10998046875	7.93872191473687e-11\\
-19.089482421875	1.09724066328517e-10\\
-19.068984375	3.35092533241917e-11\\
-19.048486328125	1.54172591812002e-10\\
-19.02798828125	1.6242729339306e-10\\
-19.007490234375	1.58704315730191e-10\\
-18.9869921875	1.71195250525399e-10\\
-18.966494140625	4.88492746021322e-10\\
-18.94599609375	4.12231832038704e-10\\
-18.925498046875	4.96577508323181e-10\\
-18.905	5.03688476601437e-10\\
-18.884501953125	4.34993446302201e-10\\
-18.86400390625	3.45078669949001e-10\\
-18.843505859375	2.71239067495171e-10\\
-18.8230078125	2.69127801034501e-10\\
-18.802509765625	1.85598721166006e-10\\
-18.78201171875	2.84644097245412e-10\\
-18.761513671875	2.16226450120511e-10\\
-18.741015625	1.6577507767169e-10\\
-18.720517578125	4.67933480537527e-10\\
-18.70001953125	2.67699650606852e-10\\
-18.679521484375	4.78822207646039e-10\\
-18.6590234375	2.59112781070606e-10\\
-18.638525390625	3.13680492608734e-10\\
-18.61802734375	1.25006601461514e-10\\
-18.597529296875	1.875658860958e-10\\
-18.57703125	3.37271907073488e-11\\
-18.556533203125	8.08711594577788e-11\\
-18.53603515625	1.63795641198661e-10\\
-18.515537109375	3.10466582064753e-10\\
-18.4950390625	2.99252792757889e-10\\
-18.474541015625	3.94330310288112e-10\\
-18.45404296875	2.6458480435719e-10\\
-18.433544921875	3.95324935462681e-10\\
-18.413046875	4.48042579779107e-10\\
-18.392548828125	3.82306140838133e-10\\
-18.37205078125	4.96203741704123e-10\\
-18.351552734375	4.43336531245225e-10\\
-18.3310546875	6.58527783266352e-10\\
-18.310556640625	6.36975763307702e-10\\
-18.29005859375	6.78486671826717e-10\\
-18.269560546875	6.3440922710601e-10\\
-18.2490625	7.17648965295566e-10\\
-18.228564453125	5.07666008115736e-10\\
-18.20806640625	5.74959684956846e-10\\
-18.187568359375	4.16900620189313e-10\\
-18.1670703125	6.3049040073498e-10\\
-18.146572265625	4.39023667087297e-10\\
-18.12607421875	6.17731418028958e-10\\
-18.105576171875	7.08461795972266e-10\\
-18.085078125	4.40450009046073e-10\\
-18.064580078125	6.533103325179e-10\\
-18.04408203125	5.49472992329832e-10\\
-18.023583984375	5.00995445967875e-10\\
-18.0030859375	4.40130004328685e-10\\
-17.982587890625	3.52902387068431e-10\\
-17.96208984375	5.70701302832597e-10\\
-17.941591796875	3.38784434600797e-10\\
-17.92109375	5.20869691623222e-10\\
-17.900595703125	5.68639843476612e-10\\
-17.88009765625	5.89162472353592e-10\\
-17.859599609375	5.4668020159709e-10\\
-17.8391015625	6.31897515070046e-10\\
-17.818603515625	4.90662942106246e-10\\
-17.79810546875	5.68742803682149e-10\\
-17.777607421875	3.69823558033424e-10\\
-17.757109375	5.57473980335288e-10\\
-17.736611328125	4.77754698160187e-10\\
-17.71611328125	4.45548639393748e-10\\
-17.695615234375	5.8266613778145e-10\\
-17.6751171875	5.53398540487906e-10\\
-17.654619140625	4.73835922987212e-10\\
-17.63412109375	5.79447677148983e-10\\
-17.613623046875	5.88841458550952e-10\\
-17.593125	6.43325099072135e-10\\
-17.572626953125	5.44750706306478e-10\\
-17.55212890625	5.87196097522025e-10\\
-17.531630859375	7.55423865104385e-10\\
-17.5111328125	4.82931251639618e-10\\
-17.490634765625	5.65334258916984e-10\\
-17.47013671875	5.82865121212905e-10\\
-17.449638671875	4.9945892364565e-10\\
-17.429140625	5.00223667174115e-10\\
-17.408642578125	5.91347682032092e-10\\
-17.38814453125	4.93118944058736e-10\\
-17.367646484375	7.03644462443981e-10\\
-17.3471484375	3.72324639161052e-10\\
-17.326650390625	8.01735414001416e-10\\
-17.30615234375	3.8571859224027e-10\\
-17.285654296875	5.26757573971807e-10\\
-17.26515625	2.4038697666187e-10\\
-17.244658203125	4.17311276101155e-10\\
-17.22416015625	2.60584978380597e-10\\
-17.203662109375	4.90038051949958e-10\\
-17.1831640625	4.84227998462725e-10\\
-17.162666015625	5.91471191416121e-10\\
-17.14216796875	5.07979465320172e-10\\
-17.121669921875	6.02486449875233e-10\\
-17.101171875	5.21518850413143e-10\\
-17.080673828125	5.21893223101472e-10\\
-17.06017578125	3.00990953295558e-10\\
-17.039677734375	3.21586959014057e-10\\
-17.0191796875	7.35430906478888e-11\\
-16.998681640625	1.06231118417245e-10\\
-16.97818359375	-8.21578790763253e-11\\
-16.957685546875	7.14037851634703e-11\\
-16.9371875	-3.50092779072548e-11\\
-16.916689453125	4.09131871609028e-11\\
-16.89619140625	1.28396570111813e-10\\
-16.875693359375	2.0105723830279e-11\\
-16.8551953125	1.99834925462207e-10\\
-16.834697265625	9.02999292144944e-11\\
-16.81419921875	1.80121332247556e-10\\
-16.793701171875	1.13247844381416e-11\\
-16.773203125	7.08069155736449e-11\\
-16.752705078125	-8.87564436431446e-11\\
-16.73220703125	5.80995164339543e-11\\
-16.711708984375	-1.03799627125594e-10\\
-16.6912109375	8.73367834785104e-11\\
-16.670712890625	5.47014300944429e-11\\
-16.65021484375	1.97950193285185e-10\\
-16.629716796875	-6.15119368921093e-12\\
-16.60921875	5.97914096994238e-12\\
-16.588720703125	-8.40284493335485e-11\\
-16.56822265625	-9.67819009960215e-11\\
-16.547724609375	-1.219154388287e-10\\
-16.5272265625	-6.36203285082972e-11\\
-16.506728515625	2.16879138737838e-11\\
-16.48623046875	6.2405470481606e-11\\
-16.465732421875	7.86463733036374e-11\\
-16.445234375	2.29616626180714e-10\\
-16.424736328125	1.47764061509975e-10\\
-16.40423828125	1.47097176636296e-10\\
-16.383740234375	7.50731585238358e-11\\
-16.3632421875	-9.68046620287261e-11\\
-16.342744140625	-1.48312596905229e-10\\
-16.32224609375	-2.49885529348673e-10\\
-16.301748046875	-2.58107110724484e-10\\
-16.28125	-3.37844858754242e-10\\
-16.260751953125	-1.82794748106481e-10\\
-16.24025390625	-1.359678273824e-10\\
-16.219755859375	-1.19197886554523e-10\\
-16.1992578125	1.06308200266318e-10\\
-16.178759765625	-6.63431444266215e-11\\
-16.15826171875	1.61468437990823e-10\\
-16.137763671875	-1.66529347365902e-10\\
-16.117265625	-8.62141367306319e-12\\
-16.096767578125	-1.54690380014273e-10\\
-16.07626953125	-1.19482141048716e-10\\
-16.055771484375	-2.89030268581745e-10\\
-16.0352734375	-1.61704267269449e-10\\
-16.014775390625	-1.146883027617e-10\\
-15.99427734375	3.04889891468266e-11\\
-15.973779296875	-9.81143439548968e-11\\
-15.95328125	1.21305080208591e-10\\
-15.932783203125	2.11269571544371e-12\\
-15.91228515625	-6.34749485260274e-11\\
-15.891787109375	-1.64574249639949e-10\\
-15.8712890625	-2.74083583702417e-10\\
-15.850791015625	-3.1691093051423e-10\\
-15.83029296875	-3.82510783502448e-10\\
-15.809794921875	-3.26087961099819e-10\\
-15.789296875	-3.04841863483256e-10\\
-15.768798828125	-2.89616185296024e-10\\
-15.74830078125	-1.85198679877158e-10\\
-15.727802734375	-4.51348929247363e-11\\
-15.7073046875	-1.09830497664275e-10\\
-15.686806640625	1.06138036801662e-11\\
-15.66630859375	-1.50245883795459e-10\\
-15.645810546875	-5.35372395906525e-11\\
-15.6253125	-2.06267444370403e-10\\
-15.604814453125	-2.14821978275606e-10\\
-15.58431640625	-1.95559631183366e-10\\
-15.563818359375	-2.60001466070005e-10\\
-15.5433203125	-8.03897340714252e-11\\
-15.522822265625	-9.78600697961114e-11\\
-15.50232421875	7.18769064424913e-11\\
-15.481826171875	1.83483603077818e-10\\
-15.461328125	1.22997064001752e-10\\
-15.440830078125	2.87946329928233e-10\\
-15.42033203125	-4.98772334415115e-11\\
-15.399833984375	9.37722302374008e-11\\
-15.3793359375	8.38008548550899e-12\\
-15.358837890625	-3.36938540409469e-12\\
-15.33833984375	-3.17689075083133e-11\\
-15.317841796875	1.44823082947032e-10\\
-15.29734375	5.2854130569967e-12\\
-15.276845703125	4.2563997736098e-10\\
-15.25634765625	2.43115604019954e-10\\
-15.235849609375	3.95350782377094e-10\\
-15.2153515625	2.84802342583977e-10\\
-15.194853515625	1.90338567617523e-10\\
-15.17435546875	1.44426121255754e-10\\
-15.153857421875	1.30399400838833e-10\\
-15.133359375	-1.06258947564259e-10\\
-15.112861328125	7.03563436324681e-11\\
-15.09236328125	-9.17832448167453e-11\\
-15.071865234375	7.17237320823339e-11\\
-15.0513671875	8.42402917524731e-11\\
-15.030869140625	1.06851693554429e-10\\
-15.01037109375	2.57996750934295e-10\\
-14.989873046875	1.04390592038972e-10\\
-14.969375	1.96971518635604e-10\\
-14.948876953125	1.15091845233386e-10\\
-14.92837890625	8.78012139993988e-11\\
-14.907880859375	5.07011318059413e-11\\
-14.8873828125	8.45005489096632e-11\\
-14.866884765625	-3.66022812028685e-11\\
-14.84638671875	2.42731277838256e-10\\
-14.825888671875	-2.42921315135128e-10\\
-14.805390625	2.70991987265752e-10\\
-14.784892578125	3.13572263079385e-11\\
-14.76439453125	2.92861311665148e-10\\
-14.743896484375	8.21646508792893e-11\\
-14.7233984375	2.55285989776887e-10\\
-14.702900390625	6.24058121754817e-11\\
-14.68240234375	9.42906834454573e-11\\
-14.661904296875	1.01752443751073e-10\\
-14.64140625	5.3182352918766e-11\\
-14.620908203125	-1.09440023753934e-10\\
-14.60041015625	6.81614755956307e-12\\
-14.579912109375	-7.1326691348226e-11\\
-14.5594140625	8.45727208739143e-11\\
-14.538916015625	-1.76219790026154e-11\\
-14.51841796875	9.54092569512089e-11\\
-14.497919921875	1.53233748677189e-10\\
-14.477421875	2.55498963066769e-10\\
-14.456923828125	1.81265350664704e-10\\
-14.43642578125	4.18489057245952e-10\\
-14.415927734375	3.03998263618382e-10\\
-14.3954296875	2.18715402972451e-10\\
-14.374931640625	3.30568616605718e-10\\
-14.35443359375	2.00741841482476e-10\\
-14.333935546875	3.68021622039624e-10\\
-14.3134375	2.4362709947501e-10\\
-14.292939453125	4.6797519313576e-10\\
-14.27244140625	5.04431990259032e-10\\
-14.251943359375	6.61230140168938e-10\\
-14.2314453125	5.15495763206134e-10\\
-14.210947265625	7.02860652856814e-10\\
-14.19044921875	5.07830066049989e-10\\
-14.169951171875	4.60988572692597e-10\\
-14.149453125	4.48305456360557e-10\\
-14.128955078125	5.62027405020219e-10\\
-14.10845703125	5.93218778465438e-10\\
-14.087958984375	6.41050470110103e-10\\
-14.0674609375	6.15523287309202e-10\\
-14.046962890625	7.73798408552095e-10\\
-14.02646484375	6.22364407259104e-10\\
-14.005966796875	6.66197847475336e-10\\
-13.98546875	5.54093918080115e-10\\
-13.964970703125	4.37142437034492e-10\\
-13.94447265625	3.22824787536881e-10\\
-13.923974609375	4.89104239564961e-10\\
-13.9034765625	4.64951956507168e-10\\
-13.882978515625	5.20543127982e-10\\
-13.86248046875	5.35785740992784e-10\\
-13.841982421875	6.43729219013786e-10\\
-13.821484375	6.53860496869404e-10\\
-13.800986328125	6.64988016247047e-10\\
-13.78048828125	6.82941313374985e-10\\
-13.759990234375	6.22739896650385e-10\\
-13.7394921875	4.31479902997642e-10\\
-13.718994140625	4.81545028952035e-10\\
-13.69849609375	4.20300428789256e-10\\
-13.677998046875	6.11032381214302e-10\\
-13.6575	4.81415872472826e-10\\
-13.637001953125	6.45779730115794e-10\\
-13.61650390625	6.66925401762622e-10\\
-13.596005859375	7.70245081775208e-10\\
-13.5755078125	7.15289292109532e-10\\
-13.555009765625	8.67875144006886e-10\\
-13.53451171875	7.28309758736416e-10\\
-13.514013671875	8.04750071023141e-10\\
-13.493515625	7.08222945761346e-10\\
-13.473017578125	9.30266376879659e-10\\
-13.45251953125	7.44598699456477e-10\\
-13.432021484375	1.02238708595702e-09\\
-13.4115234375	8.05824567246565e-10\\
-13.391025390625	1.02706634114353e-09\\
-13.37052734375	1.03187537059361e-09\\
-13.350029296875	9.40555907000339e-10\\
-13.32953125	1.17732847657142e-09\\
-13.309033203125	1.10459279116376e-09\\
-13.28853515625	1.28952190321564e-09\\
-13.268037109375	1.19288277663977e-09\\
-13.2475390625	1.25039240694261e-09\\
-13.227041015625	9.7327564728131e-10\\
-13.20654296875	1.15106770262685e-09\\
-13.186044921875	9.00100302534655e-10\\
-13.165546875	1.05092500790877e-09\\
-13.145048828125	1.00312026106894e-09\\
-13.12455078125	1.03012080358686e-09\\
-13.104052734375	9.03522057446931e-10\\
-13.0835546875	1.06715190826724e-09\\
-13.063056640625	1.05630372676596e-09\\
-13.04255859375	1.04679119841891e-09\\
-13.022060546875	1.09605646291232e-09\\
-13.0015625	9.68990032652454e-10\\
-12.981064453125	6.96814503177605e-10\\
-12.96056640625	7.39049653560256e-10\\
-12.940068359375	5.49056201606961e-10\\
-12.9195703125	6.8842922937788e-10\\
-12.899072265625	4.36375601960639e-10\\
-12.87857421875	6.25052955407305e-10\\
-12.858076171875	5.11108596229066e-10\\
-12.837578125	6.30064476233405e-10\\
-12.817080078125	4.18745970506794e-10\\
-12.79658203125	5.27604105985383e-10\\
-12.776083984375	2.28689079655415e-10\\
-12.7555859375	1.99686300333563e-10\\
-12.735087890625	-5.39516194805996e-12\\
-12.71458984375	-2.03359591023897e-11\\
-12.694091796875	-1.13660390961423e-10\\
-12.67359375	-9.57389204889011e-11\\
-12.653095703125	5.44442313822853e-11\\
-12.63259765625	2.56855099075649e-10\\
-12.612099609375	2.10402009171666e-10\\
-12.5916015625	3.87107786239178e-10\\
-12.571103515625	3.47246242114825e-10\\
-12.55060546875	2.42362495798664e-10\\
-12.530107421875	1.42531303823155e-10\\
-12.509609375	1.04709988246205e-10\\
-12.489111328125	2.85269616916944e-10\\
-12.46861328125	7.93151170984141e-11\\
-12.448115234375	1.9938246829062e-10\\
-12.4276171875	2.66400078539516e-10\\
-12.407119140625	2.38187424180813e-10\\
-12.38662109375	4.64692637444185e-10\\
-12.366123046875	4.9218044850775e-10\\
-12.345625	4.5859410870853e-10\\
-12.325126953125	6.19823952239857e-10\\
-12.30462890625	2.14894708088612e-10\\
-12.284130859375	5.30114861199279e-10\\
-12.2636328125	1.76927011890075e-10\\
-12.243134765625	2.74076690113285e-10\\
-12.22263671875	2.55763397012879e-10\\
-12.202138671875	5.68253417011126e-10\\
-12.181640625	4.69713496022229e-10\\
-12.161142578125	6.5910437203392e-10\\
-12.14064453125	6.25663816615979e-10\\
-12.120146484375	6.47951198080966e-10\\
-12.0996484375	4.60081055633213e-10\\
-12.079150390625	4.23940220322471e-10\\
-12.05865234375	3.71645960390637e-10\\
-12.038154296875	3.34459432866058e-10\\
-12.01765625	2.7502862938145e-10\\
-11.997158203125	2.8946796495186e-10\\
-11.97666015625	2.24163247406093e-10\\
-11.956162109375	4.2396122746855e-10\\
-11.9356640625	2.55929335854692e-10\\
-11.915166015625	3.57105653975278e-10\\
-11.89466796875	2.5385922313696e-10\\
-11.874169921875	5.18554377348714e-11\\
-11.853671875	9.70328788378791e-11\\
-11.833173828125	-6.02361634310419e-11\\
-11.81267578125	-3.84250051985334e-12\\
-11.792177734375	-5.41162495299225e-11\\
-11.7716796875	4.32962995856149e-11\\
-11.751181640625	-5.98755401026622e-11\\
-11.73068359375	-3.25750792230744e-11\\
-11.710185546875	-2.54642851182452e-10\\
-11.6896875	-1.04504117115123e-10\\
-11.669189453125	-2.6649223889711e-10\\
-11.64869140625	-2.69216547413656e-10\\
-11.628193359375	-3.64855265290176e-10\\
-11.6076953125	-4.34847423412192e-10\\
-11.587197265625	-5.45024849143274e-10\\
-11.56669921875	-4.75264015955458e-10\\
-11.546201171875	-6.86702079479958e-10\\
-11.525703125	-4.9485676069555e-10\\
-11.505205078125	-6.7624805187293e-10\\
-11.48470703125	-5.4922038542055e-10\\
-11.464208984375	-5.52448843242583e-10\\
-11.4437109375	-4.07920225484666e-10\\
-11.423212890625	-3.86371711068858e-10\\
-11.40271484375	-4.29158098318836e-10\\
-11.382216796875	-4.41509700244916e-10\\
-11.36171875	-4.6519821910505e-10\\
-11.341220703125	-4.52646101281368e-10\\
-11.32072265625	-4.11096447358316e-10\\
-11.300224609375	-5.67335271635647e-10\\
-11.2797265625	-3.83272456511317e-10\\
-11.259228515625	-5.13794978652328e-10\\
-11.23873046875	-3.91719826078313e-10\\
-11.218232421875	-3.1467258254626e-10\\
-11.197734375	-3.45544203115452e-10\\
-11.177236328125	-4.09218771062896e-10\\
-11.15673828125	-2.69380590166649e-10\\
-11.136240234375	-4.87908519600178e-10\\
-11.1157421875	-5.07646593652673e-10\\
-11.095244140625	-4.66920185809551e-10\\
-11.07474609375	-4.27182485711565e-10\\
-11.054248046875	-4.51898927013812e-10\\
-11.03375	-3.22304547866997e-10\\
-11.013251953125	-3.63837432062457e-10\\
-10.99275390625	-2.57096639873692e-10\\
-10.972255859375	-4.9996560051461e-10\\
-10.9517578125	-3.86508558544016e-10\\
-10.931259765625	-6.04819218162658e-10\\
-10.91076171875	-5.76840928322637e-10\\
-10.890263671875	-8.44120735383953e-10\\
-10.869765625	-7.42041897737206e-10\\
-10.849267578125	-6.84595007063814e-10\\
-10.82876953125	-8.54605977144946e-10\\
-10.808271484375	-7.25546076538976e-10\\
-10.7877734375	-7.99772826814713e-10\\
-10.767275390625	-6.88466857345707e-10\\
-10.74677734375	-8.9406435308485e-10\\
-10.726279296875	-7.32306215146411e-10\\
-10.70578125	-1.08753751664263e-09\\
-10.685283203125	-8.67659472612791e-10\\
-10.66478515625	-1.1177194989836e-09\\
-10.644287109375	-9.09061626138042e-10\\
-10.6237890625	-1.08699313918372e-09\\
-10.603291015625	-8.65677411909384e-10\\
-10.58279296875	-9.84971462860989e-10\\
-10.562294921875	-8.46412476146801e-10\\
-10.541796875	-8.78355549064311e-10\\
-10.521298828125	-7.7671186756828e-10\\
-10.50080078125	-7.32310642239585e-10\\
-10.480302734375	-7.58635609322803e-10\\
-10.4598046875	-8.15591162784836e-10\\
-10.439306640625	-6.30320663140634e-10\\
-10.41880859375	-7.93140934600698e-10\\
-10.398310546875	-5.53123935544543e-10\\
-10.3778125	-6.88536709362914e-10\\
-10.357314453125	-3.91914128656085e-10\\
-10.33681640625	-4.86529050660571e-10\\
-10.316318359375	-3.33530371744118e-10\\
-10.2958203125	-3.3978594545298e-10\\
-10.275322265625	-1.58985177778664e-10\\
-10.25482421875	-3.18564852604069e-10\\
-10.234326171875	-1.39068610870068e-10\\
-10.213828125	-1.91451867528981e-10\\
-10.193330078125	-1.36822176870276e-10\\
-10.17283203125	-1.56033265645044e-10\\
-10.152333984375	-7.23171972108594e-11\\
-10.1318359375	-2.37557237001544e-11\\
-10.111337890625	7.74521529261891e-11\\
-10.09083984375	2.17897939186594e-10\\
-10.070341796875	2.79314038650628e-10\\
-10.04984375	2.8033099645569e-10\\
-10.029345703125	2.22349789700525e-10\\
-10.00884765625	1.9158672866193e-10\\
-9.988349609375	1.02341174455589e-10\\
-9.9678515625	1.61329610959767e-10\\
-9.947353515625	-4.65090089510586e-11\\
-9.92685546875	7.52957172586786e-12\\
-9.906357421875	-1.07807068582274e-11\\
-9.885859375	9.96940543551852e-11\\
-9.865361328125	2.59818590500113e-11\\
-9.84486328125	6.72572759205883e-11\\
-9.824365234375	-1.02090387929475e-10\\
-9.8038671875	8.33211054403473e-11\\
-9.783369140625	-2.31502433206054e-10\\
-9.76287109375	-3.97068384494524e-11\\
-9.742373046875	-2.28353494171419e-10\\
-9.721875	-1.91918082561456e-10\\
-9.701376953125	-2.85579083159781e-10\\
-9.68087890625	-2.48846260341323e-10\\
-9.660380859375	-5.35038171537688e-10\\
-9.6398828125	-3.28821810029116e-10\\
-9.619384765625	-4.57423109962644e-10\\
-9.59888671874999	-4.1181824394549e-10\\
-9.578388671875	-3.55841407114744e-10\\
-9.557890625	-2.43995411586465e-10\\
-9.537392578125	-2.4003885386134e-10\\
-9.51689453125	-1.28479526401205e-10\\
-9.496396484375	-1.61296761058362e-10\\
-9.4758984375	-1.29854267056048e-11\\
-9.455400390625	-6.27634084847908e-11\\
-9.43490234375	8.55505716144211e-11\\
-9.414404296875	7.63811225714935e-11\\
-9.39390625	1.47544621778906e-10\\
-9.373408203125	1.28974248501906e-10\\
-9.35291015625	1.56478349889226e-10\\
-9.332412109375	2.28248079200108e-10\\
-9.3119140625	2.62984580497826e-10\\
-9.291416015625	4.05766787918152e-10\\
-9.27091796875	2.72296890953572e-10\\
-9.250419921875	5.43164890418454e-10\\
-9.229921875	4.5831208059876e-10\\
-9.209423828125	6.02004436477724e-10\\
-9.18892578125	4.87360042220784e-10\\
-9.168427734375	7.35192272802142e-10\\
-9.1479296875	6.14018478202576e-10\\
-9.127431640625	7.35695759132593e-10\\
-9.10693359375	7.97367210993284e-10\\
-9.08643554687499	8.49941526439088e-10\\
-9.0659375	7.70780451112754e-10\\
-9.045439453125	1.00246218741321e-09\\
-9.02494140625	9.07858701631045e-10\\
-9.004443359375	1.07888024203883e-09\\
-8.9839453125	9.05333204140395e-10\\
-8.963447265625	9.62915754272691e-10\\
-8.94294921875	9.71258475183565e-10\\
-8.922451171875	8.7905545416803e-10\\
-8.901953125	8.00222164316449e-10\\
-8.881455078125	8.39268071268064e-10\\
-8.86095703125	8.04750435145816e-10\\
-8.840458984375	8.01327731542864e-10\\
-8.8199609375	8.01317801180251e-10\\
-8.799462890625	9.28389222558492e-10\\
-8.77896484375	7.65036349113092e-10\\
-8.758466796875	8.73155534065976e-10\\
-8.73796875	8.73626579397158e-10\\
-8.717470703125	8.07313385323765e-10\\
-8.69697265625	7.98467949785207e-10\\
-8.676474609375	6.90852269696097e-10\\
-8.6559765625	5.85960185785263e-10\\
-8.635478515625	7.32856703677995e-10\\
-8.61498046875	5.18100615573351e-10\\
-8.594482421875	6.32503490420358e-10\\
-8.57398437499999	4.44999341032635e-10\\
-8.553486328125	5.23314548582773e-10\\
-8.53298828125	3.86803784447091e-10\\
-8.512490234375	7.14304770462632e-10\\
-8.4919921875	5.53707599966906e-10\\
-8.471494140625	8.27473757876519e-10\\
-8.45099609375	6.54653439525605e-10\\
-8.430498046875	8.6231019963262e-10\\
-8.41	7.48390911949782e-10\\
-8.389501953125	8.43670271904862e-10\\
-8.36900390625	7.91339848911274e-10\\
-8.348505859375	8.25451585605568e-10\\
-8.3280078125	8.04396481863852e-10\\
-8.307509765625	7.54248094811663e-10\\
-8.28701171875	9.77129039531008e-10\\
-8.266513671875	8.31267418367979e-10\\
-8.246015625	1.01082153828964e-09\\
-8.225517578125	1.04398490911046e-09\\
-8.20501953125	1.16240356947432e-09\\
-8.184521484375	1.02820662496858e-09\\
-8.1640234375	1.25582246915372e-09\\
-8.143525390625	1.09165437612866e-09\\
-8.12302734375	1.12145413519962e-09\\
-8.102529296875	9.74650556601308e-10\\
-8.08203125	1.02642662083594e-09\\
-8.06153320312499	8.70354569573e-10\\
-8.04103515625	9.79121518753446e-10\\
-8.020537109375	9.19810036011498e-10\\
-8.0000390625	9.4977533255655e-10\\
-7.979541015625	8.43030088547852e-10\\
-7.95904296875	8.67881192419668e-10\\
-7.938544921875	7.77197105074792e-10\\
-7.918046875	8.81829868378652e-10\\
-7.897548828125	6.13122914264182e-10\\
-7.87705078125	7.96845738968604e-10\\
-7.856552734375	5.45792319036931e-10\\
-7.8360546875	6.50670941319372e-10\\
-7.815556640625	4.90705889905747e-10\\
-7.79505859375	5.36804970237028e-10\\
-7.774560546875	2.88428681012261e-10\\
-7.7540625	3.67982171911619e-10\\
-7.733564453125	9.44846573833864e-11\\
-7.71306640625	7.89361681877503e-11\\
-7.692568359375	-7.37744376694167e-11\\
-7.6720703125	-7.55974731841202e-11\\
-7.651572265625	-2.89249284656855e-10\\
-7.63107421875	-1.93650835525339e-10\\
-7.610576171875	-2.86485720098302e-10\\
-7.590078125	-3.4559356019882e-10\\
-7.569580078125	-3.33447533877973e-10\\
-7.54908203125	-3.16602887408606e-10\\
-7.528583984375	-3.81575478222524e-10\\
-7.5080859375	-4.90512577111921e-10\\
-7.487587890625	-3.12912445908599e-10\\
-7.46708984375	-3.61069889768422e-10\\
-7.446591796875	-2.16416883763957e-10\\
-7.42609375	-3.38870388510267e-10\\
-7.405595703125	-1.33333604937301e-10\\
-7.38509765625	-2.58651624754631e-10\\
-7.364599609375	-8.02167361440967e-11\\
-7.3441015625	-1.40711020666973e-10\\
-7.323603515625	-7.80043104923359e-13\\
-7.30310546875	-1.58704534331592e-10\\
-7.282607421875	5.82834571409209e-11\\
-7.262109375	-7.36934215720255e-11\\
-7.241611328125	1.20396640038581e-10\\
-7.22111328125	-9.28031122237604e-11\\
-7.200615234375	1.52914838923146e-10\\
-7.1801171875	1.19396757737424e-10\\
-7.159619140625	4.02559340042414e-10\\
-7.13912109375	3.16546359174505e-10\\
-7.118623046875	4.83977760318262e-10\\
-7.098125	4.08431903797294e-10\\
-7.077626953125	2.58410195229492e-10\\
-7.05712890625	2.92894636552089e-10\\
-7.036630859375	1.68949635468465e-10\\
-7.0161328125	8.65248642362963e-11\\
-6.995634765625	1.32800197665652e-10\\
-6.97513671875	-3.01028901029068e-11\\
-6.954638671875	6.47293703788223e-11\\
-6.934140625	-3.46355423584773e-11\\
-6.913642578125	3.61653263447849e-11\\
-6.89314453125	-1.13465576432916e-10\\
-6.872646484375	-4.64493453779041e-11\\
-6.8521484375	-1.68028295385311e-10\\
-6.831650390625	-2.55705807943723e-10\\
-6.81115234375	-3.35194405034519e-10\\
-6.790654296875	-5.64300865849911e-10\\
-6.77015625	-4.87502642927773e-10\\
-6.749658203125	-7.62012879417191e-10\\
-6.72916015625	-5.62915236252038e-10\\
-6.708662109375	-8.29763497516231e-10\\
-6.6881640625	-7.47094054535594e-10\\
-6.667666015625	-9.59553375761085e-10\\
-6.64716796875	-8.28308318215761e-10\\
-6.626669921875	-9.62876974059922e-10\\
-6.606171875	-1.03697559287656e-09\\
-6.585673828125	-1.01866573160187e-09\\
-6.56517578125	-1.07452284621557e-09\\
-6.544677734375	-1.14267563674452e-09\\
-6.5241796875	-9.52793133142189e-10\\
-6.503681640625	-9.96280194648017e-10\\
-6.48318359375	-8.87743971959576e-10\\
-6.462685546875	-9.95742783193294e-10\\
-6.4421875	-1.03422926509206e-09\\
-6.421689453125	-8.95103491713194e-10\\
-6.40119140625	-8.50659271391322e-10\\
-6.380693359375	-9.58028023071113e-10\\
-6.3601953125	-8.79463876241127e-10\\
-6.339697265625	-9.60744920168843e-10\\
-6.31919921875	-8.06359631174272e-10\\
-6.298701171875	-8.33722621358048e-10\\
-6.278203125	-5.76010183143068e-10\\
-6.257705078125	-7.03836872259376e-10\\
-6.23720703125	-5.60678641626345e-10\\
-6.216708984375	-5.0567472690492e-10\\
-6.1962109375	-4.34951664928961e-10\\
-6.175712890625	-4.66409051783164e-10\\
-6.15521484375	-3.56251729847109e-10\\
-6.134716796875	-5.08296889128599e-10\\
-6.11421875	-3.15273849692102e-10\\
-6.093720703125	-6.01788264238465e-10\\
-6.07322265625	-2.85488555188109e-10\\
-6.052724609375	-4.57433886930715e-10\\
-6.0322265625	-3.08379378401985e-10\\
-6.011728515625	-4.58285577828401e-10\\
-5.99123046875	-2.34078142627014e-10\\
-5.970732421875	-4.04738041901218e-10\\
-5.950234375	-3.03555005613694e-10\\
-5.929736328125	-5.61556339976651e-10\\
-5.90923828125	-3.74805814120986e-10\\
-5.888740234375	-4.79386903848456e-10\\
-5.8682421875	-5.43273707655452e-10\\
-5.847744140625	-5.67316492455434e-10\\
-5.82724609375	-5.44169618990853e-10\\
-5.806748046875	-6.13701541953213e-10\\
-5.78625	-6.26114951826257e-10\\
-5.765751953125	-5.90087619057337e-10\\
-5.74525390625	-6.51268897742767e-10\\
-5.724755859375	-6.65282135472823e-10\\
-5.7042578125	-7.43590174506205e-10\\
-5.683759765625	-7.65608375650355e-10\\
-5.66326171875	-9.47773369813815e-10\\
-5.642763671875	-8.02922796966382e-10\\
-5.622265625	-9.4740061395701e-10\\
-5.601767578125	-7.24276128370273e-10\\
-5.58126953125	-7.90502277573121e-10\\
-5.560771484375	-6.27519383137231e-10\\
-5.5402734375	-6.06224519673119e-10\\
-5.519775390625	-5.18105155657629e-10\\
-5.49927734375	-5.109117898255e-10\\
-5.478779296875	-5.44553752289568e-10\\
-5.45828125	-6.81578233743386e-10\\
-5.437783203125	-6.8391264614612e-10\\
-5.41728515625	-8.44919478708104e-10\\
-5.396787109375	-6.14217143914575e-10\\
-5.3762890625	-7.49791803856881e-10\\
-5.355791015625	-3.84931415897749e-10\\
-5.33529296875	-5.59444899365875e-10\\
-5.314794921875	-3.00418732626437e-10\\
-5.294296875	-3.06577411731791e-10\\
-5.273798828125	-8.21028442808034e-11\\
-5.25330078125	-2.00797369648671e-10\\
-5.232802734375	5.51608646525746e-11\\
-5.2123046875	-7.89279475093333e-11\\
-5.191806640625	1.91536439685036e-10\\
-5.17130859375	7.22461461247777e-11\\
-5.150810546875	2.68475018588387e-10\\
-5.1303125	3.04822015627811e-10\\
-5.109814453125	4.11560484997489e-10\\
-5.08931640625	5.48591262743554e-10\\
-5.068818359375	4.31733977857644e-10\\
-5.0483203125	4.99236442592954e-10\\
-5.027822265625	5.4522103946403e-10\\
-5.00732421875	5.40030184757836e-10\\
-4.986826171875	4.63717756403839e-10\\
-4.966328125	4.25571581241462e-10\\
-4.945830078125	3.80503313550054e-10\\
-4.92533203125	4.99368200186202e-10\\
-4.904833984375	3.06101574780209e-10\\
-4.8843359375	4.46188428156922e-10\\
-4.863837890625	3.36510326548558e-10\\
-4.84333984375	3.72896501824571e-10\\
-4.822841796875	2.08911410437905e-10\\
-4.80234375	2.86004406513041e-10\\
-4.781845703125	1.79088733439864e-10\\
-4.76134765625	2.07702895656605e-10\\
-4.740849609375	1.81363971838788e-11\\
-4.7203515625	2.64070025331696e-10\\
-4.699853515625	-5.13146005260831e-11\\
-4.67935546875	7.48997747157868e-11\\
-4.658857421875	-1.77075484176277e-10\\
-4.638359375	-6.89497550656581e-11\\
-4.617861328125	-2.9559561078701e-10\\
-4.59736328125	-2.15527635166686e-10\\
-4.576865234375	-2.05451310770315e-10\\
-4.5563671875	-2.33714369697321e-10\\
-4.535869140625	-1.64872150156131e-10\\
-4.51537109375	-5.50263378764449e-11\\
-4.494873046875	-2.3516626459963e-11\\
-4.474375	1.85534118656762e-10\\
-4.453876953125	9.60837280362689e-11\\
-4.43337890625	2.30817173640331e-10\\
-4.412880859375	2.90676455460598e-10\\
-4.3923828125	3.24197478885051e-10\\
-4.371884765625	3.15423762052134e-10\\
-4.35138671875	3.89467029142697e-10\\
-4.330888671875	4.06893422893991e-10\\
-4.310390625	3.57228364149723e-10\\
-4.289892578125	5.84345303622355e-10\\
-4.26939453125	4.64325822864012e-10\\
-4.248896484375	6.93063791101788e-10\\
-4.2283984375	6.92487722665821e-10\\
-4.207900390625	9.87221780638233e-10\\
-4.18740234375	8.31870126016838e-10\\
-4.166904296875	1.1353280563802e-09\\
-4.14640625	1.0200403383248e-09\\
-4.125908203125	1.14778625464377e-09\\
-4.10541015625	1.0900875141676e-09\\
-4.084912109375	8.65764417403677e-10\\
-4.0644140625	8.76661217511534e-10\\
-4.043916015625	9.83948288337016e-10\\
-4.02341796875	9.07166165502564e-10\\
-4.002919921875	8.93755937846402e-10\\
-3.982421875	9.37810168632345e-10\\
-3.961923828125	1.1024491621998e-09\\
-3.94142578125	1.11932915721332e-09\\
-3.920927734375	1.20070534842447e-09\\
-3.9004296875	1.11395806410006e-09\\
-3.879931640625	1.06134287025102e-09\\
-3.85943359375	8.86775575903146e-10\\
-3.838935546875	9.65427891159556e-10\\
-3.8184375	7.48545474698536e-10\\
-3.797939453125	8.34096084643761e-10\\
-3.77744140625	8.1515421840254e-10\\
-3.756943359375	1.00168726468179e-09\\
-3.7364453125	8.04744146800657e-10\\
-3.715947265625	9.6313938340832e-10\\
-3.69544921875	9.86072385627049e-10\\
-3.674951171875	9.69480203428303e-10\\
-3.654453125	8.47349196113655e-10\\
-3.633955078125	9.92351742035547e-10\\
-3.61345703125	7.24806787677066e-10\\
-3.592958984375	9.17491811276405e-10\\
-3.5724609375	7.09826877145559e-10\\
-3.551962890625	8.17406980020263e-10\\
-3.53146484375	7.57007675538121e-10\\
-3.510966796875	8.26752328144654e-10\\
-3.49046875	6.51047706026943e-10\\
-3.469970703125	9.51674269889985e-10\\
-3.44947265625	7.8024882301666e-10\\
-3.428974609375	1.00501146614628e-09\\
-3.4084765625	8.98481813767121e-10\\
-3.387978515625	9.42237615967557e-10\\
-3.36748046875	1.00232561102115e-09\\
-3.346982421875	1.051777588436e-09\\
-3.326484375	9.37681619016527e-10\\
-3.305986328125	1.04445691770006e-09\\
-3.28548828125	9.91866071570618e-10\\
-3.264990234375	1.04202726279606e-09\\
-3.2444921875	1.0833882526036e-09\\
-3.223994140625	1.18246551744039e-09\\
-3.20349609375	1.30317975321905e-09\\
-3.182998046875	1.30050831397146e-09\\
-3.1625	1.51649591203193e-09\\
-3.142001953125	1.365614983221e-09\\
-3.12150390625	1.4464456682529e-09\\
-3.101005859375	1.22463172090079e-09\\
-3.0805078125	1.40663846522183e-09\\
-3.060009765625	1.17888680409467e-09\\
-3.03951171875	1.33252694355437e-09\\
-3.019013671875	1.14315277534481e-09\\
-2.998515625	1.32016986101371e-09\\
-2.978017578125	1.1872901878828e-09\\
-2.95751953125	1.33999094623036e-09\\
-2.937021484375	1.2840417331142e-09\\
-2.9165234375	1.32153685414536e-09\\
-2.896025390625	1.04673042751209e-09\\
-2.87552734375	1.12390770522634e-09\\
-2.855029296875	8.88778307549796e-10\\
-2.83453125	1.01948566354019e-09\\
-2.814033203125	8.58938585953092e-10\\
-2.79353515625	1.01695156511326e-09\\
-2.773037109375	7.86761592748565e-10\\
-2.7525390625	9.37524086484181e-10\\
-2.732041015625	7.37036623360609e-10\\
-2.71154296875	8.0147338723872e-10\\
-2.691044921875	4.86613077010255e-10\\
-2.670546875	4.93301209827934e-10\\
-2.650048828125	3.16419702398189e-10\\
-2.62955078125	1.90029349334029e-10\\
-2.609052734375	1.65195126616014e-10\\
-2.5885546875	1.87976839529315e-10\\
-2.568056640625	2.27911264441154e-10\\
-2.54755859375	1.23299419394295e-10\\
-2.527060546875	1.78763072883531e-10\\
-2.5065625	5.54055506421386e-11\\
-2.486064453125	1.15941748004196e-10\\
-2.46556640625	7.49739342108803e-11\\
-2.445068359375	1.39417839383367e-10\\
-2.4245703125	-1.03187937749626e-11\\
-2.404072265625	8.57939455784303e-11\\
-2.38357421875	3.64887342236221e-11\\
-2.363076171875	2.61798288155047e-10\\
-2.342578125	2.45859873323505e-10\\
-2.322080078125	4.17184124180505e-10\\
-2.30158203125	3.55188913492751e-10\\
-2.281083984375	4.46665125426063e-10\\
-2.2605859375	3.45609567251995e-10\\
-2.240087890625	4.56805844008176e-10\\
-2.21958984375	2.43207370063524e-10\\
-2.199091796875	5.24042873488609e-10\\
-2.17859375	2.4955948040892e-10\\
-2.158095703125	5.35554908552621e-10\\
-2.13759765625	4.18477762439172e-10\\
-2.117099609375	5.59241491836528e-10\\
-2.0966015625	4.72081887402819e-10\\
-2.076103515625	7.43507426497257e-10\\
-2.05560546875	6.99125650279821e-10\\
-2.035107421875	6.53803325058767e-10\\
-2.014609375	4.66436197817907e-10\\
-1.994111328125	3.68514141925325e-10\\
-1.97361328125	3.18738370798514e-10\\
-1.953115234375	2.54710438219726e-10\\
-1.9326171875	2.40908816831473e-10\\
-1.912119140625	2.28856112027948e-10\\
-1.89162109375	1.39108559733324e-10\\
-1.871123046875	1.87572018981215e-10\\
-1.850625	8.10378157414301e-12\\
-1.830126953125	-3.86658725012917e-11\\
-1.80962890625	8.96877129634948e-11\\
-1.789130859375	-2.00205755889345e-10\\
-1.7686328125	-7.11360450540623e-11\\
-1.748134765625	-3.46577041124654e-10\\
-1.72763671875	-2.81797052020081e-10\\
-1.707138671875	-3.41501856446096e-10\\
-1.686640625	-3.42105860017486e-10\\
-1.666142578125	-4.42079592762037e-10\\
-1.64564453125	-3.02080016129929e-10\\
-1.625146484375	-4.57198310241343e-10\\
-1.6046484375	-2.36751802415622e-10\\
-1.584150390625	-2.0558545085633e-10\\
-1.56365234375	-2.63374393868674e-10\\
-1.543154296875	-4.30755361908282e-10\\
-1.52265625	-3.78194621838415e-10\\
-1.502158203125	-4.72195885002411e-10\\
-1.48166015625	-5.49683719794173e-10\\
-1.461162109375	-6.84604092219007e-10\\
-1.4406640625	-6.07871657205208e-10\\
-1.420166015625	-8.64484630707952e-10\\
-1.39966796875	-6.79348938404469e-10\\
-1.379169921875	-6.11717455471817e-10\\
-1.358671875	-3.52101224536986e-10\\
-1.338173828125	-2.61973533386817e-10\\
-1.31767578125	-1.36911446931502e-10\\
-1.297177734375	-1.99599612413974e-10\\
-1.2766796875	-1.4709534168932e-10\\
-1.256181640625	-3.03764842927673e-10\\
-1.23568359375	-2.02652480557026e-10\\
-1.215185546875	-5.32809136339615e-10\\
-1.1946875	-4.93506418735028e-10\\
-1.174189453125	-6.93017720506212e-10\\
-1.15369140625	-6.08785543094068e-10\\
-1.133193359375	-7.14363020125244e-10\\
-1.1126953125	-4.86083785136893e-10\\
-1.092197265625	-4.77747513326311e-10\\
-1.07169921875	-9.82901042482739e-11\\
-1.051201171875	-3.06685099590597e-10\\
-1.030703125	1.08962873782646e-12\\
-1.010205078125	-9.10462832187425e-11\\
-0.989707031249999	-1.25776383181271e-10\\
-0.969208984375001	-3.55073347009747e-10\\
-0.948710937499996	-2.35510086352593e-10\\
-0.928212890624998	-6.93713731782009e-10\\
-0.90771484375	-5.29719156271047e-10\\
-0.887216796875002	-7.94512544955906e-10\\
-0.866718749999997	-6.04852668047307e-10\\
-0.846220703124999	-6.4640767600714e-10\\
-0.825722656250001	-4.53843926009694e-10\\
-0.805224609374996	-4.47212432255481e-10\\
-0.784726562499998	-2.4202870558611e-10\\
-0.764228515625	-2.37605032652364e-10\\
-0.743730468750002	-2.07657222935618e-10\\
-0.723232421874997	-2.70077444855649e-10\\
-0.702734374999999	-5.21437430597279e-10\\
-0.682236328125001	-6.2959684849771e-10\\
-0.661738281249995	-8.92590150928172e-10\\
-0.641240234374997	-8.62382201400707e-10\\
-0.620742187499999	-9.34911741260585e-10\\
-0.600244140625001	-8.96582856029391e-10\\
-0.579746093749996	-8.19979405327277e-10\\
-0.559248046874998	-6.20066465021419e-10\\
-0.53875	-5.35434996533563e-10\\
-0.518251953125002	-2.77576294833998e-10\\
-0.497753906249997	-4.45279312585127e-10\\
-0.477255859374999	-1.87413700333509e-10\\
-0.456757812500001	-3.73807952659279e-10\\
-0.436259765624996	-3.22361100708335e-10\\
-0.415761718749998	-3.17054093345844e-10\\
-0.395263671875	-2.18611000916655e-10\\
-0.374765625000002	-2.90562337173862e-10\\
-0.354267578124997	-5.47477432205259e-12\\
-0.333769531249999	-6.41289288940896e-11\\
-0.313271484375001	1.12849827505965e-10\\
-0.292773437499996	1.44379989361334e-10\\
-0.272275390624998	3.64757891906166e-10\\
-0.25177734375	2.71161443839944e-10\\
-0.231279296875002	4.98621344574324e-10\\
-0.210781249999997	4.98431780053795e-10\\
-0.190283203124999	7.87131357505651e-10\\
-0.169785156250001	6.92243789208473e-10\\
-0.149287109374995	9.04353838880984e-10\\
-0.128789062499997	9.80652025773319e-10\\
-0.108291015624999	9.71786447610254e-10\\
-0.0877929687500014	1.06153025599499e-09\\
-0.0672949218749963	1.05314768099806e-09\\
-0.0467968749999983	1.16820901523835e-09\\
-0.0262988281250003	1.14803473517525e-09\\
-0.00580078125000227	1.22722663170256e-09\\
0.0146972656250028	1.33719038078652e-09\\
0.0351953125000009	1.39661772446811e-09\\
0.0556933593749989	1.18342428868617e-09\\
0.076191406250004	1.33712243332106e-09\\
0.096689453125002	1.37930272360568e-09\\
0.1171875	1.53816336228713e-09\\
0.137685546874998	1.42208666526752e-09\\
0.158183593750003	1.68551056056153e-09\\
0.178681640625001	1.73698752896276e-09\\
0.199179687499999	1.94758187883085e-09\\
0.219677734375004	1.86976297010161e-09\\
0.240175781250002	2.17558832854747e-09\\
0.260673828125	1.99591197775801e-09\\
0.281171874999998	2.094368547946e-09\\
0.301669921875003	1.88337172294811e-09\\
0.322167968750001	2.07917434429279e-09\\
0.342666015624999	1.85606173778779e-09\\
0.363164062499997	1.99588652033232e-09\\
0.383662109375003	1.89271828573447e-09\\
0.404160156250001	2.03966766299494e-09\\
0.424658203124999	1.96001440754488e-09\\
0.445156250000004	2.25942363716883e-09\\
0.465654296875002	2.24943525508694e-09\\
0.48615234375	2.42453225578646e-09\\
0.506650390624998	2.4892927358175e-09\\
0.527148437500003	2.54684221006686e-09\\
0.547646484375001	2.65162538151762e-09\\
0.568144531249999	2.74414364867001e-09\\
0.588642578125004	2.79578756265128e-09\\
0.609140625000002	2.93674644729241e-09\\
0.629638671875	3.11819412247696e-09\\
0.650136718749998	3.24595224954066e-09\\
0.670634765625003	3.3872963892705e-09\\
0.691132812500001	3.53128748379504e-09\\
0.711630859374999	3.68238986190188e-09\\
0.732128906250004	3.80754193251859e-09\\
0.752626953125002	4.01020670564339e-09\\
0.773125	3.8296927560272e-09\\
0.793623046874998	4.09447869906455e-09\\
0.814121093750003	4.15249703856617e-09\\
0.834619140625001	4.19400870802378e-09\\
0.855117187499999	4.11455228615805e-09\\
0.875615234374997	4.50552881014116e-09\\
0.896113281250003	4.2952761437481e-09\\
0.916611328125001	4.53363539428854e-09\\
0.937109374999999	4.53020860456485e-09\\
0.957607421875004	4.61923539588084e-09\\
0.978105468750002	4.62665443845733e-09\\
0.998603515625	4.76133042047168e-09\\
1.0191015625	4.68856497391593e-09\\
1.039599609375	4.87362438837708e-09\\
1.06009765625	4.70827461309005e-09\\
1.080595703125	5.08732608925597e-09\\
1.10109375	5.05457458939743e-09\\
1.121591796875	5.33475158804985e-09\\
1.14208984375	5.32092067205556e-09\\
1.162587890625	5.26892953647476e-09\\
1.1830859375	5.43443839478097e-09\\
1.203583984375	5.27988832930854e-09\\
1.22408203125	5.32575771442961e-09\\
1.244580078125	5.45280865010593e-09\\
1.265078125	5.40937184497148e-09\\
1.285576171875	5.65788761975895e-09\\
1.30607421875	5.75609539810344e-09\\
1.326572265625	5.99709044034861e-09\\
1.3470703125	6.04559056392445e-09\\
1.367568359375	6.33285686232467e-09\\
1.38806640625	6.35769555752182e-09\\
1.408564453125	6.36531250926758e-09\\
1.4290625	6.2219179694431e-09\\
1.449560546875	6.42646588687517e-09\\
1.47005859375	5.97894834296338e-09\\
1.490556640625	6.23832964895636e-09\\
1.5110546875	6.24877490405552e-09\\
1.531552734375	6.46917649594989e-09\\
1.55205078125	6.53701111503978e-09\\
1.572548828125	6.97771129923236e-09\\
1.593046875	6.96302190975038e-09\\
1.613544921875	7.3497282486508e-09\\
1.63404296875	7.13951599278312e-09\\
1.654541015625	7.38602365076754e-09\\
1.6750390625	7.09794143936191e-09\\
1.695537109375	7.1386642240382e-09\\
1.71603515625	7.12868919058421e-09\\
1.736533203125	7.10883166929835e-09\\
1.75703125	7.30699265310155e-09\\
1.777529296875	7.48627638155341e-09\\
1.79802734375	7.74325068753367e-09\\
1.818525390625	8.06655751936093e-09\\
1.8390234375	8.34919298669442e-09\\
1.859521484375	8.59225922919569e-09\\
1.88001953125	8.73399952451865e-09\\
1.900517578125	8.77136144471865e-09\\
1.921015625	8.74995514653399e-09\\
1.941513671875	8.6025766400142e-09\\
1.96201171875	8.62637494807048e-09\\
1.982509765625	8.61603529470755e-09\\
2.0030078125	8.69923720590893e-09\\
2.023505859375	8.70259703400788e-09\\
2.04400390625	9.125290672975e-09\\
2.064501953125	9.04321796860394e-09\\
2.085	9.28390490978784e-09\\
2.105498046875	9.23657151367679e-09\\
2.12599609375	9.27017733366119e-09\\
2.146494140625	9.0734155892204e-09\\
2.1669921875	9.10906190499613e-09\\
2.187490234375	8.85417142666967e-09\\
2.20798828125	8.97210045327036e-09\\
2.228486328125	8.90604663152307e-09\\
2.248984375	8.96592175687034e-09\\
2.269482421875	8.82179893968664e-09\\
2.28998046875	8.93863411140629e-09\\
2.310478515625	8.71474830037481e-09\\
2.3309765625	8.7986299035719e-09\\
2.351474609375	8.54904202961749e-09\\
2.37197265625	8.55369215734157e-09\\
2.392470703125	8.43597108410855e-09\\
2.41296875	8.46875249373772e-09\\
2.433466796875	8.39845116111748e-09\\
2.45396484375	8.46850919474696e-09\\
2.474462890625	8.34257298028889e-09\\
2.4949609375	8.47284768318774e-09\\
2.515458984375	8.38828836535972e-09\\
2.53595703125	8.44109710177638e-09\\
2.556455078125	8.47189688867046e-09\\
2.576953125	8.52031804585033e-09\\
2.597451171875	8.55913811096679e-09\\
2.61794921875	8.47372662391233e-09\\
2.638447265625	8.35433514468392e-09\\
2.6589453125	8.2694045618087e-09\\
2.679443359375	8.08396353170509e-09\\
2.69994140625	8.02698881013463e-09\\
2.720439453125	8.06292868337255e-09\\
2.7409375	7.97163781435869e-09\\
2.761435546875	8.13097143155222e-09\\
2.78193359375	8.1213230153366e-09\\
2.802431640625	8.34271277696958e-09\\
2.8229296875	8.26408578891268e-09\\
2.843427734375	8.34344124786419e-09\\
2.86392578125	8.33436497613598e-09\\
2.884423828125	8.58721797236364e-09\\
2.904921875	8.39798374006986e-09\\
2.925419921875	8.55958164449833e-09\\
2.94591796875	8.41927108074938e-09\\
2.966416015625	8.50046483827936e-09\\
2.9869140625	8.40957925303248e-09\\
3.007412109375	8.36763924257875e-09\\
3.02791015625	8.26417982853809e-09\\
3.048408203125	8.07887432748514e-09\\
3.06890625	8.0684228349162e-09\\
3.089404296875	7.9062747146015e-09\\
3.10990234375	7.7844466588682e-09\\
3.130400390625	7.65861230349912e-09\\
3.1508984375	7.41798259218488e-09\\
3.171396484375	7.62405640578575e-09\\
3.19189453125	7.39417173054288e-09\\
3.212392578125	7.44408376435692e-09\\
3.232890625	7.44006461678264e-09\\
3.253388671875	7.41030191384428e-09\\
3.27388671875	7.23274000092565e-09\\
3.294384765625	7.04090243655242e-09\\
3.3148828125	6.92935094113134e-09\\
3.335380859375	6.53779365453731e-09\\
3.35587890625	6.57737099323852e-09\\
3.376376953125	6.17323124968108e-09\\
3.396875	6.35579269839559e-09\\
3.417373046875	6.26861553411176e-09\\
3.43787109375	6.43600876530702e-09\\
3.458369140625	6.38342488634272e-09\\
3.4788671875	6.37336581350719e-09\\
3.499365234375	6.3381741415658e-09\\
3.51986328125	6.17610878443545e-09\\
3.540361328125	6.11147172129586e-09\\
3.560859375	5.67901475396757e-09\\
3.581357421875	5.41236887745168e-09\\
3.60185546875	5.41810364307709e-09\\
3.622353515625	5.21641089145294e-09\\
3.6428515625	5.34545169050897e-09\\
3.663349609375	5.49204723445678e-09\\
3.68384765625	5.5899167804228e-09\\
3.704345703125	5.680106941793e-09\\
3.72484375	5.6106588831166e-09\\
3.745341796875	5.41134106053481e-09\\
3.76583984375	5.22669154596204e-09\\
3.786337890625	4.60474959912619e-09\\
3.8068359375	4.70987128226346e-09\\
3.827333984375	4.1410269681219e-09\\
3.84783203125	4.12993002562387e-09\\
3.868330078125	4.02983066083446e-09\\
3.888828125	3.96843026419928e-09\\
3.909326171875	4.04034855301432e-09\\
3.92982421875	4.33500335030248e-09\\
3.950322265625	4.29324239619803e-09\\
3.9708203125	4.52839331264002e-09\\
3.991318359375	4.13215250872457e-09\\
4.01181640625	3.98360463695729e-09\\
4.032314453125	3.73845720916034e-09\\
4.0528125	3.40380075978471e-09\\
4.073310546875	3.32748999706288e-09\\
4.09380859375	3.16154764904129e-09\\
4.114306640625	3.0558574972704e-09\\
4.1348046875	3.25732685220239e-09\\
4.155302734375	3.21489657279629e-09\\
4.17580078125	3.58017415292158e-09\\
4.196298828125	3.47622178977558e-09\\
4.216796875	3.29461391459089e-09\\
4.237294921875	3.27635199898007e-09\\
4.25779296875	2.79452375251667e-09\\
4.278291015625	2.62693839183826e-09\\
4.2987890625	2.32405963757839e-09\\
4.319287109375	2.31202217049855e-09\\
4.33978515625	2.16857448576177e-09\\
4.360283203125	2.24403486114777e-09\\
4.38078125	2.3757394032807e-09\\
4.401279296875	2.45244142889791e-09\\
4.42177734375	2.36618487358155e-09\\
4.442275390625	2.20730735961878e-09\\
4.4627734375	1.94307009486558e-09\\
4.483271484375	1.83901020831131e-09\\
4.50376953125	1.36809004985015e-09\\
4.524267578125	1.38061537647319e-09\\
4.544765625	1.14081282499009e-09\\
4.565263671875	1.19757544642265e-09\\
4.58576171875	1.13294484357075e-09\\
4.606259765625	1.42596137751531e-09\\
4.6267578125	1.45058722147118e-09\\
4.647255859375	1.65024270499554e-09\\
4.66775390625	1.48621292283868e-09\\
4.688251953125	1.61779012600228e-09\\
4.70875	1.24738735192704e-09\\
4.729248046875	1.08684677200848e-09\\
4.74974609375	8.891957711339e-10\\
4.770244140625	7.80608820770691e-10\\
4.7907421875	6.80577334609076e-10\\
4.811240234375	8.10359744074356e-10\\
4.83173828125	7.8329827832363e-10\\
4.852236328125	1.20715734536869e-09\\
4.872734375	1.12890149559986e-09\\
4.893232421875	1.49225144786761e-09\\
4.91373046875	1.32748077128098e-09\\
4.934228515625	1.50258145755399e-09\\
4.9547265625	1.19080930163959e-09\\
4.975224609375	1.08254063848763e-09\\
4.99572265625	8.89403584635471e-10\\
5.016220703125	9.1704188492703e-10\\
5.03671875	7.45052185868799e-10\\
5.057216796875	7.46753731490723e-10\\
5.07771484375	9.09514924672247e-10\\
5.098212890625	8.4631551440762e-10\\
5.1187109375	1.01576768374185e-09\\
5.139208984375	9.55112832938876e-10\\
5.15970703125	8.98596478613924e-10\\
5.180205078125	6.20572215426355e-10\\
5.200703125	4.47907591689857e-10\\
5.221201171875	1.40981312766107e-10\\
5.24169921875	6.03161961436297e-12\\
5.262197265625	-1.49916743689212e-10\\
5.2826953125	-1.63644008768049e-10\\
5.303193359375	-2.15832954264705e-10\\
5.32369140625	3.04277573804779e-11\\
5.344189453125	-9.25715076686112e-11\\
5.3646875	1.28689214031913e-10\\
5.385185546875	-3.82614634426182e-11\\
5.40568359375	-9.05470198226488e-11\\
5.426181640625	-3.94145315736713e-10\\
5.4466796875	-4.85380878718472e-10\\
5.467177734375	-8.53306026708234e-10\\
5.48767578125	-8.26064547200452e-10\\
5.508173828125	-8.85096018502406e-10\\
5.528671875	-8.10089734707517e-10\\
5.549169921875	-6.82107070375858e-10\\
5.56966796875	-5.75504287327419e-10\\
5.590166015625	-6.35366464805153e-10\\
5.6106640625	-4.59578143903497e-10\\
5.631162109375	-6.84283494022019e-10\\
5.65166015625	-3.84210421730328e-10\\
5.672158203125	-6.90031052483322e-10\\
5.69265625	-7.89956535886272e-10\\
5.713154296875	-8.85029861619649e-10\\
5.73365234375	-9.23875422070318e-10\\
5.754150390625	-9.7451777882251e-10\\
5.7746484375	-7.94932270383062e-10\\
5.795146484375	-7.22625438105223e-10\\
5.81564453125	-8.57819423102339e-10\\
5.836142578125	-6.80449502455728e-10\\
5.856640625	-8.5253797447034e-10\\
5.877138671875	-7.12604404692838e-10\\
5.89763671875	-8.82171542960349e-10\\
5.918134765625	-7.83867586071654e-10\\
5.9386328125	-9.32217800276505e-10\\
5.959130859375	-9.07342158573445e-10\\
5.97962890625	-9.0473347155528e-10\\
6.000126953125	-8.65600727774555e-10\\
6.020625	-7.45444319206941e-10\\
6.041123046875	-8.21858264977418e-10\\
6.06162109375	-7.60279309250917e-10\\
6.082119140625	-6.82729353621293e-10\\
6.1026171875	-8.66071737914169e-10\\
6.123115234375	-8.68521870489866e-10\\
6.14361328125	-8.9113406790202e-10\\
6.164111328125	-1.00144360002691e-09\\
6.184609375	-1.10210747378393e-09\\
6.205107421875	-1.07026087289886e-09\\
6.22560546875	-1.14067057606979e-09\\
6.246103515625	-1.04155068915624e-09\\
6.2666015625	-1.03387854268835e-09\\
6.287099609375	-8.78370661810972e-10\\
6.30759765625	-9.26237191339161e-10\\
6.328095703125	-7.16503110432243e-10\\
6.34859375	-8.09648452540599e-10\\
6.369091796875	-7.1059674809673e-10\\
6.38958984375	-7.86617206143406e-10\\
6.410087890625	-7.86281907190273e-10\\
6.4305859375	-8.59747528505601e-10\\
6.451083984375	-8.90894557357056e-10\\
6.47158203125	-9.6827791093315e-10\\
6.492080078125	-7.80972801360342e-10\\
6.512578125	-9.51659789034835e-10\\
6.533076171875	-8.98650108898594e-10\\
6.55357421875	-7.85504844285729e-10\\
6.574072265625	-7.58357455721717e-10\\
6.5945703125	-6.87354993076717e-10\\
6.615068359375	-6.42180263655137e-10\\
6.63556640625	-6.93984981636753e-10\\
6.656064453125	-5.04314379624645e-10\\
6.6765625	-7.94987011991812e-10\\
6.697060546875	-6.65425215781666e-10\\
6.71755859375	-6.63659682841323e-10\\
6.738056640625	-7.98629637877081e-10\\
6.7585546875	-5.55592696844707e-10\\
6.779052734375	-6.92449719760329e-10\\
6.79955078125	-6.77612559124806e-10\\
6.820048828125	-6.39423625339396e-10\\
6.840546875	-6.29406333574332e-10\\
6.861044921875	-5.53649097544706e-10\\
6.88154296875	-5.49567681198467e-10\\
6.902041015625	-4.85259643946893e-10\\
6.9225390625	-3.7512945784198e-10\\
6.943037109375	-3.79695156454096e-10\\
6.96353515625	-3.60959734000653e-10\\
6.984033203125	-3.93565747513028e-10\\
7.00453125	-3.20544110452638e-10\\
7.025029296875	-3.4751584254737e-10\\
7.04552734375	-3.71205900337295e-10\\
7.066025390625	-4.60527605610918e-10\\
7.0865234375	-3.28798862142449e-10\\
7.107021484375	-4.54523377946591e-10\\
7.12751953125	-3.50690180787133e-10\\
7.148017578125	-4.6156870705759e-10\\
7.168515625	-3.28542479283931e-10\\
7.189013671875	-4.71341412161486e-10\\
7.20951171875	-4.09224354035881e-10\\
7.230009765625	-5.53148001008657e-10\\
7.2505078125	-4.38266215009383e-10\\
7.271005859375	-4.70071109450909e-10\\
7.29150390625	-3.73946626961987e-10\\
7.312001953125	-4.01024834401867e-10\\
7.3325	-3.8258400496498e-10\\
7.352998046875	-5.75995572526869e-10\\
7.37349609375	-4.35941159782368e-10\\
7.393994140625	-7.70290446816126e-10\\
7.4144921875	-6.41596612135543e-10\\
7.434990234375	-8.22945746130519e-10\\
7.45548828125	-6.74168077677616e-10\\
7.475986328125	-8.4159345168093e-10\\
7.496484375	-6.72971410536928e-10\\
7.516982421875	-6.73517200116962e-10\\
7.53748046875	-5.59165026717639e-10\\
7.557978515625	-6.30094056862876e-10\\
7.5784765625	-6.10252720097248e-10\\
7.598974609375	-6.54034422925412e-10\\
7.61947265625	-7.19344937311716e-10\\
7.639970703125	-7.5863945924811e-10\\
7.66046875	-7.27939055701211e-10\\
7.680966796875	-7.31975121641124e-10\\
7.70146484375	-6.86592190564072e-10\\
7.721962890625	-5.77138646171245e-10\\
7.7424609375	-5.84877728242617e-10\\
7.762958984375	-3.90774485760495e-10\\
7.78345703125	-4.20667417094697e-10\\
7.803955078125	-3.5546326377568e-10\\
7.824453125	-4.00585649920667e-10\\
7.844951171875	-1.88292365635645e-10\\
7.86544921875	-4.40329650689224e-10\\
7.885947265625	-1.7532965976879e-10\\
7.9064453125	-1.99054254416805e-10\\
7.926943359375	-8.57765806498133e-11\\
7.94744140625	-2.74897424763178e-11\\
7.967939453125	1.76578655778857e-12\\
7.9884375	-2.37313398384511e-11\\
8.008935546875	-4.88240022273998e-11\\
8.02943359375	-8.67973550849472e-11\\
8.049931640625	-8.95293747899836e-11\\
8.0704296875	-1.78415463073374e-10\\
8.090927734375	-1.44471528217143e-10\\
8.11142578125	-8.22295910728897e-11\\
8.131923828125	1.05864376128438e-10\\
8.152421875	-1.32096918517569e-10\\
8.172919921875	8.00081855727287e-11\\
8.19341796875	2.50632524272852e-11\\
8.213916015625	1.24774263084771e-10\\
8.2344140625	-2.73566969785661e-12\\
8.254912109375	-3.570559706252e-11\\
8.27541015625	-1.10541511171475e-10\\
8.295908203125	-2.40705916109669e-10\\
8.31640625	-1.78458946211757e-10\\
8.336904296875	-2.70161263073776e-10\\
8.35740234375	-1.31172487716079e-10\\
8.377900390625	-2.10334357246758e-10\\
8.3983984375	1.20843069579157e-11\\
8.418896484375	-2.37335309689557e-10\\
8.43939453125	-6.40466590862438e-11\\
8.459892578125	-2.45031092614206e-10\\
8.480390625	-1.86157177956015e-10\\
8.500888671875	-2.52265730414998e-10\\
8.52138671875	-2.15018832189708e-10\\
8.541884765625	-1.48595103200893e-10\\
8.5623828125	-2.14319335635237e-10\\
8.582880859375	-1.98575797866765e-10\\
8.60337890625	-2.0974829320139e-10\\
8.623876953125	-2.40059242923677e-10\\
8.644375	-1.7951800636299e-10\\
8.664873046875	-2.27276668092311e-10\\
8.68537109375	-2.19144545716707e-10\\
8.705869140625	-1.4632809746956e-10\\
8.7263671875	7.02726376796953e-12\\
8.746865234375	-2.79652736942809e-12\\
8.76736328125	5.78773118590629e-11\\
8.787861328125	1.03476948694818e-11\\
8.808359375	3.69147135902889e-11\\
8.828857421875	-7.83086616552426e-11\\
8.84935546875	-1.17994863347608e-10\\
8.869853515625	-1.71592145097564e-10\\
8.8903515625	-1.74521932337292e-10\\
8.910849609375	-2.7055314132002e-10\\
8.93134765625	-1.93835542951525e-10\\
8.951845703125	-2.23946880390104e-10\\
8.97234375	-7.23680552267457e-11\\
8.992841796875	-8.37141938133395e-11\\
9.01333984375	-1.73831280274145e-11\\
9.033837890625	-1.39124367313696e-10\\
9.0543359375	-2.07571127608285e-10\\
9.074833984375	-2.46614687612031e-10\\
9.09533203125	-3.09148221713156e-10\\
9.115830078125	-3.84599334586188e-10\\
9.136328125	-4.1567178523366e-10\\
9.156826171875	-5.15679266988118e-10\\
9.17732421875	-2.86762769355612e-10\\
9.197822265625	-4.02878777470755e-10\\
9.2183203125	-2.73804475117833e-10\\
9.238818359375	-3.18276200153471e-10\\
9.25931640625	-3.9847507016944e-10\\
9.279814453125	-3.80331109378585e-10\\
9.3003125	-4.29147574453747e-10\\
9.320810546875	-4.65943971030005e-10\\
9.34130859375	-5.72802136892988e-10\\
9.361806640625	-5.64829815933714e-10\\
9.3823046875	-6.21275739369134e-10\\
9.402802734375	-5.97816998230039e-10\\
9.42330078125	-4.97192036064813e-10\\
9.443798828125	-6.29099025427011e-10\\
9.464296875	-6.87858228252666e-10\\
9.484794921875	-8.0001330362472e-10\\
9.50529296875	-8.33059669222106e-10\\
9.525791015625	-8.89231691861648e-10\\
9.5462890625	-9.86567285743452e-10\\
9.566787109375	-9.19766151188262e-10\\
9.58728515625	-9.62574800972634e-10\\
9.607783203125	-8.26629586318355e-10\\
9.62828125	-9.09649101069482e-10\\
9.648779296875	-7.48402552442895e-10\\
9.66927734375	-7.77431213203229e-10\\
9.689775390625	-6.33107279974849e-10\\
9.7102734375	-7.3744338396907e-10\\
9.730771484375	-5.69510495235072e-10\\
9.75126953125	-7.50271645019414e-10\\
9.771767578125	-7.31801732786624e-10\\
9.792265625	-7.43097843943465e-10\\
9.812763671875	-7.43634227501991e-10\\
9.83326171875	-7.93705120350466e-10\\
9.853759765625	-5.68031163073831e-10\\
9.8742578125	-7.568871478435e-10\\
9.894755859375	-3.81709140371211e-10\\
9.91525390625	-6.02160015232274e-10\\
9.935751953125	-3.51463112058977e-10\\
9.95625	-5.5255693674002e-10\\
9.976748046875	-4.23846707381878e-10\\
9.99724609375	-5.9601681612217e-10\\
10.017744140625	-6.0674656819193e-10\\
10.0382421875	-5.77444418102398e-10\\
10.058740234375	-5.62634386407183e-10\\
10.07923828125	-5.85967168667184e-10\\
10.099736328125	-5.84077869082055e-10\\
10.120234375	-4.05202811458407e-10\\
10.140732421875	-3.95690112439204e-10\\
10.16123046875	-4.2280116920691e-10\\
10.181728515625	-3.88317056301219e-10\\
10.2022265625	-5.69693066276407e-10\\
10.222724609375	-6.0734791005697e-10\\
10.24322265625	-6.512396364562e-10\\
10.263720703125	-8.61621416845157e-10\\
10.28421875	-8.25584674813385e-10\\
10.304716796875	-9.0376484436449e-10\\
10.32521484375	-8.94321804488293e-10\\
10.345712890625	-9.24518649240047e-10\\
10.3662109375	-7.85271140688671e-10\\
10.386708984375	-9.06475591062892e-10\\
10.40720703125	-7.86060937372108e-10\\
10.427705078125	-8.63558175524711e-10\\
10.448203125	-9.02214002612772e-10\\
10.468701171875	-9.75563234261121e-10\\
10.48919921875	-1.03421509726264e-09\\
10.509697265625	-1.00661592003783e-09\\
10.5301953125	-1.06614283711669e-09\\
10.550693359375	-1.13659210960898e-09\\
10.57119140625	-1.00188938691581e-09\\
10.591689453125	-1.09939611339849e-09\\
10.6121875	-1.04558305880029e-09\\
10.632685546875	-1.04490162671267e-09\\
10.65318359375	-1.00762486367332e-09\\
10.673681640625	-1.12419248698181e-09\\
10.6941796875	-1.07244840287474e-09\\
10.714677734375	-1.17581207838806e-09\\
10.73517578125	-1.11724907036205e-09\\
10.755673828125	-1.14713517628928e-09\\
10.776171875	-1.11764930987169e-09\\
10.796669921875	-1.09019783408355e-09\\
10.81716796875	-1.05398667940313e-09\\
10.837666015625	-9.98234553331035e-10\\
10.8581640625	-1.01776497670955e-09\\
10.878662109375	-9.29440879607476e-10\\
10.89916015625	-1.0198156764566e-09\\
10.919658203125	-1.01492941596151e-09\\
10.94015625	-1.09936820776231e-09\\
10.960654296875	-9.02303652237929e-10\\
10.98115234375	-1.01449325490506e-09\\
11.001650390625	-8.8188631282882e-10\\
11.0221484375	-9.38347218456417e-10\\
11.042646484375	-9.04009262406757e-10\\
11.06314453125	-6.78265554825804e-10\\
11.083642578125	-7.32110208745411e-10\\
11.104140625	-6.31098178365125e-10\\
11.124638671875	-7.05123765242189e-10\\
11.14513671875	-7.6800561607053e-10\\
11.165634765625	-7.22351649972179e-10\\
11.1861328125	-8.4088780054228e-10\\
11.206630859375	-8.18076807883267e-10\\
11.22712890625	-8.63245608028347e-10\\
11.247626953125	-7.88347059463081e-10\\
11.268125	-7.95829154043377e-10\\
11.288623046875	-7.59224680340875e-10\\
11.30912109375	-6.33420720502738e-10\\
11.329619140625	-5.59436984271365e-10\\
11.3501171875	-5.39404317956316e-10\\
11.370615234375	-4.4883541266996e-10\\
11.39111328125	-6.13111436517912e-10\\
11.411611328125	-4.95479921684743e-10\\
11.432109375	-5.30605331836158e-10\\
11.452607421875	-4.55395132692172e-10\\
11.47310546875	-6.19687287134263e-10\\
11.493603515625	-5.66469245368677e-10\\
11.5141015625	-4.71882626397422e-10\\
11.534599609375	-4.1364964950183e-10\\
11.55509765625	-3.79251756783622e-10\\
11.575595703125	-1.81453989934079e-10\\
11.59609375	-1.56204511647842e-10\\
11.616591796875	-3.72713811323399e-11\\
11.63708984375	-4.99411543183654e-11\\
11.657587890625	6.56864033258502e-12\\
11.6780859375	-1.32551622520048e-12\\
11.698583984375	-1.54661982732409e-11\\
11.71908203125	-9.46178976523766e-11\\
11.739580078125	-5.695909574408e-11\\
11.760078125	-2.10161305353739e-10\\
11.780576171875	-8.38117950889729e-11\\
11.80107421875	-7.79605279985335e-11\\
11.821572265625	3.29956505687922e-11\\
11.8420703125	1.12020668990875e-10\\
11.862568359375	1.59436484250644e-12\\
11.88306640625	9.17947003972496e-11\\
11.903564453125	2.46137218147662e-12\\
11.9240625	-6.24336384100748e-11\\
11.944560546875	9.70395963222051e-11\\
11.96505859375	7.3821990161139e-11\\
11.985556640625	2.0131335724208e-10\\
12.0060546875	1.45432671579458e-10\\
12.026552734375	2.45186206286263e-10\\
12.04705078125	3.28518500888375e-10\\
12.067548828125	1.8112029077129e-10\\
12.088046875	3.57252930993659e-10\\
12.108544921875	2.5297730120448e-10\\
12.12904296875	3.72282179243338e-10\\
12.149541015625	1.78242615742089e-10\\
12.1700390625	2.73018833655845e-10\\
12.190537109375	8.37345942420723e-11\\
12.21103515625	2.20623104693775e-10\\
12.231533203125	1.0308705621464e-10\\
12.25203125	2.83587072670496e-10\\
12.272529296875	1.3863081424239e-10\\
12.29302734375	1.94843378721002e-10\\
12.313525390625	2.14944126052088e-10\\
12.3340234375	1.71044242420304e-10\\
12.354521484375	1.2857819163226e-10\\
12.37501953125	1.87538792714808e-10\\
12.395517578125	7.10450817124197e-11\\
12.416015625	3.47588646345261e-10\\
12.436513671875	1.13076601129076e-10\\
12.45701171875	4.18124860476338e-10\\
12.477509765625	2.85509837222873e-10\\
12.4980078125	3.56840254554722e-10\\
12.518505859375	2.63746232801343e-10\\
12.53900390625	1.97681962964976e-10\\
12.559501953125	1.81770019862762e-10\\
12.58	2.99835550952225e-10\\
12.600498046875	1.88215775851637e-10\\
12.62099609375	2.03415334487163e-10\\
12.641494140625	2.42869353364145e-10\\
12.6619921875	2.260490921335e-10\\
12.682490234375	2.59774254436897e-10\\
12.70298828125	3.98361299467724e-10\\
12.723486328125	3.10024429670339e-10\\
12.743984375	3.53749413936263e-10\\
12.764482421875	3.91635781865122e-10\\
12.78498046875	3.76693411449065e-10\\
12.805478515625	3.76891853990365e-10\\
12.8259765625	3.6719320423784e-10\\
12.846474609375	3.58262828638759e-10\\
12.86697265625	2.28117277790167e-10\\
12.887470703125	2.23915558342099e-10\\
12.90796875	2.21916137582853e-10\\
12.928466796875	3.81085735202703e-10\\
12.94896484375	2.94948830178762e-10\\
12.969462890625	3.73936591708142e-10\\
12.9899609375	4.13224320157013e-10\\
13.010458984375	5.57113742546314e-10\\
13.03095703125	4.76729246122449e-10\\
13.051455078125	3.93417474443282e-10\\
13.071953125	3.59328649326813e-10\\
13.092451171875	4.74682428677671e-10\\
13.11294921875	2.29396017077067e-10\\
13.133447265625	4.34369421663615e-10\\
13.1539453125	3.36836460860103e-10\\
13.174443359375	4.74079764612763e-10\\
13.19494140625	4.09508819655923e-10\\
13.215439453125	5.56439895999796e-10\\
13.2359375	5.14881098167574e-10\\
13.256435546875	4.94892232193315e-10\\
13.27693359375	3.81326666023208e-10\\
13.297431640625	4.52381396237664e-10\\
13.3179296875	3.720624171005e-10\\
13.338427734375	4.38803726192109e-10\\
13.35892578125	3.68775072383372e-10\\
13.379423828125	4.2587193816541e-10\\
13.399921875	3.90306560863826e-10\\
13.420419921875	4.11644686238506e-10\\
13.44091796875	5.17933377771847e-10\\
13.461416015625	3.99508829521917e-10\\
13.4819140625	4.43823549558466e-10\\
13.502412109375	3.05013376900087e-10\\
13.52291015625	3.23658566989452e-10\\
13.543408203125	2.36448107145218e-10\\
13.56390625	2.36596259240496e-10\\
13.584404296875	3.5010677390218e-10\\
13.60490234375	2.211799747404e-10\\
13.625400390625	2.99169630880843e-10\\
13.6458984375	2.676371330091e-10\\
13.666396484375	2.75081785848325e-10\\
13.68689453125	2.59705011445295e-10\\
13.707392578125	2.23240224554674e-10\\
13.727890625	2.35340102410704e-10\\
13.748388671875	2.75852099560911e-10\\
13.76888671875	2.7144607337811e-10\\
13.789384765625	2.2724127590093e-10\\
13.8098828125	2.65047450786201e-10\\
13.830380859375	2.24409911751952e-10\\
13.85087890625	2.52889366653092e-10\\
13.871376953125	1.04736151317954e-10\\
13.891875	2.37720316365309e-10\\
13.912373046875	-4.36567423849075e-12\\
13.93287109375	1.64761299609775e-10\\
13.953369140625	5.24490684123411e-11\\
13.9738671875	1.66423125489302e-10\\
13.994365234375	2.16201427164015e-10\\
14.01486328125	9.65425577982084e-11\\
14.035361328125	4.76350171460472e-11\\
14.055859375	1.36300470250353e-10\\
14.076357421875	-1.68643995764055e-10\\
14.09685546875	-1.04520298599179e-10\\
14.117353515625	-2.93995557739859e-10\\
14.1378515625	-3.29379576995136e-10\\
14.158349609375	-4.78046454829852e-10\\
14.17884765625	-5.05809504244826e-10\\
14.199345703125	-3.92646925286233e-10\\
14.21984375	-3.18879594225789e-10\\
14.240341796875	-3.61400408098155e-10\\
14.26083984375	-2.1994328516683e-10\\
14.281337890625	-3.19302799429027e-10\\
14.3018359375	-1.92537974603498e-10\\
14.322333984375	-2.86235611227754e-10\\
14.34283203125	-3.34943420310446e-10\\
14.363330078125	-2.06656765552827e-10\\
14.383828125	-3.35050965803413e-10\\
14.404326171875	-2.45180178070292e-10\\
14.42482421875	-2.06685267824881e-10\\
14.445322265625	-4.41865610213812e-10\\
14.4658203125	-2.58309806406066e-10\\
14.486318359375	-4.26901298365531e-10\\
14.50681640625	-3.02857639536418e-10\\
14.527314453125	-3.72030463249795e-10\\
14.5478125	-4.502501839657e-10\\
14.568310546875	-3.6132431643432e-10\\
14.58880859375	-5.28788686764821e-10\\
14.609306640625	-3.5192406352817e-10\\
14.6298046875	-5.06756229233719e-10\\
14.650302734375	-3.44960692819718e-10\\
14.67080078125	-4.30773186221016e-10\\
14.691298828125	-3.5826206709986e-10\\
14.711796875	-5.33964786166406e-10\\
14.732294921875	-3.54039567037226e-10\\
14.75279296875	-4.84633235977361e-10\\
14.773291015625	-3.90364373107151e-10\\
14.7937890625	-3.51663125327453e-10\\
14.814287109375	-4.29745823196512e-10\\
14.83478515625	-4.00177386667627e-10\\
14.855283203125	-4.19644088875909e-10\\
14.87578125	-5.24342968751832e-10\\
14.896279296875	-4.59792543421901e-10\\
14.91677734375	-6.46246070159331e-10\\
14.937275390625	-3.88020382791788e-10\\
14.9577734375	-5.92101335428953e-10\\
14.978271484375	-4.19487228330103e-10\\
14.99876953125	-4.9994463127221e-10\\
15.019267578125	-3.36154574525987e-10\\
15.039765625	-3.45727097784996e-10\\
15.060263671875	-3.86222888560941e-10\\
15.08076171875	-3.88427996839579e-10\\
15.101259765625	-3.82985570845663e-10\\
15.1217578125	-5.1166954075282e-10\\
15.142255859375	-5.0875336261859e-10\\
15.16275390625	-4.95276752459642e-10\\
15.183251953125	-4.64911549701079e-10\\
15.20375	-4.97919040717408e-10\\
15.224248046875	-3.01640481677908e-10\\
15.24474609375	-5.18762861906284e-10\\
15.265244140625	-3.94224019382898e-10\\
15.2857421875	-5.00940937651774e-10\\
15.306240234375	-4.76497182166072e-10\\
15.32673828125	-4.71209919467554e-10\\
15.347236328125	-4.58655254883314e-10\\
15.367734375	-4.43982728663595e-10\\
15.388232421875	-4.1515587951312e-10\\
15.40873046875	-3.68137237943698e-10\\
15.429228515625	-5.06721611336889e-10\\
15.4497265625	-4.53753623841895e-10\\
15.470224609375	-4.08605256906583e-10\\
15.49072265625	-5.0910075219059e-10\\
15.511220703125	-5.73594637619974e-10\\
15.53171875	-4.427186130599e-10\\
15.552216796875	-4.73855994615004e-10\\
15.57271484375	-4.29746504639085e-10\\
15.593212890625	-4.08878833659389e-10\\
15.6137109375	-2.75499859361235e-10\\
15.634208984375	-3.690087238018e-10\\
15.65470703125	-2.88687913104257e-10\\
15.675205078125	-4.52380969473558e-10\\
15.695703125	-3.59605797733224e-10\\
15.716201171875	-5.77831786421307e-10\\
15.73669921875	-5.12730141977916e-10\\
15.757197265625	-6.29726280830962e-10\\
15.7776953125	-5.00684561851388e-10\\
15.798193359375	-4.46769289148183e-10\\
15.81869140625	-4.23751405664405e-10\\
15.839189453125	-3.67794799960099e-10\\
15.8596875	-2.90814577126674e-10\\
15.880185546875	-3.20250943243274e-10\\
15.90068359375	-2.48897432434038e-10\\
15.921181640625	-2.87620444476057e-10\\
15.9416796875	-4.07522366750271e-10\\
15.962177734375	-3.52641050435102e-10\\
15.98267578125	-5.30497508690173e-10\\
16.003173828125	-3.46085828975461e-10\\
16.023671875	-3.4987682951801e-10\\
16.044169921875	-3.58255323555033e-10\\
16.06466796875	-2.56649190337259e-10\\
16.085166015625	-3.75734998465492e-10\\
16.1056640625	-1.96512139449219e-10\\
16.126162109375	-2.44671946623264e-10\\
16.14666015625	-1.1831070837594e-10\\
16.167158203125	-1.88709774924762e-10\\
16.18765625	-2.0943261519231e-10\\
16.208154296875	-2.72793058788559e-10\\
16.22865234375	-3.07517314410051e-10\\
16.249150390625	-3.21999376308089e-10\\
16.2696484375	-3.16112345206104e-10\\
16.290146484375	-3.74947358120861e-10\\
16.31064453125	-3.91309707235799e-10\\
16.331142578125	-2.99432229671866e-10\\
16.351640625	-3.59991625791386e-10\\
16.372138671875	-1.70472790870965e-10\\
16.39263671875	-3.37938947325257e-10\\
16.413134765625	-8.42447735792927e-11\\
16.4336328125	-2.15603698955672e-10\\
16.454130859375	-3.76048513549705e-11\\
16.47462890625	-5.22003084145112e-11\\
16.495126953125	-2.49384091663499e-11\\
16.515625	-5.45861854844128e-11\\
16.536123046875	2.54508369981951e-12\\
16.55662109375	-9.3611644883111e-11\\
16.577119140625	1.56408564237535e-10\\
16.5976171875	7.08811885880966e-11\\
16.618115234375	3.04032205298583e-10\\
16.63861328125	2.53942715112105e-10\\
16.659111328125	3.12526901354587e-10\\
16.679609375	3.54160029031582e-10\\
16.700107421875	2.53578793051951e-10\\
16.72060546875	1.60594757256079e-10\\
16.741103515625	1.62578620684509e-10\\
16.7616015625	-2.15722652048189e-11\\
16.782099609375	1.45435339200241e-10\\
16.80259765625	6.68697525777161e-11\\
16.823095703125	2.18448580636522e-10\\
16.84359375	2.35190834424027e-10\\
16.864091796875	1.62308359404287e-10\\
16.88458984375	3.38054105270557e-10\\
16.905087890625	3.20246060021513e-10\\
16.9255859375	2.24670364089196e-10\\
16.946083984375	1.90319075503742e-10\\
16.96658203125	-7.08180288405509e-11\\
16.987080078125	1.36586657393432e-10\\
17.007578125	3.76735696876099e-11\\
17.028076171875	5.41790442844794e-11\\
17.04857421875	2.46602126715012e-10\\
17.069072265625	2.09887144547521e-10\\
17.0895703125	3.05211156860472e-10\\
17.110068359375	2.21596404682044e-10\\
17.13056640625	3.39472943809701e-10\\
17.151064453125	1.37390598487599e-10\\
17.1715625	2.91973660809056e-10\\
17.192060546875	1.77305918139658e-10\\
17.21255859375	1.81378942539052e-10\\
17.233056640625	4.88874801212326e-11\\
17.2535546875	1.50748174601939e-10\\
17.274052734375	9.23061504852391e-11\\
17.29455078125	1.13299383531985e-10\\
17.315048828125	1.72913632602524e-10\\
17.335546875	2.40237809767719e-10\\
17.356044921875	2.12956771680094e-10\\
17.37654296875	3.23509904411293e-10\\
17.397041015625	8.96458923960228e-11\\
17.4175390625	2.0253100054562e-10\\
17.438037109375	-3.48545931742277e-11\\
17.45853515625	3.2478028413866e-10\\
17.479033203125	1.66617423500335e-11\\
17.49953125	2.40934936417739e-10\\
17.520029296875	2.01721684885642e-11\\
17.54052734375	1.93494834519064e-10\\
17.561025390625	2.21452054130806e-10\\
17.5815234375	1.64693577847074e-10\\
17.602021484375	1.22382135392836e-10\\
17.62251953125	1.42582447170448e-10\\
17.643017578125	5.26123502627918e-11\\
17.663515625	-8.36066743912237e-11\\
17.684013671875	-2.90064804521463e-11\\
17.70451171875	-5.69157898811171e-11\\
17.725009765625	-9.61905690123333e-11\\
17.7455078125	4.49936946735308e-11\\
17.766005859375	-9.98416332708403e-11\\
17.78650390625	3.04060509064434e-11\\
17.807001953125	6.93576694880177e-12\\
17.8275	-5.58688218011873e-12\\
17.847998046875	-1.00504305965849e-10\\
17.86849609375	-6.9572176035682e-11\\
17.888994140625	-1.22039851573311e-10\\
17.9094921875	-1.91120392600455e-10\\
17.929990234375	-2.25187151034796e-10\\
17.95048828125	-1.84300109720613e-10\\
17.970986328125	-1.51447244780315e-10\\
17.991484375	-4.22918552675909e-11\\
18.011982421875	-1.21544061434012e-10\\
18.03248046875	-4.5238811496632e-11\\
18.052978515625	-7.21689944357999e-11\\
18.0734765625	-8.43822174987908e-11\\
18.093974609375	-1.64888007187527e-10\\
18.11447265625	-1.10330481143818e-10\\
18.134970703125	-1.53367543860308e-10\\
18.15546875	-1.76721872284895e-10\\
18.175966796875	3.54480192017876e-11\\
18.19646484375	-1.15903627884599e-10\\
18.216962890625	1.38720335209186e-10\\
18.2374609375	2.69711918740267e-11\\
18.257958984375	1.11378867821232e-10\\
18.27845703125	-6.7564949664372e-11\\
18.298955078125	-5.27295568506215e-11\\
18.319453125	-1.04865837427056e-10\\
18.339951171875	-8.8828325332323e-11\\
18.36044921875	-2.09110806202378e-10\\
18.380947265625	-2.45391992620529e-11\\
18.4014453125	-1.30771649508457e-10\\
18.421943359375	5.86956139774556e-11\\
18.44244140625	3.27595477346544e-11\\
18.462939453125	4.86650441565349e-11\\
18.4834375	6.51398176688094e-11\\
18.503935546875	-4.99242475903542e-11\\
18.52443359375	-1.78396962921763e-10\\
18.544931640625	-1.68290713826648e-10\\
18.5654296875	-4.42844100759079e-10\\
18.585927734375	-2.7517918520072e-10\\
18.60642578125	-4.42585160009369e-10\\
18.626923828125	-3.37978865334761e-10\\
18.647421875	-3.35698407363491e-10\\
18.667919921875	-3.2444388838966e-10\\
18.68841796875	-2.55394912642752e-10\\
18.708916015625	-1.51788145065554e-10\\
18.7294140625	-2.84641451565541e-10\\
18.749912109375	-2.18764845555252e-10\\
18.77041015625	-2.77802125527065e-10\\
18.790908203125	-2.79772678755664e-10\\
18.81140625	-3.43693467327825e-10\\
18.831904296875	-4.4197188945096e-10\\
18.85240234375	-2.37502781051057e-10\\
18.872900390625	-5.0497069089127e-10\\
18.8933984375	-2.89263418406425e-10\\
18.913896484375	-5.81541566318404e-10\\
18.93439453125	-3.65683407157524e-10\\
18.954892578125	-6.53486199521574e-10\\
18.975390625	-5.69043731116781e-10\\
18.995888671875	-6.38763406972916e-10\\
19.01638671875	-5.35793382032185e-10\\
19.036884765625	-7.28262132378874e-10\\
19.0573828125	-5.64537411524714e-10\\
19.077880859375	-6.8540201264855e-10\\
19.09837890625	-5.36248243076286e-10\\
19.118876953125	-6.1849024966352e-10\\
19.139375	-5.6607509754063e-10\\
19.159873046875	-5.71275523917135e-10\\
19.18037109375	-7.36794840530733e-10\\
19.200869140625	-7.86251964011567e-10\\
19.2213671875	-7.14155547074632e-10\\
19.241865234375	-8.07767549264522e-10\\
19.26236328125	-6.31550366073815e-10\\
19.282861328125	-7.35053412642929e-10\\
19.303359375	-5.33299431319869e-10\\
19.323857421875	-5.50445554960551e-10\\
19.34435546875	-6.22459234949957e-10\\
19.364853515625	-5.51728866323213e-10\\
19.3853515625	-5.47906532176555e-10\\
19.405849609375	-5.99279952234562e-10\\
19.42634765625	-6.03515422949314e-10\\
19.446845703125	-6.64359891713548e-10\\
19.46734375	-5.25363514462899e-10\\
19.487841796875	-6.91731608965226e-10\\
19.50833984375	-6.68718807622587e-10\\
19.528837890625	-6.12236751846242e-10\\
19.5493359375	-7.26423909376456e-10\\
19.569833984375	-6.75833017899084e-10\\
19.59033203125	-7.91662772328923e-10\\
19.610830078125	-5.90257706730045e-10\\
19.631328125	-7.30115139899489e-10\\
19.651826171875	-6.80603463038362e-10\\
19.67232421875	-6.49571549477902e-10\\
19.692822265625	-6.59076320272057e-10\\
19.7133203125	-8.42161645189765e-10\\
19.733818359375	-7.62219090295979e-10\\
19.75431640625	-9.89344376083583e-10\\
19.774814453125	-9.46270107076764e-10\\
19.7953125	-9.71846241310155e-10\\
19.815810546875	-8.45641241122449e-10\\
19.83630859375	-8.42816416135895e-10\\
19.856806640625	-6.13557735415303e-10\\
19.8773046875	-6.74904929295603e-10\\
19.897802734375	-4.56061302231431e-10\\
19.91830078125	-7.43926991353959e-10\\
19.938798828125	-6.17363179171479e-10\\
19.959296875	-7.94387908201134e-10\\
19.979794921875	-6.98996399389465e-10\\
20.00029296875	-8.72494853634853e-10\\
20.020791015625	-7.6971810899526e-10\\
20.0412890625	-7.96019292395514e-10\\
20.061787109375	-6.33785938381106e-10\\
20.08228515625	-5.29905148560073e-10\\
20.102783203125	-3.53321869878286e-10\\
20.12328125	-3.06329127814974e-10\\
20.143779296875	-1.99266336854603e-10\\
20.16427734375	-2.58983699592428e-10\\
20.184775390625	-2.84056920928152e-10\\
20.2052734375	-4.21250093655079e-10\\
20.225771484375	-3.66637751161527e-10\\
20.24626953125	-4.98796992153098e-10\\
20.266767578125	-3.26475865752727e-10\\
20.287265625	-5.3930638726016e-10\\
20.307763671875	-2.91573455544644e-10\\
20.32826171875	-2.99795306292802e-10\\
20.348759765625	-2.49721698245173e-10\\
20.3692578125	-3.0898600255337e-10\\
20.389755859375	-3.53426260213206e-10\\
20.41025390625	-4.52566357918825e-10\\
20.430751953125	-4.43489550710303e-10\\
20.45125	-5.89483036139816e-10\\
20.471748046875	-5.03061920796467e-10\\
20.49224609375	-3.63489417963848e-10\\
20.512744140625	-3.41972953207265e-10\\
20.5332421875	-2.47779994307105e-10\\
20.553740234375	-2.1101445983069e-10\\
20.57423828125	-2.30094084101345e-10\\
20.594736328125	-2.2062764893926e-10\\
20.615234375	-3.90250210456427e-10\\
20.635732421875	-4.32876807282891e-10\\
20.65623046875	-3.27270684414456e-10\\
20.676728515625	-5.12471340544377e-10\\
20.6972265625	-2.41253906362395e-10\\
20.717724609375	-4.37992925071564e-10\\
20.73822265625	-1.86937445471068e-10\\
20.758720703125	-3.34289297156375e-10\\
20.77921875	-8.910716500472e-11\\
20.799716796875	-2.55114048228666e-10\\
20.82021484375	-7.21582210891828e-11\\
20.840712890625	-2.52627695795615e-10\\
20.8612109375	-2.38051926764364e-10\\
20.881708984375	-3.80531127249562e-10\\
20.90220703125	-2.7179657874604e-10\\
20.922705078125	-4.83209273122945e-10\\
20.943203125	-3.17082662423134e-10\\
20.963701171875	-3.83743601446063e-10\\
20.98419921875	-3.66340335978825e-10\\
21.004697265625	-3.54573614902674e-10\\
21.0251953125	-2.01238146822278e-10\\
21.045693359375	-1.84323444705977e-10\\
21.06619140625	-5.52643049960948e-11\\
21.086689453125	-1.61454033900987e-10\\
21.1071875	-8.43720368440904e-11\\
21.127685546875	-2.3950577343947e-10\\
21.14818359375	-1.49693353229788e-10\\
21.168681640625	-1.73698506630502e-10\\
21.1891796875	-6.68596266066161e-11\\
21.209677734375	-1.14722309053926e-10\\
21.23017578125	7.79762318624399e-12\\
21.250673828125	-2.14109726345136e-11\\
21.271171875	1.05032984596803e-10\\
21.291669921875	-3.17012129397537e-11\\
21.31216796875	5.60363188796094e-11\\
21.332666015625	3.77622192036852e-11\\
21.3531640625	-1.67804112004546e-10\\
21.373662109375	1.28073428309407e-11\\
21.39416015625	-1.34828418066882e-10\\
21.414658203125	8.54941238627452e-11\\
21.43515625	-4.0654845386216e-11\\
21.455654296875	2.10142219665537e-10\\
21.47615234375	1.63046593380805e-10\\
21.496650390625	3.16878214751408e-10\\
21.5171484375	2.49606597183917e-10\\
21.537646484375	5.7234013463335e-10\\
21.55814453125	2.45682735861343e-10\\
21.578642578125	5.23739844921695e-10\\
21.599140625	4.0224623082242e-10\\
21.619638671875	5.09797815881531e-10\\
21.64013671875	5.31212921364912e-10\\
21.660634765625	7.59914325047069e-10\\
21.6811328125	8.2657342034731e-10\\
21.701630859375	9.56153688069532e-10\\
21.72212890625	8.95468430352248e-10\\
21.742626953125	9.87960136993606e-10\\
21.763125	9.08064196335014e-10\\
21.783623046875	8.66137573383286e-10\\
21.80412109375	7.21855501873456e-10\\
21.824619140625	5.43843833415931e-10\\
21.8451171875	5.05945043847305e-10\\
21.865615234375	5.22512681164374e-10\\
21.88611328125	5.26617837732046e-10\\
21.906611328125	6.09959971429226e-10\\
21.927109375	6.39825728283659e-10\\
21.947607421875	7.06996647357123e-10\\
21.96810546875	6.98278285637093e-10\\
21.988603515625	6.76596965973863e-10\\
22.0091015625	6.98646320056689e-10\\
22.029599609375	6.6102379595249e-10\\
22.05009765625	7.52065399457675e-10\\
22.070595703125	5.86581088357742e-10\\
22.09109375	6.29902900576563e-10\\
22.111591796875	5.44265358981095e-10\\
22.13208984375	6.34263305562651e-10\\
22.152587890625	5.45227898677499e-10\\
22.1730859375	6.12284938198737e-10\\
22.193583984375	4.04717509552198e-10\\
22.21408203125	6.86206552917082e-10\\
22.234580078125	5.24639635255164e-10\\
22.255078125	7.19376183494079e-10\\
22.275576171875	5.91330106602818e-10\\
22.29607421875	8.41822135962339e-10\\
22.316572265625	7.45166897015701e-10\\
22.3370703125	7.28593785241829e-10\\
22.357568359375	6.09843462355981e-10\\
22.37806640625	5.59362345991999e-10\\
22.398564453125	5.06580834076725e-10\\
22.4190625	5.21966531650258e-10\\
22.439560546875	4.43430365864962e-10\\
22.46005859375	6.13484337381553e-10\\
22.480556640625	5.54978739591596e-10\\
22.5010546875	7.28001505062172e-10\\
22.521552734375	5.77699060007984e-10\\
22.54205078125	6.07941575296138e-10\\
22.562548828125	4.18454945742215e-10\\
22.583046875	4.28477267471725e-10\\
22.603544921875	5.14985906084908e-10\\
22.62404296875	4.17766340116979e-10\\
22.644541015625	5.10583492441789e-10\\
22.6650390625	5.08725665157977e-10\\
22.685537109375	4.98342031430093e-10\\
22.70603515625	4.3198196383569e-10\\
22.726533203125	4.02138476566374e-10\\
22.74703125	4.03097017789345e-10\\
22.767529296875	2.83099501327934e-10\\
22.78802734375	4.29038399234662e-10\\
22.808525390625	9.627700108699e-11\\
22.8290234375	2.31093994554223e-10\\
22.849521484375	2.36686485382125e-10\\
22.87001953125	3.70620940782492e-10\\
22.890517578125	4.10180241531234e-10\\
22.911015625	5.05363556820035e-10\\
22.931513671875	3.69450472913729e-10\\
22.95201171875	4.06556029211888e-10\\
22.972509765625	2.49219378294031e-10\\
22.9930078125	2.83073453644199e-10\\
23.013505859375	2.31324863784675e-10\\
23.03400390625	1.55630507574625e-10\\
23.054501953125	2.24878321672733e-10\\
23.075	1.1424927808979e-10\\
23.095498046875	2.29583003733359e-10\\
23.11599609375	3.90514450337572e-10\\
23.136494140625	4.17280567886664e-10\\
23.1569921875	3.94673322971422e-10\\
23.177490234375	2.5333892096246e-10\\
23.19798828125	3.36313762607013e-11\\
23.218486328125	2.38658209381288e-10\\
23.238984375	-1.43317880576501e-10\\
23.259482421875	1.40594936464816e-10\\
23.27998046875	-1.9711309843533e-11\\
23.300478515625	1.29761331776476e-10\\
23.3209765625	3.89827132580914e-12\\
23.341474609375	1.71047982339716e-10\\
23.36197265625	2.25327886841779e-11\\
23.382470703125	2.12798852604459e-10\\
23.40296875	-9.46127421523176e-11\\
23.423466796875	1.47076360121637e-10\\
23.44396484375	2.56555595206731e-11\\
23.464462890625	1.63828219939658e-10\\
23.4849609375	2.30303149809822e-10\\
23.505458984375	2.76765537157686e-10\\
23.52595703125	2.7459683353591e-10\\
23.546455078125	3.77758111674868e-10\\
23.566953125	3.37303742528646e-10\\
23.587451171875	3.31262430177773e-10\\
23.60794921875	2.51596070556736e-10\\
23.628447265625	1.72676555317978e-10\\
23.6489453125	4.3639479191795e-11\\
23.669443359375	1.57900543896379e-10\\
23.68994140625	4.74331125528718e-12\\
23.710439453125	9.84580480326465e-11\\
23.7309375	5.73662126136864e-11\\
23.751435546875	1.10603070985482e-10\\
23.77193359375	3.28385628604477e-11\\
23.792431640625	1.22634743301783e-10\\
23.8129296875	4.2987117752369e-11\\
23.833427734375	1.29534632622077e-10\\
23.85392578125	1.74905897191833e-10\\
23.874423828125	6.16701908823797e-12\\
23.894921875	1.99244726371961e-10\\
23.915419921875	2.92045833491486e-11\\
23.93591796875	1.87469705984246e-10\\
23.956416015625	-1.33135602867365e-11\\
23.9769140625	6.46895422364143e-11\\
23.997412109375	-2.42438078309939e-10\\
24.01791015625	-2.18249783678776e-10\\
24.038408203125	-5.82470141782317e-10\\
24.05890625	-2.22139238515355e-10\\
24.079404296875	-4.88101679149214e-10\\
24.09990234375	-3.91154019218774e-10\\
24.120400390625	-3.28868824815939e-10\\
24.1408984375	-2.88200094496017e-10\\
24.161396484375	-4.85335972065858e-10\\
24.18189453125	-3.90278897304693e-10\\
24.202392578125	-5.62480997827146e-10\\
24.222890625	-5.3139132452437e-10\\
24.243388671875	-7.01559842013176e-10\\
24.26388671875	-6.42585749475967e-10\\
24.284384765625	-6.09397669095858e-10\\
24.3048828125	-6.58733334008543e-10\\
24.325380859375	-5.9288327687076e-10\\
24.34587890625	-5.47252630887146e-10\\
24.366376953125	-5.78380245830339e-10\\
24.386875	-6.00099106844486e-10\\
24.407373046875	-5.13146746917822e-10\\
24.42787109375	-4.99525958709228e-10\\
24.448369140625	-4.28414956651807e-10\\
24.4688671875	-4.21443257640527e-10\\
24.489365234375	-3.898552919716e-10\\
24.50986328125	-3.96255386932745e-10\\
24.530361328125	-4.59090788267632e-10\\
24.550859375	-5.37910860378626e-10\\
24.571357421875	-3.83728131424409e-10\\
24.59185546875	-5.79429755552124e-10\\
24.612353515625	-5.010259589138e-10\\
24.6328515625	-5.09027130481661e-10\\
24.653349609375	-4.64893901982145e-10\\
24.67384765625	-5.2649918502243e-10\\
24.694345703125	-4.07835654002413e-10\\
24.71484375	-4.10794983235772e-10\\
24.735341796875	-3.14192521238002e-10\\
24.75583984375	-4.34334892991389e-10\\
24.776337890625	-2.07282926624182e-10\\
24.7968359375	-3.99506375001229e-10\\
24.817333984375	-2.31212530485791e-10\\
24.83783203125	-3.34290596965839e-10\\
24.858330078125	-3.6497892317873e-10\\
24.878828125	-3.70176415783619e-10\\
24.899326171875	-3.23160562425774e-10\\
24.91982421875	-3.48937921003671e-10\\
24.940322265625	-3.26774213488292e-10\\
24.9608203125	-4.20106886921499e-10\\
24.981318359375	-3.32061292886486e-10\\
25.00181640625	-5.08856360536265e-10\\
25.022314453125	-3.27219286166949e-10\\
25.0428125	-4.38850028024193e-10\\
25.063310546875	-3.10621577063812e-10\\
25.08380859375	-2.91765623799463e-10\\
25.104306640625	-2.84354731163596e-10\\
25.1248046875	-1.53328840817086e-10\\
25.145302734375	-3.08392613586159e-10\\
25.16580078125	-3.14386981842612e-10\\
25.186298828125	-2.74446860177417e-10\\
25.206796875	-2.66677120054979e-10\\
25.227294921875	-3.1669598074003e-10\\
25.24779296875	-2.89266711379342e-10\\
25.268291015625	-2.42051888421298e-10\\
25.2887890625	-4.1171837280698e-10\\
25.309287109375	-2.03226552546656e-10\\
25.32978515625	-2.81108329113881e-10\\
25.350283203125	-1.37681442985231e-10\\
25.37078125	-2.39163554223932e-10\\
25.391279296875	-2.65217187482284e-10\\
25.41177734375	-3.15541575000902e-10\\
25.432275390625	-3.12030797182413e-10\\
25.4527734375	-3.74770585442142e-10\\
25.473271484375	-2.10125606147327e-10\\
25.49376953125	-3.80165983968742e-10\\
25.514267578125	-2.19376692251002e-10\\
25.534765625	-1.71124929925726e-10\\
25.555263671875	-1.49307652756774e-10\\
25.57576171875	-1.77155108987338e-10\\
25.596259765625	-2.02138467816863e-10\\
25.6167578125	-2.61713412520255e-10\\
25.637255859375	-2.18816804362813e-10\\
25.65775390625	-3.08791330153963e-10\\
25.678251953125	-1.57960958913269e-10\\
25.69875	-1.97585538315926e-11\\
25.719248046875	-1.28548691538605e-10\\
25.73974609375	2.42431751398785e-10\\
25.760244140625	1.18728659587457e-11\\
25.7807421875	1.32093872342625e-10\\
25.801240234375	-6.50995940686672e-11\\
25.82173828125	1.42723651161084e-10\\
25.842236328125	-5.07283896637891e-11\\
25.862734375	7.3416667609495e-11\\
25.883232421875	-9.19154952034722e-11\\
25.90373046875	1.9540500937663e-10\\
25.924228515625	5.44787468054366e-11\\
25.9447265625	2.24389127502929e-10\\
25.965224609375	1.79653168108618e-10\\
25.98572265625	2.44234426758183e-10\\
26.006220703125	2.03881419616575e-10\\
26.02671875	9.18470302079859e-11\\
26.047216796875	-2.84192066844722e-11\\
26.06771484375	-2.62828599549646e-11\\
26.088212890625	-1.24132142117406e-10\\
26.1087109375	-1.78830840322726e-10\\
26.129208984375	-2.2248765000387e-10\\
26.14970703125	-1.35677238300319e-10\\
26.170205078125	-1.38918318968928e-10\\
26.190703125	-7.49355637927444e-11\\
26.211201171875	-7.33961996817828e-11\\
26.23169921875	-1.67606639324555e-11\\
26.252197265625	3.99057823031182e-11\\
26.2726953125	1.56747047792151e-10\\
26.293193359375	2.31929448280898e-11\\
26.31369140625	2.07377344778184e-10\\
26.334189453125	-3.74469052939726e-11\\
26.3546875	3.32856885700091e-11\\
26.375185546875	-3.3243126230564e-11\\
26.39568359375	-2.06163424689992e-10\\
26.416181640625	6.81696374482973e-12\\
26.4366796875	-6.25928330684385e-11\\
26.457177734375	8.7170581403643e-11\\
26.47767578125	1.41486888088216e-10\\
26.498173828125	4.10417727369929e-10\\
26.518671875	3.27133384730177e-10\\
26.539169921875	6.16784766873354e-10\\
26.55966796875	4.27826770898101e-10\\
26.580166015625	5.93549737622644e-10\\
26.6006640625	4.65996924798353e-10\\
26.621162109375	4.3378473317126e-10\\
26.64166015625	3.82304617345097e-10\\
26.662158203125	4.23732411100138e-10\\
26.68265625	4.79081234490795e-10\\
26.703154296875	7.22655385170631e-10\\
26.72365234375	6.93116102188972e-10\\
26.744150390625	8.02722016812474e-10\\
26.7646484375	8.41687798240965e-10\\
26.785146484375	9.36115521972926e-10\\
26.80564453125	8.47203817927916e-10\\
26.826142578125	7.85998069906248e-10\\
26.846640625	7.13542762530509e-10\\
26.867138671875	6.49125445666967e-10\\
26.88763671875	7.30598921205925e-10\\
26.908134765625	5.76159146846771e-10\\
26.9286328125	6.28930618150029e-10\\
26.949130859375	5.63538704744057e-10\\
26.96962890625	6.1810055103476e-10\\
26.990126953125	5.44103544559877e-10\\
27.010625	6.35724664997008e-10\\
27.031123046875	6.90037764682299e-10\\
27.05162109375	6.13354540083379e-10\\
27.072119140625	5.48357425863456e-10\\
27.0926171875	6.20243571864995e-10\\
27.113115234375	5.21794203199984e-10\\
27.13361328125	5.50185944568044e-10\\
27.154111328125	4.59089178509749e-10\\
27.174609375	6.16604366510068e-10\\
27.195107421875	5.55518991651304e-10\\
27.21560546875	6.11188301371698e-10\\
27.236103515625	5.80612422342126e-10\\
27.2566015625	7.48498784580514e-10\\
27.277099609375	5.55707236467324e-10\\
27.29759765625	7.73617218742129e-10\\
27.318095703125	5.37207243533361e-10\\
27.33859375	6.13015974749989e-10\\
27.359091796875	4.10333877674427e-10\\
27.37958984375	4.47541142290899e-10\\
27.400087890625	3.78153874768544e-10\\
27.4205859375	3.43063929089672e-10\\
27.441083984375	4.43243997138196e-10\\
27.46158203125	4.74080780499217e-10\\
27.482080078125	4.5972855112893e-10\\
27.502578125	6.10196155727321e-10\\
27.523076171875	5.23171988660633e-10\\
27.54357421875	6.17875962158845e-10\\
27.564072265625	4.57128866513832e-10\\
27.5845703125	4.5222836585959e-10\\
27.605068359375	4.44384533937257e-10\\
27.62556640625	3.76018218344687e-10\\
27.646064453125	4.03256688688741e-10\\
27.6665625	4.41789310602084e-10\\
27.687060546875	3.256174054806e-10\\
27.70755859375	4.45757740807171e-10\\
27.728056640625	4.02458603843773e-10\\
27.7485546875	2.84958005839871e-10\\
27.769052734375	3.46037018695618e-10\\
27.78955078125	2.89191830710214e-10\\
27.810048828125	2.46210687927353e-10\\
27.830546875	3.64383664040264e-10\\
27.851044921875	2.38419254389258e-10\\
27.87154296875	2.75065260972036e-10\\
27.892041015625	2.95135868887394e-10\\
27.9125390625	3.15174558088123e-10\\
27.933037109375	3.5408117366759e-10\\
27.95353515625	4.3435920711071e-10\\
27.974033203125	3.01839278613283e-10\\
27.99453125	4.41755745030781e-10\\
28.015029296875	2.78495916311564e-10\\
28.03552734375	2.964314102917e-10\\
28.056025390625	3.58139284654085e-10\\
28.0765234375	2.9854374268433e-10\\
28.097021484375	2.46218259530644e-10\\
28.11751953125	2.16078246237427e-10\\
28.138017578125	1.62877083940403e-10\\
28.158515625	1.97981303090335e-10\\
28.179013671875	7.55230485288193e-11\\
28.19951171875	2.34612761968574e-11\\
28.220009765625	6.75232006932816e-11\\
28.2405078125	-2.34942309786269e-11\\
28.261005859375	2.9020572606874e-11\\
28.28150390625	1.18097397387803e-12\\
28.302001953125	6.63655425972351e-11\\
28.3225	-1.78228508716945e-11\\
28.342998046875	3.52173132917224e-11\\
28.36349609375	-6.92989255643853e-11\\
28.383994140625	2.01289075925109e-11\\
28.4044921875	-9.39146939496116e-11\\
28.424990234375	5.10863826482976e-11\\
28.44548828125	-7.84582718584364e-11\\
28.465986328125	-2.24762909830936e-12\\
28.486484375	-4.45565097087378e-11\\
28.506982421875	5.48579610986431e-11\\
28.52748046875	-1.57663162373209e-11\\
28.547978515625	2.15809324830036e-11\\
28.5684765625	-4.11969969069887e-11\\
28.588974609375	8.3834270317118e-11\\
28.60947265625	3.44337636100142e-11\\
28.629970703125	6.2902569048485e-11\\
28.65046875	3.57075094803819e-11\\
28.670966796875	2.19868954053793e-10\\
28.69146484375	1.21589639531672e-10\\
28.711962890625	1.63199910357432e-10\\
28.7324609375	1.49584409802128e-10\\
28.752958984375	5.66994463718081e-11\\
28.77345703125	3.375399647605e-11\\
28.793955078125	1.15115206730468e-10\\
28.814453125	-4.27731112258506e-11\\
28.834951171875	1.55724949602947e-10\\
28.85544921875	7.61020674566405e-12\\
28.875947265625	6.80966099306215e-11\\
28.8964453125	1.18789440480662e-10\\
28.916943359375	-1.13978573418904e-11\\
28.93744140625	2.72644543434805e-11\\
28.957939453125	-9.55069684537305e-11\\
28.9784375	-4.34687518601461e-11\\
28.998935546875	-2.16709571905934e-10\\
29.01943359375	-1.00082387038354e-10\\
29.039931640625	-2.65444637802958e-10\\
29.0604296875	-2.41592796428903e-10\\
29.080927734375	-4.2791828232909e-10\\
29.10142578125	-3.58196984125572e-10\\
29.121923828125	-3.39997599627324e-10\\
29.142421875	-5.38880428315077e-10\\
29.162919921875	-5.02733743144853e-10\\
29.18341796875	-5.0318319643344e-10\\
29.203916015625	-7.09171124243445e-10\\
29.2244140625	-5.62816490384155e-10\\
29.244912109375	-6.95326669559004e-10\\
29.26541015625	-6.02001790676247e-10\\
29.285908203125	-6.40088422179183e-10\\
29.30640625	-6.01094242019233e-10\\
29.326904296875	-6.4875227380795e-10\\
29.34740234375	-5.87685446518054e-10\\
29.367900390625	-4.67826614634007e-10\\
29.3883984375	-5.18641036261192e-10\\
29.408896484375	-4.97606660247176e-10\\
29.42939453125	-5.62208366948753e-10\\
29.449892578125	-4.50831127278148e-10\\
29.470390625	-5.57835569267463e-10\\
29.490888671875	-3.99021197197616e-10\\
29.51138671875	-4.45410074177052e-10\\
29.531884765625	-4.14227118666652e-10\\
29.5523828125	-4.37480789742357e-10\\
29.572880859375	-3.80804788324913e-10\\
29.59337890625	-4.31834666232915e-10\\
29.613876953125	-4.23518619322661e-10\\
29.634375	-4.35700953216129e-10\\
29.654873046875	-4.11515065175278e-10\\
29.67537109375	-4.007658225591e-10\\
29.695869140625	-2.85520776814818e-10\\
29.7163671875	-3.5037863954642e-10\\
29.736865234375	-2.43342718481057e-10\\
29.75736328125	-3.44589629537608e-10\\
29.777861328125	-2.91617484758786e-10\\
29.798359375	-3.45010697337007e-10\\
29.818857421875	-2.47179469156735e-10\\
29.83935546875	-3.47265539965196e-10\\
29.859853515625	-1.73249798355866e-10\\
29.8803515625	-2.25971010794954e-10\\
29.900849609375	-1.99281223306577e-10\\
29.92134765625	-1.42416082213001e-10\\
29.941845703125	-1.19585186192446e-10\\
29.96234375	-4.948054803702e-11\\
29.982841796875	-1.36884862769116e-10\\
30.00333984375	-1.82767408554413e-10\\
30.023837890625	-1.03112883030816e-10\\
30.0443359375	-2.57742095678553e-10\\
30.064833984375	-1.85161580030934e-10\\
30.08533203125	-2.72051868395109e-10\\
30.105830078125	-2.6452792231122e-10\\
30.126328125	-2.4014734606687e-10\\
30.146826171875	-2.42586025586702e-10\\
30.16732421875	-2.22518848743787e-10\\
30.187822265625	-1.21921963685557e-10\\
30.2083203125	-1.59080669444468e-10\\
30.228818359375	-8.46217846560171e-11\\
30.24931640625	-3.9354938275724e-11\\
30.269814453125	-1.12009412691702e-10\\
30.2903125	-3.09348793719382e-11\\
30.310810546875	-1.07822351326703e-10\\
30.33130859375	-1.66069986504401e-10\\
30.351806640625	-1.53646793867663e-10\\
30.3723046875	-1.54394416091994e-10\\
30.392802734375	-1.31308774103036e-10\\
30.41330078125	-1.5247965247828e-10\\
30.433798828125	-1.40076377770676e-10\\
30.454296875	-2.9827391458209e-10\\
30.474794921875	-1.63208648648446e-10\\
30.49529296875	-2.6270783678166e-10\\
30.515791015625	-6.42771312612405e-11\\
30.5362890625	-1.61451161765474e-10\\
30.556787109375	-9.52336325293631e-11\\
30.57728515625	-5.2947034665525e-11\\
30.597783203125	3.82421456429366e-12\\
30.61828125	-1.06265209247551e-10\\
30.638779296875	5.92136482066988e-12\\
30.65927734375	-4.24433230199842e-11\\
30.679775390625	-2.47507153670861e-11\\
30.7002734375	9.53012252480414e-11\\
30.720771484375	5.17923477792678e-11\\
30.74126953125	2.21518297736093e-10\\
30.761767578125	9.75962077759808e-11\\
30.782265625	3.11589208701871e-10\\
30.802763671875	1.29012462974511e-10\\
30.82326171875	1.85341490390389e-10\\
30.843759765625	8.14488078481546e-11\\
30.8642578125	1.67346028325374e-10\\
30.884755859375	5.37373922831405e-11\\
30.90525390625	1.54796267897438e-10\\
30.925751953125	1.55469740738573e-10\\
30.94625	3.17648021922339e-10\\
30.966748046875	1.90379931713419e-10\\
30.98724609375	2.56077883398684e-10\\
31.007744140625	1.15014147201185e-10\\
31.0282421875	8.74594391239245e-11\\
31.048740234375	-2.26025176269312e-11\\
31.06923828125	1.42766350924757e-10\\
31.089736328125	6.26355260225686e-12\\
31.110234375	7.91192846065375e-11\\
31.130732421875	-4.30691731891563e-11\\
31.15123046875	-3.36800070269117e-11\\
31.171728515625	-1.15191744961601e-10\\
31.1922265625	2.74209579161919e-11\\
31.212724609375	-4.75506451841451e-11\\
31.23322265625	2.75816297333768e-11\\
31.253720703125	8.68308053789439e-12\\
31.27421875	8.51435243941316e-11\\
31.294716796875	6.14067407875383e-11\\
31.31521484375	9.26036638894387e-11\\
31.335712890625	6.34382921790238e-11\\
31.3562109375	1.09424328764157e-10\\
31.376708984375	2.17698441637809e-11\\
31.39720703125	1.44784251708932e-10\\
31.417705078125	1.23397196609727e-10\\
31.438203125	3.0610505610382e-11\\
31.458701171875	2.34951049586664e-10\\
31.47919921875	3.53229806753667e-11\\
31.499697265625	2.53718361083272e-10\\
31.5201953125	1.70228714630709e-10\\
31.540693359375	2.54611595890811e-10\\
31.56119140625	2.15904555970413e-10\\
31.581689453125	3.45142463354304e-10\\
31.6021875	2.72309425864024e-10\\
31.622685546875	4.62202556013375e-10\\
31.64318359375	5.31060471643285e-10\\
31.663681640625	5.73024262722415e-10\\
31.6841796875	5.75807992424914e-10\\
31.704677734375	5.96201868537894e-10\\
31.72517578125	5.15013441276741e-10\\
31.745673828125	6.89847141221016e-10\\
31.766171875	5.92377514412181e-10\\
31.786669921875	6.11510625595278e-10\\
31.80716796875	5.35706123915006e-10\\
31.827666015625	6.11096945168064e-10\\
31.8481640625	5.33858622131192e-10\\
31.868662109375	5.47676378819732e-10\\
31.88916015625	5.85582006602363e-10\\
31.909658203125	5.55641237924576e-10\\
31.93015625	6.10298845996044e-10\\
31.950654296875	4.42031936755567e-10\\
31.97115234375	5.64945583875778e-10\\
31.991650390625	4.68706956111997e-10\\
32.0121484375	4.73861788767289e-10\\
32.032646484375	4.12855074934801e-10\\
32.05314453125	3.73317135089286e-10\\
32.073642578125	3.97098873206183e-10\\
32.094140625	3.58724108931241e-10\\
32.114638671875	4.10247490132653e-10\\
32.13513671875	4.02333223284291e-10\\
32.155634765625	2.18232750412909e-10\\
32.1761328125	3.14409220459189e-10\\
32.196630859375	1.99727962892457e-10\\
32.21712890625	2.93192150354665e-10\\
32.237626953125	2.76265850577204e-10\\
32.258125	2.95880281517467e-10\\
32.278623046875	3.23085959961051e-10\\
32.29912109375	3.45717202091502e-10\\
32.319619140625	2.47462405564191e-10\\
32.3401171875	4.11792988155013e-10\\
32.360615234375	2.71093869457298e-10\\
32.38111328125	2.51088335116868e-10\\
32.401611328125	1.99018470266229e-10\\
32.422109375	1.18395364868534e-10\\
32.442607421875	1.67821643527468e-10\\
32.46310546875	9.10845030085651e-11\\
32.483603515625	1.75244027927531e-10\\
32.5041015625	1.3868134484397e-10\\
32.524599609375	2.51296446991348e-10\\
32.54509765625	2.47408963351885e-10\\
32.565595703125	2.73428421043091e-10\\
32.58609375	2.97389295055771e-10\\
32.606591796875	2.09151677933267e-10\\
32.62708984375	1.9786304577575e-10\\
32.647587890625	1.29337654594246e-10\\
32.6680859375	2.2109044434389e-10\\
32.688583984375	1.00415329764471e-10\\
32.70908203125	2.38415544603736e-10\\
32.729580078125	1.92208559988479e-10\\
32.750078125	2.0429241387141e-10\\
32.770576171875	1.65032820823527e-10\\
32.79107421875	1.74267948436919e-10\\
32.811572265625	1.89390087744225e-10\\
32.8320703125	2.13992101624132e-10\\
32.852568359375	5.84753394089116e-11\\
32.87306640625	1.92031673364931e-10\\
32.893564453125	1.23371441603898e-10\\
32.9140625	1.94043970945921e-10\\
32.934560546875	1.90389856321222e-10\\
32.95505859375	2.31881019942417e-10\\
32.975556640625	1.37974561206197e-10\\
32.9960546875	1.71568277962387e-10\\
33.016552734375	3.37399526758974e-11\\
33.03705078125	1.44792304293374e-10\\
33.057548828125	-2.709514186764e-11\\
33.078046875	-2.65352239315757e-11\\
33.098544921875	-1.81520411395028e-11\\
33.11904296875	7.92238587199322e-11\\
33.139541015625	2.21198562937627e-11\\
33.1600390625	8.96026652164696e-11\\
33.180537109375	9.3445419113109e-11\\
33.20103515625	6.27121889456717e-11\\
33.221533203125	7.83403150235458e-11\\
33.24203125	2.29777456739138e-12\\
33.262529296875	1.97680161001603e-11\\
33.28302734375	-5.79924073195761e-11\\
33.303525390625	-4.82802086590025e-11\\
33.3240234375	-7.3205397162425e-11\\
33.344521484375	2.73472893238619e-11\\
33.36501953125	-7.21258278607017e-11\\
33.385517578125	5.14636286984683e-11\\
33.406015625	-1.1737986745188e-10\\
33.426513671875	-7.13935339596039e-12\\
33.44701171875	-1.03350590859144e-10\\
33.467509765625	1.92452106659373e-11\\
33.4880078125	-4.95600231332205e-11\\
33.508505859375	5.73552994058708e-11\\
33.52900390625	-2.47561404381851e-11\\
33.549501953125	1.03584065250627e-10\\
33.57	6.72489563252609e-11\\
33.590498046875	8.21236454226359e-11\\
33.61099609375	7.75682424697691e-11\\
33.631494140625	1.01391958445045e-10\\
33.6519921875	3.5665406557588e-11\\
33.672490234375	7.79731000676881e-11\\
33.69298828125	9.92272847316778e-11\\
33.713486328125	1.41097188807249e-10\\
33.733984375	2.16673395326599e-10\\
33.754482421875	1.96989931829583e-10\\
33.77498046875	1.33862722066979e-10\\
33.795478515625	1.63398959776987e-10\\
33.8159765625	3.34627844875809e-11\\
33.836474609375	8.16560818228052e-11\\
33.85697265625	-9.72361287759772e-11\\
33.877470703125	2.09556925748551e-12\\
33.89796875	-3.93682985328063e-11\\
33.918466796875	1.64087478437765e-11\\
33.93896484375	5.82279062920075e-11\\
33.959462890625	2.50186656785805e-11\\
33.9799609375	1.49154992791609e-10\\
34.000458984375	-2.52911373361548e-12\\
34.02095703125	4.49185253311186e-11\\
34.041455078125	-1.91825231465235e-10\\
34.061953125	-1.19571296597107e-10\\
34.082451171875	-1.64432173485583e-10\\
34.10294921875	-2.23340534540365e-10\\
34.123447265625	-2.7598721674318e-10\\
34.1439453125	-1.37901206910193e-10\\
34.164443359375	-1.0994693664645e-10\\
34.18494140625	-5.75528718117397e-11\\
};
\addplot [color=mycolor1,solid]
  table[row sep=crcr]{%
34.18494140625	-5.75528718117397e-11\\
34.205439453125	-7.94595954501071e-11\\
34.2259375	-1.16114236170854e-10\\
34.246435546875	-2.43316791051297e-10\\
34.26693359375	-1.8074979444341e-10\\
34.287431640625	-2.56800306437881e-10\\
34.3079296875	-2.46956856162337e-10\\
34.328427734375	-2.71194828169701e-10\\
34.34892578125	-1.93372913430896e-10\\
34.369423828125	-1.55661080873493e-10\\
34.389921875	-1.46820882690965e-10\\
34.410419921875	-1.49596706066909e-11\\
34.43091796875	-9.32909572460676e-11\\
34.451416015625	-7.4530905543535e-12\\
34.4719140625	-9.69797185258484e-11\\
34.492412109375	3.4846336108717e-12\\
34.51291015625	-1.28405848338161e-10\\
34.533408203125	-4.24913357112693e-11\\
34.55390625	-5.6887160350556e-11\\
34.574404296875	-9.29255879701825e-11\\
34.59490234375	5.41277820241661e-12\\
34.615400390625	-6.09250862521432e-12\\
34.6358984375	-1.231094065766e-10\\
34.656396484375	-8.10408282055805e-11\\
34.67689453125	-4.17180549839379e-11\\
34.697392578125	3.86895598602437e-11\\
34.717890625	-6.59560112065367e-11\\
34.738388671875	8.51644907278851e-12\\
34.75888671875	3.37098109784191e-11\\
34.779384765625	8.71827218060658e-11\\
34.7998828125	-1.50366897523243e-11\\
34.820380859375	1.36769612248079e-10\\
34.84087890625	6.59625422160526e-11\\
34.861376953125	1.09808878744061e-10\\
34.881875	2.98024179392792e-12\\
34.902373046875	7.54448577285376e-11\\
34.92287109375	1.39519549526357e-11\\
34.943369140625	3.99161233480488e-11\\
34.9638671875	7.10238661445541e-11\\
34.984365234375	7.29544016367325e-11\\
35.00486328125	1.7274922519063e-10\\
35.025361328125	1.3219774280295e-10\\
35.045859375	2.68049456921464e-10\\
35.066357421875	2.64719541312848e-10\\
35.08685546875	2.44514951271672e-10\\
35.107353515625	2.71402969885008e-10\\
35.1278515625	2.25635718653773e-10\\
35.148349609375	3.11179267623394e-10\\
35.16884765625	2.12113334152712e-10\\
35.189345703125	2.09997557460296e-10\\
35.20984375	1.41740807298814e-10\\
35.230341796875	1.90351892509495e-10\\
35.25083984375	1.51312457528227e-10\\
35.271337890625	2.77601083918911e-10\\
35.2918359375	2.36968526769401e-10\\
35.312333984375	3.69155956847689e-10\\
35.33283203125	3.05701637946592e-10\\
35.353330078125	3.42222930995498e-10\\
35.373828125	2.75798778218708e-10\\
35.394326171875	2.95429419711235e-10\\
35.41482421875	2.51517879322638e-10\\
35.435322265625	2.85888114823228e-10\\
35.4558203125	2.07289277892896e-10\\
35.476318359375	2.30206988192506e-10\\
35.49681640625	2.52301874009348e-10\\
35.517314453125	2.88169531917884e-10\\
35.5378125	2.94137992243487e-10\\
35.558310546875	3.91245398203541e-10\\
35.57880859375	2.96761063489012e-10\\
35.599306640625	3.55944124657452e-10\\
35.6198046875	1.99237487087378e-10\\
35.640302734375	1.82190531456156e-10\\
35.66080078125	1.34947943210167e-10\\
35.681298828125	1.7151624655573e-10\\
35.701796875	1.6618282215824e-10\\
35.722294921875	1.75043255216652e-10\\
35.74279296875	2.75220727981933e-10\\
35.763291015625	3.10952592502401e-10\\
35.7837890625	3.67105774950714e-10\\
35.804287109375	3.71801794974355e-10\\
35.82478515625	4.50732034813226e-10\\
35.845283203125	3.02119705518225e-10\\
35.86578125	3.66085719867864e-10\\
35.886279296875	2.9720422187221e-10\\
35.90677734375	3.03231356290982e-10\\
35.927275390625	2.30291142949421e-10\\
35.9477734375	3.48874542228639e-10\\
35.968271484375	3.59287071872932e-10\\
35.98876953125	5.12753331743979e-10\\
36.009267578125	3.9032480588021e-10\\
36.029765625	5.42663092818401e-10\\
36.050263671875	4.51078636707968e-10\\
36.07076171875	4.04937523239893e-10\\
36.091259765625	3.61340892763096e-10\\
36.1117578125	3.02256433362686e-10\\
36.132255859375	2.48726474363359e-10\\
36.15275390625	3.07286303437688e-10\\
36.173251953125	2.11422380229301e-10\\
36.19375	2.98257526615529e-10\\
36.214248046875	2.55182831384365e-10\\
36.23474609375	3.449835096993e-10\\
36.255244140625	3.14222213715131e-10\\
36.2757421875	3.5990559821519e-10\\
36.296240234375	3.51633584928755e-10\\
36.31673828125	4.56204360521762e-10\\
36.337236328125	3.07400667146725e-10\\
36.357734375	3.91751472857103e-10\\
36.378232421875	2.56496491165785e-10\\
36.39873046875	2.92967068531821e-10\\
36.419228515625	3.04565322822253e-10\\
36.4397265625	3.40185073217501e-10\\
36.460224609375	3.71825644891902e-10\\
36.48072265625	2.85395432295455e-10\\
36.501220703125	4.61747791698387e-10\\
36.52171875	3.75986874593878e-10\\
36.542216796875	4.87319889643514e-10\\
36.56271484375	4.24459782113817e-10\\
36.583212890625	4.0702287145313e-10\\
36.6037109375	2.62158735889302e-10\\
36.624208984375	2.93486086752732e-10\\
36.64470703125	2.03043702185676e-10\\
36.665205078125	2.58861235786694e-10\\
36.685703125	3.32536846651331e-10\\
36.706201171875	3.49037250941542e-10\\
36.72669921875	3.87942926938384e-10\\
36.747197265625	4.59289103983144e-10\\
36.7676953125	4.61282979015769e-10\\
36.788193359375	5.10081834885496e-10\\
36.80869140625	4.18956851521162e-10\\
36.829189453125	4.29126282881796e-10\\
36.8496875	2.65322093187218e-10\\
36.870185546875	2.46116363808301e-10\\
36.89068359375	2.18184873928207e-10\\
36.911181640625	2.51066179842778e-10\\
36.9316796875	3.6989116915772e-10\\
36.952177734375	3.07349261381504e-10\\
36.97267578125	4.19184522357444e-10\\
36.993173828125	2.93960653250151e-10\\
37.013671875	3.78761208993697e-10\\
37.034169921875	4.01896682283105e-10\\
37.05466796875	4.26686330814526e-10\\
37.075166015625	3.23528675630926e-10\\
37.0956640625	3.19582323848384e-10\\
37.116162109375	2.88378306121301e-10\\
37.13666015625	3.86159479802807e-10\\
37.157158203125	4.44851897474569e-10\\
37.17765625	4.39012823119536e-10\\
37.198154296875	3.70571903746158e-10\\
37.21865234375	4.03306842731631e-10\\
37.239150390625	2.84157553456806e-10\\
37.2596484375	2.42475257671915e-10\\
37.280146484375	2.22415563679207e-10\\
37.30064453125	2.39789534278559e-10\\
37.321142578125	1.84994253950988e-10\\
37.341640625	1.83795661017214e-10\\
37.362138671875	1.93222594433961e-10\\
37.38263671875	2.98475542643859e-10\\
37.403134765625	2.97636810492974e-10\\
37.4236328125	2.42921010522505e-10\\
37.444130859375	1.69363307855687e-10\\
37.46462890625	1.02589740026526e-10\\
37.485126953125	7.39352609800047e-11\\
37.505625	-2.69058341783328e-11\\
37.526123046875	1.64062831039677e-11\\
37.54662109375	-5.11483893903216e-11\\
37.567119140625	-1.85225116432361e-11\\
37.5876171875	7.78650095010637e-11\\
37.608115234375	-2.65868171877867e-12\\
37.62861328125	1.93416882548813e-10\\
37.649111328125	2.03880965629783e-11\\
37.669609375	1.15449938307512e-10\\
37.690107421875	7.27375819628737e-11\\
37.71060546875	1.78102219600104e-11\\
37.731103515625	-3.3838485689954e-11\\
37.7516015625	-4.63066849208665e-11\\
37.772099609375	-5.35246944196633e-11\\
37.79259765625	-6.21334552945189e-11\\
37.813095703125	-1.1946658730572e-10\\
37.83359375	2.71425205867911e-11\\
37.854091796875	-8.01839055980339e-11\\
37.87458984375	-3.72485209787436e-11\\
37.895087890625	-1.10127249317616e-10\\
37.9155859375	-1.59021594132247e-11\\
37.936083984375	-3.94806338863294e-11\\
37.95658203125	2.11166944635879e-12\\
37.977080078125	-2.08587214429645e-11\\
37.997578125	-8.54399554851271e-11\\
38.018076171875	-5.50956425441918e-11\\
38.03857421875	-9.20896084691136e-11\\
38.059072265625	-1.65558293755346e-10\\
38.0795703125	-6.00464307781638e-11\\
38.100068359375	-2.07143752702529e-10\\
38.12056640625	-1.98363493765456e-10\\
38.141064453125	-1.73060923086126e-10\\
38.1615625	-1.5588698186926e-10\\
38.182060546875	-2.3474471462936e-10\\
38.20255859375	-1.52454184053114e-10\\
38.223056640625	-2.2124160115847e-10\\
38.2435546875	-2.25228868502066e-10\\
38.264052734375	-3.06189402734423e-10\\
38.28455078125	-3.26349649659683e-10\\
38.305048828125	-2.2176887512316e-10\\
38.325546875	-4.16921867389873e-10\\
38.346044921875	-2.02078712268158e-10\\
38.36654296875	-3.00413238426932e-10\\
38.387041015625	-2.67865418217695e-10\\
38.4075390625	-3.05104429528281e-10\\
38.428037109375	-2.21697099084449e-10\\
38.44853515625	-3.95519744018356e-10\\
38.469033203125	-3.27977387031327e-10\\
38.48953125	-5.21396747284791e-10\\
38.510029296875	-3.27186380436655e-10\\
38.53052734375	-4.78551890086372e-10\\
38.551025390625	-3.68825710024048e-10\\
38.5715234375	-3.1697232052478e-10\\
38.592021484375	-2.54364953782062e-10\\
38.61251953125	-3.00027433906952e-10\\
38.633017578125	-1.83884280615009e-10\\
38.653515625	-1.39556862489161e-10\\
38.674013671875	-1.27370314470258e-10\\
38.69451171875	-2.45587587831669e-10\\
38.715009765625	-1.68438022956531e-10\\
38.7355078125	-3.00993679807019e-10\\
38.756005859375	-1.64208602260331e-10\\
38.77650390625	-2.03823983317236e-10\\
38.797001953125	-1.58629262452782e-10\\
38.8175	-2.90753788575257e-10\\
38.837998046875	-2.19869230593298e-10\\
38.85849609375	-2.49416588288856e-10\\
38.878994140625	-1.97299878245389e-10\\
38.8994921875	-2.3407504029841e-10\\
38.919990234375	-2.31699152051677e-10\\
38.94048828125	-2.58331098970881e-10\\
38.960986328125	-2.64401139350362e-10\\
38.981484375	-2.66336246827311e-10\\
39.001982421875	-3.23649445468669e-10\\
39.02248046875	-2.46078773074003e-10\\
39.042978515625	-2.72340318790857e-10\\
39.0634765625	-2.49300456969991e-10\\
39.083974609375	-2.48998588461857e-10\\
39.10447265625	-2.00607327233853e-10\\
39.124970703125	-3.073350356906e-10\\
39.14546875	-1.18305296362446e-10\\
39.165966796875	-2.48559420444397e-10\\
39.18646484375	-2.1701538179877e-10\\
39.206962890625	-2.6421366700195e-10\\
39.2274609375	-3.1319587001013e-10\\
39.247958984375	-3.08727087110455e-10\\
39.26845703125	-3.40441620502205e-10\\
39.288955078125	-3.14794734046668e-10\\
39.309453125	-2.60895980321559e-10\\
39.329951171875	-2.41264240326441e-10\\
39.35044921875	-2.21357306482054e-10\\
39.370947265625	-2.10525177975182e-10\\
39.3914453125	-2.24490255455847e-10\\
39.411943359375	-2.02320305143239e-10\\
39.43244140625	-2.11122537076995e-10\\
39.452939453125	-1.95230768066154e-10\\
39.4734375	-2.09344016287539e-10\\
39.493935546875	-2.19090692929793e-10\\
39.51443359375	-3.09612948752095e-10\\
39.534931640625	-2.68051133650134e-10\\
39.5554296875	-2.9323767481605e-10\\
39.575927734375	-2.27253751150356e-10\\
39.59642578125	-2.74706422693796e-10\\
39.616923828125	-2.56416768237957e-10\\
39.637421875	-1.71108111116943e-10\\
39.657919921875	-2.38457599857625e-10\\
39.67841796875	-2.15315280149192e-10\\
39.698916015625	-1.22028706150564e-10\\
39.7194140625	-8.28457859164231e-11\\
39.739912109375	-1.20752192053305e-10\\
39.76041015625	-1.03219749218748e-10\\
39.780908203125	-1.04910753242869e-10\\
39.80140625	-1.33529778927617e-10\\
39.821904296875	-1.7398516070144e-10\\
39.84240234375	-2.68572498203043e-10\\
39.862900390625	-2.21457093054598e-10\\
39.8833984375	-2.70363966306782e-10\\
39.903896484375	-2.25563860010637e-10\\
39.92439453125	-1.09520592396727e-10\\
39.944892578125	-1.17399262194747e-10\\
39.965390625	-2.64018257480449e-12\\
39.985888671875	-3.23338500141157e-11\\
40.00638671875	4.02283113715914e-11\\
40.026884765625	-1.56027603829312e-11\\
40.0473828125	2.04348157926946e-11\\
40.067880859375	-1.27221500109606e-10\\
40.08837890625	-1.58854917664454e-10\\
40.108876953125	-1.08432697554273e-10\\
40.129375	-2.05321327583669e-10\\
40.149873046875	-5.83268883111778e-11\\
40.17037109375	-1.43111254146676e-10\\
40.190869140625	-6.17087187001427e-11\\
40.2113671875	-8.52815774072879e-12\\
40.231865234375	5.12785845311164e-11\\
40.25236328125	3.79454036311147e-11\\
40.272861328125	5.25986685899243e-11\\
40.293359375	-1.44605677981569e-10\\
40.313857421875	-2.2718600182181e-11\\
40.33435546875	-1.49661049031863e-10\\
40.354853515625	-8.86367943182126e-11\\
40.3753515625	-8.5094513437756e-13\\
40.395849609375	2.3076487545441e-11\\
40.41634765625	3.53006386342922e-11\\
40.436845703125	4.41352895759384e-11\\
40.45734375	6.31470989756223e-11\\
40.477841796875	8.27216105743109e-11\\
40.49833984375	5.21965150260242e-11\\
40.518837890625	9.90660951882813e-11\\
40.5393359375	3.69914763929835e-11\\
40.559833984375	5.65885469198193e-11\\
40.58033203125	3.29323536753896e-11\\
40.600830078125	1.43213202314531e-10\\
40.621328125	1.25320094453716e-10\\
40.641826171875	2.28118100928017e-10\\
40.66232421875	1.49280657916899e-10\\
40.682822265625	1.9167256942656e-10\\
40.7033203125	1.6793472868237e-10\\
40.723818359375	2.21086912309553e-10\\
40.74431640625	1.35133810356117e-10\\
40.764814453125	1.3607612372935e-10\\
40.7853125	2.09404425254124e-10\\
40.805810546875	1.46963106452809e-10\\
40.82630859375	2.64378264334803e-10\\
40.846806640625	1.37991770651994e-10\\
40.8673046875	2.15986678274689e-10\\
40.887802734375	1.48479063959426e-10\\
40.90830078125	2.49919558742893e-10\\
40.928798828125	1.29024217527335e-10\\
40.949296875	1.8088063266333e-10\\
40.969794921875	1.10878431579847e-10\\
40.99029296875	2.53833564587514e-10\\
41.010791015625	9.33859055833166e-11\\
41.0312890625	1.81486999667621e-10\\
41.051787109375	3.57493005756223e-11\\
41.07228515625	1.13052420813615e-10\\
41.092783203125	3.9850292642083e-11\\
41.11328125	1.19391017309397e-10\\
41.133779296875	1.58152830450564e-11\\
41.15427734375	2.24464026980889e-11\\
41.174775390625	-5.69257510855547e-12\\
41.1952734375	5.04859425705204e-11\\
41.215771484375	5.8233040067538e-11\\
41.23626953125	6.20388135950585e-11\\
41.256767578125	-1.07925732307839e-11\\
41.277265625	5.92231096722652e-11\\
41.297763671875	2.49582461059939e-11\\
41.31826171875	4.54725557054618e-11\\
41.338759765625	2.7202307775341e-11\\
41.3592578125	7.99055800926031e-11\\
41.379755859375	6.48181456838055e-11\\
41.40025390625	1.0005028332671e-10\\
41.420751953125	7.70234059497654e-11\\
41.44125	1.73332180111679e-10\\
41.461748046875	1.77110199264728e-10\\
41.48224609375	1.07259249528351e-10\\
41.502744140625	1.07322058261727e-10\\
41.5232421875	-3.44291869699908e-11\\
41.543740234375	3.99820352520145e-11\\
41.56423828125	-4.58637686133644e-11\\
41.584736328125	3.93314594235278e-11\\
41.605234375	-4.61047187276284e-11\\
41.625732421875	1.57095534583261e-10\\
41.64623046875	1.13415696991708e-10\\
41.666728515625	2.68940990158623e-10\\
41.6872265625	2.21103440375274e-10\\
41.707724609375	2.37802030268497e-10\\
41.72822265625	2.56094857072867e-10\\
41.748720703125	2.06639050340581e-10\\
41.76921875	1.92572292095285e-10\\
41.789716796875	9.46658317109345e-11\\
41.81021484375	8.36134384423519e-11\\
41.830712890625	1.07651889502283e-10\\
41.8512109375	1.91927852657397e-10\\
41.871708984375	1.75349594161554e-10\\
41.89220703125	1.82894890858855e-10\\
41.912705078125	2.03261583938007e-10\\
41.933203125	1.69756878752647e-10\\
41.953701171875	2.04242513306125e-10\\
41.97419921875	2.0075033497144e-10\\
41.994697265625	2.12273449233475e-10\\
42.0151953125	2.13617255789822e-10\\
42.035693359375	1.13094027929203e-10\\
42.05619140625	2.11723928600688e-10\\
42.076689453125	1.91409090949079e-10\\
42.0971875	2.44917942852932e-10\\
42.117685546875	2.07922381830577e-10\\
42.13818359375	1.22841948103065e-10\\
42.158681640625	1.518146028659e-10\\
42.1791796875	1.69744975322962e-10\\
42.199677734375	1.55352943560306e-10\\
42.22017578125	7.09123665495745e-11\\
42.240673828125	2.01301633705911e-10\\
42.261171875	1.93533538312532e-10\\
42.281669921875	1.51580080820807e-10\\
42.30216796875	1.42988396680018e-10\\
42.322666015625	9.05219692425266e-11\\
42.3431640625	1.50627223965267e-10\\
42.363662109375	8.44029338446231e-11\\
42.38416015625	1.24770866973183e-10\\
42.404658203125	8.61535922900791e-11\\
42.42515625	1.07060898762434e-10\\
42.445654296875	-1.75886783295562e-11\\
42.46615234375	-1.88706052684911e-11\\
42.486650390625	-1.34057550730286e-11\\
42.5071484375	-1.54928102750477e-11\\
42.527646484375	-1.00270036066357e-11\\
42.54814453125	4.47040435657066e-11\\
42.568642578125	-2.8246871596731e-11\\
42.589140625	7.13624427380102e-11\\
42.609638671875	2.68541172129186e-11\\
42.63013671875	7.83310253000124e-11\\
42.650634765625	-1.11366172245153e-11\\
42.6711328125	7.28972701189111e-11\\
42.691630859375	-9.57169885398276e-11\\
42.71212890625	2.13706537143186e-11\\
42.732626953125	2.37268277683568e-13\\
42.753125	2.06825283152006e-11\\
42.773623046875	-2.36469643769098e-11\\
42.79412109375	-2.32282758492589e-11\\
42.814619140625	3.243670198408e-11\\
42.8351171875	9.2766746091783e-11\\
42.855615234375	4.19179226301586e-11\\
42.87611328125	5.48091329069284e-11\\
42.896611328125	2.64872903102293e-11\\
42.917109375	-3.33543312371419e-11\\
42.937607421875	-4.11276750628699e-11\\
42.95810546875	-1.07984701479192e-10\\
42.978603515625	-1.87738079439413e-11\\
42.9991015625	-1.1823098088411e-10\\
43.019599609375	-1.12364042987368e-10\\
43.04009765625	-5.18022763285029e-11\\
43.060595703125	-9.19418503882896e-11\\
43.08109375	-1.48069070978564e-10\\
43.101591796875	-7.27197565912053e-11\\
43.12208984375	-6.01601916868931e-11\\
43.142587890625	-8.18764546784989e-11\\
43.1630859375	-1.62013360539381e-10\\
43.183583984375	-1.23946415925324e-10\\
43.20408203125	-1.78898615502633e-10\\
43.224580078125	-2.42794529309489e-10\\
43.245078125	-2.01407009888978e-10\\
43.265576171875	-2.106129563203e-10\\
43.28607421875	-2.26786014293527e-10\\
43.306572265625	-2.24585358582778e-10\\
43.3270703125	-2.25364557624476e-10\\
43.347568359375	-1.5080371550873e-10\\
43.36806640625	-2.11925359395587e-10\\
43.388564453125	-1.01389829259004e-10\\
43.4090625	-1.81366746334471e-10\\
43.429560546875	-1.88804323586087e-10\\
43.45005859375	-2.43188607269238e-10\\
43.470556640625	-1.66090091917588e-10\\
43.4910546875	-2.97672410216328e-10\\
43.511552734375	-2.22420100799454e-10\\
43.53205078125	-2.82961670397314e-10\\
43.552548828125	-1.41655638563882e-10\\
43.573046875	-1.51911026910157e-10\\
43.593544921875	-9.92370343364747e-11\\
43.61404296875	-1.42639014354031e-10\\
43.634541015625	-3.66213202617121e-11\\
43.6550390625	-8.3672563154367e-11\\
43.675537109375	-8.23105999214754e-11\\
43.69603515625	-1.34983562318259e-10\\
43.716533203125	-1.55859209468835e-10\\
43.73703125	-1.86148716959063e-10\\
43.757529296875	-1.1620625836727e-10\\
43.77802734375	-1.98321637751628e-10\\
43.798525390625	-2.59210588633383e-11\\
43.8190234375	-1.00801712969646e-10\\
43.839521484375	-1.01593337357815e-10\\
43.86001953125	-1.15761272391415e-10\\
43.880517578125	-1.26060436682725e-10\\
43.901015625	-1.20589473832131e-10\\
43.921513671875	-1.63426460180186e-10\\
43.94201171875	-2.511180138719e-10\\
43.962509765625	-1.89270664395055e-10\\
43.9830078125	-1.01161707307373e-10\\
44.003505859375	-1.73652020952641e-10\\
44.02400390625	-1.07614557333578e-11\\
44.044501953125	-1.00700288184357e-10\\
44.065	1.07507053783988e-11\\
44.085498046875	-1.08812441981595e-10\\
44.10599609375	-7.11602963086387e-11\\
44.126494140625	-2.39885063545028e-10\\
44.1469921875	-1.46418058507806e-10\\
44.167490234375	-2.27409983220737e-10\\
44.18798828125	-1.95977271892788e-10\\
44.208486328125	-1.75024255829456e-10\\
44.228984375	-2.21248990860917e-10\\
44.249482421875	-2.18078855306912e-10\\
44.26998046875	-1.77725235950268e-10\\
44.290478515625	-1.87437219120028e-10\\
44.3109765625	-1.44869149924615e-10\\
44.331474609375	-1.69139565224362e-10\\
44.35197265625	-1.77816222093242e-10\\
44.372470703125	-2.20829040438938e-10\\
44.39296875	-1.24317027362144e-10\\
44.413466796875	-1.59860546801215e-10\\
44.43396484375	-1.44082168138047e-10\\
44.454462890625	-1.92506453138768e-10\\
44.4749609375	-1.96725315320485e-10\\
44.495458984375	-2.27677223100727e-10\\
44.51595703125	-1.57462049104119e-10\\
44.536455078125	-2.1202802571206e-10\\
44.556953125	-2.21708116781591e-10\\
44.577451171875	-2.2447597850045e-10\\
44.59794921875	-2.0371544258279e-10\\
44.618447265625	-1.48761864852565e-10\\
44.6389453125	-1.34303803820034e-10\\
44.659443359375	-6.14118216023745e-11\\
44.67994140625	-5.52610063530378e-11\\
44.700439453125	-6.7723012569565e-11\\
44.7209375	-1.25847473452305e-10\\
44.741435546875	-3.45525392409323e-11\\
44.76193359375	-1.4575619907415e-10\\
44.782431640625	-1.1828845670939e-10\\
44.8029296875	-1.49048190737576e-10\\
44.823427734375	1.93258604992212e-11\\
44.84392578125	-1.17312868762542e-10\\
44.864423828125	8.67200267531463e-12\\
44.884921875	-4.88585329305466e-11\\
44.905419921875	4.4982293940264e-11\\
44.92591796875	-4.04446742038407e-11\\
44.946416015625	3.96949596672663e-11\\
44.9669140625	-2.25757536193498e-11\\
44.987412109375	-6.81627951153804e-11\\
45.00791015625	6.50941517459236e-11\\
45.028408203125	1.68688315230487e-11\\
45.04890625	3.7711367174258e-11\\
45.069404296875	6.9971023465058e-11\\
45.08990234375	6.26673555768315e-11\\
45.110400390625	7.58840498455248e-11\\
45.1308984375	6.63412407758693e-11\\
45.151396484375	1.17994761395855e-10\\
45.17189453125	4.15046275973533e-11\\
45.192392578125	3.46384698292043e-11\\
45.212890625	3.24687123304357e-12\\
45.233388671875	7.62293691491642e-11\\
45.25388671875	-5.44439100887616e-11\\
45.274384765625	5.5993525676771e-11\\
45.2948828125	-4.33506545941495e-11\\
45.315380859375	-2.62504360459494e-11\\
45.33587890625	-7.29474259008443e-11\\
45.356376953125	6.57405659788703e-12\\
45.376875	-1.67527932084909e-11\\
45.397373046875	4.41040765906557e-11\\
45.41787109375	3.75284827247103e-11\\
45.438369140625	5.99953689722745e-11\\
45.4588671875	1.4474588072074e-10\\
45.479365234375	1.09308919761785e-10\\
45.49986328125	1.19044783181485e-10\\
45.520361328125	1.01369219110573e-10\\
45.540859375	9.35637047205992e-11\\
45.561357421875	1.1360948547605e-10\\
45.58185546875	9.2067133354175e-11\\
45.602353515625	3.91929597647801e-11\\
45.6228515625	1.26074521760131e-10\\
45.643349609375	1.14373450094821e-10\\
45.66384765625	1.52887733498849e-10\\
45.684345703125	1.95044067561517e-10\\
45.70484375	2.12413622002686e-10\\
45.725341796875	2.20044629566332e-10\\
45.74583984375	2.40942630276036e-10\\
45.766337890625	2.07627765208858e-10\\
45.7868359375	2.24096696961757e-10\\
45.807333984375	1.56261194286493e-10\\
45.82783203125	1.62449763334999e-10\\
45.848330078125	8.82827841443623e-11\\
45.868828125	1.60533806370454e-10\\
45.889326171875	5.29755159521849e-11\\
45.90982421875	1.24591265703684e-10\\
45.930322265625	1.31709760395012e-10\\
45.9508203125	2.09031901351953e-10\\
45.971318359375	1.48685169455762e-10\\
45.99181640625	2.47315767509598e-10\\
46.012314453125	1.21599303792872e-10\\
46.0328125	2.40098033650745e-10\\
46.053310546875	5.18673342799183e-11\\
46.07380859375	1.53579077276795e-10\\
46.094306640625	5.47421435279814e-11\\
46.1148046875	8.06310819213413e-11\\
46.135302734375	5.41021831304099e-11\\
46.15580078125	1.18106433669836e-10\\
46.176298828125	7.13458786472677e-11\\
46.196796875	-2.73433897862135e-11\\
46.217294921875	6.03446560122114e-11\\
46.23779296875	8.13431787476162e-11\\
46.258291015625	1.15894295046329e-11\\
46.2787890625	7.398301370118e-11\\
46.299287109375	-8.1961204079015e-12\\
46.31978515625	-6.44222655937867e-11\\
46.340283203125	3.5418823230948e-11\\
46.36078125	2.87853699932379e-11\\
46.381279296875	4.07455193112103e-11\\
46.40177734375	7.56426131728759e-11\\
46.422275390625	9.61625092290315e-11\\
46.4427734375	1.1749272930062e-10\\
46.463271484375	5.19521201257303e-11\\
46.48376953125	2.90819168736651e-11\\
46.504267578125	6.5350054739509e-11\\
46.524765625	2.79688140055213e-12\\
46.545263671875	5.91114296896475e-11\\
46.56576171875	-2.45126367944125e-11\\
46.586259765625	7.38813801641166e-11\\
46.6067578125	1.29748914796979e-11\\
46.627255859375	1.20166012099091e-10\\
46.64775390625	1.51764738896761e-11\\
46.668251953125	5.9497262319632e-11\\
46.68875	5.01698751172176e-11\\
46.709248046875	4.6479192365162e-11\\
46.72974609375	8.91786270752132e-11\\
46.750244140625	1.5764844459423e-11\\
46.7707421875	6.18485410361075e-11\\
46.791240234375	5.37372914789372e-11\\
46.81173828125	8.27822820284168e-11\\
46.832236328125	8.82868385787806e-11\\
46.852734375	5.40746197799323e-11\\
46.873232421875	9.73318146098201e-11\\
46.89373046875	7.33433441182105e-11\\
46.914228515625	5.82031382943588e-11\\
46.9347265625	1.28978638505283e-10\\
46.955224609375	1.13832446935865e-10\\
46.97572265625	1.5841278716646e-10\\
46.996220703125	1.52134782956632e-10\\
47.01671875	1.44058887450221e-10\\
47.037216796875	1.50610951321865e-10\\
47.05771484375	1.35527213974556e-10\\
47.078212890625	5.79351425567031e-11\\
47.0987109375	8.094406941031e-11\\
47.119208984375	5.83370454036199e-11\\
47.13970703125	2.29703358855343e-11\\
47.160205078125	2.70806260061528e-11\\
47.180703125	-5.16211673336491e-12\\
47.201201171875	1.19269658175609e-11\\
47.22169921875	3.04689866675508e-11\\
47.242197265625	-1.14395106967913e-12\\
47.2626953125	1.72961508532179e-11\\
47.283193359375	1.5364708790312e-11\\
47.30369140625	1.58880144621968e-11\\
47.324189453125	-1.02186821525229e-11\\
47.3446875	5.60913774187273e-11\\
47.365185546875	-6.74772137649705e-11\\
47.38568359375	9.70691889645111e-11\\
47.406181640625	-2.55959265142108e-11\\
47.4266796875	5.81775040346871e-11\\
47.447177734375	-1.6254508666589e-11\\
47.46767578125	3.89466679225451e-12\\
47.488173828125	1.31845746305952e-11\\
47.508671875	2.46392918132365e-11\\
47.529169921875	9.32628078848801e-12\\
47.54966796875	-1.48041802971853e-10\\
47.570166015625	-6.72579475285673e-11\\
47.5906640625	-1.30595322125564e-10\\
47.611162109375	-4.87231969875042e-11\\
47.63166015625	-3.49593220287699e-11\\
47.652158203125	-3.39325871606409e-11\\
47.67265625	-3.18917086736772e-11\\
47.693154296875	-1.06051484580705e-10\\
47.71365234375	-8.65006661361741e-11\\
47.734150390625	-1.32016114223432e-10\\
47.7546484375	-9.58997472591441e-11\\
47.775146484375	-8.97982813613739e-11\\
47.79564453125	-4.84255685080848e-11\\
47.816142578125	-4.34787239264586e-11\\
47.836640625	5.85479595750442e-12\\
47.857138671875	-1.113967713321e-11\\
47.87763671875	1.87507153091029e-11\\
47.898134765625	-2.80877414847369e-11\\
47.9186328125	-3.83880148393139e-11\\
47.939130859375	-1.34878691975262e-10\\
47.95962890625	-7.45694364539674e-11\\
47.980126953125	-1.27668832122583e-10\\
48.000625	-1.84797779464677e-10\\
48.021123046875	-7.7213502893752e-11\\
48.04162109375	-1.09848186733704e-10\\
48.062119140625	-1.32837455288692e-10\\
48.0826171875	-1.12304370252115e-10\\
48.103115234375	-9.50253878656717e-11\\
48.12361328125	-9.15728989921031e-11\\
48.144111328125	-1.75619227654231e-10\\
48.164609375	-1.02386571436742e-10\\
48.185107421875	-1.80722639627633e-10\\
48.20560546875	-1.17480813810613e-10\\
48.226103515625	-1.27476975525607e-10\\
48.2466015625	-1.41194886885394e-10\\
48.267099609375	-1.99808583992965e-10\\
48.28759765625	-1.85023862153537e-10\\
48.308095703125	-1.51090929722357e-10\\
48.32859375	-1.65654641056148e-10\\
48.349091796875	-1.30873586611259e-10\\
48.36958984375	-1.56782098493576e-10\\
48.390087890625	-8.77425147814041e-11\\
48.4105859375	-1.11549765783426e-10\\
48.431083984375	-7.75707183977099e-11\\
48.45158203125	-1.09394418789873e-10\\
48.472080078125	-1.23231757230514e-10\\
48.492578125	-1.81315979166512e-10\\
48.513076171875	-7.75340562985379e-11\\
48.53357421875	-2.52948144989047e-10\\
48.554072265625	-4.51526314741292e-11\\
48.5745703125	-1.20633759450787e-10\\
48.595068359375	-5.18062729344279e-11\\
48.61556640625	-2.87611626499313e-11\\
48.636064453125	-5.52393577477931e-12\\
48.6565625	-5.35952481007501e-11\\
48.677060546875	2.25396976869737e-11\\
48.69755859375	-1.25142224587862e-12\\
48.718056640625	-1.87779156521606e-11\\
48.7385546875	4.60796408794547e-11\\
48.759052734375	-8.90884138560462e-11\\
48.77955078125	-7.51716758818783e-11\\
48.800048828125	-4.6202192408664e-11\\
48.820546875	-4.05137746403902e-11\\
48.841044921875	-2.1137432007012e-11\\
48.86154296875	2.64601061122907e-11\\
48.882041015625	-7.74631712164229e-11\\
48.9025390625	6.01617545349159e-11\\
48.923037109375	-2.70109611304181e-11\\
48.94353515625	-9.15342996511709e-11\\
48.964033203125	-7.6457361866495e-11\\
48.98453125	-2.27854114509768e-11\\
49.005029296875	-2.09500537072906e-11\\
49.02552734375	-3.1215541188534e-11\\
49.046025390625	-1.68373022862268e-10\\
49.0665234375	-4.55084932124466e-11\\
49.087021484375	-1.01422457848563e-10\\
49.10751953125	-3.26876935010315e-11\\
49.128017578125	-2.31312936547193e-11\\
49.148515625	-8.26878174637495e-12\\
49.169013671875	2.25790790202e-11\\
49.18951171875	2.85727786507663e-11\\
49.210009765625	-3.14652188302454e-11\\
49.2305078125	-2.90241681027194e-11\\
49.251005859375	-1.11838098837175e-10\\
49.27150390625	-1.03982805382486e-10\\
49.292001953125	-9.47640934390075e-11\\
49.3125	-1.17056816320813e-10\\
49.332998046875	-7.72570687396048e-11\\
49.35349609375	-4.57157921355249e-11\\
49.373994140625	2.15865699243714e-11\\
49.3944921875	-2.16239735240795e-11\\
49.414990234375	-2.47831047806771e-11\\
49.43548828125	-3.82873319227192e-11\\
49.455986328125	-8.9132206101005e-11\\
49.476484375	-1.04037114266954e-10\\
49.496982421875	-5.05095394179698e-11\\
49.51748046875	-1.46827165152471e-10\\
49.537978515625	-9.30927455608006e-11\\
49.5584765625	-1.03696384706452e-10\\
49.578974609375	-3.8480143188309e-11\\
49.59947265625	-6.23919685800642e-11\\
49.619970703125	-3.86775729149727e-11\\
49.64046875	-6.68267905048385e-11\\
49.660966796875	-1.04471256512926e-11\\
49.68146484375	-9.13544671168017e-12\\
49.701962890625	-9.71022612007944e-11\\
49.7224609375	-7.80162659329993e-11\\
49.742958984375	-5.66190333503381e-11\\
49.76345703125	-1.25214182031501e-10\\
49.783955078125	-7.02276627902791e-11\\
49.804453125	-1.53116067559475e-10\\
49.824951171875	-9.42579395699518e-11\\
49.84544921875	-4.61579706161043e-11\\
49.865947265625	-2.33481399426759e-11\\
49.8864453125	-7.01894613540978e-11\\
49.906943359375	-2.06531911403875e-11\\
49.92744140625	-6.89454693301883e-11\\
49.947939453125	-8.39786072175332e-11\\
49.9684375	-1.16055306156764e-10\\
49.988935546875	-8.62218054489541e-11\\
50.00943359375	-9.91791788071014e-11\\
50.029931640625	-1.34129737254467e-10\\
50.0504296875	-1.26641918088908e-11\\
50.070927734375	-4.03022896844891e-11\\
50.09142578125	-1.04560563442752e-12\\
50.111923828125	1.93042865420967e-11\\
50.132421875	-2.43987437175116e-11\\
50.152919921875	-2.55509468505953e-11\\
50.17341796875	-1.77571969025156e-11\\
50.193916015625	7.69784447474494e-11\\
50.2144140625	-4.6341265964261e-11\\
50.234912109375	-5.09196146854704e-12\\
50.25541015625	-1.70798445892404e-11\\
50.275908203125	-2.98624349630581e-11\\
50.29640625	-8.47315369268924e-11\\
50.316904296875	-1.00012778359293e-10\\
50.33740234375	-5.38624931967489e-11\\
50.357900390625	-3.63313879214588e-11\\
50.3783984375	-4.56824369797268e-12\\
50.398896484375	-3.66091609046066e-11\\
50.41939453125	-4.63238222834516e-11\\
50.439892578125	1.79382924983542e-11\\
50.460390625	-2.40238274487779e-11\\
50.480888671875	4.5682950478409e-13\\
50.50138671875	-1.61534116527535e-12\\
50.521884765625	4.09493109753476e-11\\
50.5423828125	4.22244683393772e-11\\
50.562880859375	5.87760176372603e-11\\
50.58337890625	6.56625265541909e-11\\
50.603876953125	9.78542927354305e-11\\
50.624375	8.30843182660849e-11\\
50.644873046875	5.01320832740202e-11\\
50.66537109375	-3.81560884936679e-11\\
50.685869140625	-6.89687260511586e-11\\
50.7063671875	-1.16378305355978e-10\\
50.726865234375	-5.92599252934776e-11\\
50.74736328125	-1.0154489963818e-10\\
50.767861328125	-7.71093937099409e-11\\
50.788359375	1.11866002663988e-11\\
50.808857421875	1.87111533184244e-11\\
50.82935546875	9.61326148258335e-11\\
50.849853515625	5.60860133993126e-11\\
50.8703515625	4.0832981880573e-11\\
50.890849609375	-2.12624154734987e-11\\
50.91134765625	-9.22835705868845e-11\\
50.931845703125	-1.46914261217175e-10\\
50.95234375	-1.30889525733183e-10\\
50.972841796875	-7.76694330522719e-11\\
50.99333984375	-1.30897490753255e-10\\
51.013837890625	-1.24983162592786e-10\\
51.0343359375	3.47733622682711e-12\\
51.054833984375	-5.4470840228919e-11\\
51.07533203125	5.45585446101603e-11\\
51.095830078125	-7.20025428579681e-11\\
51.116328125	-2.89107062713803e-12\\
51.136826171875	-1.28099883059105e-10\\
51.15732421875	-1.01399641034696e-10\\
51.177822265625	-9.18618273766619e-11\\
51.1983203125	-1.24003281432888e-10\\
51.218818359375	-4.08600164358338e-11\\
51.23931640625	-1.21010577416056e-10\\
51.259814453125	-2.91850829175565e-11\\
51.2803125	-6.65365459639833e-11\\
51.300810546875	-1.0907325901344e-10\\
51.32130859375	-4.6619320454611e-11\\
51.341806640625	-1.25252772172733e-10\\
51.3623046875	-1.58049939299361e-10\\
51.382802734375	-1.65297053870871e-10\\
51.40330078125	-9.5596610041197e-11\\
51.423798828125	-3.33921857925684e-13\\
51.444296875	-1.56908892021743e-12\\
51.464794921875	-4.31284332725041e-11\\
51.48529296875	-1.2540590331691e-12\\
51.505791015625	-3.99578985666349e-11\\
51.5262890625	-1.11608279211334e-10\\
51.546787109375	2.91091583640501e-13\\
51.56728515625	-1.47699668976521e-10\\
51.587783203125	-3.00299255309537e-11\\
51.60828125	-1.94344615808155e-10\\
51.628779296875	-5.20251427805576e-11\\
51.64927734375	-1.02691358155703e-10\\
51.669775390625	-5.51404611813983e-11\\
51.6902734375	-9.54208173183765e-11\\
51.710771484375	-1.03128731560122e-10\\
51.73126953125	-1.15442098126751e-10\\
51.751767578125	-9.05216022616718e-11\\
51.772265625	-5.59404808317048e-11\\
51.792763671875	-1.05184477506128e-10\\
51.81326171875	-4.922176321815e-11\\
51.833759765625	-8.45375598712525e-11\\
51.8542578125	-7.33299668533364e-11\\
51.874755859375	-6.94483531129779e-11\\
51.89525390625	-6.9384236700821e-11\\
51.915751953125	-1.25658845683319e-10\\
51.93625	-8.1225663103568e-11\\
51.956748046875	-6.99132239477075e-11\\
51.97724609375	-9.54361681384605e-11\\
51.997744140625	-7.95824899176826e-12\\
52.0182421875	-5.552226089238e-11\\
52.038740234375	-9.64170435745358e-12\\
52.05923828125	-4.66399416745391e-11\\
52.079736328125	-6.4193753481569e-11\\
52.100234375	9.63799100146708e-12\\
52.120732421875	-2.35452677589535e-11\\
52.14123046875	-6.33606673794837e-12\\
52.161728515625	3.97910407053114e-11\\
52.1822265625	-3.19328262011863e-11\\
52.202724609375	-4.35023344614674e-11\\
52.22322265625	-6.19225879331046e-11\\
52.243720703125	-8.7864189039897e-11\\
52.26421875	-3.28620457844239e-11\\
52.284716796875	-1.42506414603971e-10\\
52.30521484375	7.48935366705781e-11\\
52.325712890625	-3.30797762460333e-11\\
52.3462109375	-3.73857738347698e-11\\
52.366708984375	-1.05615730843571e-10\\
52.38720703125	-6.75213500334814e-11\\
52.407705078125	-1.41768866335805e-10\\
52.428203125	-9.88854241434503e-11\\
52.448701171875	-1.07340604528629e-10\\
52.46919921875	-1.03128250739053e-10\\
52.489697265625	-1.36814760758519e-10\\
52.5101953125	-1.07043777311842e-10\\
52.530693359375	-6.49627171619127e-11\\
52.55119140625	-1.29579892251184e-10\\
52.571689453125	-1.36780872338338e-10\\
52.5921875	-1.58414696831169e-10\\
52.612685546875	-1.28549503867352e-10\\
52.63318359375	-2.15139352067719e-10\\
52.653681640625	-1.16198194886783e-10\\
52.6741796875	-1.51966900587094e-10\\
52.694677734375	-1.95516033003476e-10\\
52.71517578125	-1.68875241743914e-10\\
52.735673828125	-1.83812107871025e-10\\
52.756171875	-1.89341434771539e-10\\
52.776669921875	-1.85992841212477e-10\\
52.79716796875	-1.86050428580417e-10\\
52.817666015625	-1.74882377544797e-10\\
52.8381640625	-1.65191865275696e-10\\
52.858662109375	-1.31088157114956e-10\\
52.87916015625	-1.50755323550093e-10\\
52.899658203125	-1.56705371520331e-10\\
52.92015625	-1.24950855973099e-10\\
52.940654296875	-1.48234658032606e-10\\
52.96115234375	-1.44854768298729e-10\\
52.981650390625	-1.63298586669355e-10\\
53.0021484375	-1.71671525861175e-10\\
53.022646484375	-1.87699512604299e-10\\
53.04314453125	-2.21442815226756e-10\\
53.063642578125	-1.74261603736223e-10\\
53.084140625	-1.91678261703807e-10\\
53.104638671875	-1.42334288079475e-10\\
53.12513671875	-7.97586362134854e-11\\
53.145634765625	-1.19674893114213e-10\\
53.1661328125	-6.6775179917475e-11\\
53.186630859375	-4.39243519388945e-11\\
53.20712890625	2.10908929986456e-12\\
53.227626953125	-1.37706574001038e-11\\
53.248125	1.18517102622206e-11\\
53.268623046875	-9.15824603141311e-11\\
53.28912109375	-6.66184299730606e-11\\
53.309619140625	-1.04236628334619e-10\\
53.3301171875	-1.1579488885447e-10\\
53.350615234375	-1.03531001780443e-10\\
53.37111328125	-7.5229320970824e-11\\
53.391611328125	-4.74949960525773e-11\\
53.412109375	-6.81757144577432e-11\\
53.432607421875	-2.30315506110407e-11\\
53.45310546875	1.44968318606635e-12\\
53.473603515625	-1.99584474088792e-11\\
53.4941015625	3.37295676497314e-11\\
53.514599609375	6.53652108661471e-12\\
53.53509765625	-8.34965989713201e-11\\
53.555595703125	-7.83888368737207e-11\\
53.57609375	-1.15534255401832e-10\\
53.596591796875	3.04474907751007e-11\\
53.61708984375	-1.81312262822455e-10\\
53.637587890625	-9.50614478837989e-11\\
53.6580859375	-1.83562650624492e-10\\
53.678583984375	-1.22539896912088e-10\\
53.69908203125	-9.45227537330923e-11\\
53.719580078125	-5.49438049455278e-11\\
53.740078125	-2.79011000547191e-11\\
53.760576171875	-6.58067770322346e-11\\
53.78107421875	-1.13492884370941e-11\\
53.801572265625	-5.45230344983999e-11\\
53.8220703125	-4.22597053114495e-11\\
53.842568359375	-2.81935694114082e-11\\
53.86306640625	-5.72528716624586e-11\\
53.883564453125	-1.11624764541627e-11\\
53.9040625	-1.23560412183433e-11\\
53.924560546875	-6.92318809180326e-12\\
53.94505859375	-2.45961361012636e-11\\
53.965556640625	-5.93416803642831e-11\\
53.9860546875	-1.18456228945221e-12\\
54.006552734375	-1.05590744291687e-11\\
54.02705078125	5.52299145031914e-11\\
54.047548828125	6.80926713955839e-12\\
54.068046875	7.15124108462118e-11\\
54.088544921875	-1.83086832724629e-11\\
54.10904296875	3.2172869175396e-11\\
54.129541015625	-1.63953855170488e-11\\
54.1500390625	-8.08208229399756e-12\\
54.170537109375	-7.84233827549707e-11\\
54.19103515625	-2.09470354072108e-11\\
54.211533203125	-2.55400724641352e-11\\
54.23203125	3.02537220079632e-11\\
54.252529296875	7.91096519075272e-11\\
54.27302734375	1.06559523877295e-10\\
54.293525390625	1.53578334368674e-10\\
54.3140234375	8.14256475056068e-11\\
54.334521484375	8.78946943594686e-11\\
54.35501953125	-1.70602549481123e-11\\
54.375517578125	-1.55668650590274e-11\\
54.396015625	-5.17077659566553e-11\\
54.416513671875	6.25500890070247e-11\\
54.43701171875	-4.65099852851704e-11\\
54.457509765625	2.50181299958178e-11\\
54.4780078125	6.48819370398152e-11\\
54.498505859375	5.20714726061875e-11\\
54.51900390625	7.69181678769125e-11\\
54.539501953125	8.41070081398385e-11\\
54.56	3.2217393652523e-11\\
54.580498046875	2.3913539086751e-11\\
54.60099609375	-3.7281678875347e-12\\
54.621494140625	-2.15451404540085e-12\\
54.6419921875	-6.69453525075114e-12\\
54.662490234375	7.72918775702452e-12\\
54.68298828125	2.08542917953029e-11\\
54.703486328125	4.1158950476489e-11\\
54.723984375	5.95708148561433e-11\\
54.744482421875	7.95053136302873e-11\\
54.76498046875	9.31779817535569e-11\\
54.785478515625	1.28337839874422e-10\\
54.8059765625	1.92291077481219e-11\\
54.826474609375	1.84582186061152e-12\\
54.84697265625	2.37169854875688e-12\\
54.867470703125	2.0552302294618e-11\\
54.88796875	6.81920321129097e-11\\
54.908466796875	1.2936369163445e-10\\
54.92896484375	1.13981432256626e-10\\
54.949462890625	1.6525619079377e-10\\
54.9699609375	1.19093887276583e-10\\
54.990458984375	1.36318858532082e-10\\
55.01095703125	1.58380505966554e-10\\
55.031455078125	1.42620342810807e-10\\
55.051953125	9.59630627102412e-11\\
55.072451171875	5.93302537454591e-11\\
55.09294921875	9.47855170345094e-11\\
55.113447265625	6.75943491491349e-11\\
55.1339453125	1.50385159448649e-10\\
55.154443359375	9.57483660501941e-11\\
55.17494140625	1.41274868870035e-10\\
55.195439453125	1.14166775158363e-10\\
55.2159375	1.65011602620578e-10\\
55.236435546875	1.63390151656174e-10\\
55.25693359375	1.49168354277792e-10\\
55.277431640625	9.16144316327459e-11\\
55.2979296875	1.66405042868376e-10\\
55.318427734375	1.42979589459616e-10\\
55.33892578125	1.49933923427887e-10\\
55.359423828125	1.13452047344038e-10\\
55.379921875	9.90673625761979e-11\\
55.400419921875	1.41247677424823e-10\\
55.42091796875	1.10550938544812e-10\\
55.441416015625	9.3263834995553e-11\\
55.4619140625	1.00134914295444e-10\\
55.482412109375	1.42002662529553e-10\\
55.50291015625	1.53497534643117e-10\\
55.523408203125	1.29007317501608e-10\\
55.54390625	1.46490555066268e-10\\
55.564404296875	1.31484422503767e-10\\
55.58490234375	1.73206031080918e-10\\
55.605400390625	1.40533813386823e-10\\
55.6258984375	9.28544763060169e-11\\
55.646396484375	8.96581839165335e-11\\
55.66689453125	-4.59728297184322e-12\\
55.687392578125	3.89985638300699e-11\\
55.707890625	2.32593050064779e-11\\
55.728388671875	4.87810933133769e-11\\
55.74888671875	7.51859853890844e-11\\
55.769384765625	1.10941956059778e-10\\
55.7898828125	3.10757395440205e-11\\
55.810380859375	8.24347765882488e-11\\
55.83087890625	5.5772863564934e-12\\
55.851376953125	-4.30796474683527e-13\\
55.871875	3.50523590484395e-11\\
55.892373046875	1.13108761785215e-11\\
55.91287109375	2.56069120752693e-11\\
55.933369140625	-1.17317737355878e-11\\
55.9538671875	-4.65903042664983e-12\\
55.974365234375	6.45881454276934e-12\\
55.99486328125	3.92136394907815e-11\\
56.015361328125	6.00688546751227e-11\\
56.035859375	3.22676322389675e-11\\
56.056357421875	-2.23146022190982e-11\\
56.07685546875	2.23774289698149e-11\\
56.097353515625	1.42327579190406e-12\\
56.1178515625	4.99837205782479e-11\\
56.138349609375	1.33712835073706e-11\\
56.15884765625	8.75826323247903e-11\\
56.179345703125	1.10661291088063e-11\\
56.19984375	9.30654838770225e-11\\
56.220341796875	2.05769861990189e-11\\
56.24083984375	-3.4542772426087e-11\\
56.261337890625	4.96501754735032e-12\\
56.2818359375	9.21344349548385e-11\\
56.302333984375	5.45649043465656e-11\\
56.32283203125	1.13290529858898e-11\\
56.343330078125	7.81615596136907e-11\\
56.363828125	8.93892166756976e-11\\
56.384326171875	4.4040969922077e-11\\
56.40482421875	6.98421902651566e-11\\
56.425322265625	7.40584546556896e-11\\
56.4458203125	3.41305281036005e-11\\
56.466318359375	1.06813525002646e-10\\
56.48681640625	1.91789882798197e-11\\
56.507314453125	1.26145719024363e-11\\
56.5278125	-6.91947060846019e-12\\
56.548310546875	3.60423410341391e-11\\
56.56880859375	-2.14636734902733e-11\\
56.589306640625	1.34687446220729e-10\\
56.6098046875	2.61214568220278e-11\\
56.630302734375	1.13989492600842e-10\\
56.65080078125	1.03663615247507e-10\\
56.671298828125	1.05666742942154e-10\\
56.691796875	7.24680811098314e-11\\
56.712294921875	1.81875587917368e-11\\
56.73279296875	1.00839053578446e-11\\
56.753291015625	-3.64482955983214e-11\\
56.7737890625	4.88851640892699e-11\\
56.794287109375	-8.09921711837272e-11\\
56.81478515625	7.18584231630157e-11\\
56.835283203125	2.7921025686588e-11\\
56.85578125	6.14594131865118e-11\\
56.876279296875	5.12621459213696e-11\\
56.89677734375	8.83830333225692e-11\\
56.917275390625	4.88093267477574e-11\\
56.9377734375	9.44568574302251e-11\\
};
\addlegendentry{$\text{train 3 -\textgreater{} Heimdal}$};

\addplot [color=mycolor2,solid,forget plot]
  table[row sep=crcr]{%
-47.518515625	5.29155774508793e-11\\
-47.4972265625	9.70779845633674e-11\\
-47.4759375	1.1358170191362e-10\\
-47.4546484375	1.5866233813145e-10\\
-47.433359375	1.82773108164423e-10\\
-47.4120703125	2.0499602750618e-10\\
-47.39078125	2.06403589235635e-10\\
-47.3694921875	1.94801093121974e-10\\
-47.348203125	1.56818868225731e-10\\
-47.3269140625	1.53724250383499e-10\\
-47.305625	1.14902042195769e-10\\
-47.2843359375	7.12401650110102e-11\\
-47.263046875	1.53418709275903e-10\\
-47.2417578125	1.6528274191037e-10\\
-47.22046875	1.89705518989793e-10\\
-47.1991796875	1.80403345475757e-10\\
-47.177890625	1.38320152789648e-10\\
-47.1566015625	1.17298784505298e-10\\
-47.1353125	-3.02218326867466e-11\\
-47.1140234375	-5.5493105213201e-11\\
-47.092734375	-6.67682499138173e-11\\
-47.0714453125	-4.53783512082569e-11\\
-47.05015625	-2.68658544979471e-11\\
-47.0288671875	-1.9760068350689e-11\\
-47.007578125	6.54025682022775e-13\\
-46.9862890625	6.35967535233426e-11\\
-46.965	2.8964166156563e-11\\
-46.9437109375	2.0938436169745e-11\\
-46.922421875	1.64426164092928e-11\\
-46.9011328125	-9.77651032178817e-11\\
-46.87984375	-1.05207854118394e-10\\
-46.8585546875	-1.42245781688106e-10\\
-46.837265625	-1.04291301621083e-10\\
-46.8159765625	-1.94981903951364e-10\\
-46.7946875	-3.05662777859877e-11\\
-46.7733984375	-8.63706329489752e-11\\
-46.752109375	-5.42821501085714e-11\\
-46.7308203125	1.89866588951517e-11\\
-46.70953125	1.93592639855026e-11\\
-46.6882421875	-3.74372045995695e-11\\
-46.666953125	-1.16896652781365e-10\\
-46.6456640625	-1.13765217326288e-10\\
-46.624375	-1.78393072620395e-10\\
-46.6030859375	-2.07271219656159e-10\\
-46.581796875	-2.20017475798497e-10\\
-46.5605078125	-1.27392710591841e-10\\
-46.53921875	-1.20266312362538e-10\\
-46.5179296875	-1.08455013535854e-10\\
-46.496640625	-1.05981915828958e-10\\
-46.4753515625	-1.11612053497823e-10\\
-46.4540625	-2.19092036616173e-10\\
-46.4327734375	-2.69017573262011e-10\\
-46.411484375	-2.47408017021786e-10\\
-46.3901953125	-3.5310910414069e-10\\
-46.36890625	-3.77750731128014e-10\\
-46.3476171875	-2.9064497948582e-10\\
-46.326328125	-2.57971466115386e-10\\
-46.3050390625	-2.70969061779384e-10\\
-46.28375	-1.72211725787527e-10\\
-46.2624609375	-1.04230300297022e-10\\
-46.241171875	-1.31316691864027e-10\\
-46.2198828125	-1.26243044738083e-10\\
-46.19859375	-1.71369368086583e-10\\
-46.1773046875	-1.46020165686048e-10\\
-46.156015625	-2.4658239309713e-10\\
-46.1347265625	-2.92089034797037e-10\\
-46.1134375	-2.02270120123086e-10\\
-46.0921484375	-2.05161973891309e-10\\
-46.070859375	-2.08119864907179e-10\\
-46.0495703125	-1.91939238495329e-10\\
-46.02828125	-1.16371086634411e-10\\
-46.0069921875	-1.42142907371264e-10\\
-45.985703125	-1.13022814035711e-10\\
-45.9644140625	6.32203723491884e-12\\
-45.943125	-2.93581125588209e-11\\
-45.9218359375	3.44010499940286e-11\\
-45.900546875	7.50094292431183e-12\\
-45.8792578125	-8.03974671519862e-12\\
-45.85796875	-3.62383926170906e-11\\
-45.8366796875	-2.59382991003295e-11\\
-45.815390625	2.80527524866372e-11\\
-45.7941015625	-1.64469610485155e-11\\
-45.7728125	6.69100592718734e-11\\
-45.7515234375	5.60899103382088e-11\\
-45.730234375	8.31511859011465e-11\\
-45.7089453125	9.1076452243701e-11\\
-45.68765625	1.19735536301529e-10\\
-45.6663671875	4.01823131430705e-11\\
-45.645078125	1.03675822643476e-10\\
-45.6237890625	-7.2237671766161e-12\\
-45.6025	4.7976238847955e-11\\
-45.5812109375	-2.12610395758434e-11\\
-45.559921875	2.51689848839399e-11\\
-45.5386328125	4.76874801335499e-11\\
-45.51734375	5.45450533132591e-11\\
-45.4960546875	7.97941312994556e-11\\
-45.474765625	1.07876846260321e-10\\
-45.4534765625	8.80272935011499e-11\\
-45.4321875	1.26803140390802e-10\\
-45.4108984375	1.02790408654202e-10\\
-45.389609375	1.17605072189178e-10\\
-45.3683203125	1.25170578130802e-10\\
-45.34703125	1.80463999332644e-10\\
-45.3257421875	1.55573542611982e-10\\
-45.304453125	1.63863404343055e-10\\
-45.2831640625	2.14421559100987e-10\\
-45.261875	1.7131748170767e-10\\
-45.2405859375	1.72647190662726e-10\\
-45.219296875	1.73083996777157e-10\\
-45.1980078125	8.78190306287563e-11\\
-45.17671875	1.52706733222115e-10\\
-45.1554296875	1.22800064192168e-10\\
-45.134140625	9.55279103508516e-11\\
-45.1128515625	1.15195815329857e-10\\
-45.0915625	6.13119610291112e-11\\
-45.0702734375	8.21958238588483e-11\\
-45.048984375	6.33587059262955e-12\\
-45.0276953125	1.24953123466953e-10\\
-45.00640625	4.05355181481361e-12\\
-44.9851171875	-4.39125873416544e-12\\
-44.963828125	-6.91749138037958e-11\\
-44.9425390625	-6.77489754391447e-11\\
-44.92125	-1.26156796901286e-10\\
-44.8999609375	-1.41483140699026e-10\\
-44.878671875	-1.44656997076907e-10\\
-44.8573828125	-1.42197498333521e-10\\
-44.83609375	-6.2273735279944e-11\\
-44.8148046875	-5.61961832023098e-11\\
-44.793515625	4.30880216426196e-11\\
-44.7722265625	-2.24507872082202e-11\\
-44.7509375	2.66891457077e-11\\
-44.7296484375	-1.19209724225298e-11\\
-44.708359375	-3.79250065346772e-11\\
-44.6870703125	-8.59852523339514e-11\\
-44.66578125	-1.52738438065987e-10\\
-44.6444921875	-1.25804279322415e-10\\
-44.623203125	-1.31772209625488e-10\\
-44.6019140625	-1.01342378829149e-10\\
-44.580625	-9.34888923059763e-11\\
-44.5593359375	-3.28707272342485e-11\\
-44.538046875	1.05117832261884e-11\\
-44.5167578125	4.13295010350411e-11\\
-44.49546875	2.92972251333371e-12\\
-44.4741796875	1.17336429160395e-10\\
-44.452890625	1.02321350242954e-10\\
-44.4316015625	6.01676311630422e-11\\
-44.4103125	1.07066283261981e-10\\
-44.3890234375	6.01151290860026e-11\\
-44.367734375	5.29115751703482e-12\\
-44.3464453125	5.71218746339586e-12\\
-44.32515625	5.26768932272226e-11\\
-44.3038671875	9.19081430823267e-11\\
-44.282578125	8.04476084226586e-11\\
-44.2612890625	1.89982085281939e-10\\
-44.24	1.55514522715842e-10\\
-44.2187109375	1.86161043184828e-10\\
-44.197421875	1.89671827141297e-10\\
-44.1761328125	1.38000618109898e-10\\
-44.15484375	1.43909168464076e-10\\
-44.1335546875	8.44475372659271e-11\\
-44.112265625	1.1864892614385e-10\\
-44.0909765625	1.32252921816064e-10\\
-44.0696875	1.69977144577169e-10\\
-44.0483984375	2.55845343136269e-10\\
-44.027109375	2.08700848114075e-10\\
-44.0058203125	1.45428057496815e-10\\
-43.98453125	1.63607413790608e-10\\
-43.9632421875	7.99655163104507e-11\\
-43.941953125	3.73382628965705e-11\\
-43.9206640625	-4.53336683577539e-11\\
-43.899375	-2.37754346693146e-11\\
-43.8780859375	-3.52832983330116e-11\\
-43.856796875	-6.67679233776377e-11\\
-43.8355078125	9.17523551492313e-11\\
-43.81421875	1.68117221722805e-10\\
-43.7929296875	8.36755447219881e-11\\
-43.771640625	1.20369170423629e-10\\
-43.7503515625	1.62001482680679e-10\\
-43.7290625	1.77058438408373e-10\\
-43.7077734375	9.64743743340218e-11\\
-43.686484375	1.12740134639912e-10\\
-43.6651953125	1.50101033417879e-10\\
-43.64390625	1.37280791625202e-10\\
-43.6226171875	1.20152844634867e-10\\
-43.601328125	1.21987932302448e-10\\
-43.5800390625	7.38264073147845e-11\\
-43.55875	2.63828752641803e-11\\
-43.5374609375	8.39486675461049e-11\\
-43.516171875	1.36224868950572e-10\\
-43.4948828125	1.6149542297086e-10\\
-43.47359375	2.45246903017472e-10\\
-43.4523046875	2.90165197650066e-10\\
-43.431015625	3.46528757543709e-10\\
-43.4097265625	2.98714683426647e-10\\
-43.3884375	3.03001470008213e-10\\
-43.3671484375	2.11177240804052e-10\\
-43.345859375	2.04121240804478e-10\\
-43.3245703125	1.33803508201605e-10\\
-43.30328125	1.34984283340875e-10\\
-43.2819921875	2.63794719690174e-10\\
-43.260703125	1.72188335429429e-10\\
-43.2394140625	2.47293834149193e-10\\
-43.218125	3.06132850266271e-10\\
-43.1968359375	3.45497030047432e-10\\
-43.175546875	3.71271079266718e-10\\
-43.1542578125	2.98707960459007e-10\\
-43.13296875	2.20711247529499e-10\\
-43.1116796875	2.47504534998117e-10\\
-43.090390625	1.44885773826044e-10\\
-43.0691015625	1.53076313076745e-10\\
-43.0478125	1.25839527870764e-10\\
-43.0265234375	2.10634753597314e-10\\
-43.005234375	2.12579678802196e-10\\
-42.9839453125	2.13235720799761e-10\\
-42.96265625	2.55795973142701e-10\\
-42.9413671875	1.92930408963974e-10\\
-42.920078125	1.7638062088508e-10\\
-42.8987890625	8.1995654068475e-11\\
-42.8775	1.03451701342101e-10\\
-42.8562109375	1.07819265653326e-10\\
-42.834921875	4.97703482017595e-11\\
-42.8136328125	1.54541408335221e-10\\
-42.79234375	8.36980510743624e-11\\
-42.7710546875	9.39205752602593e-11\\
-42.749765625	1.41122658685855e-10\\
-42.7284765625	1.41489358339238e-10\\
-42.7071875	4.1020621356425e-11\\
-42.6858984375	5.14649658263227e-12\\
-42.664609375	3.72219224680674e-11\\
-42.6433203125	8.04728818845933e-12\\
-42.62203125	-3.9347558335268e-11\\
-42.6007421875	-5.26432172374513e-11\\
-42.579453125	3.77222553318771e-11\\
-42.5581640625	-1.2009458434963e-11\\
-42.536875	-2.84210709438008e-11\\
-42.5155859375	3.21532131813644e-11\\
-42.494296875	1.85941358885908e-10\\
-42.4730078125	2.17487481132035e-11\\
-42.45171875	6.75550240683572e-11\\
-42.4304296875	3.11630704716376e-11\\
-42.409140625	1.76315686324879e-11\\
-42.3878515625	1.52708394130478e-11\\
-42.3665625	9.0044308574867e-11\\
-42.3452734375	1.73280195717694e-11\\
-42.323984375	9.58871075313194e-11\\
-42.3026953125	1.94712388104034e-10\\
-42.28140625	2.11153831563147e-10\\
-42.2601171875	1.31162246382289e-10\\
-42.238828125	8.36871215410071e-11\\
-42.2175390625	1.69432498055825e-10\\
-42.19625	-4.66571063438124e-12\\
-42.1749609375	-3.52303510688395e-11\\
-42.153671875	-1.19334491535304e-12\\
-42.1323828125	2.54184313982607e-11\\
-42.11109375	-5.56885008377423e-12\\
-42.0898046875	8.76158801960976e-11\\
-42.068515625	1.23832923518789e-10\\
-42.0472265625	4.9844714054722e-11\\
-42.0259375	8.71902861654019e-11\\
-42.0046484375	3.13797993996881e-11\\
-41.983359375	-4.12020402899154e-12\\
-41.9620703125	-3.73060117372478e-12\\
-41.94078125	-3.81681981041899e-11\\
-41.9194921875	-2.78053188170869e-12\\
-41.898203125	-7.27381741077828e-11\\
-41.8769140625	-3.61988947059052e-11\\
-41.855625	-3.23469594222359e-12\\
-41.8343359375	3.31107914230018e-11\\
-41.813046875	2.0477121399307e-12\\
-41.7917578125	5.26350910454071e-11\\
-41.77046875	7.77710695599145e-11\\
-41.7491796875	-1.73389566747592e-11\\
-41.727890625	2.59970520420753e-11\\
-41.7066015625	-7.96238450512637e-11\\
-41.6853125	-8.81609172267562e-11\\
-41.6640234375	-1.4062313302991e-10\\
-41.642734375	-1.23682304709545e-10\\
-41.6214453125	-1.49175627277969e-10\\
-41.60015625	-1.45728720625478e-10\\
-41.5788671875	-1.01887117016857e-10\\
-41.557578125	-1.54840174981604e-10\\
-41.5362890625	-1.86429132609531e-10\\
-41.515	-2.14958817634413e-10\\
-41.4937109375	-2.40077733403898e-10\\
-41.472421875	-2.05404211207958e-10\\
-41.4511328125	-2.78759787231119e-10\\
-41.42984375	-2.90485319382371e-10\\
-41.4085546875	-2.73185143007816e-10\\
-41.387265625	-2.70243198669381e-10\\
-41.3659765625	-2.48793707389451e-10\\
-41.3446875	-2.96253374518434e-10\\
-41.3233984375	-3.05773718585211e-10\\
-41.302109375	-2.2446435374289e-10\\
-41.2808203125	-3.07474051993183e-10\\
-41.25953125	-3.39831065384178e-10\\
-41.2382421875	-3.78546634616406e-10\\
-41.216953125	-3.17695545226983e-10\\
-41.1956640625	-2.79231999560318e-10\\
-41.174375	-3.05470772189675e-10\\
-41.1530859375	-2.00282628284485e-10\\
-41.131796875	-2.59617146686948e-10\\
-41.1105078125	-2.95410740694528e-10\\
-41.08921875	-3.14595683381509e-10\\
-41.0679296875	-3.41248941068127e-10\\
-41.046640625	-3.88321151023571e-10\\
-41.0253515625	-2.90132810393941e-10\\
-41.0040625	-2.93116002540584e-10\\
-40.9827734375	-1.6999653211139e-10\\
-40.961484375	-1.93832081881825e-10\\
-40.9401953125	-1.99217527341991e-10\\
-40.91890625	-1.5287245802657e-10\\
-40.8976171875	-2.20303037881917e-10\\
-40.876328125	-2.42330089325176e-10\\
-40.8550390625	-2.5096384763819e-10\\
-40.83375	-2.95249249759827e-10\\
-40.8124609375	-2.66839733020109e-10\\
-40.791171875	-3.26700041572821e-10\\
-40.7698828125	-1.98461288794722e-10\\
-40.74859375	-1.72549161849469e-10\\
-40.7273046875	-1.31564100559487e-10\\
-40.706015625	-1.82530793752098e-10\\
-40.6847265625	-2.32887121368086e-10\\
-40.6634375	-2.30643158579715e-10\\
-40.6421484375	-2.5062533288206e-10\\
-40.620859375	-2.9354932812098e-10\\
-40.5995703125	-2.43842061959472e-10\\
-40.57828125	-2.68019884379886e-10\\
-40.5569921875	-2.34229832934204e-10\\
-40.535703125	-1.25705159672672e-10\\
-40.5144140625	-1.83092800594024e-10\\
-40.493125	-1.27326120813954e-10\\
-40.4718359375	-2.72462134399535e-10\\
-40.450546875	-1.85537808199138e-10\\
-40.4292578125	-2.60569328505369e-10\\
-40.40796875	-2.49865132449149e-10\\
-40.3866796875	-3.46853174696979e-10\\
-40.365390625	-3.04320557635117e-10\\
-40.3441015625	-2.47919011857998e-10\\
-40.3228125	-2.145732708055e-10\\
-40.3015234375	-2.26376654917507e-10\\
-40.280234375	-1.0669714787636e-10\\
-40.2589453125	-2.22832295734967e-10\\
-40.23765625	-1.80522396263465e-10\\
-40.2163671875	-1.40369321963668e-10\\
-40.195078125	-2.18631227137111e-10\\
-40.1737890625	-2.17487160610845e-10\\
-40.1525	-3.43963279546434e-10\\
-40.1312109375	-1.88645812086121e-10\\
-40.109921875	-1.39172511584895e-10\\
-40.0886328125	-1.25337199939981e-10\\
-40.06734375	1.21761460324299e-11\\
-40.0460546875	-6.80057638669146e-11\\
-40.024765625	9.74363038976825e-12\\
-40.0034765625	-5.3773147288462e-11\\
-39.9821875	5.31596104896157e-11\\
-39.9608984375	-1.38877642765224e-11\\
-39.939609375	-1.11265746201521e-10\\
-39.9183203125	1.13994212783059e-11\\
-39.89703125	-6.84364171827825e-11\\
-39.8757421875	-3.3367973408214e-11\\
-39.854453125	3.929462697447e-11\\
-39.8331640625	7.72173515895137e-11\\
-39.811875	-1.8258116725e-11\\
-39.7905859375	2.9736888506127e-11\\
-39.769296875	-2.56413053158212e-11\\
-39.7480078125	4.23926461253881e-12\\
-39.72671875	-1.36663784539977e-10\\
-39.7054296875	-7.94664648809139e-11\\
-39.684140625	-4.90135337215526e-12\\
-39.6628515625	-1.29637411456686e-10\\
-39.6415625	5.26754416126141e-11\\
-39.6202734375	1.93218832597638e-11\\
-39.598984375	5.43665546357155e-11\\
-39.5776953125	3.93907013142185e-11\\
-39.55640625	6.35205205634607e-12\\
-39.5351171875	-4.77674328645792e-12\\
-39.513828125	3.30017429362723e-11\\
-39.4925390625	-2.06196427450462e-11\\
-39.47125	-5.87190568318098e-11\\
-39.4499609375	7.56034970892879e-11\\
-39.428671875	1.52785811280768e-10\\
-39.4073828125	1.59567952758194e-10\\
-39.38609375	1.28791348837358e-10\\
-39.3648046875	1.73787582786069e-10\\
-39.343515625	8.4795009093286e-11\\
-39.3222265625	1.16621669889114e-10\\
-39.3009375	6.87179332367182e-11\\
-39.2796484375	1.06171380612462e-10\\
-39.258359375	1.18799884709861e-10\\
-39.2370703125	4.535851268441e-11\\
-39.21578125	1.44450072442705e-10\\
-39.1944921875	1.97017651793253e-10\\
-39.173203125	2.33987515084458e-10\\
-39.1519140625	1.60789952285028e-10\\
-39.130625	1.95125306129196e-10\\
-39.1093359375	2.11436017432292e-10\\
-39.088046875	1.2466263728524e-10\\
-39.0667578125	2.72972301169508e-10\\
-39.04546875	2.56869203161872e-10\\
-39.0241796875	2.30801199778122e-10\\
-39.002890625	2.65278162775078e-10\\
-38.9816015625	2.75642060824318e-10\\
-38.9603125	2.38736093321849e-10\\
-38.9390234375	2.79477423558074e-10\\
-38.917734375	2.2410548478694e-10\\
-38.8964453125	3.87236157069185e-10\\
-38.87515625	3.82064761794799e-10\\
-38.8538671875	4.0751982197572e-10\\
-38.832578125	3.51435072335164e-10\\
-38.8112890625	4.10248668881984e-10\\
-38.79	4.54117138784732e-10\\
-38.7687109375	4.26902792299895e-10\\
-38.747421875	4.56537129936012e-10\\
-38.7261328125	5.42011795419594e-10\\
-38.70484375	4.48842385689107e-10\\
-38.6835546875	5.23957372899947e-10\\
-38.662265625	5.09192794551758e-10\\
-38.6409765625	5.38286944026484e-10\\
-38.6196875	4.69393976469017e-10\\
-38.5983984375	4.14185808729266e-10\\
-38.577109375	4.95444871287085e-10\\
-38.5558203125	3.56082952777164e-10\\
-38.53453125	4.52066322351276e-10\\
-38.5132421875	5.21539134198921e-10\\
-38.491953125	5.11784122173519e-10\\
-38.4706640625	5.77334857296168e-10\\
-38.449375	6.03570644687967e-10\\
-38.4280859375	5.70769003701278e-10\\
-38.406796875	5.35471708024781e-10\\
-38.3855078125	4.89357649744275e-10\\
-38.36421875	4.1228722759887e-10\\
-38.3429296875	3.82099111043224e-10\\
-38.321640625	2.44160631958792e-10\\
-38.3003515625	3.1899273407314e-10\\
-38.2790625	3.24672605564119e-10\\
-38.2577734375	3.39798566389652e-10\\
-38.236484375	3.59520389435373e-10\\
-38.2151953125	3.74215584116242e-10\\
-38.19390625	4.16607754939199e-10\\
-38.1726171875	4.5501916984409e-10\\
-38.151328125	3.46392284923851e-10\\
-38.1300390625	4.30561401935984e-10\\
-38.10875	3.72915444123223e-10\\
-38.0874609375	3.08874949865162e-10\\
-38.066171875	3.1067902790119e-10\\
-38.0448828125	3.16143823079155e-10\\
-38.02359375	2.69224704019778e-10\\
-38.0023046875	2.48803515876662e-10\\
-37.981015625	2.62167149664987e-10\\
-37.9597265625	2.46916510814555e-10\\
-37.9384375	3.04250505715527e-10\\
-37.9171484375	2.5962931180685e-10\\
-37.895859375	2.89145477334786e-10\\
-37.8745703125	2.27718445216141e-10\\
-37.85328125	2.56595539472517e-10\\
-37.8319921875	2.99347225096771e-10\\
-37.810703125	2.30824644161756e-10\\
-37.7894140625	2.30556368007088e-10\\
-37.768125	2.52890979524412e-10\\
-37.7468359375	2.69474700975655e-10\\
-37.725546875	1.86122093788683e-10\\
-37.7042578125	2.39312668351401e-10\\
-37.68296875	2.56061811848872e-10\\
-37.6616796875	1.55411803861379e-10\\
-37.640390625	1.06516787646275e-10\\
-37.6191015625	1.03171408912548e-10\\
-37.5978125	4.47295927745142e-11\\
-37.5765234375	1.19566068856739e-10\\
-37.555234375	6.86537048295572e-11\\
-37.5339453125	-1.05830623170864e-11\\
-37.51265625	8.37529369727528e-11\\
-37.4913671875	4.42562414168419e-11\\
-37.470078125	3.04962737432649e-11\\
-37.4487890625	2.24430821970629e-10\\
-37.4275	9.25657259243452e-11\\
-37.4062109375	1.37257542244541e-10\\
-37.384921875	2.30884156549793e-10\\
-37.3636328125	3.67175300251238e-11\\
-37.34234375	9.24625184348158e-12\\
-37.3210546875	-1.72765307230731e-11\\
-37.299765625	-1.33544748141063e-10\\
-37.2784765625	-1.06761923111136e-10\\
-37.2571875	-9.65457949487038e-11\\
-37.2358984375	-7.21433126936823e-11\\
-37.214609375	-7.31982621710942e-11\\
-37.1933203125	-6.70786374897109e-11\\
-37.17203125	3.63950556151428e-11\\
-37.1507421875	3.42682488439532e-11\\
-37.129453125	-3.90799237036282e-11\\
-37.1081640625	-4.6018009745756e-11\\
-37.086875	-1.22631149376008e-10\\
-37.0655859375	-9.61273777086113e-11\\
-37.044296875	-1.28731530951276e-10\\
-37.0230078125	-2.23525073712727e-10\\
-37.00171875	-1.13847456297592e-10\\
-36.9804296875	-7.21439812529876e-11\\
-36.959140625	-1.10009176732877e-10\\
-36.9378515625	-1.29979555971997e-10\\
-36.9165625	-1.05892209741757e-10\\
-36.8952734375	-1.49693085530521e-10\\
-36.873984375	-2.42000553383525e-10\\
-36.8526953125	-2.68396096709741e-10\\
-36.83140625	-2.98254785112373e-10\\
-36.8101171875	-2.20684260789666e-10\\
-36.788828125	-3.13006325258272e-10\\
-36.7675390625	-2.92799545038089e-10\\
-36.74625	-2.01700287958614e-10\\
-36.7249609375	-2.34703613417152e-10\\
-36.703671875	-2.52017246345565e-10\\
-36.6823828125	-2.02897202173963e-10\\
-36.66109375	-2.62692039946838e-10\\
-36.6398046875	-2.95280919597997e-10\\
-36.618515625	-3.56603896732594e-10\\
-36.5972265625	-3.87421440146995e-10\\
-36.5759375	-4.15963757996609e-10\\
-36.5546484375	-4.56504443505096e-10\\
-36.533359375	-3.65332603891708e-10\\
-36.5120703125	-4.50791168102189e-10\\
-36.49078125	-3.56877716971135e-10\\
-36.4694921875	-4.08802003963519e-10\\
-36.448203125	-4.02289451563317e-10\\
-36.4269140625	-3.82915735162857e-10\\
-36.405625	-3.59544648723432e-10\\
-36.3843359375	-3.47420858705827e-10\\
-36.363046875	-4.4102486590449e-10\\
-36.3417578125	-3.76599348384558e-10\\
-36.32046875	-3.99129906501087e-10\\
-36.2991796875	-5.18591148742489e-10\\
-36.277890625	-4.06156805030712e-10\\
-36.2566015625	-4.7329450902559e-10\\
-36.2353125	-5.51646672922311e-10\\
-36.2140234375	-4.73807209251024e-10\\
-36.192734375	-5.35654145010595e-10\\
-36.1714453125	-4.9670849433066e-10\\
-36.15015625	-6.07474069476463e-10\\
-36.1288671875	-6.66184567054791e-10\\
-36.107578125	-6.0718057839295e-10\\
-36.0862890625	-7.09670225821994e-10\\
-36.065	-6.59421390530502e-10\\
-36.0437109375	-5.35272609735249e-10\\
-36.022421875	-5.68894688784845e-10\\
-36.0011328125	-4.86783662960255e-10\\
-35.97984375	-4.42766994951941e-10\\
-35.9585546875	-4.65376955565877e-10\\
-35.937265625	-5.16351620297925e-10\\
-35.9159765625	-6.03108793584168e-10\\
-35.8946875	-7.19686358425681e-10\\
-35.8733984375	-6.92315151982937e-10\\
-35.852109375	-7.86995651635926e-10\\
-35.8308203125	-7.25186685224768e-10\\
-35.80953125	-5.92797854077001e-10\\
-35.7882421875	-6.19713866187192e-10\\
-35.766953125	-5.3555974221478e-10\\
-35.7456640625	-4.52838515202553e-10\\
-35.724375	-4.08245346487132e-10\\
-35.7030859375	-4.01020660136568e-10\\
-35.681796875	-4.18200329383807e-10\\
-35.6605078125	-5.34527354723311e-10\\
-35.63921875	-5.65172708851426e-10\\
-35.6179296875	-5.05483876625737e-10\\
-35.596640625	-6.09968638342303e-10\\
-35.5753515625	-5.27234234423927e-10\\
-35.5540625	-4.49996523818606e-10\\
-35.5327734375	-3.71020650476319e-10\\
-35.511484375	-3.58079588932462e-10\\
-35.4901953125	-3.20905209758195e-10\\
-35.46890625	-3.47972973971043e-10\\
-35.4476171875	-3.25182900912537e-10\\
-35.426328125	-4.23994619834214e-10\\
-35.4050390625	-3.55239696049403e-10\\
-35.38375	-4.29405902130196e-10\\
-35.3624609375	-4.4732116492375e-10\\
-35.341171875	-3.73261970969626e-10\\
-35.3198828125	-3.94721020293304e-10\\
-35.29859375	-2.72031367008171e-10\\
-35.2773046875	-2.72559210854468e-10\\
-35.256015625	-2.74437140766103e-10\\
-35.2347265625	-2.3296239026265e-10\\
-35.2134375	-1.95531146324875e-10\\
-35.1921484375	-2.06139838059488e-10\\
-35.170859375	-1.82651593610658e-10\\
-35.1495703125	-1.77690953215239e-10\\
-35.12828125	-2.08660257625442e-10\\
-35.1069921875	-2.47338551233232e-10\\
-35.085703125	-1.16806354310474e-10\\
-35.0644140625	-9.40725233091522e-11\\
-35.043125	3.51106655926937e-11\\
-35.0218359375	-4.97765463171765e-11\\
-35.000546875	5.39876752790276e-11\\
-34.9792578125	9.96168522385431e-11\\
-34.95796875	7.18596274530621e-12\\
-34.9366796875	2.26744588482208e-11\\
-34.915390625	-3.97868287217522e-11\\
-34.8941015625	-1.03163189262888e-10\\
-34.8728125	-4.60712071808912e-11\\
-34.8515234375	-8.13507101867786e-11\\
-34.830234375	-2.57228530851769e-11\\
-34.8089453125	3.97151177297165e-11\\
-34.78765625	3.60166299127391e-11\\
-34.7663671875	1.363849799001e-10\\
-34.745078125	8.50992834582388e-11\\
-34.7237890625	1.43395192966671e-10\\
-34.7025	1.28855333751826e-10\\
-34.6812109375	-2.27263351963229e-11\\
-34.659921875	8.41429665924562e-11\\
-34.6386328125	1.24578585571977e-10\\
-34.61734375	2.06434191521059e-11\\
-34.5960546875	1.44148210104819e-10\\
-34.574765625	1.14237542465236e-10\\
-34.5534765625	9.39987299467518e-11\\
-34.5321875	2.10467652451581e-10\\
-34.5108984375	1.6252997756757e-10\\
-34.489609375	1.5240078342207e-10\\
-34.4683203125	1.81540533922962e-10\\
-34.44703125	1.55602184797537e-10\\
-34.4257421875	1.83755154493296e-10\\
-34.404453125	2.45861509776022e-10\\
-34.3831640625	2.047055233263e-10\\
-34.361875	2.68849877903484e-10\\
-34.3405859375	3.18041922400863e-10\\
-34.319296875	3.30200044716254e-10\\
-34.2980078125	3.99834632092013e-10\\
-34.27671875	4.09397455588113e-10\\
-34.2554296875	2.58513431823748e-10\\
-34.234140625	3.34789743754332e-10\\
-34.2128515625	4.03537193115146e-10\\
-34.1915625	3.79971384413243e-10\\
-34.1702734375	4.51370586948268e-10\\
-34.148984375	3.86382115990418e-10\\
-34.1276953125	4.6289444405247e-10\\
-34.10640625	4.49709008403484e-10\\
-34.0851171875	5.13911717058098e-10\\
-34.063828125	5.48445397364249e-10\\
-34.0425390625	4.52225604910426e-10\\
-34.02125	5.4379799011176e-10\\
-33.9999609375	5.47088342000232e-10\\
-33.978671875	4.71502601082647e-10\\
-33.9573828125	5.40918512129795e-10\\
-33.93609375	5.5250695991496e-10\\
-33.9148046875	5.23006298599261e-10\\
-33.893515625	5.75596223774602e-10\\
-33.8722265625	5.19173156073481e-10\\
-33.8509375	5.55682121455417e-10\\
-33.8296484375	6.0034266305283e-10\\
-33.808359375	5.93238004112519e-10\\
-33.7870703125	5.90960110536708e-10\\
-33.76578125	5.97241091673191e-10\\
-33.7444921875	5.61642562588185e-10\\
-33.723203125	6.00705067317595e-10\\
-33.7019140625	5.69066997305166e-10\\
-33.680625	6.57780295645402e-10\\
-33.6593359375	5.810955542475e-10\\
-33.638046875	6.45486744002963e-10\\
-33.6167578125	6.27587049173708e-10\\
-33.59546875	6.1329847357963e-10\\
-33.5741796875	6.87724780452477e-10\\
-33.552890625	6.77025998121777e-10\\
-33.5316015625	6.99872517223025e-10\\
-33.5103125	6.78380213122545e-10\\
-33.4890234375	6.55766317620499e-10\\
-33.467734375	6.78639229287643e-10\\
-33.4464453125	5.72924456996566e-10\\
-33.42515625	5.91680780974764e-10\\
-33.4038671875	6.32839379263767e-10\\
-33.382578125	6.58691749486599e-10\\
-33.3612890625	6.49876121825306e-10\\
-33.34	7.36776708133018e-10\\
-33.3187109375	7.16082117442217e-10\\
-33.297421875	6.90503211631552e-10\\
-33.2761328125	7.269571884884e-10\\
-33.25484375	6.29045046389757e-10\\
-33.2335546875	5.20880925485999e-10\\
-33.212265625	5.72116604134967e-10\\
-33.1909765625	5.4124961164109e-10\\
-33.1696875	5.42266212016996e-10\\
-33.1483984375	5.10664467765832e-10\\
-33.127109375	5.43867109473813e-10\\
-33.1058203125	5.63408742620253e-10\\
-33.08453125	6.06281760679857e-10\\
-33.0632421875	6.22164457743286e-10\\
-33.041953125	5.38490193859386e-10\\
-33.0206640625	5.79648332205288e-10\\
-32.999375	5.10208272733696e-10\\
-32.9780859375	4.97421725360188e-10\\
-32.956796875	3.88278000228688e-10\\
-32.9355078125	4.14137680635712e-10\\
-32.91421875	4.01852694842387e-10\\
-32.8929296875	4.25546360520212e-10\\
-32.871640625	4.31635259282376e-10\\
-32.8503515625	4.55550980544522e-10\\
-32.8290625	3.95446206079085e-10\\
-32.8077734375	4.18206939464106e-10\\
-32.786484375	3.83258086583326e-10\\
-32.7651953125	4.23756775143684e-10\\
-32.74390625	3.05705884335276e-10\\
-32.7226171875	2.46399047772098e-10\\
-32.701328125	2.63382500242584e-10\\
-32.6800390625	1.56495777257966e-10\\
-32.65875	2.49963883671318e-10\\
-32.6374609375	2.5825563555633e-10\\
-32.616171875	2.83398648828909e-10\\
-32.5948828125	2.20827458613004e-10\\
-32.57359375	2.19565242862538e-10\\
-32.5523046875	1.71158416396518e-10\\
-32.531015625	1.43553959328681e-10\\
-32.5097265625	3.25996369055945e-11\\
-32.4884375	-4.46215330772881e-11\\
-32.4671484375	-3.02891961819896e-11\\
-32.445859375	-1.65229587066262e-10\\
-32.4245703125	-4.1102173251747e-11\\
-32.40328125	-4.32717711721269e-11\\
-32.3819921875	-4.65077726542865e-11\\
-32.360703125	-1.32089014285449e-11\\
-32.3394140625	-1.76248208475156e-11\\
-32.318125	-3.49413344607311e-11\\
-32.2968359375	-6.8081593167559e-11\\
-32.275546875	-9.70790774889753e-11\\
-32.2542578125	-1.67092352234038e-10\\
-32.23296875	-1.84748631137109e-10\\
-32.2116796875	-2.98486212184368e-10\\
-32.190390625	-1.97364577311816e-10\\
-32.1691015625	-1.06412987927398e-10\\
-32.1478125	-1.97930667438569e-10\\
-32.1265234375	-1.00080111093284e-10\\
-32.105234375	-1.10051412905807e-10\\
-32.0839453125	-1.99365910429173e-10\\
-32.06265625	-2.42656738892794e-10\\
-32.0413671875	-3.49539015470662e-10\\
-32.020078125	-3.76175012335505e-10\\
-31.9987890625	-4.5742931030842e-10\\
-31.9775	-3.74891623152478e-10\\
-31.9562109375	-4.6965127438506e-10\\
-31.934921875	-3.48042012439893e-10\\
-31.9136328125	-3.48360390054994e-10\\
-31.89234375	-3.50279706862341e-10\\
-31.8710546875	-3.03576276164861e-10\\
-31.849765625	-4.14951274622131e-10\\
-31.8284765625	-4.31319621843054e-10\\
-31.8071875	-5.21217547424302e-10\\
-31.7858984375	-5.54051165619894e-10\\
-31.764609375	-5.34037005314658e-10\\
-31.7433203125	-5.96008785828491e-10\\
-31.72203125	-6.32775281659717e-10\\
-31.7007421875	-5.47908027926416e-10\\
-31.679453125	-6.4716197621427e-10\\
-31.6581640625	-5.57428692820443e-10\\
-31.636875	-6.56869158366079e-10\\
-31.6155859375	-6.76336201261324e-10\\
-31.594296875	-7.11221513766821e-10\\
-31.5730078125	-7.424496587066e-10\\
-31.55171875	-7.3922151537353e-10\\
-31.5304296875	-7.65566497098553e-10\\
-31.509140625	-7.36584550880243e-10\\
-31.4878515625	-6.85519451404097e-10\\
-31.4665625	-6.49478843837129e-10\\
-31.4452734375	-6.82928830488031e-10\\
-31.423984375	-6.61023072756227e-10\\
-31.4026953125	-6.41012130393881e-10\\
-31.38140625	-7.06318814685025e-10\\
-31.3601171875	-6.85164509973381e-10\\
-31.338828125	-8.02952677481358e-10\\
-31.3175390625	-8.00768619161067e-10\\
-31.29625	-7.63027036828861e-10\\
-31.2749609375	-8.50354861390903e-10\\
-31.253671875	-7.68548411912128e-10\\
-31.2323828125	-8.03462613742874e-10\\
-31.21109375	-8.34893826900596e-10\\
-31.1898046875	-8.3823311610475e-10\\
-31.168515625	-8.73620185868114e-10\\
-31.1472265625	-8.38730124770188e-10\\
-31.1259375	-8.71175585509523e-10\\
-31.1046484375	-8.99268935477651e-10\\
-31.083359375	-9.83419586215738e-10\\
-31.0620703125	-1.05165299636472e-09\\
-31.04078125	-1.06785410033639e-09\\
-31.0194921875	-1.13283454160014e-09\\
-30.998203125	-1.08996461167399e-09\\
-30.9769140625	-1.20137442459873e-09\\
-30.955625	-1.14912839621936e-09\\
-30.9343359375	-1.12836957230562e-09\\
-30.913046875	-1.26124605607446e-09\\
-30.8917578125	-1.15653920195177e-09\\
-30.87046875	-1.1792223637081e-09\\
-30.8491796875	-1.20934393323987e-09\\
-30.827890625	-1.20118955282118e-09\\
-30.8066015625	-1.14993349930153e-09\\
-30.7853125	-1.21251310336929e-09\\
-30.7640234375	-1.1689766516115e-09\\
-30.742734375	-1.2700503827055e-09\\
-30.7214453125	-1.21752963359725e-09\\
-30.70015625	-1.17111226693096e-09\\
-30.6788671875	-1.11424966846725e-09\\
-30.657578125	-9.92506950836917e-10\\
-30.6362890625	-9.26518601224929e-10\\
-30.615	-8.11630840133296e-10\\
-30.5937109375	-8.28996702416898e-10\\
-30.572421875	-8.74957367312035e-10\\
-30.5511328125	-8.94279758666054e-10\\
-30.52984375	-8.49401992738338e-10\\
-30.5085546875	-9.08172009936143e-10\\
-30.487265625	-8.95863209159528e-10\\
-30.4659765625	-8.10285015540614e-10\\
-30.4446875	-8.17223742656453e-10\\
-30.4233984375	-7.16243920769823e-10\\
-30.402109375	-6.79990977825034e-10\\
-30.3808203125	-5.22285725768835e-10\\
-30.35953125	-5.53175954542063e-10\\
-30.3382421875	-5.22167664065662e-10\\
-30.316953125	-6.00554149018106e-10\\
-30.2956640625	-5.84249480895313e-10\\
-30.274375	-6.27151805608494e-10\\
-30.2530859375	-6.42692037758111e-10\\
-30.231796875	-6.41356138946012e-10\\
-30.2105078125	-5.54808575715293e-10\\
-30.18921875	-5.16325407780356e-10\\
-30.1679296875	-3.791366751895e-10\\
-30.146640625	-3.56782315495228e-10\\
-30.1253515625	-2.24379052652865e-10\\
-30.1040625	-2.52481856038627e-10\\
-30.0827734375	-2.3076036279578e-10\\
-30.061484375	-2.70038046825937e-10\\
-30.0401953125	-2.78526059173824e-10\\
-30.01890625	-1.87041403150822e-10\\
-29.9976171875	-2.00433714828295e-10\\
-29.976328125	-5.03543570571243e-11\\
-29.9550390625	2.4441013194932e-11\\
-29.93375	3.80313826987994e-11\\
-29.9124609375	1.47544792088895e-10\\
-29.891171875	2.23578628879632e-10\\
-29.8698828125	1.84133693581295e-10\\
-29.84859375	9.97850347027712e-11\\
-29.8273046875	9.61096034670932e-12\\
-29.806015625	1.67643829422175e-11\\
-29.7847265625	7.18444043843777e-11\\
-29.7634375	3.31825495897203e-11\\
-29.7421484375	1.96717737346842e-11\\
-29.720859375	1.20257583785079e-10\\
-29.6995703125	1.64189999263164e-10\\
-29.67828125	1.63133240389433e-10\\
-29.6569921875	2.51794400669924e-10\\
-29.635703125	3.45495642723299e-10\\
-29.6144140625	2.74841976946654e-10\\
-29.593125	2.27442081999254e-10\\
-29.5718359375	2.60479250962366e-10\\
-29.550546875	2.6382668545328e-10\\
-29.5292578125	2.21459486047946e-10\\
-29.50796875	2.23023854628372e-10\\
-29.4866796875	2.21593780057049e-10\\
-29.465390625	1.7430935159719e-10\\
-29.4441015625	2.21671473570377e-10\\
-29.4228125	2.02095615746523e-10\\
-29.4015234375	3.25225317856209e-10\\
-29.380234375	2.32846369300223e-10\\
-29.3589453125	2.62955242642114e-10\\
-29.33765625	3.21703562352985e-10\\
-29.3163671875	2.79145031776869e-10\\
-29.295078125	2.65831405096278e-10\\
-29.2737890625	2.38103670408582e-10\\
-29.2525	2.74565003747688e-10\\
-29.2312109375	3.18454958841516e-10\\
-29.209921875	4.04939891510412e-10\\
-29.1886328125	4.54621176035542e-10\\
-29.16734375	5.52883337963705e-10\\
-29.1460546875	5.65314770459283e-10\\
-29.124765625	5.80039866113414e-10\\
-29.1034765625	6.14739592114888e-10\\
-29.0821875	6.65040219031613e-10\\
-29.0608984375	5.85037862593616e-10\\
-29.039609375	6.03853140679318e-10\\
-29.0183203125	6.54080848223535e-10\\
-28.99703125	5.08401480703915e-10\\
-28.9757421875	5.61391471193033e-10\\
-28.954453125	5.69241759779578e-10\\
-28.9331640625	5.6553262814885e-10\\
-28.911875	5.51732195528653e-10\\
-28.8905859375	5.487486007982e-10\\
-28.869296875	5.48396291762049e-10\\
-28.8480078125	4.98571514102005e-10\\
-28.82671875	4.97563490256106e-10\\
-28.8054296875	4.69713733237443e-10\\
-28.784140625	4.46161048223074e-10\\
-28.7628515625	4.74245704324781e-10\\
-28.7415625	5.85018144809166e-10\\
-28.7202734375	5.10696823774167e-10\\
-28.698984375	4.43943673968302e-10\\
-28.6776953125	5.7029087876427e-10\\
-28.65640625	5.37587543647133e-10\\
-28.6351171875	5.67731612041443e-10\\
-28.613828125	6.5448724386521e-10\\
-28.5925390625	6.10096808017276e-10\\
-28.57125	6.7682826517197e-10\\
-28.5499609375	6.04654943616986e-10\\
-28.528671875	5.99147221332227e-10\\
-28.5073828125	6.74918719475527e-10\\
-28.48609375	6.95468053953109e-10\\
-28.4648046875	7.04843844556746e-10\\
-28.443515625	7.30176627489857e-10\\
-28.4222265625	7.71127264990928e-10\\
-28.4009375	8.24933668200855e-10\\
-28.3796484375	8.15961796913893e-10\\
-28.358359375	8.12226454759251e-10\\
-28.3370703125	7.92526905139789e-10\\
-28.31578125	8.22374683765641e-10\\
-28.2944921875	6.89473236924689e-10\\
-28.273203125	7.16883811699035e-10\\
-28.2519140625	6.18923127553237e-10\\
-28.230625	6.86384383050551e-10\\
-28.2093359375	6.79447664149443e-10\\
-28.188046875	7.82576734357626e-10\\
-28.1667578125	7.77164762643359e-10\\
-28.14546875	8.19238071415612e-10\\
-28.1241796875	8.21559830560259e-10\\
-28.102890625	8.30188382119116e-10\\
-28.0816015625	8.04043981528008e-10\\
-28.0603125	7.41280236525234e-10\\
-28.0390234375	6.48476007980512e-10\\
-28.017734375	6.52620751193702e-10\\
-27.9964453125	5.05792611178719e-10\\
-27.97515625	5.59644251180256e-10\\
-27.9538671875	4.44000505032458e-10\\
-27.932578125	4.87407550590528e-10\\
-27.9112890625	5.20916565278925e-10\\
-27.89	4.98204836490491e-10\\
-27.8687109375	5.56593341677202e-10\\
-27.847421875	5.49284109274836e-10\\
-27.8261328125	5.30851777933224e-10\\
-27.80484375	4.29109439101385e-10\\
-27.7835546875	4.59916277055832e-10\\
-27.762265625	3.12219675842776e-10\\
-27.7409765625	4.05981682683559e-10\\
-27.7196875	1.9910377041188e-10\\
-27.6983984375	2.7187483445227e-10\\
-27.677109375	2.63795861684485e-10\\
-27.6558203125	3.11778465339907e-10\\
-27.63453125	3.39594396778421e-10\\
-27.6132421875	2.99726970658156e-10\\
-27.591953125	1.94652228896286e-10\\
-27.5706640625	5.52548628794195e-11\\
-27.549375	3.19977681985833e-11\\
-27.5280859375	-3.62830821104535e-11\\
-27.506796875	-1.23615848089432e-10\\
-27.4855078125	-9.45447948955552e-11\\
-27.46421875	-1.25665367088181e-10\\
-27.4429296875	-4.68141349027255e-11\\
-27.421640625	-3.14380305479298e-12\\
-27.4003515625	-4.9570374269645e-12\\
-27.3790625	1.69170038223275e-11\\
-27.3577734375	2.36091184719503e-11\\
-27.336484375	-9.55314089962218e-11\\
-27.3151953125	-2.06413640818435e-10\\
-27.29390625	-3.17921510615858e-10\\
-27.2726171875	-4.12592814276897e-10\\
-27.251328125	-3.78712667043154e-10\\
-27.2300390625	-4.41835175106488e-10\\
-27.20875	-5.15182261638711e-10\\
-27.1874609375	-4.28799681303997e-10\\
-27.166171875	-4.29247192581172e-10\\
-27.1448828125	-4.85850599496115e-10\\
-27.12359375	-3.74663024417795e-10\\
-27.1023046875	-3.46596269518453e-10\\
-27.081015625	-3.82769655194005e-10\\
-27.0597265625	-3.7104954942281e-10\\
-27.0384375	-4.52341821077162e-10\\
-27.0171484375	-5.22115827542541e-10\\
-26.995859375	-5.8722626586112e-10\\
-26.9745703125	-5.48087479556332e-10\\
-26.95328125	-6.02522504923067e-10\\
-26.9319921875	-5.21089135422913e-10\\
-26.910703125	-5.7398156344901e-10\\
-26.8894140625	-4.68123256997848e-10\\
-26.868125	-4.56378103926793e-10\\
-26.8468359375	-5.37529639780075e-10\\
-26.825546875	-5.48292452445963e-10\\
-26.8042578125	-5.2678998986298e-10\\
-26.78296875	-6.39232081120148e-10\\
-26.7616796875	-6.48316270621072e-10\\
-26.740390625	-6.85435465713656e-10\\
-26.7191015625	-7.10923621769176e-10\\
-26.6978125	-7.23764491439528e-10\\
-26.6765234375	-7.48751875117924e-10\\
-26.655234375	-7.45898451085489e-10\\
-26.6339453125	-7.00794296490695e-10\\
-26.61265625	-7.2482417826603e-10\\
-26.5913671875	-8.15895584802581e-10\\
-26.570078125	-7.8106210083419e-10\\
-26.5487890625	-8.57149674824846e-10\\
-26.5275	-9.51596456623716e-10\\
-26.5062109375	-9.26868201361277e-10\\
-26.484921875	-9.55191306620216e-10\\
-26.4636328125	-9.31464862651259e-10\\
-26.44234375	-8.70567957136617e-10\\
-26.4210546875	-8.04887748161205e-10\\
-26.399765625	-7.17302767488348e-10\\
-26.3784765625	-6.90738714156695e-10\\
-26.3571875	-6.99553085869718e-10\\
-26.3358984375	-6.89252790556224e-10\\
-26.314609375	-7.63430463029365e-10\\
-26.2933203125	-7.50705371333291e-10\\
-26.27203125	-7.43280522089023e-10\\
-26.2507421875	-8.41404980967403e-10\\
-26.229453125	-7.73149387536601e-10\\
-26.2081640625	-7.21767304725708e-10\\
-26.186875	-7.2824724607742e-10\\
-26.1655859375	-7.42236963669746e-10\\
-26.144296875	-6.6726845766659e-10\\
-26.1230078125	-6.84244917190188e-10\\
-26.10171875	-7.26581508194437e-10\\
-26.0804296875	-6.78487145652588e-10\\
-26.059140625	-7.10446585815346e-10\\
-26.0378515625	-7.73212488665039e-10\\
-26.0165625	-7.11748937013264e-10\\
-25.9952734375	-7.76949597518848e-10\\
-25.973984375	-7.83854841071486e-10\\
-25.9526953125	-6.75139292527902e-10\\
-25.93140625	-7.67226267909016e-10\\
-25.9101171875	-6.6200045508521e-10\\
-25.888828125	-6.74523128477611e-10\\
-25.8675390625	-8.03578942583699e-10\\
-25.84625	-8.19429675546244e-10\\
-25.8249609375	-8.1493064881244e-10\\
-25.803671875	-1.03535800860769e-09\\
-25.7823828125	-9.55253021172243e-10\\
-25.76109375	-9.54132613535754e-10\\
-25.7398046875	-8.90269975881944e-10\\
-25.718515625	-7.31004744592505e-10\\
-25.6972265625	-6.92616053926773e-10\\
-25.6759375	-6.18713269708741e-10\\
-25.6546484375	-5.74716749131549e-10\\
-25.633359375	-6.18606326011685e-10\\
-25.6120703125	-6.30497467073739e-10\\
-25.59078125	-6.90038858108349e-10\\
-25.5694921875	-7.56457097658574e-10\\
-25.548203125	-6.5330347718471e-10\\
-25.5269140625	-6.77898375354087e-10\\
-25.505625	-5.38788793289279e-10\\
-25.4843359375	-5.33359160227024e-10\\
-25.463046875	-3.85520358599426e-10\\
-25.4417578125	-4.42610059978516e-10\\
-25.42046875	-3.50499393223925e-10\\
-25.3991796875	-2.99217959652847e-10\\
-25.377890625	-4.31370984608619e-10\\
-25.3566015625	-3.37127845773284e-10\\
-25.3353125	-3.4683489214094e-10\\
-25.3140234375	-3.47531073342963e-10\\
-25.292734375	-2.89583833774485e-10\\
-25.2714453125	-2.19374389323949e-10\\
-25.25015625	-2.30468729977281e-10\\
-25.2288671875	-1.936811028754e-10\\
-25.207578125	-1.96364731480569e-10\\
-25.1862890625	-1.39979954384977e-10\\
-25.165	-2.73122145131829e-10\\
-25.1437109375	-1.99226107637729e-10\\
-25.122421875	-7.45912342750956e-11\\
-25.1011328125	-2.23497148889389e-12\\
-25.07984375	2.15946681266138e-12\\
-25.0585546875	1.8729636592838e-10\\
-25.037265625	1.55090762635876e-10\\
-25.0159765625	3.0419400672371e-10\\
-24.9946875	3.56597620600367e-10\\
-24.9733984375	4.1304302295945e-10\\
-24.952109375	3.5701184701493e-10\\
-24.9308203125	3.8618872694324e-10\\
-24.90953125	4.2868968111978e-10\\
-24.8882421875	4.08321077965241e-10\\
-24.866953125	4.48745495956015e-10\\
-24.8456640625	5.47531406604365e-10\\
-24.824375	5.57679507036518e-10\\
-24.8030859375	5.84871697835516e-10\\
-24.781796875	6.3751873375719e-10\\
-24.7605078125	7.2876342538193e-10\\
-24.73921875	8.10408008268917e-10\\
-24.7179296875	7.90715482549054e-10\\
-24.696640625	6.89039980363366e-10\\
-24.6753515625	8.75167802166288e-10\\
-24.6540625	8.4765805723543e-10\\
-24.6327734375	8.22558671522309e-10\\
-24.611484375	8.79142059449652e-10\\
-24.5901953125	1.00012593904379e-09\\
-24.56890625	1.0002523111013e-09\\
-24.5476171875	1.08695437428292e-09\\
-24.526328125	1.04273953843053e-09\\
-24.5050390625	1.07365321785732e-09\\
-24.48375	1.0285179135069e-09\\
-24.4624609375	9.54483312605447e-10\\
-24.441171875	1.02232891026871e-09\\
-24.4198828125	1.03324520560434e-09\\
-24.39859375	1.00141295621192e-09\\
-24.3773046875	1.16758330133629e-09\\
-24.356015625	1.2205393738784e-09\\
-24.3347265625	1.34637522873864e-09\\
-24.3134375	1.30957993756785e-09\\
-24.2921484375	1.34120784932486e-09\\
-24.270859375	1.29454346472362e-09\\
-24.2495703125	1.13718145496834e-09\\
-24.22828125	1.23707952208983e-09\\
-24.2069921875	1.14073668426621e-09\\
-24.185703125	1.20028421007336e-09\\
-24.1644140625	1.28010294001651e-09\\
-24.143125	1.30162418155658e-09\\
-24.1218359375	1.37267650399858e-09\\
-24.100546875	1.55903553497569e-09\\
-24.0792578125	1.56766617735029e-09\\
-24.05796875	1.54697967634351e-09\\
-24.0366796875	1.40084033698957e-09\\
-24.015390625	1.53650724743527e-09\\
-23.9941015625	1.39203449518782e-09\\
-23.9728125	1.4075852573863e-09\\
-23.9515234375	1.42245738562868e-09\\
-23.930234375	1.35081078419328e-09\\
-23.9089453125	1.36368569200544e-09\\
-23.88765625	1.4218902006683e-09\\
-23.8663671875	1.40600880500725e-09\\
-23.845078125	1.52886106741942e-09\\
-23.8237890625	1.42343868972546e-09\\
-23.8025	1.43734623443338e-09\\
-23.7812109375	1.38551639627771e-09\\
-23.759921875	1.36782525064713e-09\\
-23.7386328125	1.37069044889319e-09\\
-23.71734375	1.37492902011058e-09\\
-23.6960546875	1.42655144794183e-09\\
-23.674765625	1.35789340532705e-09\\
-23.6534765625	1.44033320336919e-09\\
-23.6321875	1.33055560183918e-09\\
-23.6108984375	1.36922806055126e-09\\
-23.589609375	1.38265280809897e-09\\
-23.5683203125	1.35162291933555e-09\\
-23.54703125	1.22849156343759e-09\\
-23.5257421875	1.39807890442721e-09\\
-23.504453125	1.20944933794513e-09\\
-23.4831640625	1.3252083392573e-09\\
-23.461875	1.26890109839733e-09\\
-23.4405859375	1.35137267314332e-09\\
-23.419296875	1.25460793039294e-09\\
-23.3980078125	1.28422415147233e-09\\
-23.37671875	1.26970314047687e-09\\
-23.3554296875	1.43313751150161e-09\\
-23.334140625	1.26601713406267e-09\\
-23.3128515625	1.22739860506334e-09\\
-23.2915625	1.24024368853625e-09\\
-23.2702734375	1.15411887778885e-09\\
-23.248984375	1.19976446108287e-09\\
-23.2276953125	1.07143753656135e-09\\
-23.20640625	1.11804092665073e-09\\
-23.1851171875	1.13321728373108e-09\\
-23.163828125	1.11794106978778e-09\\
-23.1425390625	1.05181198053514e-09\\
-23.12125	1.11624575874648e-09\\
-23.0999609375	1.04351158235172e-09\\
-23.078671875	1.00647760527963e-09\\
-23.0573828125	9.67822555547544e-10\\
-23.03609375	9.83220660484104e-10\\
-23.0148046875	1.09336420650327e-09\\
-22.993515625	1.15268284686659e-09\\
-22.9722265625	1.10723671260571e-09\\
-22.9509375	1.12017422594017e-09\\
-22.9296484375	1.08999883692989e-09\\
-22.908359375	9.18055598556422e-10\\
-22.8870703125	8.39052526319224e-10\\
-22.86578125	8.29520403890994e-10\\
-22.8444921875	6.48906026833189e-10\\
-22.823203125	6.48693337271658e-10\\
-22.8019140625	5.73026967847334e-10\\
-22.780625	6.84743577247663e-10\\
-22.7593359375	7.19038559494087e-10\\
-22.738046875	7.87948175738552e-10\\
-22.7167578125	8.41081180339203e-10\\
-22.69546875	8.33142087254935e-10\\
-22.6741796875	8.88175607501053e-10\\
-22.652890625	7.19937691228677e-10\\
-22.6316015625	6.50774838345122e-10\\
-22.6103125	4.24526716838152e-10\\
-22.5890234375	4.77207079385203e-10\\
-22.567734375	3.23470699475681e-10\\
-22.5464453125	3.81784289891123e-10\\
-22.52515625	3.36178084901264e-10\\
-22.5038671875	4.65912809096047e-10\\
-22.482578125	3.30805233922289e-10\\
-22.4612890625	4.32247730656248e-10\\
-22.44	4.24788313241919e-10\\
-22.4187109375	2.38321059588911e-10\\
-22.397421875	1.0301413956616e-10\\
-22.3761328125	1.22773424970085e-10\\
-22.35484375	6.60073693617234e-11\\
-22.3335546875	1.22198950582839e-10\\
-22.312265625	1.69375371461964e-10\\
-22.2909765625	2.58881790578869e-10\\
-22.2696875	2.31190916462567e-10\\
-22.2483984375	2.81572831846813e-10\\
-22.227109375	1.09775177302808e-10\\
-22.2058203125	1.11873701475952e-10\\
-22.18453125	-4.73104247186953e-12\\
-22.1632421875	-2.67775249254312e-10\\
-22.141953125	-1.6314453093096e-10\\
-22.1206640625	-3.87708727849853e-10\\
-22.099375	-3.15428769829432e-10\\
-22.0780859375	-2.4443314183724e-10\\
-22.056796875	-2.41870421317012e-10\\
-22.0355078125	-3.22853228398903e-10\\
-22.01421875	-4.27627143252783e-10\\
-21.9929296875	-3.75798694178148e-10\\
-21.971640625	-4.38732571081659e-10\\
-21.9503515625	-4.10413942096102e-10\\
-21.9290625	-4.94317122723223e-10\\
-21.9077734375	-4.31132831856281e-10\\
-21.886484375	-4.69241797771434e-10\\
-21.8651953125	-4.49262011308668e-10\\
-21.84390625	-5.506302176235e-10\\
-21.8226171875	-6.08245213903969e-10\\
-21.801328125	-7.21122724767064e-10\\
-21.7800390625	-6.99434351648021e-10\\
-21.75875	-7.99987005095637e-10\\
-21.7374609375	-9.49134700688783e-10\\
-21.716171875	-9.03281612762109e-10\\
-21.6948828125	-8.26190435094112e-10\\
-21.67359375	-8.76642940924101e-10\\
-21.6523046875	-8.5377617765796e-10\\
-21.631015625	-9.05309143643614e-10\\
-21.6097265625	-1.01652799483516e-09\\
-21.5884375	-1.12693714679833e-09\\
-21.5671484375	-1.23497383514392e-09\\
-21.545859375	-1.25846976956428e-09\\
-21.5245703125	-1.27908163677017e-09\\
-21.50328125	-1.14137008522571e-09\\
-21.4819921875	-1.14094503852134e-09\\
-21.460703125	-1.00389124894298e-09\\
-21.4394140625	-9.99264232672367e-10\\
-21.418125	-8.41568748721231e-10\\
-21.3968359375	-9.87126703718512e-10\\
-21.375546875	-9.69780266040325e-10\\
-21.3542578125	-1.17528286361232e-09\\
-21.33296875	-1.08220220832964e-09\\
-21.3116796875	-1.21168529987796e-09\\
-21.290390625	-1.10181583134474e-09\\
-21.2691015625	-1.02047904115689e-09\\
-21.2478125	-1.11813839468648e-09\\
-21.2265234375	-1.02704198882691e-09\\
-21.205234375	-1.08092311412729e-09\\
-21.1839453125	-1.07903682712856e-09\\
-21.16265625	-1.14511344879847e-09\\
-21.1413671875	-1.09527934237019e-09\\
-21.120078125	-1.12198389121177e-09\\
-21.0987890625	-1.08117036293825e-09\\
-21.0775	-1.04998273210603e-09\\
-21.0562109375	-9.27963676060451e-10\\
-21.034921875	-1.02401110268365e-09\\
-21.0136328125	-8.9816904120056e-10\\
-20.99234375	-9.50443087929162e-10\\
-20.9710546875	-8.65377795583169e-10\\
-20.949765625	-1.01809885951897e-09\\
-20.9284765625	-9.53640923707182e-10\\
-20.9071875	-8.3589795311302e-10\\
-20.8858984375	-9.8918418969387e-10\\
-20.864609375	-8.43056656073658e-10\\
-20.8433203125	-6.5931036829975e-10\\
-20.82203125	-5.45154794454733e-10\\
-20.8007421875	-4.84471011166415e-10\\
-20.779453125	-4.04341872033987e-10\\
-20.7581640625	-4.85264215546869e-10\\
-20.736875	-4.95652355573017e-10\\
-20.7155859375	-6.06145984598104e-10\\
-20.694296875	-6.36628500501159e-10\\
-20.6730078125	-6.08286475791152e-10\\
-20.65171875	-7.74238384651558e-10\\
-20.6304296875	-6.10984472842359e-10\\
-20.609140625	-4.41185763675816e-10\\
-20.5878515625	-5.16233257686894e-10\\
-20.5665625	-4.48502947024159e-10\\
-20.5452734375	-3.15488872632178e-10\\
-20.523984375	-3.0250384374475e-10\\
-20.5026953125	-4.93329500398324e-10\\
-20.48140625	-5.12887056213036e-10\\
-20.4601171875	-4.16200182725748e-10\\
-20.438828125	-5.43776149688433e-10\\
-20.4175390625	-5.03386954315621e-10\\
-20.39625	-3.99874681138889e-10\\
-20.3749609375	-2.21771968956978e-10\\
-20.353671875	-2.2169121038896e-10\\
-20.3323828125	-3.08987465000791e-10\\
-20.31109375	-2.09288960919927e-10\\
-20.2898046875	-2.85000628622531e-10\\
-20.268515625	-3.2968164330283e-10\\
-20.2472265625	-4.06289999462273e-10\\
-20.2259375	-4.35183669317586e-10\\
-20.2046484375	-4.74118670850435e-10\\
-20.183359375	-4.84129586172275e-10\\
-20.1620703125	-3.42604927717085e-10\\
-20.14078125	-1.97794739786558e-10\\
-20.1194921875	-1.75851857955908e-10\\
-20.098203125	-8.5152064025581e-11\\
-20.0769140625	-6.55147385638179e-11\\
-20.055625	-7.66902172932331e-11\\
-20.0343359375	-8.50401657502163e-11\\
-20.013046875	-2.65968303985737e-11\\
-19.9917578125	-4.01262404068826e-11\\
-19.97046875	1.17347782960791e-10\\
-19.9491796875	1.76626551108482e-10\\
-19.927890625	2.38597351420417e-10\\
-19.9066015625	3.70131202814701e-10\\
-19.8853125	3.57545998117562e-10\\
-19.8640234375	3.94649504756749e-10\\
-19.842734375	5.00823465378893e-10\\
-19.8214453125	3.77441597589263e-10\\
-19.80015625	5.27881516298413e-10\\
-19.7788671875	5.95518252012355e-10\\
-19.757578125	7.17659513634207e-10\\
-19.7362890625	9.07498044644677e-10\\
-19.715	8.66140668228884e-10\\
-19.6937109375	9.88554236739195e-10\\
-19.672421875	1.12793103670275e-09\\
-19.6511328125	1.12619346663864e-09\\
-19.62984375	1.19639703643392e-09\\
-19.6085546875	1.15402136837386e-09\\
-19.587265625	9.24424554476344e-10\\
-19.5659765625	8.96772197060194e-10\\
-19.5446875	8.4130366516048e-10\\
-19.5233984375	9.06139893917818e-10\\
-19.502109375	8.34113819252891e-10\\
-19.4808203125	9.68759773052095e-10\\
-19.45953125	1.12922321350965e-09\\
-19.4382421875	1.13812802849383e-09\\
-19.416953125	1.37583480046493e-09\\
-19.3956640625	1.44717659309542e-09\\
-19.374375	1.28594349846988e-09\\
-19.3530859375	1.14498697643925e-09\\
-19.331796875	1.09861561743904e-09\\
-19.3105078125	8.93031360296568e-10\\
-19.28921875	8.93424230796814e-10\\
-19.2679296875	7.80709079601974e-10\\
-19.246640625	7.94154991907105e-10\\
-19.2253515625	8.62107441402717e-10\\
-19.2040625	8.97737692619986e-10\\
-19.1827734375	1.03884383666991e-09\\
-19.161484375	1.15743356345499e-09\\
-19.1401953125	1.13555785850539e-09\\
-19.11890625	1.13099068966313e-09\\
-19.0976171875	1.0505148148385e-09\\
-19.076328125	1.11830353281781e-09\\
-19.0550390625	1.02496812023775e-09\\
-19.03375	1.01419860762898e-09\\
-19.0124609375	1.06407986815597e-09\\
-18.991171875	1.22432093182204e-09\\
-18.9698828125	1.256424569215e-09\\
-18.94859375	1.31576118839566e-09\\
-18.9273046875	1.30547376886111e-09\\
-18.906015625	1.35685402089296e-09\\
-18.8847265625	1.25610571044634e-09\\
-18.8634375	1.26197137018769e-09\\
-18.8421484375	1.10247750340568e-09\\
-18.820859375	1.19514561831256e-09\\
-18.7995703125	1.13608004666047e-09\\
-18.77828125	1.09919306615847e-09\\
-18.7569921875	1.26962395694028e-09\\
-18.735703125	1.21757302354824e-09\\
-18.7144140625	1.26828172356083e-09\\
-18.693125	1.155203003244e-09\\
-18.6718359375	1.17000769019596e-09\\
-18.650546875	1.18685582363255e-09\\
-18.6292578125	1.27456050241161e-09\\
-18.60796875	1.1734937627219e-09\\
-18.5866796875	1.20007655569852e-09\\
-18.565390625	1.18488179517385e-09\\
-18.5441015625	1.22907204115387e-09\\
-18.5228125	1.26946476293801e-09\\
-18.5015234375	1.12817909750561e-09\\
-18.480234375	1.18991677000742e-09\\
-18.4589453125	1.10119097028919e-09\\
-18.43765625	1.14512985779676e-09\\
-18.4163671875	1.07784384366881e-09\\
-18.395078125	1.07671853186237e-09\\
-18.3737890625	1.136956840207e-09\\
-18.3525	1.06810759051112e-09\\
-18.3312109375	1.02770831451263e-09\\
-18.309921875	1.01102644172407e-09\\
-18.2886328125	1.0583492617725e-09\\
-18.26734375	1.00927905637213e-09\\
-18.2460546875	1.07689962334702e-09\\
-18.224765625	9.75172079464575e-10\\
-18.2034765625	9.9995597023414e-10\\
-18.1821875	9.66360690405741e-10\\
-18.1608984375	8.21702269127002e-10\\
-18.139609375	7.86624206354893e-10\\
-18.1183203125	6.14884324073501e-10\\
-18.09703125	6.65196336278435e-10\\
-18.0757421875	7.5306507810574e-10\\
-18.054453125	6.85756662647055e-10\\
-18.0331640625	7.80148398718174e-10\\
-18.011875	8.54914725353053e-10\\
-17.9905859375	8.5378256796307e-10\\
-17.969296875	6.76874848626359e-10\\
-17.9480078125	7.38968249955268e-10\\
-17.92671875	7.33137028923205e-10\\
-17.9054296875	5.60361605995922e-10\\
-17.884140625	5.51941475474223e-10\\
-17.8628515625	6.11900999628971e-10\\
-17.8415625	5.71599004307824e-10\\
-17.8202734375	4.31616076537509e-10\\
-17.798984375	3.64577726523048e-10\\
-17.7776953125	4.01920129438461e-10\\
-17.75640625	4.76208310695221e-10\\
-17.7351171875	3.18897730068842e-10\\
-17.713828125	4.65877292442258e-10\\
-17.6925390625	4.88392046446434e-10\\
-17.67125	4.2350616449733e-10\\
-17.6499609375	3.16899399632964e-10\\
-17.628671875	3.45094976108864e-10\\
-17.6073828125	1.92492855777745e-10\\
-17.58609375	2.26249405002933e-10\\
-17.5648046875	8.61624397539064e-11\\
-17.543515625	1.13613467520214e-10\\
-17.5222265625	1.39621112584669e-10\\
-17.5009375	-4.62600496611685e-11\\
-17.4796484375	9.24426947997576e-11\\
-17.458359375	-4.71915795093269e-12\\
-17.4370703125	-2.1712225642894e-11\\
-17.41578125	-7.8100127835296e-11\\
-17.3944921875	-1.49732235012682e-10\\
-17.373203125	-1.34955717433735e-10\\
-17.3519140625	-2.70558646047888e-10\\
-17.330625	-1.94599307051605e-10\\
-17.3093359375	-1.6068105444083e-10\\
-17.288046875	-2.73848816623532e-10\\
-17.2667578125	-2.26553653995687e-10\\
-17.24546875	-3.08101073840917e-10\\
-17.2241796875	-4.09858098979746e-10\\
-17.202890625	-3.20116894985807e-10\\
-17.1816015625	-6.21252455631723e-10\\
-17.1603125	-6.62033365904177e-10\\
-17.1390234375	-5.90054206789492e-10\\
-17.117734375	-7.21341019018333e-10\\
-17.0964453125	-5.74374620979582e-10\\
-17.07515625	-6.27947057462263e-10\\
-17.0538671875	-7.08803131407991e-10\\
-17.032578125	-6.64550099678506e-10\\
-17.0112890625	-7.05043546655117e-10\\
-16.99	-8.24545199061267e-10\\
-16.9687109375	-7.69086687938539e-10\\
-16.947421875	-7.09947303300512e-10\\
-16.9261328125	-7.58741843923539e-10\\
-16.90484375	-6.31113376833463e-10\\
-16.8835546875	-6.64725140704422e-10\\
-16.862265625	-7.53937125052251e-10\\
-16.8409765625	-7.93907635541823e-10\\
-16.8196875	-9.08349619773842e-10\\
-16.7983984375	-1.00570338982151e-09\\
-16.777109375	-9.88523180244e-10\\
-16.7558203125	-9.78006591822898e-10\\
-16.73453125	-8.63441556217616e-10\\
-16.7132421875	-8.15607940336142e-10\\
-16.691953125	-7.25049096039686e-10\\
-16.6706640625	-6.20610948841514e-10\\
-16.649375	-6.62620696664023e-10\\
-16.6280859375	-7.43135916963132e-10\\
-16.606796875	-7.96140519153506e-10\\
-16.5855078125	-9.93523991844166e-10\\
-16.56421875	-1.15199556150223e-09\\
-16.5429296875	-1.11628796758075e-09\\
-16.521640625	-1.20777889489802e-09\\
-16.5003515625	-1.24140663287102e-09\\
-16.4790625	-1.05038731830425e-09\\
-16.4577734375	-1.02569223843905e-09\\
-16.436484375	-9.3165682346532e-10\\
-16.4151953125	-1.07997869400523e-09\\
-16.39390625	-9.92849620909745e-10\\
-16.3726171875	-1.03348088181222e-09\\
-16.351328125	-1.09830112537293e-09\\
-16.3300390625	-1.00457630470038e-09\\
-16.30875	-1.09031491977196e-09\\
-16.2874609375	-9.98331240053573e-10\\
-16.266171875	-9.67546764425019e-10\\
-16.2448828125	-8.6011157210631e-10\\
-16.22359375	-7.2555440132022e-10\\
-16.2023046875	-8.42795421978418e-10\\
-16.181015625	-8.05943168748997e-10\\
-16.1597265625	-9.31611008253309e-10\\
-16.1384375	-1.11008866462111e-09\\
-16.1171484375	-1.03189081608977e-09\\
-16.095859375	-1.06489989201871e-09\\
-16.0745703125	-1.03062655376877e-09\\
-16.05328125	-9.29495031506778e-10\\
-16.0319921875	-8.46064116931516e-10\\
-16.010703125	-8.9968736950489e-10\\
-15.9894140625	-7.73994802896273e-10\\
-15.968125	-6.90513105519616e-10\\
-15.9468359375	-8.92534875250192e-10\\
-15.925546875	-6.84165318062604e-10\\
-15.9042578125	-8.18212248345379e-10\\
-15.88296875	-8.87281267023644e-10\\
-15.8616796875	-8.69865027008192e-10\\
-15.840390625	-9.18047357400493e-10\\
-15.8191015625	-1.02083978155338e-09\\
-15.7978125	-8.81117809693582e-10\\
-15.7765234375	-1.03766003990307e-09\\
-15.755234375	-1.00530938614318e-09\\
-15.7339453125	-8.74692672168992e-10\\
-15.71265625	-8.09936541427914e-10\\
-15.6913671875	-6.90374287003933e-10\\
-15.670078125	-7.18548081855557e-10\\
-15.6487890625	-6.94405289294145e-10\\
-15.6275	-7.59074565959903e-10\\
-15.6062109375	-7.19151917758902e-10\\
-15.584921875	-6.58796092004684e-10\\
-15.5636328125	-5.18282459134438e-10\\
-15.54234375	-4.4342023145517e-10\\
-15.5210546875	-5.05363492883457e-10\\
-15.499765625	-3.24070557222207e-10\\
-15.4784765625	-1.98927632225897e-10\\
-15.4571875	-3.39203914587697e-10\\
-15.4358984375	-2.86198079141935e-10\\
-15.414609375	-2.44556707139186e-10\\
-15.3933203125	-3.22383112318095e-10\\
-15.37203125	-3.90030377028999e-10\\
-15.3507421875	-3.5144674446603e-10\\
-15.329453125	-2.50718546040072e-10\\
-15.3081640625	-2.82249414761719e-10\\
-15.286875	-2.39131590373897e-10\\
-15.2655859375	-1.17803730443798e-10\\
-15.244296875	-2.46282150862799e-11\\
-15.2230078125	6.27438439950778e-11\\
-15.20171875	7.32211024126333e-11\\
-15.1804296875	-3.37680031696976e-11\\
-15.159140625	-3.69963180340728e-11\\
-15.1378515625	-4.76365462438149e-11\\
-15.1165625	5.44992508401399e-11\\
-15.0952734375	-1.83531776980111e-11\\
-15.073984375	6.7982186370464e-13\\
-15.0526953125	1.54417359507124e-11\\
-15.03140625	1.15972683771996e-10\\
-15.0101171875	1.78569776302367e-10\\
-14.988828125	2.53854782309759e-10\\
-14.9675390625	3.08585562394201e-10\\
-14.94625	3.82321182733255e-10\\
-14.9249609375	2.88397069843786e-10\\
-14.903671875	3.43194114173446e-10\\
-14.8823828125	3.89786785985894e-10\\
-14.86109375	2.93119351484917e-10\\
-14.8398046875	3.71821653779985e-10\\
-14.818515625	3.75759819940744e-10\\
-14.7972265625	4.776104531255e-10\\
-14.7759375	4.50191836087803e-10\\
-14.7546484375	4.81884640640213e-10\\
-14.733359375	6.91229004777026e-10\\
-14.7120703125	7.43100424486935e-10\\
-14.69078125	7.2934693247041e-10\\
-14.6694921875	9.46793624463913e-10\\
-14.648203125	8.48506056091759e-10\\
-14.6269140625	8.36489375713648e-10\\
-14.605625	9.47560721205439e-10\\
-14.5843359375	8.87283608385303e-10\\
-14.563046875	1.06067948077916e-09\\
-14.5417578125	1.01962741248763e-09\\
-14.52046875	1.14274755902225e-09\\
-14.4991796875	1.36344669889242e-09\\
-14.477890625	1.37212028484498e-09\\
-14.4566015625	1.42165288711145e-09\\
-14.4353125	1.58049624782366e-09\\
-14.4140234375	1.5026963071989e-09\\
-14.392734375	1.33370009019519e-09\\
-14.3714453125	1.37922880119764e-09\\
-14.35015625	1.16205248571704e-09\\
-14.3288671875	1.14678173467862e-09\\
-14.307578125	1.19957186415923e-09\\
-14.2862890625	1.35751536675237e-09\\
-14.265	1.47197431134539e-09\\
-14.2437109375	1.70193265395901e-09\\
-14.222421875	1.68470823979324e-09\\
-14.2011328125	1.83078198539295e-09\\
-14.17984375	1.5787549993819e-09\\
-14.1585546875	1.48680175037114e-09\\
-14.137265625	1.41843267657348e-09\\
-14.1159765625	1.20805110680382e-09\\
-14.0946875	1.13735459134577e-09\\
-14.0733984375	1.23272998263755e-09\\
-14.052109375	1.30900344384589e-09\\
-14.0308203125	1.4620885856614e-09\\
-14.00953125	1.66977203353294e-09\\
-13.9882421875	1.67193798997112e-09\\
-13.966953125	1.75307747829342e-09\\
-13.9456640625	1.67525533178241e-09\\
-13.924375	1.56248546848172e-09\\
-13.9030859375	1.38120260020013e-09\\
-13.881796875	1.28990130604073e-09\\
-13.8605078125	1.16563670770707e-09\\
-13.83921875	1.19566080711727e-09\\
-13.8179296875	1.23135600309899e-09\\
-13.796640625	1.4144857949093e-09\\
-13.7753515625	1.33600541617027e-09\\
-13.7540625	1.59192898325535e-09\\
-13.7327734375	1.48111412598202e-09\\
-13.711484375	1.54035728757813e-09\\
-13.6901953125	1.39119378135378e-09\\
-13.66890625	1.26129468032442e-09\\
-13.6476171875	1.2714417953121e-09\\
-13.626328125	1.16483990557789e-09\\
-13.6050390625	1.15827502989061e-09\\
-13.58375	1.28102402075493e-09\\
-13.5624609375	1.28691403451168e-09\\
-13.541171875	1.2955789441539e-09\\
-13.5198828125	1.45207709859569e-09\\
-13.49859375	1.37777250249996e-09\\
-13.4773046875	1.44887387166489e-09\\
-13.456015625	1.31134718578782e-09\\
-13.4347265625	1.19497204763994e-09\\
-13.4134375	1.07351570851956e-09\\
-13.3921484375	8.51608078400414e-10\\
-13.370859375	9.07398500513955e-10\\
-13.3495703125	9.95222500099066e-10\\
-13.32828125	1.0112585582675e-09\\
-13.3069921875	1.08235230918922e-09\\
-13.285703125	1.17057722844068e-09\\
-13.2644140625	1.2945004377358e-09\\
-13.243125	1.11528288368325e-09\\
-13.2218359375	1.33606195978343e-09\\
-13.200546875	1.18698391896653e-09\\
-13.1792578125	1.10277826103217e-09\\
-13.15796875	9.64534122767796e-10\\
-13.1366796875	9.08796369655223e-10\\
-13.115390625	9.00037076043752e-10\\
-13.0941015625	9.86824829490498e-10\\
-13.0728125	8.9706342616689e-10\\
-13.0515234375	1.12613731523969e-09\\
-13.030234375	1.10832942116706e-09\\
-13.0089453125	9.86096140476273e-10\\
-12.98765625	1.05583197960552e-09\\
-12.9663671875	1.03853741854433e-09\\
-12.945078125	8.83762590303921e-10\\
-12.9237890625	7.40954789080577e-10\\
-12.9025	8.12208297370693e-10\\
-12.8812109375	7.89416720203158e-10\\
-12.859921875	8.7328401722445e-10\\
-12.8386328125	8.30793274142805e-10\\
-12.81734375	9.74483695280605e-10\\
-12.7960546875	9.59544079509033e-10\\
-12.774765625	8.41510169172866e-10\\
-12.7534765625	8.24452119737266e-10\\
-12.7321875	8.0655203994265e-10\\
-12.7108984375	6.7102488946513e-10\\
-12.689609375	6.3099938317085e-10\\
-12.6683203125	4.85455434168292e-10\\
-12.64703125	4.70011448542878e-10\\
-12.6257421875	4.12849518917476e-10\\
-12.604453125	5.27224665118758e-10\\
-12.5831640625	4.62952075776223e-10\\
-12.561875	4.94301521670001e-10\\
-12.5405859375	5.19164583047934e-10\\
-12.519296875	4.52047806598449e-10\\
-12.4980078125	5.32919631844328e-10\\
-12.47671875	3.89168760839471e-10\\
-12.4554296875	3.51015237531013e-10\\
-12.434140625	2.57968677155837e-10\\
-12.4128515625	2.4756014951086e-10\\
-12.3915625	2.32471059574373e-10\\
-12.3702734375	2.37440424700985e-10\\
-12.348984375	3.05460713177752e-10\\
-12.3276953125	3.56406513578336e-10\\
-12.30640625	3.48637411795885e-10\\
-12.2851171875	3.96587475942485e-10\\
-12.263828125	2.72746632879916e-10\\
-12.2425390625	2.61118649955209e-10\\
-12.22125	5.741673046862e-11\\
-12.1999609375	8.09910725282351e-11\\
-12.178671875	-9.50730609928157e-11\\
-12.1573828125	-1.26696028161017e-10\\
-12.13609375	-1.67109935336027e-10\\
-12.1148046875	-2.32137004981725e-10\\
-12.093515625	-1.1672799063861e-10\\
-12.0722265625	-1.91592437503679e-10\\
-12.0509375	-2.68915411983655e-10\\
-12.0296484375	-9.61633241658067e-11\\
-12.008359375	-2.80514293668827e-10\\
-11.9870703125	-3.69659702142593e-10\\
-11.96578125	-4.49777163317818e-10\\
-11.9444921875	-5.96739639485934e-10\\
-11.923203125	-7.1493904751635e-10\\
-11.9019140625	-6.69573272837122e-10\\
-11.880625	-5.90828820344725e-10\\
-11.8593359375	-5.40464899972549e-10\\
-11.838046875	-4.6746961655852e-10\\
-11.8167578125	-4.66537165634362e-10\\
-11.79546875	-3.83850525823727e-10\\
-11.7741796875	-4.8360203965982e-10\\
-11.752890625	-4.77855648841034e-10\\
-11.7316015625	-5.68400177240775e-10\\
-11.7103125	-6.54371931241234e-10\\
-11.6890234375	-8.39736333038219e-10\\
-11.667734375	-7.79615067308656e-10\\
-11.6464453125	-8.53422007529113e-10\\
-11.62515625	-9.51538709675395e-10\\
-11.6038671875	-9.4663860652613e-10\\
-11.582578125	-8.56136135593046e-10\\
-11.5612890625	-8.65587487777355e-10\\
-11.54	-9.22045012297048e-10\\
-11.5187109375	-1.01477277618544e-09\\
-11.497421875	-1.06896827213918e-09\\
-11.4761328125	-1.01544663098236e-09\\
-11.45484375	-1.20230996447004e-09\\
-11.4335546875	-1.19994092317984e-09\\
-11.412265625	-1.12998700782688e-09\\
-11.3909765625	-1.09350258513735e-09\\
-11.3696875	-1.02287089890036e-09\\
-11.3483984375	-1.04861679132693e-09\\
-11.327109375	-9.52892299033263e-10\\
-11.3058203125	-1.04436366378305e-09\\
-11.28453125	-9.06562694147563e-10\\
-11.2632421875	-9.02263080860174e-10\\
-11.241953125	-1.03226945019371e-09\\
-11.2206640625	-9.91904822578971e-10\\
-11.199375	-9.05802239734837e-10\\
-11.1780859375	-1.00140893286168e-09\\
-11.156796875	-1.07569670544117e-09\\
-11.1355078125	-1.03304429063179e-09\\
-11.11421875	-1.13219987629287e-09\\
-11.0929296875	-1.06636563224184e-09\\
-11.071640625	-9.67368516642697e-10\\
-11.0503515625	-9.01328038957147e-10\\
-11.0290625	-9.60461784450214e-10\\
-11.0077734375	-9.62462844132459e-10\\
-10.986484375	-1.01543119590533e-09\\
-10.9651953125	-1.03978928177669e-09\\
-10.94390625	-1.11301911037112e-09\\
-10.9226171875	-1.09068072259368e-09\\
-10.901328125	-1.15358576189008e-09\\
-10.8800390625	-9.99608714322335e-10\\
-10.85875	-1.10367376669179e-09\\
-10.8374609375	-8.77877467117335e-10\\
-10.816171875	-8.00367146301556e-10\\
-10.7948828125	-8.43511529269541e-10\\
-10.77359375	-8.85954761876472e-10\\
-10.7523046875	-9.21205450637813e-10\\
-10.731015625	-8.77648081829193e-10\\
-10.7097265625	-1.02200176731854e-09\\
-10.6884375	-7.96382313984051e-10\\
-10.6671484375	-8.34706681508176e-10\\
-10.645859375	-7.87011849110423e-10\\
-10.6245703125	-6.84156954915074e-10\\
-10.60328125	-7.1871094895053e-10\\
-10.5819921875	-7.53534749780369e-10\\
-10.560703125	-7.64308475501924e-10\\
-10.5394140625	-8.55583068331321e-10\\
-10.518125	-8.04714699715184e-10\\
-10.4968359375	-6.9560125886547e-10\\
-10.475546875	-7.1194393201808e-10\\
-10.4542578125	-6.54370275644016e-10\\
-10.43296875	-5.428033643463e-10\\
-10.4116796875	-5.50211483274938e-10\\
-10.390390625	-5.38335198332133e-10\\
-10.3691015625	-5.87934321084137e-10\\
-10.3478125	-5.29845302244804e-10\\
-10.3265234375	-6.15619581635233e-10\\
-10.305234375	-6.98095238837581e-10\\
-10.2839453125	-6.81348937718531e-10\\
-10.26265625	-5.5595016573522e-10\\
-10.2413671875	-4.7891854881939e-10\\
-10.220078125	-4.65411264005755e-10\\
-10.1987890625	-3.47213973138655e-10\\
-10.1775	-3.16212804739516e-10\\
-10.1562109375	-3.0789615488505e-10\\
-10.134921875	-3.08257988019219e-10\\
-10.1136328125	-2.81577199175265e-10\\
-10.09234375	-2.95710018710484e-10\\
-10.0710546875	-2.89324952423044e-10\\
-10.049765625	-2.94020761112213e-10\\
-10.0284765625	-1.95134338461537e-10\\
-10.0071875	-2.28587580729578e-10\\
-9.9858984375	-1.1926337665626e-10\\
-9.96460937499999	-4.70807468775207e-11\\
-9.9433203125	-1.26605887118462e-10\\
-9.92203125	2.18829502248927e-11\\
-9.9007421875	-1.01281268456872e-10\\
-9.879453125	-2.67149579854113e-11\\
-9.85816406249999	-1.61321050021088e-10\\
-9.836875	-1.42239832449289e-10\\
-9.8155859375	-5.55261723334834e-11\\
-9.794296875	-3.53944154464455e-11\\
-9.7730078125	5.90877857192865e-12\\
-9.75171874999999	6.92718362686142e-11\\
-9.7304296875	7.96759553602849e-11\\
-9.709140625	1.70031927953078e-10\\
-9.6878515625	2.20031059820128e-10\\
-9.6665625	2.09843445314669e-10\\
-9.64527343749999	2.26540117493313e-10\\
-9.623984375	2.99753354646696e-10\\
-9.6026953125	2.50952620062239e-10\\
-9.58140625	2.48581339152596e-10\\
-9.5601171875	3.77864418154415e-10\\
-9.53882812499999	3.83514788287043e-10\\
-9.5175390625	3.41477849912799e-10\\
-9.49625	5.19608071808215e-10\\
-9.4749609375	5.96462873491252e-10\\
-9.453671875	7.0772205676475e-10\\
-9.43238281249999	7.34004095132259e-10\\
-9.41109375	8.21124255475717e-10\\
-9.3898046875	7.01934548915504e-10\\
-9.368515625	8.58415522758191e-10\\
-9.3472265625	6.5467967041948e-10\\
-9.32593749999999	7.14407927145894e-10\\
-9.3046484375	7.63525032737842e-10\\
-9.283359375	7.22517134652851e-10\\
-9.2620703125	8.96676894119388e-10\\
-9.24078125	8.62068682874102e-10\\
-9.21949218749999	1.0056960222749e-09\\
-9.198203125	9.86892158429011e-10\\
-9.1769140625	9.73187157058336e-10\\
-9.155625	9.46473987281568e-10\\
-9.1343359375	1.01770586219465e-09\\
-9.11304687499999	9.96957282004635e-10\\
-9.0917578125	1.01022068233527e-09\\
-9.07046875	9.59197097061746e-10\\
-9.0491796875	9.98660120463674e-10\\
-9.027890625	1.09172052783649e-09\\
-9.00660156249999	1.1654109773781e-09\\
-8.9853125	1.1743268312076e-09\\
-8.9640234375	1.33215272062388e-09\\
-8.942734375	1.3665522703259e-09\\
-8.9214453125	1.30144420818834e-09\\
-8.90015624999999	1.22308098802572e-09\\
-8.8788671875	1.28306851846489e-09\\
-8.857578125	1.09381503406579e-09\\
-8.8362890625	1.04020383331581e-09\\
-8.815	1.06907066704502e-09\\
-8.79371093749999	9.9211003832458e-10\\
-8.772421875	1.06700084132558e-09\\
-8.7511328125	1.1225426382068e-09\\
-8.72984375	1.3428300593585e-09\\
-8.7085546875	1.3455945739815e-09\\
-8.68726562499999	1.37430230297747e-09\\
-8.6659765625	1.45422032160039e-09\\
-8.6446875	1.42913438195899e-09\\
-8.6233984375	1.23053775142473e-09\\
-8.602109375	1.31837094253214e-09\\
-8.58082031249999	1.08026429791281e-09\\
-8.55953125	1.07170314529464e-09\\
-8.5382421875	9.83966725321444e-10\\
-8.516953125	1.10732014744055e-09\\
-8.4956640625	1.04264762639727e-09\\
-8.47437499999999	1.09966290182849e-09\\
-8.4530859375	1.23794421104506e-09\\
-8.431796875	1.24379867393163e-09\\
-8.4105078125	1.2702385491426e-09\\
-8.38921875	1.20648049394507e-09\\
-8.36792968749999	1.13697225364792e-09\\
-8.346640625	9.58407150296387e-10\\
-8.3253515625	8.82232230547712e-10\\
-8.3040625	9.09930433421869e-10\\
-8.2827734375	1.03727437932946e-09\\
-8.26148437499999	1.06706898694188e-09\\
-8.2401953125	1.04046937550916e-09\\
-8.21890625	1.07995703954296e-09\\
-8.1976171875	1.13643605458584e-09\\
-8.176328125	9.03116668666763e-10\\
-8.15503906249999	1.04193881496796e-09\\
-8.13375	8.95682713393384e-10\\
-8.1124609375	7.4529981224066e-10\\
-8.091171875	7.08618176242934e-10\\
-8.0698828125	6.33276392485935e-10\\
-8.04859374999999	6.07683098456268e-10\\
-8.0273046875	6.92893009154772e-10\\
-8.006015625	7.52490209271816e-10\\
-7.9847265625	6.42352744152015e-10\\
-7.9634375	8.298382168738e-10\\
-7.94214843749999	7.74530178303587e-10\\
-7.920859375	7.26234354127325e-10\\
-7.8995703125	7.2822219050653e-10\\
-7.87828125	6.74009110288277e-10\\
-7.8569921875	5.03046665573316e-10\\
-7.83570312499999	5.12039460470855e-10\\
-7.8144140625	4.5224860312788e-10\\
-7.793125	4.35334842827011e-10\\
-7.7718359375	4.93984652619457e-10\\
-7.750546875	4.73650145935869e-10\\
-7.72925781249999	4.43007304329698e-10\\
-7.70796875	5.11428982831465e-10\\
-7.6866796875	4.28145280761617e-10\\
-7.665390625	3.81173317978388e-10\\
-7.6441015625	3.34625911866288e-10\\
-7.62281249999999	3.15270456208028e-10\\
-7.6015234375	1.98598346823385e-10\\
-7.580234375	1.68815558846968e-10\\
-7.5589453125	1.36469748587639e-10\\
-7.53765625	6.20067174197392e-11\\
-7.51636718749999	7.92096997254221e-11\\
-7.495078125	1.82775307434566e-10\\
-7.4737890625	2.10796804558737e-10\\
-7.4525	1.59392328581541e-10\\
-7.4312109375	1.14201259425893e-10\\
-7.40992187499999	3.74212273028035e-11\\
-7.3886328125	4.65427188941403e-11\\
-7.36734375	-1.33705339717186e-10\\
-7.3460546875	-1.83580781737003e-10\\
-7.324765625	-1.78078204338815e-10\\
-7.30347656249999	-2.21549224496708e-10\\
-7.2821875	-2.48880900792318e-10\\
-7.2608984375	-9.62984369337345e-11\\
-7.239609375	-6.45067901968605e-11\\
-7.2183203125	-1.15516522514325e-10\\
-7.19703124999999	-1.66453071690115e-10\\
-7.1757421875	-1.83611904305098e-10\\
-7.154453125	-2.06013124616398e-10\\
-7.1331640625	-3.79531581021483e-10\\
-7.111875	-4.158364540135e-10\\
-7.09058593749999	-3.16389228105065e-10\\
-7.069296875	-2.97620543114323e-10\\
-7.0480078125	-2.78894640730299e-10\\
-7.02671875	-2.52145667141742e-10\\
-7.0054296875	-2.08734485806745e-10\\
-6.98414062499999	-3.70878752532857e-10\\
-6.9628515625	-4.79186181957665e-10\\
-6.9415625	-4.31857339840982e-10\\
-6.9202734375	-5.71595050621385e-10\\
-6.898984375	-4.89148056066808e-10\\
-6.87769531249999	-6.02074260331453e-10\\
-6.85640625	-5.46729459917451e-10\\
-6.8351171875	-3.413160077806e-10\\
-6.813828125	-3.90986123168948e-10\\
-6.7925390625	-2.83047446935171e-10\\
-6.77124999999999	-2.2377689156714e-10\\
-6.7499609375	-3.60257539128436e-10\\
-6.728671875	-4.61496866675907e-10\\
-6.7073828125	-6.47527204942941e-10\\
-6.68609375	-6.62805002707677e-10\\
-6.66480468749999	-7.52104547388483e-10\\
-6.643515625	-7.73561690812688e-10\\
-6.6222265625	-7.35919259139094e-10\\
-6.6009375	-5.57664476713946e-10\\
-6.5796484375	-5.53192377512441e-10\\
-6.55835937499999	-5.60869996062878e-10\\
-6.5370703125	-6.04273059635788e-10\\
-6.51578125	-8.38613830535682e-10\\
-6.4944921875	-8.78523899744948e-10\\
-6.473203125	-1.00974753822389e-09\\
-6.45191406249999	-1.15812873253965e-09\\
-6.430625	-1.10495672032534e-09\\
-6.4093359375	-1.0406682515034e-09\\
-6.388046875	-8.68714897556633e-10\\
-6.3667578125	-7.47725877705589e-10\\
-6.34546874999999	-5.61994804148288e-10\\
-6.3241796875	-5.03044724517181e-10\\
-6.302890625	-5.41612644745661e-10\\
-6.2816015625	-6.71071903727698e-10\\
-6.2603125	-8.88566038916717e-10\\
-6.23902343749999	-9.94103785487151e-10\\
-6.217734375	-1.18106142214705e-09\\
-6.1964453125	-1.13643869245945e-09\\
-6.17515625	-1.20558361804052e-09\\
-6.1538671875	-1.11936020833968e-09\\
-6.13257812499999	-1.00666154733729e-09\\
-6.1112890625	-1.0130636324353e-09\\
-6.09	-9.98198886647946e-10\\
-6.0687109375	-9.65588510086416e-10\\
-6.047421875	-1.010825054891e-09\\
-6.02613281249999	-1.18364540479085e-09\\
-6.00484375	-1.25113135101295e-09\\
-5.9835546875	-1.20371828882228e-09\\
-5.962265625	-1.28561735493358e-09\\
-5.9409765625	-1.28968452717742e-09\\
-5.91968749999999	-1.25606006991222e-09\\
-5.8983984375	-1.2727450281735e-09\\
-5.877109375	-1.22726639937826e-09\\
-5.8558203125	-1.2544565381872e-09\\
-5.83453125	-1.24014795241968e-09\\
-5.81324218749999	-1.08938011575624e-09\\
-5.791953125	-1.34749476222612e-09\\
-5.7706640625	-1.33364848820476e-09\\
-5.749375	-1.42948656049349e-09\\
-5.7280859375	-1.24358878111477e-09\\
-5.70679687499999	-1.19351848214467e-09\\
-5.6855078125	-1.13231211617197e-09\\
-5.66421875	-9.56425618812489e-10\\
-5.6429296875	-8.96542520317634e-10\\
-5.621640625	-7.52277888833262e-10\\
-5.60035156249999	-7.54665933014322e-10\\
-5.5790625	-7.73124676066357e-10\\
-5.5577734375	-8.45622348528374e-10\\
-5.536484375	-9.01285141082823e-10\\
-5.5151953125	-9.85011844909415e-10\\
-5.49390624999999	-9.78117767844975e-10\\
-5.4726171875	-9.75826412799501e-10\\
-5.451328125	-9.78059572321988e-10\\
-5.4300390625	-8.95921345541302e-10\\
-5.40875	-9.35143927832754e-10\\
-5.38746093749999	-8.53182848597606e-10\\
-5.366171875	-7.10687416376316e-10\\
-5.3448828125	-6.99608528459359e-10\\
-5.32359375	-7.21991510097863e-10\\
-5.3023046875	-7.37713721677438e-10\\
-5.28101562499999	-7.37318958375549e-10\\
-5.2597265625	-7.31593023497264e-10\\
-5.2384375	-7.75026170141419e-10\\
-5.2171484375	-7.38295495155571e-10\\
-5.195859375	-7.60051278405158e-10\\
-5.17457031249999	-6.24537356180226e-10\\
-5.15328125	-4.56882070421223e-10\\
-5.1319921875	-4.12308001134668e-10\\
-5.110703125	-3.6952654437223e-10\\
-5.0894140625	-2.79771648399638e-10\\
-5.06812499999999	-3.38472392094757e-10\\
-5.0468359375	-4.88806017596852e-10\\
-5.025546875	-5.49136529027893e-10\\
-5.0042578125	-5.04739463824348e-10\\
-4.98296875	-6.4232285850522e-10\\
-4.96167968749999	-4.85452214068277e-10\\
-4.940390625	-3.5952137779615e-10\\
-4.9191015625	-2.41450140991387e-10\\
-4.8978125	-1.11192649151233e-10\\
-4.8765234375	-1.10859498522946e-11\\
-4.85523437499999	1.29030148343881e-10\\
-4.8339453125	6.0166918994582e-11\\
-4.81265625	-6.45887092878776e-12\\
-4.7913671875	-3.98264572441583e-13\\
-4.770078125	-1.36098925160136e-10\\
-4.74878906249999	-1.47847725458111e-10\\
-4.7275	-4.08677116594388e-11\\
-4.7062109375	2.62145751771093e-11\\
-4.684921875	1.91433547011111e-10\\
-4.6636328125	1.64004351382845e-10\\
-4.64234374999999	3.72758263818625e-10\\
-4.6210546875	4.96162013567856e-10\\
-4.599765625	4.75706942570768e-10\\
-4.5784765625	5.21760214449857e-10\\
-4.5571875	5.12430944679517e-10\\
-4.53589843749999	4.72553470235169e-10\\
-4.514609375	4.30474558210753e-10\\
-4.4933203125	4.09905657803008e-10\\
-4.47203125	5.01283477740531e-10\\
-4.4507421875	6.1167847226537e-10\\
-4.42945312499999	5.8155431324984e-10\\
-4.4081640625	7.79160226218718e-10\\
-4.386875	7.54343667578442e-10\\
-4.3655859375	7.4828536617011e-10\\
-4.344296875	6.68303794712081e-10\\
-4.32300781249999	6.08761138837553e-10\\
-4.30171875	6.07737035963227e-10\\
-4.2804296875	5.05766118269151e-10\\
-4.259140625	5.32416531246918e-10\\
-4.2378515625	6.58639680241222e-10\\
-4.21656249999999	7.78839971238472e-10\\
-4.1952734375	8.18711500249381e-10\\
-4.173984375	9.44236555142837e-10\\
-4.1526953125	1.06908449258045e-09\\
-4.13140625	1.07373185074822e-09\\
-4.11011718749999	8.95067684170688e-10\\
-4.088828125	9.24140768517601e-10\\
-4.0675390625	8.70927527990826e-10\\
-4.04625	8.12712787935419e-10\\
-4.0249609375	9.8270875854326e-10\\
-4.00367187499999	1.20626569743739e-09\\
-3.9823828125	1.17969841487679e-09\\
-3.96109375	1.22942609591603e-09\\
-3.9398046875	1.4131726017104e-09\\
-3.918515625	1.47573472242443e-09\\
-3.89722656249999	1.48772665039883e-09\\
-3.8759375	1.40964027536142e-09\\
-3.8546484375	1.51276837376371e-09\\
-3.833359375	1.37460725600086e-09\\
-3.8120703125	1.44842040956906e-09\\
-3.79078124999999	1.46638879780691e-09\\
-3.7694921875	1.55282210579428e-09\\
-3.748203125	1.67749721139725e-09\\
-3.7269140625	1.74050767749179e-09\\
-3.705625	1.80366579050177e-09\\
-3.68433593749999	1.81787642799362e-09\\
-3.663046875	1.82848909654939e-09\\
-3.6417578125	1.81135052921972e-09\\
-3.62046875	1.82777400496121e-09\\
-3.5991796875	1.80522328865548e-09\\
-3.57789062499999	1.82253793514367e-09\\
-3.5566015625	1.76457933262656e-09\\
-3.5353125	1.92235819567359e-09\\
-3.5140234375	1.92130572126469e-09\\
-3.492734375	1.84983695211221e-09\\
-3.47144531249999	1.81554674349365e-09\\
-3.45015625	1.81054987631072e-09\\
-3.4288671875	1.67928753054356e-09\\
-3.407578125	1.8250273983194e-09\\
-3.3862890625	1.85645066111668e-09\\
-3.36499999999999	1.8084088088369e-09\\
-3.3437109375	1.83596101070339e-09\\
-3.322421875	1.84347509099771e-09\\
-3.3011328125	1.70038034028093e-09\\
-3.27984375	1.71320953102657e-09\\
-3.25855468749999	1.63335303398755e-09\\
-3.237265625	1.54608292981111e-09\\
-3.2159765625	1.58375923586631e-09\\
-3.1946875	1.59847060247282e-09\\
-3.1733984375	1.57003543883169e-09\\
-3.15210937499999	1.45915586815742e-09\\
-3.1308203125	1.42694676355101e-09\\
-3.10953125	1.33261748944932e-09\\
-3.0882421875	1.5344557117673e-09\\
-3.066953125	1.51677998029307e-09\\
-3.04566406249999	1.41259563088029e-09\\
-3.024375	1.64008051278621e-09\\
-3.0030859375	1.62471315235667e-09\\
-2.981796875	1.63626513867745e-09\\
-2.9605078125	1.56047221570437e-09\\
-2.93921874999999	1.61312944515458e-09\\
-2.9179296875	1.39532540024814e-09\\
-2.896640625	1.5069885694666e-09\\
-2.8753515625	1.39046716204573e-09\\
-2.8540625	1.59196439087538e-09\\
-2.83277343749999	1.64678806192473e-09\\
-2.811484375	1.61578790536585e-09\\
-2.7901953125	1.80008287164464e-09\\
-2.76890625	1.71242129692845e-09\\
-2.7476171875	1.82047708374298e-09\\
-2.72632812499999	1.74460570991848e-09\\
-2.7050390625	1.72234183215771e-09\\
-2.68375	1.76027769183259e-09\\
-2.6624609375	1.51734253417082e-09\\
-2.641171875	1.63179526124615e-09\\
-2.61988281249999	1.63145531884014e-09\\
-2.59859375	1.76325321483851e-09\\
-2.5773046875	1.59120382369531e-09\\
-2.556015625	1.59186551761005e-09\\
-2.5347265625	1.66014436423289e-09\\
-2.51343749999999	1.3544862186219e-09\\
-2.4921484375	1.29220971808581e-09\\
-2.470859375	1.25769625538308e-09\\
-2.4495703125	1.14438704016384e-09\\
-2.42828125	1.02805079183183e-09\\
-2.40699218749999	1.11519596576807e-09\\
-2.385703125	1.09721679986883e-09\\
-2.3644140625	1.07482689601561e-09\\
-2.343125	1.05269755237099e-09\\
-2.3218359375	1.14500598851409e-09\\
-2.30054687499999	8.89444208290624e-10\\
-2.2792578125	9.04136981636854e-10\\
-2.25796875	7.57399556259222e-10\\
-2.2366796875	7.20445733061459e-10\\
-2.215390625	6.41870925443357e-10\\
-2.19410156249999	6.06264154191599e-10\\
-2.1728125	7.24022710915111e-10\\
-2.1515234375	5.04739413512353e-10\\
-2.130234375	5.25487609936639e-10\\
-2.1089453125	3.04151914339804e-10\\
-2.08765624999999	2.51487400055589e-10\\
-2.0663671875	1.24472993395207e-10\\
-2.045078125	2.70049146839026e-10\\
-2.0237890625	1.69451570349131e-10\\
-2.0025	2.86431793914571e-10\\
-1.98121093749999	3.98766203843645e-10\\
-1.959921875	3.5161592229607e-10\\
-1.9386328125	4.02155793445063e-10\\
-1.91734375	2.74701538019865e-10\\
-1.8960546875	3.77165053055102e-10\\
-1.87476562499999	1.1745683544574e-10\\
-1.8534765625	4.66058550116834e-11\\
-1.8321875	2.19269432787145e-11\\
-1.8108984375	1.2702748964453e-10\\
-1.789609375	1.6823829719989e-12\\
-1.76832031249999	2.54989033965588e-10\\
-1.74703125	2.18418038549869e-10\\
-1.7257421875	2.86707082179469e-10\\
-1.704453125	3.04931070655953e-10\\
-1.6831640625	3.88648015437646e-10\\
-1.66187499999999	2.10823308411998e-10\\
-1.6405859375	3.015615258607e-10\\
-1.619296875	3.76862682954403e-10\\
-1.5980078125	2.94036397118987e-10\\
-1.57671875	7.49721327232634e-11\\
-1.55542968749999	1.2807805618085e-10\\
-1.534140625	1.96037266229885e-11\\
-1.5128515625	-1.58143912293187e-10\\
-1.4915625	-1.77134503439747e-10\\
-1.4702734375	-1.30706431539221e-10\\
-1.44898437499999	-2.84906750758406e-10\\
-1.4276953125	-3.90908580228797e-10\\
-1.40640625	-3.70797148387106e-10\\
-1.3851171875	-4.35142625805515e-10\\
-1.363828125	-5.36196756873717e-10\\
-1.34253906249999	-6.46180041700992e-10\\
-1.32125	-5.39591237658486e-10\\
-1.2999609375	-4.10755812706366e-10\\
-1.278671875	-4.11612632896846e-10\\
-1.2573828125	-3.79083442374626e-10\\
-1.23609374999999	-4.40143650522866e-10\\
-1.2148046875	-4.77930720408222e-10\\
-1.193515625	-7.49884369736803e-10\\
-1.1722265625	-7.48255125744476e-10\\
-1.1509375	-7.66771126404291e-10\\
-1.12964843749999	-7.88777872501124e-10\\
-1.108359375	-7.5589827943181e-10\\
-1.0870703125	-6.46670205893961e-10\\
-1.06578125	-4.49902998112965e-10\\
-1.0444921875	-4.10959035945941e-10\\
-1.02320312499999	-3.37435464665775e-10\\
-1.0019140625	-2.39174513231421e-10\\
-0.980624999999996	-3.30527330574883e-10\\
-0.959335937500001	-5.54773079093946e-10\\
-0.938046874999998	-4.92404169640273e-10\\
-0.916757812499995	-5.69886091003242e-10\\
-0.895468749999999	-6.41932945537915e-10\\
-0.874179687499996	-5.76633946340956e-10\\
-0.852890625000001	-3.79495222657975e-10\\
-0.831601562499998	-4.18223504751419e-10\\
-0.810312499999995	-1.46855279031625e-10\\
-0.789023437499999	-5.97016422422264e-11\\
-0.767734374999996	-5.51386300257911e-12\\
-0.746445312500001	-5.61750021612574e-11\\
-0.725156249999998	-1.53838144047345e-10\\
-0.703867187499995	-3.17591816091846e-10\\
-0.682578124999999	-3.4129554580923e-10\\
-0.661289062499996	-2.05574553083217e-10\\
-0.640000000000001	-2.19439532172504e-10\\
-0.618710937499998	-1.6373007424047e-11\\
-0.597421874999995	4.79476671143458e-11\\
-0.576132812499999	3.72067332602985e-11\\
-0.554843749999996	3.50577460436398e-10\\
-0.533554687500001	2.39498896397424e-10\\
-0.512265624999998	1.21576881381135e-10\\
-0.490976562499995	2.18557298134561e-10\\
-0.469687499999999	-8.05749056171857e-11\\
-0.448398437499996	-1.81755328347737e-10\\
-0.427109375000001	-1.31744212369237e-10\\
-0.405820312499998	-9.11427911947051e-11\\
-0.384531249999995	-6.61917131471517e-11\\
-0.363242187499999	1.5468188809605e-10\\
-0.341953124999996	2.63247716583146e-10\\
-0.320664062500001	4.04273530079829e-10\\
-0.299374999999998	3.85339904474072e-10\\
-0.278085937499995	4.87174092953756e-10\\
-0.256796874999999	4.11006023272908e-10\\
-0.235507812499996	2.35852266929557e-10\\
-0.214218750000001	1.98243326338851e-10\\
-0.192929687499998	7.84639496073321e-11\\
-0.171640624999995	6.59663383809172e-11\\
-0.150351562499999	2.10802606522289e-10\\
-0.129062499999996	3.07160207470803e-10\\
-0.107773437500001	7.87135795132707e-10\\
-0.0864843749999977	8.23785227908067e-10\\
-0.0651953124999949	1.13204904851565e-09\\
-0.0439062499999991	1.25043783179799e-09\\
-0.0226171874999963	1.24680578329834e-09\\
-0.00132812500000057	1.22364818061886e-09\\
0.0199609375000023	1.31322221586536e-09\\
0.0412500000000051	1.20717427573127e-09\\
0.0625390625000009	1.34239718823269e-09\\
0.0838281250000037	1.44338979512518e-09\\
0.105117187499999	1.60585988125987e-09\\
0.126406250000002	1.78608277576463e-09\\
0.147695312500005	2.02534827384567e-09\\
0.168984375000001	2.14964312083619e-09\\
0.190273437500004	2.29555083800447e-09\\
0.211562499999999	2.3042613996321e-09\\
0.232851562500002	2.22016166598087e-09\\
0.254140625000005	2.41306697852905e-09\\
0.275429687500001	2.41760151515821e-09\\
0.296718750000004	2.32763740065647e-09\\
0.318007812499999	2.46309457515713e-09\\
0.339296875000002	2.728806656936e-09\\
0.360585937500005	2.88304234956546e-09\\
0.381875000000001	2.9798279104288e-09\\
0.403164062500004	3.29929012690009e-09\\
0.424453124999999	3.36781096858981e-09\\
0.445742187500002	3.47348717724293e-09\\
0.467031250000005	3.63488067967482e-09\\
0.488320312500001	3.69032821369128e-09\\
0.509609375000004	3.56559011477186e-09\\
0.530898437499999	3.61910686847574e-09\\
0.552187500000002	3.71236570987807e-09\\
0.573476562500005	3.9412424878924e-09\\
0.594765625000001	3.90192142237691e-09\\
0.616054687500004	4.06506088760242e-09\\
0.637343749999999	4.31312403524096e-09\\
0.658632812500002	4.4698666893617e-09\\
0.679921875000005	4.63241492302928e-09\\
0.701210937500001	4.82680532065086e-09\\
0.722500000000004	4.8265887801918e-09\\
0.743789062499999	4.66641386637561e-09\\
0.765078125000002	4.48483907492759e-09\\
0.786367187500005	4.55290940724249e-09\\
0.807656250000001	4.72618701910767e-09\\
0.828945312500004	4.82657567705056e-09\\
0.850234374999999	5.11270500750175e-09\\
0.871523437500002	5.18292573527444e-09\\
0.892812500000005	5.4742165939959e-09\\
0.914101562500001	5.64006034652269e-09\\
0.935390625000004	5.64394464282397e-09\\
0.956679687499999	5.70326075473251e-09\\
0.977968750000002	5.77680802598878e-09\\
0.999257812500005	5.83323992898655e-09\\
1.020546875	5.81981682145955e-09\\
1.0418359375	5.97079247065767e-09\\
1.063125	6.29083630276292e-09\\
1.0844140625	6.4684089631726e-09\\
1.10570312500001	6.72008605942229e-09\\
1.1269921875	6.90582015745285e-09\\
1.14828125	7.16970482893652e-09\\
1.1695703125	7.26007217035073e-09\\
1.190859375	7.26727874275485e-09\\
1.21214843750001	7.34343491453933e-09\\
1.2334375	7.41056400944912e-09\\
1.2547265625	7.29528494272609e-09\\
1.276015625	7.48241581598295e-09\\
1.2973046875	7.74538559906706e-09\\
1.31859375000001	8.05463235197612e-09\\
1.3398828125	8.25037717748591e-09\\
1.361171875	8.45975433337932e-09\\
1.3824609375	8.72851146788337e-09\\
1.40375	8.62670642827232e-09\\
1.42503906250001	8.6179177480833e-09\\
1.446328125	8.54520893820691e-09\\
1.4676171875	8.3017591193121e-09\\
1.48890625	8.35902382477197e-09\\
1.5101953125	8.47081582456615e-09\\
1.53148437500001	8.64767183125338e-09\\
1.5527734375	8.92017041660267e-09\\
1.5740625	9.08733867592785e-09\\
1.5953515625	9.14732322445163e-09\\
1.616640625	9.2804238087996e-09\\
1.63792968750001	9.26058682391066e-09\\
1.65921875	9.16518857164964e-09\\
1.6805078125	9.01695949948671e-09\\
1.701796875	9.03245408463465e-09\\
1.7230859375	8.99395293099554e-09\\
1.74437500000001	8.99478966804254e-09\\
1.7656640625	9.05146869805723e-09\\
1.786953125	9.41235450264725e-09\\
1.8082421875	9.42135811317486e-09\\
1.82953125	9.62535654106287e-09\\
1.85082031250001	9.79145228248799e-09\\
1.872109375	9.89819949254591e-09\\
1.8933984375	9.79931001876702e-09\\
1.9146875	9.87903048173564e-09\\
1.9359765625	9.84301019891917e-09\\
1.95726562500001	9.95395224794593e-09\\
1.9785546875	9.945886957282e-09\\
1.99984375	1.01469068298288e-08\\
2.0211328125	1.04267687609399e-08\\
2.042421875	1.05846652805971e-08\\
2.06371093750001	1.0587164666598e-08\\
2.085	1.07624940924369e-08\\
2.1062890625	1.06572234919157e-08\\
2.127578125	1.0604533983569e-08\\
2.1488671875	1.06850961624546e-08\\
2.17015625000001	1.06343533402358e-08\\
2.1914453125	1.04825565268034e-08\\
2.212734375	1.04062400396569e-08\\
2.2340234375	1.04157314835443e-08\\
2.2553125	1.03246354234298e-08\\
2.27660156250001	1.03984454207878e-08\\
2.297890625	1.04143399003729e-08\\
2.3191796875	1.0569346746494e-08\\
2.34046875	1.0482371185402e-08\\
2.3617578125	1.04383808972258e-08\\
2.38304687500001	1.04748680949448e-08\\
2.4043359375	1.02527732446447e-08\\
2.425625	1.01172522674915e-08\\
2.4469140625	9.95521172563915e-09\\
2.468203125	9.867003966246e-09\\
2.48949218750001	9.58757134102987e-09\\
2.51078125	9.54026951309975e-09\\
2.5320703125	9.44495788757277e-09\\
2.553359375	9.45362443484207e-09\\
2.5746484375	9.33943813199942e-09\\
2.59593750000001	9.35025992458241e-09\\
2.6172265625	9.30338038581906e-09\\
2.638515625	9.02906111901812e-09\\
2.6598046875	8.76442854148949e-09\\
2.68109375	8.54493790367716e-09\\
2.70238281250001	8.30707901189644e-09\\
2.723671875	8.24224412347915e-09\\
2.7449609375	8.20242366168811e-09\\
2.76625	8.12132429794037e-09\\
2.7875390625	8.33985094216707e-09\\
2.80882812500001	8.3929184544816e-09\\
2.8301171875	8.35494668485554e-09\\
2.85140625	8.40438942885434e-09\\
2.8726953125	8.28028604620521e-09\\
2.893984375	7.9618805734749e-09\\
2.91527343750001	7.74932627493007e-09\\
2.9365625	7.77577327423483e-09\\
2.9578515625	7.32953412251084e-09\\
2.979140625	7.48016842852474e-09\\
3.0004296875	7.33419665272865e-09\\
3.02171875000001	7.49055699883374e-09\\
3.0430078125	7.5296784105918e-09\\
3.064296875	7.46861482994419e-09\\
3.0855859375	7.41558292803734e-09\\
3.106875	7.41744817491438e-09\\
3.12816406250001	7.03092029309569e-09\\
3.149453125	7.07824132423565e-09\\
3.1707421875	6.70490600999289e-09\\
3.19203125	6.65170501046308e-09\\
3.2133203125	6.56911451097396e-09\\
3.23460937500001	6.53678171450089e-09\\
3.2558984375	6.64379464822996e-09\\
3.2771875	6.50688036022238e-09\\
3.2984765625	6.5355862644612e-09\\
3.319765625	6.44339108324196e-09\\
3.34105468750001	6.20247200408766e-09\\
3.36234375	6.10825601849273e-09\\
3.3836328125	5.91334048614683e-09\\
3.404921875	5.85825347422225e-09\\
3.4262109375	5.77802899495926e-09\\
3.44750000000001	5.8227319870691e-09\\
3.4687890625	5.95139808820141e-09\\
3.490078125	5.85650001006721e-09\\
3.5113671875	5.75702136989373e-09\\
3.53265625	5.64945592294985e-09\\
3.55394531250001	5.3016497900526e-09\\
3.575234375	5.18086560426367e-09\\
3.5965234375	4.86058938454835e-09\\
3.6178125	4.60027205656269e-09\\
3.6391015625	4.6852200645292e-09\\
3.66039062500001	4.61654536493518e-09\\
3.6816796875	4.53266575224506e-09\\
3.70296875	4.67044534835337e-09\\
3.7242578125	4.53061619357126e-09\\
3.745546875	4.49900213620128e-09\\
3.76683593750001	4.3117858353726e-09\\
3.788125	4.24509154734169e-09\\
3.8094140625	4.05317419685667e-09\\
3.830703125	3.8681175313798e-09\\
3.8519921875	3.95488005233074e-09\\
3.87328125000001	3.86374864160358e-09\\
3.8945703125	3.78332157367067e-09\\
3.915859375	3.77241751332762e-09\\
3.9371484375	3.70705397859942e-09\\
3.9584375	3.87271227331045e-09\\
3.97972656250001	3.70231575872947e-09\\
4.001015625	3.63462691371872e-09\\
4.0223046875	3.59221604284939e-09\\
4.04359375	3.34465082393279e-09\\
4.0648828125	3.30326464206175e-09\\
4.08617187500001	3.24722852521649e-09\\
4.1074609375	3.18707900229715e-09\\
4.12875	3.01914280563562e-09\\
4.1500390625	3.11661462875554e-09\\
4.171328125	2.9954636149942e-09\\
4.19261718750001	2.93016204333538e-09\\
4.21390625	2.85484937216701e-09\\
4.2351953125	2.84984256944468e-09\\
4.256484375	2.80043536034746e-09\\
4.2777734375	2.63517126088378e-09\\
4.29906250000001	2.80161672804542e-09\\
4.3203515625	2.59516335081175e-09\\
4.341640625	2.3460369011616e-09\\
4.3629296875	2.36975894903324e-09\\
4.38421875	2.33289659389546e-09\\
4.40550781250001	2.31337767661014e-09\\
4.426796875	2.31816721606206e-09\\
4.4480859375	2.31749361140016e-09\\
4.469375	2.24467704595996e-09\\
4.4906640625	2.15978521452487e-09\\
4.51195312500001	2.15875566008319e-09\\
4.5332421875	1.98153472035354e-09\\
4.55453125	2.0329153250982e-09\\
4.5758203125	1.9946175285476e-09\\
4.597109375	1.78551016896525e-09\\
4.61839843750001	1.80652480242993e-09\\
4.6396875	1.89150564739081e-09\\
4.6609765625	1.79191149172153e-09\\
4.682265625	1.85966132236774e-09\\
4.7035546875	1.82790496667989e-09\\
4.72484375000001	1.73799733816924e-09\\
4.7461328125	1.63713054439337e-09\\
4.767421875	1.52779485745915e-09\\
4.7887109375	1.47980405613641e-09\\
4.81	1.55439724957486e-09\\
4.83128906250001	1.53705195002106e-09\\
4.852578125	1.61483487241955e-09\\
4.8738671875	1.63495614466978e-09\\
4.89515625	1.58777315828746e-09\\
4.9164453125	1.65901796107728e-09\\
4.93773437500001	1.5203726469769e-09\\
4.9590234375	1.40454599443864e-09\\
4.9803125	1.25010010068075e-09\\
5.0016015625	1.13931343726864e-09\\
5.022890625	9.61934717253625e-10\\
5.04417968750001	1.00585108871351e-09\\
5.06546875	1.00489413150393e-09\\
5.0867578125	8.44863958703133e-10\\
5.108046875	9.28326048434928e-10\\
5.1293359375	8.6756608185688e-10\\
5.15062500000001	8.42037094220102e-10\\
5.1719140625	9.50640477944385e-10\\
5.193203125	8.17698811626853e-10\\
5.2144921875	7.74263860927465e-10\\
5.23578125	6.32252976488331e-10\\
5.25707031250001	5.38719302505483e-10\\
5.278359375	3.74721739694747e-10\\
5.2996484375	2.93916571947027e-10\\
5.3209375	3.35606356667351e-10\\
5.3422265625	3.83343900370086e-10\\
5.36351562500001	3.6822127158509e-10\\
5.3848046875	3.59298303736013e-10\\
5.40609375	3.2326201026584e-10\\
5.4273828125	3.04520240731294e-10\\
5.448671875	1.27415766131469e-10\\
5.46996093750001	1.91717033721573e-10\\
5.49125	2.05202436217873e-10\\
5.5125390625	1.10503548150677e-10\\
5.533828125	1.38692149376047e-10\\
5.5551171875	3.98168325940951e-11\\
5.57640625000001	2.38227119623971e-11\\
5.5976953125	1.52701171612411e-11\\
5.618984375	5.1943037480258e-13\\
5.6402734375	-7.78044475236036e-12\\
5.6615625	-6.13925006320923e-11\\
5.68285156250001	-1.95432512601747e-10\\
5.704140625	-1.59019379074972e-10\\
5.7254296875	-1.33678441516162e-10\\
5.74671875	-2.00702124104347e-10\\
5.7680078125	-2.30220995028226e-10\\
5.78929687500001	-2.0991340052675e-10\\
5.8105859375	-3.32776440333831e-10\\
5.831875	-3.19188434058291e-10\\
5.8531640625	-3.64471034393443e-10\\
5.874453125	-3.00796851149584e-10\\
5.89574218750001	-3.23800491892925e-10\\
5.91703125	-3.01550765988099e-10\\
5.9383203125	-3.31676386394257e-10\\
5.959609375	-4.83865075918569e-10\\
5.9808984375	-5.96262007339925e-10\\
6.00218750000001	-5.96852642757388e-10\\
6.0234765625	-5.72286023186301e-10\\
6.044765625	-5.50868393498921e-10\\
6.0660546875	-5.35528010063043e-10\\
6.08734375	-5.33378267549619e-10\\
6.10863281250001	-3.46711973757373e-10\\
6.129921875	-5.09688839969444e-10\\
6.1512109375	-5.55296992495767e-10\\
6.1725	-5.60986532140975e-10\\
6.1937890625	-5.47741059102499e-10\\
6.21507812500001	-6.78122278188629e-10\\
6.2363671875	-6.12211021424769e-10\\
6.25765625	-6.52090078220062e-10\\
6.2789453125	-6.10010232471465e-10\\
6.300234375	-6.64963945546291e-10\\
6.32152343750001	-6.60666448138577e-10\\
6.3428125	-6.44782697608133e-10\\
6.3641015625	-5.6590167472296e-10\\
6.385390625	-6.51673802956877e-10\\
6.4066796875	-6.10034008830738e-10\\
6.42796875000001	-7.32289154389872e-10\\
6.4492578125	-7.7240465861608e-10\\
6.470546875	-7.72604806929625e-10\\
6.4918359375	-7.64651079526119e-10\\
6.513125	-8.28254095530446e-10\\
6.53441406250001	-8.71016212640884e-10\\
6.555703125	-7.72491751531963e-10\\
6.5769921875	-8.89192925983766e-10\\
6.59828125	-8.17020951652887e-10\\
6.6195703125	-8.07964277397075e-10\\
6.64085937500001	-9.59075012370407e-10\\
6.6621484375	-9.48967236032714e-10\\
6.6834375	-1.00778868605189e-09\\
6.7047265625	-1.04519899978007e-09\\
6.726015625	-9.83808761755888e-10\\
6.74730468750001	-1.13170410140297e-09\\
6.76859375	-9.77302990975762e-10\\
6.7898828125	-9.25144138508424e-10\\
6.811171875	-1.17848963652439e-09\\
6.8324609375	-9.99833368720071e-10\\
6.85375000000001	-9.09441819615336e-10\\
6.8750390625	-9.79266366707428e-10\\
6.896328125	-1.00586324188131e-09\\
6.9176171875	-1.03198920549766e-09\\
6.93890625	-1.05199401708353e-09\\
6.96019531250001	-1.15956327183827e-09\\
6.981484375	-1.27708788603627e-09\\
7.0027734375	-1.20074580033126e-09\\
7.0240625	-1.26023707361003e-09\\
7.0453515625	-1.25216842459271e-09\\
7.06664062500001	-1.18725735923092e-09\\
7.0879296875	-1.08158875806362e-09\\
7.10921875	-1.00525246300687e-09\\
7.1305078125	-1.03541416306982e-09\\
7.151796875	-1.1119764569068e-09\\
7.17308593750001	-1.10338191693553e-09\\
7.194375	-1.25195850912669e-09\\
7.2156640625	-1.40260956321925e-09\\
7.236953125	-1.34450663872467e-09\\
7.2582421875	-1.4239192520693e-09\\
7.27953125000001	-1.48353353988992e-09\\
7.3008203125	-1.3244172909452e-09\\
7.322109375	-1.20716472406598e-09\\
7.3433984375	-1.20679851657707e-09\\
7.3646875	-1.20114398664355e-09\\
7.38597656250001	-1.28361290830269e-09\\
7.407265625	-1.23553275927443e-09\\
7.4285546875	-1.2512273949183e-09\\
7.44984375	-1.33917567613263e-09\\
7.4711328125	-1.35383668838623e-09\\
7.49242187500001	-1.32048747422824e-09\\
7.5137109375	-1.45619019250689e-09\\
7.535	-1.42229620632124e-09\\
7.5562890625	-1.32628193288935e-09\\
7.577578125	-1.28344211649737e-09\\
7.59886718750001	-1.29726308411984e-09\\
7.62015625	-1.30566040073461e-09\\
7.6414453125	-1.26067836341521e-09\\
7.662734375	-1.22399219660492e-09\\
7.6840234375	-1.2387233397987e-09\\
7.70531250000001	-1.15608027200374e-09\\
7.7266015625	-1.21447163707915e-09\\
7.747890625	-1.16009258473999e-09\\
7.7691796875	-1.24648568913231e-09\\
7.79046875	-1.17419222394481e-09\\
7.81175781250001	-1.16287236824619e-09\\
7.833046875	-1.23047184191074e-09\\
7.8543359375	-1.13302458286418e-09\\
7.875625	-1.14369209161941e-09\\
7.8969140625	-1.15493293897552e-09\\
7.91820312500001	-1.13442967614155e-09\\
7.9394921875	-1.00060899266689e-09\\
7.96078125	-9.550930005524e-10\\
7.9820703125	-9.83671737571402e-10\\
8.003359375	-1.05004314417157e-09\\
8.02464843750001	-1.07111974567175e-09\\
8.0459375	-1.08114041612567e-09\\
8.0672265625	-1.15630926314726e-09\\
8.088515625	-1.01929625972762e-09\\
8.1098046875	-1.11509950928989e-09\\
8.13109375000001	-9.69857070813849e-10\\
8.1523828125	-1.02171216132268e-09\\
8.173671875	-8.38304531083712e-10\\
8.1949609375	-8.97374811709889e-10\\
8.21625	-8.82883931975994e-10\\
8.23753906250001	-7.85540250903803e-10\\
8.258828125	-8.46324996395933e-10\\
8.2801171875	-8.71568820069401e-10\\
8.30140625	-8.46725475868315e-10\\
8.3226953125	-7.85731789499828e-10\\
8.34398437500001	-7.49575838136877e-10\\
8.3652734375	-7.17837161086166e-10\\
8.3865625	-5.64562381947216e-10\\
8.4078515625	-4.17950115757485e-10\\
8.429140625	-4.93480808703679e-10\\
8.45042968750001	-3.12196055080331e-10\\
8.47171875	-2.98788498210075e-10\\
8.4930078125	-3.61198916226391e-10\\
8.514296875	-3.41516516364217e-10\\
8.5355859375	-4.31092477999367e-10\\
8.55687500000001	-4.71117528319998e-10\\
8.5781640625	-5.92345524167939e-10\\
8.599453125	-5.67250035365239e-10\\
8.6207421875	-4.91407390511654e-10\\
8.64203125	-4.50323361032134e-10\\
8.66332031250001	-3.46066451649279e-10\\
8.684609375	-3.55532756289892e-10\\
8.7058984375	-2.33921829907548e-10\\
8.7271875	-1.87690351380989e-10\\
8.7484765625	-2.23538042350177e-10\\
8.76976562500001	-2.16679415298395e-10\\
8.7910546875	-2.9100962606081e-10\\
8.81234375	-1.99482945114521e-10\\
8.8336328125	-1.44832149444495e-10\\
8.854921875	-1.3000521298283e-10\\
8.87621093750001	-1.08155321407236e-10\\
8.8975	-1.21783859130986e-10\\
8.9187890625	-6.26677252611519e-11\\
8.940078125	-3.1008683181514e-11\\
8.9613671875	-8.3846971270501e-11\\
8.98265625000001	-1.19321597428933e-10\\
9.0039453125	-2.89455975102982e-10\\
9.025234375	-3.08736393260584e-10\\
9.0465234375	-3.83848152828937e-10\\
9.0678125	-3.80050191373936e-10\\
9.08910156250001	-2.70429990510201e-10\\
9.110390625	-3.33833854396756e-10\\
9.1316796875	-2.44886105925543e-10\\
9.15296875	-1.61197176878223e-10\\
9.1742578125	-6.47649576341075e-11\\
9.19554687500001	-3.70058612031762e-11\\
9.2168359375	4.14679926196506e-12\\
9.238125	-8.8485070240548e-11\\
9.2594140625	-1.9086243984188e-10\\
9.280703125	-1.97549566526598e-10\\
9.30199218750001	-2.71982612802826e-10\\
9.32328125	-1.81280290964915e-10\\
9.3445703125	-1.73279195268713e-10\\
9.365859375	-8.00814827768455e-11\\
9.3871484375	-1.41083875027095e-10\\
9.40843750000001	-8.24106487094544e-11\\
9.4297265625	3.05722622132267e-11\\
9.451015625	-7.7110966524661e-11\\
9.4723046875	-1.37451717250595e-11\\
9.49359375	1.2999525929895e-10\\
9.51488281250001	9.27091555652286e-11\\
9.536171875	1.16403848744857e-10\\
9.5574609375	-4.27957241502129e-11\\
9.57875	3.71259280715842e-11\\
9.6000390625	-5.67480950532586e-11\\
9.62132812500001	-1.89091223506908e-10\\
9.6426171875	-4.19417820678895e-12\\
9.66390625	-1.22294218376857e-10\\
9.6851953125	-5.07047885765041e-11\\
9.706484375	-6.07055069492822e-11\\
9.72777343750001	4.53425284297181e-13\\
9.7490625	-5.26022334883874e-13\\
9.7703515625	-4.28968991963503e-11\\
9.791640625	2.5734606505228e-11\\
9.8129296875	6.37634015988529e-12\\
9.83421875000001	3.38890389650686e-12\\
9.8555078125	6.5931677951422e-11\\
9.876796875	-2.78948397783185e-11\\
9.8980859375	8.08534663932682e-11\\
9.919375	1.07909609695399e-10\\
9.94066406250001	6.58049051503291e-11\\
9.961953125	9.09305095973823e-11\\
9.9832421875	7.68681865492882e-11\\
10.00453125	1.35302861250206e-10\\
10.0258203125	1.70650318402985e-10\\
10.047109375	1.25713566449326e-10\\
10.0683984375	1.20374162879665e-10\\
10.0896875	1.72042183124601e-10\\
10.1109765625	1.23341042218861e-10\\
10.132265625	-2.59013877213977e-11\\
10.1535546875	-1.09064012862597e-12\\
10.17484375	-1.05824794436862e-10\\
10.1961328125	-1.35082783157963e-10\\
10.217421875	-1.19020417130273e-10\\
10.2387109375	-2.47632288132423e-12\\
10.26	-1.24891933987873e-10\\
10.2812890625	-4.78244378448078e-11\\
10.302578125	-6.2471247802657e-12\\
10.3238671875	-3.85727193820908e-11\\
10.34515625	-3.43916684755637e-11\\
10.3664453125	-9.95691785523123e-11\\
10.387734375	-3.98597153747249e-11\\
10.4090234375	2.20146211438999e-11\\
10.4303125	-8.64601446348241e-11\\
10.4516015625	-1.0489925990185e-11\\
10.472890625	9.80495420531467e-11\\
10.4941796875	1.74274280485576e-10\\
10.51546875	1.96120645871177e-10\\
10.5367578125	2.28354020576039e-10\\
10.558046875	2.57820943591366e-10\\
10.5793359375	1.79942584036133e-10\\
10.600625	1.60500732196891e-10\\
10.6219140625	1.86017164958095e-10\\
10.643203125	4.25450840315515e-11\\
10.6644921875	3.9394278934571e-11\\
10.68578125	9.01053595041932e-11\\
10.7070703125	1.72950016829077e-10\\
10.728359375	3.0066572258062e-10\\
10.7496484375	2.42298049587504e-10\\
10.7709375	2.34220276592751e-10\\
10.7922265625	9.50002900348447e-11\\
10.813515625	1.3185795957748e-10\\
10.8348046875	1.973871685422e-11\\
10.85609375	4.65220858939197e-11\\
10.8773828125	-5.90627099121312e-11\\
10.898671875	-1.17491514604928e-10\\
10.9199609375	-3.43088309889496e-12\\
10.94125	4.75154202227358e-11\\
10.9625390625	-1.1814932737086e-10\\
10.983828125	-5.51620517027559e-11\\
11.0051171875	-2.96752385531495e-11\\
11.02640625	-1.03521183584383e-10\\
11.0476953125	1.09384078541321e-10\\
11.068984375	1.22111644231593e-11\\
11.0902734375	8.48164338417538e-11\\
11.1115625	1.13951193762717e-10\\
11.1328515625	6.71649985183321e-11\\
11.154140625	4.36914482702867e-11\\
11.1754296875	1.0904799051746e-11\\
11.19671875	1.17233253688775e-10\\
11.2180078125	-8.45848444925256e-11\\
11.239296875	4.24352556899351e-13\\
11.2605859375	-5.35152261532689e-11\\
11.281875	-2.61538726949503e-11\\
11.3031640625	9.56240829235851e-11\\
11.324453125	-6.92039177303635e-11\\
11.3457421875	-5.51374513552857e-11\\
11.36703125	-1.88344065110293e-11\\
11.3883203125	5.22896993269518e-11\\
11.409609375	-2.23395977507965e-11\\
11.4308984375	-4.34391552060915e-11\\
11.4521875	1.71996351539563e-10\\
11.4734765625	1.3247633926411e-10\\
11.494765625	1.53661291446267e-10\\
11.5160546875	2.80663366044925e-10\\
11.53734375	1.83687138394039e-10\\
11.5586328125	1.6688313239327e-10\\
11.579921875	5.07182060439883e-11\\
11.6012109375	8.75097842645382e-11\\
11.6225	1.24057314474289e-12\\
11.6437890625	-8.76086132348755e-11\\
11.665078125	-1.32674934633511e-10\\
11.6863671875	-1.02774334653512e-10\\
11.70765625	-9.83892636682803e-11\\
11.7289453125	-1.30844778156185e-10\\
11.750234375	-1.24330759085176e-11\\
11.7715234375	-9.04374850849649e-11\\
11.7928125	-1.47547993815622e-10\\
11.8141015625	-5.63610406623148e-11\\
11.835390625	-1.59018455796648e-10\\
11.8566796875	-1.49477480644419e-10\\
11.87796875	-3.63440854348395e-10\\
11.8992578125	-2.37195479016479e-10\\
11.920546875	-3.92530206840556e-10\\
11.9418359375	-3.60285310465723e-10\\
11.963125	-2.87279839620437e-10\\
11.9844140625	-5.13434343416345e-10\\
12.005703125	-5.51718767405859e-10\\
12.0269921875	-5.05942343621432e-10\\
12.04828125	-4.32516140048384e-10\\
12.0695703125	-3.13730207126249e-10\\
12.090859375	-3.59127648159262e-10\\
12.1121484375	-2.20169485605337e-10\\
12.1334375	-1.88027777673892e-10\\
12.1547265625	-3.29147554805875e-10\\
12.176015625	-2.02811018069667e-10\\
12.1973046875	-4.66356645960605e-10\\
12.21859375	-3.94138467927235e-10\\
12.2398828125	-4.92913146468627e-10\\
12.261171875	-5.24276880231453e-10\\
12.2824609375	-4.73750258351131e-10\\
12.30375	-4.5775246819075e-10\\
12.3250390625	-3.9169055845905e-10\\
12.346328125	-3.31314835569668e-10\\
12.3676171875	-3.01175169748132e-10\\
12.38890625	-3.15819112762318e-10\\
12.4101953125	-3.16799819660349e-10\\
12.431484375	-3.45969033016287e-10\\
12.4527734375	-4.72012844217663e-10\\
12.4740625	-4.96918990790847e-10\\
12.4953515625	-4.48595983417273e-10\\
12.516640625	-3.78780744756613e-10\\
12.5379296875	-2.16260241760088e-10\\
12.55921875	-2.56860604999545e-10\\
12.5805078125	-2.04041872353946e-10\\
12.601796875	-2.59046352573599e-10\\
12.6230859375	-2.97452433156162e-10\\
12.644375	-3.54064522134197e-10\\
12.6656640625	-4.21790529894922e-10\\
12.686953125	-5.29567238117694e-10\\
12.7082421875	-5.9574280319264e-10\\
12.72953125	-5.9213789146584e-10\\
12.7508203125	-6.16933042042447e-10\\
12.772109375	-5.09404628583841e-10\\
12.7933984375	-5.83022152929464e-10\\
12.8146875	-4.38173023523205e-10\\
12.8359765625	-5.1145756967982e-10\\
12.857265625	-5.70267590841061e-10\\
12.8785546875	-7.25480121984618e-10\\
12.89984375	-7.98122869992933e-10\\
12.9211328125	-8.06802194709481e-10\\
12.942421875	-7.70066066147038e-10\\
12.9637109375	-7.97463507516993e-10\\
12.985	-7.20897583624775e-10\\
13.0062890625	-6.55974131475669e-10\\
13.027578125	-4.88038129919684e-10\\
13.0488671875	-4.27489095703264e-10\\
13.07015625	-4.04926605688223e-10\\
13.0914453125	-4.9763619722321e-10\\
13.112734375	-5.73759559722576e-10\\
13.1340234375	-4.55276194015314e-10\\
13.1553125	-5.48247148341523e-10\\
13.1766015625	-5.3818984550554e-10\\
13.197890625	-5.02584779681455e-10\\
13.2191796875	-5.25059126989056e-10\\
13.24046875	-3.96125046014798e-10\\
13.2617578125	-3.25561636067664e-10\\
13.283046875	-3.05949421694884e-10\\
13.3043359375	-2.44680463372007e-10\\
13.325625	-2.7294532641625e-10\\
13.3469140625	-1.80058627772027e-10\\
13.368203125	-2.99034633522123e-10\\
13.3894921875	-3.45381594739627e-10\\
13.41078125	-3.91683553195793e-10\\
13.4320703125	-3.71460008849145e-10\\
13.453359375	-4.28955631402337e-10\\
13.4746484375	-3.57021708540532e-10\\
13.4959375	-3.61474176349115e-10\\
13.5172265625	-3.14303572126521e-10\\
13.538515625	-2.95088808113214e-10\\
13.5598046875	-2.70174926942611e-10\\
13.58109375	-2.329041070994e-10\\
13.6023828125	-3.69018323352614e-10\\
13.623671875	-4.5582684271652e-10\\
13.6449609375	-4.69752037865625e-10\\
13.66625	-5.92838644094382e-10\\
13.6875390625	-6.23746180565269e-10\\
13.708828125	-5.82785460467086e-10\\
13.7301171875	-4.8420043871308e-10\\
13.75140625	-5.18316926309329e-10\\
13.7726953125	-4.57977569159367e-10\\
13.793984375	-3.48752737374637e-10\\
13.8152734375	-4.5606112278073e-10\\
13.8365625	-3.79474535628062e-10\\
13.8578515625	-5.16647627086379e-10\\
13.879140625	-5.50560546174138e-10\\
13.9004296875	-5.35494736347531e-10\\
13.92171875	-5.92778749632191e-10\\
13.9430078125	-5.44689433384367e-10\\
13.964296875	-5.07992220722262e-10\\
13.9855859375	-4.41914952082659e-10\\
14.006875	-2.97402477637836e-10\\
14.0281640625	-3.81458305637564e-10\\
14.049453125	-3.03980874300889e-10\\
14.0707421875	-3.47090332352709e-10\\
14.09203125	-4.11523432213166e-10\\
14.1133203125	-4.36064680380376e-10\\
14.134609375	-4.4350787963331e-10\\
14.1558984375	-3.91884590447031e-10\\
14.1771875	-3.57244066775176e-10\\
14.1984765625	-2.63121528980225e-10\\
14.219765625	-1.67868086834727e-10\\
14.2410546875	-1.35540401773618e-10\\
14.26234375	-1.15373204848713e-10\\
14.2836328125	-6.81340392630775e-11\\
14.304921875	-1.0034643125967e-10\\
14.3262109375	-1.75491690166424e-10\\
14.3475	-2.82382506819509e-10\\
14.3687890625	-1.94309363368904e-10\\
14.390078125	-9.12285551003588e-11\\
14.4113671875	-1.75563534306998e-10\\
14.43265625	8.33764262744477e-11\\
14.4539453125	5.59267983186266e-11\\
14.475234375	1.0187434173131e-10\\
14.4965234375	2.22688421062169e-10\\
14.5178125	2.63351187314791e-10\\
14.5391015625	2.29023788693629e-10\\
14.560390625	1.08461253316664e-10\\
14.5816796875	7.0856069113369e-12\\
14.60296875	1.74529510742632e-10\\
14.6242578125	-1.20008603161534e-10\\
14.645546875	-4.59704816174068e-11\\
14.6668359375	-1.46051589546992e-11\\
14.688125	9.75635520566097e-11\\
14.7094140625	2.27364558461069e-10\\
14.730703125	3.36662810490296e-10\\
14.7519921875	2.84667700419961e-10\\
14.77328125	3.43967270950639e-10\\
14.7945703125	1.1155085549157e-10\\
14.815859375	-5.66260227402831e-12\\
14.8371484375	1.90066378611134e-11\\
14.8584375	-2.16456529549933e-10\\
14.8797265625	-2.56317555343479e-10\\
14.901015625	-2.23798918354164e-10\\
14.9223046875	1.12219388841475e-11\\
14.94359375	1.42140322122971e-10\\
14.9648828125	2.54351261060337e-10\\
14.986171875	2.79333562045557e-10\\
15.0074609375	1.74317641504569e-10\\
15.02875	5.76359760247247e-11\\
15.0500390625	4.80069544831458e-11\\
15.071328125	7.67851129790163e-11\\
15.0926171875	-4.05018039560027e-11\\
15.11390625	7.08385458045312e-11\\
15.1351953125	1.51142846349286e-10\\
15.156484375	2.12679510158087e-10\\
15.1777734375	2.25366939363345e-10\\
15.1990625	3.2659157276875e-10\\
15.2203515625	3.83940768320109e-10\\
15.241640625	2.76306599024185e-10\\
15.2629296875	2.55926671466602e-10\\
15.28421875	2.10071622442092e-10\\
15.3055078125	2.47424303911368e-10\\
15.326796875	1.68969517426654e-10\\
15.3480859375	2.59923290436867e-10\\
15.369375	1.71950728415094e-10\\
15.3906640625	1.72064030880604e-10\\
15.411953125	2.26612142396672e-10\\
15.4332421875	1.92031800590547e-10\\
15.45453125	2.31615241711511e-11\\
15.4758203125	2.16986110001409e-10\\
15.497109375	1.43057248215355e-10\\
15.5183984375	1.66336336511281e-10\\
15.5396875	3.57140319390579e-10\\
15.5609765625	4.1621707098808e-10\\
15.582265625	4.17586055362142e-10\\
15.6035546875	5.54336704188607e-10\\
15.62484375	4.7207993803448e-10\\
15.6461328125	4.00300324722451e-10\\
15.667421875	2.31319411559541e-10\\
15.6887109375	1.52315358817792e-10\\
15.71	9.05596089306727e-11\\
15.7312890625	1.87626947818087e-10\\
15.752578125	1.84209744500941e-10\\
15.7738671875	3.56478789166488e-10\\
15.79515625	3.66862114728402e-10\\
15.8164453125	4.74907518055177e-10\\
15.837734375	4.07303037990381e-10\\
15.8590234375	3.34093716803653e-10\\
15.8803125	3.42270538344779e-10\\
15.9016015625	2.22064962738596e-10\\
15.922890625	2.57054682182613e-10\\
15.9441796875	1.48418795986983e-10\\
15.96546875	1.73553816432363e-10\\
15.9867578125	1.39728471389752e-10\\
16.008046875	1.65416906182472e-10\\
16.0293359375	1.80111819881225e-10\\
16.050625	3.10978096107396e-10\\
16.0719140625	4.65969806991338e-10\\
16.093203125	4.01815619976877e-10\\
16.1144921875	5.46455753766501e-10\\
16.13578125	5.12051342133124e-10\\
16.1570703125	4.6958454917323e-10\\
16.178359375	3.87388156842104e-10\\
16.1996484375	2.18719070703845e-10\\
16.2209375	1.69309320149644e-10\\
16.2422265625	4.43304823810361e-11\\
16.263515625	3.03968303259523e-10\\
16.2848046875	1.91946305999233e-10\\
16.30609375	3.7460425034584e-10\\
16.3273828125	4.64753535274845e-10\\
16.348671875	4.77940677854824e-10\\
16.3699609375	5.85357593014815e-10\\
16.39125	4.71306895010799e-10\\
16.4125390625	2.79104694657168e-10\\
16.433828125	1.55525767355145e-10\\
16.4551171875	9.08625218024941e-11\\
16.47640625	6.1732917603607e-11\\
16.4976953125	3.07085704979022e-11\\
16.518984375	8.55343630139406e-11\\
16.5402734375	1.54733630486658e-10\\
16.5615625	2.92787389604917e-10\\
16.5828515625	2.79014155209947e-10\\
16.604140625	3.66296199500332e-10\\
16.6254296875	2.93152347668627e-10\\
16.64671875	1.00968548206226e-10\\
16.6680078125	-3.10592630948256e-11\\
16.689296875	-1.09731750330053e-10\\
16.7105859375	-1.36407143772466e-10\\
16.731875	-1.51197456335193e-10\\
16.7531640625	-1.01697081359386e-10\\
16.774453125	-1.67284705986588e-10\\
16.7957421875	4.20150223432996e-11\\
16.81703125	9.09466765045587e-11\\
16.8383203125	7.63654675330775e-11\\
16.859609375	1.5309308096094e-10\\
16.8808984375	8.55107219691432e-11\\
16.9021875	-1.51345534757065e-10\\
16.9234765625	-1.18901230711427e-10\\
16.944765625	-3.49553968420107e-10\\
16.9660546875	-3.67448700025772e-10\\
16.98734375	-4.73117886751419e-10\\
17.0086328125	-4.13889518316135e-10\\
17.029921875	-4.47372222669163e-10\\
17.0512109375	-4.7665964397166e-10\\
17.0725	-6.11025868115029e-10\\
17.0937890625	-5.712349372198e-10\\
17.115078125	-5.97901302640138e-10\\
17.1363671875	-4.23627463012828e-10\\
17.15765625	-4.37305683135427e-10\\
17.1789453125	-6.13262336517017e-10\\
17.200234375	-6.1626765072108e-10\\
17.2215234375	-7.74760366502729e-10\\
17.2428125	-6.61655909259843e-10\\
17.2641015625	-7.4605572044899e-10\\
17.285390625	-7.4987073280061e-10\\
17.3066796875	-6.7393746687368e-10\\
17.32796875	-6.36289699009198e-10\\
17.3492578125	-7.96037132900133e-10\\
17.370546875	-6.19173882275792e-10\\
17.3918359375	-5.65836005582561e-10\\
17.413125	-7.01208873812196e-10\\
17.4344140625	-6.95600043098566e-10\\
17.455703125	-5.66645094583681e-10\\
17.4769921875	-7.43673836977594e-10\\
17.49828125	-6.79516289442853e-10\\
17.5195703125	-6.24108051282057e-10\\
17.540859375	-6.6918869042267e-10\\
17.5621484375	-7.44594675238532e-10\\
17.5834375	-8.15121001652339e-10\\
17.6047265625	-8.43429058158747e-10\\
17.626015625	-9.18801817803682e-10\\
17.6473046875	-9.56188965078217e-10\\
17.66859375	-9.77479005525848e-10\\
17.6898828125	-9.0599719673162e-10\\
17.711171875	-9.54300787446579e-10\\
17.7324609375	-1.00882564939104e-09\\
17.75375	-9.6259209253336e-10\\
17.7750390625	-9.26491137920443e-10\\
17.796328125	-1.04997565929054e-09\\
17.8176171875	-8.98467289218872e-10\\
17.83890625	-9.2684083443522e-10\\
17.8601953125	-8.56209741929689e-10\\
17.881484375	-7.48583569078859e-10\\
17.9027734375	-6.1482625358518e-10\\
17.9240625	-8.63497507350047e-10\\
17.9453515625	-5.92984012436895e-10\\
17.966640625	-7.05954343545393e-10\\
17.9879296875	-6.66821916752642e-10\\
18.00921875	-7.08589436462821e-10\\
18.0305078125	-7.80388943151522e-10\\
18.051796875	-8.25976353207828e-10\\
18.0730859375	-7.49177921470883e-10\\
18.094375	-8.58241512246205e-10\\
18.1156640625	-8.44512486582708e-10\\
18.136953125	-7.41645472387418e-10\\
18.1582421875	-7.3850220101615e-10\\
18.17953125	-6.11125181429903e-10\\
18.2008203125	-5.31245036282256e-10\\
18.222109375	-5.57084530347926e-10\\
18.2433984375	-5.13265464041766e-10\\
18.2646875	-5.79815359575465e-10\\
18.2859765625	-5.84618811566001e-10\\
18.307265625	-6.2839578017577e-10\\
18.3285546875	-6.01356187131181e-10\\
18.34984375	-6.15591635397416e-10\\
18.3711328125	-4.99137411111711e-10\\
18.392421875	-6.74370019189549e-10\\
18.4137109375	-5.53019421930566e-10\\
18.435	-5.74106694376986e-10\\
18.4562890625	-5.44623855696547e-10\\
18.477578125	-5.94543885522555e-10\\
18.4988671875	-4.7075466109749e-10\\
18.52015625	-4.8227897029319e-10\\
18.5414453125	-4.35177281307749e-10\\
18.562734375	-5.82269588011885e-10\\
18.5840234375	-4.29296574299812e-10\\
18.6053125	-4.93326830307922e-10\\
18.6266015625	-6.16276038216893e-10\\
18.647890625	-5.83144330681251e-10\\
18.6691796875	-6.24268871527382e-10\\
18.69046875	-7.33170094058141e-10\\
18.7117578125	-5.71493660306998e-10\\
18.733046875	-5.07922619035429e-10\\
18.7543359375	-4.50187934385821e-10\\
18.775625	-5.44581806852917e-10\\
18.7969140625	-3.26705296818458e-10\\
18.818203125	-4.1306067681183e-10\\
18.8394921875	-3.33513732104909e-10\\
18.86078125	-2.74502080998563e-10\\
18.8820703125	-2.50683148796409e-10\\
18.903359375	-1.92901381781628e-10\\
18.9246484375	-2.70340759571009e-10\\
18.9459375	-2.64052331234805e-10\\
18.9672265625	-2.58427720027175e-10\\
18.988515625	-2.43213911107533e-10\\
19.0098046875	-1.88806613390428e-10\\
19.03109375	-8.68882558418282e-11\\
19.0523828125	-1.99235589085673e-10\\
19.073671875	-9.96104704478573e-11\\
19.0949609375	-1.05530587687344e-11\\
19.11625	8.13275955876341e-11\\
19.1375390625	1.02357709560505e-10\\
19.158828125	2.96659719612404e-10\\
19.1801171875	1.06629281812658e-10\\
19.20140625	1.54452449338217e-10\\
19.2226953125	2.06642711353415e-10\\
19.243984375	4.84012352400831e-11\\
19.2652734375	1.1990427592855e-11\\
19.2865625	5.17046141937796e-11\\
19.3078515625	2.95211910636377e-11\\
19.329140625	7.10472957839534e-11\\
19.3504296875	1.38290285659819e-10\\
19.37171875	2.26854549595499e-10\\
19.3930078125	3.13741988094871e-10\\
19.414296875	3.11848977833448e-10\\
19.4355859375	3.14370448664822e-10\\
19.456875	4.37189228242062e-10\\
19.4781640625	3.85963578937212e-10\\
19.499453125	1.8535397564866e-10\\
19.5207421875	1.89373767782541e-10\\
19.54203125	2.98134717877789e-10\\
19.5633203125	2.33276997010468e-10\\
19.584609375	4.55562755105508e-10\\
19.6058984375	3.32450012039843e-10\\
19.6271875	4.55542442569744e-10\\
19.6484765625	3.04818373651829e-10\\
19.669765625	3.3363963884811e-10\\
19.6910546875	2.12548835333278e-10\\
19.71234375	3.02743073477715e-10\\
19.7336328125	6.56081965574644e-11\\
19.754921875	1.22905387005228e-10\\
19.7762109375	2.51431595951473e-10\\
19.7975	4.1219102469686e-10\\
19.8187890625	4.16195689631092e-10\\
19.840078125	4.87900675113442e-10\\
19.8613671875	6.56389137524148e-10\\
19.88265625	3.25566351808102e-10\\
19.9039453125	4.59237594783711e-10\\
19.925234375	3.6039698668775e-10\\
19.9465234375	3.44018356171752e-10\\
19.9678125	1.41636740501456e-10\\
19.9891015625	2.02693700643967e-10\\
20.010390625	5.71518841451591e-11\\
20.0316796875	2.7326771217437e-10\\
20.05296875	1.204792239522e-10\\
20.0742578125	4.31524111269761e-10\\
20.095546875	3.26722021252313e-10\\
20.1168359375	3.90828379826812e-10\\
20.138125	2.82766268205194e-10\\
20.1594140625	4.5357770630258e-10\\
20.180703125	3.76632271031134e-10\\
20.2019921875	2.45500173319076e-10\\
20.22328125	4.07254291741511e-10\\
20.2445703125	3.78162984651363e-10\\
20.265859375	1.84638769265263e-10\\
20.2871484375	2.50171373341672e-10\\
20.3084375	3.0406662211278e-10\\
20.3297265625	2.02429200416831e-10\\
20.351015625	1.14002575511857e-10\\
20.3723046875	1.1296015645483e-10\\
20.39359375	1.31406624618823e-10\\
20.4148828125	2.16186256818215e-10\\
20.436171875	2.82805016645478e-10\\
20.4574609375	2.28642777946976e-10\\
20.47875	2.47132704897854e-10\\
20.5000390625	1.03128091869626e-10\\
20.521328125	-3.60283388583948e-11\\
20.5426171875	1.26190416865892e-11\\
20.56390625	-1.61434199981948e-10\\
20.5851953125	-1.35191857812452e-10\\
20.606484375	-4.38598615516121e-11\\
20.6277734375	-1.9604421081918e-11\\
20.6490625	-1.01064258194384e-11\\
20.6703515625	4.73486057092171e-11\\
20.691640625	-3.89528850367231e-11\\
20.7129296875	-7.20276784803274e-11\\
20.73421875	-1.49134210123352e-10\\
20.7555078125	-2.15359763878251e-10\\
20.776796875	-2.61076033679043e-10\\
20.7980859375	-3.66823922434237e-10\\
20.819375	-2.96664281254483e-10\\
20.8406640625	-3.14886485465665e-10\\
20.861953125	-3.70179007499171e-10\\
20.8832421875	-2.68307947470476e-10\\
20.90453125	-1.29301201364843e-10\\
20.9258203125	-2.5621902812582e-10\\
20.947109375	-7.55363017954187e-11\\
20.9683984375	-6.19389736127775e-11\\
20.9896875	-6.05708577647175e-12\\
21.0109765625	-1.92195057361417e-11\\
21.032265625	-2.91825893075882e-11\\
21.0535546875	-1.08384928428007e-10\\
21.07484375	9.1561692293681e-12\\
21.0961328125	-1.85964430984875e-10\\
21.117421875	-1.42433063039803e-10\\
21.1387109375	-1.60924984538475e-10\\
21.16	-2.35088711819828e-10\\
21.1812890625	-2.18092590179503e-10\\
21.202578125	-2.63841966513686e-10\\
21.2238671875	-2.24738692824314e-10\\
21.24515625	-1.14904321741964e-10\\
21.2664453125	-1.7223858373612e-10\\
21.287734375	-1.85664160080359e-10\\
21.3090234375	-1.8538402159487e-10\\
21.3303125	-1.79327590485773e-10\\
21.3516015625	-2.32180077400584e-10\\
21.372890625	-4.15991541183436e-10\\
21.3941796875	-4.00033051655002e-10\\
21.41546875	-4.12233802975071e-10\\
21.4367578125	-4.06844533082651e-10\\
21.458046875	-4.77975875259107e-10\\
21.4793359375	-2.97998251544656e-10\\
21.500625	-2.27262998829131e-10\\
21.5219140625	-1.71230874181003e-10\\
21.543203125	-1.78694112524336e-10\\
21.5644921875	-1.91366513737657e-10\\
21.58578125	-3.47187000005747e-10\\
21.6070703125	-2.80356794908447e-10\\
21.628359375	-3.64106676720851e-10\\
21.6496484375	-5.21241779366427e-10\\
21.6709375	-5.05883492462262e-10\\
21.6922265625	-4.62588534758541e-10\\
21.713515625	-5.93486619312263e-10\\
21.7348046875	-5.70978563102295e-10\\
21.75609375	-4.73629368832562e-10\\
21.7773828125	-4.61285972762007e-10\\
21.798671875	-4.77566416451115e-10\\
21.8199609375	-5.65335502394653e-10\\
21.84125	-5.72860592207927e-10\\
21.8625390625	-5.36066738230015e-10\\
21.883828125	-5.798679300093e-10\\
21.9051171875	-4.92903785911896e-10\\
21.92640625	-4.6444707936418e-10\\
21.9476953125	-4.18835148990193e-10\\
21.968984375	-5.63928937020541e-10\\
21.9902734375	-4.7742904279061e-10\\
22.0115625	-5.08947698695472e-10\\
22.0328515625	-4.50050257823457e-10\\
22.054140625	-5.92911416596657e-10\\
22.0754296875	-5.98063154167536e-10\\
22.09671875	-5.76056594205212e-10\\
22.1180078125	-6.3319472533185e-10\\
22.139296875	-7.85711996884787e-10\\
22.1605859375	-5.43502265530245e-10\\
22.181875	-6.17738574807754e-10\\
22.2031640625	-4.90809793080539e-10\\
22.224453125	-5.32827667060656e-10\\
22.2457421875	-3.20143054197333e-10\\
22.26703125	-3.30140022871736e-10\\
22.2883203125	-3.39647745415917e-10\\
22.309609375	-4.34545601633049e-10\\
22.3308984375	-4.19850225194359e-10\\
22.3521875	-7.06088778870691e-10\\
22.3734765625	-6.7769330127474e-10\\
22.394765625	-6.53885873283593e-10\\
22.4160546875	-7.51674244448011e-10\\
22.43734375	-6.73742480767222e-10\\
22.4586328125	-5.41192486621139e-10\\
22.479921875	-4.08165255865232e-10\\
22.5012109375	-3.85951558493754e-10\\
22.5225	-4.27813750620428e-10\\
22.5437890625	-3.87858635722575e-10\\
22.565078125	-3.94021104617776e-10\\
22.5863671875	-4.60637150589887e-10\\
22.60765625	-3.43323398820024e-10\\
22.6289453125	-4.26733411103813e-10\\
22.650234375	-3.18243998352654e-10\\
22.6715234375	-4.31500407861246e-10\\
22.6928125	-3.58119626841708e-10\\
22.7141015625	-3.21822312403555e-10\\
22.735390625	-4.54655243799451e-10\\
22.7566796875	-4.55270549932881e-10\\
22.77796875	-4.38728686103355e-10\\
22.7992578125	-3.62762604982264e-10\\
22.820546875	-3.39765548793166e-10\\
22.8418359375	-2.0771333625094e-10\\
22.863125	-1.50513742635491e-10\\
22.8844140625	-1.39939577294504e-10\\
22.905703125	-1.98287283854857e-10\\
22.9269921875	-2.67583692180996e-10\\
22.94828125	-2.46800956134799e-10\\
22.9695703125	-3.09274466843216e-10\\
22.990859375	-3.92843241226141e-10\\
23.0121484375	-2.26765863299396e-10\\
23.0334375	-2.48537544340347e-10\\
23.0547265625	-2.02687642332529e-10\\
23.076015625	-6.86870002671904e-11\\
23.0973046875	-4.06174369605616e-11\\
23.11859375	-1.00147203981723e-10\\
23.1398828125	-5.98109649857379e-11\\
23.161171875	5.49959681688345e-12\\
23.1824609375	-5.91957806518139e-11\\
23.20375	-2.94625795971075e-12\\
23.2250390625	1.23213300435211e-10\\
23.246328125	1.25597024245038e-10\\
23.2676171875	1.6796156342257e-10\\
23.28890625	2.57199554607515e-10\\
23.3101953125	2.07765893692284e-10\\
23.331484375	1.09755863355266e-10\\
23.3527734375	9.87637667061763e-11\\
23.3740625	1.71021109379637e-10\\
23.3953515625	1.80191565889856e-10\\
23.416640625	2.20326729642814e-10\\
23.4379296875	3.3114343411996e-10\\
23.45921875	2.9794403886097e-10\\
23.4805078125	4.32195987166538e-10\\
23.501796875	3.62097569503849e-10\\
23.5230859375	3.87786241578547e-10\\
23.544375	3.34809776100559e-10\\
23.5656640625	3.33766436246704e-10\\
23.586953125	1.59332158145087e-10\\
23.6082421875	2.91329105284716e-10\\
23.62953125	2.32813726402817e-10\\
23.6508203125	2.60776785001224e-10\\
23.672109375	2.07409303892131e-10\\
23.6933984375	2.76210164729179e-10\\
23.7146875	3.60580767207982e-10\\
23.7359765625	3.33508268558177e-10\\
23.757265625	2.39736663215031e-10\\
23.7785546875	4.40260979599285e-10\\
23.79984375	2.93259596979618e-10\\
23.8211328125	3.2015383537272e-10\\
23.842421875	2.73542778275842e-10\\
23.8637109375	3.06247746904059e-10\\
23.885	3.03022988605173e-10\\
23.9062890625	2.74154229671146e-10\\
23.927578125	4.11368239601441e-10\\
23.9488671875	3.38337982812797e-10\\
23.97015625	3.08812835517084e-10\\
23.9914453125	4.28088804724548e-10\\
24.012734375	4.44260411559382e-10\\
24.0340234375	4.20185700594336e-10\\
24.0553125	4.18728443899373e-10\\
24.0766015625	3.45186157885145e-10\\
24.097890625	5.32419653915152e-10\\
24.1191796875	4.22565775634292e-10\\
24.14046875	4.85180437861368e-10\\
24.1617578125	5.41583834896245e-10\\
24.183046875	4.7801075572e-10\\
24.2043359375	3.88462820981992e-10\\
24.225625	5.5665059262615e-10\\
24.2469140625	4.9479190687921e-10\\
24.268203125	4.83448137656942e-10\\
24.2894921875	5.34996063332583e-10\\
24.31078125	5.30227176630164e-10\\
24.3320703125	4.6655278481106e-10\\
24.353359375	5.50221714005182e-10\\
24.3746484375	4.78055078605718e-10\\
24.3959375	5.87134427253229e-10\\
24.4172265625	5.75823328370829e-10\\
24.438515625	6.19608762106059e-10\\
24.4598046875	5.3238448041523e-10\\
24.48109375	5.59903460894478e-10\\
24.5023828125	5.99749214903514e-10\\
24.523671875	6.04655112855235e-10\\
24.5449609375	5.77289205658099e-10\\
24.56625	5.73551808627878e-10\\
24.5875390625	6.70187679852663e-10\\
24.608828125	5.53372489176543e-10\\
24.6301171875	6.71684003171347e-10\\
24.65140625	6.14349688374719e-10\\
24.6726953125	5.82175481804197e-10\\
24.693984375	5.0686476429193e-10\\
24.7152734375	5.10227797834298e-10\\
24.7365625	5.20364346775287e-10\\
24.7578515625	4.07507859168598e-10\\
24.779140625	4.37831950460165e-10\\
24.8004296875	5.33834979621779e-10\\
24.82171875	3.88503371607531e-10\\
24.8430078125	4.01684290596818e-10\\
24.864296875	5.93727586012988e-10\\
24.8855859375	3.66059376040897e-10\\
24.906875	4.51327128380617e-10\\
24.9281640625	3.3407803870182e-10\\
24.949453125	2.4691273045107e-10\\
24.9707421875	2.44632380222782e-10\\
24.99203125	1.19564610384064e-10\\
25.0133203125	1.6879781003093e-10\\
25.034609375	2.35337568706465e-10\\
25.0558984375	9.54870065391588e-11\\
25.0771875	2.67953352710879e-10\\
25.0984765625	2.28700448281547e-10\\
25.119765625	2.45977841048371e-10\\
25.1410546875	2.17215005510519e-10\\
25.16234375	1.6139024155611e-10\\
25.1836328125	1.68203744255535e-10\\
25.204921875	6.30119779580467e-11\\
25.2262109375	4.46008672831023e-11\\
25.2475	1.72753725995198e-10\\
25.2687890625	1.30282095852202e-10\\
25.290078125	1.36756386341722e-10\\
25.3113671875	1.49259854299116e-10\\
25.33265625	2.51205081457195e-10\\
25.3539453125	8.22359722369527e-11\\
25.375234375	1.06690951137461e-10\\
25.3965234375	-1.49106271822334e-11\\
25.4178125	-7.20134469998308e-12\\
25.4391015625	-7.90845494399823e-11\\
25.460390625	-1.17720382736808e-11\\
25.4816796875	1.76214130459562e-10\\
25.50296875	1.99147663138806e-11\\
25.5242578125	1.17132759107758e-10\\
25.545546875	1.84754536393164e-10\\
25.5668359375	2.11431805238151e-10\\
25.588125	3.99656735839843e-12\\
25.6094140625	1.15776647901841e-10\\
25.630703125	4.05989834113848e-11\\
25.6519921875	-5.2025408576321e-11\\
25.67328125	-1.27320061785456e-10\\
25.6945703125	-9.80860214343306e-11\\
25.715859375	-1.42083124218289e-10\\
25.7371484375	-4.11556131126584e-11\\
25.7584375	-1.11703759308857e-10\\
25.7797265625	1.81096690225458e-11\\
25.801015625	-1.12411886610454e-12\\
25.8223046875	-1.38566179133179e-10\\
25.84359375	-6.67574520969663e-11\\
25.8648828125	-2.40487401927651e-11\\
25.886171875	-1.82276109743449e-10\\
25.9074609375	-2.17134844376544e-10\\
25.92875	-2.18329299844605e-10\\
25.9500390625	-2.9177232000055e-10\\
25.971328125	-2.20850759846231e-10\\
25.9926171875	-3.68482230629241e-10\\
26.01390625	-3.68858244381323e-10\\
26.0351953125	-4.16228340267806e-10\\
26.056484375	-4.17368891549091e-10\\
26.0777734375	-2.7808351217664e-10\\
26.0990625	-2.43845908184701e-10\\
26.1203515625	-3.70562005575521e-10\\
26.141640625	-1.81332397440176e-10\\
26.1629296875	-2.99353856476524e-10\\
26.18421875	-2.30418102255349e-10\\
26.2055078125	-1.85029341995942e-10\\
26.226796875	-2.62782574361159e-10\\
26.2480859375	-8.86861267141256e-11\\
26.269375	-1.52076616274208e-10\\
26.2906640625	-2.70710173599768e-10\\
26.311953125	-1.77164050780086e-10\\
26.3332421875	-2.51212567242383e-10\\
26.35453125	-2.77897297992108e-10\\
26.3758203125	-2.61358912613222e-10\\
26.397109375	-2.86131311029708e-10\\
26.4183984375	-2.73086062117656e-10\\
26.4396875	-3.11838371100408e-10\\
26.4609765625	-3.5625033025121e-10\\
26.482265625	-3.29049933360592e-10\\
26.5035546875	-3.69341969959435e-10\\
26.52484375	-2.07991458136288e-10\\
26.5461328125	-2.64121630627263e-10\\
26.567421875	-1.78064158432832e-10\\
26.5887109375	-2.39860443532075e-10\\
26.61	-1.95381774975339e-10\\
26.6312890625	-2.19238326845754e-10\\
26.652578125	-3.28448751978417e-10\\
26.6738671875	-2.40604666366865e-10\\
26.69515625	-3.08923772381805e-10\\
26.7164453125	-3.4519151707176e-10\\
26.737734375	-3.73631957645651e-10\\
26.7590234375	-2.85454896473819e-10\\
26.7803125	-4.00567039028285e-10\\
26.8016015625	-3.48927677098758e-10\\
26.822890625	-4.16677835202099e-10\\
26.8441796875	-2.95188429686572e-10\\
26.86546875	-3.97130120978007e-10\\
26.8867578125	-3.84980891622298e-10\\
26.908046875	-4.38255346506695e-10\\
26.9293359375	-3.68860679806665e-10\\
26.950625	-5.31949039556575e-10\\
26.9719140625	-6.29059307594549e-10\\
26.993203125	-6.06801378307e-10\\
27.0144921875	-7.00296821581632e-10\\
27.03578125	-6.75239333415966e-10\\
27.0570703125	-6.21459049499033e-10\\
27.078359375	-5.83803097808452e-10\\
27.0996484375	-6.37133720076893e-10\\
27.1209375	-5.11591834423817e-10\\
27.1422265625	-5.7398992472765e-10\\
27.163515625	-5.58759710237193e-10\\
27.1848046875	-5.54600033627805e-10\\
27.20609375	-6.57127594353385e-10\\
27.2273828125	-7.22421205522242e-10\\
27.248671875	-6.65888513464793e-10\\
27.2699609375	-5.34040177948345e-10\\
27.29125	-4.54806869624771e-10\\
27.3125390625	-5.64318803941405e-10\\
27.333828125	-4.41348703924603e-10\\
27.3551171875	-4.37473107106806e-10\\
27.37640625	-4.44916773074136e-10\\
27.3976953125	-3.68937199730154e-10\\
27.418984375	-4.90046426132577e-10\\
27.4402734375	-5.05927337189265e-10\\
27.4615625	-4.589817085908e-10\\
27.4828515625	-5.49060852551618e-10\\
27.504140625	-4.27452489607541e-10\\
27.5254296875	-4.28852717738228e-10\\
27.54671875	-4.51052364636539e-10\\
27.5680078125	-3.60244166791096e-10\\
27.589296875	-3.52692325242277e-10\\
27.6105859375	-3.47050859813744e-10\\
27.631875	-3.39858228825655e-10\\
27.6531640625	-4.06086612305e-10\\
27.674453125	-4.4718239503214e-10\\
27.6957421875	-4.67863601302087e-10\\
27.71703125	-3.91609494814529e-10\\
27.7383203125	-3.5516207846288e-10\\
27.759609375	-4.80018967543587e-10\\
27.7808984375	-3.01456731252436e-10\\
27.8021875	-3.42850241500008e-10\\
27.8234765625	-2.95267804090722e-10\\
27.844765625	-3.59766463076395e-10\\
27.8660546875	-4.09939453274011e-10\\
27.88734375	-2.97463837742614e-10\\
27.9086328125	-3.10862119468327e-10\\
27.929921875	-3.13334889821128e-10\\
27.9512109375	-2.204600320631e-10\\
27.9725	-2.91933416939466e-10\\
27.9937890625	-3.21854942274145e-10\\
28.015078125	-2.14425915006883e-10\\
28.0363671875	-2.57962071364026e-10\\
28.05765625	-2.24017751647452e-10\\
28.0789453125	-2.67121617720632e-10\\
28.100234375	-1.14955212645794e-10\\
28.1215234375	-1.79466410086256e-10\\
28.1428125	-9.47427869357388e-11\\
28.1641015625	-9.33403832582074e-11\\
28.185390625	-4.02190554468272e-11\\
28.2066796875	-7.19802163241432e-11\\
28.22796875	-7.04302385060008e-11\\
28.2492578125	1.94644113031019e-11\\
28.270546875	1.04075902073818e-11\\
28.2918359375	-6.34650592467331e-12\\
28.313125	-1.58203553687838e-11\\
28.3344140625	9.7388853947915e-11\\
28.355703125	7.06461778486347e-11\\
28.3769921875	1.14017356933918e-10\\
28.39828125	9.42788162894099e-11\\
28.4195703125	1.00745578582828e-10\\
28.440859375	1.21787934839332e-10\\
28.4621484375	1.39956662926474e-10\\
28.4834375	1.25872727281465e-10\\
28.5047265625	2.79590524344188e-10\\
28.526015625	1.4788386790205e-10\\
28.5473046875	2.55864209026836e-10\\
28.56859375	1.82021224637872e-10\\
28.5898828125	2.29654028111917e-10\\
28.611171875	2.07155165759246e-10\\
28.6324609375	2.16883444418952e-10\\
28.65375	2.08386971274216e-10\\
28.6750390625	2.51330898779052e-10\\
28.696328125	1.97559183892624e-10\\
28.7176171875	3.49555817078067e-10\\
28.73890625	2.42065825651307e-10\\
28.7601953125	2.25838417284177e-10\\
28.781484375	3.31345787033401e-10\\
28.8027734375	2.5724147915194e-10\\
28.8240625	1.91659278704967e-10\\
28.8453515625	2.47751381817934e-10\\
28.866640625	1.99948370588524e-10\\
28.8879296875	1.33778340822019e-10\\
28.90921875	1.56985153165563e-10\\
28.9305078125	2.44591013119932e-10\\
28.951796875	1.63339199836779e-10\\
28.9730859375	3.18570575202482e-10\\
28.994375	4.48725221972655e-10\\
29.0156640625	3.49468301226412e-10\\
29.036953125	3.40424348320363e-10\\
29.0582421875	3.22858720951941e-10\\
29.07953125	1.78917589367269e-10\\
29.1008203125	1.7924940006955e-10\\
29.122109375	1.84189414391667e-10\\
29.1433984375	2.94305693199122e-10\\
29.1646875	4.15736416652098e-10\\
29.1859765625	3.7022642308542e-10\\
29.207265625	5.58309166843532e-10\\
29.2285546875	6.94162839899795e-10\\
29.24984375	6.87176454969534e-10\\
29.2711328125	6.14369774360634e-10\\
29.292421875	7.2344318615077e-10\\
29.3137109375	6.26511411818262e-10\\
29.335	6.28943473806272e-10\\
29.3562890625	6.79114701437848e-10\\
29.377578125	6.72315149840231e-10\\
29.3988671875	6.52873328774487e-10\\
29.42015625	7.63536233457678e-10\\
29.4414453125	7.0861635119142e-10\\
29.462734375	8.2368662555686e-10\\
29.4840234375	7.63023249593939e-10\\
29.5053125	6.98184107261025e-10\\
29.5266015625	7.52323246083644e-10\\
29.547890625	6.58945000369617e-10\\
29.5691796875	6.16151357883979e-10\\
29.59046875	6.53305393038625e-10\\
29.6117578125	5.93760007245978e-10\\
29.633046875	6.58597658438641e-10\\
29.6543359375	7.00483504740278e-10\\
29.675625	7.63334796683429e-10\\
29.6969140625	7.48863781816071e-10\\
29.718203125	7.6311145596526e-10\\
29.7394921875	6.45133362530626e-10\\
29.76078125	6.08126683609173e-10\\
29.7820703125	4.58416958677814e-10\\
29.803359375	4.49659247037792e-10\\
29.8246484375	4.85874778175361e-10\\
29.8459375	3.30514893673428e-10\\
29.8672265625	4.04272578600999e-10\\
29.888515625	4.04228553067759e-10\\
29.9098046875	4.76942156593233e-10\\
29.93109375	5.75840281915306e-10\\
29.9523828125	5.23020940162123e-10\\
29.973671875	5.55451648002055e-10\\
29.9949609375	5.89446180576414e-10\\
30.01625	4.73823169603852e-10\\
30.0375390625	5.09620193139115e-10\\
30.058828125	5.239987027173e-10\\
30.0801171875	3.6907227907383e-10\\
30.10140625	4.3537555029946e-10\\
30.1226953125	4.31983720085155e-10\\
30.143984375	4.65981653647784e-10\\
30.1652734375	5.79641993711288e-10\\
30.1865625	5.94157041686545e-10\\
30.2078515625	7.74371667408452e-10\\
30.229140625	6.67740563619989e-10\\
30.2504296875	6.43021134955599e-10\\
30.27171875	7.28594129604627e-10\\
30.2930078125	5.63440010530526e-10\\
30.314296875	5.35297826137714e-10\\
30.3355859375	5.28856832770604e-10\\
30.356875	6.51975986306206e-10\\
30.3781640625	6.40105137696654e-10\\
30.399453125	6.07025829450033e-10\\
30.4207421875	5.54084043504454e-10\\
30.44203125	6.57472279965811e-10\\
30.4633203125	5.41405113582183e-10\\
30.484609375	5.30821282197883e-10\\
30.5058984375	4.8954319211909e-10\\
30.5271875	3.66497961264409e-10\\
30.5484765625	3.09759107021763e-10\\
30.569765625	4.48222294240301e-10\\
30.5910546875	3.46336991803763e-10\\
30.61234375	3.59632612028264e-10\\
30.6336328125	3.51958101292239e-10\\
30.654921875	3.333160963311e-10\\
30.6762109375	3.52545177396329e-10\\
30.6975	3.04084051307654e-10\\
30.7187890625	2.48902124104587e-10\\
30.740078125	1.84728053599363e-10\\
30.7613671875	1.15727352658401e-10\\
30.78265625	1.26796514767682e-10\\
30.8039453125	1.34829661833198e-10\\
30.825234375	3.86274965917848e-11\\
30.8465234375	-2.34409902123508e-11\\
30.8678125	1.33030105700261e-10\\
30.8891015625	4.09045832100454e-11\\
30.910390625	3.60732635415165e-11\\
30.9316796875	7.5519594828943e-11\\
30.95296875	1.10242377124951e-11\\
30.9742578125	-7.47175510105412e-11\\
30.995546875	-1.16142881681243e-10\\
31.0168359375	-5.45770405406356e-12\\
31.038125	9.00536963242957e-11\\
31.0594140625	-6.03852052130897e-11\\
31.080703125	2.88282157907034e-11\\
31.1019921875	-2.77737287628341e-11\\
31.12328125	-1.15822618448961e-10\\
31.1445703125	-2.71451437567825e-11\\
31.165859375	-1.63206592895569e-10\\
31.1871484375	-8.56696575073756e-11\\
31.2084375	-2.04427447753859e-10\\
31.2297265625	-2.11143940356056e-10\\
31.251015625	-1.51496456172505e-10\\
31.2723046875	-4.74738477796639e-11\\
31.29359375	-8.28097451670329e-11\\
31.3148828125	1.44987903219942e-12\\
31.336171875	5.7071712278092e-11\\
31.3574609375	4.77062317458401e-11\\
31.37875	4.2187645288985e-11\\
31.4000390625	-9.51377539983219e-11\\
31.421328125	-2.52509150373682e-11\\
31.4426171875	-8.13529999112685e-11\\
31.46390625	-1.28494442676832e-10\\
31.4851953125	-1.06383042272566e-10\\
31.506484375	-9.58684796621902e-11\\
31.5277734375	-3.54305177776232e-11\\
31.5490625	-1.11662835206352e-10\\
31.5703515625	-1.64528605970649e-10\\
31.591640625	-1.06681707207727e-10\\
31.6129296875	-1.80380998011691e-10\\
31.63421875	-1.72167973830661e-10\\
31.6555078125	-3.03732564640427e-10\\
31.676796875	-2.30863201659405e-10\\
31.6980859375	-2.99327674908593e-10\\
31.719375	-3.80753506318828e-10\\
31.7406640625	-3.37480292462639e-10\\
31.761953125	-3.1277522449536e-10\\
31.7832421875	-3.79954328666274e-10\\
31.80453125	-4.61590771216678e-10\\
31.8258203125	-4.18654862614316e-10\\
31.847109375	-5.413219517867e-10\\
31.8683984375	-5.40768858275717e-10\\
31.8896875	-3.44328597230181e-10\\
31.9109765625	-4.81587827014556e-10\\
31.932265625	-3.83329757837276e-10\\
31.9535546875	-3.44701308748023e-10\\
31.97484375	-3.99041524775317e-10\\
31.9961328125	-3.41182173381762e-10\\
32.017421875	-4.2772443508782e-10\\
32.0387109375	-4.82348058907236e-10\\
32.06	-6.08622475448091e-10\\
32.0812890625	-6.06609181809094e-10\\
32.102578125	-6.10938033463392e-10\\
32.1238671875	-6.05846502538455e-10\\
32.14515625	-5.98315603504086e-10\\
32.1664453125	-6.18830163460629e-10\\
32.187734375	-5.19193585385835e-10\\
32.2090234375	-6.50515803330683e-10\\
32.2303125	-5.07242091885875e-10\\
32.2516015625	-5.55369646778158e-10\\
32.272890625	-5.93333341958272e-10\\
32.2941796875	-5.5551677186017e-10\\
32.31546875	-4.93783909341499e-10\\
32.3367578125	-4.86041087105805e-10\\
32.358046875	-3.62964407872826e-10\\
32.3793359375	-4.49246892124565e-10\\
32.400625	-3.98499316160212e-10\\
32.4219140625	-4.69377749611567e-10\\
32.443203125	-3.92726721975596e-10\\
32.4644921875	-4.06308292031264e-10\\
32.48578125	-4.63615554152412e-10\\
32.5070703125	-4.02913109519366e-10\\
32.528359375	-4.06251891061028e-10\\
32.5496484375	-5.11186205350948e-10\\
32.5709375	-4.7924815283535e-10\\
32.5922265625	-5.05238297557636e-10\\
32.613515625	-4.22224531814138e-10\\
32.6348046875	-4.93971007633149e-10\\
32.65609375	-4.1890724993712e-10\\
32.6773828125	-4.43447475850842e-10\\
32.698671875	-4.92571800156842e-10\\
32.7199609375	-4.59032447321092e-10\\
32.74125	-4.38383274875572e-10\\
32.7625390625	-5.24807744765393e-10\\
32.783828125	-4.86654404442693e-10\\
32.8051171875	-4.46841252921967e-10\\
32.82640625	-5.14687940538374e-10\\
32.8476953125	-4.4146076408121e-10\\
32.868984375	-5.12532401750966e-10\\
32.8902734375	-4.41582306538467e-10\\
32.9115625	-3.02119417610295e-10\\
32.9328515625	-3.33857548161153e-10\\
32.954140625	-3.58643802377469e-10\\
32.9754296875	-3.09925602615818e-10\\
32.99671875	-3.35623216535785e-10\\
33.0180078125	-3.20919648231622e-10\\
33.039296875	-4.28625002469522e-10\\
33.0605859375	-3.74916474566707e-10\\
33.081875	-4.29204660232834e-10\\
33.1031640625	-3.057912519107e-10\\
33.124453125	-2.87483456480295e-10\\
33.1457421875	-1.90598534906418e-10\\
33.16703125	-1.88662941655161e-10\\
33.1883203125	-1.32502789324747e-10\\
33.209609375	-3.33477836842579e-11\\
33.2308984375	-9.28293884019544e-11\\
33.2521875	-1.20558695735544e-10\\
33.2734765625	-9.17729238865892e-11\\
33.294765625	-2.32800305728662e-11\\
33.3160546875	-1.2717176455252e-11\\
33.33734375	3.90138219332053e-12\\
33.3586328125	-7.1300062915412e-12\\
33.379921875	1.12958990073532e-10\\
33.4012109375	9.81123279793798e-11\\
33.4225	1.38785906930363e-10\\
33.4437890625	1.99547009251409e-10\\
33.465078125	1.31478042888102e-10\\
33.4863671875	1.78729116536502e-10\\
33.50765625	1.73238574938949e-10\\
33.5289453125	1.65022945444536e-10\\
33.550234375	2.1293635523776e-10\\
33.5715234375	3.02370459405734e-10\\
33.5928125	2.20229123689032e-10\\
33.6141015625	3.94505485156638e-10\\
33.635390625	3.28062181986693e-10\\
33.6566796875	3.66707364810235e-10\\
33.67796875	3.96397652809228e-10\\
33.6992578125	4.1213935995411e-10\\
33.720546875	4.20615122133342e-10\\
33.7418359375	4.8066686530513e-10\\
33.763125	3.17697960017258e-10\\
33.7844140625	3.20628271068746e-10\\
33.805703125	4.02363640022917e-10\\
33.8269921875	4.11392262040218e-10\\
33.84828125	5.72862264040159e-10\\
33.8695703125	4.76463198246964e-10\\
33.890859375	5.98935095356338e-10\\
33.9121484375	6.27745983336535e-10\\
33.9334375	6.17305946277924e-10\\
33.9547265625	5.51822477489751e-10\\
33.976015625	6.63836777083842e-10\\
33.9973046875	5.42682725495721e-10\\
34.01859375	5.27702726126849e-10\\
34.0398828125	5.51318301785452e-10\\
34.061171875	6.22159118982606e-10\\
34.0824609375	6.49682885197992e-10\\
34.10375	6.48024022764941e-10\\
34.1250390625	7.25194346963012e-10\\
34.146328125	7.21610323060125e-10\\
34.1676171875	7.23735864966746e-10\\
34.18890625	6.75761635772581e-10\\
34.2101953125	7.22678144435686e-10\\
34.231484375	7.64755879187265e-10\\
34.2527734375	7.56914100487274e-10\\
34.2740625	7.6333874208315e-10\\
34.2953515625	7.95837696128552e-10\\
34.316640625	9.26154306338209e-10\\
34.3379296875	7.81039767920648e-10\\
34.35921875	8.46823270518428e-10\\
34.3805078125	7.86475896788643e-10\\
34.401796875	7.51777820061122e-10\\
34.4230859375	7.50416427854727e-10\\
34.444375	7.29665642000456e-10\\
34.4656640625	6.76413344546108e-10\\
34.486953125	8.10343747521998e-10\\
34.5082421875	7.13943169745287e-10\\
34.52953125	9.28854065976921e-10\\
34.5508203125	8.71051902416254e-10\\
34.572109375	8.60045218230228e-10\\
34.5933984375	8.54281354124545e-10\\
34.6146875	8.13371198933648e-10\\
34.6359765625	7.25790487232271e-10\\
34.657265625	7.36975192514233e-10\\
34.6785546875	6.68623529819919e-10\\
34.69984375	6.78793643174012e-10\\
34.7211328125	6.76326595916279e-10\\
34.742421875	7.70021979878399e-10\\
34.7637109375	7.92975506824461e-10\\
34.785	8.10794833122016e-10\\
34.8062890625	7.25894822633615e-10\\
34.827578125	7.22007421972462e-10\\
34.8488671875	6.49186776014059e-10\\
34.87015625	6.42448242989893e-10\\
34.8914453125	6.48708348092485e-10\\
34.912734375	6.51218411513334e-10\\
34.9340234375	7.0341337528284e-10\\
34.9553125	5.84839397472281e-10\\
34.9766015625	7.1933130108976e-10\\
34.997890625	6.24580029191542e-10\\
35.0191796875	6.75052463374945e-10\\
35.04046875	7.06368319296163e-10\\
35.0617578125	7.12228978660762e-10\\
35.083046875	7.26512299129192e-10\\
35.1043359375	5.8928065692812e-10\\
35.125625	6.51488242034803e-10\\
35.1469140625	6.02357596944674e-10\\
35.168203125	6.14249137786847e-10\\
35.1894921875	6.13582536811995e-10\\
35.21078125	5.63688477805635e-10\\
35.2320703125	6.53603319319296e-10\\
35.253359375	6.54443221564287e-10\\
35.2746484375	6.72219669396968e-10\\
35.2959375	5.82282749876025e-10\\
35.3172265625	5.35401372581181e-10\\
35.338515625	4.54376061224651e-10\\
35.3598046875	3.48053128039787e-10\\
35.38109375	3.19764809058356e-10\\
35.4023828125	2.78020430791725e-10\\
35.423671875	2.7821393618926e-10\\
35.4449609375	3.57005979006244e-10\\
35.46625	2.97133362668286e-10\\
35.4875390625	3.1543836455761e-10\\
35.508828125	2.76266393679298e-10\\
35.5301171875	3.43220045712352e-10\\
35.55140625	3.5048985451832e-10\\
35.5726953125	3.03138800185108e-10\\
35.593984375	2.89202942348392e-10\\
35.6152734375	1.27769240694474e-10\\
35.6365625	7.7291840413062e-11\\
35.6578515625	3.88651041972951e-12\\
35.679140625	9.42249591534447e-11\\
35.7004296875	3.89247255247901e-11\\
35.72171875	2.43629040180982e-11\\
35.7430078125	3.94534941945384e-11\\
35.764296875	1.00773131054491e-10\\
35.7855859375	3.46191889637246e-11\\
35.806875	5.15398400846866e-11\\
35.8281640625	1.41515092234036e-11\\
35.849453125	-4.44618237901883e-11\\
35.8707421875	-9.62946766698364e-11\\
35.89203125	-1.74180195011137e-10\\
35.9133203125	-1.19235498079784e-10\\
35.934609375	-1.38042669550184e-10\\
35.9558984375	-1.69177236710772e-10\\
35.9771875	-8.51953245388116e-14\\
35.9984765625	-9.6897121764156e-11\\
36.019765625	2.1848323583762e-12\\
36.0410546875	-4.58810165322502e-11\\
36.06234375	-1.18836779714895e-10\\
36.0836328125	-1.2016029894903e-10\\
36.104921875	-5.89894170970087e-11\\
36.1262109375	-1.71003725372059e-10\\
36.1475	-8.45551630265755e-11\\
36.1687890625	-1.44138259102278e-10\\
36.190078125	-8.14748156394487e-11\\
36.2113671875	-4.94231283830443e-11\\
36.23265625	-1.79542024763026e-10\\
36.2539453125	-1.14805412807565e-10\\
36.275234375	-1.01685579294696e-10\\
36.2965234375	-1.43642405941116e-10\\
36.3178125	-1.7607180074746e-10\\
36.3391015625	-3.35244922545156e-10\\
36.360390625	-3.70074788610987e-10\\
36.3816796875	-3.18234795858477e-10\\
36.40296875	-3.49027062408565e-10\\
36.4242578125	-2.59381898855657e-10\\
36.445546875	-3.09894418242238e-10\\
36.4668359375	-2.32673362632676e-10\\
36.488125	-2.30206678376575e-10\\
36.5094140625	-1.86252627652367e-10\\
36.530703125	-1.78116217637057e-10\\
36.5519921875	-3.29719267612125e-10\\
36.57328125	-3.4336086296105e-10\\
36.5945703125	-3.46918708769335e-10\\
36.615859375	-4.78780803710583e-10\\
36.6371484375	-3.47200294044361e-10\\
36.6584375	-3.72039437806909e-10\\
36.6797265625	-3.82828177829642e-10\\
36.701015625	-3.06106619152447e-10\\
36.7223046875	-4.48669675561141e-10\\
36.74359375	-4.00938645624014e-10\\
36.7648828125	-4.39078516891499e-10\\
36.786171875	-4.0778305893512e-10\\
36.8074609375	-4.2669100008111e-10\\
36.82875	-3.31194523129707e-10\\
36.8500390625	-3.33554278396029e-10\\
36.871328125	-2.92053649399977e-10\\
36.8926171875	-3.22236193103536e-10\\
36.91390625	-3.56258548351506e-10\\
36.9351953125	-3.96716919256037e-10\\
36.956484375	-2.58469815254741e-10\\
36.9777734375	-3.94989708792995e-10\\
36.9990625	-3.1199458296724e-10\\
37.0203515625	-2.91248148619366e-10\\
37.041640625	-2.45167420192197e-10\\
37.0629296875	-2.7080057075079e-10\\
37.08421875	-1.67728099090568e-10\\
37.1055078125	-2.28150735390634e-10\\
37.126796875	-1.26282572268362e-10\\
37.1480859375	-1.88970789561177e-10\\
37.169375	-2.93955594356531e-10\\
37.1906640625	-1.80844451545025e-10\\
37.211953125	-2.35701170386706e-10\\
37.2332421875	-1.9237657061289e-10\\
37.25453125	-1.74619708619198e-10\\
37.2758203125	-1.71954472290336e-10\\
37.297109375	-2.16495328574309e-10\\
37.3183984375	-1.21407976468023e-10\\
37.3396875	-1.16838859832664e-10\\
37.3609765625	-1.42570421250398e-10\\
37.382265625	-1.33952907158366e-10\\
37.4035546875	-1.95904821467286e-10\\
37.42484375	-1.69177631780972e-10\\
37.4461328125	-1.75847357846584e-10\\
37.467421875	-1.81549305278876e-10\\
37.4887109375	-1.95013229209417e-10\\
37.51	-1.86090237115327e-10\\
37.5312890625	-2.11686421952333e-10\\
37.552578125	-7.07608071011428e-11\\
37.5738671875	-2.2155416130692e-10\\
37.59515625	-1.14952053603227e-10\\
37.6164453125	-1.47496927695399e-10\\
37.637734375	-1.63130757892233e-10\\
};
\addplot [color=mycolor2,solid]
  table[row sep=crcr]{%
37.637734375	-1.63130757892233e-10\\
37.6590234375	-7.97369229213984e-11\\
37.6803125	-6.22906475659192e-11\\
37.7016015625	-9.53500846524805e-11\\
37.722890625	-4.88999834670683e-11\\
37.7441796875	-1.58669468371525e-10\\
37.76546875	-1.39375055615487e-10\\
37.7867578125	-1.49420998803662e-10\\
37.808046875	-9.31211520617467e-11\\
37.8293359375	-8.16356283486601e-11\\
37.850625	-2.08110776668832e-11\\
37.8719140625	2.63632447276991e-11\\
37.893203125	1.42158671568527e-11\\
37.9144921875	1.65306032717386e-11\\
37.93578125	3.38142640810424e-11\\
37.9570703125	-1.19728244848439e-10\\
37.978359375	-5.0307750176837e-12\\
37.9996484375	-4.38210986775126e-12\\
38.0209375	4.67295832293252e-11\\
38.0422265625	-1.36310618358838e-11\\
38.063515625	-4.2962650457281e-11\\
38.0848046875	4.57712679004815e-11\\
38.10609375	7.86912511256517e-11\\
38.1273828125	5.13736013596312e-11\\
38.148671875	1.17706970389e-10\\
38.1699609375	3.10306584386187e-10\\
38.19125	1.02620788676245e-10\\
38.2125390625	2.33484225897024e-10\\
38.233828125	2.73615512083053e-10\\
38.2551171875	1.23973707908288e-10\\
38.27640625	1.77557526740296e-10\\
38.2976953125	2.06557818886609e-10\\
38.318984375	2.25903569233824e-10\\
38.3402734375	2.93155285987556e-10\\
38.3615625	2.59212870324516e-10\\
38.3828515625	4.71860788314101e-10\\
38.404140625	3.81390473839804e-10\\
38.4254296875	3.29720778567369e-10\\
38.44671875	3.61784877375997e-10\\
38.4680078125	3.12965110017868e-10\\
38.489296875	2.76711900322098e-10\\
38.5105859375	1.66472549061202e-10\\
38.531875	2.63906452126572e-10\\
38.5531640625	2.85583779750312e-10\\
38.574453125	2.64677564712546e-10\\
38.5957421875	3.57276440537438e-10\\
38.61703125	2.68138889188776e-10\\
38.6383203125	3.52577792389367e-10\\
38.659609375	2.62648192583584e-10\\
38.6808984375	2.82489937890508e-10\\
38.7021875	2.36417637119916e-10\\
38.7234765625	2.13462997530504e-10\\
38.744765625	2.99860903628542e-10\\
38.7660546875	2.82221305891152e-10\\
38.78734375	3.27453159355435e-10\\
38.8086328125	3.09073823441143e-10\\
38.829921875	2.96458299950243e-10\\
38.8512109375	3.23960825329706e-10\\
38.8725	3.6869662033154e-10\\
38.8937890625	3.13734832671965e-10\\
38.915078125	4.77505204829946e-10\\
38.9363671875	3.3370278750793e-10\\
38.95765625	3.94162161168433e-10\\
38.9789453125	2.79576402074904e-10\\
39.000234375	4.13088434254734e-10\\
39.0215234375	3.36547355966868e-10\\
39.0428125	3.93478802199081e-10\\
39.0641015625	3.66020665899719e-10\\
39.085390625	4.00491895445995e-10\\
39.1066796875	2.60226453798224e-10\\
39.12796875	3.98707131718854e-10\\
39.1492578125	3.14598031137777e-10\\
39.170546875	4.52995411018857e-10\\
39.1918359375	3.97196057974403e-10\\
39.213125	4.05895368463624e-10\\
39.2344140625	5.22222405737362e-10\\
39.255703125	4.81387441768823e-10\\
39.2769921875	4.48750501952441e-10\\
39.29828125	5.23349236767799e-10\\
39.3195703125	4.27992804405759e-10\\
39.340859375	3.56886042643433e-10\\
39.3621484375	3.87824400108956e-10\\
39.3834375	4.56282158232682e-10\\
39.4047265625	4.62809528270676e-10\\
39.426015625	4.82299187391733e-10\\
39.4473046875	4.64968117008089e-10\\
39.46859375	3.96996992416711e-10\\
39.4898828125	4.56582976774841e-10\\
39.511171875	2.73182972800849e-10\\
39.5324609375	3.1023609056966e-10\\
39.55375	3.25874416841148e-10\\
39.5750390625	3.12949646608642e-10\\
39.596328125	2.80744978288389e-10\\
39.6176171875	3.40838426636104e-10\\
39.63890625	2.53331539158334e-10\\
39.6601953125	3.66037221743322e-10\\
39.681484375	3.00720246904264e-10\\
39.7027734375	2.34036077097136e-10\\
39.7240625	2.35522824887897e-10\\
39.7453515625	2.15531861245647e-10\\
39.766640625	2.14317401722594e-10\\
39.7879296875	2.06980350855335e-10\\
39.80921875	2.37854234365569e-10\\
39.8305078125	1.29273076562037e-10\\
39.851796875	1.63322056739526e-10\\
39.8730859375	1.09205002736893e-10\\
39.894375	1.2308316200827e-10\\
39.9156640625	9.69543542118561e-11\\
39.936953125	7.80600980916793e-11\\
39.9582421875	1.55132191058772e-10\\
39.97953125	2.08027280682556e-10\\
40.0008203125	1.70440970196774e-10\\
40.022109375	2.28549284948631e-10\\
40.0433984375	2.22990366624807e-10\\
40.0646875	6.83722405024653e-11\\
40.0859765625	1.06514138295528e-10\\
40.107265625	2.22896560262324e-11\\
40.1285546875	4.62996801050291e-11\\
40.14984375	-3.49000502103533e-11\\
40.1711328125	2.7618847813661e-11\\
40.192421875	4.32666934508127e-11\\
40.2137109375	-5.15854130815229e-12\\
40.235	2.77323319662279e-11\\
40.2562890625	9.38944989640874e-11\\
40.277578125	8.15163759751642e-11\\
40.2988671875	4.44085857769328e-11\\
40.32015625	1.21385582402333e-10\\
40.3414453125	2.32067638089413e-11\\
40.362734375	6.71366544462286e-11\\
40.3840234375	-4.35410885185693e-11\\
40.4053125	-5.8468779688357e-11\\
40.4266015625	-4.14181588284614e-11\\
40.447890625	-3.19987195708185e-11\\
40.4691796875	-3.33818323758516e-12\\
40.49046875	2.10506173650382e-11\\
40.5117578125	-6.64941279623189e-11\\
40.533046875	-8.41342875660549e-11\\
40.5543359375	-9.23127219613598e-12\\
40.575625	2.04557072658432e-11\\
40.5969140625	1.54993368914405e-11\\
40.618203125	-2.33028155467644e-11\\
40.6394921875	-5.33517384429466e-11\\
40.66078125	-1.37840642056005e-10\\
40.6820703125	-1.7314693284749e-10\\
40.703359375	-1.41396830276317e-10\\
40.7246484375	-2.46911437075665e-10\\
40.7459375	-1.89756214022713e-10\\
40.7672265625	-1.43121619236507e-10\\
40.788515625	-9.10968996174183e-11\\
40.8098046875	-1.23194532017647e-10\\
40.83109375	-1.17466365445562e-10\\
40.8523828125	-9.89985636603439e-11\\
40.873671875	-1.64275769972076e-10\\
40.8949609375	-3.02291150644741e-10\\
40.91625	-1.94845462543462e-10\\
40.9375390625	-3.21972327292815e-10\\
40.958828125	-2.99525732666375e-10\\
40.9801171875	-2.76948989013282e-10\\
41.00140625	-1.51409597796768e-10\\
41.0226953125	-2.64297767293953e-10\\
41.043984375	-2.97072257714681e-10\\
41.0652734375	-2.94726786769071e-10\\
41.0865625	-3.03256351039257e-10\\
41.1078515625	-3.57245086184705e-10\\
41.129140625	-3.52088553916593e-10\\
41.1504296875	-3.2026915734034e-10\\
41.17171875	-2.8240475785279e-10\\
41.1930078125	-3.44250193400328e-10\\
41.214296875	-2.05960743032826e-10\\
41.2355859375	-2.81225527618874e-10\\
41.256875	-2.61287768965749e-10\\
41.2781640625	-2.36575430544425e-10\\
41.299453125	-1.91720328696052e-10\\
41.3207421875	-2.43919098180219e-10\\
41.34203125	-2.21730477814158e-10\\
41.3633203125	-2.00144102378841e-10\\
41.384609375	-1.45258977235244e-10\\
41.4058984375	-1.80881423990357e-10\\
41.4271875	-2.48143426480171e-10\\
41.4484765625	-1.84704938626927e-10\\
41.469765625	-1.76891807712032e-10\\
41.4910546875	-2.05019491780572e-10\\
41.51234375	-2.0944833590965e-10\\
41.5336328125	-1.07365465214741e-10\\
41.554921875	-1.84414026142276e-10\\
41.5762109375	-6.35473963379051e-11\\
41.5975	-2.3077093364011e-10\\
41.6187890625	-1.24290802435636e-10\\
41.640078125	-1.88627146826489e-10\\
41.6613671875	-1.45768150327051e-10\\
41.68265625	-2.6709516410181e-10\\
41.7039453125	-1.77405225930663e-10\\
41.725234375	-2.69516955924317e-10\\
41.7465234375	-2.31176709367705e-10\\
41.7678125	-2.19211175549563e-10\\
41.7891015625	-2.55642505471179e-10\\
41.810390625	-2.47912976650182e-10\\
41.8316796875	-2.2309297890334e-10\\
41.85296875	-2.65997002002246e-10\\
41.8742578125	-2.11001298766294e-10\\
41.895546875	-2.00765315364177e-10\\
41.9168359375	-2.76262275781019e-10\\
41.938125	-3.08192935408065e-10\\
41.9594140625	-2.72902908793931e-10\\
41.980703125	-2.47145637079365e-10\\
42.0019921875	-3.6545773494727e-10\\
42.02328125	-2.24393649450119e-10\\
42.0445703125	-3.09800054129895e-10\\
42.065859375	-2.18106955538812e-10\\
42.0871484375	-3.00466156784022e-10\\
42.1084375	-2.77346906322737e-10\\
42.1297265625	-2.83201912583962e-10\\
42.151015625	-2.80854761586114e-10\\
42.1723046875	-2.27166423323373e-10\\
42.19359375	-1.70591443138837e-10\\
42.2148828125	-2.25285583627496e-10\\
42.236171875	-1.58211217754603e-10\\
42.2574609375	-1.73877472431097e-10\\
42.27875	-1.94320355939359e-10\\
42.3000390625	-2.53361373113635e-10\\
42.321328125	-2.99989584264003e-10\\
42.3426171875	-2.40435498013598e-10\\
42.36390625	-2.2766661223939e-10\\
42.3851953125	-1.86851314875757e-10\\
42.406484375	-1.59682191753286e-10\\
42.4277734375	-9.55125726208351e-11\\
42.4490625	-9.72570501065853e-11\\
42.4703515625	-1.15078907354812e-10\\
42.491640625	-7.00913679649075e-11\\
42.5129296875	-9.21417512620867e-11\\
42.53421875	-1.58756253726781e-10\\
42.5555078125	-1.50604309894527e-10\\
42.576796875	-1.49289809688847e-10\\
42.5980859375	-7.49946323641638e-11\\
42.619375	-1.83085158387975e-11\\
42.6406640625	-2.59399690860064e-11\\
42.661953125	5.09195873206204e-11\\
42.6832421875	2.01514865409494e-11\\
42.70453125	7.92549345559943e-11\\
42.7258203125	-3.32236030004641e-11\\
42.747109375	5.44919932695093e-11\\
42.7683984375	-1.81855339496345e-11\\
42.7896875	-1.67581512407722e-11\\
42.8109765625	-3.46950211889681e-11\\
42.832265625	5.60071337280519e-11\\
42.8535546875	1.69630443766109e-11\\
42.87484375	8.87970070582893e-11\\
42.8961328125	4.00943469026442e-11\\
42.917421875	5.71119924397217e-11\\
42.9387109375	1.51080571028796e-10\\
42.96	5.29059768277599e-11\\
42.9812890625	9.31310219355453e-11\\
43.002578125	6.04582516909325e-11\\
43.0238671875	3.74211921879347e-11\\
43.04515625	4.16287560492174e-11\\
43.0664453125	6.41096329653826e-11\\
43.087734375	6.22966530795232e-11\\
43.1090234375	2.47867698647166e-11\\
43.1303125	-5.6433923330278e-11\\
43.1516015625	-2.10598243074068e-11\\
43.172890625	-5.05236924890767e-11\\
43.1941796875	3.52738395405593e-11\\
43.21546875	1.05086275442429e-10\\
43.2367578125	1.66735122455945e-10\\
43.258046875	1.04445387158542e-10\\
43.2793359375	1.19050861573836e-10\\
43.300625	1.49274338071988e-10\\
43.3219140625	1.16800971656498e-10\\
43.343203125	1.14359045166646e-10\\
43.3644921875	9.21917207819819e-11\\
43.38578125	8.0062268394918e-11\\
43.4070703125	1.14072841361538e-10\\
43.428359375	1.11330142860451e-10\\
43.4496484375	2.36718528040926e-10\\
43.4709375	2.01133590768491e-10\\
43.4922265625	2.92084036262483e-10\\
43.513515625	2.87159999688909e-10\\
43.5348046875	3.48933712562704e-10\\
43.55609375	2.77259131681695e-10\\
43.5773828125	3.84698984538062e-10\\
43.598671875	2.60227558725688e-10\\
43.6199609375	2.30726796728965e-10\\
43.64125	3.07856328978237e-10\\
43.6625390625	2.16372446667286e-10\\
43.683828125	2.85548811910469e-10\\
43.7051171875	3.28729575677697e-10\\
43.72640625	3.56024478691246e-10\\
43.7476953125	3.61490128583413e-10\\
43.768984375	3.91492795811699e-10\\
43.7902734375	3.92339930969708e-10\\
43.8115625	3.63559209389539e-10\\
43.8328515625	3.06141373182215e-10\\
43.854140625	2.91467353247701e-10\\
43.8754296875	2.56699384847458e-10\\
43.89671875	1.59906620980654e-10\\
43.9180078125	2.44778288316134e-10\\
43.939296875	2.17327783114968e-10\\
43.9605859375	2.5675234052539e-10\\
43.981875	2.47720656485727e-10\\
44.0031640625	3.78119416758206e-10\\
44.024453125	2.79664424380856e-10\\
44.0457421875	2.84596674696697e-10\\
44.06703125	2.37184992630408e-10\\
44.0883203125	2.17298746993376e-10\\
44.109609375	2.42303789924099e-10\\
44.1308984375	1.51331716226738e-10\\
44.1521875	1.88113143795722e-10\\
44.1734765625	2.1366540880457e-10\\
44.194765625	2.97489935438995e-10\\
44.2160546875	2.47116512885055e-10\\
44.23734375	3.65511616960407e-10\\
44.2586328125	2.32341054109234e-10\\
44.279921875	3.12704462387591e-10\\
44.3012109375	2.59790660095609e-10\\
44.3225	1.97892486892444e-10\\
44.3437890625	3.46124350110014e-10\\
44.365078125	2.72958387166512e-10\\
44.3863671875	2.8620193556993e-10\\
44.40765625	2.49971053927972e-10\\
44.4289453125	2.86482132703375e-10\\
44.450234375	3.15923093572802e-10\\
44.4715234375	3.81026469333828e-10\\
44.4928125	3.05657918454796e-10\\
44.5141015625	3.065996225455e-10\\
44.535390625	2.61178890612825e-10\\
44.5566796875	2.85614991604573e-10\\
44.57796875	2.43306586524299e-10\\
44.5992578125	2.6840031486995e-10\\
44.620546875	1.85254393974223e-10\\
44.6418359375	2.57046658555049e-10\\
44.663125	1.90519619617394e-10\\
44.6844140625	2.75672798044929e-10\\
44.705703125	2.6110013788153e-10\\
44.7269921875	2.68402947497759e-10\\
44.74828125	2.33819735191705e-10\\
44.7695703125	1.77328870581966e-10\\
44.790859375	2.04671631412901e-10\\
44.8121484375	1.55677811108702e-10\\
44.8334375	1.50958906158864e-10\\
44.8547265625	1.79754186926347e-10\\
44.876015625	2.72089140204208e-10\\
44.8973046875	2.07036173930596e-10\\
44.91859375	2.23817443172297e-10\\
44.9398828125	1.77237960087961e-10\\
44.961171875	1.68205783174071e-10\\
44.9824609375	1.40645789289223e-10\\
45.00375	1.40624677053703e-10\\
45.0250390625	1.57312800730179e-10\\
45.046328125	1.59067372923946e-10\\
45.0676171875	2.1434845360123e-10\\
45.08890625	2.58488452267647e-10\\
45.1101953125	1.63044224638978e-10\\
45.131484375	1.44829725617747e-10\\
45.1527734375	8.01415325577499e-11\\
45.1740625	1.09642943009563e-11\\
45.1953515625	-2.55777248381743e-11\\
45.216640625	6.05317126958339e-12\\
45.2379296875	-2.54205026821282e-11\\
45.25921875	-6.59685046036085e-11\\
45.2805078125	-3.06854634082998e-11\\
45.301796875	-5.01791297287168e-11\\
45.3230859375	1.18181547773312e-11\\
45.344375	6.32452649972739e-12\\
45.3656640625	-4.414188223354e-12\\
45.386953125	-1.79389799728851e-11\\
45.4082421875	-2.05675658020028e-11\\
45.42953125	-3.29530825597933e-11\\
45.4508203125	-6.65365644927129e-11\\
45.472109375	-2.36207002004863e-11\\
45.4933984375	-1.11531251921441e-10\\
45.5146875	-5.88377509897622e-11\\
45.5359765625	4.90972635104863e-11\\
45.557265625	-2.45170865684322e-11\\
45.5785546875	1.02075240174462e-11\\
45.59984375	-1.28404978377463e-11\\
45.6211328125	-4.10734970959179e-11\\
45.642421875	-5.54294078194661e-11\\
45.6637109375	-2.54978489970886e-11\\
45.685	-8.76936509181491e-12\\
45.7062890625	-4.12965687808092e-11\\
45.727578125	-1.48486007411898e-10\\
45.7488671875	-1.28167000175654e-10\\
45.77015625	-1.34520293752432e-10\\
45.7914453125	-1.83754689717492e-10\\
45.812734375	-1.77982941393663e-10\\
45.8340234375	-2.0409020902115e-10\\
45.8553125	-2.18756231519262e-10\\
45.8766015625	-1.89902257242422e-10\\
45.897890625	-2.1880437699379e-10\\
45.9191796875	-1.90372797065339e-10\\
45.94046875	-2.06664333082098e-10\\
45.9617578125	-2.19218469695636e-10\\
45.983046875	-1.68964598455308e-10\\
46.0043359375	-2.34365809779892e-10\\
46.025625	-1.70130015881482e-10\\
46.0469140625	-2.34729632770398e-10\\
46.068203125	-1.9391357845164e-10\\
46.0894921875	-2.47981061275587e-10\\
46.11078125	-2.04703056011757e-10\\
46.1320703125	-2.3775564957146e-10\\
46.153359375	-2.40164793223728e-10\\
46.1746484375	-2.26482634461194e-10\\
46.1959375	-2.83270473296108e-10\\
46.2172265625	-1.66701778994514e-10\\
46.238515625	-2.34927706676175e-10\\
46.2598046875	-1.61442814664333e-10\\
46.28109375	-2.3822747097431e-10\\
46.3023828125	-2.71558844910224e-10\\
46.323671875	-2.57512563450225e-10\\
46.3449609375	-2.20164375745587e-10\\
46.36625	-2.55868856020627e-10\\
46.3875390625	-1.49185416487955e-10\\
46.408828125	-1.53170481187188e-10\\
46.4301171875	-1.91238177035145e-10\\
46.45140625	-1.60515817490736e-10\\
46.4726953125	-2.16064620502815e-10\\
46.493984375	-1.84799586442863e-10\\
46.5152734375	-2.49477905387069e-10\\
46.5365625	-1.75426872256769e-10\\
46.5578515625	-2.48522021504493e-10\\
46.579140625	-1.95736312245963e-10\\
46.6004296875	-1.57663349415876e-10\\
46.62171875	-2.26409171010933e-10\\
46.6430078125	-1.53012878014533e-10\\
46.664296875	-1.7870033209157e-10\\
46.6855859375	-2.55128882037012e-10\\
46.706875	-2.51580890145032e-10\\
46.7281640625	-3.45592152156351e-10\\
46.749453125	-2.91575267321654e-10\\
46.7707421875	-3.28443638580461e-10\\
46.79203125	-3.40983100702461e-10\\
46.8133203125	-3.55252073097679e-10\\
46.834609375	-3.28435179668749e-10\\
46.8558984375	-3.43611230812061e-10\\
46.8771875	-3.90498307543251e-10\\
46.8984765625	-3.0472945036254e-10\\
46.919765625	-3.71831468433805e-10\\
46.9410546875	-3.14906826014969e-10\\
46.96234375	-3.83722147812474e-10\\
46.9836328125	-3.67622030169861e-10\\
47.004921875	-3.48175123135221e-10\\
47.0262109375	-2.62387421647813e-10\\
47.0475	-2.87704520041081e-10\\
47.0687890625	-2.79849072356948e-10\\
47.090078125	-2.53592943202687e-10\\
47.1113671875	-2.48976235421809e-10\\
47.13265625	-2.61348193108224e-10\\
47.1539453125	-2.2750680116816e-10\\
47.175234375	-2.37371653884111e-10\\
47.1965234375	-2.4533483974451e-10\\
47.2178125	-1.76691603022767e-10\\
47.2391015625	-2.24352947671766e-10\\
47.260390625	-2.03091370591411e-10\\
47.2816796875	-2.22932872431002e-10\\
47.30296875	-1.91289322095112e-10\\
47.3242578125	-1.56051721689725e-10\\
47.345546875	-9.60686637621637e-11\\
47.3668359375	-1.94117175898277e-10\\
47.388125	-1.79841294664402e-10\\
47.4094140625	-2.10645140208374e-10\\
47.430703125	-2.17437766495419e-10\\
47.4519921875	-2.03971290793269e-10\\
47.47328125	-1.27497279660822e-10\\
47.4945703125	-1.25689196248484e-10\\
47.515859375	-1.66800690430008e-10\\
47.5371484375	-2.04993302722349e-10\\
47.5584375	-1.52297477069621e-10\\
47.5797265625	-2.09342559801918e-10\\
47.601015625	-2.08511979799739e-10\\
47.6223046875	-2.80238141576196e-10\\
47.64359375	-1.52279912085989e-10\\
47.6648828125	-1.6400267658076e-10\\
47.686171875	-1.05773064274244e-10\\
47.7074609375	-8.32327662145003e-11\\
47.72875	-3.09932136091766e-11\\
47.7500390625	-2.18876542001053e-11\\
47.771328125	-2.6816739196855e-11\\
47.7926171875	-6.1452141123051e-11\\
47.81390625	-1.00528557059676e-10\\
47.8351953125	-9.43855924195389e-11\\
47.856484375	-1.09219351845826e-10\\
47.8777734375	-3.68202494835367e-11\\
47.8990625	-1.2908179388537e-10\\
47.9203515625	4.59319158113123e-11\\
47.941640625	4.26619060037908e-11\\
47.9629296875	-1.19928108016872e-11\\
47.98421875	4.23640864091835e-11\\
48.0055078125	-4.64236713127365e-11\\
48.026796875	-6.33460252979216e-11\\
48.0480859375	-2.79353733887273e-13\\
48.069375	-2.16390398675617e-11\\
48.0906640625	-5.06269270998122e-12\\
48.111953125	5.98435563914803e-11\\
48.1332421875	1.47926259916508e-10\\
48.15453125	1.25550066268083e-10\\
48.1758203125	6.84336304482472e-11\\
48.197109375	6.93557263472352e-11\\
48.2183984375	6.83599519227777e-11\\
48.2396875	6.83167660170539e-11\\
48.2609765625	-1.24211578760043e-12\\
48.282265625	-3.64005100266977e-11\\
48.3035546875	1.02431260406582e-10\\
48.32484375	8.13845039280011e-11\\
48.3461328125	1.219931497485e-10\\
48.367421875	1.68121851674474e-10\\
48.3887109375	1.4577959450301e-10\\
48.41	1.38679833819604e-10\\
48.4312890625	1.23140159298163e-10\\
48.452578125	4.1724858894347e-11\\
48.4738671875	1.12387129719922e-10\\
48.49515625	1.69024985960431e-11\\
48.5164453125	8.36742579381983e-11\\
48.537734375	1.04090366072935e-10\\
48.5590234375	1.36559128319006e-10\\
48.5803125	1.84364900480472e-10\\
48.6016015625	1.28557954728769e-10\\
48.622890625	1.97850069033551e-10\\
48.6441796875	2.52893026006413e-10\\
48.66546875	2.57134123722479e-10\\
48.6867578125	2.50349099530092e-10\\
48.708046875	1.01413053235942e-10\\
48.7293359375	1.92620053678704e-10\\
48.750625	1.36418858370543e-10\\
48.7719140625	8.93122843927918e-11\\
48.793203125	2.43680871217425e-10\\
48.8144921875	2.38752138825798e-10\\
48.83578125	2.54096458192268e-10\\
48.8570703125	2.50152453090741e-10\\
48.878359375	3.15548046178146e-10\\
48.8996484375	2.67860026617266e-10\\
48.9209375	2.55859963086502e-10\\
48.9422265625	1.6454942613615e-10\\
48.963515625	2.30907142681756e-10\\
48.9848046875	1.64212160962295e-10\\
49.00609375	1.49620130035255e-10\\
49.0273828125	1.74825113919606e-10\\
49.048671875	1.66226663799524e-10\\
49.0699609375	2.91936522600842e-10\\
49.09125	1.78276692939188e-10\\
49.1125390625	3.49411478916773e-10\\
49.133828125	2.32719575180493e-10\\
49.1551171875	1.97488334100383e-10\\
49.17640625	2.84634053038028e-10\\
49.1976953125	2.21800909608752e-10\\
49.218984375	2.04957652661938e-10\\
49.2402734375	1.79107962660434e-10\\
49.2615625	1.59066007948993e-10\\
49.2828515625	2.74225632567936e-10\\
49.304140625	2.46957036814852e-10\\
49.3254296875	3.10351122450439e-10\\
49.34671875	3.79500183750469e-10\\
49.3680078125	3.21617414460563e-10\\
49.389296875	2.78180461879867e-10\\
49.4105859375	2.38962485808226e-10\\
49.431875	2.40052115660293e-10\\
49.4531640625	2.44411044040737e-10\\
49.474453125	2.38049610343123e-10\\
49.4957421875	2.81046749022432e-10\\
49.51703125	2.74801777658962e-10\\
49.5383203125	3.27385663445486e-10\\
49.559609375	3.02246791927878e-10\\
49.5808984375	3.35506115475968e-10\\
49.6021875	2.51061843251354e-10\\
49.6234765625	3.09277367272397e-10\\
49.644765625	2.53365316352141e-10\\
49.6660546875	1.91444884074999e-10\\
49.68734375	2.6729248348635e-10\\
49.7086328125	1.48025188407756e-10\\
49.729921875	1.60568073531873e-10\\
49.7512109375	1.38380176818835e-10\\
49.7725	1.90123053001623e-10\\
49.7937890625	2.10648576035505e-10\\
49.815078125	1.29383875933731e-10\\
49.8363671875	1.7989010621259e-10\\
49.85765625	1.53341233711644e-10\\
49.8789453125	1.71802383808453e-10\\
49.900234375	8.79240609392982e-11\\
49.9215234375	1.060211592453e-10\\
49.9428125	1.04311225406725e-10\\
49.9641015625	1.24459634096151e-10\\
49.985390625	5.97089299498349e-11\\
50.0066796875	8.55035333234286e-11\\
50.02796875	1.21298276463266e-10\\
50.0492578125	7.57326782574982e-11\\
50.070546875	2.98608124731071e-11\\
50.0918359375	5.80077650423295e-11\\
50.113125	8.1512466865185e-11\\
50.1344140625	8.14313220191267e-11\\
50.155703125	8.09876363619705e-11\\
50.1769921875	7.7404378350458e-11\\
50.19828125	1.04848872721472e-10\\
50.2195703125	1.17591767186014e-11\\
50.240859375	6.90358592695067e-11\\
50.2621484375	6.52923442353795e-12\\
50.2834375	-1.49408079436786e-11\\
50.3047265625	-3.26643731998241e-11\\
50.326015625	-8.16049296753355e-11\\
50.3473046875	-5.77518456944703e-11\\
50.36859375	-9.82944552338231e-11\\
50.3898828125	-2.04145375449969e-10\\
50.411171875	-9.77745677074949e-11\\
50.4324609375	-1.99462546106913e-10\\
50.45375	-1.84369337703824e-10\\
50.4750390625	-1.97948553531875e-10\\
50.496328125	-1.83442743487061e-10\\
50.5176171875	-1.67679237416115e-10\\
50.53890625	-1.72263558470532e-10\\
50.5601953125	-1.75967394046841e-10\\
50.581484375	-7.84678111027916e-11\\
50.6027734375	-1.72161471787654e-10\\
50.6240625	-1.63066269760812e-10\\
50.6453515625	-2.36001653417909e-10\\
50.666640625	-2.84046229538645e-10\\
50.6879296875	-2.56427032392407e-10\\
50.70921875	-2.64954693714729e-10\\
50.7305078125	-2.28244907041225e-10\\
50.751796875	-2.38019334772543e-10\\
50.7730859375	-1.8145103849157e-10\\
50.794375	-2.39573507263056e-10\\
50.8156640625	-2.6192967924698e-10\\
50.836953125	-1.74266739774703e-10\\
50.8582421875	-2.21195454284717e-10\\
50.87953125	-2.20164027344198e-10\\
50.9008203125	-2.10520751111072e-10\\
50.922109375	-2.5840138116031e-10\\
50.9433984375	-2.79193736745121e-10\\
50.9646875	-2.78253698453802e-10\\
50.9859765625	-2.59945282435204e-10\\
51.007265625	-2.73582904683436e-10\\
51.0285546875	-2.60252279710263e-10\\
51.04984375	-3.09947583071166e-10\\
51.0711328125	-2.7968159060295e-10\\
51.092421875	-2.67023497993805e-10\\
51.1137109375	-1.8819428837756e-10\\
51.135	-2.39310747335437e-10\\
51.1562890625	-1.96698692093076e-10\\
51.177578125	-3.15088860331038e-10\\
51.1988671875	-2.17292337735987e-10\\
51.22015625	-2.37918097278024e-10\\
51.2414453125	-2.38565334973048e-10\\
51.262734375	-2.53137301471828e-10\\
51.2840234375	-2.66599667265794e-10\\
51.3053125	-2.65790852133501e-10\\
51.3266015625	-2.5789530454409e-10\\
51.347890625	-2.651966021789e-10\\
51.3691796875	-3.09064333928099e-10\\
51.39046875	-3.52376469771194e-10\\
51.4117578125	-3.60668170976372e-10\\
51.433046875	-3.08945139009024e-10\\
51.4543359375	-3.23876199860697e-10\\
51.475625	-3.44090067681477e-10\\
51.4969140625	-2.49007502348579e-10\\
51.518203125	-2.50956239699004e-10\\
51.5394921875	-2.56806922584495e-10\\
51.56078125	-2.91086756428554e-10\\
51.5820703125	-2.66399405846187e-10\\
51.603359375	-3.16556125948024e-10\\
51.6246484375	-3.07304923501774e-10\\
51.6459375	-2.44697825127162e-10\\
51.6672265625	-2.68006816789654e-10\\
51.688515625	-3.16047752311865e-10\\
51.7098046875	-2.72554981497206e-10\\
51.73109375	-2.95315663360416e-10\\
51.7523828125	-2.72720086735179e-10\\
51.773671875	-2.37132186763867e-10\\
51.7949609375	-2.35149714240116e-10\\
51.81625	-1.86830596802059e-10\\
51.8375390625	-2.1815636965338e-10\\
51.858828125	-1.84972892965782e-10\\
51.8801171875	-1.36533775594849e-10\\
51.90140625	-1.1845784794404e-10\\
51.9226953125	-1.3910685208447e-10\\
51.943984375	-1.76874807281483e-10\\
51.9652734375	-1.88969411930993e-10\\
51.9865625	-1.91547646793976e-10\\
52.0078515625	-1.83446438053929e-10\\
52.029140625	-1.94236204846914e-10\\
52.0504296875	-2.120519819652e-10\\
52.07171875	-2.18377484146641e-10\\
52.0930078125	-1.96283699814204e-10\\
52.114296875	-1.46790696059981e-10\\
52.1355859375	-1.87694159767384e-10\\
52.156875	-1.01555344105803e-10\\
52.1781640625	-1.33179120936439e-10\\
52.199453125	-1.7208045045638e-10\\
52.2207421875	-1.05529715333513e-10\\
52.24203125	-1.33504140657765e-10\\
52.2633203125	-7.79450823323596e-11\\
52.284609375	-1.35726431256701e-10\\
52.3058984375	-6.7419346439432e-11\\
52.3271875	-1.32103352627948e-10\\
52.3484765625	-1.20757128443628e-10\\
52.369765625	-7.86613506696945e-11\\
52.3910546875	-6.70653017839473e-12\\
52.41234375	-8.61985376231219e-11\\
52.4336328125	-8.16256286209305e-11\\
52.454921875	-4.996194992688e-11\\
52.4762109375	-7.72216281199842e-11\\
52.4975	-1.2112312904234e-10\\
52.5187890625	-1.51202095126833e-10\\
52.540078125	-6.023577853087e-11\\
52.5613671875	-1.32895339880105e-10\\
52.58265625	-1.18335511745287e-10\\
52.6039453125	-1.27440790035698e-10\\
52.625234375	-7.85102098067634e-11\\
52.6465234375	-5.70132222450504e-11\\
52.6678125	-1.30244925357363e-10\\
52.6891015625	-1.39036112366367e-10\\
52.710390625	-1.1110240068052e-10\\
52.7316796875	-9.43473083749992e-11\\
52.75296875	-1.08826230792133e-10\\
52.7742578125	-7.97710122191296e-11\\
52.795546875	-4.24283775753746e-11\\
52.8168359375	9.47089886880942e-13\\
52.838125	4.49241468877331e-11\\
52.8594140625	1.19200228412208e-10\\
52.880703125	4.84212713118078e-11\\
52.9019921875	3.25744565476585e-11\\
52.92328125	-1.48859353863987e-12\\
52.9445703125	1.42084136747205e-11\\
52.965859375	-4.26305902150405e-12\\
52.9871484375	7.66828712354748e-11\\
53.0084375	3.88496413858897e-11\\
53.0297265625	1.6525935406732e-10\\
53.051015625	1.16995777317927e-10\\
53.0723046875	1.17330236812863e-10\\
53.09359375	1.42280603533925e-10\\
53.1148828125	1.02879078921593e-10\\
53.136171875	6.70078418432926e-11\\
53.1574609375	6.52333537814574e-11\\
53.17875	1.68466333484806e-11\\
53.2000390625	7.6680115354467e-11\\
53.221328125	9.36542028133114e-11\\
53.2426171875	1.28466930848085e-10\\
53.26390625	9.83331952499383e-11\\
53.2851953125	8.13598600377361e-11\\
53.306484375	1.56495092422948e-10\\
53.3277734375	1.20193342081847e-10\\
53.3490625	7.79129190283366e-12\\
53.3703515625	-1.11174721602302e-11\\
53.391640625	-4.20948261501185e-11\\
53.4129296875	4.35544551237001e-12\\
53.43421875	-3.71125668359681e-12\\
53.4555078125	3.14407756106297e-12\\
53.476796875	3.95937466521607e-11\\
53.4980859375	7.18732561148128e-11\\
53.519375	8.40676029928165e-11\\
53.5406640625	3.12061046407142e-11\\
53.561953125	3.55911658701536e-11\\
53.5832421875	2.13842937561876e-11\\
53.60453125	-5.76439608275291e-11\\
53.6258203125	-6.96276330932258e-11\\
53.647109375	-4.97016675377519e-11\\
53.6683984375	3.41805664056928e-12\\
53.6896875	-1.97648262725595e-12\\
53.7109765625	-3.32760329061972e-11\\
53.732265625	3.18079784138886e-11\\
53.7535546875	-5.89924343989278e-12\\
53.77484375	-4.70309899455155e-11\\
53.7961328125	-6.1342971141446e-11\\
53.817421875	-8.59825839720011e-11\\
53.8387109375	-6.35067561328414e-11\\
53.86	1.43819762201256e-11\\
53.8812890625	2.87361263174467e-11\\
53.902578125	1.09516489792642e-10\\
53.9238671875	8.57997695816731e-11\\
53.94515625	9.39927775009494e-11\\
53.9664453125	4.62075876470106e-11\\
53.987734375	1.27705487281717e-10\\
54.0090234375	6.56933515238345e-11\\
54.0303125	8.52487478068994e-11\\
54.0516015625	6.97630021021545e-11\\
54.072890625	2.5890849570032e-11\\
54.0941796875	4.9724255717885e-11\\
54.11546875	4.695280782042e-11\\
54.1367578125	6.0346183330887e-11\\
54.158046875	4.08054563683289e-11\\
54.1793359375	1.0516585033254e-10\\
54.200625	4.74696153459742e-11\\
54.2219140625	3.70962120697142e-12\\
54.243203125	1.22523153369419e-10\\
54.2644921875	3.93502995243309e-11\\
54.28578125	5.55342271594272e-11\\
54.3070703125	3.57749132935709e-11\\
54.328359375	-8.58590369006937e-12\\
54.3496484375	3.66773016322369e-11\\
54.3709375	-8.96235363277179e-13\\
54.3922265625	1.50794212121676e-11\\
54.413515625	3.08673476545449e-11\\
54.4348046875	3.97442710680851e-11\\
54.45609375	2.72127668470492e-11\\
54.4773828125	4.35769858735069e-11\\
54.498671875	5.55927197633478e-11\\
54.5199609375	2.75323333080089e-11\\
54.54125	2.53407275512822e-11\\
54.5625390625	2.39138935207253e-11\\
54.583828125	3.14845425091175e-11\\
54.6051171875	4.63965287771768e-11\\
54.62640625	2.8288640189213e-11\\
54.6476953125	5.80484476347898e-11\\
54.668984375	8.82579702251484e-11\\
54.6902734375	3.92392306633556e-11\\
54.7115625	-4.38913640105062e-12\\
54.7328515625	6.19031702924722e-11\\
54.754140625	9.93540200841804e-12\\
54.7754296875	3.58053474089424e-11\\
54.79671875	3.38859108487719e-11\\
54.8180078125	-4.87249144835941e-11\\
54.839296875	2.3321307303914e-11\\
54.8605859375	-8.89912515033652e-11\\
54.881875	-1.87347828800952e-11\\
54.9031640625	-7.07397374891299e-11\\
54.924453125	-2.59789143478249e-11\\
54.9457421875	-2.45962861296978e-11\\
54.96703125	-7.37168195655443e-11\\
54.9883203125	8.85395102947475e-11\\
55.009609375	1.40236833255355e-11\\
55.0308984375	2.68114439984938e-11\\
55.0521875	3.64359972655441e-11\\
55.0734765625	1.0505911056485e-11\\
55.094765625	1.90303961999226e-11\\
55.1160546875	6.84411542286279e-11\\
55.13734375	4.10576150751377e-11\\
55.1586328125	5.6638262308211e-11\\
55.179921875	2.88408614806529e-11\\
55.2012109375	4.60454841352935e-11\\
55.2225	4.27936215338739e-11\\
55.2437890625	2.17194958059126e-11\\
55.265078125	-3.23450617016461e-11\\
55.2863671875	3.40458774946302e-11\\
55.30765625	-9.30664564399233e-12\\
55.3289453125	-2.2235003826988e-11\\
55.350234375	-5.22825863129762e-11\\
55.3715234375	-1.71937529020824e-11\\
55.3928125	-8.13505887506181e-11\\
55.4141015625	-4.64586332150936e-11\\
55.435390625	-5.01567642444761e-11\\
55.4566796875	-2.74881634614597e-11\\
55.47796875	-4.05189008067988e-11\\
55.4992578125	-1.10284185133704e-10\\
55.520546875	-6.7651343792692e-11\\
55.5418359375	-1.38384177682422e-10\\
55.563125	-1.08488312376555e-10\\
55.5844140625	-1.07642490793637e-10\\
55.605703125	-1.12796146960987e-10\\
55.6269921875	-1.48809509906514e-10\\
55.64828125	-1.01158772285228e-10\\
55.6695703125	-1.45338132926043e-10\\
55.690859375	-2.25845114421473e-11\\
55.7121484375	-1.13244895803737e-10\\
55.7334375	-9.17343524221726e-11\\
55.7547265625	-1.1038118762267e-10\\
55.776015625	-1.0261950413009e-10\\
55.7973046875	-7.24627675088635e-11\\
55.81859375	-1.31805527342199e-10\\
55.8398828125	-7.5243012500614e-11\\
55.861171875	-1.23290703940252e-10\\
55.8824609375	-7.25142780454316e-11\\
55.90375	-9.65929600030464e-11\\
55.9250390625	-3.93930858744962e-11\\
55.946328125	-7.08988394543966e-11\\
55.9676171875	-3.11841591482971e-11\\
55.98890625	-2.36416452071548e-11\\
56.0101953125	-3.77958634649013e-11\\
56.031484375	1.55470169360965e-11\\
56.0527734375	-8.05494394979224e-11\\
56.0740625	-1.47225446953126e-11\\
56.0953515625	-2.38842332676489e-11\\
56.116640625	1.42347947374849e-11\\
56.1379296875	-2.08452962166013e-11\\
56.15921875	-6.89710770795397e-12\\
56.1805078125	1.18698528221396e-11\\
56.201796875	2.57433684852107e-11\\
56.2230859375	9.35308157101745e-12\\
56.244375	-1.96596401346292e-12\\
56.2656640625	5.41990763363761e-11\\
56.286953125	1.93129764619551e-11\\
56.3082421875	3.28858045285779e-11\\
56.32953125	4.16507507226332e-11\\
56.3508203125	6.70334245725386e-11\\
56.372109375	-1.39373160436613e-11\\
56.3933984375	1.85171568119732e-12\\
56.4146875	1.53039696740304e-12\\
56.4359765625	2.35064356860054e-11\\
56.457265625	-1.90151523039949e-11\\
56.4785546875	-2.76505865989351e-11\\
56.49984375	-5.46609082796093e-11\\
56.5211328125	-5.56818240663076e-12\\
56.542421875	-1.65247488480776e-11\\
56.5637109375	-6.18056422592072e-11\\
56.585	-5.90358449675486e-11\\
56.6062890625	4.26082339525875e-12\\
56.627578125	3.15442186494078e-11\\
56.6488671875	-2.95248904345283e-11\\
56.67015625	1.61477753962007e-11\\
56.6914453125	1.11191437006455e-11\\
56.712734375	-3.92723870928778e-11\\
56.7340234375	-1.54718576948271e-11\\
56.7553125	-5.82543677752822e-11\\
56.7766015625	-8.25980869384605e-12\\
56.797890625	-6.31921630233928e-11\\
};
\addlegendentry{$\text{train 4 -\textgreater{} Trondheim}$};

\end{axis}
\end{tikzpicture}%