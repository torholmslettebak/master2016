\section{Analysis}

This chapter will describe how the BWIM system performs. What works? Why? How? etc.
The main focus should perhaps be placed on identifying the pros and cons of the matrix method and optimization method. 

Should include:
\begin{itemize}
\item Compare theoretical and calculated influence lines. Also include influence lines found through Abaqus.
\item Check how influence lines found through matrix method and optimization reproduces the strain history
\item Test obtained influence line by running the bwim routine on the hitherto unused freight train. (Depends on getting info about the train). Also Do this test on the other trains.
\end{itemize}

\subsection{The matrix method}
The matrix method creates an influence line for a specific strain history given a known train with known axle weights and velocity. Thus if the strain signal were recreated given with given parameters, the signal would be a almost exact replica of the measured strain signal, where the differences should originate from sensor noise. The found influence line would however be for this specific train and the passing's dynamic effects on the bridge, which is likely to vary from train to train. Therefore an averaging of a sufficient number of calculated influence lines should reduce or eliminate the dynamic effects from the influence line.

The analysis of the matrix method is based on 4 different train passings, and 3 sensor readings on each passing. The trains in these measurements is of the same type (not entirely shure!!) but the exact weight is not known. The weight of each axle were approximated by distributing carriage and locomotive gross weight. Passengers in the passing trains were not accounted for, and may lead to some deviation from ideal results. 

TODO:
\begin{itemize}
\item Show the found influence lines for some sensors
\item discuss the plots
\item reproduce strain signal, and compare with measured signal
\item show averaged influence line, and perform the same tests
\item show interpolation of this averaged influence line
\item perform the same test with this interpolated influence line
\item the alternative should also be done, interpolate each found influence line and average them, then reproduce the strain signal, and find difference through comparison.
\end{itemize} 

\subsection{Optimized influence lines}
Perform the same procedures as for the matrix method

\subsection{Differences between the methods}
Compare the optimized influence lines and the matrix method influence lines. This should be done in a thorough manner.
\subsection{Problems}
\begin{itemize}
\item Big problem with identifying exactly when train enters and leaves the bridge. This results in guesswork when placing influence line in a coordinate system. Where does the bridge begin and end in the influence line.. The only definite certainty seems to be placing the index of the maximum magnitude of the influence line in the correct position according to the measuring sensor's location.
\item This could be problematic when using the found influence lines 
\item These problems have been reduced, now the biggest problem is placing the peak of the influence line as well as possible. Possibly performing a smoothing and then finding position of peak could give a better estimate of sensorloc at influence line.. currently the max value of influence line is placed at sensorloc.
\end{itemize}
\begin{figure}[H]
\centering
\begin{subfigure}[t]{0.45\textwidth}
	\centering
	% This file was created by matlab2tikz.
%
%The latest updates can be retrieved from
%  http://www.mathworks.com/matlabcentral/fileexchange/22022-matlab2tikz-matlab2tikz
%where you can also make suggestions and rate matlab2tikz.
%
%% Preampble for Master Thesis
% !TEX root = main.tex
\documentclass[english,12pt,a4paper]{article}
\usepackage[T1]{fontenc} % --------------| More characters.
\usepackage[utf8]{inputenc} % ---------| Direct use of scandinavian letters.
\usepackage{float} % --------------------| More options for floats.
\usepackage{graphicx} % -----------------| Support more image formats.
\usepackage{booktabs} % -----------------| Better-looking tables.
\usepackage{tabularx} % -----------------| Better tables
\usepackage{subcaption} % -------------------| Subfigures.
\usepackage[a4paper]{geometry} % --------| Adjusting page margins.
\usepackage{amsmath,amssymb,amsfonts} % -| Various math, including eqref.
\usepackage{xcolor} % --------------------| Allows defn. of custom colors.
%\usepackage[english]{babel}
\usepackage{url}
\usepackage[backend=bibtex,style=numeric-comp]{biblatex}
\usepackage[hidelinks]{hyperref}
\usepackage[nottoc,numbib]{tocbibind}
\usepackage{siunitx}
\usepackage{tikz}
\usepackage[normalem]{ulem}
\usepackage{multirow}
% \usetikzlibrary{"arrows", "automata", "backgrounds", "calendar", "chains", "matrix", "mindmap", "patterns", "petri", "shadows", "shapes.geometric", "shapes.misc", "spy", "trees"}
\usetikzlibrary{"arrows"}

\usetikzlibrary{calc}
\usetikzlibrary{positioning}
\usetikzlibrary{shapes}

\usepackage{pgfplots}
\pgfplotsset{compat=newest}
  %% the following commands are sometimes needed
\usetikzlibrary{plotmarks}
\usepackage{grffile}
\usepackage{amsmath}
%% you may also want the following commands
\pgfplotsset{plot coordinates/math parser=false}
\newlength\figureheight
\newlength\figurewidth
\addbibresource{bibliography.bib}
\usetikzlibrary{external}
\tikzexternalize[]
% \renewcommand\refname{Bibliography}
%\usepgfplotslibrary{external}
%\tikzexternalize
%\usetikzlibrary{external}
%\tikzexternalize[prefix=tikz/]
% \addbibresource{references.bib}
% \renewcommand\refname{Bibliography}
% \bibliographystyle{ieeetran}

%%\usepackage{tikz,pgf}
\usetikzlibrary{calc}


% fast grunn = {x}{y}{l}{r}
% Tegner en linje med korte streker nedfra.
% x, y er startpunkter; l er lengde, r er rotasjon
% f.eks \node[fast grunn = {0}{0}{10}{60}]{};
\tikzset{
	fast grunn/.style n args={4}{%
		append after command={\pgfextra{%
			\begin{scope}[shift={(#1,#2)}]
				% Hovedlinje ==================================================
				\coordinate (A) at  (0,0); % Venstre kant
				\coordinate (B) at  (#3,0); % Høyre kant

				\draw[] let \p{A}=(A), \p{B}=(B) in
				{[rotate)=#4] (\p{A}) -- (\p{B})};

				% Korte streker ===============================================
				\foreach \i in {0, .25, ...,#3}{
					\pgfmathparse{\i}
					\draw[] let \p{I}=($ (A) + (\pgfmathresult, 0) $), \p{II} = ($ (A) + (\pgfmathresult, 0) - (.15,.15) $) in
					{[rotate=#4] (\p{I}) to (\p{II})}
				}
			\end{scope}
			}
		},
	}
}
%
%% Feste
%% feste = {x}{y}{r}
%% Tegne et fast, rotasjonsfritt feste ved x,y rotert til r grader
%% Eksempel: \node[feste={0}{0}{60}] {}
\tikzset{
	feste/.style n args={3}{%
		append after command={\pgfextra{%
			\begin{scope}[shift={(#1,#2)}]
				% Hovedlinje =================================================
				\coordinate (A) at (-.6,0);	% Venstre kant
				\coordinate (B) at (.6,0);	% Høyre kant

				\draw[] let \p{A}=(A), \p(B)=(B) in
				{[rotate=#3] (\p{A}) -- (\p{B})};

				% Korte streker ==============================================
				\foreach \i in {0,...,12} {
					\pgfmathparse{\i/10}
					\draw[] let \p{I}=($ (A) + (pgfmathresult, 0) $), \p{II}=( $ (A) + (\pgfmathresult, 0) - (.15,.15) $) in
					{[rotate=#3] (\p{I}) to (\p{II})};
				}
			\end{scope}
			},
		}
	}
}

% Fast ledd
% ledd fast={x}{y}{r}
% tegner et ledd der øverste sirkel er sentrert i x, y og rotert r grader
% Eksempel : \node[ledd fast={0}{0}{90}] {};
\tikzset{
	ledd fast/.style n args={3}{%
		append after command={\pgfextra{%
			\begin{scope}[shift={(#1,#2)}]
				% Hovedtriangel =====================================================
				\coordinate (A) at (0,0); % leddpunkt
				\coordinate (B) at (-.5,-.75); % nedre hjørne (venstre)
				\coordinate (C) at (.5,-.75);	% Nedre hjørne (høyre)

				\draw[] let \p{A}=(A),\p{B}=(B),\p{C}=(C) in
				{[rotate=#3] (\p{A}) -- (\p{B}) -- (\p{C}) -- cycle};

				% Lengre bunnlinje ==================================================
				\coordinate (BB) at (-.6,-.75);
				\coordinate (CC) at (.6,-.75);

				\draw[] let \p{BB}=(BB), \p{CC}=(CC) in
				{[rotate=#3] (\p{BB}) -- (\p{CC})};

				% Liten sirkel ======================================================
				\draw[fill=white] (A) circle (.1);
				% Korte streker =====================================================
				\foreach \i in {0,...,12} {
					\pgfmathparse{\i/10}
					\draw[] let \p{I}=($ (BB) + (\pgfmathresult, 0) $), \p{II}=($ (BB) + (\pgfmathresult, 0) - (.15,.15) $) in
					{[rotate=#3] (\p{I}) to (\p{II})};
				}
				\end{scope}
			}
		},
	}
}

% Skyveledd
% ledd skyve={x}{y}{r}
% Tegner et ledd der øverste sirkel er sentrert i x,y og rotert r grader
% Eksempel: \node[ledd skyve{3}{0}{0}] {};
\tikzset{
	ledd skyve/.style n args = {3}{%
		append after command={\pgfextra{%
		\begin{scope}[shift={(#1,#2)}]
			% Hovedtriangel ======================================================
			\coordinate (A) at (0,0); % Leddpunkt
			\coordinate (B) at (-.5,-.75); % Nedre hjørne (venstre)
			\coordinate (C) at (.5,-.75); % Nedre hørne (høyre)

			\draw[] let \p{A}=(A),\p{B}=(B),\p{C}=(C) in
			{[rotate=#3] (\p{A}) -- (\p{B}) -- (\p{C}) -- cycle};

			% Lengre bunnlinje ===================================================
			\coordinate (BB) at (-.6,-.82);
			\coordinate (CC) at (.6,-.82);

			\draw[] let \p{BB}=(BB),\p{CC}=(CC) in
			{[rotate=#3] (\p{BB}) -- (\p{CC})};

			% Liten sirkel =======================================================
			\draw[fill=white] (A) circle (.1);

			% Korte streker ======================================================
			\foreach \i in {0,...,12} {
				\pgfmathparse{\i/10}
				\draw[] let \p{I}=($ (BB) + (\pgfmathresult, 0) $), \p{II}=($ (BB) + (\pgfmathresult, 0) - (.15,.15) $) in
				{[rotate=#3] (\p{I}) to (\p{II})};
			}
			\end{scope}
			}
		},
	}
}

% Spring
% spring={x}{y}{r}
% \node[spring{3}{0}{0}] {};
\tikzset{
	ledd spring/.style n args = {3}{%
		append after command={\pgfextra{%
		\begin{scope}[shift={(#1,#2)}]
			% Hovedtriangel ======================================================
			\coordinate (A) at (0,0); % Leddpunkt
			\coordinate (B) at (-.5,-.75); % Nedre hjørne (venstre)
			\coordinate (C) at (.5,-.75); % Nedre hørne (høyre)

			\draw[] let \p{A}=(A),\p{B}=(B),\p{C}=(C) in
			{[rotate=#3] (\p{A}) -- (\p{B}) -- (\p{C}) -- cycle};


			% spring
			\draw (0,-.75) -- (0.25,-1) -- (-0.25,-1.25) -- (0.25,-1.5) -- (-0.25,-1.75); % spring""

			% Lengre bunnlinje ===================================================
			\coordinate (BB) at (-.6,-1.75);
			\coordinate (CC) at (.6,-1.75);

			\draw[] let \p{BB}=(BB),\p{CC}=(CC) in
			{[rotate=#3] (\p{BB}) -- (\p{CC})};
			% Liten sirkel =======================================================
			\draw[fill=white] (A) circle (.1);

			% Korte streker ======================================================
			\foreach \i in {0,...,12} {
				\pgfmathparse{\i/10}
				\draw[] let \p{I}=($ (BB) + (\pgfmathresult, 0) $), \p{II}=($ (BB) + (\pgfmathresult, 0) - (.15,.15) $) in
				{[rotate=#3] (\p{I}) to (\p{II})};
			}
			\end{scope}
			}
		},
	}
}

%\tikzset{
%    name/.append style={
%        /tikz/name path=qrr-node-#1
%    },
%    arrow length/.code={
%        \pgfmathsetlength\qrrArrowLength{#1}
%    },
%    m/.style={
%        arrow length=6*(.5pt+.25\pgflinewidth),
%        to path={
%            \pgfextra{
%                \path[name path=qrr-\the\qrrArrowLineCounter-path] (\tikztostart) -- (\tikztotarget);
%                \path[name intersections={of=qrr-\the\qrrArrowLineCounter-path and qrr-node-\tikztostart}] (intersection-1) coordinate (qrr-\the\qrrArrowLineCounter-start);
%                \path[name intersections={of=qrr-\the\qrrArrowLineCounter-path and qrr-node-\tikztotarget}] (intersection-1) coordinate (qrr-\the\qrrArrowLineCounter-target);
%                \path (\tikztostart) -- ($(qrr-\the\qrrArrowLineCounter-target)!\the\qrrArrowLength!(qrr-\the\qrrArrowLineCounter-start)$) \tikztonodes;
%                \global\advance\qrrArrowLineCounter by 1\relax
%            }
%            (\tikztostart) -- (\tikztotarget)
%        }
%    },
%}


%\author{Tor Holm Slettebak}
%\title{Master Thesis}

%\begin{document}
\definecolor{mycolor1}{rgb}{0.00000,0.44700,0.74100}%
%
\begin{tikzpicture}

\begin{axis}[%
width=\textwidth,
height=\textwidth,
at={(0\figurewidth,0\figureheight)},
scale only axis,
xmin=-60,
xmax=60,
ymin=-2e-09,
ymax=1e-08,
axis background/.style={fill=white},
% title style={font=\bfseries},
% title={Influencelines for train 3, middle sensor},
legend style={legend cell align=left,align=left,draw=white!15!black}
]
\addplot [color=mycolor1,solid,forget plot]
  table[row sep=crcr]{%
-47.80724609375	-2.43705295696912e-11\\
-47.786748046875	-6.76412695658217e-11\\
-47.76625	-7.04825924520673e-11\\
-47.745751953125	-3.83282736349843e-11\\
-47.72525390625	-2.7624144334012e-11\\
-47.704755859375	-1.00608535823748e-10\\
-47.6842578125	-7.85422568649401e-11\\
-47.663759765625	-1.65977371633779e-11\\
-47.64326171875	-7.25506485136496e-11\\
-47.622763671875	-4.50116083969734e-11\\
-47.602265625	-7.87973899513735e-11\\
-47.581767578125	3.47683545211713e-11\\
-47.56126953125	-8.58630388578684e-11\\
-47.540771484375	-1.32798980825908e-11\\
-47.5202734375	-1.1673518179255e-10\\
-47.499775390625	-1.39964370152935e-10\\
-47.47927734375	-1.36063957271996e-10\\
-47.458779296875	-8.77792361906869e-11\\
-47.43828125	-1.85467833701982e-10\\
-47.417783203125	-1.06862863344413e-10\\
-47.39728515625	-1.23553555612857e-10\\
-47.376787109375	-1.26641307290727e-10\\
-47.3562890625	-1.51587171796695e-10\\
-47.335791015625	-1.99949146627952e-10\\
-47.31529296875	-2.27479944788248e-10\\
-47.294794921875	-2.10526185116155e-10\\
-47.274296875	-2.23811979637214e-10\\
-47.253798828125	-1.46569325791716e-10\\
-47.23330078125	-1.88224221468316e-10\\
-47.212802734375	-8.07640215896891e-11\\
-47.1923046875	-9.29015807357987e-11\\
-47.171806640625	-1.92774932787603e-11\\
-47.15130859375	-1.83646915994842e-11\\
-47.130810546875	-5.19044743822199e-11\\
-47.1103125	-7.46409089755521e-11\\
-47.089814453125	-1.23853739657538e-11\\
-47.06931640625	-4.77479755196377e-11\\
-47.048818359375	-5.44521074571839e-11\\
-47.0283203125	-2.41290617575454e-11\\
-47.007822265625	-9.02498510558779e-11\\
-46.98732421875	7.14215183875686e-11\\
-46.966826171875	6.75498386014551e-13\\
-46.946328125	9.9651447562819e-11\\
-46.925830078125	1.09079908174467e-10\\
-46.90533203125	2.21934627700921e-10\\
-46.884833984375	1.4700887264218e-10\\
-46.8643359375	1.5871594191208e-10\\
-46.843837890625	1.71152555322276e-10\\
-46.82333984375	1.52359829651946e-10\\
-46.802841796875	1.32578405295044e-10\\
-46.78234375	1.50263623436425e-10\\
-46.761845703125	2.2299218532693e-10\\
-46.74134765625	2.70153391478226e-10\\
-46.720849609375	2.14699478997746e-10\\
-46.7003515625	2.81758095389842e-10\\
-46.679853515625	2.74383331773193e-10\\
-46.65935546875	2.7258605376745e-10\\
-46.638857421875	2.79496242337175e-10\\
-46.618359375	2.08346375006837e-10\\
-46.597861328125	2.75839669378152e-10\\
-46.57736328125	2.73914883309658e-10\\
-46.556865234375	3.05739015983724e-10\\
-46.5363671875	3.55846212504023e-10\\
-46.515869140625	4.39509362008012e-10\\
-46.49537109375	4.25758082752511e-10\\
-46.474873046875	4.40967185232544e-10\\
-46.454375	3.72164124570923e-10\\
-46.433876953125	3.34052926500276e-10\\
-46.41337890625	2.83803529930271e-10\\
-46.392880859375	2.52268957719451e-10\\
-46.3723828125	1.71816623837271e-10\\
-46.351884765625	1.28855479767502e-10\\
-46.33138671875	1.8075058518828e-10\\
-46.310888671875	1.51351669063083e-10\\
-46.290390625	2.45678150334621e-10\\
-46.269892578125	2.31982190848082e-10\\
-46.24939453125	2.44894166855285e-10\\
-46.228896484375	1.76051400084553e-10\\
-46.2083984375	1.52745484473485e-10\\
-46.187900390625	7.19769331900981e-11\\
-46.16740234375	5.40941980367577e-11\\
-46.146904296875	3.16486725849255e-11\\
-46.12640625	-5.61075002970875e-11\\
-46.105908203125	-1.0915777617896e-10\\
-46.08541015625	-8.34315606761191e-11\\
-46.064912109375	-1.75003704575635e-11\\
-46.0444140625	-5.39835180888069e-11\\
-46.023916015625	-4.34590954858808e-11\\
-46.00341796875	-5.83523501914465e-11\\
-45.982919921875	-1.29297113348402e-10\\
-45.962421875	-1.66763893681193e-10\\
-45.941923828125	-1.44182958271443e-10\\
-45.92142578125	-1.2043949691265e-10\\
-45.900927734375	-1.28925402869162e-10\\
-45.8804296875	-1.01554174236212e-10\\
-45.859931640625	-9.76174684815775e-11\\
-45.83943359375	-1.87665459629072e-10\\
-45.818935546875	-6.46840604248631e-11\\
-45.7984375	-6.46215433826728e-11\\
-45.777939453125	-1.78695048235645e-11\\
-45.75744140625	-8.00601697819881e-11\\
-45.736943359375	-1.41617331056307e-10\\
-45.7164453125	-1.6515281559771e-10\\
-45.695947265625	-2.07461317237449e-10\\
-45.67544921875	-2.01781728407864e-10\\
-45.654951171875	-1.09618071835178e-10\\
-45.634453125	-1.82726911212765e-10\\
-45.613955078125	-1.1800124579516e-10\\
-45.59345703125	-6.79153772060677e-11\\
-45.572958984375	-7.11757437741872e-12\\
-45.5524609375	-7.33873348863849e-11\\
-45.531962890625	-5.90016674212971e-11\\
-45.51146484375	-9.33080522069948e-11\\
-45.490966796875	-7.34553024616156e-11\\
-45.47046875	-1.51053705972602e-10\\
-45.449970703125	-9.05887491202441e-11\\
-45.42947265625	-1.50284472306023e-10\\
-45.408974609375	-1.2377262670428e-10\\
-45.3884765625	-9.51403030751751e-11\\
-45.367978515625	-1.86224814530952e-10\\
-45.34748046875	-1.81072120112644e-10\\
-45.326982421875	-1.05417155184408e-10\\
-45.306484375	-2.22904102481793e-10\\
-45.285986328125	-1.59476250320702e-10\\
-45.26548828125	-1.56630002223744e-10\\
-45.244990234375	-1.71157241096894e-10\\
-45.2244921875	-1.66365097593585e-10\\
-45.203994140625	-1.64062780596579e-10\\
-45.18349609375	-1.563491824785e-10\\
-45.162998046875	-1.39374729909613e-10\\
-45.1425	-1.89742318953465e-10\\
-45.122001953125	-1.68236631450442e-10\\
-45.10150390625	-9.65704705358031e-11\\
-45.081005859375	-1.91174123812455e-10\\
-45.0605078125	-2.46878396182631e-11\\
-45.040009765625	-1.22270187485887e-10\\
-45.01951171875	-3.87977299426354e-12\\
-44.999013671875	-1.73987193539183e-10\\
-44.978515625	-1.69700334430719e-10\\
-44.958017578125	-1.85088629460084e-10\\
-44.93751953125	-1.59468140802528e-10\\
-44.917021484375	-2.20422310285062e-10\\
-44.8965234375	-1.12995199104762e-10\\
-44.876025390625	-2.107899984311e-10\\
-44.85552734375	-8.76504805329389e-11\\
-44.835029296875	-3.82187192024658e-11\\
-44.81453125	-3.05482864944862e-11\\
-44.794033203125	1.68000686776317e-11\\
-44.77353515625	-4.94165401719359e-11\\
-44.753037109375	-1.29589470351391e-11\\
-44.7325390625	-4.24035917476327e-11\\
-44.712041015625	-1.28176871179358e-10\\
-44.69154296875	-8.43421923102772e-11\\
-44.671044921875	-1.14937780696804e-10\\
-44.650546875	-3.04710417685558e-11\\
-44.630048828125	-6.98856382889648e-11\\
-44.60955078125	-5.70764510523244e-11\\
-44.589052734375	-1.12693122342651e-10\\
-44.5685546875	-9.93616506929683e-11\\
-44.548056640625	-9.44395414121439e-11\\
-44.52755859375	-9.7225894117133e-11\\
-44.507060546875	-1.46772625141573e-10\\
-44.4865625	-2.33699573668156e-10\\
-44.466064453125	-9.34003972203333e-11\\
-44.44556640625	-2.46039802393467e-10\\
-44.425068359375	4.27126755965892e-12\\
-44.4045703125	-1.63011305049394e-10\\
-44.384072265625	-7.34188324976539e-11\\
-44.36357421875	-1.59305427924388e-10\\
-44.343076171875	-1.77819533003364e-10\\
-44.322578125	-1.79714981509761e-10\\
-44.302080078125	-2.72134023003952e-10\\
-44.28158203125	-2.72938537888904e-10\\
-44.261083984375	-3.03388313413932e-10\\
-44.2405859375	-2.33206275092864e-10\\
-44.220087890625	-1.75288090943113e-10\\
-44.19958984375	-1.70488479663617e-10\\
-44.179091796875	-9.09146290915524e-11\\
-44.15859375	-1.03135323511149e-10\\
-44.138095703125	-1.22172661001084e-10\\
-44.11759765625	-1.56081986923805e-10\\
-44.097099609375	-1.9802340748749e-10\\
-44.0766015625	-2.39919008866864e-10\\
-44.056103515625	-3.18560589002798e-10\\
-44.03560546875	-3.02990842537389e-10\\
-44.015107421875	-2.78918187952124e-10\\
-43.994609375	-1.70697398629842e-10\\
-43.974111328125	-1.66534380971977e-10\\
-43.95361328125	-6.09628435884452e-11\\
-43.933115234375	-6.96378691265152e-11\\
-43.9126171875	-7.52410762374544e-12\\
-43.892119140625	-8.0219256201838e-11\\
-43.87162109375	-1.44315185760575e-10\\
-43.851123046875	-1.3898157068663e-10\\
-43.830625	-2.34835487599975e-10\\
-43.810126953125	-1.68042379145465e-10\\
-43.78962890625	-1.35855969321296e-10\\
-43.769130859375	5.70750900806057e-13\\
-43.7486328125	-1.85034701455156e-11\\
-43.728134765625	8.64467195996923e-11\\
-43.70763671875	1.0078720211488e-10\\
-43.687138671875	1.16908422623544e-10\\
-43.666640625	1.71506272163838e-10\\
-43.646142578125	1.32735743925711e-10\\
-43.62564453125	8.63688373143254e-11\\
-43.605146484375	2.57114693314608e-11\\
-43.5846484375	-2.7381603287378e-12\\
-43.564150390625	7.54909378089495e-11\\
-43.54365234375	1.32891752286754e-10\\
-43.523154296875	1.82533303417523e-10\\
-43.50265625	2.06770772389354e-10\\
-43.482158203125	2.81100634251449e-10\\
-43.46166015625	3.08297977862899e-10\\
-43.441162109375	2.78436386588071e-10\\
-43.4206640625	2.94568365950705e-10\\
-43.400166015625	1.94874511308612e-10\\
-43.37966796875	1.8745003429931e-10\\
-43.359169921875	2.0061308530159e-10\\
-43.338671875	2.06198649449925e-10\\
-43.318173828125	1.98306043892118e-10\\
-43.29767578125	2.16869062280162e-10\\
-43.277177734375	2.3088982364316e-10\\
-43.2566796875	2.52906333417691e-10\\
-43.236181640625	2.68816703719254e-10\\
-43.21568359375	2.65799174064098e-10\\
-43.195185546875	2.40287300485368e-10\\
-43.1746875	1.8417325835846e-10\\
-43.154189453125	1.66636956134297e-10\\
-43.13369140625	1.6727734462647e-10\\
-43.113193359375	6.15908862884959e-11\\
-43.0926953125	9.73327337426314e-11\\
-43.072197265625	8.15084703932963e-11\\
-43.05169921875	1.62606131777735e-10\\
-43.031201171875	1.08379035999676e-10\\
-43.010703125	1.14180815683523e-10\\
-42.990205078125	3.91864815561855e-11\\
-42.96970703125	1.46081002220508e-10\\
-42.949208984375	1.7394406816843e-11\\
-42.9287109375	4.55719257741036e-11\\
-42.908212890625	-3.76196153493767e-13\\
-42.88771484375	4.35629402214319e-11\\
-42.867216796875	5.67052060158337e-11\\
-42.84671875	4.13493123813237e-11\\
-42.826220703125	7.54871935615087e-12\\
-42.80572265625	9.33175441274144e-11\\
-42.785224609375	1.45681971101056e-11\\
-42.7647265625	1.20339088425569e-10\\
-42.744228515625	9.04768114592159e-11\\
-42.72373046875	8.89286618839751e-11\\
-42.703232421875	1.03925477447608e-10\\
-42.682734375	5.08391860636421e-11\\
-42.662236328125	6.54387144727438e-11\\
-42.64173828125	7.53689990009393e-11\\
-42.621240234375	2.8206375104362e-11\\
-42.6007421875	5.58297575512182e-11\\
-42.580244140625	2.15739194839052e-12\\
-42.55974609375	1.26109199482771e-11\\
-42.539248046875	1.35420195272907e-10\\
-42.51875	3.39957741156595e-12\\
-42.498251953125	1.25784078763694e-10\\
-42.47775390625	9.80499872428331e-11\\
-42.457255859375	2.69573311528824e-11\\
-42.4367578125	7.23136684203001e-11\\
-42.416259765625	6.01680008155737e-11\\
-42.39576171875	1.39809933853925e-12\\
-42.375263671875	5.71143024505259e-11\\
-42.354765625	-3.43409202083281e-11\\
-42.334267578125	2.09896716523753e-11\\
-42.31376953125	-2.87462247597426e-11\\
-42.293271484375	-8.91518796749338e-13\\
-42.2727734375	-3.84275750202111e-11\\
-42.252275390625	-5.20943997980071e-11\\
-42.23177734375	-4.92540564323501e-11\\
-42.211279296875	-5.5706163563988e-11\\
-42.19078125	-5.35155124842659e-11\\
-42.170283203125	-2.83890736503817e-11\\
-42.14978515625	-1.03148114022762e-10\\
-42.129287109375	-1.0860558919277e-10\\
-42.1087890625	-1.08129026809343e-10\\
-42.088291015625	1.77221875292916e-11\\
-42.06779296875	-1.26119770102478e-12\\
-42.047294921875	-3.76705009100405e-11\\
-42.026796875	1.06926391923895e-11\\
-42.006298828125	-7.42925435364923e-11\\
-41.98580078125	-5.13703905510795e-11\\
-41.965302734375	-3.25046739199112e-11\\
-41.9448046875	-4.20190866368884e-12\\
-41.924306640625	-1.58658101088329e-10\\
-41.90380859375	4.75681217750274e-11\\
-41.883310546875	-6.67471845017705e-11\\
-41.8628125	8.46665029290403e-11\\
-41.842314453125	-2.35447447951291e-11\\
-41.82181640625	6.7649945274316e-11\\
-41.801318359375	8.30193779385366e-11\\
-41.7808203125	5.6640075342587e-11\\
-41.760322265625	3.33445886559711e-11\\
-41.73982421875	7.58547285317205e-11\\
-41.719326171875	-5.2081191782355e-13\\
-41.698828125	1.79813039143014e-11\\
-41.678330078125	-7.97218355350971e-11\\
-41.65783203125	-1.71498492880976e-12\\
-41.637333984375	4.30316222311015e-11\\
-41.6168359375	1.79853284036644e-11\\
-41.596337890625	1.02288976026178e-10\\
-41.57583984375	7.90099835714467e-11\\
-41.555341796875	1.09755366617671e-10\\
-41.53484375	3.57430270705511e-11\\
-41.514345703125	5.09797893862575e-11\\
-41.49384765625	6.41001100426856e-11\\
-41.473349609375	5.2091677930509e-11\\
-41.4528515625	-4.33839479396308e-11\\
-41.432353515625	6.18830578460675e-11\\
-41.41185546875	2.99304031884978e-11\\
-41.391357421875	8.75789863327373e-11\\
-41.370859375	1.03226518296032e-11\\
-41.350361328125	9.61375134575839e-11\\
-41.32986328125	2.75150975292479e-12\\
-41.309365234375	-1.24915572559457e-10\\
-41.2888671875	-4.17575866966775e-11\\
-41.268369140625	-2.02576675465645e-10\\
-41.24787109375	-1.50380253509052e-10\\
-41.227373046875	-1.89730257515556e-10\\
-41.206875	-7.97710461634346e-11\\
-41.186376953125	-7.08171131077433e-11\\
-41.16587890625	-4.25903933771955e-11\\
-41.145380859375	-7.80892761450968e-12\\
-41.1248828125	-4.46946000914061e-11\\
-41.104384765625	-1.25617467850915e-10\\
-41.08388671875	-1.66789954766834e-10\\
-41.063388671875	-2.47912598965621e-10\\
-41.042890625	-2.63493745114281e-10\\
-41.022392578125	-2.82896400312868e-10\\
-41.00189453125	-3.52791446911246e-10\\
-40.981396484375	-2.94500592688265e-10\\
-40.9608984375	-2.64647396812159e-10\\
-40.940400390625	-3.21536075939014e-10\\
-40.91990234375	-3.36802635411836e-10\\
-40.899404296875	-2.95711988841558e-10\\
-40.87890625	-3.41320973356843e-10\\
-40.858408203125	-3.55923144003864e-10\\
-40.83791015625	-3.25161034026954e-10\\
-40.817412109375	-3.70911263303419e-10\\
-40.7969140625	-4.0770496526226e-10\\
-40.776416015625	-4.32316836471414e-10\\
-40.75591796875	-4.10399255614968e-10\\
-40.735419921875	-3.62887128240708e-10\\
-40.714921875	-4.31972132421927e-10\\
-40.694423828125	-4.03745660512117e-10\\
-40.67392578125	-4.39697567013521e-10\\
-40.653427734375	-5.01210382019694e-10\\
-40.6329296875	-4.24424389015381e-10\\
-40.612431640625	-3.92063202450775e-10\\
-40.59193359375	-4.18081749102953e-10\\
-40.571435546875	-3.82183157403138e-10\\
-40.5509375	-4.03079842989865e-10\\
-40.530439453125	-3.78149580736954e-10\\
-40.50994140625	-4.11677449689777e-10\\
-40.489443359375	-3.03049809743088e-10\\
-40.4689453125	-3.75777156540767e-10\\
-40.448447265625	-2.75633593037601e-10\\
-40.42794921875	-2.95071297716759e-10\\
-40.407451171875	-2.12140161628078e-10\\
-40.386953125	-2.53772938188991e-10\\
-40.366455078125	-3.12261522334413e-10\\
-40.34595703125	-3.24855092604323e-10\\
-40.325458984375	-2.85532505901919e-10\\
-40.3049609375	-3.62693563660593e-10\\
-40.284462890625	-2.7109594833159e-10\\
-40.26396484375	-3.13059533106582e-10\\
-40.243466796875	-1.74240081913453e-10\\
-40.22296875	-3.26289586191516e-10\\
-40.202470703125	-2.32820826917063e-10\\
-40.18197265625	-2.36584242895274e-10\\
-40.161474609375	-2.53560670292011e-10\\
-40.1409765625	-2.64483111797478e-10\\
-40.120478515625	-1.95470718142144e-10\\
-40.09998046875	-3.32677866245067e-10\\
-40.079482421875	-2.53365671693715e-10\\
-40.058984375	-3.11851886815887e-10\\
-40.038486328125	-2.79258280182164e-10\\
-40.01798828125	-2.35231403978289e-10\\
-39.997490234375	-3.78926361986378e-10\\
-39.9769921875	-1.55229338754134e-10\\
-39.956494140625	-2.99302467522856e-10\\
-39.93599609375	-2.65193456385683e-10\\
-39.915498046875	-3.07173667070403e-10\\
-39.895	-2.29104825832662e-10\\
-39.874501953125	-4.31940482921727e-10\\
-39.85400390625	-2.22436885320928e-10\\
-39.833505859375	-2.79517333366643e-10\\
-39.8130078125	-2.24130750732491e-10\\
-39.792509765625	-3.33032635995573e-10\\
-39.77201171875	-1.62850608037389e-10\\
-39.751513671875	-2.23820331046041e-10\\
-39.731015625	-2.00734533576301e-10\\
-39.710517578125	-1.44227870592141e-10\\
-39.69001953125	-2.06071801756148e-10\\
-39.669521484375	-1.95836249556781e-10\\
-39.6490234375	-1.18203839682934e-10\\
-39.628525390625	-2.40733771544017e-10\\
-39.60802734375	-1.17208049006199e-10\\
-39.587529296875	-1.35573729416346e-10\\
-39.56703125	-1.54825605041532e-10\\
-39.546533203125	-1.45010173380374e-10\\
-39.52603515625	-1.59993842461513e-10\\
-39.505537109375	-1.61673471374552e-10\\
-39.4850390625	-2.11800170570594e-10\\
-39.464541015625	-1.76435921487928e-10\\
-39.44404296875	-2.33714851042441e-10\\
-39.423544921875	-1.67677295735727e-10\\
-39.403046875	-2.63606005887881e-10\\
-39.382548828125	-1.09659661857086e-10\\
-39.36205078125	-2.45529971139778e-10\\
-39.341552734375	-1.058519445589e-10\\
-39.3210546875	-2.6993415704536e-10\\
-39.300556640625	-1.40795064053159e-10\\
-39.28005859375	-2.5194984561036e-10\\
-39.259560546875	-1.87427295278064e-10\\
-39.2390625	-2.92081362933305e-10\\
-39.218564453125	-2.2382510208881e-10\\
-39.19806640625	-1.98119320707778e-10\\
-39.177568359375	-2.16523555187087e-10\\
-39.1570703125	-2.72650814014248e-10\\
-39.136572265625	-1.68844305463519e-10\\
-39.11607421875	-3.23355866927323e-10\\
-39.095576171875	-2.84826596388668e-10\\
-39.075078125	-2.17152369276605e-10\\
-39.054580078125	-3.2591998859372e-10\\
-39.03408203125	-4.15541508334372e-10\\
-39.013583984375	-3.76668961893966e-10\\
-38.9930859375	-3.30463414435603e-10\\
-38.972587890625	-3.53906602252981e-10\\
-38.95208984375	-4.26992944754937e-10\\
-38.931591796875	-3.29165363839534e-10\\
-38.91109375	-2.68291440969643e-10\\
-38.890595703125	-2.94868089891037e-10\\
-38.87009765625	-3.03386586425663e-10\\
-38.849599609375	-2.89148678046843e-10\\
-38.8291015625	-2.54350810334442e-10\\
-38.808603515625	-3.67395972880495e-10\\
-38.78810546875	-3.91992065445774e-10\\
-38.767607421875	-2.4100141470735e-10\\
-38.747109375	-2.7749074488315e-10\\
-38.726611328125	-1.88497229204873e-10\\
-38.70611328125	-1.45173420263556e-10\\
-38.685615234375	-1.33376099860491e-10\\
-38.6651171875	-1.1408134554201e-10\\
-38.644619140625	-1.24482749625956e-10\\
-38.62412109375	-9.46822070124405e-11\\
-38.603623046875	-1.42529562991262e-10\\
-38.583125	-1.4505874667946e-10\\
-38.562626953125	-4.71287670044929e-11\\
-38.54212890625	-5.20207274483341e-11\\
-38.521630859375	4.09046061365757e-11\\
-38.5011328125	1.03973183844875e-10\\
-38.480634765625	4.83483378492824e-12\\
-38.46013671875	1.03130033143908e-11\\
-38.439638671875	3.43873234409207e-11\\
-38.419140625	3.82894236308931e-11\\
-38.398642578125	3.61975536991747e-11\\
-38.37814453125	4.62878889040539e-11\\
-38.357646484375	-6.41379367529644e-12\\
-38.3371484375	9.39218509269218e-11\\
-38.316650390625	4.5249524002794e-11\\
-38.29615234375	1.49009344884079e-10\\
-38.275654296875	1.30867716383921e-10\\
-38.25515625	1.12657448154008e-10\\
-38.234658203125	7.03260423884635e-11\\
-38.21416015625	1.74945804583649e-10\\
-38.193662109375	9.96276844408104e-11\\
-38.1731640625	1.73000321616676e-10\\
-38.152666015625	1.62412321808815e-10\\
-38.13216796875	2.000380261462e-10\\
-38.111669921875	2.73716424475172e-10\\
-38.091171875	1.59200418596048e-10\\
-38.070673828125	2.00447051638319e-10\\
-38.05017578125	2.46203050656437e-10\\
-38.029677734375	1.95704788119266e-10\\
-38.0091796875	2.76695474769099e-10\\
-37.988681640625	2.18047552143217e-10\\
-37.96818359375	2.03887597507391e-10\\
-37.947685546875	1.91039321568445e-10\\
-37.9271875	2.01059618406939e-10\\
-37.906689453125	2.46010398766634e-10\\
-37.88619140625	1.40099572281507e-10\\
-37.865693359375	7.43503627724476e-11\\
-37.8451953125	8.19113790421817e-11\\
-37.824697265625	2.94607871421857e-11\\
-37.80419921875	1.51391714390144e-10\\
-37.783701171875	2.28626569909449e-11\\
-37.763203125	2.20947119880333e-10\\
-37.742705078125	5.89082437733519e-11\\
-37.72220703125	1.70222704926209e-10\\
-37.701708984375	6.43257177504404e-11\\
-37.6812109375	9.38790831845613e-11\\
-37.660712890625	-1.12665107289164e-11\\
-37.64021484375	4.79102567554217e-11\\
-37.619716796875	3.69467692633099e-11\\
-37.59921875	3.83253847573362e-12\\
-37.578720703125	6.72817370958088e-11\\
-37.55822265625	6.91338882450983e-11\\
-37.537724609375	1.13804584050741e-10\\
-37.5172265625	1.00262224977274e-10\\
-37.496728515625	1.06231211609229e-10\\
-37.47623046875	9.05038699209426e-11\\
-37.455732421875	8.6286042273127e-11\\
-37.435234375	-7.32388560086488e-11\\
-37.414736328125	7.18150019763915e-11\\
-37.39423828125	3.82488202463592e-11\\
-37.373740234375	1.82615888533702e-10\\
-37.3532421875	7.97574641086942e-11\\
-37.332744140625	2.44497498147001e-10\\
-37.31224609375	4.83904764405608e-11\\
-37.291748046875	1.85358296499497e-10\\
-37.27125	-5.84344410718559e-11\\
-37.250751953125	7.79092009841314e-11\\
-37.23025390625	-1.212969547652e-10\\
-37.209755859375	-7.06471257683655e-11\\
-37.1892578125	-1.31013336558734e-10\\
-37.168759765625	-1.84557624472317e-10\\
-37.14826171875	-4.93202579718116e-11\\
-37.127763671875	-2.72558161937925e-11\\
-37.107265625	-3.5776874428964e-11\\
-37.086767578125	-5.82211246964659e-11\\
-37.06626953125	-2.63002285144156e-11\\
-37.045771484375	-8.85701180598952e-11\\
-37.0252734375	-1.2387264314426e-10\\
-37.004775390625	-9.43784941405918e-11\\
-36.98427734375	-1.10110662572702e-10\\
-36.963779296875	-1.19727470092679e-10\\
-36.94328125	-8.38830441259712e-11\\
-36.922783203125	2.12514487474505e-12\\
-36.90228515625	4.09425612094497e-11\\
-36.881787109375	2.36991738061953e-11\\
-36.8612890625	5.84432531413368e-11\\
-36.840791015625	-6.15344297603022e-11\\
-36.82029296875	1.37824706232304e-10\\
-36.799794921875	-1.019797725903e-10\\
-36.779296875	3.2159058768942e-11\\
-36.758798828125	-1.23334065489452e-10\\
-36.73830078125	3.71828071017845e-11\\
-36.717802734375	1.4488686105761e-11\\
-36.6973046875	8.72153538969797e-12\\
-36.676806640625	1.06200929836358e-10\\
-36.65630859375	1.49726397764548e-10\\
-36.635810546875	6.49241859726044e-11\\
-36.6153125	2.24434346369326e-10\\
-36.594814453125	1.00849352120549e-10\\
-36.57431640625	1.34220096381132e-10\\
-36.553818359375	6.29185357731046e-11\\
-36.5333203125	1.42468298133508e-10\\
-36.512822265625	5.52756820168553e-11\\
-36.49232421875	1.48478601153346e-10\\
-36.471826171875	1.29929756013825e-10\\
-36.451328125	2.68652546861895e-10\\
-36.430830078125	1.63235664867152e-10\\
-36.41033203125	1.87650593474265e-10\\
-36.389833984375	1.82276762943576e-11\\
-36.3693359375	1.1965557255204e-10\\
-36.348837890625	2.1424736303768e-11\\
-36.32833984375	7.59518321805785e-11\\
-36.307841796875	3.421353543801e-11\\
-36.28734375	-3.89194636686715e-11\\
-36.266845703125	3.46588507445628e-11\\
-36.24634765625	6.3063979072795e-11\\
-36.225849609375	8.57933086034676e-12\\
-36.2053515625	2.45972379558824e-11\\
-36.184853515625	-5.45482939957858e-11\\
-36.16435546875	-3.05470128903915e-11\\
-36.143857421875	-4.36426381847847e-12\\
-36.123359375	-1.21167789022449e-10\\
-36.102861328125	-7.47399030211622e-11\\
-36.08236328125	-1.78838283627482e-10\\
-36.061865234375	-1.31358210136702e-10\\
-36.0413671875	-1.32648472043892e-10\\
-36.020869140625	-1.85368525708294e-10\\
-36.00037109375	-1.93756360082739e-10\\
-35.979873046875	-1.89446413158324e-10\\
-35.959375	-3.13946865600641e-10\\
-35.938876953125	-2.10746681090881e-10\\
-35.91837890625	-3.04570619071647e-10\\
-35.897880859375	-2.60110681163038e-10\\
-35.8773828125	-2.45977444659216e-10\\
-35.856884765625	-2.27142467602347e-10\\
-35.83638671875	-2.72160203091543e-10\\
-35.815888671875	-1.76843962197611e-10\\
-35.795390625	-2.19242361712812e-10\\
-35.774892578125	-1.45054786593115e-10\\
-35.75439453125	-3.15904488040781e-10\\
-35.733896484375	-2.2244734216784e-10\\
-35.7133984375	-2.64372481762778e-10\\
-35.692900390625	-2.34569401662901e-10\\
-35.67240234375	-2.44030415911244e-10\\
-35.651904296875	-1.38751304165167e-10\\
-35.63140625	-2.41352971979471e-10\\
-35.610908203125	-2.82729133405877e-10\\
-35.59041015625	-1.97052494142862e-10\\
-35.569912109375	-2.65861880468044e-10\\
-35.5494140625	-2.35757909917697e-10\\
-35.528916015625	-2.94079380066253e-10\\
-35.50841796875	-3.6824099875944e-10\\
-35.487919921875	-2.87991961853027e-10\\
-35.467421875	-3.22129357013582e-10\\
-35.446923828125	-2.96561671131761e-10\\
-35.42642578125	-3.46221187438232e-10\\
-35.405927734375	-3.47725828853588e-10\\
-35.3854296875	-3.30027573591559e-10\\
-35.364931640625	-1.73731284123401e-10\\
-35.34443359375	-2.2244846189882e-10\\
-35.323935546875	-1.3696888385376e-10\\
-35.3034375	-2.27281404675606e-10\\
-35.282939453125	-1.36813509408624e-10\\
-35.26244140625	-2.77858960865139e-10\\
-35.241943359375	-8.80690603536957e-11\\
-35.2214453125	-2.65461955741111e-10\\
-35.200947265625	-1.68241019058176e-10\\
-35.18044921875	-3.22660094506431e-10\\
-35.159951171875	-2.29349846743522e-10\\
-35.139453125	-3.19987671575588e-10\\
-35.118955078125	-2.1322181320087e-10\\
-35.09845703125	-2.84919654364432e-10\\
-35.077958984375	-1.54620530829645e-10\\
-35.0574609375	-1.87558125755801e-10\\
-35.036962890625	-1.68784672592815e-10\\
-35.01646484375	-1.67637237930322e-10\\
-34.995966796875	-2.29276656316151e-10\\
-34.97546875	-2.05317415452382e-10\\
-34.954970703125	-2.08408669519605e-10\\
-34.93447265625	-1.56402916505216e-10\\
-34.913974609375	-2.7248123319837e-10\\
-34.8934765625	-1.90715752166324e-10\\
-34.872978515625	-2.64549420315457e-10\\
-34.85248046875	-1.1005499935472e-10\\
-34.831982421875	-1.8915989678251e-10\\
-34.811484375	-1.15919103174151e-10\\
-34.790986328125	-8.02725039962858e-11\\
-34.77048828125	3.35815758696776e-12\\
-34.749990234375	-3.80005475644415e-11\\
-34.7294921875	1.17638321855294e-11\\
-34.708994140625	1.05857647792467e-11\\
-34.68849609375	1.01696273460718e-10\\
-34.667998046875	-3.60314018279977e-11\\
-34.6475	4.69420065244623e-11\\
-34.627001953125	1.12810805637331e-10\\
-34.60650390625	9.01424204601594e-11\\
-34.586005859375	1.27445728900914e-10\\
-34.5655078125	7.14019116154892e-11\\
-34.545009765625	1.80761493287762e-10\\
-34.52451171875	1.82046135889404e-10\\
-34.504013671875	2.4382118006815e-10\\
-34.483515625	2.3291571031761e-10\\
-34.463017578125	1.91613471082963e-10\\
-34.44251953125	2.10283566342071e-10\\
-34.422021484375	1.8747382751256e-10\\
-34.4015234375	1.40351325728888e-10\\
-34.381025390625	2.38483463547974e-10\\
-34.36052734375	9.85793181454672e-11\\
-34.340029296875	2.21674745698262e-10\\
-34.31953125	8.21320997704624e-11\\
-34.299033203125	1.69674901260055e-10\\
-34.27853515625	1.31037196265216e-12\\
-34.258037109375	4.47829753485436e-11\\
-34.2375390625	-1.17343873801697e-10\\
-34.217041015625	-5.31599541020613e-11\\
-34.19654296875	-1.29266027056518e-10\\
-34.176044921875	-1.23763339429829e-11\\
-34.155546875	-4.45441571997517e-11\\
-34.135048828125	-1.43784132530104e-10\\
-34.11455078125	-1.12372965999297e-10\\
-34.094052734375	-1.90697208604662e-11\\
-34.0735546875	-1.91233040540656e-10\\
-34.053056640625	-1.4394844140838e-10\\
-34.03255859375	-3.03711649215377e-10\\
-34.012060546875	-2.33192413387561e-10\\
-33.9915625	-2.17612319505937e-10\\
-33.971064453125	-2.30030716728244e-10\\
-33.95056640625	-2.11091935515531e-10\\
-33.930068359375	-1.63547356198702e-10\\
-33.9095703125	-2.0003572298937e-10\\
-33.889072265625	1.79010192067642e-11\\
-33.86857421875	-8.44236816717907e-11\\
-33.848076171875	2.41548725372791e-11\\
-33.827578125	-1.451206047699e-10\\
-33.807080078125	-9.38109569454244e-11\\
-33.78658203125	-8.66613773231233e-11\\
-33.766083984375	-1.2993098653522e-10\\
-33.7455859375	-3.32939928262442e-11\\
-33.725087890625	1.32564627487682e-11\\
-33.70458984375	1.15287461193302e-11\\
-33.684091796875	3.16895539212481e-11\\
-33.66359375	6.58842948530677e-11\\
-33.643095703125	7.20805593406148e-11\\
-33.62259765625	1.80442932618295e-11\\
-33.602099609375	-1.1726651752777e-10\\
-33.5816015625	-2.05172145649521e-11\\
-33.561103515625	1.50813802969478e-11\\
-33.54060546875	3.48016281098005e-11\\
-33.520107421875	1.18386116794343e-10\\
-33.499609375	1.92521177868274e-10\\
-33.479111328125	2.46483411518674e-10\\
-33.45861328125	2.49863887876151e-10\\
-33.438115234375	2.64599990303974e-10\\
-33.4176171875	3.92123099158466e-10\\
-33.397119140625	2.33359805254272e-10\\
-33.37662109375	2.58425101183085e-10\\
-33.356123046875	1.78073847927076e-10\\
-33.335625	2.25166699017868e-10\\
-33.315126953125	1.36589852197979e-10\\
-33.29462890625	1.51322890307245e-10\\
-33.274130859375	1.0698245697143e-10\\
-33.2536328125	2.132185615783e-10\\
-33.233134765625	2.01887837192646e-10\\
-33.21263671875	2.84470907312481e-10\\
-33.192138671875	2.35904334382018e-10\\
-33.171640625	2.28917352151971e-10\\
-33.151142578125	1.68237623609121e-10\\
-33.13064453125	1.76692630549994e-10\\
-33.110146484375	1.69203201801323e-10\\
-33.0896484375	1.76351573602819e-10\\
-33.069150390625	1.51192198750007e-10\\
-33.04865234375	2.32551563648404e-10\\
-33.028154296875	2.18804460550951e-10\\
-33.00765625	2.47626317775727e-10\\
-32.987158203125	2.76949779597622e-10\\
-32.96666015625	2.86177238469753e-10\\
-32.946162109375	1.67783366967882e-10\\
-32.9256640625	1.85294298971715e-10\\
-32.905166015625	1.21145439813666e-10\\
-32.88466796875	2.04202406645455e-10\\
-32.864169921875	1.06394505446589e-10\\
-32.843671875	2.43256343494502e-10\\
-32.823173828125	1.94688989477166e-10\\
-32.80267578125	2.50657737136572e-10\\
-32.782177734375	1.50183773399713e-10\\
-32.7616796875	2.14929264756756e-10\\
-32.741181640625	1.25886972029337e-10\\
-32.72068359375	1.91857563362932e-10\\
-32.700185546875	9.39655711516493e-11\\
-32.6796875	2.21548225385039e-10\\
-32.659189453125	8.10567332998721e-11\\
-32.63869140625	1.97828269279178e-10\\
-32.618193359375	1.09768444992132e-10\\
-32.5976953125	2.83641474833096e-10\\
-32.577197265625	1.36833704563114e-10\\
-32.55669921875	2.66470369785803e-10\\
-32.536201171875	7.63067005612642e-11\\
-32.515703125	1.34500045611448e-10\\
-32.495205078125	8.53008970778084e-11\\
-32.47470703125	3.88868729879023e-11\\
-32.454208984375	8.16072469724982e-11\\
-32.4337109375	2.2036576484755e-11\\
-32.413212890625	7.80654326920302e-11\\
-32.39271484375	-3.11572156874141e-11\\
-32.372216796875	1.70934279069003e-10\\
-32.35171875	-2.85030334222485e-12\\
-32.331220703125	1.54271922417094e-10\\
-32.31072265625	-6.1201312581544e-11\\
-32.290224609375	2.30647919978706e-11\\
-32.2697265625	-1.28691780956718e-10\\
-32.249228515625	-7.44026541461635e-11\\
-32.22873046875	-2.33946413779677e-10\\
-32.208232421875	-2.00365224654428e-10\\
-32.187734375	-2.07012084400355e-10\\
-32.167236328125	-3.08065071061691e-10\\
-32.14673828125	-3.2923293162217e-10\\
-32.126240234375	-3.05024166107625e-10\\
-32.1057421875	-3.18057421924298e-10\\
-32.085244140625	-3.6248964005213e-10\\
-32.06474609375	-2.93390451057643e-10\\
-32.044248046875	-4.2341561187878e-10\\
-32.02375	-4.23715313243362e-10\\
-32.003251953125	-5.06210503935136e-10\\
-31.98275390625	-5.30546865275647e-10\\
-31.962255859375	-5.47588901843848e-10\\
-31.9417578125	-5.81019816827201e-10\\
-31.921259765625	-5.94841189445576e-10\\
-31.90076171875	-5.59996909621933e-10\\
-31.880263671875	-6.43438408633824e-10\\
-31.859765625	-4.92462302341826e-10\\
-31.839267578125	-6.16928984391516e-10\\
-31.81876953125	-3.77264983205633e-10\\
-31.798271484375	-4.91722395618548e-10\\
-31.7777734375	-2.95471999425628e-10\\
-31.757275390625	-4.22805014321533e-10\\
-31.73677734375	-2.49312334246659e-10\\
-31.716279296875	-3.24782416030547e-10\\
-31.69578125	-1.72536215398124e-10\\
-31.675283203125	-3.59465194175305e-10\\
-31.65478515625	-2.52611592345664e-10\\
-31.634287109375	-2.94052545272836e-10\\
-31.6137890625	-1.99942513210539e-10\\
-31.593291015625	-2.78231513664007e-10\\
-31.57279296875	-1.37055977445849e-10\\
-31.552294921875	-1.90398891933584e-10\\
-31.531796875	-1.49192812745467e-10\\
-31.511298828125	-1.47733674267556e-10\\
-31.49080078125	-3.29329519324962e-11\\
-31.470302734375	-1.84050936607243e-10\\
-31.4498046875	-1.12259430589829e-10\\
-31.429306640625	-1.34167611714569e-10\\
-31.40880859375	-1.63093687549816e-10\\
-31.388310546875	-2.3720690610579e-10\\
-31.3678125	-1.20792052370813e-10\\
-31.347314453125	-3.08392290891971e-10\\
-31.32681640625	-1.56646901572918e-10\\
-31.306318359375	-2.01766509344413e-10\\
-31.2858203125	-1.7072909800821e-10\\
-31.265322265625	-1.7784208648308e-10\\
-31.24482421875	-2.1317930276112e-10\\
-31.224326171875	-2.58112018997327e-10\\
-31.203828125	-2.86220653895345e-10\\
-31.183330078125	-3.92372042033462e-10\\
-31.16283203125	-3.67306136980351e-10\\
-31.142333984375	-3.33641411889008e-10\\
-31.1218359375	-2.6014788230901e-10\\
-31.101337890625	-2.40356523147226e-10\\
-31.08083984375	-2.96380845767929e-10\\
-31.060341796875	-2.46275562999253e-10\\
-31.03984375	-3.59734436378868e-10\\
-31.019345703125	-2.99128532137434e-10\\
-30.99884765625	-3.22804478134041e-10\\
-30.978349609375	-4.2360450605639e-10\\
-30.9578515625	-4.44236035864364e-10\\
-30.937353515625	-5.78690558472403e-10\\
-30.91685546875	-5.6138601141847e-10\\
-30.896357421875	-5.43100574569932e-10\\
-30.875859375	-6.41766084862431e-10\\
-30.855361328125	-4.65045147091411e-10\\
-30.83486328125	-4.44513415492352e-10\\
-30.814365234375	-3.92473271732638e-10\\
-30.7938671875	-4.09989779841488e-10\\
-30.773369140625	-3.74040894078568e-10\\
-30.75287109375	-4.53415752285521e-10\\
-30.732373046875	-3.66659183530585e-10\\
-30.711875	-5.33251310745361e-10\\
-30.691376953125	-5.39351115960254e-10\\
-30.67087890625	-6.01175376546962e-10\\
-30.650380859375	-5.97111200560555e-10\\
-30.6298828125	-5.47716573853284e-10\\
-30.609384765625	-5.2321029719581e-10\\
-30.58888671875	-5.13879191152601e-10\\
-30.568388671875	-4.52918471821387e-10\\
-30.547890625	-4.61064392564769e-10\\
-30.527392578125	-4.16204047868864e-10\\
-30.50689453125	-4.51602654039996e-10\\
-30.486396484375	-4.83309350291469e-10\\
-30.4658984375	-5.40480676181337e-10\\
-30.445400390625	-6.1240291802484e-10\\
-30.42490234375	-6.9056551361534e-10\\
-30.404404296875	-5.87917001825043e-10\\
-30.38390625	-6.85396890488235e-10\\
-30.363408203125	-6.11096519873672e-10\\
-30.34291015625	-6.60744014741398e-10\\
-30.322412109375	-5.41745360567067e-10\\
-30.3019140625	-6.43126551553997e-10\\
-30.281416015625	-4.71974244081527e-10\\
-30.26091796875	-4.78161569860462e-10\\
-30.240419921875	-5.08157623839926e-10\\
-30.219921875	-5.398599257645e-10\\
-30.199423828125	-5.25976707357913e-10\\
-30.17892578125	-7.09746623814949e-10\\
-30.158427734375	-5.66006378465421e-10\\
-30.1379296875	-7.386472246946e-10\\
-30.117431640625	-5.97451457357162e-10\\
-30.09693359375	-6.76122427412786e-10\\
-30.076435546875	-6.43207344825355e-10\\
-30.0559375	-6.33724225825745e-10\\
-30.035439453125	-4.99624313643317e-10\\
-30.01494140625	-6.94868661500714e-10\\
-29.994443359375	-6.15971068556158e-10\\
-29.9739453125	-6.70398013307514e-10\\
-29.953447265625	-8.2191411970249e-10\\
-29.93294921875	-7.10587769169451e-10\\
-29.912451171875	-8.22347206885721e-10\\
-29.891953125	-5.89459813338398e-10\\
-29.871455078125	-8.03939222241849e-10\\
-29.85095703125	-5.79338260989873e-10\\
-29.830458984375	-6.73121525878508e-10\\
-29.8099609375	-5.29587817419377e-10\\
-29.789462890625	-6.66625333128542e-10\\
-29.76896484375	-5.43093037658006e-10\\
-29.748466796875	-6.65176270958881e-10\\
-29.72796875	-5.08644124530748e-10\\
-29.707470703125	-5.7688430968257e-10\\
-29.68697265625	-5.04945763904081e-10\\
-29.666474609375	-4.81788006758791e-10\\
-29.6459765625	-3.63061058343069e-10\\
-29.625478515625	-3.25093328380624e-10\\
-29.60498046875	-1.72785702956999e-10\\
-29.584482421875	-1.31288840481602e-10\\
-29.563984375	-9.67179461755092e-11\\
-29.543486328125	-1.24703786805788e-10\\
-29.52298828125	-1.6551457359748e-10\\
-29.502490234375	-1.74441011826686e-11\\
-29.4819921875	-2.62498089193254e-11\\
-29.461494140625	9.63971093953089e-11\\
-29.44099609375	5.53762791164291e-11\\
-29.420498046875	1.16970632131207e-10\\
-29.4	2.03674039102718e-10\\
-29.379501953125	2.14407497326207e-10\\
-29.35900390625	1.62892183562076e-10\\
-29.338505859375	2.65368431708031e-10\\
-29.3180078125	1.13864317534071e-10\\
-29.297509765625	2.37187254305105e-10\\
-29.27701171875	8.08348712407892e-11\\
-29.256513671875	1.59844583958699e-10\\
-29.236015625	1.09628545337205e-10\\
-29.215517578125	1.74825477734886e-10\\
-29.19501953125	1.05878692152172e-10\\
-29.174521484375	3.03531976236528e-10\\
-29.1540234375	1.05831251292924e-10\\
-29.133525390625	1.98919002915899e-10\\
-29.11302734375	4.00004880522366e-11\\
-29.092529296875	-6.03547658398e-12\\
-29.07203125	-8.271192994848e-11\\
-29.051533203125	-5.6832958419947e-11\\
-29.03103515625	-1.12767891878213e-10\\
-29.010537109375	-9.80605128115017e-11\\
-28.9900390625	-1.80665865413098e-10\\
-28.969541015625	-1.3348406408401e-10\\
-28.94904296875	-1.48348869323676e-10\\
-28.928544921875	-7.89883045249923e-11\\
-28.908046875	-1.38629759545825e-10\\
-28.887548828125	-1.5707125970648e-10\\
-28.86705078125	-2.30947743073988e-10\\
-28.846552734375	-1.38067239811052e-10\\
-28.8260546875	-2.91833799222184e-10\\
-28.805556640625	-1.20503155860695e-10\\
-28.78505859375	-2.05707527312886e-10\\
-28.764560546875	-3.95877339096604e-11\\
-28.7440625	-9.50352341733775e-11\\
-28.723564453125	-1.02308670982197e-10\\
-28.70306640625	-1.08361157550089e-10\\
-28.682568359375	1.30220776082879e-11\\
-28.6620703125	-3.62605524171018e-11\\
-28.641572265625	3.08225403563653e-11\\
-28.62107421875	7.25884321330888e-11\\
-28.600576171875	6.23901725022546e-11\\
-28.580078125	7.85144799853516e-11\\
-28.559580078125	-1.5022789712931e-11\\
-28.53908203125	4.41228276978601e-11\\
-28.518583984375	-5.36149673750907e-11\\
-28.4980859375	-1.82438886398194e-11\\
-28.477587890625	9.79966958210713e-12\\
-28.45708984375	1.63025208292688e-10\\
-28.436591796875	1.92612438206156e-10\\
-28.41609375	2.167741293193e-10\\
-28.395595703125	3.28203802220666e-10\\
-28.37509765625	3.44320731897049e-10\\
-28.354599609375	2.10497358365128e-10\\
-28.3341015625	3.15752964519593e-10\\
-28.313603515625	1.75792115772226e-10\\
-28.29310546875	9.87682029592843e-11\\
-28.272607421875	4.28532093573201e-11\\
-28.252109375	5.68336427805671e-11\\
-28.231611328125	4.10728885136771e-11\\
-28.21111328125	3.09786633703127e-10\\
-28.190615234375	2.02135334886325e-10\\
-28.1701171875	3.68011409974797e-10\\
-28.149619140625	3.73556409909117e-10\\
-28.12912109375	3.68424759131772e-10\\
-28.108623046875	3.33497756916911e-10\\
-28.088125	3.41319863511745e-10\\
-28.067626953125	2.43350446703982e-10\\
-28.04712890625	1.93641100503023e-10\\
-28.026630859375	8.16713825613025e-11\\
-28.0061328125	2.44102805773459e-10\\
-27.985634765625	3.43031577115334e-10\\
-27.96513671875	3.96390785619135e-10\\
-27.944638671875	5.25598240259149e-10\\
-27.924140625	5.66915962461263e-10\\
-27.903642578125	3.87294225925344e-10\\
-27.88314453125	4.64329692584739e-10\\
-27.862646484375	3.77824068698341e-10\\
-27.8421484375	3.58822991543999e-10\\
-27.821650390625	2.44413461044083e-10\\
-27.80115234375	3.05624048168844e-10\\
-27.780654296875	3.14902747105997e-10\\
-27.76015625	3.87427289966702e-10\\
-27.739658203125	3.25468265471835e-10\\
-27.71916015625	3.84690905644364e-10\\
-27.698662109375	3.62692516407118e-10\\
-27.6781640625	4.95441701982264e-10\\
-27.657666015625	4.08646117788393e-10\\
-27.63716796875	4.93570992632801e-10\\
-27.616669921875	2.87585596638182e-10\\
-27.596171875	4.24473830303618e-10\\
-27.575673828125	3.14146641158087e-10\\
-27.55517578125	4.16804996648024e-10\\
-27.534677734375	3.31228159579342e-10\\
-27.5141796875	3.53864040336441e-10\\
-27.493681640625	3.80176515974373e-10\\
-27.47318359375	4.3455484971258e-10\\
-27.452685546875	5.00312676323037e-10\\
-27.4321875	4.31263093352119e-10\\
-27.411689453125	5.51775132350384e-10\\
-27.39119140625	3.82395165017079e-10\\
-27.370693359375	4.25972591410735e-10\\
-27.3501953125	3.11864148222014e-10\\
-27.329697265625	4.91986054395607e-10\\
-27.30919921875	2.34549310702442e-10\\
-27.288701171875	4.30409903316231e-10\\
-27.268203125	2.41458670941011e-10\\
-27.247705078125	3.80058041424944e-10\\
-27.22720703125	2.57497233398809e-10\\
-27.206708984375	3.15041460193005e-10\\
-27.1862109375	1.78613261211203e-10\\
-27.165712890625	2.22218780147741e-10\\
-27.14521484375	1.41275906659502e-10\\
-27.124716796875	5.23228013707437e-11\\
-27.10421875	5.50512914704859e-11\\
-27.083720703125	5.77374376288308e-13\\
-27.06322265625	-1.43505694365446e-10\\
-27.042724609375	-1.27880964916839e-10\\
-27.0222265625	-2.07298536785748e-10\\
-27.001728515625	-1.85180934364783e-10\\
-26.98123046875	-1.96571105588254e-10\\
-26.960732421875	-2.32933806927177e-10\\
-26.940234375	-2.27863748813029e-10\\
-26.919736328125	-3.64130847777452e-10\\
-26.89923828125	-4.20927804864054e-10\\
-26.878740234375	-5.29064101665882e-10\\
-26.8582421875	-5.47644604732973e-10\\
-26.837744140625	-6.55520232803403e-10\\
-26.81724609375	-4.74444588300903e-10\\
-26.796748046875	-6.22542857582195e-10\\
-26.77625	-3.600075668467e-10\\
-26.755751953125	-4.67281823535086e-10\\
-26.73525390625	-2.26310585744298e-10\\
-26.714755859375	-3.89829618151522e-10\\
-26.6942578125	-1.77248013824442e-10\\
-26.673759765625	-3.53405854241671e-10\\
-26.65326171875	-2.95863910602781e-10\\
-26.632763671875	-5.23094188727653e-10\\
-26.612265625	-3.7715456637735e-10\\
-26.591767578125	-5.32332668129498e-10\\
-26.57126953125	-4.33783743315569e-10\\
-26.550771484375	-4.38126440193259e-10\\
-26.5302734375	-2.35454321730349e-10\\
-26.509775390625	-2.2138745253265e-10\\
-26.48927734375	-1.58635343314882e-10\\
-26.468779296875	-1.20443727840783e-10\\
-26.44828125	-2.95954455263914e-11\\
-26.427783203125	-1.69656910461383e-10\\
-26.40728515625	-1.49640448149117e-10\\
-26.386787109375	-7.71708789709696e-11\\
-26.3662890625	-1.97336397813891e-10\\
-26.345791015625	-1.98963610175176e-10\\
-26.32529296875	-1.60901641565084e-10\\
-26.304794921875	-1.55368358245746e-10\\
-26.284296875	-4.49082121857232e-11\\
-26.263798828125	-2.33367307264461e-10\\
-26.24330078125	-1.31445902528433e-11\\
-26.222802734375	-1.25387998546557e-10\\
-26.2023046875	-1.26628094086963e-10\\
-26.181806640625	-1.97587739069607e-10\\
-26.16130859375	-1.55093920548396e-10\\
-26.140810546875	-2.34958575223599e-10\\
-26.1203125	-2.44436527481089e-10\\
-26.099814453125	-2.11852166625443e-10\\
-26.07931640625	-1.85114070898158e-10\\
-26.058818359375	-1.53456012311295e-10\\
-26.0383203125	-1.93708471760091e-10\\
-26.017822265625	-1.73814999117219e-10\\
-25.99732421875	-1.73632647692179e-10\\
-25.976826171875	-1.90947052169913e-10\\
-25.956328125	-2.86382183801622e-10\\
-25.935830078125	-4.39261309617766e-10\\
-25.91533203125	-4.80917808632753e-10\\
-25.894833984375	-5.81911272481082e-10\\
-25.8743359375	-5.49797181590928e-10\\
-25.853837890625	-6.06338612467341e-10\\
-25.83333984375	-6.47662212101812e-10\\
-25.812841796875	-5.34738971738871e-10\\
-25.79234375	-5.7497744240267e-10\\
-25.771845703125	-5.63671676454569e-10\\
-25.75134765625	-6.16792576046291e-10\\
-25.730849609375	-5.68694401316383e-10\\
-25.7103515625	-7.21905238973256e-10\\
-25.689853515625	-5.81144045090617e-10\\
-25.66935546875	-7.96104755847739e-10\\
-25.648857421875	-6.80409016494201e-10\\
-25.628359375	-6.28111759757319e-10\\
-25.607861328125	-7.65591341472487e-10\\
-25.58736328125	-6.52607505190426e-10\\
-25.566865234375	-6.95695847988677e-10\\
-25.5463671875	-7.23974665084902e-10\\
-25.525869140625	-5.91443870986594e-10\\
-25.50537109375	-6.9585384111776e-10\\
-25.484873046875	-6.38227837283618e-10\\
-25.464375	-7.26888388987428e-10\\
-25.443876953125	-8.37359129403379e-10\\
-25.42337890625	-8.76180885738629e-10\\
-25.402880859375	-8.29863655757017e-10\\
-25.3823828125	-8.70030285371956e-10\\
-25.361884765625	-7.14211369196505e-10\\
-25.34138671875	-7.75759186397001e-10\\
-25.320888671875	-6.13954280005404e-10\\
-25.300390625	-6.01735023921947e-10\\
-25.279892578125	-5.19907426235139e-10\\
-25.25939453125	-7.41643216034341e-10\\
-25.238896484375	-6.16389463340787e-10\\
-25.2183984375	-8.58942832840968e-10\\
-25.197900390625	-7.86465433812048e-10\\
-25.17740234375	-7.79604706534199e-10\\
-25.156904296875	-7.49200659315848e-10\\
-25.13640625	-7.57767369138055e-10\\
-25.115908203125	-4.95482024229156e-10\\
-25.09541015625	-4.74136970004202e-10\\
-25.074912109375	-4.06120510865629e-10\\
-25.0544140625	-5.10360331108694e-10\\
-25.033916015625	-3.97520790569756e-10\\
-25.01341796875	-5.70809312798279e-10\\
-24.992919921875	-5.84543385213371e-10\\
-24.972421875	-6.90630947470857e-10\\
-24.951923828125	-5.75314441182521e-10\\
-24.93142578125	-6.33000686191616e-10\\
-24.910927734375	-7.45671598587667e-10\\
-24.8904296875	-5.69710022077677e-10\\
-24.869931640625	-4.97626568851973e-10\\
-24.84943359375	-3.46485845282948e-10\\
-24.828935546875	-4.37802139098719e-10\\
-24.8084375	-2.19652844761408e-10\\
-24.787939453125	-3.50807675775231e-10\\
-24.76744140625	-1.62823692629748e-10\\
-24.746943359375	-3.36093661706271e-10\\
-24.7264453125	-1.40781420427282e-10\\
-24.705947265625	-3.35451746652683e-10\\
-24.68544921875	-2.01976077971607e-10\\
-24.664951171875	-2.86580453061014e-10\\
-24.644453125	-1.17007653542985e-10\\
-24.623955078125	-1.02237102083155e-10\\
-24.60345703125	6.49557993842721e-11\\
-24.582958984375	9.25344116449014e-11\\
-24.5624609375	2.68848960003735e-10\\
-24.541962890625	3.23359949727907e-10\\
-24.52146484375	3.75101311602251e-10\\
-24.500966796875	4.02466918064426e-10\\
-24.48046875	4.05680923281675e-10\\
-24.459970703125	3.89282341787163e-10\\
-24.43947265625	4.03887360594605e-10\\
-24.418974609375	5.08250497553544e-10\\
-24.3984765625	5.90721690758441e-10\\
-24.377978515625	6.93615558584366e-10\\
-24.35748046875	7.49983199602069e-10\\
-24.336982421875	8.36571099285064e-10\\
-24.316484375	7.38390269507214e-10\\
-24.295986328125	8.97847255473588e-10\\
-24.27548828125	6.82568806358489e-10\\
-24.254990234375	8.39956993424498e-10\\
-24.2344921875	6.96029006303244e-10\\
-24.213994140625	8.12854421338418e-10\\
-24.19349609375	6.53381135543278e-10\\
-24.172998046875	9.01569718591716e-10\\
-24.1525	7.61135736354055e-10\\
-24.132001953125	1.07220631823358e-09\\
-24.11150390625	8.89424069299075e-10\\
-24.091005859375	1.07553830183878e-09\\
-24.0705078125	9.22634108201924e-10\\
-24.050009765625	8.15926940368085e-10\\
-24.02951171875	6.81466226697773e-10\\
-24.009013671875	7.12056480407344e-10\\
-23.988515625	6.39895974410481e-10\\
-23.968017578125	5.47673817163969e-10\\
-23.94751953125	4.23565522260601e-10\\
-23.927021484375	5.58781535971922e-10\\
-23.9065234375	5.99504678636822e-10\\
-23.886025390625	4.98647427719573e-10\\
-23.86552734375	6.40258424417758e-10\\
-23.845029296875	5.54460385815783e-10\\
-23.82453125	5.9447203152916e-10\\
-23.804033203125	4.98904793856494e-10\\
-23.78353515625	4.69069927312811e-10\\
-23.763037109375	6.0544640868016e-10\\
-23.7425390625	4.22411529760707e-10\\
-23.722041015625	6.24112345735304e-10\\
-23.70154296875	4.65250925728572e-10\\
-23.681044921875	5.86474132095106e-10\\
-23.660546875	6.09340044751481e-10\\
-23.640048828125	6.38282756026853e-10\\
-23.61955078125	6.01771267576185e-10\\
-23.599052734375	7.23275644487022e-10\\
-23.5785546875	6.00459830798433e-10\\
-23.558056640625	6.01200120191943e-10\\
-23.53755859375	6.39634258712568e-10\\
-23.517060546875	6.16641418681675e-10\\
-23.4965625	6.20260813390563e-10\\
-23.476064453125	7.11897058309213e-10\\
-23.45556640625	6.8077142825278e-10\\
-23.435068359375	8.68923769818716e-10\\
-23.4145703125	8.54639270151203e-10\\
-23.394072265625	1.00586372581475e-09\\
-23.37357421875	1.12235458302105e-09\\
-23.353076171875	1.18081058692039e-09\\
-23.332578125	1.15103786784728e-09\\
-23.312080078125	1.18180382462626e-09\\
-23.29158203125	1.35514223971541e-09\\
-23.271083984375	1.21656138907517e-09\\
-23.2505859375	1.3262019803161e-09\\
-23.230087890625	1.25796293080688e-09\\
-23.20958984375	1.4151803592418e-09\\
-23.189091796875	1.36451853439134e-09\\
-23.16859375	1.5817325858206e-09\\
-23.148095703125	1.43105758011692e-09\\
-23.12759765625	1.49659232148965e-09\\
-23.107099609375	1.53528826160855e-09\\
-23.0866015625	1.26347450727743e-09\\
-23.066103515625	1.38197828920227e-09\\
-23.04560546875	1.29087431556004e-09\\
-23.025107421875	1.07104133758633e-09\\
-23.004609375	1.18623168001136e-09\\
-22.984111328125	1.10963200352015e-09\\
-22.96361328125	1.20127925651425e-09\\
-22.943115234375	1.2515735054182e-09\\
-22.9226171875	1.22896489651992e-09\\
-22.902119140625	1.18179266364472e-09\\
-22.88162109375	1.16825755967319e-09\\
-22.861123046875	1.02561111921729e-09\\
-22.840625	1.02938820900196e-09\\
-22.820126953125	8.65947516388412e-10\\
-22.79962890625	6.99621434461686e-10\\
-22.779130859375	5.01661474421144e-10\\
-22.7586328125	7.10899764636231e-10\\
-22.738134765625	5.87545204790211e-10\\
-22.71763671875	7.71161740384642e-10\\
-22.697138671875	6.50538206859006e-10\\
-22.676640625	7.49067403675316e-10\\
-22.656142578125	5.71097464322075e-10\\
-22.63564453125	6.16731232834048e-10\\
-22.615146484375	3.48862774904497e-10\\
-22.5946484375	3.85870189923397e-10\\
-22.574150390625	2.00406985452281e-10\\
-22.55365234375	2.84330438040096e-10\\
-22.533154296875	3.30169467887547e-10\\
-22.51265625	4.23104418981409e-10\\
-22.492158203125	3.56658183593776e-10\\
-22.47166015625	5.73660872446669e-10\\
-22.451162109375	4.45605421699668e-10\\
-22.4306640625	3.61299493731502e-10\\
-22.410166015625	3.79545205943015e-10\\
-22.38966796875	-7.72121597461469e-13\\
-22.369169921875	1.11409378854824e-10\\
-22.348671875	-1.18975975132174e-10\\
-22.328173828125	1.55162563634641e-10\\
-22.30767578125	1.84209163975101e-11\\
-22.287177734375	2.87253706225225e-10\\
-22.2666796875	1.55776072565827e-10\\
-22.246181640625	5.95276015079022e-10\\
-22.22568359375	2.87163001982804e-10\\
-22.205185546875	5.48355410915693e-10\\
-22.1846875	2.86694633565218e-10\\
-22.164189453125	2.89264925495114e-10\\
-22.14369140625	8.66741824072562e-11\\
-22.123193359375	1.22413538672947e-10\\
-22.1026953125	-8.29972384102758e-11\\
-22.082197265625	2.17925189936873e-11\\
-22.06169921875	2.49450491684089e-11\\
-22.041201171875	8.19839081660152e-11\\
-22.020703125	4.30624543010721e-11\\
-22.000205078125	1.37487688265725e-10\\
-21.97970703125	2.01482231601381e-10\\
-21.959208984375	3.16826280669173e-10\\
-21.9387109375	2.01914194215592e-10\\
-21.918212890625	1.13424679144872e-10\\
-21.89771484375	9.81874366026376e-11\\
-21.877216796875	-4.72448477542831e-11\\
-21.85671875	-1.19295515794117e-12\\
-21.836220703125	-1.20992114038462e-10\\
-21.81572265625	-8.67374506601598e-12\\
-21.795224609375	-1.32687959484098e-10\\
-21.7747265625	2.15677436017177e-11\\
-21.754228515625	-1.83755497603358e-10\\
-21.73373046875	1.53950867737631e-11\\
-21.713232421875	-1.86436422606618e-10\\
-21.692734375	-1.27229729092916e-10\\
-21.672236328125	-3.56981026785066e-10\\
-21.65173828125	-3.09478736301434e-10\\
-21.631240234375	-5.45184946402655e-10\\
-21.6107421875	-5.26666527853294e-10\\
-21.590244140625	-5.50342675111319e-10\\
-21.56974609375	-4.33889185723131e-10\\
-21.549248046875	-4.66156895457161e-10\\
-21.52875	-2.96145547884648e-10\\
-21.508251953125	-2.42247840619871e-10\\
-21.48775390625	-2.25222455784124e-10\\
-21.467255859375	-1.00247355288605e-10\\
-21.4467578125	-2.10787564724746e-10\\
-21.426259765625	-2.49325887039036e-10\\
-21.40576171875	-3.85463366915432e-10\\
-21.385263671875	-2.27821991826621e-10\\
-21.364765625	-2.74808958973623e-10\\
-21.344267578125	-1.94811559385095e-10\\
-21.32376953125	-1.79518640501692e-10\\
-21.303271484375	3.98236437966376e-11\\
-21.2827734375	9.0450604091216e-11\\
-21.262275390625	1.36835796038806e-11\\
-21.24177734375	2.40335925203355e-10\\
-21.221279296875	6.22227275918352e-11\\
-21.20078125	2.37198488374225e-10\\
-21.180283203125	7.20296015248876e-11\\
-21.15978515625	2.27040928487136e-10\\
-21.139287109375	1.27016134387216e-10\\
-21.1187890625	1.99737891701956e-10\\
-21.098291015625	1.05251654811962e-10\\
-21.07779296875	2.30354468497827e-10\\
-21.057294921875	1.63181377449173e-10\\
-21.036796875	1.26966113477056e-10\\
-21.016298828125	7.27444474885078e-11\\
-20.99580078125	1.10048046440004e-10\\
-20.975302734375	-1.38588394323927e-10\\
-20.9548046875	-3.34480255599101e-11\\
-20.934306640625	-2.2767749634487e-10\\
-20.91380859375	-3.12296619345648e-10\\
-20.893310546875	-3.6307264498644e-10\\
-20.8728125	-4.19200264277836e-10\\
-20.852314453125	-3.61974409123057e-10\\
-20.83181640625	-4.18587524682248e-10\\
-20.811318359375	-3.51730243147511e-10\\
-20.7908203125	-5.5830454845349e-10\\
-20.770322265625	-2.65679417859927e-10\\
-20.74982421875	-4.87301500600753e-10\\
-20.729326171875	-3.57185462631221e-10\\
-20.708828125	-4.96609841275818e-10\\
-20.688330078125	-4.17843888292328e-10\\
-20.66783203125	-6.69446975200703e-10\\
-20.647333984375	-6.52002437301948e-10\\
-20.6268359375	-6.94421688937825e-10\\
-20.606337890625	-7.59701191752861e-10\\
-20.58583984375	-5.21471139463458e-10\\
-20.565341796875	-6.09229244394644e-10\\
-20.54484375	-6.00654459503299e-10\\
-20.524345703125	-4.98478678685929e-10\\
-20.50384765625	-6.55352099659392e-10\\
-20.483349609375	-4.33263476481658e-10\\
-20.4628515625	-7.20257423005953e-10\\
-20.442353515625	-5.48260840205078e-10\\
-20.42185546875	-6.90733149684236e-10\\
-20.401357421875	-6.6670700210634e-10\\
-20.380859375	-6.42100727280871e-10\\
-20.360361328125	-4.75130708070476e-10\\
-20.33986328125	-5.64182912849423e-10\\
-20.319365234375	-4.5622163027748e-10\\
-20.2988671875	-4.38230521944817e-10\\
-20.278369140625	-3.7631378905255e-10\\
-20.25787109375	-6.15894393025248e-10\\
-20.237373046875	-4.58186172619211e-10\\
-20.216875	-4.28867003422204e-10\\
-20.196376953125	-4.88890854523688e-10\\
-20.17587890625	-5.19416236882332e-10\\
-20.155380859375	-3.10882172381399e-10\\
-20.1348828125	-2.59664998989225e-10\\
-20.114384765625	-3.55280649994725e-10\\
-20.09388671875	-4.6015211555691e-10\\
-20.073388671875	-3.62248666868513e-10\\
-20.052890625	-4.78855838942855e-10\\
-20.032392578125	-6.64825021707195e-10\\
-20.01189453125	-4.56193931024715e-10\\
-19.991396484375	-5.02265207883117e-10\\
-19.9708984375	-6.06972366124695e-10\\
-19.950400390625	-4.14176426313625e-10\\
-19.92990234375	-3.45602510438655e-10\\
-19.909404296875	-4.03824111780111e-10\\
-19.88890625	-7.83532069994269e-11\\
-19.868408203125	-4.52597455416137e-10\\
-19.84791015625	-1.53499505044951e-10\\
-19.827412109375	-6.21174081974946e-10\\
-19.8069140625	-4.42878189148015e-10\\
-19.786416015625	-5.9679271314183e-10\\
-19.76591796875	-4.88238419274499e-10\\
-19.745419921875	-7.78890567379068e-10\\
-19.724921875	-3.83511906695978e-10\\
-19.704423828125	-4.92260817213037e-10\\
-19.68392578125	-2.1082185778566e-10\\
-19.663427734375	-3.38490560738268e-10\\
-19.6429296875	-1.70557343701868e-10\\
-19.622431640625	-3.66844528630623e-10\\
-19.60193359375	-3.53448468529959e-10\\
-19.581435546875	-4.29405324263619e-10\\
-19.5609375	-4.41201127651865e-10\\
-19.540439453125	-6.19872782139401e-10\\
-19.51994140625	-4.11570802999955e-10\\
-19.499443359375	-4.30027741145084e-10\\
-19.4789453125	-2.76915719104535e-10\\
-19.458447265625	-1.34645267721603e-10\\
-19.43794921875	-4.52250595600916e-11\\
-19.417451171875	2.95592286852425e-12\\
-19.396953125	-1.22788226140305e-10\\
-19.376455078125	-6.39758648629864e-11\\
-19.35595703125	-2.01703437216587e-10\\
-19.335458984375	-2.32310997863054e-10\\
-19.3149609375	-3.35283568284347e-10\\
-19.294462890625	-2.75064202412166e-10\\
-19.27396484375	-3.27248188288309e-10\\
-19.253466796875	-2.59585903387177e-11\\
-19.23296875	-1.54686500337408e-10\\
-19.212470703125	6.74184371473896e-11\\
-19.19197265625	3.087568294331e-11\\
-19.171474609375	2.11194312634093e-10\\
-19.1509765625	2.98109090541633e-11\\
-19.130478515625	2.18575056875788e-10\\
-19.10998046875	7.93872191473836e-11\\
-19.089482421875	1.09724066328554e-10\\
-19.068984375	3.35092533241994e-11\\
-19.048486328125	1.54172591812023e-10\\
-19.02798828125	1.62427293393061e-10\\
-19.007490234375	1.58704315730214e-10\\
-18.9869921875	1.71195250525395e-10\\
-18.966494140625	4.88492746021333e-10\\
-18.94599609375	4.12231832038702e-10\\
-18.925498046875	4.96577508323199e-10\\
-18.905	5.03688476601446e-10\\
-18.884501953125	4.3499344630222e-10\\
-18.86400390625	3.45078669949011e-10\\
-18.843505859375	2.7123906749519e-10\\
-18.8230078125	2.691278010345e-10\\
-18.802509765625	1.85598721166023e-10\\
-18.78201171875	2.84644097245409e-10\\
-18.761513671875	2.16226450120528e-10\\
-18.741015625	1.65775077671695e-10\\
-18.720517578125	4.67933480537548e-10\\
-18.70001953125	2.67699650606872e-10\\
-18.679521484375	4.78822207646066e-10\\
-18.6590234375	2.59112781070615e-10\\
-18.638525390625	3.13680492608768e-10\\
-18.61802734375	1.25006601461509e-10\\
-18.597529296875	1.87565886095819e-10\\
-18.57703125	3.37271907073364e-11\\
-18.556533203125	8.08711594577989e-11\\
-18.53603515625	1.63795641198648e-10\\
-18.515537109375	3.10466582064776e-10\\
-18.4950390625	2.99252792757893e-10\\
-18.474541015625	3.94330310288136e-10\\
-18.45404296875	2.64584804357188e-10\\
-18.433544921875	3.95324935462703e-10\\
-18.413046875	4.48042579779106e-10\\
-18.392548828125	3.82306140838143e-10\\
-18.37205078125	4.96203741704122e-10\\
-18.351552734375	4.43336531245234e-10\\
-18.3310546875	6.58527783266348e-10\\
-18.310556640625	6.36975763307709e-10\\
-18.29005859375	6.7848667182672e-10\\
-18.269560546875	6.34409227106023e-10\\
-18.2490625	7.17648965295585e-10\\
-18.228564453125	5.07666008115745e-10\\
-18.20806640625	5.74959684956857e-10\\
-18.187568359375	4.16900620189304e-10\\
-18.1670703125	6.30490400734994e-10\\
-18.146572265625	4.39023667087291e-10\\
-18.12607421875	6.17731418028974e-10\\
-18.105576171875	7.08461795972264e-10\\
-18.085078125	4.40450009046096e-10\\
-18.064580078125	6.53310332517895e-10\\
-18.04408203125	5.49472992329849e-10\\
-18.023583984375	5.00995445967862e-10\\
-18.0030859375	4.401300043287e-10\\
-17.982587890625	3.52902387068413e-10\\
-17.96208984375	5.70701302832612e-10\\
-17.941591796875	3.38784434600776e-10\\
-17.92109375	5.20869691623229e-10\\
-17.900595703125	5.68639843476592e-10\\
-17.88009765625	5.89162472353604e-10\\
-17.859599609375	5.46680201597061e-10\\
-17.8391015625	6.31897515070049e-10\\
-17.818603515625	4.90662942106221e-10\\
-17.79810546875	5.68742803682158e-10\\
-17.777607421875	3.69823558033397e-10\\
-17.757109375	5.57473980335295e-10\\
-17.736611328125	4.77754698160173e-10\\
-17.71611328125	4.45548639393748e-10\\
-17.695615234375	5.8266613778143e-10\\
-17.6751171875	5.53398540487918e-10\\
-17.654619140625	4.73835922987189e-10\\
-17.63412109375	5.79447677148987e-10\\
-17.613623046875	5.88841458550918e-10\\
-17.593125	6.4332509907213e-10\\
-17.572626953125	5.44750706306448e-10\\
-17.55212890625	5.8719609752202e-10\\
-17.531630859375	7.55423865104352e-10\\
-17.5111328125	4.82931251639621e-10\\
-17.490634765625	5.65334258916957e-10\\
-17.47013671875	5.82865121212922e-10\\
-17.449638671875	4.99458923645612e-10\\
-17.429140625	5.00223667174132e-10\\
-17.408642578125	5.91347682032056e-10\\
-17.38814453125	4.93118944058737e-10\\
-17.367646484375	7.03644462443943e-10\\
-17.3471484375	3.72324639161051e-10\\
-17.326650390625	8.01735414001389e-10\\
-17.30615234375	3.85718592240272e-10\\
-17.285654296875	5.26757573971792e-10\\
-17.26515625	2.40386976661874e-10\\
-17.244658203125	4.17311276101137e-10\\
-17.22416015625	2.60584978380589e-10\\
-17.203662109375	4.90038051949943e-10\\
-17.1831640625	4.84227998462701e-10\\
-17.162666015625	5.91471191416091e-10\\
-17.14216796875	5.07979465320138e-10\\
-17.121669921875	6.02486449875216e-10\\
-17.101171875	5.21518850413121e-10\\
-17.080673828125	5.21893223101474e-10\\
-17.06017578125	3.00990953295536e-10\\
-17.039677734375	3.21586959014066e-10\\
-17.0191796875	7.35430906478745e-11\\
-16.998681640625	1.06231118417232e-10\\
-16.97818359375	-8.21578790763479e-11\\
-16.957685546875	7.14037851634502e-11\\
-16.9371875	-3.50092779072766e-11\\
-16.916689453125	4.09131871608851e-11\\
-16.89619140625	1.28396570111802e-10\\
-16.875693359375	2.01057238302713e-11\\
-16.8551953125	1.99834925462187e-10\\
-16.834697265625	9.02999292144782e-11\\
-16.81419921875	1.8012133224753e-10\\
-16.793701171875	1.13247844381186e-11\\
-16.773203125	7.08069155736232e-11\\
-16.752705078125	-8.87564436431616e-11\\
-16.73220703125	5.80995164339363e-11\\
-16.711708984375	-1.03799627125621e-10\\
-16.6912109375	8.7336783478503e-11\\
-16.670712890625	5.4701430094425e-11\\
-16.65021484375	1.97950193285162e-10\\
-16.629716796875	-6.15119368922555e-12\\
-16.60921875	5.97914096992572e-12\\
-16.588720703125	-8.40284493335758e-11\\
-16.56822265625	-9.67819009960329e-11\\
-16.547724609375	-1.2191543882873e-10\\
-16.5272265625	-6.36203285083026e-11\\
-16.506728515625	2.16879138737656e-11\\
-16.48623046875	6.24054704816145e-11\\
-16.465732421875	7.86463733036229e-11\\
-16.445234375	2.29616626180711e-10\\
-16.424736328125	1.47764061509949e-10\\
-16.40423828125	1.47097176636292e-10\\
-16.383740234375	7.50731585238211e-11\\
-16.3632421875	-9.68046620287378e-11\\
-16.342744140625	-1.48312596905251e-10\\
-16.32224609375	-2.49885529348674e-10\\
-16.301748046875	-2.58107110724502e-10\\
-16.28125	-3.37844858754234e-10\\
-16.260751953125	-1.82794748106494e-10\\
-16.24025390625	-1.359678273824e-10\\
-16.219755859375	-1.19197886554537e-10\\
-16.1992578125	1.06308200266305e-10\\
-16.178759765625	-6.63431444266498e-11\\
-16.15826171875	1.61468437990808e-10\\
-16.137763671875	-1.6652934736593e-10\\
-16.117265625	-8.62141367305741e-12\\
-16.096767578125	-1.54690380014298e-10\\
-16.07626953125	-1.19482141048705e-10\\
-16.055771484375	-2.89030268581768e-10\\
-16.0352734375	-1.61704267269441e-10\\
-16.014775390625	-1.1468830276172e-10\\
-15.99427734375	3.04889891468247e-11\\
-15.973779296875	-9.81143439549222e-11\\
-15.95328125	1.2130508020859e-10\\
-15.932783203125	2.11269571542038e-12\\
-15.91228515625	-6.34749485260219e-11\\
-15.891787109375	-1.64574249639964e-10\\
-15.8712890625	-2.74083583702414e-10\\
-15.850791015625	-3.16910930514231e-10\\
-15.83029296875	-3.82510783502444e-10\\
-15.809794921875	-3.26087961099828e-10\\
-15.789296875	-3.04841863483253e-10\\
-15.768798828125	-2.89616185296031e-10\\
-15.74830078125	-1.85198679877158e-10\\
-15.727802734375	-4.51348929247416e-11\\
-15.7073046875	-1.09830497664289e-10\\
-15.686806640625	1.06138036801671e-11\\
-15.66630859375	-1.50245883795484e-10\\
-15.645810546875	-5.35372395906459e-11\\
-15.6253125	-2.06267444370417e-10\\
-15.604814453125	-2.14821978275596e-10\\
-15.58431640625	-1.9555963118339e-10\\
-15.563818359375	-2.60001466070002e-10\\
-15.5433203125	-8.03897340714489e-11\\
-15.522822265625	-9.7860069796105e-11\\
-15.50232421875	7.18769064424708e-11\\
-15.481826171875	1.83483603077833e-10\\
-15.461328125	1.22997064001745e-10\\
-15.440830078125	2.8794632992825e-10\\
-15.42033203125	-4.98772334415226e-11\\
-15.399833984375	9.3772230237427e-11\\
-15.3793359375	8.38008548549892e-12\\
-15.358837890625	-3.36938540406795e-12\\
-15.33833984375	-3.17689075083169e-11\\
-15.317841796875	1.44823082947062e-10\\
-15.29734375	5.28541305698829e-12\\
-15.276845703125	4.25639977361009e-10\\
-15.25634765625	2.43115604019945e-10\\
-15.235849609375	3.95350782377115e-10\\
-15.2153515625	2.84802342583976e-10\\
-15.194853515625	1.90338567617542e-10\\
-15.17435546875	1.4442612125575e-10\\
-15.153857421875	1.30399400838855e-10\\
-15.133359375	-1.06258947564266e-10\\
-15.112861328125	7.03563436324994e-11\\
-15.09236328125	-9.17832448167472e-11\\
-15.071865234375	7.17237320823661e-11\\
-15.0513671875	8.42402917524766e-11\\
-15.030869140625	1.06851693554461e-10\\
-15.01037109375	2.57996750934297e-10\\
-14.989873046875	1.04390592039009e-10\\
-14.969375	1.96971518635594e-10\\
-14.948876953125	1.15091845233419e-10\\
-14.92837890625	8.7801213999375e-11\\
-14.907880859375	5.07011318059707e-11\\
-14.8873828125	8.45005489096614e-11\\
-14.866884765625	-3.66022812028269e-11\\
-14.84638671875	2.42731277838262e-10\\
-14.825888671875	-2.4292131513509e-10\\
-14.805390625	2.70991987265763e-10\\
-14.784892578125	3.13572263079618e-11\\
-14.76439453125	2.92861311665146e-10\\
-14.743896484375	8.21646508793061e-11\\
-14.7233984375	2.55285989776894e-10\\
-14.702900390625	6.24058121755039e-11\\
-14.68240234375	9.42906834454754e-11\\
-14.661904296875	1.01752443751093e-10\\
-14.64140625	5.31823529187901e-11\\
-14.620908203125	-1.09440023753915e-10\\
-14.60041015625	6.81614755959235e-12\\
-14.579912109375	-7.13266913482088e-11\\
-14.5594140625	8.45727208739464e-11\\
-14.538916015625	-1.76219790026245e-11\\
-14.51841796875	9.54092569512309e-11\\
-14.497919921875	1.53233748677204e-10\\
-14.477421875	2.55498963066794e-10\\
-14.456923828125	1.81265350664723e-10\\
-14.43642578125	4.18489057245982e-10\\
-14.415927734375	3.03998263618409e-10\\
-14.3954296875	2.18715402972475e-10\\
-14.374931640625	3.30568616605731e-10\\
-14.35443359375	2.00741841482495e-10\\
-14.333935546875	3.68021622039631e-10\\
-14.3134375	2.43627099475029e-10\\
-14.292939453125	4.67975193135769e-10\\
-14.27244140625	5.04431990259051e-10\\
-14.251943359375	6.61230140168964e-10\\
-14.2314453125	5.15495763206153e-10\\
-14.210947265625	7.02860652856827e-10\\
-14.19044921875	5.07830066050008e-10\\
-14.169951171875	4.60988572692618e-10\\
-14.149453125	4.4830545636057e-10\\
-14.128955078125	5.62027405020231e-10\\
-14.10845703125	5.93218778465468e-10\\
-14.087958984375	6.41050470110124e-10\\
-14.0674609375	6.15523287309213e-10\\
-14.046962890625	7.73798408552128e-10\\
-14.02646484375	6.22364407259101e-10\\
-14.005966796875	6.66197847475358e-10\\
-13.98546875	5.54093918080119e-10\\
-13.964970703125	4.37142437034519e-10\\
-13.94447265625	3.22824787536884e-10\\
-13.923974609375	4.89104239564983e-10\\
-13.9034765625	4.64951956507172e-10\\
-13.882978515625	5.20543127982029e-10\\
-13.86248046875	5.35785740992794e-10\\
-13.841982421875	6.43729219013809e-10\\
-13.821484375	6.53860496869405e-10\\
-13.800986328125	6.6498801624706e-10\\
-13.78048828125	6.82941313374977e-10\\
-13.759990234375	6.22739896650391e-10\\
-13.7394921875	4.31479902997638e-10\\
-13.718994140625	4.81545028952045e-10\\
-13.69849609375	4.20300428789258e-10\\
-13.677998046875	6.11032381214322e-10\\
-13.6575	4.81415872472837e-10\\
-13.637001953125	6.45779730115818e-10\\
-13.61650390625	6.66925401762635e-10\\
-13.596005859375	7.70245081775245e-10\\
-13.5755078125	7.15289292109522e-10\\
-13.555009765625	8.67875144006907e-10\\
-13.53451171875	7.28309758736403e-10\\
-13.514013671875	8.04750071023161e-10\\
-13.493515625	7.0822294576134e-10\\
-13.473017578125	9.30266376879687e-10\\
-13.45251953125	7.44598699456481e-10\\
-13.432021484375	1.02238708595704e-09\\
-13.4115234375	8.05824567246562e-10\\
-13.391025390625	1.02706634114355e-09\\
-13.37052734375	1.03187537059361e-09\\
-13.350029296875	9.40555907000349e-10\\
-13.32953125	1.17732847657142e-09\\
-13.309033203125	1.10459279116377e-09\\
-13.28853515625	1.28952190321563e-09\\
-13.268037109375	1.19288277663978e-09\\
-13.2475390625	1.25039240694262e-09\\
-13.227041015625	9.73275647281332e-10\\
-13.20654296875	1.15106770262687e-09\\
-13.186044921875	9.00100302534656e-10\\
-13.165546875	1.05092500790879e-09\\
-13.145048828125	1.00312026106892e-09\\
-13.12455078125	1.03012080358687e-09\\
-13.104052734375	9.03522057446915e-10\\
-13.0835546875	1.06715190826725e-09\\
-13.063056640625	1.05630372676595e-09\\
-13.04255859375	1.04679119841893e-09\\
-13.022060546875	1.09605646291231e-09\\
-13.0015625	9.68990032652473e-10\\
-12.981064453125	6.96814503177592e-10\\
-12.96056640625	7.39049653560259e-10\\
-12.940068359375	5.49056201606935e-10\\
-12.9195703125	6.88429229377888e-10\\
-12.899072265625	4.36375601960608e-10\\
-12.87857421875	6.25052955407313e-10\\
-12.858076171875	5.11108596229046e-10\\
-12.837578125	6.30064476233422e-10\\
-12.817080078125	4.18745970506768e-10\\
-12.79658203125	5.2760410598539e-10\\
-12.776083984375	2.28689079655389e-10\\
-12.7555859375	1.99686300333567e-10\\
-12.735087890625	-5.39516194808691e-12\\
-12.71458984375	-2.0335959102382e-11\\
-12.694091796875	-1.1366039096144e-10\\
-12.67359375	-9.5738920488898e-11\\
-12.653095703125	5.44442313822749e-11\\
-12.63259765625	2.56855099075657e-10\\
-12.612099609375	2.1040200917164e-10\\
-12.5916015625	3.87107786239177e-10\\
-12.571103515625	3.47246242114796e-10\\
-12.55060546875	2.42362495798662e-10\\
-12.530107421875	1.42531303823123e-10\\
-12.509609375	1.04709988246203e-10\\
-12.489111328125	2.85269616916911e-10\\
-12.46861328125	7.93151170984271e-11\\
-12.448115234375	1.99382468290585e-10\\
-12.4276171875	2.66400078539524e-10\\
-12.407119140625	2.38187424180777e-10\\
-12.38662109375	4.64692637444187e-10\\
-12.366123046875	4.92180448507715e-10\\
-12.345625	4.58594108708522e-10\\
-12.325126953125	6.1982395223982e-10\\
-12.30462890625	2.1489470808861e-10\\
-12.284130859375	5.30114861199252e-10\\
-12.2636328125	1.76927011890079e-10\\
-12.243134765625	2.74076690113271e-10\\
-12.22263671875	2.55763397012883e-10\\
-12.202138671875	5.68253417011112e-10\\
-12.181640625	4.6971349602222e-10\\
-12.161142578125	6.59104372033895e-10\\
-12.14064453125	6.2566381661595e-10\\
-12.120146484375	6.47951198080932e-10\\
-12.0996484375	4.60081055633185e-10\\
-12.079150390625	4.23940220322456e-10\\
-12.05865234375	3.71645960390612e-10\\
-12.038154296875	3.34459432866057e-10\\
-12.01765625	2.75028629381425e-10\\
-11.997158203125	2.8946796495186e-10\\
-11.97666015625	2.24163247406074e-10\\
-11.956162109375	4.23961227468538e-10\\
-11.9356640625	2.55929335854664e-10\\
-11.915166015625	3.57105653975263e-10\\
-11.89466796875	2.53859223136934e-10\\
-11.874169921875	5.18554377348625e-11\\
-11.853671875	9.70328788378568e-11\\
-11.833173828125	-6.02361634310501e-11\\
-11.81267578125	-3.84250051987571e-12\\
-11.792177734375	-5.41162495299309e-11\\
-11.7716796875	4.32962995855892e-11\\
-11.751181640625	-5.98755401026837e-11\\
-11.73068359375	-3.25750792231027e-11\\
-11.710185546875	-2.54642851182468e-10\\
-11.6896875	-1.04504117115151e-10\\
-11.669189453125	-2.66492238897123e-10\\
-11.64869140625	-2.69216547413668e-10\\
-11.628193359375	-3.64855265290189e-10\\
-11.6076953125	-4.34847423412214e-10\\
-11.587197265625	-5.45024849143284e-10\\
-11.56669921875	-4.75264015955469e-10\\
-11.546201171875	-6.86702079479991e-10\\
-11.525703125	-4.94856760695555e-10\\
-11.505205078125	-6.7624805187296e-10\\
-11.48470703125	-5.49220385420559e-10\\
-11.464208984375	-5.52448843242609e-10\\
-11.4437109375	-4.0792022548467e-10\\
-11.423212890625	-3.86371711068883e-10\\
-11.40271484375	-4.29158098318848e-10\\
-11.382216796875	-4.41509700244943e-10\\
-11.36171875	-4.65198219105057e-10\\
-11.341220703125	-4.52646101281391e-10\\
-11.32072265625	-4.1109644735833e-10\\
-11.300224609375	-5.67335271635662e-10\\
-11.2797265625	-3.83272456511323e-10\\
-11.259228515625	-5.13794978652337e-10\\
-11.23873046875	-3.91719826078308e-10\\
-11.218232421875	-3.14672582546268e-10\\
-11.197734375	-3.45544203115448e-10\\
-11.177236328125	-4.09218771062912e-10\\
-11.15673828125	-2.6938059016665e-10\\
-11.136240234375	-4.87908519600198e-10\\
-11.1157421875	-5.07646593652678e-10\\
-11.095244140625	-4.66920185809571e-10\\
-11.07474609375	-4.27182485711551e-10\\
-11.054248046875	-4.51898927013824e-10\\
-11.03375	-3.22304547866978e-10\\
-11.013251953125	-3.63837432062471e-10\\
-10.99275390625	-2.57096639873676e-10\\
-10.972255859375	-4.99965600514627e-10\\
-10.9517578125	-3.8650855854401e-10\\
-10.931259765625	-6.04819218162677e-10\\
-10.91076171875	-5.7684092832263e-10\\
-10.890263671875	-8.44120735383971e-10\\
-10.869765625	-7.42041897737203e-10\\
-10.849267578125	-6.84595007063824e-10\\
-10.82876953125	-8.54605977144939e-10\\
-10.808271484375	-7.25546076538979e-10\\
-10.7877734375	-7.99772826814706e-10\\
-10.767275390625	-6.88466857345722e-10\\
-10.74677734375	-8.94064353084852e-10\\
-10.726279296875	-7.32306215146421e-10\\
-10.70578125	-1.08753751664264e-09\\
-10.685283203125	-8.67659472612799e-10\\
-10.66478515625	-1.11771949898361e-09\\
-10.644287109375	-9.0906162613803e-10\\
-10.6237890625	-1.08699313918373e-09\\
-10.603291015625	-8.65677411909356e-10\\
-10.58279296875	-9.84971462860995e-10\\
-10.562294921875	-8.46412476146786e-10\\
-10.541796875	-8.78355549064327e-10\\
-10.521298828125	-7.76711867568271e-10\\
-10.50080078125	-7.32310642239602e-10\\
-10.480302734375	-7.58635609322785e-10\\
-10.4598046875	-8.15591162784838e-10\\
-10.439306640625	-6.30320663140605e-10\\
-10.41880859375	-7.93140934600699e-10\\
-10.398310546875	-5.5312393554451e-10\\
-10.3778125	-6.88536709362916e-10\\
-10.357314453125	-3.91914128656051e-10\\
-10.33681640625	-4.86529050660575e-10\\
-10.316318359375	-3.33530371744083e-10\\
-10.2958203125	-3.39785945452984e-10\\
-10.275322265625	-1.58985177778639e-10\\
-10.25482421875	-3.18564852604073e-10\\
-10.234326171875	-1.39068610870038e-10\\
-10.213828125	-1.91451867528988e-10\\
-10.193330078125	-1.36822176870253e-10\\
-10.17283203125	-1.56033265645045e-10\\
-10.152333984375	-7.23171972108433e-11\\
-10.1318359375	-2.37557237001565e-11\\
-10.111337890625	7.74521529262106e-11\\
-10.09083984375	2.17897939186593e-10\\
-10.070341796875	2.79314038650657e-10\\
-10.04984375	2.80330996455691e-10\\
-10.029345703125	2.22349789700561e-10\\
-10.00884765625	1.91586728661935e-10\\
-9.988349609375	1.02341174455618e-10\\
-9.9678515625	1.61329610959759e-10\\
-9.947353515625	-4.65090089510319e-11\\
-9.92685546875	7.52957172585548e-12\\
-9.906357421875	-1.07807068582007e-11\\
-9.885859375	9.96940543551757e-11\\
-9.865361328125	2.59818590500426e-11\\
-9.84486328125	6.72572759205931e-11\\
-9.824365234375	-1.02090387929436e-10\\
-9.8038671875	8.33211054403545e-11\\
-9.783369140625	-2.31502433206019e-10\\
-9.76287109375	-3.97068384494519e-11\\
-9.742373046875	-2.28353494171397e-10\\
-9.721875	-1.91918082561459e-10\\
-9.701376953125	-2.85579083159768e-10\\
-9.68087890625	-2.48846260341317e-10\\
-9.660380859375	-5.35038171537669e-10\\
-9.6398828125	-3.28821810029095e-10\\
-9.619384765625	-4.57423109962618e-10\\
-9.59888671874999	-4.11818243945463e-10\\
-9.578388671875	-3.55841407114723e-10\\
-9.557890625	-2.43995411586441e-10\\
-9.537392578125	-2.40038853861329e-10\\
-9.51689453125	-1.28479526401187e-10\\
-9.496396484375	-1.61296761058371e-10\\
-9.4758984375	-1.29854267055858e-11\\
-9.455400390625	-6.27634084847859e-11\\
-9.43490234375	8.55505716144461e-11\\
-9.414404296875	7.63811225715e-11\\
-9.39390625	1.47544621778943e-10\\
-9.373408203125	1.28974248501922e-10\\
-9.35291015625	1.56478349889255e-10\\
-9.332412109375	2.28248079200116e-10\\
-9.3119140625	2.62984580497849e-10\\
-9.291416015625	4.05766787918155e-10\\
-9.27091796875	2.72296890953593e-10\\
-9.250419921875	5.43164890418465e-10\\
-9.229921875	4.58312080598786e-10\\
-9.209423828125	6.02004436477737e-10\\
-9.18892578125	4.8736004222081e-10\\
-9.168427734375	7.3519227280215e-10\\
-9.1479296875	6.14018478202594e-10\\
-9.127431640625	7.35695759132607e-10\\
-9.10693359375	7.97367210993302e-10\\
-9.08643554687499	8.49941526439105e-10\\
-9.0659375	7.70780451112769e-10\\
-9.045439453125	1.00246218741324e-09\\
-9.02494140625	9.07858701631056e-10\\
-9.004443359375	1.07888024203886e-09\\
-8.9839453125	9.05333204140399e-10\\
-8.963447265625	9.62915754272714e-10\\
-8.94294921875	9.71258475183571e-10\\
-8.922451171875	8.7905545416805e-10\\
-8.901953125	8.00222164316454e-10\\
-8.881455078125	8.39268071268083e-10\\
-8.86095703125	8.04750435145821e-10\\
-8.840458984375	8.01327731542885e-10\\
-8.8199609375	8.01317801180262e-10\\
-8.799462890625	9.28389222558504e-10\\
-8.77896484375	7.65036349113096e-10\\
-8.758466796875	8.73155534065985e-10\\
-8.73796875	8.73626579397157e-10\\
-8.717470703125	8.07313385323772e-10\\
-8.69697265625	7.984679497852e-10\\
-8.676474609375	6.90852269696108e-10\\
-8.6559765625	5.85960185785261e-10\\
-8.635478515625	7.32856703678014e-10\\
-8.61498046875	5.18100615573356e-10\\
-8.594482421875	6.32503490420374e-10\\
-8.57398437499999	4.4499934103263e-10\\
-8.553486328125	5.23314548582792e-10\\
-8.53298828125	3.86803784447075e-10\\
-8.512490234375	7.14304770462647e-10\\
-8.4919921875	5.53707599966885e-10\\
-8.471494140625	8.27473757876535e-10\\
-8.45099609375	6.54653439525593e-10\\
-8.430498046875	8.62310199632642e-10\\
-8.41	7.48390911949779e-10\\
-8.389501953125	8.4367027190488e-10\\
-8.36900390625	7.91339848911266e-10\\
-8.348505859375	8.25451585605575e-10\\
-8.3280078125	8.04396481863844e-10\\
-8.307509765625	7.54248094811668e-10\\
-8.28701171875	9.77129039530997e-10\\
-8.266513671875	8.31267418367988e-10\\
-8.246015625	1.01082153828963e-09\\
-8.225517578125	1.04398490911047e-09\\
-8.20501953125	1.16240356947433e-09\\
-8.184521484375	1.02820662496859e-09\\
-8.1640234375	1.25582246915372e-09\\
-8.143525390625	1.09165437612866e-09\\
-8.12302734375	1.12145413519963e-09\\
-8.102529296875	9.74650556601287e-10\\
-8.08203125	1.02642662083595e-09\\
-8.06153320312499	8.70354569572979e-10\\
-8.04103515625	9.79121518753457e-10\\
-8.020537109375	9.19810036011483e-10\\
-8.0000390625	9.49775332556563e-10\\
-7.979541015625	8.43030088547839e-10\\
-7.95904296875	8.67881192419683e-10\\
-7.938544921875	7.77197105074764e-10\\
-7.918046875	8.8182986837865e-10\\
-7.897548828125	6.13122914264146e-10\\
-7.87705078125	7.96845738968603e-10\\
-7.856552734375	5.45792319036893e-10\\
-7.8360546875	6.50670941319376e-10\\
-7.815556640625	4.90705889905715e-10\\
-7.79505859375	5.36804970237031e-10\\
-7.774560546875	2.8842868101223e-10\\
-7.7540625	3.6798217191162e-10\\
-7.733564453125	9.44846573833556e-11\\
-7.71306640625	7.8936168187752e-11\\
-7.692568359375	-7.37744376694435e-11\\
-7.6720703125	-7.55974731841149e-11\\
-7.651572265625	-2.89249284656874e-10\\
-7.63107421875	-1.9365083552534e-10\\
-7.610576171875	-2.86485720098316e-10\\
-7.590078125	-3.45593560198815e-10\\
-7.569580078125	-3.33447533877999e-10\\
-7.54908203125	-3.16602887408609e-10\\
-7.528583984375	-3.81575478222558e-10\\
-7.5080859375	-4.90512577111926e-10\\
-7.487587890625	-3.12912445908629e-10\\
-7.46708984375	-3.61069889768422e-10\\
-7.446591796875	-2.16416883763984e-10\\
-7.42609375	-3.38870388510258e-10\\
-7.405595703125	-1.33333604937324e-10\\
-7.38509765625	-2.58651624754622e-10\\
-7.364599609375	-8.02167361441271e-11\\
-7.3441015625	-1.40711020666976e-10\\
-7.323603515625	-7.80043104964093e-13\\
-7.30310546875	-1.58704534331598e-10\\
-7.282607421875	5.82834571408853e-11\\
-7.262109375	-7.36934215720313e-11\\
-7.241611328125	1.20396640038555e-10\\
-7.22111328125	-9.28031122237544e-11\\
-7.200615234375	1.52914838923133e-10\\
-7.1801171875	1.1939675773742e-10\\
-7.159619140625	4.02559340042399e-10\\
-7.13912109375	3.16546359174488e-10\\
-7.118623046875	4.83977760318237e-10\\
-7.098125	4.08431903797268e-10\\
-7.077626953125	2.58410195229469e-10\\
-7.05712890625	2.92894636552069e-10\\
-7.036630859375	1.68949635468452e-10\\
-7.0161328125	8.65248642362789e-11\\
-6.995634765625	1.32800197665651e-10\\
-6.97513671875	-3.01028901029293e-11\\
-6.954638671875	6.47293703788239e-11\\
-6.934140625	-3.46355423584991e-11\\
-6.913642578125	3.61653263447785e-11\\
-6.89314453125	-1.13465576432954e-10\\
-6.872646484375	-4.64493453779212e-11\\
-6.8521484375	-1.68028295385345e-10\\
-6.831650390625	-2.55705807943732e-10\\
-6.81115234375	-3.35194405034549e-10\\
-6.790654296875	-5.64300865849918e-10\\
-6.77015625	-4.87502642927795e-10\\
-6.749658203125	-7.62012879417194e-10\\
-6.72916015625	-5.62915236252066e-10\\
-6.708662109375	-8.29763497516241e-10\\
-6.6881640625	-7.47094054535625e-10\\
-6.667666015625	-9.59553375761097e-10\\
-6.64716796875	-8.28308318215791e-10\\
-6.626669921875	-9.62876974059933e-10\\
-6.606171875	-1.03697559287658e-09\\
-6.585673828125	-1.01866573160189e-09\\
-6.56517578125	-1.07452284621559e-09\\
-6.544677734375	-1.14267563674454e-09\\
-6.5241796875	-9.52793133142204e-10\\
-6.503681640625	-9.96280194648043e-10\\
-6.48318359375	-8.87743971959575e-10\\
-6.462685546875	-9.95742783193313e-10\\
-6.4421875	-1.03422926509207e-09\\
-6.421689453125	-8.9510349171321e-10\\
-6.40119140625	-8.5065927139132e-10\\
-6.380693359375	-9.58028023071129e-10\\
-6.3601953125	-8.79463876241131e-10\\
-6.339697265625	-9.60744920168854e-10\\
-6.31919921875	-8.06359631174275e-10\\
-6.298701171875	-8.3372262135806e-10\\
-6.278203125	-5.76010183143074e-10\\
-6.257705078125	-7.03836872259382e-10\\
-6.23720703125	-5.60678641626345e-10\\
-6.216708984375	-5.05674726904924e-10\\
-6.1962109375	-4.34951664928956e-10\\
-6.175712890625	-4.66409051783173e-10\\
-6.15521484375	-3.5625172984711e-10\\
-6.134716796875	-5.08296889128616e-10\\
-6.11421875	-3.15273849692107e-10\\
-6.093720703125	-6.01788264238487e-10\\
-6.07322265625	-2.8548855518811e-10\\
-6.052724609375	-4.57433886930738e-10\\
-6.0322265625	-3.08379378401967e-10\\
-6.011728515625	-4.58285577828416e-10\\
-5.99123046875	-2.3407814262699e-10\\
-5.970732421875	-4.04738041901235e-10\\
-5.950234375	-3.0355500561368e-10\\
-5.929736328125	-5.61556339976678e-10\\
-5.90923828125	-3.74805814120983e-10\\
-5.888740234375	-4.79386903848477e-10\\
-5.8682421875	-5.43273707655453e-10\\
-5.847744140625	-5.67316492455445e-10\\
-5.82724609375	-5.44169618990842e-10\\
-5.806748046875	-6.13701541953216e-10\\
-5.78625	-6.26114951826238e-10\\
-5.765751953125	-5.90087619057342e-10\\
-5.74525390625	-6.51268897742752e-10\\
-5.724755859375	-6.65282135472834e-10\\
-5.7042578125	-7.435901745062e-10\\
-5.683759765625	-7.6560837565037e-10\\
-5.66326171875	-9.47773369813823e-10\\
-5.642763671875	-8.02922796966388e-10\\
-5.622265625	-9.47400613957018e-10\\
-5.601767578125	-7.24276128370257e-10\\
-5.58126953125	-7.90502277573128e-10\\
-5.560771484375	-6.27519383137209e-10\\
-5.5402734375	-6.06224519673127e-10\\
-5.519775390625	-5.18105155657613e-10\\
-5.49927734375	-5.1091178982551e-10\\
-5.478779296875	-5.44553752289555e-10\\
-5.45828125	-6.81578233743399e-10\\
-5.437783203125	-6.83912646146095e-10\\
-5.41728515625	-8.44919478708107e-10\\
-5.396787109375	-6.14217143914544e-10\\
-5.3762890625	-7.49791803856879e-10\\
-5.355791015625	-3.84931415897712e-10\\
-5.33529296875	-5.59444899365873e-10\\
-5.314794921875	-3.00418732626404e-10\\
-5.294296875	-3.06577411731791e-10\\
-5.273798828125	-8.21028442807684e-11\\
-5.25330078125	-2.00797369648673e-10\\
-5.232802734375	5.51608646526036e-11\\
-5.2123046875	-7.89279475093295e-11\\
-5.191806640625	1.91536439685071e-10\\
-5.17130859375	7.22461461247724e-11\\
-5.150810546875	2.6847501858841e-10\\
-5.1303125	3.04822015627813e-10\\
-5.109814453125	4.11560484997503e-10\\
-5.08931640625	5.48591262743555e-10\\
-5.068818359375	4.3173397785767e-10\\
-5.0483203125	4.99236442592958e-10\\
-5.027822265625	5.45221039464052e-10\\
-5.00732421875	5.4003018475784e-10\\
-4.986826171875	4.63717756403874e-10\\
-4.966328125	4.25571581241464e-10\\
-4.945830078125	3.80503313550078e-10\\
-4.92533203125	4.99368200186195e-10\\
-4.904833984375	3.06101574780234e-10\\
-4.8843359375	4.46188428156914e-10\\
-4.863837890625	3.36510326548588e-10\\
-4.84333984375	3.72896501824568e-10\\
-4.822841796875	2.08911410437943e-10\\
-4.80234375	2.86004406513046e-10\\
-4.781845703125	1.79088733439901e-10\\
-4.76134765625	2.0770289565661e-10\\
-4.740849609375	1.8136397183906e-11\\
-4.7203515625	2.64070025331699e-10\\
-4.699853515625	-5.13146005260651e-11\\
-4.67935546875	7.48997747157924e-11\\
-4.658857421875	-1.77075484176266e-10\\
-4.638359375	-6.89497550656449e-11\\
-4.617861328125	-2.95595610786983e-10\\
-4.59736328125	-2.1552763516666e-10\\
-4.576865234375	-2.05451310770285e-10\\
-4.5563671875	-2.33714369697295e-10\\
-4.535869140625	-1.64872150156113e-10\\
-4.51537109375	-5.50263378764235e-11\\
-4.494873046875	-2.35166264599557e-11\\
-4.474375	1.85534118656782e-10\\
-4.453876953125	9.6083728036268e-11\\
-4.43337890625	2.30817173640356e-10\\
-4.412880859375	2.90676455460602e-10\\
-4.3923828125	3.24197478885077e-10\\
-4.371884765625	3.1542376205214e-10\\
-4.35138671875	3.89467029142733e-10\\
-4.330888671875	4.06893422894002e-10\\
-4.310390625	3.57228364149754e-10\\
-4.289892578125	5.84345303622357e-10\\
-4.26939453125	4.64325822864031e-10\\
-4.248896484375	6.93063791101784e-10\\
-4.2283984375	6.92487722665849e-10\\
-4.207900390625	9.87221780638242e-10\\
-4.18740234375	8.31870126016867e-10\\
-4.166904296875	1.13532805638022e-09\\
-4.14640625	1.02004033832483e-09\\
-4.125908203125	1.14778625464378e-09\\
-4.10541015625	1.09008751416762e-09\\
-4.084912109375	8.65764417403688e-10\\
-4.0644140625	8.76661217511556e-10\\
-4.043916015625	9.8394828833703e-10\\
-4.02341796875	9.07166165502578e-10\\
-4.002919921875	8.9375593784642e-10\\
-3.982421875	9.37810168632346e-10\\
-3.961923828125	1.10244916219982e-09\\
-3.94142578125	1.11932915721332e-09\\
-3.920927734375	1.20070534842449e-09\\
-3.9004296875	1.11395806410006e-09\\
-3.879931640625	1.06134287025103e-09\\
-3.85943359375	8.86775575903146e-10\\
-3.838935546875	9.65427891159564e-10\\
-3.8184375	7.48545474698534e-10\\
-3.797939453125	8.34096084643774e-10\\
-3.77744140625	8.15154218402539e-10\\
-3.756943359375	1.0016872646818e-09\\
-3.7364453125	8.04744146800655e-10\\
-3.715947265625	9.6313938340832e-10\\
-3.69544921875	9.86072385627043e-10\\
-3.674951171875	9.69480203428302e-10\\
-3.654453125	8.47349196113653e-10\\
-3.633955078125	9.92351742035558e-10\\
-3.61345703125	7.24806787677068e-10\\
-3.592958984375	9.17491811276429e-10\\
-3.5724609375	7.09826877145562e-10\\
-3.551962890625	8.17406980020286e-10\\
-3.53146484375	7.57007675538109e-10\\
-3.510966796875	8.26752328144678e-10\\
-3.49046875	6.51047706026917e-10\\
-3.469970703125	9.51674269890005e-10\\
-3.44947265625	7.80248823016637e-10\\
-3.428974609375	1.0050114661463e-09\\
-3.4084765625	8.98481813767114e-10\\
-3.387978515625	9.42237615967583e-10\\
-3.36748046875	1.00232561102115e-09\\
-3.346982421875	1.05177758843602e-09\\
-3.326484375	9.37681619016524e-10\\
-3.305986328125	1.04445691770007e-09\\
-3.28548828125	9.91866071570608e-10\\
-3.264990234375	1.04202726279606e-09\\
-3.2444921875	1.08338825260358e-09\\
-3.223994140625	1.18246551744041e-09\\
-3.20349609375	1.30317975321904e-09\\
-3.182998046875	1.30050831397149e-09\\
-3.1625	1.51649591203193e-09\\
-3.142001953125	1.36561498322102e-09\\
-3.12150390625	1.4464456682529e-09\\
-3.101005859375	1.22463172090078e-09\\
-3.0805078125	1.40663846522184e-09\\
-3.060009765625	1.17888680409465e-09\\
-3.03951171875	1.33252694355437e-09\\
-3.019013671875	1.14315277534479e-09\\
-2.998515625	1.32016986101373e-09\\
-2.978017578125	1.18729018788279e-09\\
-2.95751953125	1.33999094623038e-09\\
-2.937021484375	1.28404173311419e-09\\
-2.9165234375	1.32153685414537e-09\\
-2.896025390625	1.04673042751207e-09\\
-2.87552734375	1.12390770522634e-09\\
-2.855029296875	8.88778307549767e-10\\
-2.83453125	1.01948566354019e-09\\
-2.814033203125	8.58938585953058e-10\\
-2.79353515625	1.01695156511326e-09\\
-2.773037109375	7.86761592748534e-10\\
-2.7525390625	9.37524086484186e-10\\
-2.732041015625	7.37036623360582e-10\\
-2.71154296875	8.01473387238725e-10\\
-2.691044921875	4.8661307701023e-10\\
-2.670546875	4.93301209827946e-10\\
-2.650048828125	3.16419702398169e-10\\
-2.62955078125	1.90029349334032e-10\\
-2.609052734375	1.65195126616002e-10\\
-2.5885546875	1.8797683952931e-10\\
-2.568056640625	2.27911264441145e-10\\
-2.54755859375	1.23299419394295e-10\\
-2.527060546875	1.78763072883509e-10\\
-2.5065625	5.54055506421351e-11\\
-2.486064453125	1.15941748004172e-10\\
-2.46556640625	7.49739342108789e-11\\
-2.445068359375	1.39417839383347e-10\\
-2.4245703125	-1.031879377496e-11\\
-2.404072265625	8.57939455784022e-11\\
-2.38357421875	3.64887342236344e-11\\
-2.363076171875	2.6179828815502e-10\\
-2.342578125	2.45859873323511e-10\\
-2.322080078125	4.1718412418047e-10\\
-2.30158203125	3.55188913492746e-10\\
-2.281083984375	4.46665125426022e-10\\
-2.2605859375	3.45609567251982e-10\\
-2.240087890625	4.5680584400814e-10\\
-2.21958984375	2.43207370063517e-10\\
-2.199091796875	5.24042873488586e-10\\
-2.17859375	2.49559480408921e-10\\
-2.158095703125	5.35554908552607e-10\\
-2.13759765625	4.18477762439159e-10\\
-2.117099609375	5.59241491836508e-10\\
-2.0966015625	4.72081887402799e-10\\
-2.076103515625	7.43507426497227e-10\\
-2.05560546875	6.99125650279793e-10\\
-2.035107421875	6.53803325058743e-10\\
-2.014609375	4.66436197817881e-10\\
-1.994111328125	3.68514141925315e-10\\
-1.97361328125	3.18738370798498e-10\\
-1.953115234375	2.54710438219728e-10\\
-1.9326171875	2.40908816831456e-10\\
-1.912119140625	2.28856112027945e-10\\
-1.89162109375	1.39108559733311e-10\\
-1.871123046875	1.8757201898121e-10\\
-1.850625	8.1037815741068e-12\\
-1.830126953125	-3.86658725012992e-11\\
-1.80962890625	8.9687712963462e-11\\
-1.789130859375	-2.00205755889345e-10\\
-1.7686328125	-7.11360450540897e-11\\
-1.748134765625	-3.46577041124643e-10\\
-1.72763671875	-2.817970520201e-10\\
-1.707138671875	-3.41501856446093e-10\\
-1.686640625	-3.42105860017513e-10\\
-1.666142578125	-4.42079592762051e-10\\
-1.64564453125	-3.02080016129961e-10\\
-1.625146484375	-4.57198310241357e-10\\
-1.6046484375	-2.36751802415647e-10\\
-1.584150390625	-2.05585450856336e-10\\
-1.56365234375	-2.63374393868697e-10\\
-1.543154296875	-4.30755361908287e-10\\
-1.52265625	-3.78194621838431e-10\\
-1.502158203125	-4.72195885002426e-10\\
-1.48166015625	-5.49683719794182e-10\\
-1.461162109375	-6.8460409221903e-10\\
-1.4406640625	-6.07871657205191e-10\\
-1.420166015625	-8.64484630707967e-10\\
-1.39966796875	-6.79348938404467e-10\\
-1.379169921875	-6.11717455471824e-10\\
-1.358671875	-3.52101224536977e-10\\
-1.338173828125	-2.61973533386822e-10\\
-1.31767578125	-1.36911446931495e-10\\
-1.297177734375	-1.99599612413978e-10\\
-1.2766796875	-1.47095341689314e-10\\
-1.256181640625	-3.03764842927679e-10\\
-1.23568359375	-2.02652480557025e-10\\
-1.215185546875	-5.3280913633961e-10\\
-1.1946875	-4.93506418735025e-10\\
-1.174189453125	-6.93017720506208e-10\\
-1.15369140625	-6.08785543094064e-10\\
-1.133193359375	-7.1436302012525e-10\\
-1.1126953125	-4.8608378513689e-10\\
-1.092197265625	-4.77747513326333e-10\\
-1.07169921875	-9.82901042482708e-11\\
-1.051201171875	-3.06685099590617e-10\\
-1.030703125	1.08962873783246e-12\\
-1.010205078125	-9.10462832187623e-11\\
-0.989707031249999	-1.25776383181246e-10\\
-0.969208984375001	-3.55073347009764e-10\\
-0.948710937499996	-2.35510086352566e-10\\
-0.928212890624998	-6.93713731782026e-10\\
-0.90771484375	-5.29719156271039e-10\\
-0.887216796875002	-7.94512544955936e-10\\
-0.866718749999997	-6.04852668047306e-10\\
-0.846220703124999	-6.46407676007171e-10\\
-0.825722656250001	-4.53843926009696e-10\\
-0.805224609374996	-4.47212432255496e-10\\
-0.784726562499998	-2.42028705586099e-10\\
-0.764228515625	-2.37605032652367e-10\\
-0.743730468750002	-2.076572229356e-10\\
-0.723232421874997	-2.70077444855668e-10\\
-0.702734374999999	-5.21437430597273e-10\\
-0.682236328125001	-6.29596848497733e-10\\
-0.661738281249995	-8.92590150928165e-10\\
-0.641240234374997	-8.62382201400737e-10\\
-0.620742187499999	-9.34911741260589e-10\\
-0.600244140625001	-8.96582856029409e-10\\
-0.579746093749996	-8.19979405327283e-10\\
-0.559248046874998	-6.200664650214e-10\\
-0.53875	-5.35434996533569e-10\\
-0.518251953125002	-2.77576294833992e-10\\
-0.497753906249997	-4.45279312585137e-10\\
-0.477255859374999	-1.87413700333495e-10\\
-0.456757812500001	-3.73807952659298e-10\\
-0.436259765624996	-3.22361100708327e-10\\
-0.415761718749998	-3.17054093345858e-10\\
-0.395263671875	-2.18611000916642e-10\\
-0.374765625000002	-2.90562337173867e-10\\
-0.354267578124997	-5.47477432202269e-12\\
-0.333769531249999	-6.41289288940673e-11\\
-0.313271484375001	1.12849827506011e-10\\
-0.292773437499996	1.44379989361341e-10\\
-0.272275390624998	3.64757891906201e-10\\
-0.25177734375	2.71161443839943e-10\\
-0.231279296875002	4.98621344574356e-10\\
-0.210781249999997	4.98431780053783e-10\\
-0.190283203124999	7.87131357505675e-10\\
-0.169785156250001	6.92243789208462e-10\\
-0.149287109374995	9.04353838880999e-10\\
-0.128789062499997	9.80652025773304e-10\\
-0.108291015624999	9.71786447610259e-10\\
-0.0877929687500014	1.06153025599499e-09\\
-0.0672949218749963	1.05314768099806e-09\\
-0.0467968749999983	1.16820901523835e-09\\
-0.0262988281250003	1.14803473517527e-09\\
-0.00580078125000227	1.22722663170255e-09\\
0.0146972656250028	1.33719038078653e-09\\
0.0351953125000009	1.3966177244681e-09\\
0.0556933593749989	1.18342428868619e-09\\
0.076191406250004	1.33712243332107e-09\\
0.096689453125002	1.37930272360571e-09\\
0.1171875	1.53816336228712e-09\\
0.137685546874998	1.42208666526754e-09\\
0.158183593750003	1.68551056056152e-09\\
0.178681640625001	1.73698752896279e-09\\
0.199179687499999	1.94758187883086e-09\\
0.219677734375004	1.86976297010165e-09\\
0.240175781250002	2.17558832854747e-09\\
0.260673828125	1.99591197775805e-09\\
0.281171874999998	2.09436854794601e-09\\
0.301669921875003	1.88337172294814e-09\\
0.322167968750001	2.07917434429279e-09\\
0.342666015624999	1.8560617377878e-09\\
0.363164062499997	1.99588652033232e-09\\
0.383662109375003	1.89271828573449e-09\\
0.404160156250001	2.03966766299496e-09\\
0.424658203124999	1.96001440754492e-09\\
0.445156250000004	2.25942363716886e-09\\
0.465654296875002	2.24943525508698e-09\\
0.48615234375	2.42453225578649e-09\\
0.506650390624998	2.4892927358175e-09\\
0.527148437500003	2.54684221006687e-09\\
0.547646484375001	2.65162538151762e-09\\
0.568144531249999	2.74414364867003e-09\\
0.588642578125004	2.79578756265129e-09\\
0.609140625000002	2.93674644729242e-09\\
0.629638671875	3.11819412247697e-09\\
0.650136718749998	3.24595224954067e-09\\
0.670634765625003	3.38729638927049e-09\\
0.691132812500001	3.53128748379507e-09\\
0.711630859374999	3.68238986190188e-09\\
0.732128906250004	3.80754193251861e-09\\
0.752626953125002	4.01020670564337e-09\\
0.773125	3.8296927560272e-09\\
0.793623046874998	4.09447869906455e-09\\
0.814121093750003	4.1524970385662e-09\\
0.834619140625001	4.1940087080238e-09\\
0.855117187499999	4.11455228615808e-09\\
0.875615234374997	4.50552881014117e-09\\
0.896113281250003	4.29527614374813e-09\\
0.916611328125001	4.53363539428854e-09\\
0.937109374999999	4.53020860456487e-09\\
0.957607421875004	4.61923539588084e-09\\
0.978105468750002	4.62665443845735e-09\\
0.998603515625	4.7613304204717e-09\\
1.0191015625	4.68856497391595e-09\\
1.039599609375	4.87362438837712e-09\\
1.06009765625	4.70827461309004e-09\\
1.080595703125	5.08732608925599e-09\\
1.10109375	5.05457458939742e-09\\
1.121591796875	5.33475158804985e-09\\
1.14208984375	5.32092067205555e-09\\
1.162587890625	5.26892953647476e-09\\
1.1830859375	5.43443839478097e-09\\
1.203583984375	5.27988832930853e-09\\
1.22408203125	5.3257577144296e-09\\
1.244580078125	5.45280865010593e-09\\
1.265078125	5.40937184497147e-09\\
1.285576171875	5.65788761975894e-09\\
1.30607421875	5.75609539810344e-09\\
1.326572265625	5.99709044034859e-09\\
1.3470703125	6.04559056392444e-09\\
1.367568359375	6.33285686232466e-09\\
1.38806640625	6.35769555752182e-09\\
1.408564453125	6.3653125092676e-09\\
1.4290625	6.2219179694431e-09\\
1.449560546875	6.42646588687519e-09\\
1.47005859375	5.97894834296338e-09\\
1.490556640625	6.23832964895638e-09\\
1.5110546875	6.2487749040555e-09\\
1.531552734375	6.46917649594991e-09\\
1.55205078125	6.53701111503975e-09\\
1.572548828125	6.97771129923239e-09\\
1.593046875	6.96302190975036e-09\\
1.613544921875	7.34972824865083e-09\\
1.63404296875	7.13951599278312e-09\\
1.654541015625	7.38602365076758e-09\\
1.6750390625	7.09794143936191e-09\\
1.695537109375	7.13866422403821e-09\\
1.71603515625	7.12868919058419e-09\\
1.736533203125	7.10883166929835e-09\\
1.75703125	7.30699265310153e-09\\
1.777529296875	7.48627638155341e-09\\
1.79802734375	7.74325068753366e-09\\
1.818525390625	8.06655751936096e-09\\
1.8390234375	8.34919298669442e-09\\
1.859521484375	8.59225922919574e-09\\
1.88001953125	8.73399952451866e-09\\
1.900517578125	8.77136144471868e-09\\
1.921015625	8.74995514653399e-09\\
1.941513671875	8.60257664001419e-09\\
1.96201171875	8.62637494807048e-09\\
1.982509765625	8.61603529470752e-09\\
2.0030078125	8.69923720590893e-09\\
2.023505859375	8.70259703400786e-09\\
2.04400390625	9.12529067297501e-09\\
2.064501953125	9.04321796860391e-09\\
2.085	9.28390490978786e-09\\
2.105498046875	9.23657151367677e-09\\
2.12599609375	9.27017733366119e-09\\
2.146494140625	9.07341558922037e-09\\
2.1669921875	9.10906190499613e-09\\
2.187490234375	8.85417142666965e-09\\
2.20798828125	8.97210045327035e-09\\
2.228486328125	8.90604663152304e-09\\
2.248984375	8.96592175687034e-09\\
2.269482421875	8.8217989396866e-09\\
2.28998046875	8.9386341114063e-09\\
2.310478515625	8.71474830037479e-09\\
2.3309765625	8.7986299035719e-09\\
2.351474609375	8.54904202961748e-09\\
2.37197265625	8.55369215734158e-09\\
2.392470703125	8.43597108410854e-09\\
2.41296875	8.46875249373771e-09\\
2.433466796875	8.39845116111748e-09\\
2.45396484375	8.46850919474696e-09\\
2.474462890625	8.34257298028888e-09\\
2.4949609375	8.47284768318775e-09\\
2.515458984375	8.38828836535971e-09\\
2.53595703125	8.44109710177638e-09\\
2.556455078125	8.47189688867044e-09\\
2.576953125	8.52031804585033e-09\\
2.597451171875	8.55913811096678e-09\\
2.61794921875	8.47372662391232e-09\\
2.638447265625	8.35433514468389e-09\\
2.6589453125	8.26940456180873e-09\\
2.679443359375	8.08396353170507e-09\\
2.69994140625	8.02698881013463e-09\\
2.720439453125	8.06292868337253e-09\\
2.7409375	7.97163781435869e-09\\
2.761435546875	8.13097143155218e-09\\
2.78193359375	8.1213230153366e-09\\
2.802431640625	8.34271277696955e-09\\
2.8229296875	8.26408578891268e-09\\
2.843427734375	8.34344124786416e-09\\
2.86392578125	8.33436497613599e-09\\
2.884423828125	8.58721797236361e-09\\
2.904921875	8.39798374006984e-09\\
2.925419921875	8.5595816444983e-09\\
2.94591796875	8.41927108074935e-09\\
2.966416015625	8.50046483827934e-09\\
2.9869140625	8.40957925303245e-09\\
3.007412109375	8.36763924257873e-09\\
3.02791015625	8.26417982853808e-09\\
3.048408203125	8.07887432748514e-09\\
3.06890625	8.06842283491618e-09\\
3.089404296875	7.90627471460151e-09\\
3.10990234375	7.78444665886819e-09\\
3.130400390625	7.65861230349911e-09\\
3.1508984375	7.41798259218486e-09\\
3.171396484375	7.62405640578575e-09\\
3.19189453125	7.39417173054285e-09\\
3.212392578125	7.44408376435693e-09\\
3.232890625	7.44006461678262e-09\\
3.253388671875	7.41030191384428e-09\\
3.27388671875	7.23274000092562e-09\\
3.294384765625	7.04090243655242e-09\\
3.3148828125	6.92935094113132e-09\\
3.335380859375	6.53779365453731e-09\\
3.35587890625	6.57737099323851e-09\\
3.376376953125	6.17323124968107e-09\\
3.396875	6.35579269839557e-09\\
3.417373046875	6.26861553411176e-09\\
3.43787109375	6.436008765307e-09\\
3.458369140625	6.38342488634272e-09\\
3.4788671875	6.37336581350718e-09\\
3.499365234375	6.33817414156578e-09\\
3.51986328125	6.17610878443545e-09\\
3.540361328125	6.11147172129584e-09\\
3.560859375	5.67901475396757e-09\\
3.581357421875	5.41236887745167e-09\\
3.60185546875	5.4181036430771e-09\\
3.622353515625	5.21641089145293e-09\\
3.6428515625	5.34545169050897e-09\\
3.663349609375	5.49204723445676e-09\\
3.68384765625	5.58991678042281e-09\\
3.704345703125	5.680106941793e-09\\
3.72484375	5.61065888311661e-09\\
3.745341796875	5.4113410605348e-09\\
3.76583984375	5.22669154596205e-09\\
3.786337890625	4.6047495991262e-09\\
3.8068359375	4.70987128226347e-09\\
3.827333984375	4.14102696812191e-09\\
3.84783203125	4.12993002562388e-09\\
3.868330078125	4.02983066083446e-09\\
3.888828125	3.96843026419929e-09\\
3.909326171875	4.0403485530143e-09\\
3.92982421875	4.33500335030249e-09\\
3.950322265625	4.293242396198e-09\\
3.9708203125	4.52839331264002e-09\\
3.991318359375	4.13215250872455e-09\\
4.01181640625	3.9836046369573e-09\\
4.032314453125	3.73845720916033e-09\\
4.0528125	3.40380075978475e-09\\
4.073310546875	3.32748999706286e-09\\
4.09380859375	3.16154764904133e-09\\
4.114306640625	3.05585749727038e-09\\
4.1348046875	3.25732685220241e-09\\
4.155302734375	3.21489657279625e-09\\
4.17580078125	3.58017415292158e-09\\
4.196298828125	3.47622178977554e-09\\
4.216796875	3.29461391459089e-09\\
4.237294921875	3.27635199898006e-09\\
4.25779296875	2.79452375251668e-09\\
4.278291015625	2.62693839183826e-09\\
4.2987890625	2.3240596375784e-09\\
4.319287109375	2.31202217049854e-09\\
4.33978515625	2.16857448576178e-09\\
4.360283203125	2.24403486114774e-09\\
4.38078125	2.37573940328071e-09\\
4.401279296875	2.45244142889788e-09\\
4.42177734375	2.36618487358155e-09\\
4.442275390625	2.20730735961876e-09\\
4.4627734375	1.94307009486558e-09\\
4.483271484375	1.83901020831133e-09\\
4.50376953125	1.36809004985015e-09\\
4.524267578125	1.38061537647319e-09\\
4.544765625	1.14081282499008e-09\\
4.565263671875	1.19757544642266e-09\\
4.58576171875	1.13294484357073e-09\\
4.606259765625	1.42596137751531e-09\\
4.6267578125	1.45058722147117e-09\\
4.647255859375	1.65024270499556e-09\\
4.66775390625	1.48621292283868e-09\\
4.688251953125	1.6177901260023e-09\\
4.70875	1.24738735192705e-09\\
4.729248046875	1.08684677200851e-09\\
4.74974609375	8.89195771133911e-10\\
4.770244140625	7.80608820770718e-10\\
4.7907421875	6.80577334609079e-10\\
4.811240234375	8.10359744074384e-10\\
4.83173828125	7.83298278323637e-10\\
4.852236328125	1.2071573453687e-09\\
4.872734375	1.12890149559986e-09\\
4.893232421875	1.49225144786763e-09\\
4.91373046875	1.32748077128098e-09\\
4.934228515625	1.50258145755399e-09\\
4.9547265625	1.1908093016396e-09\\
4.975224609375	1.08254063848763e-09\\
4.99572265625	8.89403584635476e-10\\
5.016220703125	9.17041884927047e-10\\
5.03671875	7.45052185868798e-10\\
5.057216796875	7.4675373149074e-10\\
5.07771484375	9.09514924672255e-10\\
5.098212890625	8.46315514407641e-10\\
5.1187109375	1.01576768374185e-09\\
5.139208984375	9.55112832938894e-10\\
5.15970703125	8.98596478613913e-10\\
5.180205078125	6.20572215426376e-10\\
5.200703125	4.47907591689857e-10\\
5.221201171875	1.40981312766134e-10\\
5.24169921875	6.03161961436029e-12\\
5.262197265625	-1.49916743689176e-10\\
5.2826953125	-1.63644008768049e-10\\
5.303193359375	-2.15832954264673e-10\\
5.32369140625	3.04277573804665e-11\\
5.344189453125	-9.25715076685974e-11\\
5.3646875	1.28689214031898e-10\\
5.385185546875	-3.82614634426035e-11\\
5.40568359375	-9.05470198226552e-11\\
5.426181640625	-3.94145315736702e-10\\
5.4466796875	-4.85380878718467e-10\\
5.467177734375	-8.53306026708216e-10\\
5.48767578125	-8.26064547200432e-10\\
5.508173828125	-8.85096018502397e-10\\
5.528671875	-8.10089734707505e-10\\
5.549169921875	-6.82107070375858e-10\\
5.56966796875	-5.75504287327418e-10\\
5.590166015625	-6.35366464805158e-10\\
5.6106640625	-4.59578143903484e-10\\
5.631162109375	-6.84283494022021e-10\\
5.65166015625	-3.84210421730314e-10\\
5.672158203125	-6.90031052483324e-10\\
5.69265625	-7.89956535886252e-10\\
5.713154296875	-8.85029861619654e-10\\
5.73365234375	-9.23875422070292e-10\\
5.754150390625	-9.7451777882251e-10\\
5.7746484375	-7.94932270383048e-10\\
5.795146484375	-7.22625438105233e-10\\
5.81564453125	-8.57819423102325e-10\\
5.836142578125	-6.80449502455731e-10\\
5.856640625	-8.52537974470326e-10\\
5.877138671875	-7.12604404692839e-10\\
5.89763671875	-8.82171542960335e-10\\
5.918134765625	-7.83867586071657e-10\\
5.9386328125	-9.32217800276492e-10\\
5.959130859375	-9.07342158573442e-10\\
5.97962890625	-9.04733471555279e-10\\
6.000126953125	-8.65600727774555e-10\\
6.020625	-7.4544431920695e-10\\
6.041123046875	-8.2185826497741e-10\\
6.06162109375	-7.60279309250926e-10\\
6.082119140625	-6.82729353621283e-10\\
6.1026171875	-8.6607173791418e-10\\
6.123115234375	-8.68521870489855e-10\\
6.14361328125	-8.91134067902034e-10\\
6.164111328125	-1.00144360002691e-09\\
6.184609375	-1.10210747378394e-09\\
6.205107421875	-1.07026087289885e-09\\
6.22560546875	-1.1406705760698e-09\\
6.246103515625	-1.04155068915624e-09\\
6.2666015625	-1.03387854268836e-09\\
6.287099609375	-8.78370661810964e-10\\
6.30759765625	-9.26237191339172e-10\\
6.328095703125	-7.16503110432243e-10\\
6.34859375	-8.09648452540608e-10\\
6.369091796875	-7.10596748096732e-10\\
6.38958984375	-7.86617206143421e-10\\
6.410087890625	-7.86281907190264e-10\\
6.4305859375	-8.59747528505607e-10\\
6.451083984375	-8.90894557357034e-10\\
6.47158203125	-9.68277910933152e-10\\
6.492080078125	-7.8097280136032e-10\\
6.512578125	-9.51659789034839e-10\\
6.533076171875	-8.98650108898572e-10\\
6.55357421875	-7.85504844285749e-10\\
6.574072265625	-7.58357455721696e-10\\
6.5945703125	-6.87354993076747e-10\\
6.615068359375	-6.4218026365511e-10\\
6.63556640625	-6.93984981636774e-10\\
6.656064453125	-5.04314379624613e-10\\
6.6765625	-7.9498701199182e-10\\
6.697060546875	-6.65425215781635e-10\\
6.71755859375	-6.63659682841329e-10\\
6.738056640625	-7.98629637877061e-10\\
6.7585546875	-5.55592696844725e-10\\
6.779052734375	-6.92449719760322e-10\\
6.79955078125	-6.77612559124829e-10\\
6.820048828125	-6.39423625339388e-10\\
6.840546875	-6.29406333574349e-10\\
6.861044921875	-5.53649097544684e-10\\
6.88154296875	-5.49567681198473e-10\\
6.902041015625	-4.85259643946861e-10\\
6.9225390625	-3.75129457841985e-10\\
6.943037109375	-3.79695156454076e-10\\
6.96353515625	-3.60959734000657e-10\\
6.984033203125	-3.9356574751302e-10\\
7.00453125	-3.20544110452639e-10\\
7.025029296875	-3.47515842547371e-10\\
7.04552734375	-3.71205900337296e-10\\
7.066025390625	-4.60527605610922e-10\\
7.0865234375	-3.28798862142439e-10\\
7.107021484375	-4.54523377946592e-10\\
7.12751953125	-3.50690180787128e-10\\
7.148017578125	-4.61568707057604e-10\\
7.168515625	-3.28542479283926e-10\\
7.189013671875	-4.71341412161501e-10\\
7.20951171875	-4.09224354035885e-10\\
7.230009765625	-5.53148001008677e-10\\
7.2505078125	-4.38266215009385e-10\\
7.271005859375	-4.70071109450928e-10\\
7.29150390625	-3.73946626961989e-10\\
7.312001953125	-4.01024834401882e-10\\
7.3325	-3.82584004964973e-10\\
7.352998046875	-5.75995572526883e-10\\
7.37349609375	-4.35941159782376e-10\\
7.393994140625	-7.70290446816139e-10\\
7.4144921875	-6.41596612135542e-10\\
7.434990234375	-8.22945746130532e-10\\
7.45548828125	-6.74168077677621e-10\\
7.475986328125	-8.41593451680934e-10\\
7.496484375	-6.72971410536941e-10\\
7.516982421875	-6.73517200116971e-10\\
7.53748046875	-5.59165026717644e-10\\
7.557978515625	-6.30094056862894e-10\\
7.5784765625	-6.10252720097256e-10\\
7.598974609375	-6.54034422925427e-10\\
7.61947265625	-7.19344937311713e-10\\
7.639970703125	-7.58639459248126e-10\\
7.66046875	-7.27939055701202e-10\\
7.680966796875	-7.31975121641139e-10\\
7.70146484375	-6.8659219056406e-10\\
7.721962890625	-5.77138646171264e-10\\
7.7424609375	-5.8487772824261e-10\\
7.762958984375	-3.90774485760525e-10\\
7.78345703125	-4.20667417094695e-10\\
7.803955078125	-3.55463263775712e-10\\
7.824453125	-4.00585649920655e-10\\
7.844951171875	-1.88292365635665e-10\\
7.86544921875	-4.40329650689212e-10\\
7.885947265625	-1.753296597688e-10\\
7.9064453125	-1.99054254416789e-10\\
7.926943359375	-8.57765806498247e-11\\
7.94744140625	-2.74897424763197e-11\\
7.967939453125	1.76578655776571e-12\\
7.9884375	-2.37313398384604e-11\\
8.008935546875	-4.88240022274196e-11\\
8.02943359375	-8.679735508496e-11\\
8.049931640625	-8.95293747899908e-11\\
8.0704296875	-1.78415463073371e-10\\
8.090927734375	-1.4447152821714e-10\\
8.11142578125	-8.22295910728889e-11\\
8.131923828125	1.05864376128446e-10\\
8.152421875	-1.32096918517577e-10\\
8.172919921875	8.00081855727289e-11\\
8.19341796875	2.50632524272707e-11\\
8.213916015625	1.2477426308477e-10\\
8.2344140625	-2.73566969787237e-12\\
8.254912109375	-3.5705597062514e-11\\
8.27541015625	-1.10541511171487e-10\\
8.295908203125	-2.40705916109662e-10\\
8.31640625	-1.78458946211768e-10\\
8.336904296875	-2.70161263073764e-10\\
8.35740234375	-1.31172487716089e-10\\
8.377900390625	-2.10334357246755e-10\\
8.3983984375	1.20843069579049e-11\\
8.418896484375	-2.37335309689561e-10\\
8.43939453125	-6.40466590862536e-11\\
8.459892578125	-2.45031092614203e-10\\
8.480390625	-1.86157177956013e-10\\
8.500888671875	-2.52265730414999e-10\\
8.52138671875	-2.15018832189702e-10\\
8.541884765625	-1.485951032009e-10\\
8.5623828125	-2.14319335635229e-10\\
8.582880859375	-1.98575797866773e-10\\
8.60337890625	-2.09748293201374e-10\\
8.623876953125	-2.40059242923688e-10\\
8.644375	-1.79518006362972e-10\\
8.664873046875	-2.27276668092317e-10\\
8.68537109375	-2.19144545716692e-10\\
8.705869140625	-1.46328097469573e-10\\
8.7263671875	7.02726376798214e-12\\
8.746865234375	-2.79652736943483e-12\\
8.76736328125	5.78773118590714e-11\\
8.787861328125	1.03476948694684e-11\\
8.808359375	3.69147135902988e-11\\
8.828857421875	-7.83086616552474e-11\\
8.84935546875	-1.17994863347601e-10\\
8.869853515625	-1.71592145097567e-10\\
8.8903515625	-1.74521932337284e-10\\
8.910849609375	-2.70553141320031e-10\\
8.93134765625	-1.93835542951518e-10\\
8.951845703125	-2.23946880390118e-10\\
8.97234375	-7.23680552267433e-11\\
8.992841796875	-8.37141938133673e-11\\
9.01333984375	-1.7383128027411e-11\\
9.033837890625	-1.39124367313716e-10\\
9.0543359375	-2.07571127608266e-10\\
9.074833984375	-2.46614687612051e-10\\
9.09533203125	-3.09148221713129e-10\\
9.115830078125	-3.84599334586212e-10\\
9.136328125	-4.15671785233632e-10\\
9.156826171875	-5.15679266988149e-10\\
9.17732421875	-2.867627693556e-10\\
9.197822265625	-4.0287877747079e-10\\
9.2183203125	-2.73804475117825e-10\\
9.238818359375	-3.18276200153498e-10\\
9.25931640625	-3.98475070169427e-10\\
9.279814453125	-3.80331109378599e-10\\
9.3003125	-4.29147574453722e-10\\
9.320810546875	-4.65943971030016e-10\\
9.34130859375	-5.7280213689297e-10\\
9.361806640625	-5.64829815933734e-10\\
9.3823046875	-6.21275739369124e-10\\
9.402802734375	-5.97816998230066e-10\\
9.42330078125	-4.97192036064808e-10\\
9.443798828125	-6.29099025427038e-10\\
9.464296875	-6.87858228252662e-10\\
9.484794921875	-8.00013303624737e-10\\
9.50529296875	-8.33059669222094e-10\\
9.525791015625	-8.89231691861648e-10\\
9.5462890625	-9.86567285743451e-10\\
9.566787109375	-9.19766151188266e-10\\
9.58728515625	-9.62574800972635e-10\\
9.607783203125	-8.2662958631835e-10\\
9.62828125	-9.0964910106949e-10\\
9.648779296875	-7.48402552442889e-10\\
9.66927734375	-7.77431213203237e-10\\
9.689775390625	-6.33107279974837e-10\\
9.7102734375	-7.37443383969071e-10\\
9.730771484375	-5.69510495235058e-10\\
9.75126953125	-7.50271645019416e-10\\
9.771767578125	-7.3180173278661e-10\\
9.792265625	-7.43097843943462e-10\\
9.812763671875	-7.43634227501976e-10\\
9.83326171875	-7.93705120350467e-10\\
9.853759765625	-5.6803116307381e-10\\
9.8742578125	-7.56887147843493e-10\\
9.894755859375	-3.817091403712e-10\\
9.91525390625	-6.02160015232271e-10\\
9.935751953125	-3.51463112058964e-10\\
9.95625	-5.52556936740015e-10\\
9.976748046875	-4.23846707381878e-10\\
9.99724609375	-5.9601681612216e-10\\
10.017744140625	-6.06746568191928e-10\\
10.0382421875	-5.77444418102392e-10\\
10.058740234375	-5.62634386407171e-10\\
10.07923828125	-5.85967168667177e-10\\
10.099736328125	-5.84077869082041e-10\\
10.120234375	-4.0520281145841e-10\\
10.140732421875	-3.95690112439192e-10\\
10.16123046875	-4.22801169206922e-10\\
10.181728515625	-3.88317056301213e-10\\
10.2022265625	-5.69693066276421e-10\\
10.222724609375	-6.07347910056954e-10\\
10.24322265625	-6.5123963645621e-10\\
10.263720703125	-8.61621416845137e-10\\
10.28421875	-8.25584674813395e-10\\
10.304716796875	-9.03764844364459e-10\\
10.32521484375	-8.94321804488299e-10\\
10.345712890625	-9.24518649240021e-10\\
10.3662109375	-7.85271140688681e-10\\
10.386708984375	-9.06475591062876e-10\\
10.40720703125	-7.86060937372125e-10\\
10.427705078125	-8.63558175524696e-10\\
10.448203125	-9.0221400261278e-10\\
10.468701171875	-9.75563234261104e-10\\
10.48919921875	-1.03421509726264e-09\\
10.509697265625	-1.00661592003781e-09\\
10.5301953125	-1.06614283711668e-09\\
10.550693359375	-1.13659210960897e-09\\
10.57119140625	-1.00188938691581e-09\\
10.591689453125	-1.09939611339849e-09\\
10.6121875	-1.04558305880029e-09\\
10.632685546875	-1.04490162671268e-09\\
10.65318359375	-1.00762486367332e-09\\
10.673681640625	-1.12419248698182e-09\\
10.6941796875	-1.07244840287473e-09\\
10.714677734375	-1.17581207838806e-09\\
10.73517578125	-1.11724907036203e-09\\
10.755673828125	-1.14713517628929e-09\\
10.776171875	-1.11764930987167e-09\\
10.796669921875	-1.09019783408356e-09\\
10.81716796875	-1.05398667940312e-09\\
10.837666015625	-9.98234553331049e-10\\
10.8581640625	-1.01776497670953e-09\\
10.878662109375	-9.29440879607481e-10\\
10.89916015625	-1.0198156764566e-09\\
10.919658203125	-1.01492941596151e-09\\
10.94015625	-1.09936820776231e-09\\
10.960654296875	-9.02303652237927e-10\\
10.98115234375	-1.01449325490506e-09\\
11.001650390625	-8.81886312828817e-10\\
11.0221484375	-9.38347218456425e-10\\
11.042646484375	-9.04009262406756e-10\\
11.06314453125	-6.78265554825816e-10\\
11.083642578125	-7.32110208745405e-10\\
11.104140625	-6.31098178365139e-10\\
11.124638671875	-7.05123765242174e-10\\
11.14513671875	-7.68005616070548e-10\\
11.165634765625	-7.22351649972165e-10\\
11.1861328125	-8.40887800542301e-10\\
11.206630859375	-8.18076807883256e-10\\
11.22712890625	-8.63245608028363e-10\\
11.247626953125	-7.88347059463073e-10\\
11.268125	-7.95829154043389e-10\\
11.288623046875	-7.59224680340864e-10\\
11.30912109375	-6.33420720502752e-10\\
11.329619140625	-5.59436984271356e-10\\
11.3501171875	-5.39404317956328e-10\\
11.370615234375	-4.48835412669954e-10\\
11.39111328125	-6.13111436517923e-10\\
11.411611328125	-4.95479921684735e-10\\
11.432109375	-5.30605331836171e-10\\
11.452607421875	-4.55395132692155e-10\\
11.47310546875	-6.1968728713427e-10\\
11.493603515625	-5.66469245368653e-10\\
11.5141015625	-4.71882626397429e-10\\
11.534599609375	-4.13649649501806e-10\\
11.55509765625	-3.79251756783631e-10\\
11.575595703125	-1.81453989934055e-10\\
11.59609375	-1.56204511647866e-10\\
11.616591796875	-3.72713811323209e-11\\
11.63708984375	-4.99411543183921e-11\\
11.657587890625	6.56864033261238e-12\\
11.6780859375	-1.32551622521574e-12\\
11.698583984375	-1.54661982732073e-11\\
11.71908203125	-9.46178976523838e-11\\
11.739580078125	-5.6959095744044e-11\\
11.760078125	-2.10161305353752e-10\\
11.780576171875	-8.38117950889522e-11\\
11.80107421875	-7.79605279985619e-11\\
11.821572265625	3.29956505688046e-11\\
11.8420703125	1.12020668990848e-10\\
11.862568359375	1.59436484252021e-12\\
11.88306640625	9.17947003972352e-11\\
11.903564453125	2.4613721815001e-12\\
11.9240625	-6.24336384100812e-11\\
11.944560546875	9.70395963222315e-11\\
11.96505859375	7.38219901611288e-11\\
11.985556640625	2.01313357242098e-10\\
12.0060546875	1.45432671579443e-10\\
12.026552734375	2.45186206286269e-10\\
12.04705078125	3.28518500888364e-10\\
12.067548828125	1.81120290771291e-10\\
12.088046875	3.57252930993652e-10\\
12.108544921875	2.52977301204481e-10\\
12.12904296875	3.72282179243353e-10\\
12.149541015625	1.78242615742089e-10\\
12.1700390625	2.73018833655851e-10\\
12.190537109375	8.37345942420681e-11\\
12.21103515625	2.20623104693787e-10\\
12.231533203125	1.03087056214633e-10\\
12.25203125	2.83587072670495e-10\\
12.272529296875	1.3863081424238e-10\\
12.29302734375	1.94843378721006e-10\\
12.313525390625	2.14944126052075e-10\\
12.3340234375	1.71044242420309e-10\\
12.354521484375	1.28578191632246e-10\\
12.37501953125	1.87538792714813e-10\\
12.395517578125	7.10450817124075e-11\\
12.416015625	3.47588646345259e-10\\
12.436513671875	1.13076601129067e-10\\
12.45701171875	4.18124860476336e-10\\
12.477509765625	2.8550983722287e-10\\
12.4980078125	3.56840254554716e-10\\
12.518505859375	2.63746232801342e-10\\
12.53900390625	1.97681962964962e-10\\
12.559501953125	1.81770019862754e-10\\
12.58	2.99835550952223e-10\\
12.600498046875	1.88215775851626e-10\\
12.62099609375	2.03415334487163e-10\\
12.641494140625	2.42869353364138e-10\\
12.6619921875	2.26049092133509e-10\\
12.682490234375	2.59774254436892e-10\\
12.70298828125	3.98361299467737e-10\\
12.723486328125	3.10024429670334e-10\\
12.743984375	3.53749413936281e-10\\
12.764482421875	3.91635781865105e-10\\
12.78498046875	3.76693411449078e-10\\
12.805478515625	3.76891853990341e-10\\
12.8259765625	3.67193204237849e-10\\
12.846474609375	3.58262828638741e-10\\
12.86697265625	2.28117277790182e-10\\
12.887470703125	2.23915558342083e-10\\
12.90796875	2.21916137582871e-10\\
12.928466796875	3.81085735202691e-10\\
12.94896484375	2.94948830178779e-10\\
12.969462890625	3.73936591708125e-10\\
12.9899609375	4.13224320157015e-10\\
13.010458984375	5.57113742546289e-10\\
13.03095703125	4.76729246122444e-10\\
13.051455078125	3.93417474443261e-10\\
13.071953125	3.59328649326808e-10\\
13.092451171875	4.74682428677662e-10\\
13.11294921875	2.29396017077069e-10\\
13.133447265625	4.34369421663614e-10\\
13.1539453125	3.36836460860103e-10\\
13.174443359375	4.74079764612759e-10\\
13.19494140625	4.09508819655914e-10\\
13.215439453125	5.5643989599978e-10\\
13.2359375	5.14881098167559e-10\\
13.256435546875	4.94892232193306e-10\\
13.27693359375	3.81326666023195e-10\\
13.297431640625	4.52381396237667e-10\\
13.3179296875	3.72062417100488e-10\\
13.338427734375	4.3880372619212e-10\\
13.35892578125	3.68775072383364e-10\\
13.379423828125	4.2587193816542e-10\\
13.399921875	3.90306560863822e-10\\
13.420419921875	4.11644686238501e-10\\
13.44091796875	5.17933377771843e-10\\
13.461416015625	3.99508829521914e-10\\
13.4819140625	4.43823549558467e-10\\
13.502412109375	3.05013376900086e-10\\
13.52291015625	3.23658566989454e-10\\
13.543408203125	2.36448107145216e-10\\
13.56390625	2.36596259240507e-10\\
13.584404296875	3.50106773902172e-10\\
13.60490234375	2.21179974740414e-10\\
13.625400390625	2.99169630880831e-10\\
13.6458984375	2.67637133009116e-10\\
13.666396484375	2.75081785848305e-10\\
13.68689453125	2.59705011445315e-10\\
13.707392578125	2.23240224554661e-10\\
13.727890625	2.35340102410718e-10\\
13.748388671875	2.75852099560896e-10\\
13.76888671875	2.7144607337812e-10\\
13.789384765625	2.27241275900914e-10\\
13.8098828125	2.65047450786207e-10\\
13.830380859375	2.24409911751933e-10\\
13.85087890625	2.52889366653107e-10\\
13.871376953125	1.04736151317943e-10\\
13.891875	2.37720316365321e-10\\
13.912373046875	-4.36567423849805e-12\\
13.93287109375	1.64761299609786e-10\\
13.953369140625	5.244906841232e-11\\
13.9738671875	1.66423125489314e-10\\
13.994365234375	2.16201427163994e-10\\
14.01486328125	9.65425577982131e-11\\
14.035361328125	4.76350171460156e-11\\
14.055859375	1.36300470250361e-10\\
14.076357421875	-1.68643995764078e-10\\
14.09685546875	-1.0452029859916e-10\\
14.117353515625	-2.93995557739879e-10\\
14.1378515625	-3.29379576995112e-10\\
14.158349609375	-4.78046454829873e-10\\
14.17884765625	-5.05809504244804e-10\\
14.199345703125	-3.92646925286261e-10\\
14.21984375	-3.18879594225773e-10\\
14.240341796875	-3.61400408098187e-10\\
14.26083984375	-2.19943285166813e-10\\
14.281337890625	-3.19302799429051e-10\\
14.3018359375	-1.92537974603474e-10\\
14.322333984375	-2.86235611227769e-10\\
14.34283203125	-3.34943420310418e-10\\
14.363330078125	-2.06656765552838e-10\\
14.383828125	-3.35050965803396e-10\\
14.404326171875	-2.45180178070316e-10\\
14.42482421875	-2.06685267824875e-10\\
14.445322265625	-4.4186561021384e-10\\
14.4658203125	-2.58309806406057e-10\\
14.486318359375	-4.26901298365554e-10\\
14.50681640625	-3.02857639536403e-10\\
14.527314453125	-3.72030463249806e-10\\
14.5478125	-4.50250183965688e-10\\
14.568310546875	-3.61324316434326e-10\\
14.58880859375	-5.28788686764818e-10\\
14.609306640625	-3.51924063528175e-10\\
14.6298046875	-5.06756229233726e-10\\
14.650302734375	-3.44960692819718e-10\\
14.67080078125	-4.30773186221027e-10\\
14.691298828125	-3.58262067099863e-10\\
14.711796875	-5.33964786166416e-10\\
14.732294921875	-3.54039567037224e-10\\
14.75279296875	-4.84633235977367e-10\\
14.773291015625	-3.90364373107148e-10\\
14.7937890625	-3.51663125327464e-10\\
14.814287109375	-4.29745823196505e-10\\
14.83478515625	-4.00177386667633e-10\\
14.855283203125	-4.19644088875897e-10\\
14.87578125	-5.24342968751839e-10\\
14.896279296875	-4.59792543421886e-10\\
14.91677734375	-6.46246070159334e-10\\
14.937275390625	-3.88020382791779e-10\\
14.9577734375	-5.92101335428952e-10\\
14.978271484375	-4.19487228330098e-10\\
14.99876953125	-4.99944631272212e-10\\
15.019267578125	-3.36154574525989e-10\\
15.039765625	-3.4572709778499e-10\\
15.060263671875	-3.86222888560941e-10\\
15.08076171875	-3.88427996839577e-10\\
15.101259765625	-3.82985570845654e-10\\
15.1217578125	-5.11669540752819e-10\\
15.142255859375	-5.08753362618586e-10\\
15.16275390625	-4.95276752459651e-10\\
15.183251953125	-4.64911549701075e-10\\
15.20375	-4.97919040717424e-10\\
15.224248046875	-3.01640481677908e-10\\
15.24474609375	-5.18762861906299e-10\\
15.265244140625	-3.94224019382883e-10\\
15.2857421875	-5.0094093765179e-10\\
15.306240234375	-4.76497182166054e-10\\
15.32673828125	-4.71209919467562e-10\\
15.347236328125	-4.58655254883292e-10\\
15.367734375	-4.43982728663609e-10\\
15.388232421875	-4.15155879513104e-10\\
15.40873046875	-3.68137237943719e-10\\
15.429228515625	-5.0672161133688e-10\\
15.4497265625	-4.53753623841917e-10\\
15.470224609375	-4.08605256906568e-10\\
15.49072265625	-5.09100752190598e-10\\
15.511220703125	-5.73594637619956e-10\\
15.53171875	-4.42718613059901e-10\\
15.552216796875	-4.73855994614982e-10\\
15.57271484375	-4.29746504639083e-10\\
15.593212890625	-4.08878833659376e-10\\
15.6137109375	-2.75499859361241e-10\\
15.634208984375	-3.69008723801796e-10\\
15.65470703125	-2.88687913104265e-10\\
15.675205078125	-4.52380969473552e-10\\
15.695703125	-3.59605797733222e-10\\
15.716201171875	-5.77831786421294e-10\\
15.73669921875	-5.12730141977907e-10\\
15.757197265625	-6.29726280830945e-10\\
15.7776953125	-5.00684561851372e-10\\
15.798193359375	-4.46769289148175e-10\\
15.81869140625	-4.23751405664394e-10\\
15.839189453125	-3.67794799960107e-10\\
15.8596875	-2.9081457712667e-10\\
15.880185546875	-3.20250943243279e-10\\
15.90068359375	-2.48897432434028e-10\\
15.921181640625	-2.87620444476055e-10\\
15.9416796875	-4.07522366750269e-10\\
15.962177734375	-3.52641050435094e-10\\
15.98267578125	-5.30497508690168e-10\\
16.003173828125	-3.46085828975459e-10\\
16.023671875	-3.49876829518014e-10\\
16.044169921875	-3.58255323555033e-10\\
16.06466796875	-2.5664919033727e-10\\
16.085166015625	-3.75734998465485e-10\\
16.1056640625	-1.96512139449233e-10\\
16.126162109375	-2.44671946623252e-10\\
16.14666015625	-1.18310708375958e-10\\
16.167158203125	-1.88709774924748e-10\\
16.18765625	-2.09432615192335e-10\\
16.208154296875	-2.72793058788546e-10\\
16.22865234375	-3.07517314410067e-10\\
16.249150390625	-3.21999376308072e-10\\
16.2696484375	-3.16112345206118e-10\\
16.290146484375	-3.74947358120844e-10\\
16.31064453125	-3.91309707235806e-10\\
16.331142578125	-2.99432229671846e-10\\
16.351640625	-3.59991625791401e-10\\
16.372138671875	-1.70472790870939e-10\\
16.39263671875	-3.37938947325264e-10\\
16.413134765625	-8.42447735792799e-11\\
16.4336328125	-2.15603698955679e-10\\
16.454130859375	-3.7604851354953e-11\\
16.47462890625	-5.22003084145205e-11\\
16.495126953125	-2.49384091663218e-11\\
16.515625	-5.45861854844173e-11\\
16.536123046875	2.54508369984631e-12\\
16.55662109375	-9.36116448831215e-11\\
16.577119140625	1.56408564237559e-10\\
16.5976171875	7.08811885880861e-11\\
16.618115234375	3.04032205298607e-10\\
16.63861328125	2.5394271511208e-10\\
16.659111328125	3.12526901354605e-10\\
16.679609375	3.54160029031562e-10\\
16.700107421875	2.53578793051978e-10\\
16.72060546875	1.60594757256058e-10\\
16.741103515625	1.62578620684542e-10\\
16.7616015625	-2.15722652048334e-11\\
16.782099609375	1.45435339200273e-10\\
16.80259765625	6.68697525776992e-11\\
16.823095703125	2.18448580636544e-10\\
16.84359375	2.35190834424003e-10\\
16.864091796875	1.62308359404299e-10\\
16.88458984375	3.38054105270541e-10\\
16.905087890625	3.20246060021533e-10\\
16.9255859375	2.24670364089181e-10\\
16.946083984375	1.90319075503767e-10\\
16.96658203125	-7.08180288405604e-11\\
16.987080078125	1.3658665739346e-10\\
17.007578125	3.76735696875938e-11\\
17.028076171875	5.41790442844972e-11\\
17.04857421875	2.46602126714997e-10\\
17.069072265625	2.09887144547527e-10\\
17.0895703125	3.05211156860467e-10\\
17.110068359375	2.21596404682053e-10\\
17.13056640625	3.39472943809708e-10\\
17.151064453125	1.37390598487613e-10\\
17.1715625	2.91973660809071e-10\\
17.192060546875	1.77305918139665e-10\\
17.21255859375	1.81378942539063e-10\\
17.233056640625	4.88874801212445e-11\\
17.2535546875	1.50748174601956e-10\\
17.274052734375	9.23061504852416e-11\\
17.29455078125	1.13299383531999e-10\\
17.315048828125	1.72913632602517e-10\\
17.335546875	2.40237809767725e-10\\
17.356044921875	2.12956771680085e-10\\
17.37654296875	3.23509904411306e-10\\
17.397041015625	8.96458923960107e-11\\
17.4175390625	2.02531000545626e-10\\
17.438037109375	-3.48545931742346e-11\\
17.45853515625	3.24780284138665e-10\\
17.479033203125	1.66617423500263e-11\\
17.49953125	2.40934936417741e-10\\
17.520029296875	2.01721684885601e-11\\
17.54052734375	1.93494834519057e-10\\
17.561025390625	2.21452054130802e-10\\
17.5815234375	1.6469357784707e-10\\
17.602021484375	1.22382135392837e-10\\
17.62251953125	1.42582447170451e-10\\
17.643017578125	5.26123502627935e-11\\
17.663515625	-8.36066743912191e-11\\
17.684013671875	-2.90064804521495e-11\\
17.70451171875	-5.69157898811023e-11\\
17.725009765625	-9.61905690123328e-11\\
17.7455078125	4.49936946735475e-11\\
17.766005859375	-9.98416332708416e-11\\
17.78650390625	3.04060509064636e-11\\
17.807001953125	6.93576694878922e-12\\
17.8275	-5.58688218011077e-12\\
17.847998046875	-1.0050430596587e-10\\
17.86849609375	-6.95721760356687e-11\\
17.888994140625	-1.22039851573327e-10\\
17.9094921875	-1.91120392600435e-10\\
17.929990234375	-2.25187151034808e-10\\
17.95048828125	-1.84300109720588e-10\\
17.970986328125	-1.51447244780326e-10\\
17.991484375	-4.22918552675726e-11\\
18.011982421875	-1.21544061434025e-10\\
18.03248046875	-4.52388114966252e-11\\
18.052978515625	-7.2168994435822e-11\\
18.0734765625	-8.43822174987906e-11\\
18.093974609375	-1.64888007187545e-10\\
18.11447265625	-1.10330481143813e-10\\
18.134970703125	-1.53367543860318e-10\\
18.15546875	-1.76721872284882e-10\\
18.175966796875	3.54480192017825e-11\\
18.19646484375	-1.15903627884587e-10\\
18.216962890625	1.38720335209179e-10\\
18.2374609375	2.69711918740233e-11\\
18.257958984375	1.11378867821217e-10\\
18.27845703125	-6.75649496643789e-11\\
18.298955078125	-5.2729556850635e-11\\
18.319453125	-1.04865837427067e-10\\
18.339951171875	-8.88283253323298e-11\\
18.36044921875	-2.09110806202387e-10\\
18.380947265625	-2.45391992620517e-11\\
18.4014453125	-1.30771649508463e-10\\
18.421943359375	5.8695613977463e-11\\
18.44244140625	3.27595477346518e-11\\
18.462939453125	4.86650441565336e-11\\
18.4834375	6.51398176688085e-11\\
18.503935546875	-4.99242475903443e-11\\
18.52443359375	-1.78396962921756e-10\\
18.544931640625	-1.68290713826646e-10\\
18.5654296875	-4.42844100759066e-10\\
18.585927734375	-2.75179185200718e-10\\
18.60642578125	-4.4258516000935e-10\\
18.626923828125	-3.37978865334773e-10\\
18.647421875	-3.35698407363472e-10\\
18.667919921875	-3.24443888389673e-10\\
18.68841796875	-2.55394912642731e-10\\
18.708916015625	-1.51788145065574e-10\\
18.7294140625	-2.84641451565522e-10\\
18.749912109375	-2.18764845555269e-10\\
18.77041015625	-2.77802125527049e-10\\
18.790908203125	-2.79772678755682e-10\\
18.81140625	-3.43693467327812e-10\\
18.831904296875	-4.41971889450978e-10\\
18.85240234375	-2.37502781051042e-10\\
18.872900390625	-5.04970690891291e-10\\
18.8933984375	-2.89263418406408e-10\\
18.913896484375	-5.81541566318417e-10\\
18.93439453125	-3.65683407157514e-10\\
18.954892578125	-6.53486199521588e-10\\
18.975390625	-5.69043731116779e-10\\
18.995888671875	-6.38763406972941e-10\\
19.01638671875	-5.35793382032174e-10\\
19.036884765625	-7.282621323789e-10\\
19.0573828125	-5.64537411524708e-10\\
19.077880859375	-6.8540201264858e-10\\
19.09837890625	-5.36248243076275e-10\\
19.118876953125	-6.18490249663541e-10\\
19.139375	-5.66075097540608e-10\\
19.159873046875	-5.71275523917154e-10\\
19.18037109375	-7.36794840530714e-10\\
19.200869140625	-7.86251964011595e-10\\
19.2213671875	-7.14155547074614e-10\\
19.241865234375	-8.07767549264555e-10\\
19.26236328125	-6.31550366073805e-10\\
19.282861328125	-7.35053412642963e-10\\
19.303359375	-5.33299431319852e-10\\
19.323857421875	-5.50445554960576e-10\\
19.34435546875	-6.22459234949934e-10\\
19.364853515625	-5.51728866323225e-10\\
19.3853515625	-5.47906532176527e-10\\
19.405849609375	-5.99279952234572e-10\\
19.42634765625	-6.03515422949297e-10\\
19.446845703125	-6.6435989171358e-10\\
19.46734375	-5.25363514462893e-10\\
19.487841796875	-6.91731608965259e-10\\
19.50833984375	-6.68718807622578e-10\\
19.528837890625	-6.12236751846267e-10\\
19.5493359375	-7.26423909376442e-10\\
19.569833984375	-6.75833017899098e-10\\
19.59033203125	-7.91662772328922e-10\\
19.610830078125	-5.90257706730057e-10\\
19.631328125	-7.30115139899494e-10\\
19.651826171875	-6.80603463038375e-10\\
19.67232421875	-6.49571549477911e-10\\
19.692822265625	-6.59076320272072e-10\\
19.7133203125	-8.42161645189778e-10\\
19.733818359375	-7.62219090295988e-10\\
19.75431640625	-9.89344376083591e-10\\
19.774814453125	-9.46270107076756e-10\\
19.7953125	-9.71846241310165e-10\\
19.815810546875	-8.45641241122448e-10\\
19.83630859375	-8.42816416135913e-10\\
19.856806640625	-6.13557735415297e-10\\
19.8773046875	-6.74904929295616e-10\\
19.897802734375	-4.56061302231422e-10\\
19.91830078125	-7.4392699135397e-10\\
19.938798828125	-6.17363179171471e-10\\
19.959296875	-7.94387908201143e-10\\
19.979794921875	-6.98996399389456e-10\\
20.00029296875	-8.72494853634863e-10\\
20.020791015625	-7.69718108995257e-10\\
20.0412890625	-7.96019292395519e-10\\
20.061787109375	-6.33785938381106e-10\\
20.08228515625	-5.29905148560071e-10\\
20.102783203125	-3.5332186987829e-10\\
20.12328125	-3.06329127814968e-10\\
20.143779296875	-1.99266336854594e-10\\
20.16427734375	-2.58983699592426e-10\\
20.184775390625	-2.84056920928149e-10\\
20.2052734375	-4.2125009365509e-10\\
20.225771484375	-3.66637751161525e-10\\
20.24626953125	-4.98796992153115e-10\\
20.266767578125	-3.2647586575273e-10\\
20.287265625	-5.39306387260173e-10\\
20.307763671875	-2.91573455544631e-10\\
20.32826171875	-2.99795306292815e-10\\
20.348759765625	-2.49721698245154e-10\\
20.3692578125	-3.08986002553379e-10\\
20.389755859375	-3.53426260213182e-10\\
20.41025390625	-4.52566357918842e-10\\
20.430751953125	-4.43489550710287e-10\\
20.45125	-5.89483036139839e-10\\
20.471748046875	-5.03061920796456e-10\\
20.49224609375	-3.63489417963869e-10\\
20.512744140625	-3.41972953207255e-10\\
20.5332421875	-2.47779994307119e-10\\
20.553740234375	-2.11014459830675e-10\\
20.57423828125	-2.30094084101353e-10\\
20.594736328125	-2.20627648939245e-10\\
20.615234375	-3.90250210456439e-10\\
20.635732421875	-4.32876807282879e-10\\
20.65623046875	-3.27270684414468e-10\\
20.676728515625	-5.12471340544369e-10\\
20.6972265625	-2.41253906362406e-10\\
20.717724609375	-4.37992925071553e-10\\
20.73822265625	-1.8693744547107e-10\\
20.758720703125	-3.34289297156363e-10\\
20.77921875	-8.91071650047149e-11\\
20.799716796875	-2.55114048228649e-10\\
20.82021484375	-7.21582210891823e-11\\
20.840712890625	-2.52627695795608e-10\\
20.8612109375	-2.3805192676437e-10\\
20.881708984375	-3.80531127249569e-10\\
20.90220703125	-2.71796578746042e-10\\
20.922705078125	-4.83209273122952e-10\\
20.943203125	-3.17082662423129e-10\\
20.963701171875	-3.8374360144607e-10\\
20.98419921875	-3.66340335978827e-10\\
21.004697265625	-3.54573614902675e-10\\
21.0251953125	-2.0123814682228e-10\\
21.045693359375	-1.84323444705981e-10\\
21.06619140625	-5.52643049961058e-11\\
21.086689453125	-1.61454033900982e-10\\
21.1071875	-8.4372036844102e-11\\
21.127685546875	-2.39505773439456e-10\\
21.14818359375	-1.49693353229807e-10\\
21.168681640625	-1.73698506630494e-10\\
21.1891796875	-6.68596266066393e-11\\
21.209677734375	-1.14722309053915e-10\\
21.23017578125	7.79762318621532e-12\\
21.250673828125	-2.14109726345004e-11\\
21.271171875	1.05032984596783e-10\\
21.291669921875	-3.17012129397377e-11\\
21.31216796875	5.60363188795888e-11\\
21.332666015625	3.77622192037033e-11\\
21.3531640625	-1.67804112004556e-10\\
21.373662109375	1.28073428309559e-11\\
21.39416015625	-1.34828418066901e-10\\
21.414658203125	8.54941238627627e-11\\
21.43515625	-4.06548453862277e-11\\
21.455654296875	2.10142219665545e-10\\
21.47615234375	1.63046593380796e-10\\
21.496650390625	3.16878214751422e-10\\
21.5171484375	2.49606597183901e-10\\
21.537646484375	5.72340134633373e-10\\
21.55814453125	2.4568273586133e-10\\
21.578642578125	5.23739844921714e-10\\
21.599140625	4.022462308224e-10\\
21.619638671875	5.09797815881551e-10\\
21.64013671875	5.31212921364895e-10\\
21.660634765625	7.59914325047087e-10\\
21.6811328125	8.26573420347283e-10\\
21.701630859375	9.56153688069551e-10\\
21.72212890625	8.95468430352227e-10\\
21.742626953125	9.87960136993635e-10\\
21.763125	9.08064196334997e-10\\
21.783623046875	8.66137573383318e-10\\
21.80412109375	7.21855501873442e-10\\
21.824619140625	5.43843833415959e-10\\
21.8451171875	5.05945043847289e-10\\
21.865615234375	5.22512681164398e-10\\
21.88611328125	5.26617837732023e-10\\
21.906611328125	6.09959971429245e-10\\
21.927109375	6.39825728283649e-10\\
21.947607421875	7.06996647357154e-10\\
21.96810546875	6.98278285637087e-10\\
21.988603515625	6.765969659739e-10\\
22.0091015625	6.98646320056692e-10\\
22.029599609375	6.61023795952524e-10\\
22.05009765625	7.52065399457669e-10\\
22.070595703125	5.86581088357763e-10\\
22.09109375	6.2990290057656e-10\\
22.111591796875	5.44265358981111e-10\\
22.13208984375	6.34263305562656e-10\\
22.152587890625	5.4522789867752e-10\\
22.1730859375	6.1228493819875e-10\\
22.193583984375	4.04717509552218e-10\\
22.21408203125	6.86206552917094e-10\\
22.234580078125	5.24639635255173e-10\\
22.255078125	7.19376183494094e-10\\
22.275576171875	5.91330106602818e-10\\
22.29607421875	8.41822135962362e-10\\
22.316572265625	7.45166897015703e-10\\
22.3370703125	7.28593785241849e-10\\
22.357568359375	6.09843462355978e-10\\
22.37806640625	5.59362345992014e-10\\
22.398564453125	5.06580834076723e-10\\
22.4190625	5.21966531650274e-10\\
22.439560546875	4.43430365864951e-10\\
22.46005859375	6.13484337381568e-10\\
22.480556640625	5.54978739591592e-10\\
22.5010546875	7.28001505062192e-10\\
22.521552734375	5.77699060007984e-10\\
22.54205078125	6.07941575296155e-10\\
22.562548828125	4.18454945742216e-10\\
22.583046875	4.28477267471728e-10\\
22.603544921875	5.14985906084912e-10\\
22.62404296875	4.17766340116979e-10\\
22.644541015625	5.10583492441787e-10\\
22.6650390625	5.0872566515798e-10\\
22.685537109375	4.98342031430091e-10\\
22.70603515625	4.31981963835697e-10\\
22.726533203125	4.02138476566367e-10\\
22.74703125	4.03097017789359e-10\\
22.767529296875	2.83099501327933e-10\\
22.78802734375	4.29038399234672e-10\\
22.808525390625	9.6277001086979e-11\\
22.8290234375	2.31093994554239e-10\\
22.849521484375	2.36686485382105e-10\\
22.87001953125	3.70620940782502e-10\\
22.890517578125	4.10180241531214e-10\\
22.911015625	5.0536355682005e-10\\
22.931513671875	3.69450472913712e-10\\
22.95201171875	4.06556029211907e-10\\
22.972509765625	2.4921937829402e-10\\
22.9930078125	2.8307345364422e-10\\
23.013505859375	2.31324863784659e-10\\
23.03400390625	1.55630507574637e-10\\
23.054501953125	2.24878321672714e-10\\
23.075	1.14249278089797e-10\\
23.095498046875	2.29583003733338e-10\\
23.11599609375	3.90514450337581e-10\\
23.136494140625	4.17280567886649e-10\\
23.1569921875	3.94673322971434e-10\\
23.177490234375	2.53338920962455e-10\\
23.19798828125	3.36313762607153e-11\\
23.218486328125	2.38658209381284e-10\\
23.238984375	-1.4331788057649e-10\\
23.259482421875	1.40594936464805e-10\\
23.27998046875	-1.97113098435325e-11\\
23.300478515625	1.2976133177646e-10\\
23.3209765625	3.89827132581402e-12\\
23.341474609375	1.71047982339703e-10\\
23.36197265625	2.25327886841852e-11\\
23.382470703125	2.12798852604459e-10\\
23.40296875	-9.46127421523108e-11\\
23.423466796875	1.47076360121647e-10\\
23.44396484375	2.56555595206733e-11\\
23.464462890625	1.63828219939665e-10\\
23.4849609375	2.30303149809816e-10\\
23.505458984375	2.7676553715768e-10\\
23.52595703125	2.7459683353591e-10\\
23.546455078125	3.77758111674871e-10\\
23.566953125	3.37303742528649e-10\\
23.587451171875	3.31262430177768e-10\\
23.60794921875	2.51596070556753e-10\\
23.628447265625	1.72676555317977e-10\\
23.6489453125	4.36394791918139e-11\\
23.669443359375	1.5790054389637e-10\\
23.68994140625	4.74331125531168e-12\\
23.710439453125	9.84580480326379e-11\\
23.7309375	5.73662126137104e-11\\
23.751435546875	1.10603070985466e-10\\
23.77193359375	3.28385628604666e-11\\
23.792431640625	1.22634743301761e-10\\
23.8129296875	4.29871177523813e-11\\
23.833427734375	1.29534632622051e-10\\
23.85392578125	1.74905897191843e-10\\
23.874423828125	6.16701908822457e-12\\
23.894921875	1.99244726371976e-10\\
23.915419921875	2.92045833491304e-11\\
23.93591796875	1.87469705984261e-10\\
23.956416015625	-1.33135602867434e-11\\
23.9769140625	6.46895422364243e-11\\
23.997412109375	-2.42438078309947e-10\\
24.01791015625	-2.18249783678762e-10\\
24.038408203125	-5.82470141782331e-10\\
24.05890625	-2.22139238515339e-10\\
24.079404296875	-4.88101679149231e-10\\
24.09990234375	-3.91154019218757e-10\\
24.120400390625	-3.28868824815956e-10\\
24.1408984375	-2.88200094495995e-10\\
24.161396484375	-4.85335972065871e-10\\
24.18189453125	-3.90278897304671e-10\\
24.202392578125	-5.62480997827166e-10\\
24.222890625	-5.31391324524352e-10\\
24.243388671875	-7.01559842013202e-10\\
24.26388671875	-6.42585749475949e-10\\
24.284384765625	-6.09397669095891e-10\\
24.3048828125	-6.58733334008532e-10\\
24.325380859375	-5.92883276870791e-10\\
24.34587890625	-5.47252630887131e-10\\
24.366376953125	-5.78380245830366e-10\\
24.386875	-6.00099106844465e-10\\
24.407373046875	-5.13146746917845e-10\\
24.42787109375	-4.99525958709214e-10\\
24.448369140625	-4.28414956651829e-10\\
24.4688671875	-4.21443257640516e-10\\
24.489365234375	-3.8985529197163e-10\\
24.50986328125	-3.96255386932744e-10\\
24.530361328125	-4.59090788267658e-10\\
24.550859375	-5.37910860378624e-10\\
24.571357421875	-3.83728131424435e-10\\
24.59185546875	-5.79429755552123e-10\\
24.612353515625	-5.01025958913821e-10\\
24.6328515625	-5.09027130481667e-10\\
24.653349609375	-4.64893901982166e-10\\
24.67384765625	-5.26499185022446e-10\\
24.694345703125	-4.07835654002436e-10\\
24.71484375	-4.10794983235793e-10\\
24.735341796875	-3.14192521238019e-10\\
24.75583984375	-4.34334892991408e-10\\
24.776337890625	-2.07282926624187e-10\\
24.7968359375	-3.99506375001248e-10\\
24.817333984375	-2.31212530485784e-10\\
24.83783203125	-3.34290596965861e-10\\
24.858330078125	-3.64978923178733e-10\\
24.878828125	-3.70176415783635e-10\\
24.899326171875	-3.2316056242577e-10\\
24.91982421875	-3.48937921003689e-10\\
24.940322265625	-3.26774213488282e-10\\
24.9608203125	-4.20106886921522e-10\\
24.981318359375	-3.32061292886481e-10\\
25.00181640625	-5.08856360536288e-10\\
25.022314453125	-3.27219286166944e-10\\
25.0428125	-4.38850028024209e-10\\
25.063310546875	-3.1062157706381e-10\\
25.08380859375	-2.91765623799469e-10\\
25.104306640625	-2.84354731163595e-10\\
25.1248046875	-1.53328840817091e-10\\
25.145302734375	-3.08392613586159e-10\\
25.16580078125	-3.14386981842619e-10\\
25.186298828125	-2.74446860177414e-10\\
25.206796875	-2.66677120054987e-10\\
25.227294921875	-3.16695980740026e-10\\
25.24779296875	-2.89266711379353e-10\\
25.268291015625	-2.42051888421291e-10\\
25.2887890625	-4.11718372806988e-10\\
25.309287109375	-2.03226552546652e-10\\
25.32978515625	-2.81108329113894e-10\\
25.350283203125	-1.3768144298521e-10\\
25.37078125	-2.39163554223942e-10\\
25.391279296875	-2.65217187482259e-10\\
25.41177734375	-3.15541575000909e-10\\
25.432275390625	-3.12030797182394e-10\\
25.4527734375	-3.74770585442153e-10\\
25.473271484375	-2.1012560614731e-10\\
25.49376953125	-3.80165983968758e-10\\
25.514267578125	-2.19376692250986e-10\\
25.534765625	-1.71124929925745e-10\\
25.555263671875	-1.49307652756758e-10\\
25.57576171875	-1.77155108987348e-10\\
25.596259765625	-2.0213846781684e-10\\
25.6167578125	-2.61713412520263e-10\\
25.637255859375	-2.18816804362796e-10\\
25.65775390625	-3.08791330153972e-10\\
25.678251953125	-1.57960958913256e-10\\
25.69875	-1.97585538316024e-11\\
25.719248046875	-1.28548691538603e-10\\
25.73974609375	2.42431751398775e-10\\
25.760244140625	1.18728659587543e-11\\
25.7807421875	1.32093872342616e-10\\
25.801240234375	-6.50995940686532e-11\\
25.82173828125	1.42723651161082e-10\\
25.842236328125	-5.07283896637689e-11\\
25.862734375	7.34166676094877e-11\\
25.883232421875	-9.19154952034644e-11\\
25.90373046875	1.95405009376621e-10\\
25.924228515625	5.44787468054305e-11\\
25.9447265625	2.24389127502925e-10\\
25.965224609375	1.79653168108612e-10\\
25.98572265625	2.44234426758181e-10\\
26.006220703125	2.03881419616572e-10\\
26.02671875	9.18470302079794e-11\\
26.047216796875	-2.84192066844754e-11\\
26.06771484375	-2.62828599549674e-11\\
26.088212890625	-1.24132142117405e-10\\
26.1087109375	-1.78830840322741e-10\\
26.129208984375	-2.22487650003867e-10\\
26.14970703125	-1.35677238300329e-10\\
26.170205078125	-1.38918318968919e-10\\
26.190703125	-7.49355637927679e-11\\
26.211201171875	-7.33961996817704e-11\\
26.23169921875	-1.67606639324763e-11\\
26.252197265625	3.99057823031327e-11\\
26.2726953125	1.56747047792129e-10\\
26.293193359375	2.31929448281075e-11\\
26.31369140625	2.07377344778168e-10\\
26.334189453125	-3.74469052939564e-11\\
26.3546875	3.32856885699906e-11\\
26.375185546875	-3.32431262305475e-11\\
26.39568359375	-2.06163424690003e-10\\
26.416181640625	6.81696374484487e-12\\
26.4366796875	-6.25928330684528e-11\\
26.457177734375	8.71705814036562e-11\\
26.47767578125	1.41486888088205e-10\\
26.498173828125	4.1041772736994e-10\\
26.518671875	3.27133384730172e-10\\
26.539169921875	6.16784766873367e-10\\
26.55966796875	4.2782677089809e-10\\
26.580166015625	5.93549737622659e-10\\
26.6006640625	4.6599692479834e-10\\
26.621162109375	4.33784733171275e-10\\
26.64166015625	3.82304617345079e-10\\
26.662158203125	4.23732411100152e-10\\
26.68265625	4.79081234490776e-10\\
26.703154296875	7.22655385170643e-10\\
26.72365234375	6.93116102188947e-10\\
26.744150390625	8.02722016812494e-10\\
26.7646484375	8.4168779824095e-10\\
26.785146484375	9.36115521972957e-10\\
26.80564453125	8.47203817927906e-10\\
26.826142578125	7.85998069906276e-10\\
26.846640625	7.13542762530489e-10\\
26.867138671875	6.49125445666993e-10\\
26.88763671875	7.30598921205907e-10\\
26.908134765625	5.76159146846794e-10\\
26.9286328125	6.28930618150007e-10\\
26.949130859375	5.6353870474408e-10\\
26.96962890625	6.18100551034753e-10\\
26.990126953125	5.44103544559904e-10\\
27.010625	6.35724664997006e-10\\
27.031123046875	6.90037764682327e-10\\
27.05162109375	6.13354540083378e-10\\
27.072119140625	5.48357425863482e-10\\
27.0926171875	6.20243571864988e-10\\
27.113115234375	5.21794203200005e-10\\
27.13361328125	5.50185944568044e-10\\
27.154111328125	4.59089178509767e-10\\
27.174609375	6.1660436651008e-10\\
27.195107421875	5.55518991651325e-10\\
27.21560546875	6.11188301371717e-10\\
27.236103515625	5.8061242234214e-10\\
27.2566015625	7.48498784580532e-10\\
27.277099609375	5.55707236467334e-10\\
27.29759765625	7.73617218742143e-10\\
27.318095703125	5.37207243533362e-10\\
27.33859375	6.13015974750008e-10\\
27.359091796875	4.10333877674425e-10\\
27.37958984375	4.47541142290918e-10\\
27.400087890625	3.78153874768547e-10\\
27.4205859375	3.43063929089697e-10\\
27.441083984375	4.43243997138192e-10\\
27.46158203125	4.74080780499239e-10\\
27.482080078125	4.59728551128917e-10\\
27.502578125	6.10196155727345e-10\\
27.523076171875	5.23171988660627e-10\\
27.54357421875	6.17875962158863e-10\\
27.564072265625	4.57128866513832e-10\\
27.5845703125	4.52228365859602e-10\\
27.605068359375	4.44384533937253e-10\\
27.62556640625	3.76018218344695e-10\\
27.646064453125	4.03256688688737e-10\\
27.6665625	4.41789310602089e-10\\
27.687060546875	3.25617405480597e-10\\
27.70755859375	4.45757740807183e-10\\
27.728056640625	4.02458603843773e-10\\
27.7485546875	2.84958005839881e-10\\
27.769052734375	3.46037018695617e-10\\
27.78955078125	2.89191830710231e-10\\
27.810048828125	2.46210687927352e-10\\
27.830546875	3.64383664040273e-10\\
27.851044921875	2.38419254389246e-10\\
27.87154296875	2.75065260972048e-10\\
27.892041015625	2.95135868887374e-10\\
27.9125390625	3.15174558088133e-10\\
27.933037109375	3.54081173667574e-10\\
27.95353515625	4.34359207110718e-10\\
27.974033203125	3.01839278613265e-10\\
27.99453125	4.41755745030795e-10\\
28.015029296875	2.78495916311548e-10\\
28.03552734375	2.96431410291719e-10\\
28.056025390625	3.58139284654064e-10\\
28.0765234375	2.9854374268434e-10\\
28.097021484375	2.46218259530626e-10\\
28.11751953125	2.16078246237436e-10\\
28.138017578125	1.62877083940384e-10\\
28.158515625	1.97981303090342e-10\\
28.179013671875	7.55230485288062e-11\\
28.19951171875	2.34612761968656e-11\\
28.220009765625	6.75232006932753e-11\\
28.2405078125	-2.34942309786185e-11\\
28.261005859375	2.90205726068655e-11\\
28.28150390625	1.18097397388795e-12\\
28.302001953125	6.63655425972239e-11\\
28.3225	-1.78228508716873e-11\\
28.342998046875	3.5217313291707e-11\\
28.36349609375	-6.92989255643752e-11\\
28.383994140625	2.01289075925026e-11\\
28.4044921875	-9.39146939495988e-11\\
28.424990234375	5.10863826483003e-11\\
28.44548828125	-7.84582718584257e-11\\
28.465986328125	-2.24762909830128e-12\\
28.486484375	-4.4556509708729e-11\\
28.506982421875	5.48579610986475e-11\\
28.52748046875	-1.57663162373144e-11\\
28.547978515625	2.15809324830065e-11\\
28.5684765625	-4.11969969069755e-11\\
28.588974609375	8.38342703171281e-11\\
28.60947265625	3.44337636100239e-11\\
28.629970703125	6.29025690484843e-11\\
28.65046875	3.57075094803941e-11\\
28.670966796875	2.19868954053786e-10\\
28.69146484375	1.21589639531688e-10\\
28.711962890625	1.6319991035742e-10\\
28.7324609375	1.49584409802147e-10\\
28.752958984375	5.66994463717957e-11\\
28.77345703125	3.37539964760721e-11\\
28.793955078125	1.15115206730454e-10\\
28.814453125	-4.27731112258366e-11\\
28.834951171875	1.55724949602929e-10\\
28.85544921875	7.61020674567502e-12\\
28.875947265625	6.80966099306e-11\\
28.8964453125	1.18789440480671e-10\\
28.916943359375	-1.13978573419039e-11\\
28.93744140625	2.72644543434916e-11\\
28.957939453125	-9.55069684537496e-11\\
28.9784375	-4.34687518601347e-11\\
28.998935546875	-2.16709571905945e-10\\
29.01943359375	-1.00082387038349e-10\\
29.039931640625	-2.6544463780297e-10\\
29.0604296875	-2.41592796428895e-10\\
29.080927734375	-4.27918282329105e-10\\
29.10142578125	-3.58196984125566e-10\\
29.121923828125	-3.39997599627338e-10\\
29.142421875	-5.38880428315065e-10\\
29.162919921875	-5.02733743144867e-10\\
29.18341796875	-5.0318319643342e-10\\
29.203916015625	-7.09171124243455e-10\\
29.2244140625	-5.62816490384138e-10\\
29.244912109375	-6.95326669559023e-10\\
29.26541015625	-6.02001790676231e-10\\
29.285908203125	-6.40088422179205e-10\\
29.30640625	-6.01094242019222e-10\\
29.326904296875	-6.48752273807981e-10\\
29.34740234375	-5.87685446518045e-10\\
29.367900390625	-4.67826614634035e-10\\
29.3883984375	-5.18641036261177e-10\\
29.408896484375	-4.97606660247197e-10\\
29.42939453125	-5.62208366948733e-10\\
29.449892578125	-4.50831127278171e-10\\
29.470390625	-5.57835569267453e-10\\
29.490888671875	-3.99021197197635e-10\\
29.51138671875	-4.45410074177046e-10\\
29.531884765625	-4.14227118666679e-10\\
29.5523828125	-4.37480789742352e-10\\
29.572880859375	-3.80804788324935e-10\\
29.59337890625	-4.31834666232907e-10\\
29.613876953125	-4.2351861932268e-10\\
29.634375	-4.35700953216121e-10\\
29.654873046875	-4.11515065175286e-10\\
29.67537109375	-4.00765822559101e-10\\
29.695869140625	-2.85520776814828e-10\\
29.7163671875	-3.50378639546429e-10\\
29.736865234375	-2.4334271848107e-10\\
29.75736328125	-3.44589629537624e-10\\
29.777861328125	-2.91617484758797e-10\\
29.798359375	-3.45010697337018e-10\\
29.818857421875	-2.47179469156737e-10\\
29.83935546875	-3.47265539965208e-10\\
29.859853515625	-1.73249798355864e-10\\
29.8803515625	-2.25971010794976e-10\\
29.900849609375	-1.99281223306573e-10\\
29.92134765625	-1.42416082213018e-10\\
29.941845703125	-1.19585186192439e-10\\
29.96234375	-4.94805480370404e-11\\
29.982841796875	-1.36884862769105e-10\\
30.00333984375	-1.82767408554433e-10\\
30.023837890625	-1.03112883030806e-10\\
30.0443359375	-2.57742095678571e-10\\
30.064833984375	-1.85161580030929e-10\\
30.08533203125	-2.72051868395126e-10\\
30.105830078125	-2.64527922311216e-10\\
30.126328125	-2.40147346066881e-10\\
30.146826171875	-2.42586025586696e-10\\
30.16732421875	-2.22518848743792e-10\\
30.187822265625	-1.21921963685551e-10\\
30.2083203125	-1.59080669444477e-10\\
30.228818359375	-8.46217846560165e-11\\
30.24931640625	-3.93549382757339e-11\\
30.269814453125	-1.12009412691699e-10\\
30.2903125	-3.09348793719503e-11\\
30.310810546875	-1.07822351326696e-10\\
30.33130859375	-1.66069986504409e-10\\
30.351806640625	-1.53646793867655e-10\\
30.3723046875	-1.54394416092008e-10\\
30.392802734375	-1.3130877410302e-10\\
30.41330078125	-1.52479652478292e-10\\
30.433798828125	-1.40076377770657e-10\\
30.454296875	-2.982739145821e-10\\
30.474794921875	-1.63208648648432e-10\\
30.49529296875	-2.62707836781675e-10\\
30.515791015625	-6.42771312612235e-11\\
30.5362890625	-1.61451161765488e-10\\
30.556787109375	-9.5233632529343e-11\\
30.57728515625	-5.29470346655394e-11\\
30.597783203125	3.82421456431067e-12\\
30.61828125	-1.06265209247558e-10\\
30.638779296875	5.92136482068935e-12\\
30.65927734375	-4.24433230199931e-11\\
30.679775390625	-2.47507153670762e-11\\
30.7002734375	9.53012252480302e-11\\
30.720771484375	5.17923477792729e-11\\
30.74126953125	2.21518297736077e-10\\
30.761767578125	9.75962077759815e-11\\
30.782265625	3.11589208701858e-10\\
30.802763671875	1.29012462974519e-10\\
30.82326171875	1.85341490390384e-10\\
30.843759765625	8.14488078481684e-11\\
30.8642578125	1.67346028325367e-10\\
30.884755859375	5.37373922831519e-11\\
30.90525390625	1.54796267897429e-10\\
30.925751953125	1.55469740738575e-10\\
30.94625	3.17648021922327e-10\\
30.966748046875	1.90379931713415e-10\\
30.98724609375	2.56077883398674e-10\\
31.007744140625	1.15014147201182e-10\\
31.0282421875	8.74594391239208e-11\\
31.048740234375	-2.26025176269314e-11\\
31.06923828125	1.42766350924752e-10\\
31.089736328125	6.26355260225331e-12\\
31.110234375	7.91192846065299e-11\\
31.130732421875	-4.30691731891635e-11\\
31.15123046875	-3.36800070269275e-11\\
31.171728515625	-1.15191744961598e-10\\
31.1922265625	2.74209579161823e-11\\
31.212724609375	-4.75506451841368e-11\\
31.23322265625	2.75816297333601e-11\\
31.253720703125	8.68308053790695e-12\\
31.27421875	8.51435243941174e-11\\
31.294716796875	6.14067407875521e-11\\
31.31521484375	9.26036638894266e-11\\
31.335712890625	6.34382921790393e-11\\
31.3562109375	1.09424328764148e-10\\
31.376708984375	2.1769844163799e-11\\
31.39720703125	1.44784251708922e-10\\
31.417705078125	1.23397196609741e-10\\
31.438203125	3.06105056103693e-11\\
31.458701171875	2.34951049586679e-10\\
31.47919921875	3.53229806753545e-11\\
31.499697265625	2.53718361083284e-10\\
31.5201953125	1.70228714630706e-10\\
31.540693359375	2.54611595890826e-10\\
31.56119140625	2.1590455597041e-10\\
31.581689453125	3.45142463354318e-10\\
31.6021875	2.72309425864021e-10\\
31.622685546875	4.62202556013389e-10\\
31.64318359375	5.3106047164328e-10\\
31.663681640625	5.7302426272243e-10\\
31.6841796875	5.75807992424904e-10\\
31.704677734375	5.96201868537907e-10\\
31.72517578125	5.15013441276729e-10\\
31.745673828125	6.89847141221027e-10\\
31.766171875	5.92377514412167e-10\\
31.786669921875	6.11510625595297e-10\\
31.80716796875	5.35706123914996e-10\\
31.827666015625	6.11096945168089e-10\\
31.8481640625	5.33858622131181e-10\\
31.868662109375	5.47676378819754e-10\\
31.88916015625	5.8558200660235e-10\\
31.909658203125	5.55641237924598e-10\\
31.93015625	6.10298845996029e-10\\
31.950654296875	4.42031936755586e-10\\
31.97115234375	5.64945583875764e-10\\
31.991650390625	4.68706956112017e-10\\
32.0121484375	4.73861788767281e-10\\
32.032646484375	4.12855074934825e-10\\
32.05314453125	3.73317135089282e-10\\
32.073642578125	3.97098873206206e-10\\
32.094140625	3.58724108931236e-10\\
32.114638671875	4.10247490132673e-10\\
32.13513671875	4.02333223284284e-10\\
32.155634765625	2.18232750412919e-10\\
32.1761328125	3.14409220459187e-10\\
32.196630859375	1.99727962892463e-10\\
32.21712890625	2.93192150354671e-10\\
32.237626953125	2.76265850577214e-10\\
32.258125	2.95880281517475e-10\\
32.278623046875	3.2308595996106e-10\\
32.29912109375	3.45717202091512e-10\\
32.319619140625	2.47462405564196e-10\\
32.3401171875	4.11792988155024e-10\\
32.360615234375	2.71093869457294e-10\\
32.38111328125	2.51088335116883e-10\\
32.401611328125	1.99018470266217e-10\\
32.422109375	1.18395364868549e-10\\
32.442607421875	1.67821643527461e-10\\
32.46310546875	9.10845030085825e-11\\
32.483603515625	1.75244027927519e-10\\
32.5041015625	1.38681344843986e-10\\
32.524599609375	2.5129644699134e-10\\
32.54509765625	2.47408963351904e-10\\
32.565595703125	2.73428421043083e-10\\
32.58609375	2.97389295055791e-10\\
32.606591796875	2.09151677933261e-10\\
32.62708984375	1.97863045775763e-10\\
32.647587890625	1.29337654594239e-10\\
32.6680859375	2.210904443439e-10\\
32.688583984375	1.00415329764467e-10\\
32.70908203125	2.38415544603744e-10\\
32.729580078125	1.92208559988478e-10\\
32.750078125	2.04292413871421e-10\\
32.770576171875	1.65032820823525e-10\\
32.79107421875	1.74267948436932e-10\\
32.811572265625	1.89390087744219e-10\\
32.8320703125	2.13992101624146e-10\\
32.852568359375	5.84753394089038e-11\\
32.87306640625	1.92031673364939e-10\\
32.893564453125	1.23371441603881e-10\\
32.9140625	1.9404397094593e-10\\
32.934560546875	1.90389856321203e-10\\
32.95505859375	2.31881019942429e-10\\
32.975556640625	1.37974561206182e-10\\
32.9960546875	1.71568277962398e-10\\
33.016552734375	3.3739952675882e-11\\
33.03705078125	1.44792304293392e-10\\
33.057548828125	-2.7095141867654e-11\\
33.078046875	-2.65352239315624e-11\\
33.098544921875	-1.81520411395224e-11\\
33.11904296875	7.92238587199407e-11\\
33.139541015625	2.21198562937465e-11\\
33.1600390625	8.96026652164778e-11\\
33.180537109375	9.3445419113095e-11\\
33.20103515625	6.27121889456822e-11\\
33.221533203125	7.83403150235383e-11\\
33.24203125	2.29777456740658e-12\\
33.262529296875	1.9768016100158e-11\\
33.28302734375	-5.7992407319566e-11\\
33.303525390625	-4.8280208659008e-11\\
33.3240234375	-7.3205397162418e-11\\
33.344521484375	2.73472893238518e-11\\
33.36501953125	-7.21258278607005e-11\\
33.385517578125	5.14636286984561e-11\\
33.406015625	-1.17379867451872e-10\\
33.426513671875	-7.13935339596384e-12\\
33.44701171875	-1.0335059085913e-10\\
33.467509765625	1.92452106659397e-11\\
33.4880078125	-4.95600231332081e-11\\
33.508505859375	5.73552994058749e-11\\
33.52900390625	-2.47561404381811e-11\\
33.549501953125	1.03584065250627e-10\\
33.57	6.72489563252631e-11\\
33.590498046875	8.21236454226368e-11\\
33.61099609375	7.75682424697736e-11\\
33.631494140625	1.01391958445049e-10\\
33.6519921875	3.56654065575956e-11\\
33.672490234375	7.79731000676856e-11\\
33.69298828125	9.92272847316872e-11\\
33.713486328125	1.41097188807241e-10\\
33.733984375	2.16673395326607e-10\\
33.754482421875	1.96989931829566e-10\\
33.77498046875	1.33862722066987e-10\\
33.795478515625	1.63398959776979e-10\\
33.8159765625	3.34627844875944e-11\\
33.836474609375	8.16560818227925e-11\\
33.85697265625	-9.72361287759683e-11\\
33.877470703125	2.09556925747016e-12\\
33.89796875	-3.93682985327977e-11\\
33.918466796875	1.64087478437603e-11\\
33.93896484375	5.82279062920178e-11\\
33.959462890625	2.5018665678567e-11\\
33.9799609375	1.49154992791619e-10\\
34.000458984375	-2.52911373362584e-12\\
34.02095703125	4.4918525331125e-11\\
34.041455078125	-1.91825231465246e-10\\
34.061953125	-1.19571296597105e-10\\
34.082451171875	-1.64432173485596e-10\\
34.10294921875	-2.23340534540366e-10\\
34.123447265625	-2.75987216743197e-10\\
34.1439453125	-1.37901206910193e-10\\
34.164443359375	-1.09946936646463e-10\\
34.18494140625	-5.75528718117355e-11\\
};
\addplot [color=mycolor1,solid]
  table[row sep=crcr]{%
34.18494140625	-5.75528718117355e-11\\
34.205439453125	-7.94595954501198e-11\\
34.2259375	-1.16114236170846e-10\\
34.246435546875	-2.43316791051307e-10\\
34.26693359375	-1.80749794443402e-10\\
34.287431640625	-2.568003064379e-10\\
34.3079296875	-2.46956856162327e-10\\
34.328427734375	-2.71194828169722e-10\\
34.34892578125	-1.9337291343089e-10\\
34.369423828125	-1.55661080873516e-10\\
34.389921875	-1.46820882690956e-10\\
34.410419921875	-1.4959670606709e-11\\
34.43091796875	-9.3290957246054e-11\\
34.451416015625	-7.45309055436923e-12\\
34.4719140625	-9.69797185258352e-11\\
34.492412109375	3.48463361085069e-12\\
34.51291015625	-1.28405848338154e-10\\
34.533408203125	-4.24913357112896e-11\\
34.55390625	-5.68871603505517e-11\\
34.574404296875	-9.29255879702043e-11\\
34.59490234375	5.4127782024235e-12\\
34.615400390625	-6.09250862523116e-12\\
34.6358984375	-1.23109406576595e-10\\
34.656396484375	-8.10408282055923e-11\\
34.67689453125	-4.17180549839348e-11\\
34.697392578125	3.86895598602346e-11\\
34.717890625	-6.59560112065399e-11\\
34.738388671875	8.51644907277833e-12\\
34.75888671875	3.37098109784111e-11\\
34.779384765625	8.71827218060544e-11\\
34.7998828125	-1.50366897523329e-11\\
34.820380859375	1.36769612248074e-10\\
34.84087890625	6.59625422160468e-11\\
34.861376953125	1.09808878744067e-10\\
34.881875	2.98024179391692e-12\\
34.902373046875	7.54448577285468e-11\\
34.92287109375	1.39519549526192e-11\\
34.943369140625	3.9916123348057e-11\\
34.9638671875	7.10238661445388e-11\\
34.984365234375	7.29544016367381e-11\\
35.00486328125	1.72749225190608e-10\\
35.025361328125	1.32197742802957e-10\\
35.045859375	2.68049456921444e-10\\
35.066357421875	2.64719541312855e-10\\
35.08685546875	2.44514951271649e-10\\
35.107353515625	2.71402969885012e-10\\
35.1278515625	2.25635718653754e-10\\
35.148349609375	3.111792676234e-10\\
35.16884765625	2.12113334152702e-10\\
35.189345703125	2.09997557460303e-10\\
35.20984375	1.4174080729881e-10\\
35.230341796875	1.903518925095e-10\\
35.25083984375	1.51312457528215e-10\\
35.271337890625	2.77601083918911e-10\\
35.2918359375	2.36968526769388e-10\\
35.312333984375	3.6915595684769e-10\\
35.33283203125	3.05701637946576e-10\\
35.353330078125	3.42222930995506e-10\\
35.373828125	2.75798778218699e-10\\
35.394326171875	2.95429419711242e-10\\
35.41482421875	2.51517879322627e-10\\
35.435322265625	2.85888114823244e-10\\
35.4558203125	2.07289277892886e-10\\
35.476318359375	2.30206988192524e-10\\
35.49681640625	2.5230187400934e-10\\
35.517314453125	2.88169531917897e-10\\
35.5378125	2.94137992243473e-10\\
35.558310546875	3.91245398203555e-10\\
35.57880859375	2.96761063489001e-10\\
35.599306640625	3.55944124657468e-10\\
35.6198046875	1.99237487087367e-10\\
35.640302734375	1.82190531456171e-10\\
35.66080078125	1.3494794321016e-10\\
35.681298828125	1.71516246555746e-10\\
35.701796875	1.66182822158228e-10\\
35.722294921875	1.75043255216663e-10\\
35.74279296875	2.75220727981922e-10\\
35.763291015625	3.10952592502406e-10\\
35.7837890625	3.67105774950702e-10\\
35.804287109375	3.71801794974358e-10\\
35.82478515625	4.50732034813216e-10\\
35.845283203125	3.02119705518229e-10\\
35.86578125	3.66085719867859e-10\\
35.886279296875	2.97204221872217e-10\\
35.90677734375	3.03231356290971e-10\\
35.927275390625	2.30291142949426e-10\\
35.9477734375	3.48874542228629e-10\\
35.968271484375	3.59287071872935e-10\\
35.98876953125	5.1275333174397e-10\\
36.009267578125	3.90324805880207e-10\\
36.029765625	5.42663092818397e-10\\
36.050263671875	4.51078636707967e-10\\
36.07076171875	4.04937523239895e-10\\
36.091259765625	3.61340892763098e-10\\
36.1117578125	3.02256433362687e-10\\
36.132255859375	2.48726474363359e-10\\
36.15275390625	3.0728630343768e-10\\
36.173251953125	2.11422380229301e-10\\
36.19375	2.9825752661552e-10\\
36.214248046875	2.55182831384374e-10\\
36.23474609375	3.44983509699293e-10\\
36.255244140625	3.14222213715138e-10\\
36.2757421875	3.59905598215179e-10\\
36.296240234375	3.51633584928762e-10\\
36.31673828125	4.56204360521753e-10\\
36.337236328125	3.07400667146735e-10\\
36.357734375	3.91751472857097e-10\\
36.378232421875	2.56496491165797e-10\\
36.39873046875	2.92967068531814e-10\\
36.419228515625	3.0456532282227e-10\\
36.4397265625	3.40185073217493e-10\\
36.460224609375	3.71825644891912e-10\\
36.48072265625	2.85395432295444e-10\\
36.501220703125	4.61747791698396e-10\\
36.52171875	3.7598687459387e-10\\
36.542216796875	4.87319889643525e-10\\
36.56271484375	4.24459782113819e-10\\
36.583212890625	4.07022871453142e-10\\
36.6037109375	2.621587358893e-10\\
36.624208984375	2.93486086752745e-10\\
36.64470703125	2.03043702185673e-10\\
36.665205078125	2.58861235786703e-10\\
36.685703125	3.32536846651329e-10\\
36.706201171875	3.49037250941555e-10\\
36.72669921875	3.87942926938378e-10\\
36.747197265625	4.59289103983153e-10\\
36.7676953125	4.61282979015765e-10\\
36.788193359375	5.10081834885508e-10\\
36.80869140625	4.18956851521151e-10\\
36.829189453125	4.29126282881815e-10\\
36.8496875	2.65322093187213e-10\\
36.870185546875	2.46116363808319e-10\\
36.89068359375	2.181848739282e-10\\
36.911181640625	2.51066179842794e-10\\
36.9316796875	3.6989116915771e-10\\
36.952177734375	3.07349261381518e-10\\
36.97267578125	4.1918452235743e-10\\
36.993173828125	2.93960653250167e-10\\
37.013671875	3.78761208993687e-10\\
37.034169921875	4.01896682283123e-10\\
37.05466796875	4.26686330814523e-10\\
37.075166015625	3.23528675630947e-10\\
37.0956640625	3.19582323848383e-10\\
37.116162109375	2.88378306121318e-10\\
37.13666015625	3.86159479802804e-10\\
37.157158203125	4.44851897474582e-10\\
37.17765625	4.3901282311953e-10\\
37.198154296875	3.70571903746166e-10\\
37.21865234375	4.0330684273163e-10\\
37.239150390625	2.84157553456814e-10\\
37.2596484375	2.42475257671921e-10\\
37.280146484375	2.22415563679216e-10\\
37.30064453125	2.39789534278568e-10\\
37.321142578125	1.84994253950997e-10\\
37.341640625	1.83795661017228e-10\\
37.362138671875	1.93222594433965e-10\\
37.38263671875	2.98475542643874e-10\\
37.403134765625	2.9763681049297e-10\\
37.4236328125	2.42921010522519e-10\\
37.444130859375	1.69363307855681e-10\\
37.46462890625	1.02589740026541e-10\\
37.485126953125	7.39352609800006e-11\\
37.505625	-2.69058341783165e-11\\
37.526123046875	1.64062831039552e-11\\
37.54662109375	-5.11483893903014e-11\\
37.567119140625	-1.8522511643244e-11\\
37.5876171875	7.78650095010815e-11\\
37.608115234375	-2.65868171878966e-12\\
37.62861328125	1.93416882548831e-10\\
37.649111328125	2.03880965629743e-11\\
37.669609375	1.15449938307525e-10\\
37.690107421875	7.2737581962869e-11\\
37.71060546875	1.781022196002e-11\\
37.731103515625	-3.3838485689956e-11\\
37.7516015625	-4.63066849208536e-11\\
37.772099609375	-5.35246944196631e-11\\
37.79259765625	-6.21334552945036e-11\\
37.813095703125	-1.19466587305717e-10\\
37.83359375	2.71425205868077e-11\\
37.854091796875	-8.01839055980398e-11\\
37.87458984375	-3.72485209787321e-11\\
37.895087890625	-1.10127249317621e-10\\
37.9155859375	-1.59021594132147e-11\\
37.936083984375	-3.94806338863406e-11\\
37.95658203125	2.11166944637044e-12\\
37.977080078125	-2.08587214429764e-11\\
37.997578125	-8.54399554851142e-11\\
38.018076171875	-5.50956425442025e-11\\
38.03857421875	-9.20896084691012e-11\\
38.059072265625	-1.65558293755356e-10\\
38.0795703125	-6.00464307781486e-11\\
38.100068359375	-2.07143752702539e-10\\
38.12056640625	-1.98363493765444e-10\\
38.141064453125	-1.73060923086141e-10\\
38.1615625	-1.55886981869252e-10\\
38.182060546875	-2.34744714629373e-10\\
38.20255859375	-1.52454184053102e-10\\
38.223056640625	-2.21241601158481e-10\\
38.2435546875	-2.25228868502054e-10\\
38.264052734375	-3.06189402734427e-10\\
38.28455078125	-3.26349649659669e-10\\
38.305048828125	-2.21768875123163e-10\\
38.325546875	-4.16921867389866e-10\\
38.346044921875	-2.02078712268163e-10\\
38.36654296875	-3.00413238426928e-10\\
38.387041015625	-2.67865418217704e-10\\
38.4075390625	-3.05104429528278e-10\\
38.428037109375	-2.21697099084461e-10\\
38.44853515625	-3.95519744018352e-10\\
38.469033203125	-3.27977387031333e-10\\
38.48953125	-5.21396747284785e-10\\
38.510029296875	-3.27186380436657e-10\\
38.53052734375	-4.78551890086371e-10\\
38.551025390625	-3.68825710024047e-10\\
38.5715234375	-3.16972320524783e-10\\
38.592021484375	-2.54364953782064e-10\\
38.61251953125	-3.00027433906954e-10\\
38.633017578125	-1.83884280615012e-10\\
38.653515625	-1.39556862489158e-10\\
38.674013671875	-1.2737031447026e-10\\
38.69451171875	-2.45587587831663e-10\\
38.715009765625	-1.68438022956535e-10\\
38.7355078125	-3.00993679807013e-10\\
38.756005859375	-1.64208602260338e-10\\
38.77650390625	-2.03823983317234e-10\\
38.797001953125	-1.58629262452794e-10\\
38.8175	-2.90753788575253e-10\\
38.837998046875	-2.19869230593302e-10\\
38.85849609375	-2.4941658828885e-10\\
38.878994140625	-1.97299878245399e-10\\
38.8994921875	-2.34075040298405e-10\\
38.919990234375	-2.31699152051691e-10\\
38.94048828125	-2.58331098970876e-10\\
38.960986328125	-2.64401139350374e-10\\
38.981484375	-2.66336246827307e-10\\
39.001982421875	-3.2364944546868e-10\\
39.02248046875	-2.46078773073999e-10\\
39.042978515625	-2.72340318790867e-10\\
39.0634765625	-2.49300456969992e-10\\
39.083974609375	-2.4899858846187e-10\\
39.10447265625	-2.00607327233854e-10\\
39.124970703125	-3.07335035690612e-10\\
39.14546875	-1.18305296362446e-10\\
39.165966796875	-2.48559420444406e-10\\
39.18646484375	-2.17015381798768e-10\\
39.206962890625	-2.64213667001958e-10\\
39.2274609375	-3.13195870010127e-10\\
39.247958984375	-3.08727087110463e-10\\
39.26845703125	-3.404416205022e-10\\
39.288955078125	-3.14794734046674e-10\\
39.309453125	-2.60895980321553e-10\\
39.329951171875	-2.41264240326456e-10\\
39.35044921875	-2.2135730648205e-10\\
39.370947265625	-2.10525177975199e-10\\
39.3914453125	-2.24490255455845e-10\\
39.411943359375	-2.02320305143254e-10\\
39.43244140625	-2.11122537076989e-10\\
39.452939453125	-1.95230768066167e-10\\
39.4734375	-2.09344016287526e-10\\
39.493935546875	-2.19090692929805e-10\\
39.51443359375	-3.09612948752083e-10\\
39.534931640625	-2.68051133650146e-10\\
39.5554296875	-2.93237674816043e-10\\
39.575927734375	-2.27253751150369e-10\\
39.59642578125	-2.74706422693794e-10\\
39.616923828125	-2.56416768237969e-10\\
39.637421875	-1.71108111116938e-10\\
39.657919921875	-2.38457599857638e-10\\
39.67841796875	-2.15315280149187e-10\\
39.698916015625	-1.22028706150574e-10\\
39.7194140625	-8.28457859164207e-11\\
39.739912109375	-1.20752192053312e-10\\
39.76041015625	-1.03219749218749e-10\\
39.780908203125	-1.04910753242875e-10\\
39.80140625	-1.33529778927626e-10\\
39.821904296875	-1.73985160701449e-10\\
39.84240234375	-2.68572498203055e-10\\
39.862900390625	-2.21457093054602e-10\\
39.8833984375	-2.70363966306792e-10\\
39.903896484375	-2.25563860010637e-10\\
39.92439453125	-1.09520592396739e-10\\
39.944892578125	-1.17399262194743e-10\\
39.965390625	-2.64018257481734e-12\\
39.985888671875	-3.23338500141077e-11\\
40.00638671875	4.02283113715792e-11\\
40.026884765625	-1.56027603829199e-11\\
40.0473828125	2.04348157926766e-11\\
40.067880859375	-1.27221500109594e-10\\
40.08837890625	-1.58854917664472e-10\\
40.108876953125	-1.08432697554265e-10\\
40.129375	-2.05321327583689e-10\\
40.149873046875	-5.8326888311176e-11\\
40.17037109375	-1.4311125414669e-10\\
40.190869140625	-6.1708718700139e-11\\
40.2113671875	-8.52815774073841e-12\\
40.231865234375	5.12785845311222e-11\\
40.25236328125	3.79454036311059e-11\\
40.272861328125	5.25986685899278e-11\\
40.293359375	-1.44605677981581e-10\\
40.313857421875	-2.27186001821796e-11\\
40.33435546875	-1.49661049031874e-10\\
40.354853515625	-8.86367943182108e-11\\
40.3753515625	-8.50945134390957e-13\\
40.395849609375	2.3076487545447e-11\\
40.41634765625	3.53006386342834e-11\\
40.436845703125	4.41352895759444e-11\\
40.45734375	6.31470989756102e-11\\
40.477841796875	8.27216105743207e-11\\
40.49833984375	5.21965150260106e-11\\
40.518837890625	9.90660951882899e-11\\
40.5393359375	3.69914763929722e-11\\
40.559833984375	5.65885469198262e-11\\
40.58033203125	3.2932353675376e-11\\
40.600830078125	1.43213202314541e-10\\
40.621328125	1.25320094453705e-10\\
40.641826171875	2.28118100928028e-10\\
40.66232421875	1.4928065791689e-10\\
40.682822265625	1.91672569426574e-10\\
40.7033203125	1.67934728682363e-10\\
40.723818359375	2.21086912309565e-10\\
40.74431640625	1.35133810356104e-10\\
40.764814453125	1.36076123729355e-10\\
40.7853125	2.09404425254113e-10\\
40.805810546875	1.46963106452812e-10\\
40.82630859375	2.64378264334795e-10\\
40.846806640625	1.37991770651998e-10\\
40.8673046875	2.15986678274683e-10\\
40.887802734375	1.48479063959434e-10\\
40.90830078125	2.49919558742893e-10\\
40.928798828125	1.29024217527343e-10\\
40.949296875	1.80880632663323e-10\\
40.969794921875	1.10878431579851e-10\\
40.99029296875	2.53833564587508e-10\\
41.010791015625	9.33859055833159e-11\\
41.0312890625	1.81486999667618e-10\\
41.051787109375	3.57493005756196e-11\\
41.07228515625	1.13052420813618e-10\\
41.092783203125	3.98502926420846e-11\\
41.11328125	1.19391017309399e-10\\
41.133779296875	1.58152830450607e-11\\
41.15427734375	2.24464026980912e-11\\
41.174775390625	-5.69257510855491e-12\\
41.1952734375	5.04859425705175e-11\\
41.215771484375	5.82330400675384e-11\\
41.23626953125	6.20388135950554e-11\\
41.256767578125	-1.07925732307795e-11\\
41.277265625	5.92231096722633e-11\\
41.297763671875	2.49582461060007e-11\\
41.31826171875	4.54725557054591e-11\\
41.338759765625	2.72023077753465e-11\\
41.3592578125	7.99055800925978e-11\\
41.379755859375	6.48181456838102e-11\\
41.40025390625	1.00050283326704e-10\\
41.420751953125	7.70234059497738e-11\\
41.44125	1.73332180111676e-10\\
41.461748046875	1.77110199264742e-10\\
41.48224609375	1.07259249528347e-10\\
41.502744140625	1.07322058261734e-10\\
41.5232421875	-3.44291869699968e-11\\
41.543740234375	3.99820352520186e-11\\
41.56423828125	-4.58637686133686e-11\\
41.584736328125	3.93314594235375e-11\\
41.605234375	-4.6104718727626e-11\\
41.625732421875	1.57095534583271e-10\\
41.64623046875	1.13415696991712e-10\\
41.666728515625	2.68940990158631e-10\\
41.6872265625	2.21103440375273e-10\\
41.707724609375	2.37802030268502e-10\\
41.72822265625	2.56094857072866e-10\\
41.748720703125	2.06639050340586e-10\\
41.76921875	1.92572292095281e-10\\
41.789716796875	9.46658317109378e-11\\
41.81021484375	8.36134384423502e-11\\
41.830712890625	1.07651889502289e-10\\
41.8512109375	1.91927852657393e-10\\
41.871708984375	1.75349594161567e-10\\
41.89220703125	1.82894890858852e-10\\
41.912705078125	2.03261583938018e-10\\
41.933203125	1.69756878752642e-10\\
41.953701171875	2.04242513306133e-10\\
41.97419921875	2.00750334971431e-10\\
41.994697265625	2.12273449233484e-10\\
42.0151953125	2.1361725578981e-10\\
42.035693359375	1.13094027929213e-10\\
42.05619140625	2.11723928600683e-10\\
42.076689453125	1.91409090949089e-10\\
42.0971875	2.4491794285293e-10\\
42.117685546875	2.07922381830586e-10\\
42.13818359375	1.2284194810306e-10\\
42.158681640625	1.51814602865909e-10\\
42.1791796875	1.69744975322956e-10\\
42.199677734375	1.55352943560315e-10\\
42.22017578125	7.09123665495687e-11\\
42.240673828125	2.0130163370592e-10\\
42.261171875	1.93533538312532e-10\\
42.281669921875	1.51580080820811e-10\\
42.30216796875	1.42988396680025e-10\\
42.322666015625	9.05219692425345e-11\\
42.3431640625	1.50627223965274e-10\\
42.363662109375	8.44029338446272e-11\\
42.38416015625	1.24770866973191e-10\\
42.404658203125	8.6153592290081e-11\\
42.42515625	1.07060898762442e-10\\
42.445654296875	-1.75886783295593e-11\\
42.46615234375	-1.88706052684791e-11\\
42.486650390625	-1.34057550730336e-11\\
42.5071484375	-1.54928102750331e-11\\
42.527646484375	-1.0027003606642e-11\\
42.54814453125	4.47040435657207e-11\\
42.568642578125	-2.82468715967454e-11\\
42.589140625	7.13624427380281e-11\\
42.609638671875	2.68541172129101e-11\\
42.63013671875	7.83310253000295e-11\\
42.650634765625	-1.11366172245217e-11\\
42.6711328125	7.28972701189251e-11\\
42.691630859375	-9.57169885398313e-11\\
42.71212890625	2.13706537143304e-11\\
42.732626953125	2.37268277676651e-13\\
42.753125	2.0682528315207e-11\\
42.773623046875	-2.36469643769145e-11\\
42.79412109375	-2.32282758492476e-11\\
42.814619140625	3.24367019840797e-11\\
42.8351171875	9.2766746091796e-11\\
42.855615234375	4.19179226301644e-11\\
42.87611328125	5.48091329069413e-11\\
42.896611328125	2.64872903102251e-11\\
42.917109375	-3.33543312371333e-11\\
42.937607421875	-4.11276750628748e-11\\
42.95810546875	-1.07984701479185e-10\\
42.978603515625	-1.87738079439491e-11\\
42.9991015625	-1.18230980884099e-10\\
43.019599609375	-1.12364042987377e-10\\
43.04009765625	-5.18022763284937e-11\\
43.060595703125	-9.1941850388296e-11\\
43.08109375	-1.48069070978554e-10\\
43.101591796875	-7.27197565912137e-11\\
43.12208984375	-6.01601916868838e-11\\
43.142587890625	-8.18764546785101e-11\\
43.1630859375	-1.62013360539372e-10\\
43.183583984375	-1.23946415925336e-10\\
43.20408203125	-1.78898615502626e-10\\
43.224580078125	-2.42794529309496e-10\\
43.245078125	-2.01407009888969e-10\\
43.265576171875	-2.10612956320305e-10\\
43.28607421875	-2.26786014293516e-10\\
43.306572265625	-2.2458535858278e-10\\
43.3270703125	-2.25364557624465e-10\\
43.347568359375	-1.50803715508732e-10\\
43.36806640625	-2.1192535939558e-10\\
43.388564453125	-1.01389829259009e-10\\
43.4090625	-1.81366746334468e-10\\
43.429560546875	-1.88804323586093e-10\\
43.45005859375	-2.43188607269233e-10\\
43.470556640625	-1.66090091917593e-10\\
43.4910546875	-2.97672410216324e-10\\
43.511552734375	-2.22420100799454e-10\\
43.53205078125	-2.8296167039731e-10\\
43.552548828125	-1.41655638563882e-10\\
43.573046875	-1.51911026910157e-10\\
43.593544921875	-9.92370343364719e-11\\
43.61404296875	-1.42639014354033e-10\\
43.634541015625	-3.66213202617136e-11\\
43.6550390625	-8.36725631543692e-11\\
43.675537109375	-8.23105999214769e-11\\
43.69603515625	-1.3498356231826e-10\\
43.716533203125	-1.55859209468832e-10\\
43.73703125	-1.8614871695906e-10\\
43.757529296875	-1.16206258367268e-10\\
43.77802734375	-1.98321637751626e-10\\
43.798525390625	-2.59210588633455e-11\\
43.8190234375	-1.00801712969644e-10\\
43.839521484375	-1.01593337357822e-10\\
43.86001953125	-1.15761272391411e-10\\
43.880517578125	-1.26060436682732e-10\\
43.901015625	-1.20589473832129e-10\\
43.921513671875	-1.63426460180193e-10\\
43.94201171875	-2.511180138719e-10\\
43.962509765625	-1.89270664395066e-10\\
43.9830078125	-1.01161707307371e-10\\
44.003505859375	-1.7365202095265e-10\\
44.02400390625	-1.0761455733353e-11\\
44.044501953125	-1.00700288184363e-10\\
44.065	1.07507053784045e-11\\
44.085498046875	-1.088124419816e-10\\
44.10599609375	-7.11602963086385e-11\\
44.126494140625	-2.39885063545035e-10\\
44.1469921875	-1.46418058507809e-10\\
44.167490234375	-2.27409983220743e-10\\
44.18798828125	-1.95977271892788e-10\\
44.208486328125	-1.75024255829458e-10\\
44.228984375	-2.21248990860916e-10\\
44.249482421875	-2.18078855306915e-10\\
44.26998046875	-1.77725235950268e-10\\
44.290478515625	-1.87437219120034e-10\\
44.3109765625	-1.44869149924616e-10\\
44.331474609375	-1.69139565224365e-10\\
44.35197265625	-1.77816222093239e-10\\
44.372470703125	-2.20829040438948e-10\\
44.39296875	-1.24317027362143e-10\\
44.413466796875	-1.59860546801227e-10\\
44.43396484375	-1.44082168138044e-10\\
44.454462890625	-1.92506453138776e-10\\
44.4749609375	-1.96725315320477e-10\\
44.495458984375	-2.27677223100733e-10\\
44.51595703125	-1.57462049104113e-10\\
44.536455078125	-2.12028025712068e-10\\
44.556953125	-2.21708116781584e-10\\
44.577451171875	-2.24475978500457e-10\\
44.59794921875	-2.03715442582787e-10\\
44.618447265625	-1.48761864852574e-10\\
44.6389453125	-1.3430380382003e-10\\
44.659443359375	-6.14118216023796e-11\\
44.67994140625	-5.52610063530331e-11\\
44.700439453125	-6.77230125695716e-11\\
44.7209375	-1.258474734523e-10\\
44.741435546875	-3.45525392409377e-11\\
44.76193359375	-1.45756199074149e-10\\
44.782431640625	-1.18288456709395e-10\\
44.8029296875	-1.49048190737578e-10\\
44.823427734375	1.93258604992159e-11\\
44.84392578125	-1.17312868762546e-10\\
44.864423828125	8.67200267530642e-12\\
44.884921875	-4.88585329305544e-11\\
44.905419921875	4.49822939402611e-11\\
44.92591796875	-4.04446742038492e-11\\
44.946416015625	3.9694959667268e-11\\
44.9669140625	-2.25757536193604e-11\\
44.987412109375	-6.81627951153753e-11\\
45.00791015625	6.50941517459086e-11\\
45.028408203125	1.6868831523054e-11\\
45.04890625	3.77113671742438e-11\\
45.069404296875	6.99710234650657e-11\\
45.08990234375	6.26673555768172e-11\\
45.110400390625	7.58840498455331e-11\\
45.1308984375	6.63412407758568e-11\\
45.151396484375	1.1799476139586e-10\\
45.17189453125	4.15046275973373e-11\\
45.192392578125	3.46384698292102e-11\\
45.212890625	3.24687123303396e-12\\
45.233388671875	7.62293691491699e-11\\
45.25388671875	-5.444391008877e-11\\
45.274384765625	5.59935256767759e-11\\
45.2948828125	-4.33506545941564e-11\\
45.315380859375	-2.62504360459473e-11\\
45.33587890625	-7.29474259008547e-11\\
45.356376953125	6.57405659788638e-12\\
45.376875	-1.67527932084991e-11\\
45.397373046875	4.41040765906563e-11\\
45.41787109375	3.75284827247029e-11\\
45.438369140625	5.99953689722792e-11\\
45.4588671875	1.44745880720734e-10\\
45.479365234375	1.09308919761792e-10\\
45.49986328125	1.19044783181477e-10\\
45.520361328125	1.0136921911058e-10\\
45.540859375	9.35637047205903e-11\\
45.561357421875	1.13609485476057e-10\\
45.58185546875	9.20671333541664e-11\\
45.602353515625	3.9192959764787e-11\\
45.6228515625	1.26074521760124e-10\\
45.643349609375	1.14373450094829e-10\\
45.66384765625	1.52887733498842e-10\\
45.684345703125	1.95044067561528e-10\\
45.70484375	2.12413622002681e-10\\
45.725341796875	2.20044629566342e-10\\
45.74583984375	2.40942630276033e-10\\
45.766337890625	2.07627765208865e-10\\
45.7868359375	2.24096696961749e-10\\
45.807333984375	1.56261194286495e-10\\
45.82783203125	1.62449763334991e-10\\
45.848330078125	8.82827841443662e-11\\
45.868828125	1.6053380637045e-10\\
45.889326171875	5.29755159521891e-11\\
45.90982421875	1.24591265703681e-10\\
45.930322265625	1.31709760395018e-10\\
45.9508203125	2.09031901351951e-10\\
45.971318359375	1.48685169455766e-10\\
45.99181640625	2.47315767509595e-10\\
46.012314453125	1.21599303792872e-10\\
46.0328125	2.4009803365074e-10\\
46.053310546875	5.18673342799185e-11\\
46.07380859375	1.53579077276792e-10\\
46.094306640625	5.47421435279796e-11\\
46.1148046875	8.06310819213412e-11\\
46.135302734375	5.41021831304074e-11\\
46.15580078125	1.18106433669836e-10\\
46.176298828125	7.1345878647265e-11\\
46.196796875	-2.73433897862147e-11\\
46.217294921875	6.03446560122073e-11\\
46.23779296875	8.13431787476145e-11\\
46.258291015625	1.15894295046302e-11\\
46.2787890625	7.39830137011755e-11\\
46.299287109375	-8.19612040790132e-12\\
46.31978515625	-6.44222655937907e-11\\
46.340283203125	3.54188232309531e-11\\
46.36078125	2.87853699932361e-11\\
46.381279296875	4.07455193112153e-11\\
46.40177734375	7.56426131728744e-11\\
46.422275390625	9.61625092290348e-11\\
46.4427734375	1.17492729300615e-10\\
46.463271484375	5.19521201257342e-11\\
46.48376953125	2.90819168736619e-11\\
46.504267578125	6.53500547395156e-11\\
46.524765625	2.79688140054667e-12\\
46.545263671875	5.91114296896531e-11\\
46.56576171875	-2.45126367944189e-11\\
46.586259765625	7.38813801641197e-11\\
46.6067578125	1.29748914796972e-11\\
46.627255859375	1.20166012099097e-10\\
46.64775390625	1.51764738896772e-11\\
46.668251953125	5.94972623196368e-11\\
46.68875	5.01698751172192e-11\\
46.709248046875	4.64791923651639e-11\\
46.72974609375	8.91786270752109e-11\\
46.750244140625	1.57648444594235e-11\\
46.7707421875	6.1848541036106e-11\\
46.791240234375	5.37372914789388e-11\\
46.81173828125	8.27822820284155e-11\\
46.832236328125	8.82868385787825e-11\\
46.852734375	5.40746197799336e-11\\
46.873232421875	9.73318146098264e-11\\
46.89373046875	7.33433441182103e-11\\
46.914228515625	5.82031382943679e-11\\
46.9347265625	1.28978638505283e-10\\
46.955224609375	1.13832446935874e-10\\
46.97572265625	1.58412787166458e-10\\
46.996220703125	1.52134782956638e-10\\
47.01671875	1.44058887450216e-10\\
47.037216796875	1.50610951321869e-10\\
47.05771484375	1.35527213974552e-10\\
47.078212890625	5.79351425567104e-11\\
47.0987109375	8.09440694103061e-11\\
47.119208984375	5.83370454036261e-11\\
47.13970703125	2.29703358855315e-11\\
47.160205078125	2.70806260061599e-11\\
47.180703125	-5.16211673336684e-12\\
47.201201171875	1.19269658175665e-11\\
47.22169921875	3.04689866675484e-11\\
47.242197265625	-1.14395106967426e-12\\
47.2626953125	1.72961508532174e-11\\
47.283193359375	1.53647087903169e-11\\
47.30369140625	1.58880144621969e-11\\
47.324189453125	-1.021868215252e-11\\
47.3446875	5.60913774187311e-11\\
47.365185546875	-6.74772137649656e-11\\
47.38568359375	9.70691889645131e-11\\
47.406181640625	-2.55959265142068e-11\\
47.4266796875	5.81775040346921e-11\\
47.447177734375	-1.62545086665897e-11\\
47.46767578125	3.89466679226188e-12\\
47.488173828125	1.31845746305911e-11\\
47.508671875	2.46392918132466e-11\\
47.529169921875	9.32628078848289e-12\\
47.54966796875	-1.48041802971843e-10\\
47.570166015625	-6.72579475285726e-11\\
47.5906640625	-1.30595322125554e-10\\
47.611162109375	-4.8723196987512e-11\\
47.63166015625	-3.49593220287587e-11\\
47.652158203125	-3.39325871606457e-11\\
47.67265625	-3.18917086736648e-11\\
47.693154296875	-1.06051484580709e-10\\
47.71365234375	-8.65006661361629e-11\\
47.734150390625	-1.32016114223435e-10\\
47.7546484375	-9.5899747259138e-11\\
47.775146484375	-8.97982813613813e-11\\
47.79564453125	-4.84255685080826e-11\\
47.816142578125	-4.34787239264622e-11\\
47.836640625	5.85479595750902e-12\\
47.857138671875	-1.11396771332115e-11\\
47.87763671875	1.87507153091103e-11\\
47.898134765625	-2.80877414847357e-11\\
47.9186328125	-3.83880148393069e-11\\
47.939130859375	-1.34878691975264e-10\\
47.95962890625	-7.45694364539609e-11\\
47.980126953125	-1.27668832122586e-10\\
48.000625	-1.84797779464671e-10\\
48.021123046875	-7.72135028937574e-11\\
48.04162109375	-1.09848186733697e-10\\
48.062119140625	-1.32837455288698e-10\\
48.0826171875	-1.12304370252109e-10\\
48.103115234375	-9.50253878656772e-11\\
48.12361328125	-9.15728989920973e-11\\
48.144111328125	-1.7561922765424e-10\\
48.164609375	-1.02386571436739e-10\\
48.185107421875	-1.80722639627642e-10\\
48.20560546875	-1.17480813810612e-10\\
48.226103515625	-1.27476975525618e-10\\
48.2466015625	-1.41194886885393e-10\\
48.267099609375	-1.99808583992973e-10\\
48.28759765625	-1.85023862153535e-10\\
48.308095703125	-1.51090929722363e-10\\
48.32859375	-1.65654641056142e-10\\
48.349091796875	-1.3087358661126e-10\\
48.36958984375	-1.5678209849357e-10\\
48.390087890625	-8.77425147814064e-11\\
48.4105859375	-1.11549765783423e-10\\
48.431083984375	-7.75707183977135e-11\\
48.45158203125	-1.09394418789872e-10\\
48.472080078125	-1.23231757230516e-10\\
48.492578125	-1.81315979166508e-10\\
48.513076171875	-7.75340562985404e-11\\
48.53357421875	-2.52948144989044e-10\\
48.554072265625	-4.51526314741281e-11\\
48.5745703125	-1.20633759450784e-10\\
48.595068359375	-5.1806272934429e-11\\
48.61556640625	-2.87611626499309e-11\\
48.636064453125	-5.52393577477823e-12\\
48.6565625	-5.35952481007516e-11\\
48.677060546875	2.2539697686976e-11\\
48.69755859375	-1.25142224587832e-12\\
48.718056640625	-1.87779156521568e-11\\
48.7385546875	4.6079640879458e-11\\
48.759052734375	-8.90884138560423e-11\\
48.77955078125	-7.51716758818728e-11\\
48.800048828125	-4.62021924086604e-11\\
48.820546875	-4.05137746403853e-11\\
48.841044921875	-2.11374320070119e-11\\
48.86154296875	2.64601061122954e-11\\
48.882041015625	-7.74631712164225e-11\\
48.9025390625	6.01617545349186e-11\\
48.923037109375	-2.70109611304176e-11\\
48.94353515625	-9.15342996511684e-11\\
48.964033203125	-7.64573618664986e-11\\
48.98453125	-2.27854114509722e-11\\
49.005029296875	-2.09500537072947e-11\\
49.02552734375	-3.12155411885293e-11\\
49.046025390625	-1.68373022862273e-10\\
49.0665234375	-4.55084932124417e-11\\
49.087021484375	-1.01422457848565e-10\\
49.10751953125	-3.2687693501027e-11\\
49.128017578125	-2.3131293654721e-11\\
49.148515625	-8.26878174637514e-12\\
49.169013671875	2.2579079020196e-11\\
49.18951171875	2.8572778650763e-11\\
49.210009765625	-3.14652188302493e-11\\
49.2305078125	-2.90241681027188e-11\\
49.251005859375	-1.11838098837176e-10\\
49.27150390625	-1.03982805382486e-10\\
49.292001953125	-9.47640934390056e-11\\
49.3125	-1.17056816320812e-10\\
49.332998046875	-7.72570687396058e-11\\
49.35349609375	-4.57157921355264e-11\\
49.373994140625	2.15865699243711e-11\\
49.3944921875	-2.16239735240783e-11\\
49.414990234375	-2.47831047806832e-11\\
49.43548828125	-3.82873319227194e-11\\
49.455986328125	-8.91322061010097e-11\\
49.476484375	-1.04037114266954e-10\\
49.496982421875	-5.05095394179729e-11\\
49.51748046875	-1.46827165152468e-10\\
49.537978515625	-9.30927455608045e-11\\
49.5584765625	-1.03696384706451e-10\\
49.578974609375	-3.84801431883138e-11\\
49.59947265625	-6.23919685800623e-11\\
49.619970703125	-3.86775729149781e-11\\
49.64046875	-6.68267905048368e-11\\
49.660966796875	-1.04471256512985e-11\\
49.68146484375	-9.13544671167737e-12\\
49.701962890625	-9.71022612007971e-11\\
49.7224609375	-7.80162659329991e-11\\
49.742958984375	-5.66190333503416e-11\\
49.76345703125	-1.25214182031499e-10\\
49.783955078125	-7.02276627902827e-11\\
49.804453125	-1.53116067559475e-10\\
49.824951171875	-9.42579395699554e-11\\
49.84544921875	-4.6157970616106e-11\\
49.865947265625	-2.33481399426793e-11\\
49.8864453125	-7.01894613540986e-11\\
49.906943359375	-2.06531911403914e-11\\
49.92744140625	-6.89454693301918e-11\\
49.947939453125	-8.39786072175351e-11\\
49.9684375	-1.1605530615677e-10\\
49.988935546875	-8.62218054489529e-11\\
50.00943359375	-9.91791788071079e-11\\
50.029931640625	-1.34129737254466e-10\\
50.0504296875	-1.26641918088989e-11\\
50.070927734375	-4.03022896844865e-11\\
50.09142578125	-1.04560563443436e-12\\
50.111923828125	1.93042865421022e-11\\
50.132421875	-2.43987437175172e-11\\
50.152919921875	-2.55509468505875e-11\\
50.17341796875	-1.77571969025208e-11\\
50.193916015625	7.69784447474548e-11\\
50.2144140625	-4.63412659642703e-11\\
50.234912109375	-5.0919614685414e-12\\
50.25541015625	-1.7079844589245e-11\\
50.275908203125	-2.98624349630535e-11\\
50.29640625	-8.47315369268963e-11\\
50.316904296875	-1.00012778359289e-10\\
50.33740234375	-5.38624931967526e-11\\
50.357900390625	-3.63313879214579e-11\\
50.3783984375	-4.56824369797839e-12\\
50.398896484375	-3.66091609046077e-11\\
50.41939453125	-4.63238222834574e-11\\
50.439892578125	1.79382924983552e-11\\
50.460390625	-2.40238274487812e-11\\
50.480888671875	4.56829504786647e-13\\
50.50138671875	-1.6153411652773e-12\\
50.521884765625	4.09493109753498e-11\\
50.5423828125	4.22244683393706e-11\\
50.562880859375	5.87760176372619e-11\\
50.58337890625	6.56625265541873e-11\\
50.603876953125	9.78542927354341e-11\\
50.624375	8.30843182660818e-11\\
50.644873046875	5.01320832740263e-11\\
50.66537109375	-3.81560884936703e-11\\
50.685869140625	-6.89687260511519e-11\\
50.7063671875	-1.16378305355978e-10\\
50.726865234375	-5.92599252934686e-11\\
50.74736328125	-1.01544899638181e-10\\
50.767861328125	-7.71093937099319e-11\\
50.788359375	1.11866002664004e-11\\
50.808857421875	1.87111533184297e-11\\
50.82935546875	9.61326148258298e-11\\
50.849853515625	5.60860133993163e-11\\
50.8703515625	4.08329818805701e-11\\
50.890849609375	-2.12624154734961e-11\\
50.91134765625	-9.22835705868867e-11\\
50.931845703125	-1.46914261217173e-10\\
50.95234375	-1.30889525733183e-10\\
50.972841796875	-7.76694330522683e-11\\
50.99333984375	-1.30897490753257e-10\\
51.013837890625	-1.24983162592784e-10\\
51.0343359375	3.4773362268232e-12\\
51.054833984375	-5.44708402289195e-11\\
51.07533203125	5.45585446101572e-11\\
51.095830078125	-7.2002542857968e-11\\
51.116328125	-2.89107062714039e-12\\
51.136826171875	-1.28099883059108e-10\\
51.15732421875	-1.01399641034695e-10\\
51.177822265625	-9.1861827376663e-11\\
51.1983203125	-1.24003281432888e-10\\
51.218818359375	-4.08600164358356e-11\\
51.23931640625	-1.21010577416057e-10\\
51.259814453125	-2.91850829175592e-11\\
51.2803125	-6.65365459639861e-11\\
51.300810546875	-1.09073259013443e-10\\
51.32130859375	-4.66193204546159e-11\\
51.341806640625	-1.25252772172737e-10\\
51.3623046875	-1.58049939299365e-10\\
51.382802734375	-1.65297053870873e-10\\
51.40330078125	-9.55966100411984e-11\\
51.423798828125	-3.33921857926554e-13\\
51.444296875	-1.56908892021777e-12\\
51.464794921875	-4.31284332725031e-11\\
51.48529296875	-1.25405903317124e-12\\
51.505791015625	-3.99578985666285e-11\\
51.5262890625	-1.11608279211336e-10\\
51.546787109375	2.91091583643607e-13\\
51.56728515625	-1.47699668976525e-10\\
51.587783203125	-3.00299255309512e-11\\
51.60828125	-1.94344615808158e-10\\
51.628779296875	-5.20251427805552e-11\\
51.64927734375	-1.02691358155704e-10\\
51.669775390625	-5.51404611813947e-11\\
51.6902734375	-9.54208173183757e-11\\
51.710771484375	-1.03128731560119e-10\\
51.73126953125	-1.1544209812675e-10\\
51.751767578125	-9.05216022616714e-11\\
51.772265625	-5.5940480831706e-11\\
51.792763671875	-1.05184477506129e-10\\
51.81326171875	-4.9221763218151e-11\\
51.833759765625	-8.4537559871252e-11\\
51.8542578125	-7.33299668533361e-11\\
51.874755859375	-6.94483531129779e-11\\
51.89525390625	-6.93842367008192e-11\\
51.915751953125	-1.25658845683316e-10\\
51.93625	-8.12256631035695e-11\\
51.956748046875	-6.99132239477047e-11\\
51.97724609375	-9.54361681384611e-11\\
51.997744140625	-7.95824899176667e-12\\
52.0182421875	-5.55222608923812e-11\\
52.038740234375	-9.64170435745147e-12\\
52.05923828125	-4.6639941674543e-11\\
52.079736328125	-6.41937534815664e-11\\
52.100234375	9.63799100146614e-12\\
52.120732421875	-2.35452677589479e-11\\
52.14123046875	-6.33606673794791e-12\\
52.161728515625	3.97910407053155e-11\\
52.1822265625	-3.19328262011878e-11\\
52.202724609375	-4.35023344614647e-11\\
52.22322265625	-6.19225879331052e-11\\
52.243720703125	-8.78641890398966e-11\\
52.26421875	-3.28620457844258e-11\\
52.284716796875	-1.42506414603969e-10\\
52.30521484375	7.4893536670579e-11\\
52.325712890625	-3.30797762460295e-11\\
52.3462109375	-3.73857738347679e-11\\
52.366708984375	-1.05615730843566e-10\\
52.38720703125	-6.7521350033477e-11\\
52.407705078125	-1.417688663358e-10\\
52.428203125	-9.88854241434473e-11\\
52.448701171875	-1.07340604528627e-10\\
52.46919921875	-1.03128250739052e-10\\
52.489697265625	-1.3681476075852e-10\\
52.5101953125	-1.0704377731184e-10\\
52.530693359375	-6.4962717161915e-11\\
52.55119140625	-1.2957989225118e-10\\
52.571689453125	-1.3678087233834e-10\\
52.5921875	-1.58414696831165e-10\\
52.612685546875	-1.28549503867356e-10\\
52.63318359375	-2.15139352067715e-10\\
52.653681640625	-1.16198194886789e-10\\
52.6741796875	-1.51966900587089e-10\\
52.694677734375	-1.95516033003481e-10\\
52.71517578125	-1.6887524174391e-10\\
52.735673828125	-1.8381210787103e-10\\
52.756171875	-1.89341434771534e-10\\
52.776669921875	-1.85992841212478e-10\\
52.79716796875	-1.86050428580414e-10\\
52.817666015625	-1.74882377544802e-10\\
52.8381640625	-1.65191865275694e-10\\
52.858662109375	-1.31088157114959e-10\\
52.87916015625	-1.5075532355009e-10\\
52.899658203125	-1.5670537152033e-10\\
52.92015625	-1.24950855973095e-10\\
52.940654296875	-1.48234658032606e-10\\
52.96115234375	-1.44854768298725e-10\\
52.981650390625	-1.63298586669355e-10\\
53.0021484375	-1.71671525861173e-10\\
53.022646484375	-1.87699512604301e-10\\
53.04314453125	-2.21442815226754e-10\\
53.063642578125	-1.74261603736224e-10\\
53.084140625	-1.91678261703803e-10\\
53.104638671875	-1.42334288079475e-10\\
53.12513671875	-7.97586362134813e-11\\
53.145634765625	-1.19674893114215e-10\\
53.1661328125	-6.6775179917473e-11\\
53.186630859375	-4.39243519388991e-11\\
53.20712890625	2.10908929986575e-12\\
53.227626953125	-1.37706574001105e-11\\
53.248125	1.1851710262221e-11\\
53.268623046875	-9.15824603141385e-11\\
53.28912109375	-6.66184299730606e-11\\
53.309619140625	-1.04236628334623e-10\\
53.3301171875	-1.1579488885447e-10\\
53.350615234375	-1.03531001780447e-10\\
53.37111328125	-7.52293209708208e-11\\
53.391611328125	-4.74949960525788e-11\\
53.412109375	-6.81757144577415e-11\\
53.432607421875	-2.30315506110429e-11\\
53.45310546875	1.44968318606594e-12\\
53.473603515625	-1.99584474088811e-11\\
53.4941015625	3.3729567649732e-11\\
53.514599609375	6.53652108661318e-12\\
53.53509765625	-8.34965989713188e-11\\
53.555595703125	-7.83888368737223e-11\\
53.57609375	-1.1553425540183e-10\\
53.596591796875	3.04474907751007e-11\\
53.61708984375	-1.81312262822456e-10\\
53.637587890625	-9.50614478837989e-11\\
53.6580859375	-1.83562650624492e-10\\
53.678583984375	-1.22539896912087e-10\\
53.69908203125	-9.4522753733092e-11\\
53.719580078125	-5.49438049455272e-11\\
53.740078125	-2.79011000547192e-11\\
53.760576171875	-6.58067770322339e-11\\
53.78107421875	-1.13492884370938e-11\\
53.801572265625	-5.45230344983984e-11\\
53.8220703125	-4.22597053114489e-11\\
53.842568359375	-2.81935694114078e-11\\
53.86306640625	-5.72528716624593e-11\\
53.883564453125	-1.11624764541627e-11\\
53.9040625	-1.23560412183424e-11\\
53.924560546875	-6.92318809180404e-12\\
53.94505859375	-2.45961361012635e-11\\
53.965556640625	-5.93416803642842e-11\\
53.9860546875	-1.18456228945269e-12\\
54.006552734375	-1.05590744291719e-11\\
54.02705078125	5.52299145031947e-11\\
54.047548828125	6.8092671395552e-12\\
54.068046875	7.15124108462135e-11\\
54.088544921875	-1.83086832724642e-11\\
54.10904296875	3.21728691754e-11\\
54.129541015625	-1.63953855170485e-11\\
54.1500390625	-8.08208229399595e-12\\
54.170537109375	-7.84233827549739e-11\\
54.19103515625	-2.09470354072112e-11\\
54.211533203125	-2.554007246414e-11\\
54.23203125	3.02537220079604e-11\\
54.252529296875	7.91096519075251e-11\\
54.27302734375	1.06559523877294e-10\\
54.293525390625	1.53578334368674e-10\\
54.3140234375	8.14256475056063e-11\\
54.334521484375	8.78946943594697e-11\\
54.35501953125	-1.70602549481118e-11\\
54.375517578125	-1.55668650590282e-11\\
54.396015625	-5.1707765956657e-11\\
54.416513671875	6.25500890070242e-11\\
54.43701171875	-4.65099852851707e-11\\
54.457509765625	2.50181299958153e-11\\
54.4780078125	6.48819370398152e-11\\
54.498505859375	5.20714726061857e-11\\
54.51900390625	7.69181678769116e-11\\
54.539501953125	8.4107008139838e-11\\
54.56	3.22173936525247e-11\\
54.580498046875	2.3913539086748e-11\\
54.60099609375	-3.72816788753425e-12\\
54.621494140625	-2.15451404540373e-12\\
54.6419921875	-6.69453525075142e-12\\
54.662490234375	7.72918775702171e-12\\
54.68298828125	2.08542917953015e-11\\
54.703486328125	4.11589504764856e-11\\
54.723984375	5.95708148561443e-11\\
54.744482421875	7.95053136302866e-11\\
54.76498046875	9.31779817535561e-11\\
54.785478515625	1.2833783987442e-10\\
54.8059765625	1.92291077481218e-11\\
54.826474609375	1.84582186060999e-12\\
54.84697265625	2.37169854875756e-12\\
54.867470703125	2.05523022946169e-11\\
54.88796875	6.81920321129087e-11\\
54.908466796875	1.2936369163445e-10\\
54.92896484375	1.13981432256627e-10\\
54.949462890625	1.65256190793771e-10\\
54.9699609375	1.19093887276582e-10\\
54.990458984375	1.36318858532083e-10\\
55.01095703125	1.58380505966555e-10\\
55.031455078125	1.42620342810809e-10\\
55.051953125	9.59630627102404e-11\\
55.072451171875	5.93302537454603e-11\\
55.09294921875	9.47855170345082e-11\\
55.113447265625	6.75943491491362e-11\\
55.1339453125	1.50385159448645e-10\\
55.154443359375	9.57483660501976e-11\\
55.17494140625	1.41274868870033e-10\\
55.195439453125	1.14166775158367e-10\\
55.2159375	1.65011602620577e-10\\
55.236435546875	1.63390151656176e-10\\
55.25693359375	1.4916835427779e-10\\
55.277431640625	9.16144316327495e-11\\
55.2979296875	1.66405042868376e-10\\
55.318427734375	1.42979589459619e-10\\
55.33892578125	1.49933923427886e-10\\
55.359423828125	1.13452047344039e-10\\
55.379921875	9.90673625761973e-11\\
55.400419921875	1.41247677424824e-10\\
55.42091796875	1.1055093854481e-10\\
55.441416015625	9.32638349955511e-11\\
55.4619140625	1.00134914295442e-10\\
55.482412109375	1.42002662529554e-10\\
55.50291015625	1.53497534643117e-10\\
55.523408203125	1.2900731750161e-10\\
55.54390625	1.46490555066269e-10\\
55.564404296875	1.31484422503769e-10\\
55.58490234375	1.73206031080916e-10\\
55.605400390625	1.40533813386824e-10\\
55.6258984375	9.28544763060149e-11\\
55.646396484375	8.96581839165342e-11\\
55.66689453125	-4.59728297184434e-12\\
55.687392578125	3.89985638300723e-11\\
55.707890625	2.32593050064768e-11\\
55.728388671875	4.87810933133795e-11\\
55.74888671875	7.51859853890841e-11\\
55.769384765625	1.10941956059781e-10\\
55.7898828125	3.10757395440189e-11\\
55.810380859375	8.2434776588252e-11\\
55.83087890625	5.57728635649445e-12\\
55.851376953125	-4.30796474682074e-13\\
55.871875	3.50523590484394e-11\\
55.892373046875	1.1310876178524e-11\\
55.91287109375	2.56069120752693e-11\\
55.933369140625	-1.17317737355858e-11\\
55.9538671875	-4.65903042664967e-12\\
55.974365234375	6.45881454277093e-12\\
55.99486328125	3.92136394907818e-11\\
56.015361328125	6.00688546751237e-11\\
56.035859375	3.22676322389657e-11\\
56.056357421875	-2.23146022190987e-11\\
56.07685546875	2.23774289698132e-11\\
56.097353515625	1.42327579190321e-12\\
56.1178515625	4.9983720578248e-11\\
56.138349609375	1.33712835073701e-11\\
56.15884765625	8.75826323247906e-11\\
56.179345703125	1.10661291088065e-11\\
56.19984375	9.30654838770237e-11\\
56.220341796875	2.05769861990184e-11\\
56.24083984375	-3.4542772426088e-11\\
56.261337890625	4.96501754735037e-12\\
56.2818359375	9.21344349548384e-11\\
56.302333984375	5.45649043465652e-11\\
56.32283203125	1.13290529858909e-11\\
56.343330078125	7.81615596136907e-11\\
56.363828125	8.93892166756978e-11\\
56.384326171875	4.40409699220769e-11\\
56.40482421875	6.98421902651562e-11\\
56.425322265625	7.40584546556902e-11\\
56.4458203125	3.41305281035997e-11\\
56.466318359375	1.06813525002646e-10\\
56.48681640625	1.91789882798185e-11\\
56.507314453125	1.26145719024356e-11\\
56.5278125	-6.91947060846123e-12\\
56.548310546875	3.6042341034142e-11\\
56.56880859375	-2.1463673490275e-11\\
56.589306640625	1.34687446220729e-10\\
56.6098046875	2.61214568220266e-11\\
56.630302734375	1.13989492600845e-10\\
56.65080078125	1.03663615247506e-10\\
56.671298828125	1.05666742942155e-10\\
56.691796875	7.24680811098307e-11\\
56.712294921875	1.8187558791739e-11\\
56.73279296875	1.00839053578453e-11\\
56.753291015625	-3.64482955983206e-11\\
56.7737890625	4.88851640892706e-11\\
56.794287109375	-8.09921711837282e-11\\
56.81478515625	7.1858423163015e-11\\
56.835283203125	2.79210256865845e-11\\
56.85578125	6.14594131865105e-11\\
56.876279296875	5.12621459213703e-11\\
56.89677734375	8.83830333225702e-11\\
56.917275390625	4.88093267477575e-11\\
56.9377734375	9.4456857430227e-11\\
};
\addlegendentry{$\text{train 3 -\textgreater{} Heimdal}$};

\end{axis}
\end{tikzpicture}%
%\end{document}
	\label{fig:train3}
\end{subfigure}
\qquad
\begin{subfigure}[t]{0.45\textwidth}
	\centering
	% This file was created by matlab2tikz.
%
%The latest updates can be retrieved from
%  http://www.mathworks.com/matlabcentral/fileexchange/22022-matlab2tikz-matlab2tikz
%where you can also make suggestions and rate matlab2tikz.
%
\definecolor{mycolor1}{rgb}{0.00000,0.44700,0.74100}%
%
\begin{tikzpicture}

\begin{axis}[%
width=\textwidth,
height=\textwidth,
at={(0\figurewidth,0\figureheight)},
scale only axis,
xmin=-60,
xmax=60,
ymin=-2e-09,
ymax=1.2e-08,
axis background/.style={fill=white},
% title style={font=\bfseries},
% title={Influencelines for train 4 , middle sensor},
legend style={legend cell align=left,align=left,draw=white!15!black}
]
\addplot [color=mycolor1,solid,forget plot]
  table[row sep=crcr]{%
-47.518515625	5.29155774508828e-11\\
-47.4972265625	9.70779845633673e-11\\
-47.4759375	1.13581701913621e-10\\
-47.4546484375	1.58662338131455e-10\\
-47.433359375	1.82773108164428e-10\\
-47.4120703125	2.04996027506188e-10\\
-47.39078125	2.06403589235636e-10\\
-47.3694921875	1.94801093121976e-10\\
-47.348203125	1.56818868225733e-10\\
-47.3269140625	1.53724250383501e-10\\
-47.305625	1.14902042195776e-10\\
-47.2843359375	7.12401650110145e-11\\
-47.263046875	1.53418709275908e-10\\
-47.2417578125	1.65282741910374e-10\\
-47.22046875	1.89705518989793e-10\\
-47.1991796875	1.80403345475759e-10\\
-47.177890625	1.3832015278965e-10\\
-47.1566015625	1.17298784505299e-10\\
-47.1353125	-3.02218326867449e-11\\
-47.1140234375	-5.54931052131998e-11\\
-47.092734375	-6.67682499138156e-11\\
-47.0714453125	-4.53783512082528e-11\\
-47.05015625	-2.68658544979426e-11\\
-47.0288671875	-1.9760068350685e-11\\
-47.007578125	6.54025682030477e-13\\
-46.9862890625	6.35967535233481e-11\\
-46.965	2.89641661565706e-11\\
-46.9437109375	2.09384361697574e-11\\
-46.922421875	1.6442616409302e-11\\
-46.9011328125	-9.77651032178706e-11\\
-46.87984375	-1.05207854118384e-10\\
-46.8585546875	-1.42245781688098e-10\\
-46.837265625	-1.0429130162108e-10\\
-46.8159765625	-1.9498190395136e-10\\
-46.7946875	-3.0566277785989e-11\\
-46.7733984375	-8.63706329489739e-11\\
-46.752109375	-5.42821501085703e-11\\
-46.7308203125	1.89866588951538e-11\\
-46.70953125	1.9359263985507e-11\\
-46.6882421875	-3.74372045995671e-11\\
-46.666953125	-1.1689665278136e-10\\
-46.6456640625	-1.13765217326284e-10\\
-46.624375	-1.78393072620391e-10\\
-46.6030859375	-2.07271219656156e-10\\
-46.581796875	-2.20017475798494e-10\\
-46.5605078125	-1.27392710591837e-10\\
-46.53921875	-1.20266312362535e-10\\
-46.5179296875	-1.08455013535852e-10\\
-46.496640625	-1.05981915828954e-10\\
-46.4753515625	-1.11612053497822e-10\\
-46.4540625	-2.19092036616169e-10\\
-46.4327734375	-2.69017573262009e-10\\
-46.411484375	-2.47408017021783e-10\\
-46.3901953125	-3.53109104140687e-10\\
-46.36890625	-3.7775073112801e-10\\
-46.3476171875	-2.90644979485823e-10\\
-46.326328125	-2.57971466115388e-10\\
-46.3050390625	-2.70969061779388e-10\\
-46.28375	-1.72211725787536e-10\\
-46.2624609375	-1.04230300297028e-10\\
-46.241171875	-1.31316691864035e-10\\
-46.2198828125	-1.26243044738091e-10\\
-46.19859375	-1.71369368086588e-10\\
-46.1773046875	-1.46020165686051e-10\\
-46.156015625	-2.46582393097134e-10\\
-46.1347265625	-2.92089034797039e-10\\
-46.1134375	-2.02270120123085e-10\\
-46.0921484375	-2.05161973891312e-10\\
-46.070859375	-2.08119864907182e-10\\
-46.0495703125	-1.91939238495336e-10\\
-46.02828125	-1.16371086634417e-10\\
-46.0069921875	-1.42142907371269e-10\\
-45.985703125	-1.13022814035716e-10\\
-45.9644140625	6.32203723491569e-12\\
-45.943125	-2.93581125588241e-11\\
-45.9218359375	3.440104999403e-11\\
-45.900546875	7.5009429243114e-12\\
-45.8792578125	-8.03974671520022e-12\\
-45.85796875	-3.62383926170932e-11\\
-45.8366796875	-2.59382991003335e-11\\
-45.815390625	2.80527524866306e-11\\
-45.7941015625	-1.6446961048521e-11\\
-45.7728125	6.69100592718681e-11\\
-45.7515234375	5.60899103382012e-11\\
-45.730234375	8.31511859011359e-11\\
-45.7089453125	9.10764522436921e-11\\
-45.68765625	1.19735536301525e-10\\
-45.6663671875	4.01823131430636e-11\\
-45.645078125	1.0367582264347e-10\\
-45.6237890625	-7.22376717662354e-12\\
-45.6025	4.7976238847949e-11\\
-45.5812109375	-2.12610395758464e-11\\
-45.559921875	2.51689848839379e-11\\
-45.5386328125	4.76874801335447e-11\\
-45.51734375	5.45450533132569e-11\\
-45.4960546875	7.97941312994581e-11\\
-45.474765625	1.07876846260321e-10\\
-45.4534765625	8.80272935011509e-11\\
-45.4321875	1.26803140390802e-10\\
-45.4108984375	1.02790408654201e-10\\
-45.389609375	1.17605072189174e-10\\
-45.3683203125	1.251705781308e-10\\
-45.34703125	1.80463999332642e-10\\
-45.3257421875	1.55573542611982e-10\\
-45.304453125	1.63863404343057e-10\\
-45.2831640625	2.1442155910099e-10\\
-45.261875	1.7131748170767e-10\\
-45.2405859375	1.72647190662725e-10\\
-45.219296875	1.73083996777155e-10\\
-45.1980078125	8.78190306287545e-11\\
-45.17671875	1.52706733222112e-10\\
-45.1554296875	1.22800064192166e-10\\
-45.134140625	9.55279103508487e-11\\
-45.1128515625	1.15195815329857e-10\\
-45.0915625	6.13119610291091e-11\\
-45.0702734375	8.21958238588454e-11\\
-45.048984375	6.33587059262676e-12\\
-45.0276953125	1.24953123466954e-10\\
-45.00640625	4.05355181481148e-12\\
-44.9851171875	-4.39125873416276e-12\\
-44.963828125	-6.91749138037992e-11\\
-44.9425390625	-6.77489754391465e-11\\
-44.92125	-1.26156796901288e-10\\
-44.8999609375	-1.41483140699027e-10\\
-44.878671875	-1.4465699707691e-10\\
-44.8573828125	-1.42197498333525e-10\\
-44.83609375	-6.22737352799471e-11\\
-44.8148046875	-5.61961832023139e-11\\
-44.793515625	4.30880216426179e-11\\
-44.7722265625	-2.24507872082199e-11\\
-44.7509375	2.66891457076969e-11\\
-44.7296484375	-1.19209724225299e-11\\
-44.708359375	-3.79250065346803e-11\\
-44.6870703125	-8.59852523339577e-11\\
-44.66578125	-1.52738438065989e-10\\
-44.6444921875	-1.25804279322416e-10\\
-44.623203125	-1.31772209625496e-10\\
-44.6019140625	-1.01342378829151e-10\\
-44.580625	-9.3488892305982e-11\\
-44.5593359375	-3.28707272342547e-11\\
-44.538046875	1.05117832261841e-11\\
-44.5167578125	4.13295010350331e-11\\
-44.49546875	2.92972251332303e-12\\
-44.4741796875	1.17336429160387e-10\\
-44.452890625	1.02321350242939e-10\\
-44.4316015625	6.01676311630318e-11\\
-44.4103125	1.07066283261968e-10\\
-44.3890234375	6.01151290859904e-11\\
-44.367734375	5.29115751702051e-12\\
-44.3464453125	5.71218746338359e-12\\
-44.32515625	5.26768932272041e-11\\
-44.3038671875	9.19081430823166e-11\\
-44.282578125	8.04476084226438e-11\\
-44.2612890625	1.89982085281926e-10\\
-44.24	1.55514522715833e-10\\
-44.2187109375	1.86161043184817e-10\\
-44.197421875	1.89671827141288e-10\\
-44.1761328125	1.38000618109889e-10\\
-44.15484375	1.43909168464065e-10\\
-44.1335546875	8.44475372659193e-11\\
-44.112265625	1.18648926143841e-10\\
-44.0909765625	1.32252921816056e-10\\
-44.0696875	1.69977144577158e-10\\
-44.0483984375	2.55845343136266e-10\\
-44.027109375	2.08700848114071e-10\\
-44.0058203125	1.45428057496812e-10\\
-43.98453125	1.63607413790602e-10\\
-43.9632421875	7.99655163104418e-11\\
-43.941953125	3.73382628965661e-11\\
-43.9206640625	-4.53336683577573e-11\\
-43.899375	-2.37754346693145e-11\\
-43.8780859375	-3.52832983330141e-11\\
-43.856796875	-6.67679233776391e-11\\
-43.8355078125	9.17523551492325e-11\\
-43.81421875	1.68117221722803e-10\\
-43.7929296875	8.36755447219899e-11\\
-43.771640625	1.20369170423633e-10\\
-43.7503515625	1.62001482680683e-10\\
-43.7290625	1.77058438408382e-10\\
-43.7077734375	9.64743743340239e-11\\
-43.686484375	1.12740134639916e-10\\
-43.6651953125	1.50101033417881e-10\\
-43.64390625	1.37280791625206e-10\\
-43.6226171875	1.20152844634869e-10\\
-43.601328125	1.21987932302451e-10\\
-43.5800390625	7.38264073147857e-11\\
-43.55875	2.63828752641845e-11\\
-43.5374609375	8.3948667546111e-11\\
-43.516171875	1.36224868950572e-10\\
-43.4948828125	1.61495422970864e-10\\
-43.47359375	2.45246903017477e-10\\
-43.4523046875	2.90165197650069e-10\\
-43.431015625	3.46528757543712e-10\\
-43.4097265625	2.98714683426647e-10\\
-43.3884375	3.03001470008212e-10\\
-43.3671484375	2.1117724080405e-10\\
-43.345859375	2.04121240804475e-10\\
-43.3245703125	1.33803508201607e-10\\
-43.30328125	1.34984283340879e-10\\
-43.2819921875	2.63794719690174e-10\\
-43.260703125	1.72188335429436e-10\\
-43.2394140625	2.47293834149195e-10\\
-43.218125	3.06132850266276e-10\\
-43.1968359375	3.45497030047438e-10\\
-43.175546875	3.71271079266726e-10\\
-43.1542578125	2.98707960459016e-10\\
-43.13296875	2.20711247529509e-10\\
-43.1116796875	2.47504534998121e-10\\
-43.090390625	1.44885773826049e-10\\
-43.0691015625	1.5307631307675e-10\\
-43.0478125	1.25839527870767e-10\\
-43.0265234375	2.10634753597317e-10\\
-43.005234375	2.12579678802195e-10\\
-42.9839453125	2.13235720799763e-10\\
-42.96265625	2.55795973142706e-10\\
-42.9413671875	1.92930408963978e-10\\
-42.920078125	1.7638062088508e-10\\
-42.8987890625	8.19956540684781e-11\\
-42.8775	1.03451701342103e-10\\
-42.8562109375	1.07819265653327e-10\\
-42.834921875	4.9770348201761e-11\\
-42.8136328125	1.54541408335225e-10\\
-42.79234375	8.36980510743645e-11\\
-42.7710546875	9.39205752602607e-11\\
-42.749765625	1.41122658685857e-10\\
-42.7284765625	1.4148935833924e-10\\
-42.7071875	4.10206213564273e-11\\
-42.6858984375	5.14649658263423e-12\\
-42.664609375	3.72219224680721e-11\\
-42.6433203125	8.04728818845983e-12\\
-42.62203125	-3.93475583352655e-11\\
-42.6007421875	-5.26432172374498e-11\\
-42.579453125	3.77222553318786e-11\\
-42.5581640625	-1.20094584349622e-11\\
-42.536875	-2.84210709438006e-11\\
-42.5155859375	3.21532131813685e-11\\
-42.494296875	1.85941358885913e-10\\
-42.4730078125	2.17487481132048e-11\\
-42.45171875	6.75550240683597e-11\\
-42.4304296875	3.11630704716366e-11\\
-42.409140625	1.763156863249e-11\\
-42.3878515625	1.52708394130503e-11\\
-42.3665625	9.00443085748699e-11\\
-42.3452734375	1.73280195717726e-11\\
-42.323984375	9.58871075313241e-11\\
-42.3026953125	1.94712388104035e-10\\
-42.28140625	2.11153831563148e-10\\
-42.2601171875	1.31162246382295e-10\\
-42.238828125	8.36871215410136e-11\\
-42.2175390625	1.69432498055827e-10\\
-42.19625	-4.66571063437227e-12\\
-42.1749609375	-3.52303510688327e-11\\
-42.153671875	-1.19334491534037e-12\\
-42.1323828125	2.54184313982693e-11\\
-42.11109375	-5.56885008376202e-12\\
-42.0898046875	8.76158801961017e-11\\
-42.068515625	1.238329235188e-10\\
-42.0472265625	4.98447140547283e-11\\
-42.0259375	8.71902861654118e-11\\
-42.0046484375	3.13797993996978e-11\\
-41.983359375	-4.1202040289824e-12\\
-41.9620703125	-3.73060117371297e-12\\
-41.94078125	-3.81681981041761e-11\\
-41.9194921875	-2.7805318817002e-12\\
-41.898203125	-7.27381741077695e-11\\
-41.8769140625	-3.619889470589e-11\\
-41.855625	-3.23469594220853e-12\\
-41.8343359375	3.31107914230206e-11\\
-41.813046875	2.04771213994468e-12\\
-41.7917578125	5.26350910454213e-11\\
-41.77046875	7.7771069559931e-11\\
-41.7491796875	-1.73389566747472e-11\\
-41.727890625	2.59970520420876e-11\\
-41.7066015625	-7.96238450512483e-11\\
-41.6853125	-8.81609172267471e-11\\
-41.6640234375	-1.40623133029898e-10\\
-41.642734375	-1.23682304709536e-10\\
-41.6214453125	-1.49175627277957e-10\\
-41.60015625	-1.45728720625465e-10\\
-41.5788671875	-1.01887117016844e-10\\
-41.557578125	-1.54840174981589e-10\\
-41.5362890625	-1.86429132609521e-10\\
-41.515	-2.14958817634404e-10\\
-41.4937109375	-2.40077733403892e-10\\
-41.472421875	-2.05404211207946e-10\\
-41.4511328125	-2.78759787231111e-10\\
-41.42984375	-2.90485319382362e-10\\
-41.4085546875	-2.73185143007808e-10\\
-41.387265625	-2.70243198669376e-10\\
-41.3659765625	-2.48793707389448e-10\\
-41.3446875	-2.96253374518431e-10\\
-41.3233984375	-3.05773718585205e-10\\
-41.302109375	-2.24464353742882e-10\\
-41.2808203125	-3.07474051993183e-10\\
-41.25953125	-3.39831065384177e-10\\
-41.2382421875	-3.78546634616406e-10\\
-41.216953125	-3.17695545226984e-10\\
-41.1956640625	-2.79231999560322e-10\\
-41.174375	-3.05470772189679e-10\\
-41.1530859375	-2.00282628284482e-10\\
-41.131796875	-2.59617146686949e-10\\
-41.1105078125	-2.95410740694527e-10\\
-41.08921875	-3.14595683381512e-10\\
-41.0679296875	-3.41248941068124e-10\\
-41.046640625	-3.8832115102357e-10\\
-41.0253515625	-2.9013281039394e-10\\
-41.0040625	-2.93116002540581e-10\\
-40.9827734375	-1.69996532111395e-10\\
-40.961484375	-1.93832081881828e-10\\
-40.9401953125	-1.99217527341992e-10\\
-40.91890625	-1.52872458026569e-10\\
-40.8976171875	-2.20303037881916e-10\\
-40.876328125	-2.42330089325172e-10\\
-40.8550390625	-2.50963847638187e-10\\
-40.83375	-2.95249249759825e-10\\
-40.8124609375	-2.66839733020107e-10\\
-40.791171875	-3.26700041572815e-10\\
-40.7698828125	-1.9846128879472e-10\\
-40.74859375	-1.72549161849469e-10\\
-40.7273046875	-1.31564100559484e-10\\
-40.706015625	-1.82530793752101e-10\\
-40.6847265625	-2.32887121368086e-10\\
-40.6634375	-2.30643158579722e-10\\
-40.6421484375	-2.50625332882063e-10\\
-40.620859375	-2.93549328120984e-10\\
-40.5995703125	-2.43842061959478e-10\\
-40.57828125	-2.68019884379895e-10\\
-40.5569921875	-2.34229832934205e-10\\
-40.535703125	-1.25705159672674e-10\\
-40.5144140625	-1.83092800594028e-10\\
-40.493125	-1.27326120813954e-10\\
-40.4718359375	-2.72462134399536e-10\\
-40.450546875	-1.85537808199139e-10\\
-40.4292578125	-2.6056932850537e-10\\
-40.40796875	-2.49865132449155e-10\\
-40.3866796875	-3.46853174696986e-10\\
-40.365390625	-3.04320557635124e-10\\
-40.3441015625	-2.47919011858e-10\\
-40.3228125	-2.14573270805504e-10\\
-40.3015234375	-2.2637665491751e-10\\
-40.280234375	-1.0669714787636e-10\\
-40.2589453125	-2.22832295734968e-10\\
-40.23765625	-1.80522396263466e-10\\
-40.2163671875	-1.40369321963672e-10\\
-40.195078125	-2.18631227137114e-10\\
-40.1737890625	-2.17487160610847e-10\\
-40.1525	-3.43963279546435e-10\\
-40.1312109375	-1.8864581208612e-10\\
-40.109921875	-1.39172511584901e-10\\
-40.0886328125	-1.25337199939983e-10\\
-40.06734375	1.2176146032424e-11\\
-40.0460546875	-6.80057638669151e-11\\
-40.024765625	9.74363038976618e-12\\
-40.0034765625	-5.3773147288463e-11\\
-39.9821875	5.31596104896161e-11\\
-39.9608984375	-1.38877642765201e-11\\
-39.939609375	-1.1126574620152e-10\\
-39.9183203125	1.1399421278309e-11\\
-39.89703125	-6.84364171827815e-11\\
-39.8757421875	-3.33679734082112e-11\\
-39.854453125	3.92946269744669e-11\\
-39.8331640625	7.72173515895099e-11\\
-39.811875	-1.82581167250013e-11\\
-39.7905859375	2.97368885061263e-11\\
-39.769296875	-2.56413053158267e-11\\
-39.7480078125	4.23926461253369e-12\\
-39.72671875	-1.36663784539983e-10\\
-39.7054296875	-7.94664648809193e-11\\
-39.684140625	-4.90135337216291e-12\\
-39.6628515625	-1.29637411456692e-10\\
-39.6415625	5.26754416126048e-11\\
-39.6202734375	1.93218832597504e-11\\
-39.598984375	5.43665546357042e-11\\
-39.5776953125	3.93907013142097e-11\\
-39.55640625	6.35205205633448e-12\\
-39.5351171875	-4.77674328646681e-12\\
-39.513828125	3.30017429362602e-11\\
-39.4925390625	-2.06196427450563e-11\\
-39.47125	-5.87190568318182e-11\\
-39.4499609375	7.56034970892773e-11\\
-39.428671875	1.52785811280755e-10\\
-39.4073828125	1.5956795275818e-10\\
-39.38609375	1.28791348837337e-10\\
-39.3648046875	1.73787582786056e-10\\
-39.343515625	8.47950090932683e-11\\
-39.3222265625	1.16621669889096e-10\\
-39.3009375	6.87179332367048e-11\\
-39.2796484375	1.0617138061245e-10\\
-39.258359375	1.18799884709846e-10\\
-39.2370703125	4.53585126844015e-11\\
-39.21578125	1.44450072442691e-10\\
-39.1944921875	1.97017651793238e-10\\
-39.173203125	2.33987515084444e-10\\
-39.1519140625	1.60789952285013e-10\\
-39.130625	1.95125306129184e-10\\
-39.1093359375	2.11436017432279e-10\\
-39.088046875	1.24662637285224e-10\\
-39.0667578125	2.72972301169494e-10\\
-39.04546875	2.56869203161857e-10\\
-39.0241796875	2.30801199778109e-10\\
-39.002890625	2.65278162775064e-10\\
-38.9816015625	2.75642060824307e-10\\
-38.9603125	2.38736093321844e-10\\
-38.9390234375	2.79477423558068e-10\\
-38.917734375	2.24105484786927e-10\\
-38.8964453125	3.87236157069175e-10\\
-38.87515625	3.82064761794789e-10\\
-38.8538671875	4.07519821975718e-10\\
-38.832578125	3.51435072335162e-10\\
-38.8112890625	4.10248668881976e-10\\
-38.79	4.54117138784731e-10\\
-38.7687109375	4.2690279229989e-10\\
-38.747421875	4.56537129936008e-10\\
-38.7261328125	5.42011795419592e-10\\
-38.70484375	4.48842385689104e-10\\
-38.6835546875	5.23957372899945e-10\\
-38.662265625	5.09192794551758e-10\\
-38.6409765625	5.38286944026483e-10\\
-38.6196875	4.69393976469014e-10\\
-38.5983984375	4.14185808729265e-10\\
-38.577109375	4.95444871287083e-10\\
-38.5558203125	3.56082952777156e-10\\
-38.53453125	4.52066322351278e-10\\
-38.5132421875	5.21539134198918e-10\\
-38.491953125	5.11784122173516e-10\\
-38.4706640625	5.7733485729617e-10\\
-38.449375	6.03570644687967e-10\\
-38.4280859375	5.7076900370127e-10\\
-38.406796875	5.35471708024782e-10\\
-38.3855078125	4.89357649744278e-10\\
-38.36421875	4.12287227598871e-10\\
-38.3429296875	3.82099111043219e-10\\
-38.321640625	2.44160631958796e-10\\
-38.3003515625	3.18992734073135e-10\\
-38.2790625	3.24672605564117e-10\\
-38.2577734375	3.39798566389652e-10\\
-38.236484375	3.59520389435369e-10\\
-38.2151953125	3.74215584116237e-10\\
-38.19390625	4.16607754939197e-10\\
-38.1726171875	4.55019169844085e-10\\
-38.151328125	3.46392284923854e-10\\
-38.1300390625	4.30561401935982e-10\\
-38.10875	3.72915444123227e-10\\
-38.0874609375	3.08874949865166e-10\\
-38.066171875	3.1067902790119e-10\\
-38.0448828125	3.16143823079156e-10\\
-38.02359375	2.6922470401978e-10\\
-38.0023046875	2.48803515876659e-10\\
-37.981015625	2.62167149664989e-10\\
-37.9597265625	2.46916510814561e-10\\
-37.9384375	3.0425050571553e-10\\
-37.9171484375	2.59629311806858e-10\\
-37.895859375	2.89145477334797e-10\\
-37.8745703125	2.27718445216148e-10\\
-37.85328125	2.56595539472526e-10\\
-37.8319921875	2.99347225096776e-10\\
-37.810703125	2.30824644161761e-10\\
-37.7894140625	2.30556368007092e-10\\
-37.768125	2.52890979524415e-10\\
-37.7468359375	2.69474700975659e-10\\
-37.725546875	1.86122093788683e-10\\
-37.7042578125	2.39312668351404e-10\\
-37.68296875	2.56061811848873e-10\\
-37.6616796875	1.55411803861385e-10\\
-37.640390625	1.06516787646282e-10\\
-37.6191015625	1.03171408912553e-10\\
-37.5978125	4.47295927745188e-11\\
-37.5765234375	1.1956606885674e-10\\
-37.555234375	6.86537048295593e-11\\
-37.5339453125	-1.05830623170888e-11\\
-37.51265625	8.37529369727591e-11\\
-37.4913671875	4.42562414168455e-11\\
-37.470078125	3.04962737432687e-11\\
-37.4487890625	2.24430821970636e-10\\
-37.4275	9.25657259243503e-11\\
-37.4062109375	1.37257542244545e-10\\
-37.384921875	2.30884156549798e-10\\
-37.3636328125	3.67175300251269e-11\\
-37.34234375	9.24625184348515e-12\\
-37.3210546875	-1.72765307230657e-11\\
-37.299765625	-1.33544748141057e-10\\
-37.2784765625	-1.06761923111133e-10\\
-37.2571875	-9.65457949486961e-11\\
-37.2358984375	-7.21433126936781e-11\\
-37.214609375	-7.31982621710865e-11\\
-37.1933203125	-6.70786374897018e-11\\
-37.17203125	3.63950556151537e-11\\
-37.1507421875	3.42682488439649e-11\\
-37.129453125	-3.90799237036078e-11\\
-37.1081640625	-4.60180097457445e-11\\
-37.086875	-1.22631149375995e-10\\
-37.0655859375	-9.61273777085958e-11\\
-37.044296875	-1.28731530951259e-10\\
-37.0230078125	-2.23525073712723e-10\\
-37.00171875	-1.13847456297573e-10\\
-36.9804296875	-7.21439812529796e-11\\
-36.959140625	-1.10009176732864e-10\\
-36.9378515625	-1.29979555971978e-10\\
-36.9165625	-1.0589220974174e-10\\
-36.8952734375	-1.49693085530504e-10\\
-36.873984375	-2.42000553383503e-10\\
-36.8526953125	-2.68396096709724e-10\\
-36.83140625	-2.98254785112352e-10\\
-36.8101171875	-2.20684260789647e-10\\
-36.788828125	-3.13006325258258e-10\\
-36.7675390625	-2.92799545038071e-10\\
-36.74625	-2.01700287958598e-10\\
-36.7249609375	-2.3470361341714e-10\\
-36.703671875	-2.52017246345549e-10\\
-36.6823828125	-2.02897202173952e-10\\
-36.66109375	-2.62692039946825e-10\\
-36.6398046875	-2.95280919597982e-10\\
-36.618515625	-3.56603896732583e-10\\
-36.5972265625	-3.87421440146981e-10\\
-36.5759375	-4.15963757996598e-10\\
-36.5546484375	-4.56504443505084e-10\\
-36.533359375	-3.65332603891696e-10\\
-36.5120703125	-4.50791168102178e-10\\
-36.49078125	-3.56877716971125e-10\\
-36.4694921875	-4.08802003963513e-10\\
-36.448203125	-4.0228945156331e-10\\
-36.4269140625	-3.82915735162851e-10\\
-36.405625	-3.59544648723426e-10\\
-36.3843359375	-3.47420858705821e-10\\
-36.363046875	-4.4102486590448e-10\\
-36.3417578125	-3.76599348384553e-10\\
-36.32046875	-3.9912990650108e-10\\
-36.2991796875	-5.18591148742485e-10\\
-36.277890625	-4.06156805030707e-10\\
-36.2566015625	-4.73294509025581e-10\\
-36.2353125	-5.51646672922303e-10\\
-36.2140234375	-4.7380720925102e-10\\
-36.192734375	-5.35654145010592e-10\\
-36.1714453125	-4.96708494330653e-10\\
-36.15015625	-6.07474069476459e-10\\
-36.1288671875	-6.66184567054785e-10\\
-36.107578125	-6.07180578392946e-10\\
-36.0862890625	-7.09670225821992e-10\\
-36.065	-6.59421390530497e-10\\
-36.0437109375	-5.35272609735244e-10\\
-36.022421875	-5.68894688784847e-10\\
-36.0011328125	-4.86783662960249e-10\\
-35.97984375	-4.42766994951936e-10\\
-35.9585546875	-4.65376955565868e-10\\
-35.937265625	-5.16351620297922e-10\\
-35.9159765625	-6.03108793584164e-10\\
-35.8946875	-7.19686358425679e-10\\
-35.8733984375	-6.92315151982934e-10\\
-35.852109375	-7.86995651635917e-10\\
-35.8308203125	-7.25186685224764e-10\\
-35.80953125	-5.92797854076996e-10\\
-35.7882421875	-6.19713866187185e-10\\
-35.766953125	-5.35559742214777e-10\\
-35.7456640625	-4.52838515202557e-10\\
-35.724375	-4.08245346487129e-10\\
-35.7030859375	-4.01020660136568e-10\\
-35.681796875	-4.18200329383809e-10\\
-35.6605078125	-5.34527354723305e-10\\
-35.63921875	-5.65172708851425e-10\\
-35.6179296875	-5.05483876625729e-10\\
-35.596640625	-6.09968638342305e-10\\
-35.5753515625	-5.27234234423925e-10\\
-35.5540625	-4.49996523818605e-10\\
-35.5327734375	-3.71020650476315e-10\\
-35.511484375	-3.58079588932469e-10\\
-35.4901953125	-3.20905209758194e-10\\
-35.46890625	-3.47972973971048e-10\\
-35.4476171875	-3.25182900912535e-10\\
-35.426328125	-4.23994619834217e-10\\
-35.4050390625	-3.55239696049411e-10\\
-35.38375	-4.29405902130201e-10\\
-35.3624609375	-4.4732116492376e-10\\
-35.341171875	-3.73261970969635e-10\\
-35.3198828125	-3.94721020293316e-10\\
-35.29859375	-2.72031367008179e-10\\
-35.2773046875	-2.72559210854474e-10\\
-35.256015625	-2.74437140766109e-10\\
-35.2347265625	-2.32962390262655e-10\\
-35.2134375	-1.95531146324881e-10\\
-35.1921484375	-2.06139838059492e-10\\
-35.170859375	-1.82651593610662e-10\\
-35.1495703125	-1.7769095321524e-10\\
-35.12828125	-2.08660257625443e-10\\
-35.1069921875	-2.47338551233235e-10\\
-35.085703125	-1.16806354310474e-10\\
-35.0644140625	-9.40725233091553e-11\\
-35.043125	3.51106655926893e-11\\
-35.0218359375	-4.97765463171777e-11\\
-35.000546875	5.39876752790229e-11\\
-34.9792578125	9.9616852238541e-11\\
-34.95796875	7.18596274530056e-12\\
-34.9366796875	2.26744588482131e-11\\
-34.915390625	-3.97868287217583e-11\\
-34.8941015625	-1.03163189262894e-10\\
-34.8728125	-4.60712071808955e-11\\
-34.8515234375	-8.13507101867789e-11\\
-34.830234375	-2.57228530851824e-11\\
-34.8089453125	3.97151177297098e-11\\
-34.78765625	3.60166299127328e-11\\
-34.7663671875	1.3638497990009e-10\\
-34.745078125	8.50992834582307e-11\\
-34.7237890625	1.43395192966666e-10\\
-34.7025	1.28855333751822e-10\\
-34.6812109375	-2.27263351963295e-11\\
-34.659921875	8.41429665924483e-11\\
-34.6386328125	1.24578585571968e-10\\
-34.61734375	2.06434191520971e-11\\
-34.5960546875	1.44148210104813e-10\\
-34.574765625	1.14237542465225e-10\\
-34.5534765625	9.3998729946739e-11\\
-34.5321875	2.10467652451573e-10\\
-34.5108984375	1.62529977567568e-10\\
-34.489609375	1.5240078342206e-10\\
-34.4683203125	1.81540533922956e-10\\
-34.44703125	1.55602184797522e-10\\
-34.4257421875	1.83755154493279e-10\\
-34.404453125	2.45861509776006e-10\\
-34.3831640625	2.04705523326278e-10\\
-34.361875	2.68849877903463e-10\\
-34.3405859375	3.18041922400845e-10\\
-34.319296875	3.3020004471623e-10\\
-34.2980078125	3.99834632091996e-10\\
-34.27671875	4.09397455588098e-10\\
-34.2554296875	2.58513431823725e-10\\
-34.234140625	3.34789743754314e-10\\
-34.2128515625	4.03537193115129e-10\\
-34.1915625	3.79971384413225e-10\\
-34.1702734375	4.51370586948254e-10\\
-34.148984375	3.86382115990401e-10\\
-34.1276953125	4.62894444052451e-10\\
-34.10640625	4.49709008403468e-10\\
-34.0851171875	5.13911717058081e-10\\
-34.063828125	5.48445397364232e-10\\
-34.0425390625	4.52225604910409e-10\\
-34.02125	5.43797990111742e-10\\
-33.9999609375	5.47088342000215e-10\\
-33.978671875	4.71502601082631e-10\\
-33.9573828125	5.40918512129787e-10\\
-33.93609375	5.52506959914953e-10\\
-33.9148046875	5.23006298599254e-10\\
-33.893515625	5.75596223774596e-10\\
-33.8722265625	5.19173156073476e-10\\
-33.8509375	5.55682121455411e-10\\
-33.8296484375	6.00342663052827e-10\\
-33.808359375	5.93238004112519e-10\\
-33.7870703125	5.9096011053671e-10\\
-33.76578125	5.97241091673187e-10\\
-33.7444921875	5.61642562588183e-10\\
-33.723203125	6.00705067317588e-10\\
-33.7019140625	5.69066997305164e-10\\
-33.680625	6.57780295645397e-10\\
-33.6593359375	5.81095554247498e-10\\
-33.638046875	6.45486744002964e-10\\
-33.6167578125	6.27587049173702e-10\\
-33.59546875	6.13298473579627e-10\\
-33.5741796875	6.87724780452472e-10\\
-33.552890625	6.7702599812177e-10\\
-33.5316015625	6.99872517223023e-10\\
-33.5103125	6.78380213122545e-10\\
-33.4890234375	6.55766317620497e-10\\
-33.467734375	6.78639229287639e-10\\
-33.4464453125	5.7292445699656e-10\\
-33.42515625	5.91680780974767e-10\\
-33.4038671875	6.32839379263763e-10\\
-33.382578125	6.586917494866e-10\\
-33.3612890625	6.498761218253e-10\\
-33.34	7.36776708133014e-10\\
-33.3187109375	7.16082117442211e-10\\
-33.297421875	6.9050321163155e-10\\
-33.2761328125	7.2695718848839e-10\\
-33.25484375	6.29045046389763e-10\\
-33.2335546875	5.20880925485999e-10\\
-33.212265625	5.72116604134964e-10\\
-33.1909765625	5.4124961164109e-10\\
-33.1696875	5.42266212016999e-10\\
-33.1483984375	5.1066446776583e-10\\
-33.127109375	5.43867109473817e-10\\
-33.1058203125	5.63408742620249e-10\\
-33.08453125	6.06281760679858e-10\\
-33.0632421875	6.22164457743285e-10\\
-33.041953125	5.38490193859388e-10\\
-33.0206640625	5.79648332205293e-10\\
-32.999375	5.10208272733699e-10\\
-32.9780859375	4.97421725360186e-10\\
-32.956796875	3.88278000228689e-10\\
-32.9355078125	4.14137680635708e-10\\
-32.91421875	4.01852694842394e-10\\
-32.8929296875	4.25546360520216e-10\\
-32.871640625	4.31635259282382e-10\\
-32.8503515625	4.55550980544529e-10\\
-32.8290625	3.95446206079097e-10\\
-32.8077734375	4.18206939464114e-10\\
-32.786484375	3.83258086583338e-10\\
-32.7651953125	4.2375677514369e-10\\
-32.74390625	3.05705884335284e-10\\
-32.7226171875	2.46399047772104e-10\\
-32.701328125	2.63382500242591e-10\\
-32.6800390625	1.56495777257977e-10\\
-32.65875	2.49963883671324e-10\\
-32.6374609375	2.58255635556337e-10\\
-32.616171875	2.83398648828913e-10\\
-32.5948828125	2.20827458613009e-10\\
-32.57359375	2.19565242862542e-10\\
-32.5523046875	1.71158416396522e-10\\
-32.531015625	1.43553959328685e-10\\
-32.5097265625	3.25996369055984e-11\\
-32.4884375	-4.46215330772838e-11\\
-32.4671484375	-3.02891961819846e-11\\
-32.445859375	-1.65229587066257e-10\\
-32.4245703125	-4.11021732517395e-11\\
-32.40328125	-4.32717711721221e-11\\
-32.3819921875	-4.65077726542774e-11\\
-32.360703125	-1.32089014285406e-11\\
-32.3394140625	-1.76248208475082e-11\\
-32.318125	-3.49413344607201e-11\\
-32.2968359375	-6.80815931675525e-11\\
-32.275546875	-9.7079077488966e-11\\
-32.2542578125	-1.67092352234028e-10\\
-32.23296875	-1.84748631137099e-10\\
-32.2116796875	-2.9848621218436e-10\\
-32.190390625	-1.9736457731181e-10\\
-32.1691015625	-1.06412987927393e-10\\
-32.1478125	-1.97930667438566e-10\\
-32.1265234375	-1.00080111093279e-10\\
-32.105234375	-1.10051412905805e-10\\
-32.0839453125	-1.99365910429166e-10\\
-32.06265625	-2.42656738892781e-10\\
-32.0413671875	-3.49539015470653e-10\\
-32.020078125	-3.761750123355e-10\\
-31.9987890625	-4.57429310308417e-10\\
-31.9775	-3.74891623152464e-10\\
-31.9562109375	-4.69651274385058e-10\\
-31.934921875	-3.48042012439886e-10\\
-31.9136328125	-3.48360390054987e-10\\
-31.89234375	-3.50279706862328e-10\\
-31.8710546875	-3.03576276164842e-10\\
-31.849765625	-4.14951274622114e-10\\
-31.8284765625	-4.31319621843033e-10\\
-31.8071875	-5.21217547424278e-10\\
-31.7858984375	-5.54051165619874e-10\\
-31.764609375	-5.34037005314631e-10\\
-31.7433203125	-5.96008785828467e-10\\
-31.72203125	-6.32775281659699e-10\\
-31.7007421875	-5.47908027926398e-10\\
-31.679453125	-6.4716197621425e-10\\
-31.6581640625	-5.57428692820422e-10\\
-31.636875	-6.5686915836605e-10\\
-31.6155859375	-6.76336201261299e-10\\
-31.594296875	-7.11221513766795e-10\\
-31.5730078125	-7.42449658706572e-10\\
-31.55171875	-7.39221515373505e-10\\
-31.5304296875	-7.65566497098533e-10\\
-31.509140625	-7.36584550880227e-10\\
-31.4878515625	-6.85519451404085e-10\\
-31.4665625	-6.49478843837113e-10\\
-31.4452734375	-6.82928830488023e-10\\
-31.423984375	-6.61023072756211e-10\\
-31.4026953125	-6.41012130393871e-10\\
-31.38140625	-7.06318814685019e-10\\
-31.3601171875	-6.85164509973378e-10\\
-31.338828125	-8.02952677481352e-10\\
-31.3175390625	-8.0076861916106e-10\\
-31.29625	-7.63027036828856e-10\\
-31.2749609375	-8.50354861390898e-10\\
-31.253671875	-7.68548411912124e-10\\
-31.2323828125	-8.03462613742871e-10\\
-31.21109375	-8.34893826900591e-10\\
-31.1898046875	-8.38233116104744e-10\\
-31.168515625	-8.73620185868108e-10\\
-31.1472265625	-8.38730124770186e-10\\
-31.1259375	-8.71175585509518e-10\\
-31.1046484375	-8.99268935477645e-10\\
-31.083359375	-9.83419586215735e-10\\
-31.0620703125	-1.05165299636472e-09\\
-31.04078125	-1.06785410033638e-09\\
-31.0194921875	-1.13283454160014e-09\\
-30.998203125	-1.08996461167398e-09\\
-30.9769140625	-1.20137442459873e-09\\
-30.955625	-1.14912839621937e-09\\
-30.9343359375	-1.12836957230563e-09\\
-30.913046875	-1.26124605607446e-09\\
-30.8917578125	-1.15653920195177e-09\\
-30.87046875	-1.1792223637081e-09\\
-30.8491796875	-1.20934393323987e-09\\
-30.827890625	-1.20118955282119e-09\\
-30.8066015625	-1.14993349930153e-09\\
-30.7853125	-1.21251310336929e-09\\
-30.7640234375	-1.16897665161151e-09\\
-30.742734375	-1.2700503827055e-09\\
-30.7214453125	-1.21752963359724e-09\\
-30.70015625	-1.17111226693096e-09\\
-30.6788671875	-1.11424966846726e-09\\
-30.657578125	-9.92506950836915e-10\\
-30.6362890625	-9.26518601224931e-10\\
-30.615	-8.11630840133294e-10\\
-30.5937109375	-8.28996702416901e-10\\
-30.572421875	-8.74957367312044e-10\\
-30.5511328125	-8.94279758666055e-10\\
-30.52984375	-8.4940199273835e-10\\
-30.5085546875	-9.08172009936147e-10\\
-30.487265625	-8.9586320915953e-10\\
-30.4659765625	-8.10285015540627e-10\\
-30.4446875	-8.17223742656464e-10\\
-30.4233984375	-7.16243920769838e-10\\
-30.402109375	-6.79990977825052e-10\\
-30.3808203125	-5.22285725768849e-10\\
-30.35953125	-5.53175954542076e-10\\
-30.3382421875	-5.22167664065684e-10\\
-30.316953125	-6.00554149018116e-10\\
-30.2956640625	-5.84249480895324e-10\\
-30.274375	-6.27151805608503e-10\\
-30.2530859375	-6.4269203775812e-10\\
-30.231796875	-6.41356138946022e-10\\
-30.2105078125	-5.54808575715302e-10\\
-30.18921875	-5.16325407780366e-10\\
-30.1679296875	-3.79136675189515e-10\\
-30.146640625	-3.56782315495237e-10\\
-30.1253515625	-2.24379052652878e-10\\
-30.1040625	-2.52481856038633e-10\\
-30.0827734375	-2.30760362795792e-10\\
-30.061484375	-2.70038046825945e-10\\
-30.0401953125	-2.78526059173835e-10\\
-30.01890625	-1.87041403150829e-10\\
-29.9976171875	-2.00433714828296e-10\\
-29.976328125	-5.03543570571302e-11\\
-29.9550390625	2.44410131949273e-11\\
-29.93375	3.80313826987905e-11\\
-29.9124609375	1.47544792088889e-10\\
-29.891171875	2.23578628879623e-10\\
-29.8698828125	1.84133693581288e-10\\
-29.84859375	9.97850347027706e-11\\
-29.8273046875	9.61096034670938e-12\\
-29.806015625	1.67643829422165e-11\\
-29.7847265625	7.18444043843812e-11\\
-29.7634375	3.31825495897224e-11\\
-29.7421484375	1.96717737346884e-11\\
-29.720859375	1.20257583785079e-10\\
-29.6995703125	1.64189999263169e-10\\
-29.67828125	1.63133240389436e-10\\
-29.6569921875	2.51794400669931e-10\\
-29.635703125	3.45495642723297e-10\\
-29.6144140625	2.74841976946654e-10\\
-29.593125	2.27442081999254e-10\\
-29.5718359375	2.60479250962361e-10\\
-29.550546875	2.63826685453283e-10\\
-29.5292578125	2.21459486047951e-10\\
-29.50796875	2.23023854628379e-10\\
-29.4866796875	2.21593780057049e-10\\
-29.465390625	1.74309351597191e-10\\
-29.4441015625	2.21671473570382e-10\\
-29.4228125	2.0209561574652e-10\\
-29.4015234375	3.25225317856208e-10\\
-29.380234375	2.32846369300212e-10\\
-29.3589453125	2.62955242642104e-10\\
-29.33765625	3.21703562352977e-10\\
-29.3163671875	2.79145031776855e-10\\
-29.295078125	2.65831405096261e-10\\
-29.2737890625	2.38103670408574e-10\\
-29.2525	2.74565003747673e-10\\
-29.2312109375	3.18454958841506e-10\\
-29.209921875	4.049398915104e-10\\
-29.1886328125	4.54621176035527e-10\\
-29.16734375	5.5288333796369e-10\\
-29.1460546875	5.65314770459273e-10\\
-29.124765625	5.80039866113399e-10\\
-29.1034765625	6.14739592114875e-10\\
-29.0821875	6.65040219031597e-10\\
-29.0608984375	5.85037862593596e-10\\
-29.039609375	6.03853140679303e-10\\
-29.0183203125	6.54080848223514e-10\\
-28.99703125	5.08401480703892e-10\\
-28.9757421875	5.61391471193008e-10\\
-28.954453125	5.69241759779556e-10\\
-28.9331640625	5.65532628148831e-10\\
-28.911875	5.51732195528636e-10\\
-28.8905859375	5.48748600798188e-10\\
-28.869296875	5.48396291762042e-10\\
-28.8480078125	4.98571514101992e-10\\
-28.82671875	4.97563490256092e-10\\
-28.8054296875	4.69713733237431e-10\\
-28.784140625	4.46161048223059e-10\\
-28.7628515625	4.74245704324771e-10\\
-28.7415625	5.85018144809157e-10\\
-28.7202734375	5.10696823774161e-10\\
-28.698984375	4.43943673968295e-10\\
-28.6776953125	5.70290878764262e-10\\
-28.65640625	5.37587543647126e-10\\
-28.6351171875	5.67731612041439e-10\\
-28.613828125	6.54487243865201e-10\\
-28.5925390625	6.10096808017274e-10\\
-28.57125	6.7682826517197e-10\\
-28.5499609375	6.04654943616979e-10\\
-28.528671875	5.99147221332224e-10\\
-28.5073828125	6.74918719475524e-10\\
-28.48609375	6.95468053953108e-10\\
-28.4648046875	7.04843844556747e-10\\
-28.443515625	7.30176627489859e-10\\
-28.4222265625	7.71127264990927e-10\\
-28.4009375	8.24933668200854e-10\\
-28.3796484375	8.15961796913888e-10\\
-28.358359375	8.12226454759248e-10\\
-28.3370703125	7.92526905139794e-10\\
-28.31578125	8.22374683765647e-10\\
-28.2944921875	6.89473236924688e-10\\
-28.273203125	7.16883811699029e-10\\
-28.2519140625	6.18923127553242e-10\\
-28.230625	6.86384383050547e-10\\
-28.2093359375	6.79447664149441e-10\\
-28.188046875	7.82576734357629e-10\\
-28.1667578125	7.77164762643361e-10\\
-28.14546875	8.19238071415611e-10\\
-28.1241796875	8.21559830560266e-10\\
-28.102890625	8.30188382119121e-10\\
-28.0816015625	8.04043981528016e-10\\
-28.0603125	7.41280236525241e-10\\
-28.0390234375	6.4847600798052e-10\\
-28.017734375	6.52620751193711e-10\\
-27.9964453125	5.05792611178727e-10\\
-27.97515625	5.59644251180268e-10\\
-27.9538671875	4.44000505032471e-10\\
-27.932578125	4.87407550590538e-10\\
-27.9112890625	5.20916565278933e-10\\
-27.89	4.98204836490507e-10\\
-27.8687109375	5.56593341677213e-10\\
-27.847421875	5.49284109274862e-10\\
-27.8261328125	5.30851777933246e-10\\
-27.80484375	4.29109439101411e-10\\
-27.7835546875	4.59916277055856e-10\\
-27.762265625	3.122196758428e-10\\
-27.7409765625	4.05981682683579e-10\\
-27.7196875	1.99103770411897e-10\\
-27.6983984375	2.71874834452284e-10\\
-27.677109375	2.637958616845e-10\\
-27.6558203125	3.11778465339923e-10\\
-27.63453125	3.39594396778434e-10\\
-27.6132421875	2.99726970658171e-10\\
-27.591953125	1.94652228896297e-10\\
-27.5706640625	5.52548628794315e-11\\
-27.549375	3.19977681985931e-11\\
-27.5280859375	-3.62830821104444e-11\\
-27.506796875	-1.23615848089424e-10\\
-27.4855078125	-9.45447948955415e-11\\
-27.46421875	-1.2566536708817e-10\\
-27.4429296875	-4.68141349027137e-11\\
-27.421640625	-3.14380305478693e-12\\
-27.4003515625	-4.95703742695615e-12\\
-27.3790625	1.69170038223324e-11\\
-27.3577734375	2.36091184719583e-11\\
-27.336484375	-9.5531408996223e-11\\
-27.3151953125	-2.06413640818431e-10\\
-27.29390625	-3.17921510615855e-10\\
-27.2726171875	-4.12592814276901e-10\\
-27.251328125	-3.78712667043156e-10\\
-27.2300390625	-4.41835175106487e-10\\
-27.20875	-5.1518226163871e-10\\
-27.1874609375	-4.28799681303998e-10\\
-27.166171875	-4.29247192581176e-10\\
-27.1448828125	-4.85850599496122e-10\\
-27.12359375	-3.74663024417796e-10\\
-27.1023046875	-3.46596269518459e-10\\
-27.081015625	-3.82769655194011e-10\\
-27.0597265625	-3.71049549422809e-10\\
-27.0384375	-4.52341821077161e-10\\
-27.0171484375	-5.2211582754254e-10\\
-26.995859375	-5.87226265861116e-10\\
-26.9745703125	-5.48087479556326e-10\\
-26.95328125	-6.02522504923071e-10\\
-26.9319921875	-5.2108913542291e-10\\
-26.910703125	-5.73981563449005e-10\\
-26.8894140625	-4.68123256997848e-10\\
-26.868125	-4.56378103926779e-10\\
-26.8468359375	-5.37529639780062e-10\\
-26.825546875	-5.48292452445951e-10\\
-26.8042578125	-5.26789989862958e-10\\
-26.78296875	-6.39232081120132e-10\\
-26.7616796875	-6.48316270621058e-10\\
-26.740390625	-6.85435465713639e-10\\
-26.7191015625	-7.10923621769163e-10\\
-26.6978125	-7.2376449143951e-10\\
-26.6765234375	-7.4875187511791e-10\\
-26.655234375	-7.45898451085485e-10\\
-26.6339453125	-7.00794296490682e-10\\
-26.61265625	-7.24824178266011e-10\\
-26.5913671875	-8.1589558480257e-10\\
-26.570078125	-7.81062100834167e-10\\
-26.5487890625	-8.57149674824825e-10\\
-26.5275	-9.51596456623694e-10\\
-26.5062109375	-9.2686820136125e-10\\
-26.484921875	-9.55191306620192e-10\\
-26.4636328125	-9.31464862651245e-10\\
-26.44234375	-8.7056795713659e-10\\
-26.4210546875	-8.04887748161187e-10\\
-26.399765625	-7.17302767488324e-10\\
-26.3784765625	-6.90738714156679e-10\\
-26.3571875	-6.99553085869699e-10\\
-26.3358984375	-6.89252790556211e-10\\
-26.314609375	-7.63430463029352e-10\\
-26.2933203125	-7.50705371333276e-10\\
-26.27203125	-7.43280522089009e-10\\
-26.2507421875	-8.41404980967389e-10\\
-26.229453125	-7.73149387536587e-10\\
-26.2081640625	-7.21767304725698e-10\\
-26.186875	-7.28247246077409e-10\\
-26.1655859375	-7.42236963669736e-10\\
-26.144296875	-6.67268457666574e-10\\
-26.1230078125	-6.84244917190179e-10\\
-26.10171875	-7.26581508194422e-10\\
-26.0804296875	-6.78487145652583e-10\\
-26.059140625	-7.10446585815334e-10\\
-26.0378515625	-7.73212488665028e-10\\
-26.0165625	-7.11748937013257e-10\\
-25.9952734375	-7.76949597518846e-10\\
-25.973984375	-7.83854841071482e-10\\
-25.9526953125	-6.75139292527901e-10\\
-25.93140625	-7.6722626790901e-10\\
-25.9101171875	-6.62000455085208e-10\\
-25.888828125	-6.74523128477608e-10\\
-25.8675390625	-8.03578942583696e-10\\
-25.84625	-8.19429675546238e-10\\
-25.8249609375	-8.14930648812436e-10\\
-25.803671875	-1.03535800860768e-09\\
-25.7823828125	-9.55253021172239e-10\\
-25.76109375	-9.54132613535756e-10\\
-25.7398046875	-8.90269975881949e-10\\
-25.718515625	-7.31004744592501e-10\\
-25.6972265625	-6.92616053926774e-10\\
-25.6759375	-6.18713269708745e-10\\
-25.6546484375	-5.74716749131553e-10\\
-25.633359375	-6.18606326011695e-10\\
-25.6120703125	-6.30497467073756e-10\\
-25.59078125	-6.90038858108356e-10\\
-25.5694921875	-7.56457097658585e-10\\
-25.548203125	-6.53303477184721e-10\\
-25.5269140625	-6.77898375354098e-10\\
-25.505625	-5.38788793289291e-10\\
-25.4843359375	-5.33359160227034e-10\\
-25.463046875	-3.85520358599429e-10\\
-25.4417578125	-4.42610059978523e-10\\
-25.42046875	-3.50499393223929e-10\\
-25.3991796875	-2.99217959652858e-10\\
-25.377890625	-4.31370984608631e-10\\
-25.3566015625	-3.37127845773308e-10\\
-25.3353125	-3.46834892140958e-10\\
-25.3140234375	-3.47531073342987e-10\\
-25.292734375	-2.89583833774504e-10\\
-25.2714453125	-2.19374389323976e-10\\
-25.25015625	-2.30468729977303e-10\\
-25.2288671875	-1.93681102875423e-10\\
-25.207578125	-1.96364731480588e-10\\
-25.1862890625	-1.3997995438499e-10\\
-25.165	-2.73122145131845e-10\\
-25.1437109375	-1.99226107637743e-10\\
-25.122421875	-7.45912342751065e-11\\
-25.1011328125	-2.23497148891542e-12\\
-25.07984375	2.15946681264476e-12\\
-25.0585546875	1.8729636592836e-10\\
-25.037265625	1.55090762635868e-10\\
-25.0159765625	3.0419400672369e-10\\
-24.9946875	3.56597620600351e-10\\
-24.9733984375	4.13043022959436e-10\\
-24.952109375	3.57011847014916e-10\\
-24.9308203125	3.86188726943233e-10\\
-24.90953125	4.28689681119769e-10\\
-24.8882421875	4.08321077965231e-10\\
-24.866953125	4.4874549595601e-10\\
-24.8456640625	5.47531406604358e-10\\
-24.824375	5.57679507036517e-10\\
-24.8030859375	5.84871697835507e-10\\
-24.781796875	6.37518733757191e-10\\
-24.7605078125	7.28763425381935e-10\\
-24.73921875	8.1040800826891e-10\\
-24.7179296875	7.90715482549055e-10\\
-24.696640625	6.8903998036336e-10\\
-24.6753515625	8.75167802166288e-10\\
-24.6540625	8.47658057235433e-10\\
-24.6327734375	8.22558671522306e-10\\
-24.611484375	8.79142059449652e-10\\
-24.5901953125	1.0001259390438e-09\\
-24.56890625	1.0002523111013e-09\\
-24.5476171875	1.08695437428291e-09\\
-24.526328125	1.04273953843052e-09\\
-24.5050390625	1.07365321785732e-09\\
-24.48375	1.02851791350689e-09\\
-24.4624609375	9.54483312605443e-10\\
-24.441171875	1.02232891026871e-09\\
-24.4198828125	1.03324520560433e-09\\
-24.39859375	1.00141295621192e-09\\
-24.3773046875	1.16758330133628e-09\\
-24.356015625	1.22053937387838e-09\\
-24.3347265625	1.34637522873864e-09\\
-24.3134375	1.30957993756784e-09\\
-24.2921484375	1.34120784932484e-09\\
-24.270859375	1.29454346472359e-09\\
-24.2495703125	1.13718145496832e-09\\
-24.22828125	1.23707952208981e-09\\
-24.2069921875	1.1407366842662e-09\\
-24.185703125	1.20028421007333e-09\\
-24.1644140625	1.2801029400165e-09\\
-24.143125	1.30162418155656e-09\\
-24.1218359375	1.37267650399856e-09\\
-24.100546875	1.55903553497567e-09\\
-24.0792578125	1.56766617735028e-09\\
-24.05796875	1.54697967634348e-09\\
-24.0366796875	1.40084033698955e-09\\
-24.015390625	1.53650724743525e-09\\
-23.9941015625	1.39203449518779e-09\\
-23.9728125	1.40758525738629e-09\\
-23.9515234375	1.42245738562865e-09\\
-23.930234375	1.35081078419325e-09\\
-23.9089453125	1.36368569200543e-09\\
-23.88765625	1.42189020066828e-09\\
-23.8663671875	1.40600880500724e-09\\
-23.845078125	1.5288610674194e-09\\
-23.8237890625	1.42343868972545e-09\\
-23.8025	1.43734623443336e-09\\
-23.7812109375	1.3855163962777e-09\\
-23.759921875	1.36782525064711e-09\\
-23.7386328125	1.37069044889317e-09\\
-23.71734375	1.37492902011056e-09\\
-23.6960546875	1.42655144794181e-09\\
-23.674765625	1.35789340532703e-09\\
-23.6534765625	1.44033320336917e-09\\
-23.6321875	1.33055560183916e-09\\
-23.6108984375	1.36922806055125e-09\\
-23.589609375	1.38265280809895e-09\\
-23.5683203125	1.35162291933554e-09\\
-23.54703125	1.22849156343757e-09\\
-23.5257421875	1.3980789044272e-09\\
-23.504453125	1.20944933794512e-09\\
-23.4831640625	1.32520833925729e-09\\
-23.461875	1.26890109839731e-09\\
-23.4405859375	1.35137267314331e-09\\
-23.419296875	1.25460793039293e-09\\
-23.3980078125	1.28422415147233e-09\\
-23.37671875	1.26970314047686e-09\\
-23.3554296875	1.43313751150159e-09\\
-23.334140625	1.26601713406266e-09\\
-23.3128515625	1.22739860506334e-09\\
-23.2915625	1.24024368853624e-09\\
-23.2702734375	1.15411887778884e-09\\
-23.248984375	1.19976446108286e-09\\
-23.2276953125	1.07143753656135e-09\\
-23.20640625	1.11804092665073e-09\\
-23.1851171875	1.13321728373108e-09\\
-23.163828125	1.11794106978777e-09\\
-23.1425390625	1.05181198053515e-09\\
-23.12125	1.11624575874647e-09\\
-23.0999609375	1.04351158235173e-09\\
-23.078671875	1.00647760527963e-09\\
-23.0573828125	9.67822555547567e-10\\
-23.03609375	9.83220660484108e-10\\
-23.0148046875	1.09336420650328e-09\\
-22.993515625	1.15268284686659e-09\\
-22.9722265625	1.10723671260572e-09\\
-22.9509375	1.12017422594017e-09\\
-22.9296484375	1.0899988369299e-09\\
-22.908359375	9.18055598556426e-10\\
-22.8870703125	8.39052526319235e-10\\
-22.86578125	8.29520403891003e-10\\
-22.8444921875	6.48906026833202e-10\\
-22.823203125	6.48693337271668e-10\\
-22.8019140625	5.7302696784734e-10\\
-22.780625	6.84743577247674e-10\\
-22.7593359375	7.19038559494095e-10\\
-22.738046875	7.87948175738561e-10\\
-22.7167578125	8.41081180339216e-10\\
-22.69546875	8.33142087254951e-10\\
-22.6741796875	8.88175607501062e-10\\
-22.652890625	7.19937691228698e-10\\
-22.6316015625	6.50774838345136e-10\\
-22.6103125	4.24526716838169e-10\\
-22.5890234375	4.77207079385218e-10\\
-22.567734375	3.23470699475695e-10\\
-22.5464453125	3.81784289891136e-10\\
-22.52515625	3.36178084901272e-10\\
-22.5038671875	4.65912809096063e-10\\
-22.482578125	3.30805233922295e-10\\
-22.4612890625	4.32247730656263e-10\\
-22.44	4.24788313241925e-10\\
-22.4187109375	2.38321059588922e-10\\
-22.397421875	1.03014139566169e-10\\
-22.3761328125	1.2277342497008e-10\\
-22.35484375	6.60073693617214e-11\\
-22.3335546875	1.22198950582845e-10\\
-22.312265625	1.69375371461963e-10\\
-22.2909765625	2.5888179057887e-10\\
-22.2696875	2.31190916462549e-10\\
-22.2483984375	2.81572831846812e-10\\
-22.227109375	1.09775177302805e-10\\
-22.2058203125	1.11873701475947e-10\\
-22.18453125	-4.73104247185898e-12\\
-22.1632421875	-2.67775249254311e-10\\
-22.141953125	-1.63144530930947e-10\\
-22.1206640625	-3.87708727849844e-10\\
-22.099375	-3.15428769829429e-10\\
-22.0780859375	-2.44433141837238e-10\\
-22.056796875	-2.41870421317001e-10\\
-22.0355078125	-3.22853228398898e-10\\
-22.01421875	-4.27627143252774e-10\\
-21.9929296875	-3.75798694178149e-10\\
-21.971640625	-4.38732571081656e-10\\
-21.9503515625	-4.10413942096093e-10\\
-21.9290625	-4.94317122723212e-10\\
-21.9077734375	-4.31132831856275e-10\\
-21.886484375	-4.69241797771434e-10\\
-21.8651953125	-4.49262011308657e-10\\
-21.84390625	-5.50630217623499e-10\\
-21.8226171875	-6.08245213903956e-10\\
-21.801328125	-7.21122724767059e-10\\
-21.7800390625	-6.99434351648002e-10\\
-21.75875	-7.99987005095622e-10\\
-21.7374609375	-9.49134700688765e-10\\
-21.716171875	-9.03281612762082e-10\\
-21.6948828125	-8.26190435094093e-10\\
-21.67359375	-8.76642940924072e-10\\
-21.6523046875	-8.53776177657948e-10\\
-21.631015625	-9.05309143643582e-10\\
-21.6097265625	-1.01652799483514e-09\\
-21.5884375	-1.12693714679831e-09\\
-21.5671484375	-1.23497383514389e-09\\
-21.545859375	-1.25846976956425e-09\\
-21.5245703125	-1.27908163677014e-09\\
-21.50328125	-1.14137008522567e-09\\
-21.4819921875	-1.14094503852131e-09\\
-21.460703125	-1.00389124894296e-09\\
-21.4394140625	-9.99264232672339e-10\\
-21.418125	-8.41568748721211e-10\\
-21.3968359375	-9.87126703718498e-10\\
-21.375546875	-9.69780266040301e-10\\
-21.3542578125	-1.1752828636123e-09\\
-21.33296875	-1.08220220832963e-09\\
-21.3116796875	-1.21168529987795e-09\\
-21.290390625	-1.10181583134472e-09\\
-21.2691015625	-1.02047904115687e-09\\
-21.2478125	-1.11813839468647e-09\\
-21.2265234375	-1.02704198882688e-09\\
-21.205234375	-1.08092311412726e-09\\
-21.1839453125	-1.07903682712854e-09\\
-21.16265625	-1.14511344879845e-09\\
-21.1413671875	-1.09527934237017e-09\\
-21.120078125	-1.12198389121174e-09\\
-21.0987890625	-1.08117036293823e-09\\
-21.0775	-1.049982732106e-09\\
-21.0562109375	-9.27963676060425e-10\\
-21.034921875	-1.02401110268363e-09\\
-21.0136328125	-8.98169041200539e-10\\
-20.99234375	-9.5044308792914e-10\\
-20.9710546875	-8.65377795583154e-10\\
-20.949765625	-1.01809885951896e-09\\
-20.9284765625	-9.53640923707166e-10\\
-20.9071875	-8.3589795311301e-10\\
-20.8858984375	-9.89184189693864e-10\\
-20.864609375	-8.43056656073649e-10\\
-20.8433203125	-6.59310368299746e-10\\
-20.82203125	-5.45154794454723e-10\\
-20.8007421875	-4.84471011166408e-10\\
-20.779453125	-4.04341872033977e-10\\
-20.7581640625	-4.85264215546868e-10\\
-20.736875	-4.95652355573012e-10\\
-20.7155859375	-6.06145984598095e-10\\
-20.694296875	-6.36628500501164e-10\\
-20.6730078125	-6.08286475791147e-10\\
-20.65171875	-7.74238384651559e-10\\
-20.6304296875	-6.10984472842371e-10\\
-20.609140625	-4.41185763675818e-10\\
-20.5878515625	-5.16233257686901e-10\\
-20.5665625	-4.48502947024158e-10\\
-20.5452734375	-3.15488872632191e-10\\
-20.523984375	-3.02503843744751e-10\\
-20.5026953125	-4.93329500398328e-10\\
-20.48140625	-5.12887056213037e-10\\
-20.4601171875	-4.16200182725746e-10\\
-20.438828125	-5.43776149688432e-10\\
-20.4175390625	-5.03386954315628e-10\\
-20.39625	-3.9987468113889e-10\\
-20.3749609375	-2.21771968956987e-10\\
-20.353671875	-2.21691210388963e-10\\
-20.3323828125	-3.08987465000803e-10\\
-20.31109375	-2.09288960919935e-10\\
-20.2898046875	-2.85000628622538e-10\\
-20.268515625	-3.29681643302835e-10\\
-20.2472265625	-4.06289999462279e-10\\
-20.2259375	-4.35183669317589e-10\\
-20.2046484375	-4.74118670850442e-10\\
-20.183359375	-4.84129586172271e-10\\
-20.1620703125	-3.42604927717076e-10\\
-20.14078125	-1.97794739786551e-10\\
-20.1194921875	-1.75851857955901e-10\\
-20.098203125	-8.51520640255799e-11\\
-20.0769140625	-6.5514738563819e-11\\
-20.055625	-7.66902172932335e-11\\
-20.0343359375	-8.50401657502171e-11\\
-20.013046875	-2.65968303985712e-11\\
-19.9917578125	-4.01262404068836e-11\\
-19.97046875	1.17347782960796e-10\\
-19.9491796875	1.76626551108486e-10\\
-19.927890625	2.3859735142042e-10\\
-19.9066015625	3.70131202814693e-10\\
-19.8853125	3.57545998117568e-10\\
-19.8640234375	3.94649504756752e-10\\
-19.842734375	5.00823465378894e-10\\
-19.8214453125	3.77441597589265e-10\\
-19.80015625	5.27881516298416e-10\\
-19.7788671875	5.95518252012361e-10\\
-19.757578125	7.17659513634207e-10\\
-19.7362890625	9.07498044644676e-10\\
-19.715	8.66140668228895e-10\\
-19.6937109375	9.88554236739197e-10\\
-19.672421875	1.12793103670275e-09\\
-19.6511328125	1.12619346663864e-09\\
-19.62984375	1.19639703643391e-09\\
-19.6085546875	1.15402136837386e-09\\
-19.587265625	9.2442455447634e-10\\
-19.5659765625	8.96772197060197e-10\\
-19.5446875	8.41303665160482e-10\\
-19.5233984375	9.06139893917818e-10\\
-19.502109375	8.34113819252887e-10\\
-19.4808203125	9.68759773052088e-10\\
-19.45953125	1.12922321350964e-09\\
-19.4382421875	1.13812802849383e-09\\
-19.416953125	1.37583480046493e-09\\
-19.3956640625	1.44717659309542e-09\\
-19.374375	1.28594349846987e-09\\
-19.3530859375	1.14498697643923e-09\\
-19.331796875	1.09861561743903e-09\\
-19.3105078125	8.93031360296565e-10\\
-19.28921875	8.93424230796791e-10\\
-19.2679296875	7.80709079601962e-10\\
-19.246640625	7.9415499190709e-10\\
-19.2253515625	8.62107441402703e-10\\
-19.2040625	8.97737692619971e-10\\
-19.1827734375	1.0388438366699e-09\\
-19.161484375	1.15743356345498e-09\\
-19.1401953125	1.13555785850538e-09\\
-19.11890625	1.1309906896631e-09\\
-19.0976171875	1.05051481483848e-09\\
-19.076328125	1.11830353281779e-09\\
-19.0550390625	1.02496812023772e-09\\
-19.03375	1.01419860762895e-09\\
-19.0124609375	1.06407986815594e-09\\
-18.991171875	1.224320931822e-09\\
-18.9698828125	1.25642456921497e-09\\
-18.94859375	1.31576118839564e-09\\
-18.9273046875	1.30547376886108e-09\\
-18.906015625	1.35685402089294e-09\\
-18.8847265625	1.25610571044632e-09\\
-18.8634375	1.26197137018767e-09\\
-18.8421484375	1.10247750340567e-09\\
-18.820859375	1.19514561831254e-09\\
-18.7995703125	1.13608004666045e-09\\
-18.77828125	1.09919306615845e-09\\
-18.7569921875	1.26962395694027e-09\\
-18.735703125	1.21757302354823e-09\\
-18.7144140625	1.26828172356081e-09\\
-18.693125	1.15520300324398e-09\\
-18.6718359375	1.17000769019593e-09\\
-18.650546875	1.18685582363253e-09\\
-18.6292578125	1.27456050241158e-09\\
-18.60796875	1.17349376272186e-09\\
-18.5866796875	1.20007655569849e-09\\
-18.565390625	1.18488179517382e-09\\
-18.5441015625	1.22907204115384e-09\\
-18.5228125	1.26946476293798e-09\\
-18.5015234375	1.12817909750558e-09\\
-18.480234375	1.1899167700074e-09\\
-18.4589453125	1.10119097028916e-09\\
-18.43765625	1.14512985779674e-09\\
-18.4163671875	1.07784384366879e-09\\
-18.395078125	1.07671853186235e-09\\
-18.3737890625	1.13695684020698e-09\\
-18.3525	1.0681075905111e-09\\
-18.3312109375	1.02770831451261e-09\\
-18.309921875	1.01102644172405e-09\\
-18.2886328125	1.05834926177249e-09\\
-18.26734375	1.00927905637211e-09\\
-18.2460546875	1.07689962334701e-09\\
-18.224765625	9.7517207946456e-10\\
-18.2034765625	9.99955970234136e-10\\
-18.1821875	9.66360690405734e-10\\
-18.1608984375	8.21702269126991e-10\\
-18.139609375	7.8662420635489e-10\\
-18.1183203125	6.14884324073491e-10\\
-18.09703125	6.65196336278416e-10\\
-18.0757421875	7.53065078105731e-10\\
-18.054453125	6.85756662647044e-10\\
-18.0331640625	7.80148398718169e-10\\
-18.011875	8.54914725353024e-10\\
-17.9905859375	8.53782567963065e-10\\
-17.969296875	6.76874848626343e-10\\
-17.9480078125	7.38968249955265e-10\\
-17.92671875	7.33137028923198e-10\\
-17.9054296875	5.60361605995921e-10\\
-17.884140625	5.51941475474213e-10\\
-17.8628515625	6.11900999628973e-10\\
-17.8415625	5.71599004307822e-10\\
-17.8202734375	4.31616076537517e-10\\
-17.798984375	3.64577726523051e-10\\
-17.7776953125	4.01920129438468e-10\\
-17.75640625	4.76208310695231e-10\\
-17.7351171875	3.18897730068861e-10\\
-17.713828125	4.65877292442257e-10\\
-17.6925390625	4.88392046446444e-10\\
-17.67125	4.23506164497324e-10\\
-17.6499609375	3.16899399632966e-10\\
-17.628671875	3.45094976108866e-10\\
-17.6073828125	1.92492855777737e-10\\
-17.58609375	2.26249405002925e-10\\
-17.5648046875	8.61624397539014e-11\\
-17.543515625	1.13613467520218e-10\\
-17.5222265625	1.39621112584672e-10\\
-17.5009375	-4.62600496611633e-11\\
-17.4796484375	9.24426947997661e-11\\
-17.458359375	-4.71915795094077e-12\\
-17.4370703125	-2.17122256429e-11\\
-17.41578125	-7.81001278353012e-11\\
-17.3944921875	-1.49732235012685e-10\\
-17.373203125	-1.34955717433748e-10\\
-17.3519140625	-2.70558646047897e-10\\
-17.330625	-1.94599307051622e-10\\
-17.3093359375	-1.60681054440851e-10\\
-17.288046875	-2.7384881662355e-10\\
-17.2667578125	-2.26553653995702e-10\\
-17.24546875	-3.08101073840923e-10\\
-17.2241796875	-4.09858098979762e-10\\
-17.202890625	-3.20116894985819e-10\\
-17.1816015625	-6.21252455631733e-10\\
-17.1603125	-6.62033365904184e-10\\
-17.1390234375	-5.9005420678951e-10\\
-17.117734375	-7.21341019018338e-10\\
-17.0964453125	-5.7437462097959e-10\\
-17.07515625	-6.27947057462268e-10\\
-17.0538671875	-7.08803131407994e-10\\
-17.032578125	-6.64550099678514e-10\\
-17.0112890625	-7.05043546655126e-10\\
-16.99	-8.24545199061271e-10\\
-16.9687109375	-7.69086687938541e-10\\
-16.947421875	-7.09947303300517e-10\\
-16.9261328125	-7.58741843923543e-10\\
-16.90484375	-6.31113376833482e-10\\
-16.8835546875	-6.64725140704439e-10\\
-16.862265625	-7.53937125052249e-10\\
-16.8409765625	-7.93907635541819e-10\\
-16.8196875	-9.08349619773849e-10\\
-16.7983984375	-1.0057033898215e-09\\
-16.777109375	-9.88523180243989e-10\\
-16.7558203125	-9.78006591822904e-10\\
-16.73453125	-8.63441556217607e-10\\
-16.7132421875	-8.15607940336127e-10\\
-16.691953125	-7.25049096039673e-10\\
-16.6706640625	-6.20610948841503e-10\\
-16.649375	-6.6262069666402e-10\\
-16.6280859375	-7.43135916963126e-10\\
-16.606796875	-7.96140519153494e-10\\
-16.5855078125	-9.93523991844169e-10\\
-16.56421875	-1.15199556150221e-09\\
-16.5429296875	-1.11628796758074e-09\\
-16.521640625	-1.20777889489801e-09\\
-16.5003515625	-1.24140663287101e-09\\
-16.4790625	-1.05038731830423e-09\\
-16.4577734375	-1.02569223843903e-09\\
-16.436484375	-9.31656823465281e-10\\
-16.4151953125	-1.0799786940052e-09\\
-16.39390625	-9.92849620909729e-10\\
-16.3726171875	-1.03348088181218e-09\\
-16.351328125	-1.09830112537291e-09\\
-16.3300390625	-1.00457630470035e-09\\
-16.30875	-1.09031491977195e-09\\
-16.2874609375	-9.98331240053558e-10\\
-16.266171875	-9.67546764425003e-10\\
-16.2448828125	-8.60111572106286e-10\\
-16.22359375	-7.25554401320209e-10\\
-16.2023046875	-8.42795421978402e-10\\
-16.181015625	-8.05943168748983e-10\\
-16.1597265625	-9.31611008253288e-10\\
-16.1384375	-1.11008866462109e-09\\
-16.1171484375	-1.03189081608973e-09\\
-16.095859375	-1.06489989201869e-09\\
-16.0745703125	-1.03062655376873e-09\\
-16.05328125	-9.29495031506745e-10\\
-16.0319921875	-8.46064116931488e-10\\
-16.010703125	-8.99687369504853e-10\\
-15.9894140625	-7.73994802896248e-10\\
-15.968125	-6.90513105519585e-10\\
-15.9468359375	-8.92534875250168e-10\\
-15.925546875	-6.84165318062578e-10\\
-15.9042578125	-8.18212248345355e-10\\
-15.88296875	-8.87281267023622e-10\\
-15.8616796875	-8.69865027008179e-10\\
-15.840390625	-9.18047357400479e-10\\
-15.8191015625	-1.02083978155338e-09\\
-15.7978125	-8.81117809693575e-10\\
-15.7765234375	-1.03766003990306e-09\\
-15.755234375	-1.00530938614317e-09\\
-15.7339453125	-8.74692672168983e-10\\
-15.71265625	-8.09936541427901e-10\\
-15.6913671875	-6.90374287003927e-10\\
-15.670078125	-7.18548081855543e-10\\
-15.6487890625	-6.94405289294128e-10\\
-15.6275	-7.59074565959877e-10\\
-15.6062109375	-7.19151917758872e-10\\
-15.584921875	-6.58796092004666e-10\\
-15.5636328125	-5.18282459134418e-10\\
-15.54234375	-4.43420231455152e-10\\
-15.5210546875	-5.05363492883434e-10\\
-15.499765625	-3.24070557222181e-10\\
-15.4784765625	-1.98927632225897e-10\\
-15.4571875	-3.39203914587669e-10\\
-15.4358984375	-2.86198079141916e-10\\
-15.414609375	-2.44556707139157e-10\\
-15.3933203125	-3.2238311231807e-10\\
-15.37203125	-3.90030377028983e-10\\
-15.3507421875	-3.51446744466022e-10\\
-15.329453125	-2.50718546040052e-10\\
-15.3081640625	-2.82249414761725e-10\\
-15.286875	-2.39131590373905e-10\\
-15.2655859375	-1.17803730443802e-10\\
-15.244296875	-2.46282150863002e-11\\
-15.2230078125	6.27438439950771e-11\\
-15.20171875	7.32211024126401e-11\\
-15.1804296875	-3.3768003169713e-11\\
-15.159140625	-3.69963180340742e-11\\
-15.1378515625	-4.7636546243828e-11\\
-15.1165625	5.44992508401268e-11\\
-15.0952734375	-1.83531776980161e-11\\
-15.073984375	6.79821863696775e-13\\
-15.0526953125	1.54417359507076e-11\\
-15.03140625	1.15972683772e-10\\
-15.0101171875	1.78569776302372e-10\\
-14.988828125	2.53854782309771e-10\\
-14.9675390625	3.08585562394214e-10\\
-14.94625	3.82321182733261e-10\\
-14.9249609375	2.88397069843787e-10\\
-14.903671875	3.4319411417344e-10\\
-14.8823828125	3.89786785985905e-10\\
-14.86109375	2.93119351484934e-10\\
-14.8398046875	3.71821653779997e-10\\
-14.818515625	3.75759819940761e-10\\
-14.7972265625	4.77610453125512e-10\\
-14.7759375	4.5019183608782e-10\\
-14.7546484375	4.81884640640225e-10\\
-14.733359375	6.91229004777042e-10\\
-14.7120703125	7.43100424486945e-10\\
-14.69078125	7.29346932470424e-10\\
-14.6694921875	9.46793624463925e-10\\
-14.648203125	8.48506056091773e-10\\
-14.6269140625	8.36489375713661e-10\\
-14.605625	9.47560721205441e-10\\
-14.5843359375	8.87283608385325e-10\\
-14.563046875	1.06067948077917e-09\\
-14.5417578125	1.01962741248765e-09\\
-14.52046875	1.14274755902226e-09\\
-14.4991796875	1.36344669889243e-09\\
-14.477890625	1.37212028484499e-09\\
-14.4566015625	1.42165288711146e-09\\
-14.4353125	1.58049624782368e-09\\
-14.4140234375	1.50269630719893e-09\\
-14.392734375	1.3337000901952e-09\\
-14.3714453125	1.37922880119766e-09\\
-14.35015625	1.16205248571705e-09\\
-14.3288671875	1.14678173467863e-09\\
-14.307578125	1.19957186415925e-09\\
-14.2862890625	1.35751536675238e-09\\
-14.265	1.4719743113454e-09\\
-14.2437109375	1.70193265395903e-09\\
-14.222421875	1.68470823979323e-09\\
-14.2011328125	1.83078198539294e-09\\
-14.17984375	1.57875499938189e-09\\
-14.1585546875	1.48680175037113e-09\\
-14.137265625	1.41843267657346e-09\\
-14.1159765625	1.20805110680382e-09\\
-14.0946875	1.13735459134576e-09\\
-14.0733984375	1.23272998263755e-09\\
-14.052109375	1.3090034438459e-09\\
-14.0308203125	1.46208858566141e-09\\
-14.00953125	1.66977203353293e-09\\
-13.9882421875	1.67193798997111e-09\\
-13.966953125	1.75307747829341e-09\\
-13.9456640625	1.67525533178241e-09\\
-13.924375	1.56248546848171e-09\\
-13.9030859375	1.38120260020013e-09\\
-13.881796875	1.28990130604071e-09\\
-13.8605078125	1.16563670770706e-09\\
-13.83921875	1.19566080711726e-09\\
-13.8179296875	1.23135600309897e-09\\
-13.796640625	1.41448579490927e-09\\
-13.7753515625	1.33600541617025e-09\\
-13.7540625	1.59192898325533e-09\\
-13.7327734375	1.481114125982e-09\\
-13.711484375	1.54035728757812e-09\\
-13.6901953125	1.39119378135376e-09\\
-13.66890625	1.26129468032441e-09\\
-13.6476171875	1.27144179531207e-09\\
-13.626328125	1.16483990557786e-09\\
-13.6050390625	1.15827502989058e-09\\
-13.58375	1.2810240207549e-09\\
-13.5624609375	1.28691403451166e-09\\
-13.541171875	1.29557894415388e-09\\
-13.5198828125	1.45207709859566e-09\\
-13.49859375	1.37777250249992e-09\\
-13.4773046875	1.44887387166486e-09\\
-13.456015625	1.31134718578778e-09\\
-13.4347265625	1.19497204763991e-09\\
-13.4134375	1.07351570851954e-09\\
-13.3921484375	8.51608078400382e-10\\
-13.370859375	9.07398500513931e-10\\
-13.3495703125	9.95222500099036e-10\\
-13.32828125	1.01125855826747e-09\\
-13.3069921875	1.08235230918919e-09\\
-13.285703125	1.17057722844064e-09\\
-13.2644140625	1.29450043773576e-09\\
-13.243125	1.11528288368321e-09\\
-13.2218359375	1.3360619597834e-09\\
-13.200546875	1.1869839189665e-09\\
-13.1792578125	1.10277826103214e-09\\
-13.15796875	9.64534122767765e-10\\
-13.1366796875	9.0879636965519e-10\\
-13.115390625	9.00037076043728e-10\\
-13.0941015625	9.86824829490466e-10\\
-13.0728125	8.97063426166857e-10\\
-13.0515234375	1.12613731523965e-09\\
-13.030234375	1.10832942116701e-09\\
-13.0089453125	9.86096140476228e-10\\
-12.98765625	1.05583197960548e-09\\
-12.9663671875	1.0385374185443e-09\\
-12.945078125	8.83762590303873e-10\\
-12.9237890625	7.40954789080543e-10\\
-12.9025	8.12208297370656e-10\\
-12.8812109375	7.89416720203116e-10\\
-12.859921875	8.73284017224412e-10\\
-12.8386328125	8.30793274142765e-10\\
-12.81734375	9.74483695280576e-10\\
-12.7960546875	9.59544079509013e-10\\
-12.774765625	8.41510169172842e-10\\
-12.7534765625	8.24452119737253e-10\\
-12.7321875	8.06552039942643e-10\\
-12.7108984375	6.71024889465116e-10\\
-12.689609375	6.30999383170854e-10\\
-12.6683203125	4.85455434168291e-10\\
-12.64703125	4.7001144854286e-10\\
-12.6257421875	4.12849518917471e-10\\
-12.604453125	5.27224665118736e-10\\
-12.5831640625	4.62952075776215e-10\\
-12.561875	4.94301521669986e-10\\
-12.5405859375	5.19164583047921e-10\\
-12.519296875	4.52047806598441e-10\\
-12.4980078125	5.32919631844338e-10\\
-12.47671875	3.8916876083947e-10\\
-12.4554296875	3.51015237531017e-10\\
-12.434140625	2.57968677155832e-10\\
-12.4128515625	2.47560149510863e-10\\
-12.3915625	2.32471059574361e-10\\
-12.3702734375	2.37440424700981e-10\\
-12.348984375	3.05460713177751e-10\\
-12.3276953125	3.56406513578338e-10\\
-12.30640625	3.4863741179588e-10\\
-12.2851171875	3.96587475942478e-10\\
-12.263828125	2.72746632879911e-10\\
-12.2425390625	2.61118649955198e-10\\
-12.22125	5.74167304686116e-11\\
-12.1999609375	8.09910725282302e-11\\
-12.178671875	-9.50730609928324e-11\\
-12.1573828125	-1.26696028161026e-10\\
-12.13609375	-1.67109935336043e-10\\
-12.1148046875	-2.32137004981735e-10\\
-12.093515625	-1.16727990638627e-10\\
-12.0722265625	-1.91592437503699e-10\\
-12.0509375	-2.6891541198367e-10\\
-12.0296484375	-9.61633241658282e-11\\
-12.008359375	-2.80514293668848e-10\\
-11.9870703125	-3.69659702142613e-10\\
-11.96578125	-4.49777163317829e-10\\
-11.9444921875	-5.9673963948595e-10\\
-11.923203125	-7.14939047516368e-10\\
-11.9019140625	-6.6957327283713e-10\\
-11.880625	-5.90828820344736e-10\\
-11.8593359375	-5.40464899972578e-10\\
-11.838046875	-4.67469616558548e-10\\
-11.8167578125	-4.66537165634387e-10\\
-11.79546875	-3.83850525823751e-10\\
-11.7741796875	-4.83602039659827e-10\\
-11.752890625	-4.77855648841048e-10\\
-11.7316015625	-5.68400177240783e-10\\
-11.7103125	-6.54371931241227e-10\\
-11.6890234375	-8.39736333038222e-10\\
-11.667734375	-7.79615067308655e-10\\
-11.6464453125	-8.53422007529106e-10\\
-11.62515625	-9.51538709675399e-10\\
-11.6038671875	-9.4663860652614e-10\\
-11.582578125	-8.56136135593051e-10\\
-11.5612890625	-8.65587487777361e-10\\
-11.54	-9.22045012297054e-10\\
-11.5187109375	-1.01477277618545e-09\\
-11.497421875	-1.06896827213918e-09\\
-11.4761328125	-1.01544663098236e-09\\
-11.45484375	-1.20230996447005e-09\\
-11.4335546875	-1.19994092317984e-09\\
-11.412265625	-1.12998700782687e-09\\
-11.3909765625	-1.09350258513733e-09\\
-11.3696875	-1.02287089890035e-09\\
-11.3483984375	-1.04861679132692e-09\\
-11.327109375	-9.52892299033259e-10\\
-11.3058203125	-1.04436366378303e-09\\
-11.28453125	-9.06562694147553e-10\\
-11.2632421875	-9.02263080860167e-10\\
-11.241953125	-1.0322694501937e-09\\
-11.2206640625	-9.9190482257896e-10\\
-11.199375	-9.05802239734824e-10\\
-11.1780859375	-1.00140893286167e-09\\
-11.156796875	-1.07569670544116e-09\\
-11.1355078125	-1.03304429063175e-09\\
-11.11421875	-1.13219987629285e-09\\
-11.0929296875	-1.0663656322418e-09\\
-11.071640625	-9.67368516642671e-10\\
-11.0503515625	-9.01328038957114e-10\\
-11.0290625	-9.60461784450193e-10\\
-11.0077734375	-9.62462844132425e-10\\
-10.986484375	-1.0154311959053e-09\\
-10.9651953125	-1.03978928177667e-09\\
-10.94390625	-1.11301911037109e-09\\
-10.9226171875	-1.09068072259365e-09\\
-10.901328125	-1.15358576189004e-09\\
-10.8800390625	-9.99608714322299e-10\\
-10.85875	-1.10367376669175e-09\\
-10.8374609375	-8.77877467117294e-10\\
-10.816171875	-8.00367146301521e-10\\
-10.7948828125	-8.43511529269509e-10\\
-10.77359375	-8.85954761876426e-10\\
-10.7523046875	-9.21205450637779e-10\\
-10.731015625	-8.7764808182915e-10\\
-10.7097265625	-1.0220017673185e-09\\
-10.6884375	-7.96382313984008e-10\\
-10.6671484375	-8.34706681508137e-10\\
-10.645859375	-7.87011849110382e-10\\
-10.6245703125	-6.84156954915035e-10\\
-10.60328125	-7.1871094895049e-10\\
-10.5819921875	-7.53534749780331e-10\\
-10.560703125	-7.64308475501884e-10\\
-10.5394140625	-8.55583068331273e-10\\
-10.518125	-8.04714699715145e-10\\
-10.4968359375	-6.95601258865437e-10\\
-10.475546875	-7.11943932018046e-10\\
-10.4542578125	-6.54370275643979e-10\\
-10.43296875	-5.4280336434626e-10\\
-10.4116796875	-5.50211483274911e-10\\
-10.390390625	-5.38335198332093e-10\\
-10.3691015625	-5.87934321084095e-10\\
-10.3478125	-5.29845302244768e-10\\
-10.3265234375	-6.156195816352e-10\\
-10.305234375	-6.98095238837546e-10\\
-10.2839453125	-6.81348937718506e-10\\
-10.26265625	-5.55950165735194e-10\\
-10.2413671875	-4.78918548819377e-10\\
-10.220078125	-4.65411264005737e-10\\
-10.1987890625	-3.47213973138648e-10\\
-10.1775	-3.16212804739505e-10\\
-10.1562109375	-3.07896154885029e-10\\
-10.134921875	-3.08257988019203e-10\\
-10.1136328125	-2.81577199175255e-10\\
-10.09234375	-2.95710018710462e-10\\
-10.0710546875	-2.89324952423032e-10\\
-10.049765625	-2.94020761112192e-10\\
-10.0284765625	-1.95134338461519e-10\\
-10.0071875	-2.28587580729578e-10\\
-9.9858984375	-1.19263376656248e-10\\
-9.96460937499999	-4.708074687751e-11\\
-9.9433203125	-1.26605887118457e-10\\
-9.92203125	2.18829502248947e-11\\
-9.9007421875	-1.01281268456865e-10\\
-9.879453125	-2.67149579854067e-11\\
-9.85816406249999	-1.61321050021085e-10\\
-9.836875	-1.42239832449291e-10\\
-9.8155859375	-5.55261723334784e-11\\
-9.794296875	-3.53944154464422e-11\\
-9.7730078125	5.90877857193006e-12\\
-9.75171874999999	6.9271836268622e-11\\
-9.7304296875	7.96759553602951e-11\\
-9.709140625	1.70031927953081e-10\\
-9.6878515625	2.20031059820131e-10\\
-9.6665625	2.09843445314674e-10\\
-9.64527343749999	2.26540117493316e-10\\
-9.623984375	2.99753354646698e-10\\
-9.6026953125	2.50952620062246e-10\\
-9.58140625	2.48581339152606e-10\\
-9.5601171875	3.77864418154425e-10\\
-9.53882812499999	3.83514788287047e-10\\
-9.5175390625	3.41477849912814e-10\\
-9.49625	5.19608071808227e-10\\
-9.4749609375	5.96462873491265e-10\\
-9.453671875	7.0772205676476e-10\\
-9.43238281249999	7.34004095132275e-10\\
-9.41109375	8.21124255475726e-10\\
-9.3898046875	7.01934548915513e-10\\
-9.368515625	8.58415522758201e-10\\
-9.3472265625	6.546796704195e-10\\
-9.32593749999999	7.14407927145915e-10\\
-9.3046484375	7.63525032737861e-10\\
-9.283359375	7.2251713465287e-10\\
-9.2620703125	8.966768941194e-10\\
-9.24078125	8.62068682874115e-10\\
-9.21949218749999	1.0056960222749e-09\\
-9.198203125	9.86892158429021e-10\\
-9.1769140625	9.73187157058341e-10\\
-9.155625	9.46473987281571e-10\\
-9.1343359375	1.01770586219465e-09\\
-9.11304687499999	9.96957282004654e-10\\
-9.0917578125	1.01022068233528e-09\\
-9.07046875	9.59197097061761e-10\\
-9.0491796875	9.98660120463684e-10\\
-9.027890625	1.0917205278365e-09\\
-9.00660156249999	1.1654109773781e-09\\
-8.9853125	1.1743268312076e-09\\
-8.9640234375	1.33215272062388e-09\\
-8.942734375	1.3665522703259e-09\\
-8.9214453125	1.30144420818832e-09\\
-8.90015624999999	1.22308098802571e-09\\
-8.8788671875	1.28306851846488e-09\\
-8.857578125	1.09381503406578e-09\\
-8.8362890625	1.0402038333158e-09\\
-8.815	1.06907066704501e-09\\
-8.79371093749999	9.92110038324565e-10\\
-8.772421875	1.06700084132557e-09\\
-8.7511328125	1.12254263820679e-09\\
-8.72984375	1.34283005935849e-09\\
-8.7085546875	1.3455945739815e-09\\
-8.68726562499999	1.37430230297746e-09\\
-8.6659765625	1.45422032160038e-09\\
-8.6446875	1.42913438195897e-09\\
-8.6233984375	1.23053775142471e-09\\
-8.602109375	1.31837094253211e-09\\
-8.58082031249999	1.08026429791279e-09\\
-8.55953125	1.07170314529461e-09\\
-8.5382421875	9.83966725321416e-10\\
-8.516953125	1.10732014744053e-09\\
-8.4956640625	1.04264762639724e-09\\
-8.47437499999999	1.09966290182847e-09\\
-8.4530859375	1.23794421104502e-09\\
-8.431796875	1.24379867393159e-09\\
-8.4105078125	1.27023854914256e-09\\
-8.38921875	1.20648049394504e-09\\
-8.36792968749999	1.13697225364788e-09\\
-8.346640625	9.58407150296346e-10\\
-8.3253515625	8.82232230547668e-10\\
-8.3040625	9.09930433421833e-10\\
-8.2827734375	1.03727437932942e-09\\
-8.26148437499999	1.06706898694184e-09\\
-8.2401953125	1.04046937550912e-09\\
-8.21890625	1.07995703954292e-09\\
-8.1976171875	1.1364360545858e-09\\
-8.176328125	9.03116668666717e-10\\
-8.15503906249999	1.04193881496791e-09\\
-8.13375	8.95682713393341e-10\\
-8.1124609375	7.45299812240618e-10\\
-8.091171875	7.08618176242894e-10\\
-8.0698828125	6.33276392485898e-10\\
-8.04859374999999	6.07683098456227e-10\\
-8.0273046875	6.92893009154731e-10\\
-8.006015625	7.52490209271775e-10\\
-7.9847265625	6.42352744151973e-10\\
-7.9634375	8.29838216873756e-10\\
-7.94214843749999	7.74530178303546e-10\\
-7.920859375	7.26234354127294e-10\\
-7.8995703125	7.28222190506501e-10\\
-7.87828125	6.74009110288232e-10\\
-7.8569921875	5.0304666557328e-10\\
-7.83570312499999	5.12039460470822e-10\\
-7.8144140625	4.5224860312784e-10\\
-7.793125	4.35334842826986e-10\\
-7.7718359375	4.93984652619434e-10\\
-7.750546875	4.73650145935852e-10\\
-7.72925781249999	4.43007304329691e-10\\
-7.70796875	5.11428982831449e-10\\
-7.6866796875	4.28145280761605e-10\\
-7.665390625	3.81173317978375e-10\\
-7.6441015625	3.34625911866268e-10\\
-7.62281249999999	3.15270456208015e-10\\
-7.6015234375	1.98598346823362e-10\\
-7.580234375	1.68815558846939e-10\\
-7.5589453125	1.36469748587617e-10\\
-7.53765625	6.2006717419707e-11\\
-7.51636718749999	7.92096997253999e-11\\
-7.495078125	1.82775307434544e-10\\
-7.4737890625	2.10796804558713e-10\\
-7.4525	1.59392328581519e-10\\
-7.4312109375	1.14201259425887e-10\\
-7.40992187499999	3.74212273027816e-11\\
-7.3886328125	4.65427188941287e-11\\
-7.36734375	-1.33705339717194e-10\\
-7.3460546875	-1.83580781737012e-10\\
-7.324765625	-1.78078204338814e-10\\
-7.30347656249999	-2.21549224496709e-10\\
-7.2821875	-2.48880900792317e-10\\
-7.2608984375	-9.62984369337344e-11\\
-7.239609375	-6.45067901968666e-11\\
-7.2183203125	-1.15516522514323e-10\\
-7.19703124999999	-1.66453071690114e-10\\
-7.1757421875	-1.83611904305096e-10\\
-7.154453125	-2.06013124616401e-10\\
-7.1331640625	-3.79531581021478e-10\\
-7.111875	-4.158364540135e-10\\
-7.09058593749999	-3.16389228105064e-10\\
-7.069296875	-2.97620543114324e-10\\
-7.0480078125	-2.78894640730302e-10\\
-7.02671875	-2.52145667141744e-10\\
-7.0054296875	-2.08734485806748e-10\\
-6.98414062499999	-3.70878752532857e-10\\
-6.9628515625	-4.79186181957664e-10\\
-6.9415625	-4.31857339840987e-10\\
-6.9202734375	-5.71595050621388e-10\\
-6.898984375	-4.89148056066812e-10\\
-6.87769531249999	-6.02074260331454e-10\\
-6.85640625	-5.46729459917458e-10\\
-6.8351171875	-3.4131600778061e-10\\
-6.813828125	-3.90986123168957e-10\\
-6.7925390625	-2.83047446935183e-10\\
-6.77124999999999	-2.2377689156715e-10\\
-6.7499609375	-3.60257539128449e-10\\
-6.728671875	-4.61496866675905e-10\\
-6.7073828125	-6.47527204942939e-10\\
-6.68609375	-6.62805002707684e-10\\
-6.66480468749999	-7.52104547388479e-10\\
-6.643515625	-7.73561690812686e-10\\
-6.6222265625	-7.35919259139098e-10\\
-6.6009375	-5.57664476713952e-10\\
-6.5796484375	-5.53192377512442e-10\\
-6.55835937499999	-5.6086999606289e-10\\
-6.5370703125	-6.04273059635795e-10\\
-6.51578125	-8.38613830535684e-10\\
-6.4944921875	-8.78523899744953e-10\\
-6.473203125	-1.00974753822389e-09\\
-6.45191406249999	-1.15812873253964e-09\\
-6.430625	-1.10495672032533e-09\\
-6.4093359375	-1.04066825150339e-09\\
-6.388046875	-8.6871489755663e-10\\
-6.3667578125	-7.47725877705578e-10\\
-6.34546874999999	-5.61994804148273e-10\\
-6.3241796875	-5.03044724517167e-10\\
-6.302890625	-5.41612644745638e-10\\
-6.2816015625	-6.71071903727668e-10\\
-6.2603125	-8.88566038916699e-10\\
-6.23902343749999	-9.94103785487125e-10\\
-6.217734375	-1.18106142214703e-09\\
-6.1964453125	-1.13643869245944e-09\\
-6.17515625	-1.2055836180405e-09\\
-6.1538671875	-1.11936020833967e-09\\
-6.13257812499999	-1.00666154733727e-09\\
-6.1112890625	-1.01306363243527e-09\\
-6.09	-9.98198886647924e-10\\
-6.0687109375	-9.65588510086379e-10\\
-6.047421875	-1.01082505489096e-09\\
-6.02613281249999	-1.18364540479081e-09\\
-6.00484375	-1.25113135101292e-09\\
-5.9835546875	-1.20371828882224e-09\\
-5.962265625	-1.28561735493354e-09\\
-5.9409765625	-1.28968452717737e-09\\
-5.91968749999999	-1.25606006991218e-09\\
-5.8983984375	-1.27274502817345e-09\\
-5.877109375	-1.22726639937821e-09\\
-5.8558203125	-1.25445653818715e-09\\
-5.83453125	-1.24014795241964e-09\\
-5.81324218749999	-1.08938011575619e-09\\
-5.791953125	-1.34749476222607e-09\\
-5.7706640625	-1.3336484882047e-09\\
-5.749375	-1.42948656049344e-09\\
-5.7280859375	-1.24358878111473e-09\\
-5.70679687499999	-1.19351848214461e-09\\
-5.6855078125	-1.13231211617192e-09\\
-5.66421875	-9.56425618812447e-10\\
-5.6429296875	-8.96542520317592e-10\\
-5.621640625	-7.5227788883321e-10\\
-5.60035156249999	-7.5466593301428e-10\\
-5.5790625	-7.73124676066316e-10\\
-5.5577734375	-8.45622348528341e-10\\
-5.536484375	-9.01285141082782e-10\\
-5.5151953125	-9.85011844909367e-10\\
-5.49390624999999	-9.78117767844926e-10\\
-5.4726171875	-9.75826412799458e-10\\
-5.451328125	-9.78059572321944e-10\\
-5.4300390625	-8.95921345541262e-10\\
-5.40875	-9.35143927832717e-10\\
-5.38746093749999	-8.53182848597567e-10\\
-5.366171875	-7.10687416376263e-10\\
-5.3448828125	-6.99608528459319e-10\\
-5.32359375	-7.21991510097816e-10\\
-5.3023046875	-7.37713721677392e-10\\
-5.28101562499999	-7.3731895837551e-10\\
-5.2597265625	-7.31593023497229e-10\\
-5.2384375	-7.75026170141391e-10\\
-5.2171484375	-7.38295495155563e-10\\
-5.195859375	-7.60051278405139e-10\\
-5.17457031249999	-6.24537356180218e-10\\
-5.15328125	-4.5688207042121e-10\\
-5.1319921875	-4.12308001134659e-10\\
-5.110703125	-3.69526544372217e-10\\
-5.0894140625	-2.79771648399612e-10\\
-5.06812499999999	-3.38472392094737e-10\\
-5.0468359375	-4.88806017596834e-10\\
-5.025546875	-5.49136529027866e-10\\
-5.0042578125	-5.04739463824323e-10\\
-4.98296875	-6.42322858505204e-10\\
-4.96167968749999	-4.85452214068252e-10\\
-4.940390625	-3.5952137779614e-10\\
-4.9191015625	-2.41450140991367e-10\\
-4.8978125	-1.11192649151217e-10\\
-4.8765234375	-1.10859498522826e-11\\
-4.85523437499999	1.29030148343892e-10\\
-4.8339453125	6.01669189946011e-11\\
-4.81265625	-6.4588709287777e-12\\
-4.7913671875	-3.98264572435302e-13\\
-4.770078125	-1.36098925160132e-10\\
-4.74878906249999	-1.47847725458109e-10\\
-4.7275	-4.08677116594368e-11\\
-4.7062109375	2.62145751771183e-11\\
-4.684921875	1.91433547011119e-10\\
-4.6636328125	1.64004351382855e-10\\
-4.64234374999999	3.72758263818626e-10\\
-4.6210546875	4.96162013567855e-10\\
-4.599765625	4.75706942570772e-10\\
-4.5784765625	5.21760214449851e-10\\
-4.5571875	5.12430944679519e-10\\
-4.53589843749999	4.7255347023517e-10\\
-4.514609375	4.3047455821075e-10\\
-4.4933203125	4.09905657803013e-10\\
-4.47203125	5.01283477740528e-10\\
-4.4507421875	6.11678472265368e-10\\
-4.42945312499999	5.81554313249846e-10\\
-4.4081640625	7.79160226218718e-10\\
-4.386875	7.54343667578448e-10\\
-4.3655859375	7.48285366170112e-10\\
-4.344296875	6.68303794712086e-10\\
-4.32300781249999	6.08761138837556e-10\\
-4.30171875	6.07737035963235e-10\\
-4.2804296875	5.05766118269159e-10\\
-4.259140625	5.32416531246934e-10\\
-4.2378515625	6.58639680241238e-10\\
-4.21656249999999	7.78839971238476e-10\\
-4.1952734375	8.18711500249374e-10\\
-4.173984375	9.44236555142841e-10\\
-4.1526953125	1.06908449258044e-09\\
-4.13140625	1.07373185074823e-09\\
-4.11011718749999	8.95067684170688e-10\\
-4.088828125	9.24140768517609e-10\\
-4.0675390625	8.7092752799082e-10\\
-4.04625	8.12712787935434e-10\\
-4.0249609375	9.82708758543256e-10\\
-4.00367187499999	1.20626569743739e-09\\
-3.9823828125	1.17969841487679e-09\\
-3.96109375	1.22942609591602e-09\\
-3.9398046875	1.41317260171039e-09\\
-3.918515625	1.47573472242442e-09\\
-3.89722656249999	1.48772665039882e-09\\
-3.8759375	1.40964027536141e-09\\
-3.8546484375	1.51276837376369e-09\\
-3.833359375	1.37460725600085e-09\\
-3.8120703125	1.44842040956903e-09\\
-3.79078124999999	1.46638879780688e-09\\
-3.7694921875	1.55282210579425e-09\\
-3.748203125	1.67749721139723e-09\\
-3.7269140625	1.74050767749176e-09\\
-3.705625	1.80366579050176e-09\\
-3.68433593749999	1.81787642799359e-09\\
-3.663046875	1.82848909654935e-09\\
-3.6417578125	1.81135052921969e-09\\
-3.62046875	1.82777400496118e-09\\
-3.5991796875	1.80522328865544e-09\\
-3.57789062499999	1.82253793514363e-09\\
-3.5566015625	1.76457933262651e-09\\
-3.5353125	1.92235819567355e-09\\
-3.5140234375	1.92130572126463e-09\\
-3.492734375	1.84983695211217e-09\\
-3.47144531249999	1.8155467434936e-09\\
-3.45015625	1.81054987631068e-09\\
-3.4288671875	1.67928753054351e-09\\
-3.407578125	1.82502739831935e-09\\
-3.3862890625	1.85645066111662e-09\\
-3.36499999999999	1.80840880883684e-09\\
-3.3437109375	1.83596101070333e-09\\
-3.322421875	1.84347509099767e-09\\
-3.3011328125	1.70038034028087e-09\\
-3.27984375	1.71320953102651e-09\\
-3.25855468749999	1.63335303398749e-09\\
-3.237265625	1.54608292981106e-09\\
-3.2159765625	1.58375923586625e-09\\
-3.1946875	1.59847060247276e-09\\
-3.1733984375	1.57003543883165e-09\\
-3.15210937499999	1.45915586815736e-09\\
-3.1308203125	1.42694676355096e-09\\
-3.10953125	1.33261748944926e-09\\
-3.0882421875	1.53445571176724e-09\\
-3.066953125	1.51677998029302e-09\\
-3.04566406249999	1.41259563088024e-09\\
-3.024375	1.64008051278615e-09\\
-3.0030859375	1.62471315235662e-09\\
-2.981796875	1.6362651386774e-09\\
-2.9605078125	1.56047221570432e-09\\
-2.93921874999999	1.61312944515454e-09\\
-2.9179296875	1.39532540024809e-09\\
-2.896640625	1.50698856946656e-09\\
-2.8753515625	1.39046716204569e-09\\
-2.8540625	1.59196439087534e-09\\
-2.83277343749999	1.64678806192468e-09\\
-2.811484375	1.61578790536579e-09\\
-2.7901953125	1.80008287164458e-09\\
-2.76890625	1.7124212969284e-09\\
-2.7476171875	1.82047708374293e-09\\
-2.72632812499999	1.74460570991845e-09\\
-2.7050390625	1.72234183215769e-09\\
-2.68375	1.76027769183256e-09\\
-2.6624609375	1.5173425341708e-09\\
-2.641171875	1.63179526124613e-09\\
-2.61988281249999	1.63145531884013e-09\\
-2.59859375	1.76325321483849e-09\\
-2.5773046875	1.5912038236953e-09\\
-2.556015625	1.59186551761003e-09\\
-2.5347265625	1.66014436423286e-09\\
-2.51343749999999	1.35448621862187e-09\\
-2.4921484375	1.29220971808578e-09\\
-2.470859375	1.25769625538306e-09\\
-2.4495703125	1.14438704016381e-09\\
-2.42828125	1.02805079183181e-09\\
-2.40699218749999	1.11519596576804e-09\\
-2.385703125	1.09721679986882e-09\\
-2.3644140625	1.07482689601558e-09\\
-2.343125	1.05269755237097e-09\\
-2.3218359375	1.14500598851407e-09\\
-2.30054687499999	8.89444208290613e-10\\
-2.2792578125	9.04136981636834e-10\\
-2.25796875	7.57399556259211e-10\\
-2.2366796875	7.20445733061443e-10\\
-2.215390625	6.4187092544335e-10\\
-2.19410156249999	6.06264154191585e-10\\
-2.1728125	7.24022710915097e-10\\
-2.1515234375	5.04739413512345e-10\\
-2.130234375	5.25487609936641e-10\\
-2.1089453125	3.04151914339804e-10\\
-2.08765624999999	2.5148740005559e-10\\
-2.0663671875	1.24472993395214e-10\\
-2.045078125	2.70049146839023e-10\\
-2.0237890625	1.69451570349131e-10\\
-2.0025	2.86431793914573e-10\\
-1.98121093749999	3.98766203843643e-10\\
-1.959921875	3.51615922296071e-10\\
-1.9386328125	4.02155793445059e-10\\
-1.91734375	2.74701538019859e-10\\
-1.8960546875	3.77165053055109e-10\\
-1.87476562499999	1.17456835445735e-10\\
-1.8534765625	4.66058550116955e-11\\
-1.8321875	2.19269432787142e-11\\
-1.8108984375	1.27027489644534e-10\\
-1.789609375	1.68238297199375e-12\\
-1.76832031249999	2.54989033965589e-10\\
-1.74703125	2.18418038549861e-10\\
-1.7257421875	2.86707082179468e-10\\
-1.704453125	3.04931070655957e-10\\
-1.6831640625	3.88648015437645e-10\\
-1.66187499999999	2.10823308412001e-10\\
-1.6405859375	3.0156152586071e-10\\
-1.619296875	3.76862682954422e-10\\
-1.5980078125	2.94036397118999e-10\\
-1.57671875	7.49721327232663e-11\\
-1.55542968749999	1.28078056180863e-10\\
-1.534140625	1.96037266229951e-11\\
-1.5128515625	-1.58143912293171e-10\\
-1.4915625	-1.77134503439737e-10\\
-1.4702734375	-1.30706431539197e-10\\
-1.44898437499999	-2.84906750758388e-10\\
-1.4276953125	-3.90908580228779e-10\\
-1.40640625	-3.70797148387089e-10\\
-1.3851171875	-4.35142625805482e-10\\
-1.363828125	-5.36196756873695e-10\\
-1.34253906249999	-6.46180041700959e-10\\
-1.32125	-5.39591237658471e-10\\
-1.2999609375	-4.10755812706336e-10\\
-1.278671875	-4.11612632896822e-10\\
-1.2573828125	-3.79083442374607e-10\\
-1.23609374999999	-4.40143650522839e-10\\
-1.2148046875	-4.77930720408191e-10\\
-1.193515625	-7.49884369736789e-10\\
-1.1722265625	-7.48255125744435e-10\\
-1.1509375	-7.66771126404274e-10\\
-1.12964843749999	-7.88777872501105e-10\\
-1.108359375	-7.55898279431772e-10\\
-1.0870703125	-6.46670205893926e-10\\
-1.06578125	-4.4990299811293e-10\\
-1.0444921875	-4.1095903594589e-10\\
-1.02320312499999	-3.37435464665737e-10\\
-1.0019140625	-2.39174513231379e-10\\
-0.980624999999996	-3.30527330574838e-10\\
-0.959335937500001	-5.54773079093903e-10\\
-0.938046874999998	-4.92404169640213e-10\\
-0.916757812499995	-5.69886091003184e-10\\
-0.895468749999999	-6.41932945537864e-10\\
-0.874179687499996	-5.76633946340881e-10\\
-0.852890625000001	-3.79495222657904e-10\\
-0.831601562499998	-4.18223504751345e-10\\
-0.810312499999995	-1.46855279031556e-10\\
-0.789023437499999	-5.97016422421557e-11\\
-0.767734374999996	-5.51386300252021e-12\\
-0.746445312500001	-5.61750021611966e-11\\
-0.725156249999998	-1.53838144047269e-10\\
-0.703867187499995	-3.1759181609178e-10\\
-0.682578124999999	-3.41295545809179e-10\\
-0.661289062499996	-2.0557455308315e-10\\
-0.640000000000001	-2.19439532172439e-10\\
-0.618710937499998	-1.63730074239787e-11\\
-0.597421874999995	4.79476671144098e-11\\
-0.576132812499999	3.72067332603516e-11\\
-0.554843749999996	3.5057746043647e-10\\
-0.533554687500001	2.39498896397493e-10\\
-0.512265624999998	1.21576881381203e-10\\
-0.490976562499995	2.18557298134623e-10\\
-0.469687499999999	-8.05749056171173e-11\\
-0.448398437499996	-1.8175532834768e-10\\
-0.427109375000001	-1.31744212369158e-10\\
-0.405820312499998	-9.11427911946386e-11\\
-0.384531249999995	-6.6191713147101e-11\\
-0.363242187499999	1.54681888096111e-10\\
-0.341953124999996	2.63247716583206e-10\\
-0.320664062500001	4.04273530079893e-10\\
-0.299374999999998	3.85339904474149e-10\\
-0.278085937499995	4.87174092953813e-10\\
-0.256796874999999	4.11006023272987e-10\\
-0.235507812499996	2.35852266929631e-10\\
-0.214218750000001	1.98243326338886e-10\\
-0.192929687499998	7.84639496073747e-11\\
-0.171640624999995	6.59663383809524e-11\\
-0.150351562499999	2.10802606522308e-10\\
-0.129062499999996	3.0716020747084e-10\\
-0.107773437500001	7.87135795132734e-10\\
-0.0864843749999977	8.23785227908103e-10\\
-0.0651953124999949	1.13204904851568e-09\\
-0.0439062499999991	1.25043783179802e-09\\
-0.0226171874999963	1.24680578329839e-09\\
-0.00132812500000057	1.22364818061891e-09\\
0.0199609375000023	1.31322221586539e-09\\
0.0412500000000051	1.20717427573132e-09\\
0.0625390625000009	1.34239718823273e-09\\
0.0838281250000037	1.44338979512521e-09\\
0.105117187499999	1.6058598812599e-09\\
0.126406250000002	1.78608277576464e-09\\
0.147695312500005	2.02534827384568e-09\\
0.168984375000001	2.14964312083621e-09\\
0.190273437500004	2.29555083800449e-09\\
0.211562499999999	2.30426139963212e-09\\
0.232851562500002	2.2201616659809e-09\\
0.254140625000005	2.41306697852907e-09\\
0.275429687500001	2.41760151515823e-09\\
0.296718750000004	2.32763740065648e-09\\
0.318007812499999	2.46309457515716e-09\\
0.339296875000002	2.72880665693602e-09\\
0.360585937500005	2.88304234956547e-09\\
0.381875000000001	2.97982791042881e-09\\
0.403164062500004	3.29929012690009e-09\\
0.424453124999999	3.36781096858981e-09\\
0.445742187500002	3.47348717724293e-09\\
0.467031250000005	3.63488067967484e-09\\
0.488320312500001	3.69032821369129e-09\\
0.509609375000004	3.56559011477187e-09\\
0.530898437499999	3.61910686847574e-09\\
0.552187500000002	3.71236570987807e-09\\
0.573476562500005	3.94124248789241e-09\\
0.594765625000001	3.9019214223769e-09\\
0.616054687500004	4.06506088760242e-09\\
0.637343749999999	4.31312403524096e-09\\
0.658632812500002	4.46986668936171e-09\\
0.679921875000005	4.6324149230293e-09\\
0.701210937500001	4.82680532065087e-09\\
0.722500000000004	4.82658878019181e-09\\
0.743789062499999	4.66641386637561e-09\\
0.765078125000002	4.4848390749276e-09\\
0.786367187500005	4.55290940724249e-09\\
0.807656250000001	4.72618701910768e-09\\
0.828945312500004	4.82657567705054e-09\\
0.850234374999999	5.11270500750173e-09\\
0.871523437500002	5.18292573527442e-09\\
0.892812500000005	5.47421659399586e-09\\
0.914101562500001	5.64006034652268e-09\\
0.935390625000004	5.64394464282397e-09\\
0.956679687499999	5.70326075473248e-09\\
0.977968750000002	5.77680802598876e-09\\
0.999257812500005	5.83323992898653e-09\\
1.020546875	5.81981682145955e-09\\
1.0418359375	5.97079247065765e-09\\
1.063125	6.29083630276289e-09\\
1.0844140625	6.46840896317255e-09\\
1.10570312500001	6.72008605942225e-09\\
1.1269921875	6.90582015745282e-09\\
1.14828125	7.16970482893648e-09\\
1.1695703125	7.2600721703507e-09\\
1.190859375	7.26727874275482e-09\\
1.21214843750001	7.34343491453929e-09\\
1.2334375	7.4105640094491e-09\\
1.2547265625	7.29528494272607e-09\\
1.276015625	7.48241581598291e-09\\
1.2973046875	7.74538559906703e-09\\
1.31859375000001	8.05463235197611e-09\\
1.3398828125	8.25037717748588e-09\\
1.361171875	8.45975433337933e-09\\
1.3824609375	8.72851146788335e-09\\
1.40375	8.62670642827232e-09\\
1.42503906250001	8.61791774808329e-09\\
1.446328125	8.54520893820689e-09\\
1.4676171875	8.30175911931206e-09\\
1.48890625	8.35902382477191e-09\\
1.5101953125	8.47081582456608e-09\\
1.53148437500001	8.64767183125331e-09\\
1.5527734375	8.92017041660261e-09\\
1.5740625	9.08733867592781e-09\\
1.5953515625	9.14732322445159e-09\\
1.616640625	9.28042380879954e-09\\
1.63792968750001	9.26058682391058e-09\\
1.65921875	9.16518857164958e-09\\
1.6805078125	9.01695949948663e-09\\
1.701796875	9.03245408463458e-09\\
1.7230859375	8.99395293099547e-09\\
1.74437500000001	8.99478966804249e-09\\
1.7656640625	9.05146869805717e-09\\
1.786953125	9.41235450264718e-09\\
1.8082421875	9.4213581131748e-09\\
1.82953125	9.6253565410628e-09\\
1.85082031250001	9.79145228248793e-09\\
1.872109375	9.89819949254585e-09\\
1.8933984375	9.79931001876697e-09\\
1.9146875	9.87903048173558e-09\\
1.9359765625	9.84301019891909e-09\\
1.95726562500001	9.95395224794586e-09\\
1.9785546875	9.94588695728193e-09\\
1.99984375	1.01469068298287e-08\\
2.0211328125	1.04267687609398e-08\\
2.042421875	1.0584665280597e-08\\
2.06371093750001	1.05871646665979e-08\\
2.085	1.07624940924368e-08\\
2.1062890625	1.06572234919156e-08\\
2.127578125	1.0604533983569e-08\\
2.1488671875	1.06850961624546e-08\\
2.17015625000001	1.06343533402357e-08\\
2.1914453125	1.04825565268033e-08\\
2.212734375	1.04062400396568e-08\\
2.2340234375	1.04157314835442e-08\\
2.2553125	1.03246354234297e-08\\
2.27660156250001	1.03984454207877e-08\\
2.297890625	1.04143399003729e-08\\
2.3191796875	1.0569346746494e-08\\
2.34046875	1.0482371185402e-08\\
2.3617578125	1.04383808972258e-08\\
2.38304687500001	1.04748680949448e-08\\
2.4043359375	1.02527732446447e-08\\
2.425625	1.01172522674915e-08\\
2.4469140625	9.95521172563912e-09\\
2.468203125	9.86700396624595e-09\\
2.48949218750001	9.58757134102981e-09\\
2.51078125	9.54026951309969e-09\\
2.5320703125	9.44495788757271e-09\\
2.553359375	9.45362443484199e-09\\
2.5746484375	9.33943813199935e-09\\
2.59593750000001	9.35025992458235e-09\\
2.6172265625	9.30338038581899e-09\\
2.638515625	9.02906111901811e-09\\
2.6598046875	8.76442854148943e-09\\
2.68109375	8.54493790367713e-09\\
2.70238281250001	8.30707901189641e-09\\
2.723671875	8.24224412347915e-09\\
2.7449609375	8.20242366168809e-09\\
2.76625	8.12132429794035e-09\\
2.7875390625	8.33985094216705e-09\\
2.80882812500001	8.39291845448158e-09\\
2.8301171875	8.35494668485552e-09\\
2.85140625	8.40438942885432e-09\\
2.8726953125	8.28028604620518e-09\\
2.893984375	7.96188057347489e-09\\
2.91527343750001	7.74932627493004e-09\\
2.9365625	7.77577327423483e-09\\
2.9578515625	7.32953412251082e-09\\
2.979140625	7.48016842852473e-09\\
3.0004296875	7.33419665272864e-09\\
3.02171875000001	7.49055699883374e-09\\
3.0430078125	7.5296784105918e-09\\
3.064296875	7.46861482994417e-09\\
3.0855859375	7.41558292803734e-09\\
3.106875	7.41744817491438e-09\\
3.12816406250001	7.03092029309568e-09\\
3.149453125	7.07824132423564e-09\\
3.1707421875	6.70490600999289e-09\\
3.19203125	6.65170501046307e-09\\
3.2133203125	6.56911451097396e-09\\
3.23460937500001	6.53678171450089e-09\\
3.2558984375	6.64379464822998e-09\\
3.2771875	6.5068803602224e-09\\
3.2984765625	6.53558626446122e-09\\
3.319765625	6.44339108324199e-09\\
3.34105468750001	6.20247200408768e-09\\
3.36234375	6.10825601849275e-09\\
3.3836328125	5.91334048614688e-09\\
3.404921875	5.85825347422229e-09\\
3.4262109375	5.77802899495929e-09\\
3.44750000000001	5.82273198706913e-09\\
3.4687890625	5.95139808820145e-09\\
3.490078125	5.85650001006726e-09\\
3.5113671875	5.75702136989376e-09\\
3.53265625	5.64945592294988e-09\\
3.55394531250001	5.30164979005266e-09\\
3.575234375	5.18086560426371e-09\\
3.5965234375	4.86058938454839e-09\\
3.6178125	4.60027205656275e-09\\
3.6391015625	4.68522006452926e-09\\
3.66039062500001	4.61654536493522e-09\\
3.6816796875	4.53266575224511e-09\\
3.70296875	4.6704453483534e-09\\
3.7242578125	4.53061619357131e-09\\
3.745546875	4.49900213620132e-09\\
3.76683593750001	4.31178583537266e-09\\
3.788125	4.24509154734172e-09\\
3.8094140625	4.05317419685669e-09\\
3.830703125	3.86811753137983e-09\\
3.8519921875	3.95488005233078e-09\\
3.87328125000001	3.8637486416036e-09\\
3.8945703125	3.7833215736707e-09\\
3.915859375	3.77241751332764e-09\\
3.9371484375	3.70705397859944e-09\\
3.9584375	3.87271227331049e-09\\
3.97972656250001	3.7023157587295e-09\\
4.001015625	3.63462691371873e-09\\
4.0223046875	3.59221604284941e-09\\
4.04359375	3.34465082393283e-09\\
4.0648828125	3.30326464206178e-09\\
4.08617187500001	3.24722852521653e-09\\
4.1074609375	3.18707900229719e-09\\
4.12875	3.01914280563568e-09\\
4.1500390625	3.11661462875558e-09\\
4.171328125	2.99546361499426e-09\\
4.19261718750001	2.93016204333544e-09\\
4.21390625	2.85484937216706e-09\\
4.2351953125	2.84984256944474e-09\\
4.256484375	2.80043536034751e-09\\
4.2777734375	2.63517126088383e-09\\
4.29906250000001	2.80161672804547e-09\\
4.3203515625	2.59516335081181e-09\\
4.341640625	2.34603690116168e-09\\
4.3629296875	2.36975894903329e-09\\
4.38421875	2.33289659389552e-09\\
4.40550781250001	2.3133776766102e-09\\
4.426796875	2.31816721606215e-09\\
4.4480859375	2.31749361140023e-09\\
4.469375	2.24467704596003e-09\\
4.4906640625	2.15978521452495e-09\\
4.51195312500001	2.15875566008327e-09\\
4.5332421875	1.98153472035361e-09\\
4.55453125	2.03291532509828e-09\\
4.5758203125	1.99461752854768e-09\\
4.597109375	1.78551016896533e-09\\
4.61839843750001	1.80652480242999e-09\\
4.6396875	1.89150564739088e-09\\
4.6609765625	1.7919114917216e-09\\
4.682265625	1.85966132236782e-09\\
4.7035546875	1.82790496667996e-09\\
4.72484375000001	1.73799733816931e-09\\
4.7461328125	1.63713054439346e-09\\
4.767421875	1.52779485745923e-09\\
4.7887109375	1.47980405613648e-09\\
4.81	1.55439724957492e-09\\
4.83128906250001	1.53705195002111e-09\\
4.852578125	1.61483487241959e-09\\
4.8738671875	1.63495614466983e-09\\
4.89515625	1.5877731582875e-09\\
4.9164453125	1.65901796107733e-09\\
4.93773437500001	1.52037264697695e-09\\
4.9590234375	1.40454599443869e-09\\
4.9803125	1.2501001006808e-09\\
5.0016015625	1.13931343726869e-09\\
5.022890625	9.6193471725368e-10\\
5.04417968750001	1.00585108871357e-09\\
5.06546875	1.004894131504e-09\\
5.0867578125	8.44863958703184e-10\\
5.108046875	9.28326048434986e-10\\
5.1293359375	8.67566081856928e-10\\
5.15062500000001	8.42037094220152e-10\\
5.1719140625	9.50640477944436e-10\\
5.193203125	8.17698811626887e-10\\
5.2144921875	7.74263860927498e-10\\
5.23578125	6.32252976488371e-10\\
5.25707031250001	5.3871930250551e-10\\
5.278359375	3.74721739694766e-10\\
5.2996484375	2.9391657194705e-10\\
5.3209375	3.35606356667371e-10\\
5.3422265625	3.83343900370114e-10\\
5.36351562500001	3.68221271585108e-10\\
5.3848046875	3.59298303736028e-10\\
5.40609375	3.23262010265857e-10\\
5.4273828125	3.04520240731312e-10\\
5.448671875	1.27415766131478e-10\\
5.46996093750001	1.91717033721582e-10\\
5.49125	2.05202436217875e-10\\
5.5125390625	1.10503548150688e-10\\
5.533828125	1.38692149376063e-10\\
5.5551171875	3.98168325941102e-11\\
5.57640625000001	2.38227119624034e-11\\
5.5976953125	1.52701171612431e-11\\
5.618984375	5.19430374804421e-13\\
5.6402734375	-7.78044475236458e-12\\
5.6615625	-6.13925006320952e-11\\
5.68285156250001	-1.95432512601751e-10\\
5.704140625	-1.59019379074968e-10\\
5.7254296875	-1.3367844151617e-10\\
5.74671875	-2.00702124104351e-10\\
5.7680078125	-2.3022099502824e-10\\
5.78929687500001	-2.09913400526756e-10\\
5.8105859375	-3.32776440333848e-10\\
5.831875	-3.19188434058311e-10\\
5.8531640625	-3.64471034393461e-10\\
5.874453125	-3.00796851149608e-10\\
5.89574218750001	-3.23800491892953e-10\\
5.91703125	-3.01550765988131e-10\\
5.9383203125	-3.31676386394291e-10\\
5.959609375	-4.83865075918597e-10\\
5.9808984375	-5.96262007339958e-10\\
6.00218750000001	-5.96852642757413e-10\\
6.0234765625	-5.72286023186325e-10\\
6.044765625	-5.50868393498955e-10\\
6.0660546875	-5.35528010063086e-10\\
6.08734375	-5.33378267549645e-10\\
6.10863281250001	-3.46711973757412e-10\\
6.129921875	-5.09688839969489e-10\\
6.1512109375	-5.55296992495807e-10\\
6.1725	-5.60986532141015e-10\\
6.1937890625	-5.47741059102542e-10\\
6.21507812500001	-6.78122278188669e-10\\
6.2363671875	-6.12211021424814e-10\\
6.25765625	-6.52090078220101e-10\\
6.2789453125	-6.10010232471506e-10\\
6.300234375	-6.64963945546327e-10\\
6.32152343750001	-6.60666448138612e-10\\
6.3428125	-6.44782697608175e-10\\
6.3641015625	-5.65901674722992e-10\\
6.385390625	-6.51673802956901e-10\\
6.4066796875	-6.10034008830768e-10\\
6.42796875000001	-7.32289154389893e-10\\
6.4492578125	-7.72404658616114e-10\\
6.470546875	-7.72604806929654e-10\\
6.4918359375	-7.64651079526152e-10\\
6.513125	-8.28254095530477e-10\\
6.53441406250001	-8.71016212640934e-10\\
6.555703125	-7.72491751531994e-10\\
6.5769921875	-8.89192925983801e-10\\
6.59828125	-8.17020951652923e-10\\
6.6195703125	-8.07964277397114e-10\\
6.64085937500001	-9.59075012370451e-10\\
6.6621484375	-9.48967236032755e-10\\
6.6834375	-1.00778868605194e-09\\
6.7047265625	-1.04519899978012e-09\\
6.726015625	-9.83808761755933e-10\\
6.74730468750001	-1.13170410140301e-09\\
6.76859375	-9.77302990975808e-10\\
6.7898828125	-9.25144138508468e-10\\
6.811171875	-1.17848963652442e-09\\
6.8324609375	-9.99833368720122e-10\\
6.85375000000001	-9.09441819615379e-10\\
6.8750390625	-9.79266366707477e-10\\
6.896328125	-1.00586324188137e-09\\
6.9176171875	-1.03198920549771e-09\\
6.93890625	-1.05199401708359e-09\\
6.96019531250001	-1.15956327183832e-09\\
6.981484375	-1.27708788603633e-09\\
7.0027734375	-1.20074580033133e-09\\
7.0240625	-1.26023707361009e-09\\
7.0453515625	-1.25216842459278e-09\\
7.06664062500001	-1.18725735923099e-09\\
7.0879296875	-1.08158875806369e-09\\
7.10921875	-1.00525246300693e-09\\
7.1305078125	-1.03541416306988e-09\\
7.151796875	-1.11197645690686e-09\\
7.17308593750001	-1.10338191693558e-09\\
7.194375	-1.25195850912675e-09\\
7.2156640625	-1.40260956321932e-09\\
7.236953125	-1.34450663872473e-09\\
7.2582421875	-1.42391925206937e-09\\
7.27953125000001	-1.48353353988998e-09\\
7.3008203125	-1.32441729094526e-09\\
7.322109375	-1.20716472406604e-09\\
7.3433984375	-1.20679851657713e-09\\
7.3646875	-1.2011439866436e-09\\
7.38597656250001	-1.28361290830275e-09\\
7.407265625	-1.23553275927447e-09\\
7.4285546875	-1.25122739491835e-09\\
7.44984375	-1.33917567613268e-09\\
7.4711328125	-1.35383668838628e-09\\
7.49242187500001	-1.32048747422829e-09\\
7.5137109375	-1.45619019250694e-09\\
7.535	-1.42229620632129e-09\\
7.5562890625	-1.3262819328894e-09\\
7.577578125	-1.28344211649741e-09\\
7.59886718750001	-1.29726308411989e-09\\
7.62015625	-1.30566040073466e-09\\
7.6414453125	-1.26067836341525e-09\\
7.662734375	-1.22399219660496e-09\\
7.6840234375	-1.23872333979874e-09\\
7.70531250000001	-1.15608027200378e-09\\
7.7266015625	-1.21447163707919e-09\\
7.747890625	-1.16009258474003e-09\\
7.7691796875	-1.24648568913233e-09\\
7.79046875	-1.17419222394484e-09\\
7.81175781250001	-1.16287236824621e-09\\
7.833046875	-1.23047184191076e-09\\
7.8543359375	-1.13302458286421e-09\\
7.875625	-1.14369209161942e-09\\
7.8969140625	-1.15493293897555e-09\\
7.91820312500001	-1.13442967614157e-09\\
7.9394921875	-1.00060899266691e-09\\
7.96078125	-9.55093000552412e-10\\
7.9820703125	-9.83671737571411e-10\\
8.003359375	-1.05004314417157e-09\\
8.02464843750001	-1.07111974567176e-09\\
8.0459375	-1.08114041612568e-09\\
8.0672265625	-1.15630926314727e-09\\
8.088515625	-1.01929625972762e-09\\
8.1098046875	-1.1150995092899e-09\\
8.13109375000001	-9.69857070813847e-10\\
8.1523828125	-1.02171216132267e-09\\
8.173671875	-8.38304531083713e-10\\
8.1949609375	-8.97374811709888e-10\\
8.21625	-8.82883931975991e-10\\
8.23753906250001	-7.85540250903798e-10\\
8.258828125	-8.46324996395935e-10\\
8.2801171875	-8.71568820069388e-10\\
8.30140625	-8.46725475868304e-10\\
8.3226953125	-7.85731789499805e-10\\
8.34398437500001	-7.49575838136859e-10\\
8.3652734375	-7.17837161086142e-10\\
8.3865625	-5.64562381947194e-10\\
8.4078515625	-4.17950115757459e-10\\
8.429140625	-4.93480808703649e-10\\
8.45042968750001	-3.12196055080307e-10\\
8.47171875	-2.98788498210051e-10\\
8.4930078125	-3.61198916226366e-10\\
8.514296875	-3.41516516364186e-10\\
8.5355859375	-4.31092477999344e-10\\
8.55687500000001	-4.71117528319969e-10\\
8.5781640625	-5.92345524167907e-10\\
8.599453125	-5.6725003536521e-10\\
8.6207421875	-4.91407390511621e-10\\
8.64203125	-4.503233610321e-10\\
8.66332031250001	-3.46066451649242e-10\\
8.684609375	-3.55532756289861e-10\\
8.7058984375	-2.33921829907511e-10\\
8.7271875	-1.87690351380962e-10\\
8.7484765625	-2.23538042350144e-10\\
8.76976562500001	-2.16679415298367e-10\\
8.7910546875	-2.91009626060774e-10\\
8.81234375	-1.99482945114489e-10\\
8.8336328125	-1.44832149444455e-10\\
8.854921875	-1.30005212982793e-10\\
8.87621093750001	-1.08155321407208e-10\\
8.8975	-1.2178385913095e-10\\
8.9187890625	-6.26677252611206e-11\\
8.940078125	-3.10086831814846e-11\\
8.9613671875	-8.38469712704667e-11\\
8.98265625000001	-1.19321597428893e-10\\
9.0039453125	-2.89455975102947e-10\\
9.025234375	-3.08736393260528e-10\\
9.0465234375	-3.83848152828894e-10\\
9.0678125	-3.80050191373888e-10\\
9.08910156250001	-2.70429990510156e-10\\
9.110390625	-3.33833854396713e-10\\
9.1316796875	-2.44886105925507e-10\\
9.15296875	-1.61197176878186e-10\\
9.1742578125	-6.47649576340738e-11\\
9.19554687500001	-3.70058612031352e-11\\
9.2168359375	4.14679926200903e-12\\
9.238125	-8.84850702405015e-11\\
9.2594140625	-1.90862439841837e-10\\
9.280703125	-1.97549566526562e-10\\
9.30199218750001	-2.71982612802796e-10\\
9.32328125	-1.81280290964877e-10\\
9.3445703125	-1.73279195268677e-10\\
9.365859375	-8.00814827768084e-11\\
9.3871484375	-1.41083875027049e-10\\
9.40843750000001	-8.24106487094064e-11\\
9.4297265625	3.05722622132675e-11\\
9.451015625	-7.71109665246187e-11\\
9.4723046875	-1.37451717250184e-11\\
9.49359375	1.29995259299e-10\\
9.51488281250001	9.27091555652741e-11\\
9.536171875	1.1640384874491e-10\\
9.5574609375	-4.27957241501612e-11\\
9.57875	3.7125928071636e-11\\
9.6000390625	-5.67480950532055e-11\\
9.62132812500001	-1.89091223506858e-10\\
9.6426171875	-4.19417820673361e-12\\
9.66390625	-1.22294218376806e-10\\
9.6851953125	-5.07047885764507e-11\\
9.706484375	-6.07055069492344e-11\\
9.72777343750001	4.53425284341584e-13\\
9.7490625	-5.26022334838168e-13\\
9.7703515625	-4.28968991962944e-11\\
9.791640625	2.57346065052814e-11\\
9.8129296875	6.37634015993814e-12\\
9.83421875000001	3.38890389655368e-12\\
9.8555078125	6.59316779514623e-11\\
9.876796875	-2.78948397782764e-11\\
9.8980859375	8.08534663933094e-11\\
9.919375	1.07909609695429e-10\\
9.94066406250001	6.58049051503751e-11\\
9.961953125	9.09305095974248e-11\\
9.9832421875	7.6868186549333e-11\\
10.00453125	1.35302861250249e-10\\
10.0258203125	1.70650318403028e-10\\
10.047109375	1.25713566449366e-10\\
10.0683984375	1.20374162879705e-10\\
10.0896875	1.72042183124634e-10\\
10.1109765625	1.23341042218904e-10\\
10.132265625	-2.59013877213621e-11\\
10.1535546875	-1.09064012860261e-12\\
10.17484375	-1.05824794436827e-10\\
10.1961328125	-1.35082783157943e-10\\
10.217421875	-1.19020417130257e-10\\
10.2387109375	-2.4763228812977e-12\\
10.26	-1.24891933987864e-10\\
10.2812890625	-4.78244378447873e-11\\
10.302578125	-6.24712478023861e-12\\
10.3238671875	-3.8572719382075e-11\\
10.34515625	-3.43916684755471e-11\\
10.3664453125	-9.95691785522889e-11\\
10.387734375	-3.98597153747035e-11\\
10.4090234375	2.20146211439176e-11\\
10.4303125	-8.64601446348053e-11\\
10.4516015625	-1.04899259901681e-11\\
10.472890625	9.80495420531599e-11\\
10.4941796875	1.74274280485583e-10\\
10.51546875	1.96120645871174e-10\\
10.5367578125	2.28354020576045e-10\\
10.558046875	2.57820943591359e-10\\
10.5793359375	1.79942584036134e-10\\
10.600625	1.60500732196897e-10\\
10.6219140625	1.86017164958094e-10\\
10.643203125	4.25450840315391e-11\\
10.6644921875	3.93942789345694e-11\\
10.68578125	9.01053595041899e-11\\
10.7070703125	1.72950016829067e-10\\
10.728359375	3.00665722580611e-10\\
10.7496484375	2.422980495875e-10\\
10.7709375	2.34220276592748e-10\\
10.7922265625	9.50002900348288e-11\\
10.813515625	1.3185795957747e-10\\
10.8348046875	1.97387168542033e-11\\
10.85609375	4.65220858939073e-11\\
10.8773828125	-5.90627099121476e-11\\
10.898671875	-1.17491514604944e-10\\
10.9199609375	-3.4308830989113e-12\\
10.94125	4.75154202227198e-11\\
10.9625390625	-1.18149327370882e-10\\
10.983828125	-5.51620517027774e-11\\
11.0051171875	-2.96752385531754e-11\\
11.02640625	-1.03521183584411e-10\\
11.0476953125	1.09384078541303e-10\\
11.068984375	1.22111644231348e-11\\
11.0902734375	8.48164338417279e-11\\
11.1115625	1.13951193762679e-10\\
11.1328515625	6.71649985183037e-11\\
11.154140625	4.36914482702579e-11\\
11.1754296875	1.09047990517093e-11\\
11.19671875	1.17233253688741e-10\\
11.2180078125	-8.45848444925557e-11\\
11.239296875	4.24352556864367e-13\\
11.2605859375	-5.3515226153302e-11\\
11.281875	-2.61538726949776e-11\\
11.3031640625	9.5624082923557e-11\\
11.324453125	-6.92039177303902e-11\\
11.3457421875	-5.5137451355313e-11\\
11.36703125	-1.88344065110531e-11\\
11.3883203125	5.22896993269277e-11\\
11.409609375	-2.23395977508261e-11\\
11.4308984375	-4.34391552061175e-11\\
11.4521875	1.71996351539535e-10\\
11.4734765625	1.3247633926408e-10\\
11.494765625	1.53661291446237e-10\\
11.5160546875	2.80663366044888e-10\\
11.53734375	1.83687138393999e-10\\
11.5586328125	1.66883132393231e-10\\
11.579921875	5.07182060439446e-11\\
11.6012109375	8.75097842644948e-11\\
11.6225	1.24057314470624e-12\\
11.6437890625	-8.76086132349139e-11\\
11.665078125	-1.32674934633557e-10\\
11.6863671875	-1.02774334653549e-10\\
11.70765625	-9.83892636683269e-11\\
11.7289453125	-1.30844778156222e-10\\
11.750234375	-1.24330759085616e-11\\
11.7715234375	-9.04374850850066e-11\\
11.7928125	-1.4754799381566e-10\\
11.8141015625	-5.63610406623563e-11\\
11.835390625	-1.5901845579669e-10\\
11.8566796875	-1.4947748064446e-10\\
11.87796875	-3.63440854348432e-10\\
11.8992578125	-2.37195479016527e-10\\
11.920546875	-3.92530206840603e-10\\
11.9418359375	-3.60285310465766e-10\\
11.963125	-2.8727983962048e-10\\
11.9844140625	-5.13434343416386e-10\\
12.005703125	-5.517187674059e-10\\
12.0269921875	-5.05942343621475e-10\\
12.04828125	-4.32516140048428e-10\\
12.0695703125	-3.13730207126299e-10\\
12.090859375	-3.59127648159307e-10\\
12.1121484375	-2.20169485605384e-10\\
12.1334375	-1.88027777673931e-10\\
12.1547265625	-3.2914755480592e-10\\
12.176015625	-2.0281101806971e-10\\
12.1973046875	-4.66356645960648e-10\\
12.21859375	-3.94138467927281e-10\\
12.2398828125	-4.92913146468672e-10\\
12.261171875	-5.242768802315e-10\\
12.2824609375	-4.73750258351175e-10\\
12.30375	-4.57752468190797e-10\\
12.3250390625	-3.91690558459091e-10\\
12.346328125	-3.31314835569714e-10\\
12.3676171875	-3.01175169748171e-10\\
12.38890625	-3.15819112762353e-10\\
12.4101953125	-3.16799819660377e-10\\
12.431484375	-3.45969033016317e-10\\
12.4527734375	-4.72012844217697e-10\\
12.4740625	-4.96918990790883e-10\\
12.4953515625	-4.48595983417308e-10\\
12.516640625	-3.78780744756647e-10\\
12.5379296875	-2.16260241760121e-10\\
12.55921875	-2.5686060499958e-10\\
12.5805078125	-2.04041872353979e-10\\
12.601796875	-2.59046352573633e-10\\
12.6230859375	-2.97452433156197e-10\\
12.644375	-3.54064522134218e-10\\
12.6656640625	-4.21790529894945e-10\\
12.686953125	-5.29567238117714e-10\\
12.7082421875	-5.95742803192647e-10\\
12.72953125	-5.92137891465855e-10\\
12.7508203125	-6.16933042042466e-10\\
12.772109375	-5.09404628583847e-10\\
12.7933984375	-5.83022152929477e-10\\
12.8146875	-4.38173023523214e-10\\
12.8359765625	-5.11457569679817e-10\\
12.857265625	-5.70267590841068e-10\\
12.8785546875	-7.25480121984621e-10\\
12.89984375	-7.9812286999294e-10\\
12.9211328125	-8.06802194709488e-10\\
12.942421875	-7.70066066147046e-10\\
12.9637109375	-7.97463507516998e-10\\
12.985	-7.20897583624782e-10\\
13.0062890625	-6.55974131475669e-10\\
13.027578125	-4.88038129919684e-10\\
13.0488671875	-4.27489095703259e-10\\
13.07015625	-4.04926605688215e-10\\
13.0914453125	-4.97636197223205e-10\\
13.112734375	-5.73759559722577e-10\\
13.1340234375	-4.55276194015306e-10\\
13.1553125	-5.48247148341516e-10\\
13.1766015625	-5.38189845505534e-10\\
13.197890625	-5.02584779681438e-10\\
13.2191796875	-5.25059126989043e-10\\
13.24046875	-3.96125046014795e-10\\
13.2617578125	-3.25561636067655e-10\\
13.283046875	-3.05949421694876e-10\\
13.3043359375	-2.44680463371991e-10\\
13.325625	-2.72945326416243e-10\\
13.3469140625	-1.80058627772013e-10\\
13.368203125	-2.99034633522106e-10\\
13.3894921875	-3.45381594739605e-10\\
13.41078125	-3.91683553195778e-10\\
13.4320703125	-3.71460008849123e-10\\
13.453359375	-4.28955631402314e-10\\
13.4746484375	-3.57021708540506e-10\\
13.4959375	-3.61474176349095e-10\\
13.5172265625	-3.14303572126492e-10\\
13.538515625	-2.95088808113196e-10\\
13.5598046875	-2.70174926942594e-10\\
13.58109375	-2.32904107099376e-10\\
13.6023828125	-3.69018323352589e-10\\
13.623671875	-4.55826842716484e-10\\
13.6449609375	-4.69752037865598e-10\\
13.66625	-5.92838644094353e-10\\
13.6875390625	-6.23746180565224e-10\\
13.708828125	-5.82785460467056e-10\\
13.7301171875	-4.84200438713045e-10\\
13.75140625	-5.1831692630929e-10\\
13.7726953125	-4.57977569159325e-10\\
13.793984375	-3.48752737374604e-10\\
13.8152734375	-4.56061122780693e-10\\
13.8365625	-3.79474535628032e-10\\
13.8578515625	-5.16647627086349e-10\\
13.879140625	-5.50560546174113e-10\\
13.9004296875	-5.35494736347508e-10\\
13.92171875	-5.92778749632167e-10\\
13.9430078125	-5.44689433384349e-10\\
13.964296875	-5.07992220722238e-10\\
13.9855859375	-4.41914952082631e-10\\
14.006875	-2.97402477637805e-10\\
14.0281640625	-3.81458305637532e-10\\
14.049453125	-3.03980874300844e-10\\
14.0707421875	-3.47090332352673e-10\\
14.09203125	-4.11523432213119e-10\\
14.1133203125	-4.36064680380331e-10\\
14.134609375	-4.43507879633271e-10\\
14.1558984375	-3.91884590446986e-10\\
14.1771875	-3.57244066775131e-10\\
14.1984765625	-2.63121528980186e-10\\
14.219765625	-1.67868086834682e-10\\
14.2410546875	-1.35540401773574e-10\\
14.26234375	-1.15373204848663e-10\\
14.2836328125	-6.81340392630359e-11\\
14.304921875	-1.00346431259625e-10\\
14.3262109375	-1.75491690166383e-10\\
14.3475	-2.82382506819472e-10\\
14.3687890625	-1.94309363368867e-10\\
14.390078125	-9.12285551003216e-11\\
14.4113671875	-1.75563534306954e-10\\
14.43265625	8.33764262744851e-11\\
14.4539453125	5.5926798318669e-11\\
14.475234375	1.01874341731353e-10\\
14.4965234375	2.22688421062209e-10\\
14.5178125	2.63351187314829e-10\\
14.5391015625	2.29023788693668e-10\\
14.560390625	1.08461253316711e-10\\
14.5816796875	7.08560691138354e-12\\
14.60296875	1.74529510742675e-10\\
14.6242578125	-1.20008603161493e-10\\
14.645546875	-4.59704816173716e-11\\
14.6668359375	-1.46051589546591e-11\\
14.688125	9.75635520566423e-11\\
14.7094140625	2.27364558461106e-10\\
14.730703125	3.36662810490336e-10\\
14.7519921875	2.84667700419997e-10\\
14.77328125	3.43967270950673e-10\\
14.7945703125	1.1155085549161e-10\\
14.815859375	-5.66260227399294e-12\\
14.8371484375	1.9006637861154e-11\\
14.8584375	-2.16456529549898e-10\\
14.8797265625	-2.56317555343451e-10\\
14.901015625	-2.23798918354144e-10\\
14.9223046875	1.12219388841665e-11\\
14.94359375	1.42140322122987e-10\\
14.9648828125	2.54351261060362e-10\\
14.986171875	2.79333562045571e-10\\
15.0074609375	1.7431764150459e-10\\
15.02875	5.76359760247466e-11\\
15.0500390625	4.80069544831724e-11\\
15.071328125	7.67851129790386e-11\\
15.0926171875	-4.05018039559816e-11\\
15.11390625	7.08385458045498e-11\\
15.1351953125	1.51142846349309e-10\\
15.156484375	2.12679510158101e-10\\
15.1777734375	2.25366939363363e-10\\
15.1990625	3.26591572768761e-10\\
15.2203515625	3.83940768320121e-10\\
15.241640625	2.76306599024204e-10\\
15.2629296875	2.55926671466615e-10\\
15.28421875	2.10071622442098e-10\\
15.3055078125	2.47424303911379e-10\\
15.326796875	1.68969517426654e-10\\
15.3480859375	2.5992329043687e-10\\
15.369375	1.71950728415097e-10\\
15.3906640625	1.72064030880606e-10\\
15.411953125	2.26612142396675e-10\\
15.4332421875	1.92031800590555e-10\\
15.45453125	2.31615241711561e-11\\
15.4758203125	2.16986110001423e-10\\
15.497109375	1.43057248215369e-10\\
15.5183984375	1.66336336511286e-10\\
15.5396875	3.57140319390572e-10\\
15.5609765625	4.16217070988074e-10\\
15.582265625	4.17586055362133e-10\\
15.6035546875	5.54336704188598e-10\\
15.62484375	4.72079938034469e-10\\
15.6461328125	4.00300324722439e-10\\
15.667421875	2.31319411559527e-10\\
15.6887109375	1.52315358817779e-10\\
15.71	9.05596089306523e-11\\
15.7312890625	1.87626947818077e-10\\
15.752578125	1.84209744500925e-10\\
15.7738671875	3.56478789166474e-10\\
15.79515625	3.6686211472839e-10\\
15.8164453125	4.74907518055168e-10\\
15.837734375	4.07303037990371e-10\\
15.8590234375	3.34093716803628e-10\\
15.8803125	3.42270538344757e-10\\
15.9016015625	2.22064962738579e-10\\
15.922890625	2.57054682182592e-10\\
15.9441796875	1.48418795986957e-10\\
15.96546875	1.73553816432336e-10\\
15.9867578125	1.39728471389723e-10\\
16.008046875	1.65416906182442e-10\\
16.0293359375	1.80111819881189e-10\\
16.050625	3.10978096107362e-10\\
16.0719140625	4.65969806991303e-10\\
16.093203125	4.01815619976839e-10\\
16.1144921875	5.46455753766467e-10\\
16.13578125	5.12051342133077e-10\\
16.1570703125	4.69584549173205e-10\\
16.178359375	3.87388156842054e-10\\
16.1996484375	2.18719070703806e-10\\
16.2209375	1.69309320149603e-10\\
16.2422265625	4.43304823809942e-11\\
16.263515625	3.03968303259473e-10\\
16.2848046875	1.91946305999186e-10\\
16.30609375	3.74604250345802e-10\\
16.3273828125	4.64753535274799e-10\\
16.348671875	4.77940677854782e-10\\
16.3699609375	5.85357593014778e-10\\
16.39125	4.71306895010763e-10\\
16.4125390625	2.79104694657134e-10\\
16.433828125	1.55525767355113e-10\\
16.4551171875	9.08625218024622e-11\\
16.47640625	6.17329176035726e-11\\
16.4976953125	3.0708570497865e-11\\
16.518984375	8.5534363013908e-11\\
16.5402734375	1.54733630486621e-10\\
16.5615625	2.92787389604878e-10\\
16.5828515625	2.79014155209914e-10\\
16.604140625	3.66296199500282e-10\\
16.6254296875	2.93152347668601e-10\\
16.64671875	1.00968548206186e-10\\
16.6680078125	-3.10592630948613e-11\\
16.689296875	-1.09731750330095e-10\\
16.7105859375	-1.36407143772497e-10\\
16.731875	-1.51197456335238e-10\\
16.7531640625	-1.0169708135943e-10\\
16.774453125	-1.67284705986638e-10\\
16.7957421875	4.2015022343257e-11\\
16.81703125	9.09466765045111e-11\\
16.8383203125	7.63654675330357e-11\\
16.859609375	1.53093080960898e-10\\
16.8808984375	8.55107219691084e-11\\
16.9021875	-1.51345534757112e-10\\
16.9234765625	-1.18901230711469e-10\\
16.944765625	-3.49553968420155e-10\\
16.9660546875	-3.67448700025819e-10\\
16.98734375	-4.7311788675146e-10\\
17.0086328125	-4.13889518316173e-10\\
17.029921875	-4.47372222669197e-10\\
17.0512109375	-4.76659643971706e-10\\
17.0725	-6.11025868115071e-10\\
17.0937890625	-5.71234937219846e-10\\
17.115078125	-5.97901302640179e-10\\
17.1363671875	-4.23627463012876e-10\\
17.15765625	-4.37305683135465e-10\\
17.1789453125	-6.1326233651706e-10\\
17.200234375	-6.16267650721113e-10\\
17.2215234375	-7.74760366502774e-10\\
17.2428125	-6.61655909259882e-10\\
17.2641015625	-7.46055720449024e-10\\
17.285390625	-7.49870732800646e-10\\
17.3066796875	-6.73937466873711e-10\\
17.32796875	-6.3628969900923e-10\\
17.3492578125	-7.96037132900163e-10\\
17.370546875	-6.19173882275818e-10\\
17.3918359375	-5.65836005582586e-10\\
17.413125	-7.0120887381222e-10\\
17.4344140625	-6.95600043098582e-10\\
17.455703125	-5.66645094583696e-10\\
17.4769921875	-7.43673836977604e-10\\
17.49828125	-6.79516289442867e-10\\
17.5195703125	-6.2410805128208e-10\\
17.540859375	-6.6918869042269e-10\\
17.5621484375	-7.44594675238552e-10\\
17.5834375	-8.15121001652357e-10\\
17.6047265625	-8.43429058158762e-10\\
17.626015625	-9.18801817803697e-10\\
17.6473046875	-9.56188965078235e-10\\
17.66859375	-9.77479005525853e-10\\
17.6898828125	-9.05997196731634e-10\\
17.711171875	-9.54300787446586e-10\\
17.7324609375	-1.00882564939105e-09\\
17.75375	-9.62592092533369e-10\\
17.7750390625	-9.26491137920442e-10\\
17.796328125	-1.04997565929055e-09\\
17.8176171875	-8.98467289218884e-10\\
17.83890625	-9.26840834435231e-10\\
17.8601953125	-8.56209741929705e-10\\
17.881484375	-7.48583569078873e-10\\
17.9027734375	-6.14826253585193e-10\\
17.9240625	-8.63497507350064e-10\\
17.9453515625	-5.92984012436908e-10\\
17.966640625	-7.05954343545392e-10\\
17.9879296875	-6.66821916752644e-10\\
18.00921875	-7.08589436462824e-10\\
18.0305078125	-7.80388943151508e-10\\
18.051796875	-8.2597635320783e-10\\
18.0730859375	-7.49177921470879e-10\\
18.094375	-8.58241512246201e-10\\
18.1156640625	-8.44512486582696e-10\\
18.136953125	-7.41645472387407e-10\\
18.1582421875	-7.3850220101614e-10\\
18.17953125	-6.11125181429894e-10\\
18.2008203125	-5.31245036282244e-10\\
18.222109375	-5.57084530347917e-10\\
18.2433984375	-5.13265464041755e-10\\
18.2646875	-5.79815359575449e-10\\
18.2859765625	-5.84618811565991e-10\\
18.307265625	-6.28395780175767e-10\\
18.3285546875	-6.01356187131167e-10\\
18.34984375	-6.15591635397404e-10\\
18.3711328125	-4.99137411111695e-10\\
18.392421875	-6.74370019189544e-10\\
18.4137109375	-5.53019421930544e-10\\
18.435	-5.74106694376963e-10\\
18.4562890625	-5.44623855696522e-10\\
18.477578125	-5.94543885522526e-10\\
18.4988671875	-4.70754661097458e-10\\
18.52015625	-4.82278970293155e-10\\
18.5414453125	-4.35177281307706e-10\\
18.562734375	-5.82269588011842e-10\\
18.5840234375	-4.29296574299757e-10\\
18.6053125	-4.93326830307882e-10\\
18.6266015625	-6.16276038216849e-10\\
18.647890625	-5.8314433068121e-10\\
18.6691796875	-6.24268871527337e-10\\
18.69046875	-7.33170094058089e-10\\
18.7117578125	-5.71493660306949e-10\\
18.733046875	-5.07922619035383e-10\\
18.7543359375	-4.50187934385777e-10\\
18.775625	-5.44581806852876e-10\\
18.7969140625	-3.26705296818418e-10\\
18.818203125	-4.1306067681179e-10\\
18.8394921875	-3.33513732104865e-10\\
18.86078125	-2.74502080998529e-10\\
18.8820703125	-2.50683148796377e-10\\
18.903359375	-1.92901381781594e-10\\
18.9246484375	-2.70340759570973e-10\\
18.9459375	-2.64052331234765e-10\\
18.9672265625	-2.58427720027136e-10\\
18.988515625	-2.43213911107493e-10\\
19.0098046875	-1.88806613390387e-10\\
19.03109375	-8.68882558417828e-11\\
19.0523828125	-1.99235589085621e-10\\
19.073671875	-9.96104704478081e-11\\
19.0949609375	-1.05530587686806e-11\\
19.11625	8.13275955876842e-11\\
19.1375390625	1.02357709560552e-10\\
19.158828125	2.96659719612454e-10\\
19.1801171875	1.06629281812705e-10\\
19.20140625	1.5445244933825e-10\\
19.2226953125	2.06642711353456e-10\\
19.243984375	4.84012352401262e-11\\
19.2652734375	1.19904275928943e-11\\
19.2865625	5.17046141938178e-11\\
19.3078515625	2.95211910636684e-11\\
19.329140625	7.10472957839944e-11\\
19.3504296875	1.38290285659855e-10\\
19.37171875	2.26854549595537e-10\\
19.3930078125	3.13741988094909e-10\\
19.414296875	3.11848977833485e-10\\
19.4355859375	3.14370448664859e-10\\
19.456875	4.37189228242101e-10\\
19.4781640625	3.85963578937255e-10\\
19.499453125	1.85353975648695e-10\\
19.5207421875	1.8937376778258e-10\\
19.54203125	2.98134717877828e-10\\
19.5633203125	2.33276997010504e-10\\
19.584609375	4.55562755105542e-10\\
19.6058984375	3.32450012039884e-10\\
19.6271875	4.55542442569783e-10\\
19.6484765625	3.04818373651868e-10\\
19.669765625	3.33639638848142e-10\\
19.6910546875	2.12548835333313e-10\\
19.71234375	3.02743073477739e-10\\
19.7336328125	6.56081965574911e-11\\
19.754921875	1.22905387005245e-10\\
19.7762109375	2.51431595951492e-10\\
19.7975	4.12191024696878e-10\\
19.8187890625	4.16195689631112e-10\\
19.840078125	4.8790067511346e-10\\
19.8613671875	6.56389137524163e-10\\
19.88265625	3.25566351808117e-10\\
19.9039453125	4.59237594783728e-10\\
19.925234375	3.60396986687758e-10\\
19.9465234375	3.44018356171757e-10\\
19.9678125	1.41636740501451e-10\\
19.9891015625	2.02693700643974e-10\\
20.010390625	5.71518841451582e-11\\
20.0316796875	2.73267712174378e-10\\
20.05296875	1.20479223952198e-10\\
20.0742578125	4.31524111269769e-10\\
20.095546875	3.26722021252318e-10\\
20.1168359375	3.9082837982682e-10\\
20.138125	2.82766268205197e-10\\
20.1594140625	4.53577706302583e-10\\
20.180703125	3.76632271031133e-10\\
20.2019921875	2.45500173319079e-10\\
20.22328125	4.07254291741508e-10\\
20.2445703125	3.78162984651364e-10\\
20.265859375	1.84638769265256e-10\\
20.2871484375	2.50171373341671e-10\\
20.3084375	3.0406662211279e-10\\
20.3297265625	2.02429200416844e-10\\
20.351015625	1.1400257551186e-10\\
20.3723046875	1.12960156454842e-10\\
20.39359375	1.31406624618827e-10\\
20.4148828125	2.16186256818221e-10\\
20.436171875	2.82805016645491e-10\\
20.4574609375	2.2864277794698e-10\\
20.47875	2.47132704897867e-10\\
20.5000390625	1.03128091869654e-10\\
20.521328125	-3.60283388583814e-11\\
20.5426171875	1.26190416866051e-11\\
20.56390625	-1.61434199981932e-10\\
20.5851953125	-1.35191857812447e-10\\
20.606484375	-4.38598615516162e-11\\
20.6277734375	-1.96044210819211e-11\\
20.6490625	-1.01064258194401e-11\\
20.6703515625	4.73486057092217e-11\\
20.691640625	-3.89528850367248e-11\\
20.7129296875	-7.20276784803276e-11\\
20.73421875	-1.4913421012335e-10\\
20.7555078125	-2.15359763878259e-10\\
20.776796875	-2.61076033679055e-10\\
20.7980859375	-3.66823922434238e-10\\
20.819375	-2.96664281254491e-10\\
20.8406640625	-3.14886485465672e-10\\
20.861953125	-3.70179007499176e-10\\
20.8832421875	-2.6830794747048e-10\\
20.90453125	-1.2930120136486e-10\\
20.9258203125	-2.56219028125845e-10\\
20.947109375	-7.5536301795438e-11\\
20.9683984375	-6.19389736127973e-11\\
20.9896875	-6.05708577649746e-12\\
21.0109765625	-1.92195057361671e-11\\
21.032265625	-2.91825893076134e-11\\
21.0535546875	-1.08384928428033e-10\\
21.07484375	9.15616922933472e-12\\
21.0961328125	-1.85964430984904e-10\\
21.117421875	-1.42433063039832e-10\\
21.1387109375	-1.60924984538512e-10\\
21.16	-2.35088711819873e-10\\
21.1812890625	-2.1809259017954e-10\\
21.202578125	-2.63841966513739e-10\\
21.2238671875	-2.24738692824358e-10\\
21.24515625	-1.14904321742011e-10\\
21.2664453125	-1.72238583736163e-10\\
21.287734375	-1.856641600804e-10\\
21.3090234375	-1.85384021594916e-10\\
21.3303125	-1.79327590485818e-10\\
21.3516015625	-2.32180077400624e-10\\
21.372890625	-4.15991541183478e-10\\
21.3941796875	-4.00033051655045e-10\\
21.41546875	-4.12233802975104e-10\\
21.4367578125	-4.06844533082686e-10\\
21.458046875	-4.77975875259138e-10\\
21.4793359375	-2.9799825154469e-10\\
21.500625	-2.27262998829166e-10\\
21.5219140625	-1.71230874181041e-10\\
21.543203125	-1.78694112524375e-10\\
21.5644921875	-1.91366513737701e-10\\
21.58578125	-3.47187000005788e-10\\
21.6070703125	-2.80356794908488e-10\\
21.628359375	-3.64106676720898e-10\\
21.6496484375	-5.21241779366466e-10\\
21.6709375	-5.05883492462318e-10\\
21.6922265625	-4.62588534758572e-10\\
21.713515625	-5.93486619312306e-10\\
21.7348046875	-5.70978563102335e-10\\
21.75609375	-4.73629368832601e-10\\
21.7773828125	-4.61285972762036e-10\\
21.798671875	-4.77566416451153e-10\\
21.8199609375	-5.6533550239469e-10\\
21.84125	-5.72860592207966e-10\\
21.8625390625	-5.36066738230043e-10\\
21.883828125	-5.7986793000933e-10\\
21.9051171875	-4.92903785911923e-10\\
21.92640625	-4.64447079364213e-10\\
21.9476953125	-4.1883514899022e-10\\
21.968984375	-5.63928937020578e-10\\
21.9902734375	-4.77429042790644e-10\\
22.0115625	-5.08947698695505e-10\\
22.0328515625	-4.50050257823498e-10\\
22.054140625	-5.9291141659669e-10\\
22.0754296875	-5.98063154167562e-10\\
22.09671875	-5.76056594205239e-10\\
22.1180078125	-6.33194725331882e-10\\
22.139296875	-7.85711996884814e-10\\
22.1605859375	-5.43502265530275e-10\\
22.181875	-6.1773857480778e-10\\
22.2031640625	-4.90809793080565e-10\\
22.224453125	-5.32827667060671e-10\\
22.2457421875	-3.20143054197355e-10\\
22.26703125	-3.30140022871752e-10\\
22.2883203125	-3.39647745415942e-10\\
22.309609375	-4.3454560163307e-10\\
22.3308984375	-4.19850225194381e-10\\
22.3521875	-7.06088778870711e-10\\
22.3734765625	-6.77693301274759e-10\\
22.394765625	-6.53885873283609e-10\\
22.4160546875	-7.51674244448027e-10\\
22.43734375	-6.73742480767235e-10\\
22.4586328125	-5.41192486621144e-10\\
22.479921875	-4.08165255865241e-10\\
22.5012109375	-3.85951558493764e-10\\
22.5225	-4.27813750620437e-10\\
22.5437890625	-3.8785863572258e-10\\
22.565078125	-3.94021104617788e-10\\
22.5863671875	-4.60637150589907e-10\\
22.60765625	-3.43323398820031e-10\\
22.6289453125	-4.26733411103825e-10\\
22.650234375	-3.18243998352665e-10\\
22.6715234375	-4.31500407861262e-10\\
22.6928125	-3.58119626841714e-10\\
22.7141015625	-3.21822312403562e-10\\
22.735390625	-4.54655243799452e-10\\
22.7566796875	-4.55270549932889e-10\\
22.77796875	-4.38728686103357e-10\\
22.7992578125	-3.62762604982271e-10\\
22.820546875	-3.39765548793172e-10\\
22.8418359375	-2.07713336250939e-10\\
22.863125	-1.50513742635492e-10\\
22.8844140625	-1.39939577294515e-10\\
22.905703125	-1.98287283854861e-10\\
22.9269921875	-2.67583692181e-10\\
22.94828125	-2.46800956134809e-10\\
22.9695703125	-3.0927446684322e-10\\
22.990859375	-3.9284324122616e-10\\
23.0121484375	-2.26765863299406e-10\\
23.0334375	-2.48537544340357e-10\\
23.0547265625	-2.02687642332541e-10\\
23.076015625	-6.86870002672065e-11\\
23.0973046875	-4.06174369605629e-11\\
23.11859375	-1.0014720398173e-10\\
23.1398828125	-5.98109649857359e-11\\
23.161171875	5.49959681688473e-12\\
23.1824609375	-5.91957806518117e-11\\
23.20375	-2.94625795971378e-12\\
23.2250390625	1.23213300435204e-10\\
23.246328125	1.25597024245034e-10\\
23.2676171875	1.67961563422571e-10\\
23.28890625	2.57199554607513e-10\\
23.3101953125	2.07765893692279e-10\\
23.331484375	1.09755863355264e-10\\
23.3527734375	9.87637667061732e-11\\
23.3740625	1.71021109379626e-10\\
23.3953515625	1.80191565889855e-10\\
23.416640625	2.20326729642814e-10\\
23.4379296875	3.31143434119968e-10\\
23.45921875	2.97944038860975e-10\\
23.4805078125	4.3219598716655e-10\\
23.501796875	3.6209756950386e-10\\
23.5230859375	3.87786241578554e-10\\
23.544375	3.34809776100568e-10\\
23.5656640625	3.3376643624672e-10\\
23.586953125	1.59332158145106e-10\\
23.6082421875	2.91329105284735e-10\\
23.62953125	2.32813726402835e-10\\
23.6508203125	2.60776785001253e-10\\
23.672109375	2.07409303892155e-10\\
23.6933984375	2.76210164729203e-10\\
23.7146875	3.60580767208015e-10\\
23.7359765625	3.33508268558205e-10\\
23.757265625	2.39736663215065e-10\\
23.7785546875	4.40260979599314e-10\\
23.79984375	2.9325959697965e-10\\
23.8211328125	3.20153835372755e-10\\
23.842421875	2.73542778275877e-10\\
23.8637109375	3.06247746904092e-10\\
23.885	3.03022988605213e-10\\
23.9062890625	2.74154229671186e-10\\
23.927578125	4.1136823960147e-10\\
23.9488671875	3.38337982812823e-10\\
23.97015625	3.08812835517115e-10\\
23.9914453125	4.28088804724572e-10\\
24.012734375	4.4426041155941e-10\\
24.0340234375	4.20185700594362e-10\\
24.0553125	4.18728443899401e-10\\
24.0766015625	3.45186157885172e-10\\
24.097890625	5.32419653915176e-10\\
24.1191796875	4.22565775634325e-10\\
24.14046875	4.85180437861396e-10\\
24.1617578125	5.41583834896278e-10\\
24.183046875	4.78010755720034e-10\\
24.2043359375	3.8846282098202e-10\\
24.225625	5.56650592626186e-10\\
24.2469140625	4.94791906879243e-10\\
24.268203125	4.8344813765697e-10\\
24.2894921875	5.34996063332616e-10\\
24.31078125	5.30227176630202e-10\\
24.3320703125	4.66552784811088e-10\\
24.353359375	5.5022171400521e-10\\
24.3746484375	4.78055078605746e-10\\
24.3959375	5.87134427253259e-10\\
24.4172265625	5.75823328370856e-10\\
24.438515625	6.19608762106085e-10\\
24.4598046875	5.32384480415252e-10\\
24.48109375	5.59903460894505e-10\\
24.5023828125	5.9974921490354e-10\\
24.523671875	6.04655112855267e-10\\
24.5449609375	5.77289205658126e-10\\
24.56625	5.73551808627898e-10\\
24.5875390625	6.70187679852679e-10\\
24.608828125	5.53372489176565e-10\\
24.6301171875	6.71684003171367e-10\\
24.65140625	6.14349688374733e-10\\
24.6726953125	5.82175481804222e-10\\
24.693984375	5.06864764291951e-10\\
24.7152734375	5.10227797834319e-10\\
24.7365625	5.20364346775298e-10\\
24.7578515625	4.07507859168619e-10\\
24.779140625	4.37831950460178e-10\\
24.8004296875	5.33834979621794e-10\\
24.82171875	3.88503371607542e-10\\
24.8430078125	4.01684290596833e-10\\
24.864296875	5.93727586013004e-10\\
24.8855859375	3.66059376040908e-10\\
24.906875	4.51327128380624e-10\\
24.9281640625	3.34078038701835e-10\\
24.949453125	2.46912730451081e-10\\
24.9707421875	2.44632380222796e-10\\
24.99203125	1.19564610384077e-10\\
25.0133203125	1.68797810030939e-10\\
25.034609375	2.35337568706468e-10\\
25.0558984375	9.54870065391638e-11\\
25.0771875	2.67953352710895e-10\\
25.0984765625	2.28700448281563e-10\\
25.119765625	2.45977841048379e-10\\
25.1410546875	2.17215005510536e-10\\
25.16234375	1.61390241556127e-10\\
25.1836328125	1.68203744255551e-10\\
25.204921875	6.30119779580567e-11\\
25.2262109375	4.46008672831202e-11\\
25.2475	1.72753725995211e-10\\
25.2687890625	1.30282095852213e-10\\
25.290078125	1.36756386341729e-10\\
25.3113671875	1.49259854299124e-10\\
25.33265625	2.51205081457202e-10\\
25.3539453125	8.22359722369564e-11\\
25.375234375	1.06690951137466e-10\\
25.3965234375	-1.49106271822256e-11\\
25.4178125	-7.20134469998724e-12\\
25.4391015625	-7.90845494399774e-11\\
25.460390625	-1.17720382736818e-11\\
25.4816796875	1.76214130459561e-10\\
25.50296875	1.99147663138801e-11\\
25.5242578125	1.17132759107759e-10\\
25.545546875	1.84754536393158e-10\\
25.5668359375	2.11431805238155e-10\\
25.588125	3.99656735839518e-12\\
25.6094140625	1.15776647901837e-10\\
25.630703125	4.05989834113948e-11\\
25.6519921875	-5.20254085763132e-11\\
25.67328125	-1.27320061785447e-10\\
25.6945703125	-9.8086021434328e-11\\
25.715859375	-1.42083124218279e-10\\
25.7371484375	-4.1155613112651e-11\\
25.7584375	-1.11703759308849e-10\\
25.7797265625	1.81096690225563e-11\\
25.801015625	-1.12411886609605e-12\\
25.8223046875	-1.38566179133168e-10\\
25.84359375	-6.67574520969646e-11\\
25.8648828125	-2.40487401927559e-11\\
25.886171875	-1.82276109743443e-10\\
25.9074609375	-2.17134844376538e-10\\
25.92875	-2.18329299844603e-10\\
25.9500390625	-2.9177232000055e-10\\
25.971328125	-2.2085075984623e-10\\
25.9926171875	-3.68482230629246e-10\\
26.01390625	-3.68858244381326e-10\\
26.0351953125	-4.16228340267807e-10\\
26.056484375	-4.17368891549097e-10\\
26.0777734375	-2.78083512176645e-10\\
26.0990625	-2.43845908184709e-10\\
26.1203515625	-3.70562005575528e-10\\
26.141640625	-1.81332397440185e-10\\
26.1629296875	-2.99353856476542e-10\\
26.18421875	-2.30418102255355e-10\\
26.2055078125	-1.85029341995952e-10\\
26.226796875	-2.6278257436117e-10\\
26.2480859375	-8.86861267141408e-11\\
26.269375	-1.52076616274226e-10\\
26.2906640625	-2.7071017359979e-10\\
26.311953125	-1.77164050780105e-10\\
26.3332421875	-2.51212567242407e-10\\
26.35453125	-2.77897297992131e-10\\
26.3758203125	-2.61358912613243e-10\\
26.397109375	-2.86131311029736e-10\\
26.4183984375	-2.73086062117682e-10\\
26.4396875	-3.11838371100426e-10\\
26.4609765625	-3.56250330251235e-10\\
26.482265625	-3.29049933360613e-10\\
26.5035546875	-3.69341969959457e-10\\
26.52484375	-2.07991458136311e-10\\
26.5461328125	-2.64121630627288e-10\\
26.567421875	-1.78064158432853e-10\\
26.5887109375	-2.39860443532099e-10\\
26.61	-1.95381774975363e-10\\
26.6312890625	-2.19238326845777e-10\\
26.652578125	-3.28448751978437e-10\\
26.6738671875	-2.40604666366882e-10\\
26.69515625	-3.08923772381818e-10\\
26.7164453125	-3.45191517071772e-10\\
26.737734375	-3.73631957645664e-10\\
26.7590234375	-2.85454896473827e-10\\
26.7803125	-4.00567039028297e-10\\
26.8016015625	-3.48927677098778e-10\\
26.822890625	-4.16677835202119e-10\\
26.8441796875	-2.95188429686589e-10\\
26.86546875	-3.97130120978031e-10\\
26.8867578125	-3.84980891622316e-10\\
26.908046875	-4.38255346506717e-10\\
26.9293359375	-3.6886067980668e-10\\
26.950625	-5.31949039556595e-10\\
26.9719140625	-6.29059307594569e-10\\
26.993203125	-6.06801378307022e-10\\
27.0144921875	-7.0029682158165e-10\\
27.03578125	-6.75239333415982e-10\\
27.0570703125	-6.2145904949905e-10\\
27.078359375	-5.83803097808466e-10\\
27.0996484375	-6.3713372007691e-10\\
27.1209375	-5.11591834423829e-10\\
27.1422265625	-5.73989924727654e-10\\
27.163515625	-5.58759710237203e-10\\
27.1848046875	-5.54600033627821e-10\\
27.20609375	-6.57127594353395e-10\\
27.2273828125	-7.22421205522255e-10\\
27.248671875	-6.65888513464801e-10\\
27.2699609375	-5.34040177948356e-10\\
27.29125	-4.54806869624772e-10\\
27.3125390625	-5.64318803941415e-10\\
27.333828125	-4.41348703924612e-10\\
27.3551171875	-4.3747310710682e-10\\
27.37640625	-4.4491677307415e-10\\
27.3976953125	-3.68937199730169e-10\\
27.418984375	-4.90046426132591e-10\\
27.4402734375	-5.05927337189284e-10\\
27.4615625	-4.58981708590815e-10\\
27.4828515625	-5.49060852551632e-10\\
27.504140625	-4.27452489607551e-10\\
27.5254296875	-4.28852717738239e-10\\
27.54671875	-4.51052364636549e-10\\
27.5680078125	-3.60244166791106e-10\\
27.589296875	-3.52692325242294e-10\\
27.6105859375	-3.47050859813758e-10\\
27.631875	-3.39858228825667e-10\\
27.6531640625	-4.06086612305022e-10\\
27.674453125	-4.47182395032155e-10\\
27.6957421875	-4.67863601302105e-10\\
27.71703125	-3.91609494814539e-10\\
27.7383203125	-3.55162078462899e-10\\
27.759609375	-4.80018967543599e-10\\
27.7808984375	-3.01456731252452e-10\\
27.8021875	-3.42850241500017e-10\\
27.8234765625	-2.95267804090739e-10\\
27.844765625	-3.59766463076406e-10\\
27.8660546875	-4.0993945327402e-10\\
27.88734375	-2.97463837742629e-10\\
27.9086328125	-3.10862119468335e-10\\
27.929921875	-3.13334889821135e-10\\
27.9512109375	-2.20460032063107e-10\\
27.9725	-2.91933416939467e-10\\
27.9937890625	-3.21854942274148e-10\\
28.015078125	-2.14425915006893e-10\\
28.0363671875	-2.57962071364024e-10\\
28.05765625	-2.24017751647463e-10\\
28.0789453125	-2.67121617720638e-10\\
28.100234375	-1.14955212645794e-10\\
28.1215234375	-1.79466410086259e-10\\
28.1428125	-9.47427869357416e-11\\
28.1641015625	-9.33403832582024e-11\\
28.185390625	-4.02190554468251e-11\\
28.2066796875	-7.19802163241436e-11\\
28.22796875	-7.04302385060073e-11\\
28.2492578125	1.94644113030931e-11\\
28.270546875	1.04075902073731e-11\\
28.2918359375	-6.34650592467762e-12\\
28.313125	-1.58203553687879e-11\\
28.3344140625	9.73888539479208e-11\\
28.355703125	7.06461778486352e-11\\
28.3769921875	1.14017356933918e-10\\
28.39828125	9.42788162894079e-11\\
28.4195703125	1.00745578582822e-10\\
28.440859375	1.21787934839329e-10\\
28.4621484375	1.39956662926472e-10\\
28.4834375	1.25872727281467e-10\\
28.5047265625	2.79590524344191e-10\\
28.526015625	1.4788386790205e-10\\
28.5473046875	2.55864209026841e-10\\
28.56859375	1.82021224637869e-10\\
28.5898828125	2.29654028111921e-10\\
28.611171875	2.07155165759251e-10\\
28.6324609375	2.16883444418954e-10\\
28.65375	2.08386971274226e-10\\
28.6750390625	2.5133089877906e-10\\
28.696328125	1.9755918389263e-10\\
28.7176171875	3.49555817078076e-10\\
28.73890625	2.42065825651319e-10\\
28.7601953125	2.25838417284189e-10\\
28.781484375	3.31345787033415e-10\\
28.8027734375	2.57241479151951e-10\\
28.8240625	1.91659278704981e-10\\
28.8453515625	2.47751381817953e-10\\
28.866640625	1.99948370588542e-10\\
28.8879296875	1.33778340822034e-10\\
28.90921875	1.56985153165584e-10\\
28.9305078125	2.44591013119948e-10\\
28.951796875	1.63339199836795e-10\\
28.9730859375	3.18570575202494e-10\\
28.994375	4.48725221972665e-10\\
29.0156640625	3.49468301226423e-10\\
29.036953125	3.40424348320372e-10\\
29.0582421875	3.22858720951954e-10\\
29.07953125	1.7891758936729e-10\\
29.1008203125	1.79249400069567e-10\\
29.122109375	1.84189414391685e-10\\
29.1433984375	2.94305693199139e-10\\
29.1646875	4.15736416652113e-10\\
29.1859765625	3.70226423085435e-10\\
29.207265625	5.58309166843542e-10\\
29.2285546875	6.94162839899804e-10\\
29.24984375	6.87176454969545e-10\\
29.2711328125	6.14369774360639e-10\\
29.292421875	7.23443186150778e-10\\
29.3137109375	6.26511411818267e-10\\
29.335	6.28943473806274e-10\\
29.3562890625	6.79114701437854e-10\\
29.377578125	6.72315149840238e-10\\
29.3988671875	6.52873328774492e-10\\
29.42015625	7.63536233457689e-10\\
29.4414453125	7.08616351191427e-10\\
29.462734375	8.23686625556865e-10\\
29.4840234375	7.63023249593945e-10\\
29.5053125	6.98184107261037e-10\\
29.5266015625	7.52323246083654e-10\\
29.547890625	6.58945000369631e-10\\
29.5691796875	6.16151357883989e-10\\
29.59046875	6.53305393038639e-10\\
29.6117578125	5.93760007245985e-10\\
29.633046875	6.58597658438648e-10\\
29.6543359375	7.00483504740285e-10\\
29.675625	7.63334796683438e-10\\
29.6969140625	7.48863781816072e-10\\
29.718203125	7.63111455965264e-10\\
29.7394921875	6.45133362530636e-10\\
29.76078125	6.08126683609181e-10\\
29.7820703125	4.58416958677814e-10\\
29.803359375	4.49659247037792e-10\\
29.8246484375	4.8587477817537e-10\\
29.8459375	3.30514893673433e-10\\
29.8672265625	4.04272578601008e-10\\
29.888515625	4.04228553067768e-10\\
29.9098046875	4.76942156593241e-10\\
29.93109375	5.75840281915313e-10\\
29.9523828125	5.23020940162133e-10\\
29.973671875	5.55451648002061e-10\\
29.9949609375	5.89446180576422e-10\\
30.01625	4.73823169603863e-10\\
30.0375390625	5.09620193139124e-10\\
30.058828125	5.2399870271731e-10\\
30.0801171875	3.69072279073834e-10\\
30.10140625	4.35375550299473e-10\\
30.1226953125	4.31983720085162e-10\\
30.143984375	4.65981653647796e-10\\
30.1652734375	5.79641993711298e-10\\
30.1865625	5.94157041686558e-10\\
30.2078515625	7.74371667408459e-10\\
30.229140625	6.67740563619993e-10\\
30.2504296875	6.43021134955607e-10\\
30.27171875	7.2859412960463e-10\\
30.2930078125	5.63440010530527e-10\\
30.314296875	5.35297826137718e-10\\
30.3355859375	5.28856832770611e-10\\
30.356875	6.51975986306211e-10\\
30.3781640625	6.40105137696664e-10\\
30.399453125	6.07025829450042e-10\\
30.4207421875	5.54084043504461e-10\\
30.44203125	6.57472279965824e-10\\
30.4633203125	5.41405113582194e-10\\
30.484609375	5.30821282197887e-10\\
30.5058984375	4.895431921191e-10\\
30.5271875	3.66497961264413e-10\\
30.5484765625	3.09759107021767e-10\\
30.569765625	4.48222294240312e-10\\
30.5910546875	3.46336991803764e-10\\
30.61234375	3.59632612028264e-10\\
30.6336328125	3.51958101292241e-10\\
30.654921875	3.33316096331099e-10\\
30.6762109375	3.52545177396323e-10\\
30.6975	3.04084051307655e-10\\
30.7187890625	2.48902124104581e-10\\
30.740078125	1.84728053599365e-10\\
30.7613671875	1.15727352658399e-10\\
30.78265625	1.26796514767682e-10\\
30.8039453125	1.34829661833202e-10\\
30.825234375	3.86274965917873e-11\\
30.8465234375	-2.34409902123533e-11\\
30.8678125	1.33030105700264e-10\\
30.8891015625	4.09045832100416e-11\\
30.910390625	3.60732635415184e-11\\
30.9316796875	7.55195948289473e-11\\
30.95296875	1.10242377124982e-11\\
30.9742578125	-7.47175510105421e-11\\
30.995546875	-1.16142881681247e-10\\
31.0168359375	-5.45770405406494e-12\\
31.038125	9.00536963242888e-11\\
31.0594140625	-6.03852052130937e-11\\
31.080703125	2.88282157906977e-11\\
31.1019921875	-2.77737287628367e-11\\
31.12328125	-1.15822618448963e-10\\
31.1445703125	-2.71451437567864e-11\\
31.165859375	-1.63206592895577e-10\\
31.1871484375	-8.56696575073816e-11\\
31.2084375	-2.04427447753867e-10\\
31.2297265625	-2.11143940356065e-10\\
31.251015625	-1.51496456172513e-10\\
31.2723046875	-4.74738477796713e-11\\
31.29359375	-8.28097451670382e-11\\
31.3148828125	1.4498790321883e-12\\
31.336171875	5.70717122780795e-11\\
31.3574609375	4.77062317458265e-11\\
31.37875	4.21876452889772e-11\\
31.4000390625	-9.51377539983313e-11\\
31.421328125	-2.52509150373814e-11\\
31.4426171875	-8.13529999112821e-11\\
31.46390625	-1.2849444267684e-10\\
31.4851953125	-1.06383042272568e-10\\
31.506484375	-9.58684796621911e-11\\
31.5277734375	-3.54305177776258e-11\\
31.5490625	-1.11662835206352e-10\\
31.5703515625	-1.64528605970653e-10\\
31.591640625	-1.06681707207736e-10\\
31.6129296875	-1.80380998011698e-10\\
31.63421875	-1.72167973830673e-10\\
31.6555078125	-3.03732564640437e-10\\
31.676796875	-2.30863201659417e-10\\
31.6980859375	-2.99327674908605e-10\\
31.719375	-3.80753506318839e-10\\
31.7406640625	-3.37480292462656e-10\\
31.761953125	-3.12775224495374e-10\\
31.7832421875	-3.79954328666279e-10\\
31.80453125	-4.61590771216684e-10\\
31.8258203125	-4.1865486261432e-10\\
31.847109375	-5.41321951786698e-10\\
31.8683984375	-5.40768858275724e-10\\
31.8896875	-3.44328597230182e-10\\
31.9109765625	-4.81587827014558e-10\\
31.932265625	-3.83329757837285e-10\\
31.9535546875	-3.44701308748028e-10\\
31.97484375	-3.99041524775324e-10\\
31.9961328125	-3.41182173381774e-10\\
32.017421875	-4.27724435087828e-10\\
32.0387109375	-4.8234805890724e-10\\
32.06	-6.08622475448094e-10\\
32.0812890625	-6.066091818091e-10\\
32.102578125	-6.10938033463399e-10\\
32.1238671875	-6.05846502538458e-10\\
32.14515625	-5.98315603504085e-10\\
32.1664453125	-6.1883016346063e-10\\
32.187734375	-5.19193585385833e-10\\
32.2090234375	-6.5051580333068e-10\\
32.2303125	-5.07242091885877e-10\\
32.2516015625	-5.55369646778163e-10\\
32.272890625	-5.9333334195827e-10\\
32.2941796875	-5.55516771860169e-10\\
32.31546875	-4.937839093415e-10\\
32.3367578125	-4.86041087105805e-10\\
32.358046875	-3.62964407872819e-10\\
32.3793359375	-4.49246892124569e-10\\
32.400625	-3.98499316160214e-10\\
32.4219140625	-4.6937774961157e-10\\
32.443203125	-3.92726721975597e-10\\
32.4644921875	-4.06308292031266e-10\\
32.48578125	-4.63615554152411e-10\\
32.5070703125	-4.02913109519368e-10\\
32.528359375	-4.06251891061027e-10\\
32.5496484375	-5.11186205350952e-10\\
32.5709375	-4.79248152835347e-10\\
32.5922265625	-5.05238297557631e-10\\
32.613515625	-4.22224531814137e-10\\
32.6348046875	-4.93971007633149e-10\\
32.65609375	-4.1890724993712e-10\\
32.6773828125	-4.43447475850846e-10\\
32.698671875	-4.92571800156843e-10\\
32.7199609375	-4.59032447321095e-10\\
32.74125	-4.38383274875566e-10\\
32.7625390625	-5.24807744765388e-10\\
32.783828125	-4.86654404442689e-10\\
32.8051171875	-4.46841252921964e-10\\
32.82640625	-5.14687940538368e-10\\
32.8476953125	-4.41460764081208e-10\\
32.868984375	-5.12532401750963e-10\\
32.8902734375	-4.41582306538466e-10\\
32.9115625	-3.02119417610294e-10\\
32.9328515625	-3.33857548161153e-10\\
32.954140625	-3.58643802377471e-10\\
32.9754296875	-3.09925602615817e-10\\
32.99671875	-3.35623216535786e-10\\
33.0180078125	-3.20919648231627e-10\\
33.039296875	-4.28625002469525e-10\\
33.0605859375	-3.74916474566712e-10\\
33.081875	-4.29204660232841e-10\\
33.1031640625	-3.05791251910703e-10\\
33.124453125	-2.87483456480301e-10\\
33.1457421875	-1.90598534906425e-10\\
33.16703125	-1.88662941655157e-10\\
33.1883203125	-1.3250278932475e-10\\
33.209609375	-3.33477836842592e-11\\
33.2308984375	-9.28293884019534e-11\\
33.2521875	-1.20558695735544e-10\\
33.2734765625	-9.17729238865862e-11\\
33.294765625	-2.32800305728636e-11\\
33.3160546875	-1.27171764552487e-11\\
33.33734375	3.90138219332115e-12\\
33.3586328125	-7.13000629153796e-12\\
33.379921875	1.12958990073535e-10\\
33.4012109375	9.81123279793846e-11\\
33.4225	1.38785906930368e-10\\
33.4437890625	1.99547009251408e-10\\
33.465078125	1.31478042888106e-10\\
33.4863671875	1.78729116536505e-10\\
33.50765625	1.73238574938948e-10\\
33.5289453125	1.65022945444541e-10\\
33.550234375	2.12936355237769e-10\\
33.5715234375	3.02370459405742e-10\\
33.5928125	2.20229123689038e-10\\
33.6141015625	3.94505485156645e-10\\
33.635390625	3.28062181986701e-10\\
33.6566796875	3.66707364810242e-10\\
33.67796875	3.96397652809237e-10\\
33.6992578125	4.12139359954121e-10\\
33.720546875	4.20615122133353e-10\\
33.7418359375	4.80666865305143e-10\\
33.763125	3.17697960017265e-10\\
33.7844140625	3.20628271068752e-10\\
33.805703125	4.02363640022926e-10\\
33.8269921875	4.11392262040223e-10\\
33.84828125	5.72862264040168e-10\\
33.8695703125	4.76463198246973e-10\\
33.890859375	5.98935095356347e-10\\
33.9121484375	6.27745983336543e-10\\
33.9334375	6.17305946277937e-10\\
33.9547265625	5.51822477489757e-10\\
33.976015625	6.63836777083851e-10\\
33.9973046875	5.42682725495723e-10\\
34.01859375	5.27702726126858e-10\\
34.0398828125	5.51318301785455e-10\\
34.061171875	6.22159118982606e-10\\
34.0824609375	6.49682885197995e-10\\
34.10375	6.48024022764948e-10\\
34.1250390625	7.25194346963017e-10\\
34.146328125	7.21610323060133e-10\\
34.1676171875	7.23735864966753e-10\\
34.18890625	6.7576163577259e-10\\
34.2101953125	7.22678144435696e-10\\
34.231484375	7.64755879187276e-10\\
34.2527734375	7.56914100487286e-10\\
34.2740625	7.63338742083156e-10\\
34.2953515625	7.95837696128556e-10\\
34.316640625	9.26154306338221e-10\\
34.3379296875	7.81039767920654e-10\\
34.35921875	8.46823270518439e-10\\
34.3805078125	7.86475896788657e-10\\
34.401796875	7.51777820061131e-10\\
34.4230859375	7.5041642785474e-10\\
34.444375	7.29665642000465e-10\\
34.4656640625	6.76413344546107e-10\\
34.486953125	8.10343747521999e-10\\
34.5082421875	7.13943169745285e-10\\
34.52953125	9.28854065976921e-10\\
34.5508203125	8.71051902416255e-10\\
34.572109375	8.60045218230235e-10\\
34.5933984375	8.54281354124551e-10\\
34.6146875	8.13371198933652e-10\\
34.6359765625	7.25790487232269e-10\\
34.657265625	7.36975192514235e-10\\
34.6785546875	6.68623529819918e-10\\
34.69984375	6.78793643174012e-10\\
34.7211328125	6.76326595916278e-10\\
34.742421875	7.700219798784e-10\\
34.7637109375	7.92975506824458e-10\\
34.785	8.10794833122018e-10\\
34.8062890625	7.25894822633616e-10\\
34.827578125	7.22007421972463e-10\\
34.8488671875	6.49186776014053e-10\\
34.87015625	6.42448242989892e-10\\
34.8914453125	6.48708348092487e-10\\
34.912734375	6.51218411513337e-10\\
34.9340234375	7.03413375282838e-10\\
34.9553125	5.84839397472278e-10\\
34.9766015625	7.19331301089756e-10\\
34.997890625	6.24580029191536e-10\\
35.0191796875	6.75052463374933e-10\\
35.04046875	7.06368319296153e-10\\
35.0617578125	7.12228978660752e-10\\
35.083046875	7.26512299129178e-10\\
35.1043359375	5.89280656928108e-10\\
35.125625	6.51488242034798e-10\\
35.1469140625	6.02357596944663e-10\\
35.168203125	6.14249137786842e-10\\
35.1894921875	6.13582536811986e-10\\
35.21078125	5.63688477805633e-10\\
35.2320703125	6.5360331931929e-10\\
35.253359375	6.54443221564278e-10\\
35.2746484375	6.72219669396964e-10\\
35.2959375	5.82282749876021e-10\\
35.3172265625	5.35401372581174e-10\\
35.338515625	4.54376061224644e-10\\
35.3598046875	3.48053128039786e-10\\
35.38109375	3.19764809058352e-10\\
35.4023828125	2.78020430791722e-10\\
35.423671875	2.78213936189253e-10\\
35.4449609375	3.57005979006238e-10\\
35.46625	2.97133362668279e-10\\
35.4875390625	3.15438364557601e-10\\
35.508828125	2.76266393679294e-10\\
35.5301171875	3.43220045712347e-10\\
35.55140625	3.50489854518312e-10\\
35.5726953125	3.03138800185101e-10\\
35.593984375	2.8920294234839e-10\\
35.6152734375	1.27769240694467e-10\\
35.6365625	7.72918404130641e-11\\
35.6578515625	3.88651041972087e-12\\
35.679140625	9.42249591534427e-11\\
35.7004296875	3.89247255247886e-11\\
35.72171875	2.43629040180961e-11\\
35.7430078125	3.94534941945335e-11\\
35.764296875	1.00773131054493e-10\\
35.7855859375	3.4619188963728e-11\\
35.806875	5.15398400846899e-11\\
35.8281640625	1.41515092234051e-11\\
35.849453125	-4.4461823790186e-11\\
35.8707421875	-9.62946766698387e-11\\
35.89203125	-1.74180195011138e-10\\
35.9133203125	-1.19235498079789e-10\\
35.934609375	-1.38042669550183e-10\\
35.9558984375	-1.69177236710774e-10\\
35.9771875	-8.51953245373013e-14\\
35.9984765625	-9.68971217641521e-11\\
36.019765625	2.18483235837652e-12\\
36.0410546875	-4.58810165322539e-11\\
36.06234375	-1.18836779714902e-10\\
36.0836328125	-1.20160298949032e-10\\
36.104921875	-5.89894170970123e-11\\
36.1262109375	-1.71003725372062e-10\\
36.1475	-8.45551630265801e-11\\
36.1687890625	-1.44138259102279e-10\\
36.190078125	-8.14748156394559e-11\\
36.2113671875	-4.94231283830485e-11\\
36.23265625	-1.7954202476303e-10\\
36.2539453125	-1.14805412807567e-10\\
36.275234375	-1.01685579294695e-10\\
36.2965234375	-1.43642405941119e-10\\
36.3178125	-1.76071800747459e-10\\
36.3391015625	-3.35244922545154e-10\\
36.360390625	-3.70074788610989e-10\\
36.3816796875	-3.18234795858478e-10\\
36.40296875	-3.49027062408566e-10\\
36.4242578125	-2.59381898855663e-10\\
36.445546875	-3.09894418242237e-10\\
36.4668359375	-2.32673362632674e-10\\
36.488125	-2.30206678376576e-10\\
36.5094140625	-1.86252627652367e-10\\
36.530703125	-1.78116217637055e-10\\
36.5519921875	-3.29719267612119e-10\\
36.57328125	-3.43360862961043e-10\\
36.5945703125	-3.46918708769332e-10\\
36.615859375	-4.78780803710584e-10\\
36.6371484375	-3.4720029404436e-10\\
36.6584375	-3.7203943780691e-10\\
36.6797265625	-3.82828177829643e-10\\
36.701015625	-3.06106619152454e-10\\
36.7223046875	-4.48669675561141e-10\\
36.74359375	-4.00938645624016e-10\\
36.7648828125	-4.39078516891502e-10\\
36.786171875	-4.07783058935124e-10\\
36.8074609375	-4.26691000081122e-10\\
36.82875	-3.31194523129717e-10\\
36.8500390625	-3.33554278396036e-10\\
36.871328125	-2.92053649399986e-10\\
36.8926171875	-3.22236193103545e-10\\
36.91390625	-3.56258548351512e-10\\
36.9351953125	-3.96716919256048e-10\\
36.956484375	-2.58469815254747e-10\\
36.9777734375	-3.94989708793007e-10\\
36.9990625	-3.1199458296725e-10\\
37.0203515625	-2.91248148619379e-10\\
37.041640625	-2.45167420192207e-10\\
37.0629296875	-2.70800570750804e-10\\
37.08421875	-1.67728099090571e-10\\
37.1055078125	-2.28150735390642e-10\\
37.126796875	-1.26282572268367e-10\\
37.1480859375	-1.88970789561184e-10\\
37.169375	-2.93955594356535e-10\\
37.1906640625	-1.80844451545027e-10\\
37.211953125	-2.35701170386711e-10\\
37.2332421875	-1.92376570612891e-10\\
37.25453125	-1.74619708619199e-10\\
37.2758203125	-1.71954472290337e-10\\
37.297109375	-2.16495328574308e-10\\
37.3183984375	-1.21407976468025e-10\\
37.3396875	-1.16838859832664e-10\\
37.3609765625	-1.42570421250399e-10\\
37.382265625	-1.33952907158364e-10\\
37.4035546875	-1.95904821467281e-10\\
37.42484375	-1.69177631780966e-10\\
37.4461328125	-1.75847357846583e-10\\
37.467421875	-1.81549305278873e-10\\
37.4887109375	-1.95013229209412e-10\\
37.51	-1.86090237115321e-10\\
37.5312890625	-2.11686421952329e-10\\
37.552578125	-7.07608071011366e-11\\
37.5738671875	-2.21554161306915e-10\\
37.59515625	-1.14952053603221e-10\\
37.6164453125	-1.47496927695395e-10\\
37.637734375	-1.63130757892225e-10\\
};
\addplot [color=mycolor1,solid]
  table[row sep=crcr]{%
37.637734375	-1.63130757892225e-10\\
37.6590234375	-7.97369229213885e-11\\
37.6803125	-6.22906475659067e-11\\
37.7016015625	-9.53500846524714e-11\\
37.722890625	-4.88999834670626e-11\\
37.7441796875	-1.58669468371519e-10\\
37.76546875	-1.39375055615476e-10\\
37.7867578125	-1.49420998803656e-10\\
37.808046875	-9.3121152061739e-11\\
37.8293359375	-8.1635628348648e-11\\
37.850625	-2.08110776668755e-11\\
37.8719140625	2.63632447277036e-11\\
37.893203125	1.42158671568583e-11\\
37.9144921875	1.65306032717471e-11\\
37.93578125	3.38142640810466e-11\\
37.9570703125	-1.19728244848437e-10\\
37.978359375	-5.03077501767871e-12\\
37.9996484375	-4.38210986774569e-12\\
38.0209375	4.67295832293269e-11\\
38.0422265625	-1.36310618358766e-11\\
38.063515625	-4.29626504572772e-11\\
38.0848046875	4.57712679004824e-11\\
38.10609375	7.86912511256589e-11\\
38.1273828125	5.13736013596336e-11\\
38.148671875	1.17706970389003e-10\\
38.1699609375	3.10306584386193e-10\\
38.19125	1.02620788676247e-10\\
38.2125390625	2.33484225897023e-10\\
38.233828125	2.7361551208306e-10\\
38.2551171875	1.23973707908288e-10\\
38.27640625	1.77557526740291e-10\\
38.2976953125	2.06557818886609e-10\\
38.318984375	2.25903569233823e-10\\
38.3402734375	2.93155285987555e-10\\
38.3615625	2.59212870324512e-10\\
38.3828515625	4.71860788314097e-10\\
38.404140625	3.81390473839802e-10\\
38.4254296875	3.29720778567367e-10\\
38.44671875	3.61784877375994e-10\\
38.4680078125	3.12965110017867e-10\\
38.489296875	2.76711900322095e-10\\
38.5105859375	1.66472549061198e-10\\
38.531875	2.63906452126571e-10\\
38.5531640625	2.85583779750312e-10\\
38.574453125	2.64677564712546e-10\\
38.5957421875	3.57276440537439e-10\\
38.61703125	2.68138889188777e-10\\
38.6383203125	3.52577792389365e-10\\
38.659609375	2.6264819258358e-10\\
38.6808984375	2.82489937890505e-10\\
38.7021875	2.36417637119918e-10\\
38.7234765625	2.134629975305e-10\\
38.744765625	2.99860903628545e-10\\
38.7660546875	2.82221305891155e-10\\
38.78734375	3.27453159355438e-10\\
38.8086328125	3.09073823441147e-10\\
38.829921875	2.96458299950243e-10\\
38.8512109375	3.23960825329706e-10\\
38.8725	3.68696620331543e-10\\
38.8937890625	3.13734832671966e-10\\
38.915078125	4.77505204829949e-10\\
38.9363671875	3.33702787507932e-10\\
38.95765625	3.94162161168435e-10\\
38.9789453125	2.79576402074906e-10\\
39.000234375	4.13088434254741e-10\\
39.0215234375	3.36547355966868e-10\\
39.0428125	3.93478802199085e-10\\
39.0641015625	3.66020665899717e-10\\
39.085390625	4.00491895445997e-10\\
39.1066796875	2.60226453798223e-10\\
39.12796875	3.98707131718856e-10\\
39.1492578125	3.14598031137782e-10\\
39.170546875	4.52995411018862e-10\\
39.1918359375	3.97196057974408e-10\\
39.213125	4.05895368463632e-10\\
39.2344140625	5.22222405737365e-10\\
39.255703125	4.81387441768832e-10\\
39.2769921875	4.4875050195245e-10\\
39.29828125	5.23349236767809e-10\\
39.3195703125	4.27992804405767e-10\\
39.340859375	3.56886042643439e-10\\
39.3621484375	3.87824400108962e-10\\
39.3834375	4.56282158232691e-10\\
39.4047265625	4.62809528270682e-10\\
39.426015625	4.82299187391743e-10\\
39.4473046875	4.649681170081e-10\\
39.46859375	3.96996992416719e-10\\
39.4898828125	4.56582976774853e-10\\
39.511171875	2.7318297280086e-10\\
39.5324609375	3.10236090569664e-10\\
39.55375	3.2587441684116e-10\\
39.5750390625	3.1294964660865e-10\\
39.596328125	2.80744978288394e-10\\
39.6176171875	3.40838426636109e-10\\
39.63890625	2.5333153915834e-10\\
39.6601953125	3.66037221743326e-10\\
39.681484375	3.00720246904267e-10\\
39.7027734375	2.34036077097141e-10\\
39.7240625	2.35522824887901e-10\\
39.7453515625	2.15531861245649e-10\\
39.766640625	2.14317401722598e-10\\
39.7879296875	2.06980350855339e-10\\
39.80921875	2.37854234365573e-10\\
39.8305078125	1.2927307656204e-10\\
39.851796875	1.63322056739532e-10\\
39.8730859375	1.09205002736899e-10\\
39.894375	1.23083162008274e-10\\
39.9156640625	9.69543542118517e-11\\
39.936953125	7.8060098091682e-11\\
39.9582421875	1.55132191058775e-10\\
39.97953125	2.08027280682556e-10\\
40.0008203125	1.70440970196775e-10\\
40.022109375	2.28549284948632e-10\\
40.0433984375	2.22990366624805e-10\\
40.0646875	6.83722405024638e-11\\
40.0859765625	1.06514138295526e-10\\
40.107265625	2.22896560262307e-11\\
40.1285546875	4.62996801050248e-11\\
40.14984375	-3.49000502103545e-11\\
40.1711328125	2.76188478136547e-11\\
40.192421875	4.32666934508108e-11\\
40.2137109375	-5.15854130815537e-12\\
40.235	2.77323319662272e-11\\
40.2562890625	9.38944989640884e-11\\
40.277578125	8.15163759751565e-11\\
40.2988671875	4.44085857769311e-11\\
40.32015625	1.21385582402326e-10\\
40.3414453125	2.32067638089357e-11\\
40.362734375	6.71366544462215e-11\\
40.3840234375	-4.35410885185747e-11\\
40.4053125	-5.84687796883608e-11\\
40.4266015625	-4.14181588284635e-11\\
40.447890625	-3.19987195708211e-11\\
40.4691796875	-3.33818323758341e-12\\
40.49046875	2.10506173650344e-11\\
40.5117578125	-6.64941279623174e-11\\
40.533046875	-8.41342875660557e-11\\
40.5543359375	-9.23127219613322e-12\\
40.575625	2.04557072658477e-11\\
40.5969140625	1.54993368914426e-11\\
40.618203125	-2.3302815546764e-11\\
40.6394921875	-5.33517384429479e-11\\
40.66078125	-1.37840642056005e-10\\
40.6820703125	-1.73146932847494e-10\\
40.703359375	-1.41396830276314e-10\\
40.7246484375	-2.46911437075672e-10\\
40.7459375	-1.89756214022714e-10\\
40.7672265625	-1.43121619236505e-10\\
40.788515625	-9.10968996174194e-11\\
40.8098046875	-1.23194532017648e-10\\
40.83109375	-1.17466365445559e-10\\
40.8523828125	-9.89985636603393e-11\\
40.873671875	-1.64275769972072e-10\\
40.8949609375	-3.02291150644738e-10\\
40.91625	-1.94845462543459e-10\\
40.9375390625	-3.2197232729281e-10\\
40.958828125	-2.9952573266637e-10\\
40.9801171875	-2.7694898901328e-10\\
41.00140625	-1.51409597796761e-10\\
41.0226953125	-2.64297767293946e-10\\
41.043984375	-2.97072257714676e-10\\
41.0652734375	-2.94726786769058e-10\\
41.0865625	-3.03256351039253e-10\\
41.1078515625	-3.57245086184701e-10\\
41.129140625	-3.5208855391659e-10\\
41.1504296875	-3.20269157340335e-10\\
41.17171875	-2.82404757852787e-10\\
41.1930078125	-3.44250193400323e-10\\
41.214296875	-2.05960743032821e-10\\
41.2355859375	-2.81225527618869e-10\\
41.256875	-2.61287768965749e-10\\
41.2781640625	-2.36575430544422e-10\\
41.299453125	-1.91720328696053e-10\\
41.3207421875	-2.43919098180219e-10\\
41.34203125	-2.21730477814154e-10\\
41.3633203125	-2.0014410237884e-10\\
41.384609375	-1.45258977235242e-10\\
41.4058984375	-1.80881423990352e-10\\
41.4271875	-2.4814342648017e-10\\
41.4484765625	-1.8470493862693e-10\\
41.469765625	-1.76891807712035e-10\\
41.4910546875	-2.05019491780577e-10\\
41.51234375	-2.09448335909649e-10\\
41.5336328125	-1.0736546521474e-10\\
41.554921875	-1.84414026142275e-10\\
41.5762109375	-6.35473963379058e-11\\
41.5975	-2.30770933640111e-10\\
41.6187890625	-1.24290802435631e-10\\
41.640078125	-1.88627146826496e-10\\
41.6613671875	-1.45768150327054e-10\\
41.68265625	-2.67095164101819e-10\\
41.7039453125	-1.77405225930672e-10\\
41.725234375	-2.69516955924327e-10\\
41.7465234375	-2.3117670936771e-10\\
41.7678125	-2.19211175549573e-10\\
41.7891015625	-2.55642505471185e-10\\
41.810390625	-2.47912976650192e-10\\
41.8316796875	-2.23092978903346e-10\\
41.85296875	-2.65997002002252e-10\\
41.8742578125	-2.11001298766304e-10\\
41.895546875	-2.00765315364186e-10\\
41.9168359375	-2.76262275781023e-10\\
41.938125	-3.08192935408068e-10\\
41.9594140625	-2.72902908793936e-10\\
41.980703125	-2.47145637079367e-10\\
42.0019921875	-3.65457734947279e-10\\
42.02328125	-2.24393649450118e-10\\
42.0445703125	-3.09800054129899e-10\\
42.065859375	-2.18106955538817e-10\\
42.0871484375	-3.00466156784025e-10\\
42.1084375	-2.77346906322737e-10\\
42.1297265625	-2.83201912583972e-10\\
42.151015625	-2.8085476158612e-10\\
42.1723046875	-2.27166423323378e-10\\
42.19359375	-1.70591443138839e-10\\
42.2148828125	-2.25285583627504e-10\\
42.236171875	-1.58211217754609e-10\\
42.2574609375	-1.73877472431099e-10\\
42.27875	-1.94320355939363e-10\\
42.3000390625	-2.53361373113642e-10\\
42.321328125	-2.99989584264006e-10\\
42.3426171875	-2.40435498013601e-10\\
42.36390625	-2.27666612239393e-10\\
42.3851953125	-1.86851314875763e-10\\
42.406484375	-1.59682191753286e-10\\
42.4277734375	-9.55125726208373e-11\\
42.4490625	-9.72570501065874e-11\\
42.4703515625	-1.15078907354809e-10\\
42.491640625	-7.00913679649101e-11\\
42.5129296875	-9.21417512620898e-11\\
42.53421875	-1.58756253726784e-10\\
42.5555078125	-1.50604309894526e-10\\
42.576796875	-1.49289809688844e-10\\
42.5980859375	-7.49946323641649e-11\\
42.619375	-1.83085158387984e-11\\
42.6406640625	-2.59399690860064e-11\\
42.661953125	5.09195873206197e-11\\
42.6832421875	2.01514865409473e-11\\
42.70453125	7.92549345559907e-11\\
42.7258203125	-3.32236030004635e-11\\
42.747109375	5.44919932695028e-11\\
42.7683984375	-1.81855339496337e-11\\
42.7896875	-1.67581512407736e-11\\
42.8109765625	-3.46950211889705e-11\\
42.832265625	5.60071337280547e-11\\
42.8535546875	1.6963044376617e-11\\
42.87484375	8.87970070582943e-11\\
42.8961328125	4.00943469026483e-11\\
42.917421875	5.71119924397235e-11\\
42.9387109375	1.51080571028799e-10\\
42.96	5.29059768277594e-11\\
42.9812890625	9.31310219355455e-11\\
43.002578125	6.04582516909351e-11\\
43.0238671875	3.74211921879328e-11\\
43.04515625	4.1628756049216e-11\\
43.0664453125	6.41096329653806e-11\\
43.087734375	6.2296653079519e-11\\
43.1090234375	2.47867698647124e-11\\
43.1303125	-5.64339233302818e-11\\
43.1516015625	-2.1059824307411e-11\\
43.172890625	-5.05236924890796e-11\\
43.1941796875	3.52738395405565e-11\\
43.21546875	1.05086275442423e-10\\
43.2367578125	1.66735122455943e-10\\
43.258046875	1.04445387158536e-10\\
43.2793359375	1.19050861573831e-10\\
43.300625	1.49274338071985e-10\\
43.3219140625	1.16800971656494e-10\\
43.343203125	1.14359045166638e-10\\
43.3644921875	9.21917207819751e-11\\
43.38578125	8.006226839491e-11\\
43.4070703125	1.14072841361535e-10\\
43.428359375	1.11330142860445e-10\\
43.4496484375	2.36718528040921e-10\\
43.4709375	2.01133590768485e-10\\
43.4922265625	2.92084036262478e-10\\
43.513515625	2.87159999688902e-10\\
43.5348046875	3.48933712562698e-10\\
43.55609375	2.77259131681684e-10\\
43.5773828125	3.84698984538053e-10\\
43.598671875	2.60227558725681e-10\\
43.6199609375	2.30726796728958e-10\\
43.64125	3.07856328978232e-10\\
43.6625390625	2.16372446667279e-10\\
43.683828125	2.85548811910464e-10\\
43.7051171875	3.28729575677689e-10\\
43.72640625	3.56024478691239e-10\\
43.7476953125	3.61490128583409e-10\\
43.768984375	3.91492795811695e-10\\
43.7902734375	3.92339930969701e-10\\
43.8115625	3.63559209389537e-10\\
43.8328515625	3.0614137318221e-10\\
43.854140625	2.91467353247698e-10\\
43.8754296875	2.56699384847458e-10\\
43.89671875	1.59906620980652e-10\\
43.9180078125	2.44778288316132e-10\\
43.939296875	2.17327783114972e-10\\
43.9605859375	2.56752340525391e-10\\
43.981875	2.4772065648573e-10\\
44.0031640625	3.78119416758209e-10\\
44.024453125	2.79664424380854e-10\\
44.0457421875	2.84596674696701e-10\\
44.06703125	2.37184992630409e-10\\
44.0883203125	2.17298746993377e-10\\
44.109609375	2.42303789924099e-10\\
44.1308984375	1.51331716226736e-10\\
44.1521875	1.88113143795726e-10\\
44.1734765625	2.13665408804574e-10\\
44.194765625	2.97489935439001e-10\\
44.2160546875	2.47116512885062e-10\\
44.23734375	3.6551161696042e-10\\
44.2586328125	2.3234105410924e-10\\
44.279921875	3.12704462387597e-10\\
44.3012109375	2.59790660095617e-10\\
44.3225	1.97892486892453e-10\\
44.3437890625	3.4612435011002e-10\\
44.365078125	2.72958387166521e-10\\
44.3863671875	2.86201935569937e-10\\
44.40765625	2.49971053927978e-10\\
44.4289453125	2.86482132703379e-10\\
44.450234375	3.15923093572811e-10\\
44.4715234375	3.81026469333829e-10\\
44.4928125	3.056579184548e-10\\
44.5141015625	3.06599622545502e-10\\
44.535390625	2.6117889061283e-10\\
44.5566796875	2.85614991604578e-10\\
44.57796875	2.43306586524304e-10\\
44.5992578125	2.68400314869954e-10\\
44.620546875	1.85254393974226e-10\\
44.6418359375	2.57046658555056e-10\\
44.663125	1.90519619617397e-10\\
44.6844140625	2.75672798044931e-10\\
44.705703125	2.61100137881529e-10\\
44.7269921875	2.68402947497765e-10\\
44.74828125	2.33819735191705e-10\\
44.7695703125	1.77328870581969e-10\\
44.790859375	2.04671631412902e-10\\
44.8121484375	1.55677811108701e-10\\
44.8334375	1.50958906158865e-10\\
44.8547265625	1.79754186926346e-10\\
44.876015625	2.72089140204209e-10\\
44.8973046875	2.07036173930595e-10\\
44.91859375	2.23817443172295e-10\\
44.9398828125	1.77237960087965e-10\\
44.961171875	1.68205783174071e-10\\
44.9824609375	1.40645789289222e-10\\
45.00375	1.40624677053702e-10\\
45.0250390625	1.5731280073018e-10\\
45.046328125	1.59067372923951e-10\\
45.0676171875	2.14348453601228e-10\\
45.08890625	2.58488452267647e-10\\
45.1101953125	1.63044224638978e-10\\
45.131484375	1.44829725617746e-10\\
45.1527734375	8.01415325577528e-11\\
45.1740625	1.09642943009563e-11\\
45.1953515625	-2.55777248381712e-11\\
45.216640625	6.05317126958304e-12\\
45.2379296875	-2.54205026821302e-11\\
45.25921875	-6.59685046036079e-11\\
45.2805078125	-3.06854634083e-11\\
45.301796875	-5.01791297287142e-11\\
45.3230859375	1.18181547773331e-11\\
45.344375	6.32452649972998e-12\\
45.3656640625	-4.41418822335738e-12\\
45.386953125	-1.79389799728878e-11\\
45.4082421875	-2.05675658020037e-11\\
45.42953125	-3.29530825597936e-11\\
45.4508203125	-6.65365644927157e-11\\
45.472109375	-2.36207002004853e-11\\
45.4933984375	-1.11531251921438e-10\\
45.5146875	-5.8837750989759e-11\\
45.5359765625	4.90972635104895e-11\\
45.557265625	-2.45170865684303e-11\\
45.5785546875	1.0207524017454e-11\\
45.59984375	-1.28404978377437e-11\\
45.6211328125	-4.10734970959147e-11\\
45.642421875	-5.54294078194594e-11\\
45.6637109375	-2.54978489970842e-11\\
45.685	-8.76936509181234e-12\\
45.7062890625	-4.12965687808031e-11\\
45.727578125	-1.48486007411895e-10\\
45.7488671875	-1.2816700017565e-10\\
45.77015625	-1.34520293752426e-10\\
45.7914453125	-1.8375468971749e-10\\
45.812734375	-1.77982941393657e-10\\
45.8340234375	-2.04090209021142e-10\\
45.8553125	-2.18756231519253e-10\\
45.8766015625	-1.89902257242416e-10\\
45.897890625	-2.18804376993779e-10\\
45.9191796875	-1.90372797065334e-10\\
45.94046875	-2.0666433308209e-10\\
45.9617578125	-2.19218469695629e-10\\
45.983046875	-1.689645984553e-10\\
46.0043359375	-2.34365809779885e-10\\
46.025625	-1.70130015881475e-10\\
46.0469140625	-2.34729632770391e-10\\
46.068203125	-1.93913578451632e-10\\
46.0894921875	-2.4798106127558e-10\\
46.11078125	-2.04703056011748e-10\\
46.1320703125	-2.37755649571451e-10\\
46.153359375	-2.40164793223722e-10\\
46.1746484375	-2.26482634461189e-10\\
46.1959375	-2.83270473296104e-10\\
46.2172265625	-1.66701778994503e-10\\
46.238515625	-2.34927706676169e-10\\
46.2598046875	-1.61442814664326e-10\\
46.28109375	-2.38227470974303e-10\\
46.3023828125	-2.71558844910216e-10\\
46.323671875	-2.57512563450225e-10\\
46.3449609375	-2.20164375745581e-10\\
46.36625	-2.55868856020625e-10\\
46.3875390625	-1.4918541648795e-10\\
46.408828125	-1.53170481187185e-10\\
46.4301171875	-1.91238177035142e-10\\
46.45140625	-1.60515817490733e-10\\
46.4726953125	-2.16064620502811e-10\\
46.493984375	-1.84799586442864e-10\\
46.5152734375	-2.49477905387068e-10\\
46.5365625	-1.75426872256769e-10\\
46.5578515625	-2.48522021504494e-10\\
46.579140625	-1.95736312245962e-10\\
46.6004296875	-1.57663349415874e-10\\
46.62171875	-2.26409171010932e-10\\
46.6430078125	-1.53012878014532e-10\\
46.664296875	-1.78700332091571e-10\\
46.6855859375	-2.55128882037008e-10\\
46.706875	-2.51580890145035e-10\\
46.7281640625	-3.45592152156353e-10\\
46.749453125	-2.91575267321658e-10\\
46.7707421875	-3.28443638580461e-10\\
46.79203125	-3.40983100702466e-10\\
46.8133203125	-3.55252073097681e-10\\
46.834609375	-3.28435179668752e-10\\
46.8558984375	-3.43611230812063e-10\\
46.8771875	-3.90498307543257e-10\\
46.8984765625	-3.04729450362546e-10\\
46.919765625	-3.71831468433809e-10\\
46.9410546875	-3.14906826014973e-10\\
46.96234375	-3.83722147812479e-10\\
46.9836328125	-3.67622030169865e-10\\
47.004921875	-3.48175123135226e-10\\
47.0262109375	-2.62387421647817e-10\\
47.0475	-2.87704520041085e-10\\
47.0687890625	-2.79849072356951e-10\\
47.090078125	-2.53592943202691e-10\\
47.1113671875	-2.48976235421811e-10\\
47.13265625	-2.61348193108227e-10\\
47.1539453125	-2.27506801168161e-10\\
47.175234375	-2.37371653884113e-10\\
47.1965234375	-2.4533483974451e-10\\
47.2178125	-1.76691603022769e-10\\
47.2391015625	-2.24352947671766e-10\\
47.260390625	-2.0309137059141e-10\\
47.2816796875	-2.22932872431004e-10\\
47.30296875	-1.9128932209511e-10\\
47.3242578125	-1.56051721689726e-10\\
47.345546875	-9.60686637621614e-11\\
47.3668359375	-1.94117175898276e-10\\
47.388125	-1.79841294664403e-10\\
47.4094140625	-2.10645140208371e-10\\
47.430703125	-2.17437766495418e-10\\
47.4519921875	-2.03971290793268e-10\\
47.47328125	-1.27497279660821e-10\\
47.4945703125	-1.25689196248482e-10\\
47.515859375	-1.66800690430005e-10\\
47.5371484375	-2.04993302722348e-10\\
47.5584375	-1.52297477069618e-10\\
47.5797265625	-2.09342559801917e-10\\
47.601015625	-2.08511979799738e-10\\
47.6223046875	-2.80238141576194e-10\\
47.64359375	-1.52279912085986e-10\\
47.6648828125	-1.64002676580756e-10\\
47.686171875	-1.05773064274242e-10\\
47.7074609375	-8.32327662145007e-11\\
47.72875	-3.09932136091722e-11\\
47.7500390625	-2.18876542001032e-11\\
47.771328125	-2.68167391968544e-11\\
47.7926171875	-6.14521411230485e-11\\
47.81390625	-1.00528557059673e-10\\
47.8351953125	-9.43855924195391e-11\\
47.856484375	-1.09219351845826e-10\\
47.8777734375	-3.68202494835339e-11\\
47.8990625	-1.29081793885371e-10\\
47.9203515625	4.59319158113165e-11\\
47.941640625	4.26619060037908e-11\\
47.9629296875	-1.19928108016847e-11\\
47.98421875	4.2364086409181e-11\\
48.0055078125	-4.6423671312738e-11\\
48.026796875	-6.3346025297926e-11\\
48.0480859375	-2.79353733889074e-13\\
48.069375	-2.16390398675656e-11\\
48.0906640625	-5.06269270998746e-12\\
48.111953125	5.98435563914766e-11\\
48.1332421875	1.47926259916502e-10\\
48.15453125	1.25550066268078e-10\\
48.1758203125	6.84336304482438e-11\\
48.197109375	6.9355726347231e-11\\
48.2183984375	6.83599519227746e-11\\
48.2396875	6.83167660170569e-11\\
48.2609765625	-1.24211578760006e-12\\
48.282265625	-3.64005100267e-11\\
48.3035546875	1.02431260406582e-10\\
48.32484375	8.13845039279955e-11\\
48.3461328125	1.21993149748496e-10\\
48.367421875	1.68121851674469e-10\\
48.3887109375	1.45779594503001e-10\\
48.41	1.38679833819596e-10\\
48.4312890625	1.23140159298158e-10\\
48.452578125	4.17248588943406e-11\\
48.4738671875	1.12387129719919e-10\\
48.49515625	1.69024985960387e-11\\
48.5164453125	8.36742579381939e-11\\
48.537734375	1.04090366072926e-10\\
48.5590234375	1.36559128318998e-10\\
48.5803125	1.84364900480462e-10\\
48.6016015625	1.28557954728763e-10\\
48.622890625	1.97850069033541e-10\\
48.6441796875	2.52893026006406e-10\\
48.66546875	2.57134123722471e-10\\
48.6867578125	2.50349099530084e-10\\
48.708046875	1.01413053235935e-10\\
48.7293359375	1.92620053678699e-10\\
48.750625	1.36418858370538e-10\\
48.7719140625	8.93122843927843e-11\\
48.793203125	2.43680871217418e-10\\
48.8144921875	2.38752138825793e-10\\
48.83578125	2.54096458192264e-10\\
48.8570703125	2.50152453090734e-10\\
48.878359375	3.15548046178141e-10\\
48.8996484375	2.6786002661726e-10\\
48.9209375	2.55859963086497e-10\\
48.9422265625	1.64549426136144e-10\\
48.963515625	2.30907142681751e-10\\
48.9848046875	1.64212160962291e-10\\
49.00609375	1.49620130035252e-10\\
49.0273828125	1.74825113919605e-10\\
49.048671875	1.66226663799521e-10\\
49.0699609375	2.9193652260084e-10\\
49.09125	1.78276692939182e-10\\
49.1125390625	3.4941147891677e-10\\
49.133828125	2.32719575180486e-10\\
49.1551171875	1.9748833410038e-10\\
49.17640625	2.84634053038023e-10\\
49.1976953125	2.21800909608748e-10\\
49.218984375	2.04957652661936e-10\\
49.2402734375	1.79107962660431e-10\\
49.2615625	1.59066007948993e-10\\
49.2828515625	2.74225632567935e-10\\
49.304140625	2.46957036814852e-10\\
49.3254296875	3.10351122450438e-10\\
49.34671875	3.79500183750468e-10\\
49.3680078125	3.2161741446056e-10\\
49.389296875	2.78180461879868e-10\\
49.4105859375	2.38962485808225e-10\\
49.431875	2.40052115660292e-10\\
49.4531640625	2.44411044040738e-10\\
49.474453125	2.38049610343122e-10\\
49.4957421875	2.81046749022433e-10\\
49.51703125	2.74801777658964e-10\\
49.5383203125	3.27385663445488e-10\\
49.559609375	3.02246791927881e-10\\
49.5808984375	3.35506115475972e-10\\
49.6021875	2.51061843251357e-10\\
49.6234765625	3.09277367272402e-10\\
49.644765625	2.53365316352147e-10\\
49.6660546875	1.91444884075003e-10\\
49.68734375	2.67292483486354e-10\\
49.7086328125	1.48025188407762e-10\\
49.729921875	1.60568073531875e-10\\
49.7512109375	1.38380176818836e-10\\
49.7725	1.90123053001622e-10\\
49.7937890625	2.10648576035502e-10\\
49.815078125	1.29383875933727e-10\\
49.8363671875	1.79890106212587e-10\\
49.85765625	1.53341233711639e-10\\
49.8789453125	1.7180238380845e-10\\
49.900234375	8.79240609392944e-11\\
49.9215234375	1.06021159245298e-10\\
49.9428125	1.04311225406721e-10\\
49.9641015625	1.24459634096145e-10\\
49.985390625	5.97089299498318e-11\\
50.0066796875	8.55035333234255e-11\\
50.02796875	1.21298276463263e-10\\
50.0492578125	7.5732678257494e-11\\
50.070546875	2.98608124731032e-11\\
50.0918359375	5.8007765042326e-11\\
50.113125	8.15124668651816e-11\\
50.1344140625	8.14313220191218e-11\\
50.155703125	8.09876363619656e-11\\
50.1769921875	7.74043783504518e-11\\
50.19828125	1.04848872721468e-10\\
50.2195703125	1.17591767185976e-11\\
50.240859375	6.90358592695009e-11\\
50.2621484375	6.52923442353459e-12\\
50.2834375	-1.49408079436833e-11\\
50.3047265625	-3.26643731998282e-11\\
50.326015625	-8.16049296753419e-11\\
50.3473046875	-5.77518456944725e-11\\
50.36859375	-9.8294455233826e-11\\
50.3898828125	-2.0414537544997e-10\\
50.411171875	-9.77745677074983e-11\\
50.4324609375	-1.99462546106917e-10\\
50.45375	-1.84369337703825e-10\\
50.4750390625	-1.97948553531878e-10\\
50.496328125	-1.83442743487062e-10\\
50.5176171875	-1.67679237416117e-10\\
50.53890625	-1.72263558470533e-10\\
50.5601953125	-1.75967394046843e-10\\
50.581484375	-7.84678111027892e-11\\
50.6027734375	-1.72161471787651e-10\\
50.6240625	-1.63066269760809e-10\\
50.6453515625	-2.36001653417905e-10\\
50.666640625	-2.84046229538639e-10\\
50.6879296875	-2.56427032392404e-10\\
50.70921875	-2.64954693714725e-10\\
50.7305078125	-2.28244907041225e-10\\
50.751796875	-2.38019334772541e-10\\
50.7730859375	-1.81451038491566e-10\\
50.794375	-2.39573507263055e-10\\
50.8156640625	-2.61929679246982e-10\\
50.836953125	-1.74266739774705e-10\\
50.8582421875	-2.21195454284719e-10\\
50.87953125	-2.20164027344202e-10\\
50.9008203125	-2.10520751111072e-10\\
50.922109375	-2.58401381160312e-10\\
50.9433984375	-2.7919373674512e-10\\
50.9646875	-2.78253698453798e-10\\
50.9859765625	-2.599452824352e-10\\
51.007265625	-2.7358290468343e-10\\
51.0285546875	-2.60252279710257e-10\\
51.04984375	-3.0994758307116e-10\\
51.0711328125	-2.79681590602945e-10\\
51.092421875	-2.670234979938e-10\\
51.1137109375	-1.88194288377554e-10\\
51.135	-2.39310747335429e-10\\
51.1562890625	-1.9669869209307e-10\\
51.177578125	-3.15088860331032e-10\\
51.1988671875	-2.17292337735983e-10\\
51.22015625	-2.37918097278018e-10\\
51.2414453125	-2.38565334973043e-10\\
51.262734375	-2.53137301471822e-10\\
51.2840234375	-2.66599667265788e-10\\
51.3053125	-2.65790852133498e-10\\
51.3266015625	-2.57895304544086e-10\\
51.347890625	-2.65196602178895e-10\\
51.3691796875	-3.09064333928095e-10\\
51.39046875	-3.52376469771193e-10\\
51.4117578125	-3.60668170976367e-10\\
51.433046875	-3.0894513900902e-10\\
51.4543359375	-3.2387619986069e-10\\
51.475625	-3.44090067681472e-10\\
51.4969140625	-2.49007502348571e-10\\
51.518203125	-2.50956239698999e-10\\
51.5394921875	-2.56806922584487e-10\\
51.56078125	-2.91086756428546e-10\\
51.5820703125	-2.66399405846179e-10\\
51.603359375	-3.16556125948015e-10\\
51.6246484375	-3.07304923501766e-10\\
51.6459375	-2.44697825127154e-10\\
51.6672265625	-2.68006816789647e-10\\
51.688515625	-3.16047752311857e-10\\
51.7098046875	-2.72554981497196e-10\\
51.73109375	-2.95315663360412e-10\\
51.7523828125	-2.72720086735172e-10\\
51.773671875	-2.37132186763865e-10\\
51.7949609375	-2.35149714240115e-10\\
51.81625	-1.86830596802056e-10\\
51.8375390625	-2.18156369653378e-10\\
51.858828125	-1.84972892965779e-10\\
51.8801171875	-1.36533775594845e-10\\
51.90140625	-1.18457847944038e-10\\
51.9226953125	-1.39106852084466e-10\\
51.943984375	-1.76874807281478e-10\\
51.9652734375	-1.88969411930988e-10\\
51.9865625	-1.91547646793971e-10\\
52.0078515625	-1.83446438053925e-10\\
52.029140625	-1.9423620484691e-10\\
52.0504296875	-2.12051981965194e-10\\
52.07171875	-2.18377484146638e-10\\
52.0930078125	-1.96283699814201e-10\\
52.114296875	-1.4679069605998e-10\\
52.1355859375	-1.87694159767384e-10\\
52.156875	-1.01555344105802e-10\\
52.1781640625	-1.33179120936437e-10\\
52.199453125	-1.72080450456383e-10\\
52.2207421875	-1.05529715333513e-10\\
52.24203125	-1.33504140657764e-10\\
52.2633203125	-7.79450823323604e-11\\
52.284609375	-1.35726431256703e-10\\
52.3058984375	-6.74193464394329e-11\\
52.3271875	-1.32103352627948e-10\\
52.3484765625	-1.20757128443627e-10\\
52.369765625	-7.86613506696919e-11\\
52.3910546875	-6.70653017839171e-12\\
52.41234375	-8.61985376231187e-11\\
52.4336328125	-8.16256286209276e-11\\
52.454921875	-4.99619499268788e-11\\
52.4762109375	-7.72216281199818e-11\\
52.4975	-1.21123129042339e-10\\
52.5187890625	-1.51202095126831e-10\\
52.540078125	-6.02357785308662e-11\\
52.5613671875	-1.32895339880103e-10\\
52.58265625	-1.18335511745284e-10\\
52.6039453125	-1.27440790035695e-10\\
52.625234375	-7.85102098067599e-11\\
52.6465234375	-5.70132222450488e-11\\
52.6678125	-1.3024492535736e-10\\
52.6891015625	-1.39036112366362e-10\\
52.710390625	-1.11102400680516e-10\\
52.7316796875	-9.43473083749975e-11\\
52.75296875	-1.08826230792132e-10\\
52.7742578125	-7.97710122191311e-11\\
52.795546875	-4.2428377575374e-11\\
52.8168359375	9.47089886882609e-13\\
52.838125	4.4924146887733e-11\\
52.8594140625	1.19200228412208e-10\\
52.880703125	4.84212713118084e-11\\
52.9019921875	3.25744565476586e-11\\
52.92328125	-1.48859353863952e-12\\
52.9445703125	1.420841367472e-11\\
52.965859375	-4.26305902150627e-12\\
52.9871484375	7.66828712354761e-11\\
53.0084375	3.88496413858881e-11\\
53.0297265625	1.65259354067318e-10\\
53.051015625	1.16995777317922e-10\\
53.0723046875	1.17330236812863e-10\\
53.09359375	1.42280603533921e-10\\
53.1148828125	1.02879078921591e-10\\
53.136171875	6.70078418432896e-11\\
53.1574609375	6.52333537814536e-11\\
53.17875	1.68466333484758e-11\\
53.2000390625	7.66801153544622e-11\\
53.221328125	9.36542028133066e-11\\
53.2426171875	1.28466930848081e-10\\
53.26390625	9.83331952499334e-11\\
53.2851953125	8.13598600377335e-11\\
53.306484375	1.5649509242295e-10\\
53.3277734375	1.20193342081847e-10\\
53.3490625	7.79129190283336e-12\\
53.3703515625	-1.11174721602284e-11\\
53.391640625	-4.20948261501183e-11\\
53.4129296875	4.35544551237123e-12\\
53.43421875	-3.71125668359601e-12\\
53.4555078125	3.14407756106331e-12\\
53.476796875	3.95937466521631e-11\\
53.4980859375	7.18732561148173e-11\\
53.519375	8.40676029928188e-11\\
53.5406640625	3.12061046407164e-11\\
53.561953125	3.55911658701554e-11\\
53.5832421875	2.13842937561875e-11\\
53.60453125	-5.76439608275325e-11\\
53.6258203125	-6.96276330932273e-11\\
53.647109375	-4.97016675377541e-11\\
53.6683984375	3.41805664056753e-12\\
53.6896875	-1.97648262726032e-12\\
53.7109765625	-3.32760329062001e-11\\
53.732265625	3.18079784138847e-11\\
53.7535546875	-5.89924343989572e-12\\
53.77484375	-4.70309899455199e-11\\
53.7961328125	-6.1342971141449e-11\\
53.817421875	-8.5982583972006e-11\\
53.8387109375	-6.35067561328459e-11\\
53.86	1.4381976220122e-11\\
53.8812890625	2.87361263174467e-11\\
53.902578125	1.09516489792638e-10\\
53.9238671875	8.57997695816678e-11\\
53.94515625	9.39927775009447e-11\\
53.9664453125	4.62075876470051e-11\\
53.987734375	1.27705487281712e-10\\
54.0090234375	6.56933515238292e-11\\
54.0303125	8.52487478068955e-11\\
54.0516015625	6.976300210215e-11\\
54.072890625	2.5890849570028e-11\\
54.0941796875	4.97242557178808e-11\\
54.11546875	4.69528078204146e-11\\
54.1367578125	6.03461833308819e-11\\
54.158046875	4.08054563683182e-11\\
54.1793359375	1.05165850332533e-10\\
54.200625	4.74696153459652e-11\\
54.2219140625	3.70962120696454e-12\\
54.243203125	1.22523153369414e-10\\
54.2644921875	3.9350299524326e-11\\
54.28578125	5.55342271594244e-11\\
54.3070703125	3.57749132935661e-11\\
54.328359375	-8.58590369007052e-12\\
54.3496484375	3.66773016322348e-11\\
54.3709375	-8.9623536327923e-13\\
54.3922265625	1.50794212121639e-11\\
54.413515625	3.08673476545425e-11\\
54.4348046875	3.97442710680819e-11\\
54.45609375	2.72127668470473e-11\\
54.4773828125	4.35769858735044e-11\\
54.498671875	5.55927197633464e-11\\
54.5199609375	2.75323333080045e-11\\
54.54125	2.53407275512783e-11\\
54.5625390625	2.39138935207223e-11\\
54.583828125	3.1484542509114e-11\\
54.6051171875	4.63965287771756e-11\\
54.62640625	2.82886401892076e-11\\
54.6476953125	5.80484476347894e-11\\
54.668984375	8.82579702251461e-11\\
54.6902734375	3.92392306633563e-11\\
54.7115625	-4.38913640105162e-12\\
54.7328515625	6.19031702924719e-11\\
54.754140625	9.9354020084183e-12\\
54.7754296875	3.58053474089429e-11\\
54.79671875	3.38859108487732e-11\\
54.8180078125	-4.87249144835964e-11\\
54.839296875	2.33213073039135e-11\\
54.8605859375	-8.89912515033651e-11\\
54.881875	-1.87347828800954e-11\\
54.9031640625	-7.07397374891286e-11\\
54.924453125	-2.59789143478251e-11\\
54.9457421875	-2.4596286129698e-11\\
54.96703125	-7.37168195655457e-11\\
54.9883203125	8.85395102947459e-11\\
55.009609375	1.40236833255344e-11\\
55.0308984375	2.68114439984891e-11\\
55.0521875	3.64359972655425e-11\\
55.0734765625	1.05059110564841e-11\\
55.094765625	1.90303961999225e-11\\
55.1160546875	6.84411542286255e-11\\
55.13734375	4.10576150751371e-11\\
55.1586328125	5.66382623082111e-11\\
55.179921875	2.88408614806518e-11\\
55.2012109375	4.6045484135292e-11\\
55.2225	4.2793621533873e-11\\
55.2437890625	2.17194958059111e-11\\
55.265078125	-3.23450617016479e-11\\
55.2863671875	3.4045877494629e-11\\
55.30765625	-9.30664564399474e-12\\
55.3289453125	-2.22350038269891e-11\\
55.350234375	-5.22825863129782e-11\\
55.3715234375	-1.71937529020843e-11\\
55.3928125	-8.13505887506194e-11\\
55.4141015625	-4.64586332150933e-11\\
55.435390625	-5.01567642444771e-11\\
55.4566796875	-2.74881634614578e-11\\
55.47796875	-4.05189008068012e-11\\
55.4992578125	-1.10284185133706e-10\\
55.520546875	-6.76513437926942e-11\\
55.5418359375	-1.38384177682421e-10\\
55.563125	-1.08488312376556e-10\\
55.5844140625	-1.07642490793638e-10\\
55.605703125	-1.12796146960985e-10\\
55.6269921875	-1.48809509906514e-10\\
55.64828125	-1.01158772285226e-10\\
55.6695703125	-1.45338132926043e-10\\
55.690859375	-2.25845114421461e-11\\
55.7121484375	-1.13244895803736e-10\\
55.7334375	-9.1734352422172e-11\\
55.7547265625	-1.10381187622671e-10\\
55.776015625	-1.02619504130089e-10\\
55.7973046875	-7.24627675088653e-11\\
55.81859375	-1.318055273422e-10\\
55.8398828125	-7.52430125006117e-11\\
55.861171875	-1.23290703940252e-10\\
55.8824609375	-7.25142780454313e-11\\
55.90375	-9.65929600030472e-11\\
55.9250390625	-3.93930858744981e-11\\
55.946328125	-7.08988394543992e-11\\
55.9676171875	-3.11841591482985e-11\\
55.98890625	-2.36416452071595e-11\\
56.0101953125	-3.77958634649023e-11\\
56.031484375	1.55470169360926e-11\\
56.0527734375	-8.05494394979234e-11\\
56.0740625	-1.47225446953159e-11\\
56.0953515625	-2.38842332676461e-11\\
56.116640625	1.42347947374852e-11\\
56.1379296875	-2.08452962165995e-11\\
56.15921875	-6.89710770795322e-12\\
56.1805078125	1.18698528221413e-11\\
56.201796875	2.57433684852129e-11\\
56.2230859375	9.35308157101708e-12\\
56.244375	-1.96596401346099e-12\\
56.2656640625	5.4199076336378e-11\\
56.286953125	1.93129764619557e-11\\
56.3082421875	3.28858045285789e-11\\
56.32953125	4.16507507226336e-11\\
56.3508203125	6.70334245725394e-11\\
56.372109375	-1.39373160436613e-11\\
56.3933984375	1.85171568119809e-12\\
56.4146875	1.5303969674034e-12\\
56.4359765625	2.35064356860054e-11\\
56.457265625	-1.90151523039946e-11\\
56.4785546875	-2.76505865989347e-11\\
56.49984375	-5.46609082796067e-11\\
56.5211328125	-5.56818240662938e-12\\
56.542421875	-1.65247488480773e-11\\
56.5637109375	-6.18056422592052e-11\\
56.585	-5.90358449675473e-11\\
56.6062890625	4.26082339526161e-12\\
56.627578125	3.15442186494099e-11\\
56.6488671875	-2.9524890434523e-11\\
56.67015625	1.61477753962055e-11\\
56.6914453125	1.11191437006492e-11\\
56.712734375	-3.92723870928728e-11\\
56.7340234375	-1.54718576948258e-11\\
56.7553125	-5.8254367775279e-11\\
56.7766015625	-8.25980869384452e-12\\
56.797890625	-6.31921630233898e-11\\
};
\addlegendentry{$\text{train 4 -\textgreater{} Trondheim}$};

\end{axis}
\end{tikzpicture}%	
	\label{fig:train4}
\end{subfigure}

\begin{subfigure}[t]{0.45\textwidth}
	\centering
	% This file was created by matlab2tikz.
%
%The latest updates can be retrieved from
%  http://www.mathworks.com/matlabcentral/fileexchange/22022-matlab2tikz-matlab2tikz
%where you can also make suggestions and rate matlab2tikz.
%
\definecolor{mycolor1}{rgb}{0.00000,0.44700,0.74100}%
%
\begin{tikzpicture}

\begin{axis}[%
width=\textwidth,
height=\textwidth,
at={(0\figurewidth,0\figureheight)},
scale only axis,
xmin=-60,
xmax=60,
ymin=-4e-09,
ymax=1e-08,
axis background/.style={fill=white},
% title style={font=\bfseries},
% title={Influencelines for train 5, middle sensor},
legend style={legend cell align=left,align=left,draw=white!15!black}
]
\addplot [color=mycolor1,solid,forget plot]
  table[row sep=crcr]{%
-42.122044921875	-6.36711615647638e-10\\
-42.10205078125	-6.94657200635323e-10\\
-42.082056640625	-7.20491132803054e-10\\
-42.0620625	-7.0630087107732e-10\\
-42.042068359375	-7.83475183126038e-10\\
-42.02207421875	-7.28434238760549e-10\\
-42.002080078125	-7.94166577154994e-10\\
-41.9820859375	-7.71298334022731e-10\\
-41.962091796875	-8.21755087169326e-10\\
-41.94209765625	-8.8442288518353e-10\\
-41.922103515625	-9.48851713001819e-10\\
-41.902109375	-1.03313144740452e-09\\
-41.882115234375	-9.9092612001472e-10\\
-41.86212109375	-1.01436263796855e-09\\
-41.842126953125	-9.56753795601444e-10\\
-41.8221328125	-9.95512405418778e-10\\
-41.802138671875	-9.5007243766237e-10\\
-41.78214453125	-9.05966089858972e-10\\
-41.762150390625	-8.87240128310315e-10\\
-41.74215625	-8.15867557969108e-10\\
-41.722162109375	-7.62018375407651e-10\\
-41.70216796875	-8.3991060248494e-10\\
-41.682173828125	-8.05749886675556e-10\\
-41.6621796875	-8.4720355262605e-10\\
-41.642185546875	-8.96766636609685e-10\\
-41.62219140625	-8.41276264496886e-10\\
-41.602197265625	-9.24204467876266e-10\\
-41.582203125	-9.06595383810749e-10\\
-41.562208984375	-9.25131042064211e-10\\
-41.54221484375	-9.30899587609836e-10\\
-41.522220703125	-8.41160904101124e-10\\
-41.5022265625	-8.23975948374616e-10\\
-41.482232421875	-8.43204760138811e-10\\
-41.46223828125	-8.19386621144996e-10\\
-41.442244140625	-7.65363027028711e-10\\
-41.42225	-7.82557326390553e-10\\
-41.402255859375	-7.82418322618448e-10\\
-41.38226171875	-7.57148628666961e-10\\
-41.362267578125	-8.03113673220812e-10\\
-41.3422734375	-8.16470452936008e-10\\
-41.322279296875	-7.98240914392697e-10\\
-41.30228515625	-8.49023721488167e-10\\
-41.282291015625	-7.92136724910609e-10\\
-41.262296875	-7.90766229107266e-10\\
-41.242302734375	-7.58145829861446e-10\\
-41.22230859375	-6.96472422372365e-10\\
-41.202314453125	-6.46745295093059e-10\\
-41.1823203125	-7.07042489504156e-10\\
-41.162326171875	-6.88584933855046e-10\\
-41.14233203125	-7.31672779038377e-10\\
-41.122337890625	-7.8185979296303e-10\\
-41.10234375	-7.45914603611069e-10\\
-41.082349609375	-7.31033899051653e-10\\
-41.06235546875	-7.69631412824923e-10\\
-41.042361328125	-7.02577596248465e-10\\
-41.0223671875	-5.7418036024572e-10\\
-41.002373046875	-4.95507066296161e-10\\
-40.98237890625	-4.80933372310137e-10\\
-40.962384765625	-4.25552044292854e-10\\
-40.942390625	-3.98697285150774e-10\\
-40.922396484375	-5.08695097176722e-10\\
-40.90240234375	-5.35800048611741e-10\\
-40.882408203125	-5.18277313436753e-10\\
-40.8624140625	-6.23933434180161e-10\\
-40.842419921875	-5.84677936756177e-10\\
-40.82242578125	-4.81355376513295e-10\\
-40.802431640625	-4.56427772366349e-10\\
-40.7824375	-4.76599198040946e-10\\
-40.762443359375	-3.56716219608645e-10\\
-40.74244921875	-3.37434202565744e-10\\
-40.722455078125	-4.1475764834427e-10\\
-40.7024609375	-3.80657443647227e-10\\
-40.682466796875	-5.03862607136507e-10\\
-40.66247265625	-4.92855506373994e-10\\
-40.642478515625	-4.94902357053211e-10\\
-40.622484375	-4.67417132133816e-10\\
-40.602490234375	-4.32779371806154e-10\\
-40.58249609375	-3.61276295219484e-10\\
-40.562501953125	-4.16658024003546e-10\\
-40.5425078125	-2.8259088877512e-10\\
-40.522513671875	-2.6144866229099e-10\\
-40.50251953125	-3.44919510076703e-10\\
-40.482525390625	-2.90787470387237e-10\\
-40.46253125	-3.53723218657425e-10\\
-40.442537109375	-3.90553970575117e-10\\
-40.42254296875	-4.41310144748519e-10\\
-40.402548828125	-4.01015937985221e-10\\
-40.3825546875	-3.59516755777636e-10\\
-40.362560546875	-3.07927456011718e-10\\
-40.34256640625	-2.87167202557025e-10\\
-40.322572265625	-1.91903668345051e-10\\
-40.302578125	-1.73513963430295e-10\\
-40.282583984375	-2.69446389379091e-10\\
-40.26258984375	-2.52072048745655e-10\\
-40.242595703125	-2.3045510522733e-10\\
-40.2226015625	-3.17218089766929e-10\\
-40.202607421875	-3.20541223589898e-10\\
-40.18261328125	-3.80736055686952e-10\\
-40.162619140625	-2.78646662522191e-10\\
-40.142625	-3.38452453130159e-10\\
-40.122630859375	-2.99995188965512e-10\\
-40.10263671875	-2.93953745253606e-10\\
-40.082642578125	-2.54186212488497e-10\\
-40.0626484375	-3.01751413734835e-10\\
-40.042654296875	-2.34204879313374e-10\\
-40.02266015625	-2.77542852326181e-10\\
-40.002666015625	-2.81413191379824e-10\\
-39.982671875	-2.96825100610439e-10\\
-39.962677734375	-3.48621457728537e-10\\
-39.94268359375	-3.619817841221e-10\\
-39.922689453125	-3.95431763464876e-10\\
-39.9026953125	-3.97230524121624e-10\\
-39.882701171875	-3.73260857688856e-10\\
-39.86270703125	-3.07580573132698e-10\\
-39.842712890625	-2.81626615585108e-10\\
-39.82271875	-3.26422802773755e-10\\
-39.802724609375	-2.85054503232998e-10\\
-39.78273046875	-2.92205290483529e-10\\
-39.762736328125	-3.23627529913731e-10\\
-39.7427421875	-2.31840653959845e-10\\
-39.722748046875	-2.45901769965464e-10\\
-39.70275390625	-1.27759233541803e-10\\
-39.682759765625	-1.47878008364698e-10\\
-39.662765625	-1.86216959511758e-10\\
-39.642771484375	-3.26025262171294e-11\\
-39.62277734375	2.04084539821345e-10\\
-39.602783203125	6.88061995054516e-11\\
-39.5827890625	3.47170801665962e-10\\
-39.562794921875	4.65268137283909e-10\\
-39.54280078125	4.87566845504653e-10\\
-39.522806640625	5.40556275605837e-10\\
-39.5028125	6.06153535411739e-10\\
-39.482818359375	5.44659115994708e-10\\
-39.46282421875	5.91387665094854e-10\\
-39.442830078125	6.49812912393328e-10\\
-39.4228359375	7.34482758216857e-10\\
-39.402841796875	7.35284455150824e-10\\
-39.38284765625	7.4456594154429e-10\\
-39.362853515625	7.92760693925352e-10\\
-39.342859375	7.66419819629553e-10\\
-39.322865234375	7.42421039873288e-10\\
-39.30287109375	7.09890215124637e-10\\
-39.282876953125	6.984290340155e-10\\
-39.2628828125	6.41886083860263e-10\\
-39.242888671875	6.89947946159618e-10\\
-39.22289453125	6.47724194220416e-10\\
-39.202900390625	6.75532128322439e-10\\
-39.18290625	6.37392885047407e-10\\
-39.162912109375	6.14043001044907e-10\\
-39.14291796875	6.18744749306428e-10\\
-39.122923828125	5.41614326845042e-10\\
-39.1029296875	5.27786449629539e-10\\
-39.082935546875	5.37424010155363e-10\\
-39.06294140625	5.04594826456211e-10\\
-39.042947265625	4.69184463389609e-10\\
-39.022953125	4.80911941518879e-10\\
-39.002958984375	5.41770748056919e-10\\
-38.98296484375	4.98922254349057e-10\\
-38.962970703125	5.70299829756446e-10\\
-38.9429765625	4.97379033736935e-10\\
-38.922982421875	6.65336930059731e-10\\
-38.90298828125	5.47812553582003e-10\\
-38.882994140625	3.83362380709407e-10\\
-38.863	4.24622018691609e-10\\
-38.843005859375	3.57187367144976e-10\\
-38.82301171875	2.88997737902264e-10\\
-38.803017578125	4.31098262081462e-10\\
-38.7830234375	3.98574260433607e-10\\
-38.763029296875	4.83524165946537e-10\\
-38.74303515625	4.7740963916441e-10\\
-38.723041015625	6.05874220819413e-10\\
-38.703046875	5.73997309416435e-10\\
-38.683052734375	5.85764192030221e-10\\
-38.66305859375	5.10501839106894e-10\\
-38.643064453125	5.26190522620959e-10\\
-38.6230703125	5.38011039630653e-10\\
-38.603076171875	5.33814360935282e-10\\
-38.58308203125	5.58224306598678e-10\\
-38.563087890625	6.80930546824679e-10\\
-38.54309375	5.56582439988806e-10\\
-38.523099609375	6.24449420957214e-10\\
-38.50310546875	5.55095020187697e-10\\
-38.483111328125	5.51665266963308e-10\\
-38.4631171875	4.73856569134245e-10\\
-38.443123046875	4.25430485602361e-10\\
-38.42312890625	4.32126055764189e-10\\
-38.403134765625	2.97692677928374e-10\\
-38.383140625	3.98998522797533e-10\\
-38.363146484375	3.78527299967212e-10\\
-38.34315234375	3.75193289709849e-10\\
-38.323158203125	3.98642389321404e-10\\
-38.3031640625	4.52316653011879e-10\\
-38.283169921875	4.38466638635457e-10\\
-38.26317578125	3.33947724359413e-10\\
-38.243181640625	3.53038318202656e-10\\
-38.2231875	1.90378507824888e-10\\
-38.203193359375	1.39426344180589e-10\\
-38.18319921875	1.54114098469398e-10\\
-38.163205078125	1.11940233904629e-10\\
-38.1432109375	1.68455951437599e-10\\
-38.123216796875	1.89125552197029e-10\\
-38.10322265625	1.75683295516135e-10\\
-38.083228515625	1.63866840496013e-10\\
-38.063234375	1.64589846115148e-10\\
-38.043240234375	1.24361929277433e-10\\
-38.02324609375	1.02650098964767e-10\\
-38.003251953125	2.08608053891449e-11\\
-37.9832578125	2.41293902096759e-11\\
-37.963263671875	-2.06294710203101e-11\\
-37.94326953125	3.27281994880827e-11\\
-37.923275390625	6.57721486577038e-11\\
-37.90328125	6.48200187220821e-11\\
-37.883287109375	8.46739428212078e-11\\
-37.86329296875	1.97841979171167e-11\\
-37.843298828125	5.54522945011197e-12\\
-37.8233046875	-2.90763253842257e-11\\
-37.803310546875	-1.04912100526681e-10\\
-37.78331640625	-1.23191697780851e-10\\
-37.763322265625	-1.53851466749229e-10\\
-37.743328125	-1.50963895041792e-10\\
-37.723333984375	-5.85913207653046e-11\\
-37.70333984375	-8.56935121485696e-11\\
-37.683345703125	7.14087278641389e-11\\
-37.6633515625	1.39163967163659e-10\\
-37.643357421875	1.2894618990868e-10\\
-37.62336328125	7.62372250279763e-11\\
-37.603369140625	5.06456571334916e-11\\
-37.583375	-3.89942193254899e-11\\
-37.563380859375	-3.26472121184995e-11\\
-37.54338671875	-1.27864160449981e-10\\
-37.523392578125	-9.11470021442527e-11\\
-37.5033984375	-1.31060435134998e-10\\
-37.483404296875	-1.6294323085666e-10\\
-37.46341015625	-5.96017362449199e-11\\
-37.443416015625	-3.46178570804556e-11\\
-37.423421875	3.53392936783062e-11\\
-37.403427734375	6.55376038142219e-11\\
-37.38343359375	7.3609619006351e-11\\
-37.363439453125	3.74879152971326e-11\\
-37.3434453125	3.21755415230661e-11\\
-37.323451171875	4.76704045311935e-12\\
-37.30345703125	-4.76133861181266e-12\\
-37.283462890625	-2.70995437133826e-12\\
-37.26346875	-5.92343024251348e-11\\
-37.243474609375	-1.69878076175132e-10\\
-37.22348046875	-2.41003370667842e-11\\
-37.203486328125	-1.30634460312452e-10\\
-37.1834921875	-2.01635384550615e-11\\
-37.163498046875	-5.46850399869658e-11\\
-37.14350390625	-1.15514091391284e-10\\
-37.123509765625	-1.59093484988544e-10\\
-37.103515625	-2.0178755891662e-10\\
-37.083521484375	-4.34109285405005e-10\\
-37.06352734375	-4.61029337187082e-10\\
-37.043533203125	-5.851423555645e-10\\
-37.0235390625	-6.74185148990892e-10\\
-37.003544921875	-7.93009444301687e-10\\
-36.98355078125	-7.3327410157888e-10\\
-36.963556640625	-7.4054088936749e-10\\
-36.9435625	-8.49590953779038e-10\\
-36.923568359375	-7.69683093718468e-10\\
-36.90357421875	-9.38536671514739e-10\\
-36.883580078125	-9.74555843805639e-10\\
-36.8635859375	-1.03513573071809e-09\\
-36.843591796875	-1.006808528567e-09\\
-36.82359765625	-1.00089520168648e-09\\
-36.803603515625	-1.00636677653681e-09\\
-36.783609375	-1.01667115937808e-09\\
-36.763615234375	-8.72979833872775e-10\\
-36.74362109375	-9.01889249105896e-10\\
-36.723626953125	-7.55739856197368e-10\\
-36.7036328125	-8.22786449545075e-10\\
-36.683638671875	-7.71952939728002e-10\\
-36.66364453125	-8.19772774456398e-10\\
-36.643650390625	-7.11180062367246e-10\\
-36.62365625	-6.88832063317933e-10\\
-36.603662109375	-7.34575864684741e-10\\
-36.58366796875	-7.24354268009383e-10\\
-36.563673828125	-6.46433882076658e-10\\
-36.5436796875	-6.56556870286679e-10\\
-36.523685546875	-5.95415936774408e-10\\
-36.50369140625	-5.02425801634152e-10\\
-36.483697265625	-5.20792909762858e-10\\
-36.463703125	-5.46098696152927e-10\\
-36.443708984375	-4.80205453991569e-10\\
-36.42371484375	-6.33583349096295e-10\\
-36.403720703125	-5.13852066154055e-10\\
-36.3837265625	-6.28407662413542e-10\\
-36.363732421875	-6.82977755930444e-10\\
-36.34373828125	-6.31702020491335e-10\\
-36.323744140625	-6.01420432668859e-10\\
-36.30375	-6.95728976209286e-10\\
-36.283755859375	-5.2070592179746e-10\\
-36.26376171875	-6.34325832952304e-10\\
-36.243767578125	-5.02363458787421e-10\\
-36.2237734375	-6.68064999521782e-10\\
-36.203779296875	-6.32569920475819e-10\\
-36.18378515625	-7.00047576824564e-10\\
-36.163791015625	-6.68183041832438e-10\\
-36.143796875	-6.99660274511271e-10\\
-36.123802734375	-7.97418155752885e-10\\
-36.10380859375	-7.36160833960996e-10\\
-36.083814453125	-8.25212686812988e-10\\
-36.0638203125	-8.22162349406562e-10\\
-36.043826171875	-8.21615650208775e-10\\
-36.02383203125	-9.8416794491531e-10\\
-36.003837890625	-7.98656753136069e-10\\
-35.98384375	-9.63710157909908e-10\\
-35.963849609375	-8.6492386025324e-10\\
-35.94385546875	-7.93051222139389e-10\\
-35.923861328125	-8.01071593753657e-10\\
-35.9038671875	-8.28035433118467e-10\\
-35.883873046875	-7.49296375803986e-10\\
-35.86387890625	-6.89815319130866e-10\\
-35.843884765625	-6.78143989091015e-10\\
-35.823890625	-8.06952963662984e-10\\
-35.803896484375	-7.78445617463122e-10\\
-35.78390234375	-7.49770413209305e-10\\
-35.763908203125	-7.9163984215761e-10\\
-35.7439140625	-8.39923165997773e-10\\
-35.723919921875	-7.68323346476437e-10\\
-35.70392578125	-7.36111551636572e-10\\
-35.683931640625	-6.81536653902034e-10\\
-35.6639375	-6.57445805653767e-10\\
-35.643943359375	-6.07330806842391e-10\\
-35.62394921875	-5.45623973336933e-10\\
-35.603955078125	-5.5352236894806e-10\\
-35.5839609375	-5.14709205869166e-10\\
-35.563966796875	-4.3647340854112e-10\\
-35.54397265625	-4.78700505473577e-10\\
-35.523978515625	-4.58440000609489e-10\\
-35.503984375	-3.46065784647124e-10\\
-35.483990234375	-3.88565819104747e-10\\
-35.46399609375	-2.37444257897843e-10\\
-35.444001953125	-3.5186147430318e-10\\
-35.4240078125	-2.43157857336085e-10\\
-35.404013671875	-3.20935623634777e-10\\
-35.38401953125	-3.06769886991137e-10\\
-35.364025390625	-3.33137226954431e-10\\
-35.34403125	-3.36046813174787e-10\\
-35.324037109375	-2.76002161857847e-10\\
-35.30404296875	-2.22879401630464e-10\\
-35.284048828125	-2.08916761106225e-10\\
-35.2640546875	-1.93954701332309e-10\\
-35.244060546875	-1.23013587034024e-10\\
-35.22406640625	-9.10907126217708e-11\\
-35.204072265625	-7.84566289251771e-11\\
-35.184078125	-1.19103252411614e-10\\
-35.164083984375	-1.74438965246565e-10\\
-35.14408984375	-2.26757624864483e-10\\
-35.124095703125	-3.1995881494428e-10\\
-35.1041015625	-3.08231114312664e-10\\
-35.084107421875	-2.82541903959407e-10\\
-35.06411328125	-2.83603298825912e-10\\
-35.044119140625	-2.52365556486449e-10\\
-35.024125	-2.03058382856045e-10\\
-35.004130859375	-1.12175999325422e-10\\
-34.98413671875	-1.14472437201618e-10\\
-34.964142578125	-5.13305563671487e-11\\
-34.9441484375	-3.50946261433615e-11\\
-34.924154296875	2.26899623323423e-11\\
-34.90416015625	-5.44786470526566e-11\\
-34.884166015625	-1.10112932904195e-10\\
-34.864171875	-6.12461101687621e-11\\
-34.844177734375	-5.89144189903171e-11\\
-34.82418359375	-1.59210598302412e-10\\
-34.804189453125	-5.22636733456315e-11\\
-34.7841953125	-1.31404174562562e-10\\
-34.764201171875	-4.49641815502866e-11\\
-34.74420703125	-8.46883101724563e-12\\
-34.724212890625	5.20141726217625e-12\\
-34.70421875	1.48551021776595e-10\\
-34.684224609375	2.03492493147817e-10\\
-34.66423046875	2.6979607365974e-10\\
-34.644236328125	2.42470190740663e-10\\
-34.6242421875	3.4492311898939e-10\\
-34.604248046875	3.69594843702078e-10\\
-34.58425390625	3.78620541127883e-10\\
-34.564259765625	4.64785147730273e-10\\
-34.544265625	6.29183681380483e-10\\
-34.524271484375	7.08905311555129e-10\\
-34.50427734375	8.16002515128932e-10\\
-34.484283203125	8.46664867321784e-10\\
-34.4642890625	9.05443260272492e-10\\
-34.444294921875	9.59739807180887e-10\\
-34.42430078125	9.14335718973699e-10\\
-34.404306640625	1.09414724767285e-09\\
-34.3843125	9.71234844128619e-10\\
-34.364318359375	1.12314161217242e-09\\
-34.34432421875	1.17779862876251e-09\\
-34.324330078125	1.24528821568577e-09\\
-34.3043359375	1.27025718647978e-09\\
-34.284341796875	1.36592244133412e-09\\
-34.26434765625	1.30492098781711e-09\\
-34.244353515625	1.2938323833345e-09\\
-34.224359375	1.17215919593924e-09\\
-34.204365234375	1.16889182261377e-09\\
-34.18437109375	1.07300407235337e-09\\
-34.164376953125	1.03870820174055e-09\\
-34.1443828125	9.88970984730512e-10\\
-34.124388671875	1.02524846732479e-09\\
-34.10439453125	1.04520297882293e-09\\
-34.084400390625	9.51056107354644e-10\\
-34.06440625	1.04552709612731e-09\\
-34.044412109375	1.02080635639286e-09\\
-34.02441796875	1.04251782500785e-09\\
-34.004423828125	1.00690799588333e-09\\
-33.9844296875	9.67492699937697e-10\\
-33.964435546875	9.17095232004432e-10\\
-33.94444140625	8.45613670771432e-10\\
-33.924447265625	7.95514512044267e-10\\
-33.904453125	7.73372728081515e-10\\
-33.884458984375	8.39566350722159e-10\\
-33.86446484375	7.93532336162259e-10\\
-33.844470703125	8.52428377446333e-10\\
-33.8244765625	8.68277000692583e-10\\
-33.804482421875	8.86301219232487e-10\\
-33.78448828125	9.76043813669927e-10\\
-33.764494140625	1.00065225741374e-09\\
-33.7445	1.01997583885424e-09\\
-33.724505859375	1.1624126935027e-09\\
-33.70451171875	1.0971193844787e-09\\
-33.684517578125	1.13787896142892e-09\\
-33.6645234375	1.10429130209653e-09\\
-33.644529296875	1.09110156408946e-09\\
-33.62453515625	1.11719104519424e-09\\
-33.604541015625	9.83643727962862e-10\\
-33.584546875	1.04179915912629e-09\\
-33.564552734375	1.0515122634121e-09\\
-33.54455859375	1.13042974754211e-09\\
-33.524564453125	1.24525599990282e-09\\
-33.5045703125	1.20891549662815e-09\\
-33.484576171875	1.4170538140511e-09\\
-33.46458203125	1.29609563000194e-09\\
-33.444587890625	1.35791621271542e-09\\
-33.42459375	1.29633072554974e-09\\
-33.404599609375	1.34208640653426e-09\\
-33.38460546875	1.22583648986162e-09\\
-33.364611328125	1.28670284378368e-09\\
-33.3446171875	1.21240293042747e-09\\
-33.324623046875	1.21827747842329e-09\\
-33.30462890625	1.16970186502973e-09\\
-33.284634765625	1.2552608702948e-09\\
-33.264640625	1.19607942271924e-09\\
-33.244646484375	1.26021954700386e-09\\
-33.22465234375	1.19877916897132e-09\\
-33.204658203125	1.13139257048453e-09\\
-33.1846640625	1.1892101952979e-09\\
-33.164669921875	1.18452672878391e-09\\
-33.14467578125	1.15090901038289e-09\\
-33.124681640625	1.16968707540058e-09\\
-33.1046875	1.1754379075756e-09\\
-33.084693359375	1.17389014264565e-09\\
-33.06469921875	1.02319640161582e-09\\
-33.044705078125	1.11877298183936e-09\\
-33.0247109375	9.31208243725754e-10\\
-33.004716796875	8.85188901752744e-10\\
-32.98472265625	8.47928015525068e-10\\
-32.964728515625	7.53121208208723e-10\\
-32.944734375	8.03082350942345e-10\\
-32.924740234375	7.65937223380648e-10\\
-32.90474609375	7.97109891916937e-10\\
-32.884751953125	8.20615875608704e-10\\
-32.8647578125	9.08890712667531e-10\\
-32.844763671875	8.97598667859933e-10\\
-32.82476953125	8.8348510757992e-10\\
-32.804775390625	8.93117339396433e-10\\
-32.78478125	7.61641766285244e-10\\
-32.764787109375	7.51721452254994e-10\\
-32.74479296875	7.12420519826303e-10\\
-32.724798828125	6.18105667418462e-10\\
-32.7048046875	6.35719138864738e-10\\
-32.684810546875	6.43466026563616e-10\\
-32.66481640625	7.21088094980432e-10\\
-32.644822265625	7.79646288767402e-10\\
-32.624828125	8.58157399347707e-10\\
-32.604833984375	8.03751752508245e-10\\
-32.58483984375	9.00101274285527e-10\\
-32.564845703125	8.21090972323519e-10\\
-32.5448515625	8.70223366845735e-10\\
-32.524857421875	8.22077867562575e-10\\
-32.50486328125	8.49248096323022e-10\\
-32.484869140625	7.70815883529817e-10\\
-32.464875	7.83552266025623e-10\\
-32.444880859375	7.8897282782918e-10\\
-32.42488671875	7.18412762405667e-10\\
-32.404892578125	7.43733605291186e-10\\
-32.3848984375	7.0494077022902e-10\\
-32.364904296875	6.81358885600887e-10\\
-32.34491015625	6.94386059563486e-10\\
-32.324916015625	7.45840908533203e-10\\
-32.304921875	7.72908198670756e-10\\
-32.284927734375	8.60595478383634e-10\\
-32.26493359375	8.13870444834446e-10\\
-32.244939453125	8.1349446626648e-10\\
-32.2249453125	8.37282632737346e-10\\
-32.204951171875	7.41361821505677e-10\\
-32.18495703125	7.62137660454012e-10\\
-32.164962890625	6.30170092037385e-10\\
-32.14496875	5.99630623624606e-10\\
-32.124974609375	5.11591867076227e-10\\
-32.10498046875	4.17747911222808e-10\\
-32.084986328125	4.26930509278182e-10\\
-32.0649921875	2.935786806625e-10\\
-32.044998046875	3.42042612432723e-10\\
-32.02500390625	2.29516365471776e-10\\
-32.005009765625	7.98021518891244e-11\\
-31.985015625	1.21968588323325e-11\\
-31.965021484375	-7.40710644294213e-11\\
-31.94502734375	-1.36098419120869e-10\\
-31.925033203125	-1.38104513990578e-10\\
-31.9050390625	-1.82569218690943e-10\\
-31.885044921875	-1.33445544510016e-10\\
-31.86505078125	-3.0661006291902e-10\\
-31.845056640625	-1.46771885016297e-10\\
-31.8250625	-2.4771798541782e-10\\
-31.805068359375	-2.36378217306388e-10\\
-31.78507421875	-4.06999547469168e-10\\
-31.765080078125	-3.00408853455352e-10\\
-31.7450859375	-4.37745075324565e-10\\
-31.725091796875	-3.59529101770558e-10\\
-31.70509765625	-3.38128683634116e-10\\
-31.685103515625	-2.1497607810208e-10\\
-31.665109375	-2.48321311221288e-10\\
-31.645115234375	-1.80870979047837e-10\\
-31.62512109375	-5.8450859928208e-11\\
-31.605126953125	-3.92552083786562e-11\\
-31.5851328125	-5.00224164614575e-11\\
-31.565138671875	-4.93365219561847e-11\\
-31.54514453125	-1.46133395267574e-11\\
-31.525150390625	2.29079758146365e-11\\
-31.50515625	2.08377495441623e-11\\
-31.485162109375	2.24501271673569e-11\\
-31.46516796875	3.07419696690002e-11\\
-31.445173828125	2.97720489132681e-11\\
-31.4251796875	1.18494178769486e-10\\
-31.405185546875	1.25811136437289e-10\\
-31.38519140625	1.6459215894239e-10\\
-31.365197265625	1.96087466622547e-10\\
-31.345203125	1.67180530985987e-10\\
-31.325208984375	1.17053974678486e-10\\
-31.30521484375	1.19965136309802e-10\\
-31.285220703125	-8.74302640085624e-12\\
-31.2652265625	9.56932756206682e-11\\
-31.245232421875	-1.29976806167644e-12\\
-31.22523828125	-1.12358978212291e-10\\
-31.205244140625	-4.29240027488772e-11\\
-31.18525	-1.51235582211152e-10\\
-31.165255859375	-1.9293049846883e-10\\
-31.14526171875	-2.27509010726189e-10\\
-31.125267578125	-2.25395568602621e-10\\
-31.1052734375	-1.9223764536951e-10\\
-31.085279296875	-2.53909418320593e-10\\
-31.06528515625	-1.89815557221123e-10\\
-31.045291015625	-2.06153549805397e-10\\
-31.025296875	-3.22955342324114e-10\\
-31.005302734375	-2.90104053727068e-10\\
-30.98530859375	-3.98524239597648e-10\\
-30.965314453125	-4.48403485546921e-10\\
-30.9453203125	-5.52441254877496e-10\\
-30.925326171875	-4.74221767311241e-10\\
-30.90533203125	-4.87855631147499e-10\\
-30.885337890625	-4.19175433000088e-10\\
-30.86534375	-4.25093537276212e-10\\
-30.845349609375	-3.42160707149124e-10\\
-30.82535546875	-3.10582952453457e-10\\
-30.805361328125	-2.79502187571012e-10\\
-30.7853671875	-3.04784999923151e-10\\
-30.765373046875	-2.3133188222433e-10\\
-30.74537890625	-3.46942769031531e-10\\
-30.725384765625	-2.53995236952152e-10\\
-30.705390625	-3.11912160772023e-10\\
-30.685396484375	-2.53524767713174e-10\\
-30.66540234375	-1.83328186104355e-10\\
-30.645408203125	-1.37808675173704e-10\\
-30.6254140625	-1.61085001335592e-10\\
-30.605419921875	-1.51952285436096e-10\\
-30.58542578125	-1.79196299176027e-10\\
-30.565431640625	-1.29417981164011e-10\\
-30.5454375	-2.50170982577508e-10\\
-30.525443359375	-1.027747320028e-10\\
-30.50544921875	-1.43832106608575e-10\\
-30.485455078125	1.3729450994281e-12\\
-30.4654609375	1.2468331338343e-10\\
-30.445466796875	1.19572868346877e-10\\
-30.42547265625	1.67395838938434e-10\\
-30.405478515625	1.32261031768263e-10\\
-30.385484375	1.81637732380521e-10\\
-30.365490234375	1.38751087565644e-10\\
-30.34549609375	1.61335816743787e-10\\
-30.325501953125	1.08280090142531e-10\\
-30.3055078125	7.31319155017134e-11\\
-30.285513671875	1.5658470148513e-10\\
-30.26551953125	1.19994571193711e-10\\
-30.245525390625	1.90709392238417e-10\\
-30.22553125	2.21562266886704e-10\\
-30.205537109375	1.71881471568341e-10\\
-30.18554296875	2.27665575326279e-10\\
-30.165548828125	1.76565315056941e-10\\
-30.1455546875	2.09445994142043e-10\\
-30.125560546875	2.09984230718637e-10\\
-30.10556640625	1.78141816841015e-10\\
-30.085572265625	1.15727151498714e-10\\
-30.065578125	1.60992098745126e-10\\
-30.045583984375	9.26746327147188e-11\\
-30.02558984375	1.28911663435938e-10\\
-30.005595703125	5.40869242955399e-11\\
-29.9856015625	6.69115966791108e-11\\
-29.965607421875	-9.64061492386887e-12\\
-29.94561328125	1.27406326712033e-10\\
-29.925619140625	9.43975514523342e-11\\
-29.905625	9.69220061441549e-11\\
-29.885630859375	1.17349401847917e-10\\
-29.86563671875	1.6608340535419e-10\\
-29.845642578125	1.56197406951191e-10\\
-29.8256484375	1.67346141776787e-10\\
-29.805654296875	8.46857360185627e-11\\
-29.78566015625	6.36993296034095e-11\\
-29.765666015625	7.41901347356865e-11\\
-29.745671875	1.06554207796522e-10\\
-29.725677734375	1.08707262921851e-10\\
-29.70568359375	-1.04908327564295e-11\\
-29.685689453125	6.31145109936836e-11\\
-29.6656953125	4.93625746592072e-11\\
-29.645701171875	1.38354447447961e-10\\
-29.62570703125	1.8206102096181e-10\\
-29.605712890625	1.28091836655507e-10\\
-29.58571875	1.11657327678603e-10\\
-29.565724609375	1.18764509739492e-10\\
-29.54573046875	1.02075556497631e-10\\
-29.525736328125	1.38742730222867e-10\\
-29.5057421875	2.26857945315048e-10\\
-29.485748046875	1.85287144866315e-10\\
-29.46575390625	3.01918297928656e-10\\
-29.445759765625	3.33987603452128e-10\\
-29.425765625	3.56511141057519e-10\\
-29.405771484375	4.0725121168467e-10\\
-29.38577734375	3.58102141026897e-10\\
-29.365783203125	4.8144289436914e-10\\
-29.3457890625	3.54019480834865e-10\\
-29.325794921875	4.92234333478775e-10\\
-29.30580078125	4.52032909011299e-10\\
-29.285806640625	5.05387258290185e-10\\
-29.2658125	4.02658712527421e-10\\
-29.245818359375	5.18971775195941e-10\\
-29.22582421875	4.66560983664766e-10\\
-29.205830078125	4.84279402041919e-10\\
-29.1858359375	3.98067646878872e-10\\
-29.165841796875	3.77833648393142e-10\\
-29.14584765625	2.71439236289425e-10\\
-29.125853515625	2.43004783493735e-10\\
-29.105859375	2.32371720208161e-10\\
-29.085865234375	7.81831739799259e-11\\
-29.06587109375	7.35378038319564e-11\\
-29.045876953125	4.9981905465622e-12\\
-29.0258828125	-3.14488677627953e-11\\
-29.005888671875	-5.10069238475418e-11\\
-28.98589453125	-8.58856841529688e-11\\
-28.965900390625	-7.41731329143496e-11\\
-28.94590625	-1.8883900759116e-10\\
-28.925912109375	-2.43712944637695e-10\\
-28.90591796875	-2.00003587385647e-10\\
-28.885923828125	-3.04800210067253e-10\\
-28.8659296875	-4.0758017079981e-10\\
-28.845935546875	-3.55289383207815e-10\\
-28.82594140625	-4.21193886377486e-10\\
-28.805947265625	-4.98499802629319e-10\\
-28.785953125	-3.97484028048115e-10\\
-28.765958984375	-5.09201157999692e-10\\
-28.74596484375	-4.95497263098307e-10\\
-28.725970703125	-4.97378060711345e-10\\
-28.7059765625	-4.67639056339027e-10\\
-28.685982421875	-3.86629614737548e-10\\
-28.66598828125	-3.30128169336888e-10\\
-28.645994140625	-4.31094222415043e-10\\
-28.626	-3.35077916283737e-10\\
-28.606005859375	-3.60268564181836e-10\\
-28.58601171875	-3.25117766672526e-10\\
-28.566017578125	-4.10278395787661e-10\\
-28.5460234375	-3.18634155673816e-10\\
-28.526029296875	-3.27586634989775e-10\\
-28.50603515625	-3.84526278253743e-10\\
-28.486041015625	-3.0657662291806e-10\\
-28.466046875	-3.65415657522421e-10\\
-28.446052734375	-2.32167857515672e-10\\
-28.42605859375	-2.72052001308946e-10\\
-28.406064453125	-2.15451349796327e-10\\
-28.3860703125	-2.37245919818081e-10\\
-28.366076171875	-1.55305169157071e-10\\
-28.34608203125	-1.89019531222627e-10\\
-28.326087890625	-2.48965740512932e-10\\
-28.30609375	-1.58785556490839e-10\\
-28.286099609375	-1.90343683086112e-10\\
-28.26610546875	-1.9426893054753e-10\\
-28.246111328125	-2.61146008462869e-10\\
-28.2261171875	-3.88566654892468e-10\\
-28.206123046875	-4.11847242636812e-10\\
-28.18612890625	-4.90410960570357e-10\\
-28.166134765625	-5.32349087629261e-10\\
-28.146140625	-3.86891801714742e-10\\
-28.126146484375	-3.91757957528933e-10\\
-28.10615234375	-5.26346596082102e-10\\
-28.086158203125	-4.22315400546392e-10\\
-28.0661640625	-4.14549508928473e-10\\
-28.046169921875	-5.09525783738643e-10\\
-28.02617578125	-5.63900763456738e-10\\
-28.006181640625	-5.89644739274269e-10\\
-27.9861875	-7.3780806630875e-10\\
-27.966193359375	-7.18061437528153e-10\\
-27.94619921875	-7.81358903084567e-10\\
-27.926205078125	-7.43412443690272e-10\\
-27.9062109375	-8.39788028093868e-10\\
-27.886216796875	-8.95926697461706e-10\\
-27.86622265625	-7.98029302983188e-10\\
-27.846228515625	-8.3184354396612e-10\\
-27.826234375	-8.39378567644044e-10\\
-27.806240234375	-8.99845632257358e-10\\
-27.78624609375	-9.13464517346092e-10\\
-27.766251953125	-9.17773218719524e-10\\
-27.7462578125	-9.53663211873259e-10\\
-27.726263671875	-9.183735820439e-10\\
-27.70626953125	-9.14976961374895e-10\\
-27.686275390625	-9.01130024334176e-10\\
-27.66628125	-8.65320816257362e-10\\
-27.646287109375	-8.60099776217985e-10\\
-27.62629296875	-7.819330392814e-10\\
-27.606298828125	-7.73120536260312e-10\\
-27.5863046875	-7.30602326508031e-10\\
-27.566310546875	-8.38177451625558e-10\\
-27.54631640625	-9.28727091458676e-10\\
-27.526322265625	-1.01808549675775e-09\\
-27.506328125	-1.04542917385156e-09\\
-27.486333984375	-1.00586385533737e-09\\
-27.46633984375	-1.06377002865417e-09\\
-27.446345703125	-1.00911876491814e-09\\
-27.4263515625	-9.23356163497834e-10\\
-27.406357421875	-9.72414750934701e-10\\
-27.38636328125	-9.66228485025489e-10\\
-27.366369140625	-8.90522641768178e-10\\
-27.346375	-9.45703200181337e-10\\
-27.326380859375	-1.01200841512774e-09\\
-27.30638671875	-9.98298954180648e-10\\
-27.286392578125	-1.06726389940821e-09\\
-27.2663984375	-1.06671199651719e-09\\
-27.246404296875	-1.17240989919249e-09\\
-27.22641015625	-1.08490158225317e-09\\
-27.206416015625	-1.13009855035612e-09\\
-27.186421875	-1.20116166363115e-09\\
-27.166427734375	-1.2312172021874e-09\\
-27.14643359375	-1.27307140509938e-09\\
-27.126439453125	-1.28817116442968e-09\\
-27.1064453125	-1.36550153385371e-09\\
-27.086451171875	-1.33339955104659e-09\\
-27.06645703125	-1.31620059176322e-09\\
-27.046462890625	-1.4146841569565e-09\\
-27.02646875	-1.33012573786123e-09\\
-27.006474609375	-1.44422229176449e-09\\
-26.98648046875	-1.51442756745819e-09\\
-26.966486328125	-1.66723014778546e-09\\
-26.9464921875	-1.67182169845041e-09\\
-26.926498046875	-1.80240384692109e-09\\
-26.90650390625	-1.8410496203435e-09\\
-26.886509765625	-1.83604044881349e-09\\
-26.866515625	-1.8220205327206e-09\\
-26.846521484375	-1.78959674855654e-09\\
-26.82652734375	-1.90041625996539e-09\\
-26.806533203125	-1.85050160978563e-09\\
-26.7865390625	-1.89464380102877e-09\\
-26.766544921875	-1.85151824434683e-09\\
-26.74655078125	-1.90300711885074e-09\\
-26.726556640625	-1.80453684873924e-09\\
-26.7065625	-1.82914491318695e-09\\
-26.686568359375	-1.75057907753913e-09\\
-26.66657421875	-1.71673597026607e-09\\
-26.646580078125	-1.69235853447465e-09\\
-26.6265859375	-1.72904146259416e-09\\
-26.606591796875	-1.67015045768621e-09\\
-26.58659765625	-1.53785848454453e-09\\
-26.566603515625	-1.57428636279066e-09\\
-26.546609375	-1.57447601258113e-09\\
-26.526615234375	-1.4913307803037e-09\\
-26.50662109375	-1.47286995891707e-09\\
-26.486626953125	-1.56419070512536e-09\\
-26.4666328125	-1.4587555833727e-09\\
-26.446638671875	-1.43338621901002e-09\\
-26.42664453125	-1.52498651453971e-09\\
-26.406650390625	-1.4091375961839e-09\\
-26.38665625	-1.35355785359008e-09\\
-26.366662109375	-1.30651509429087e-09\\
-26.34666796875	-1.22905244585285e-09\\
-26.326673828125	-1.24240433945182e-09\\
-26.3066796875	-1.16721266947404e-09\\
-26.286685546875	-1.22117697475612e-09\\
-26.26669140625	-1.27664592002642e-09\\
-26.246697265625	-1.24834175274934e-09\\
-26.226703125	-1.29346706952852e-09\\
-26.206708984375	-1.36526590337921e-09\\
-26.18671484375	-1.34816798462085e-09\\
-26.166720703125	-1.42195249576921e-09\\
-26.1467265625	-1.37964104306054e-09\\
-26.126732421875	-1.46813924189022e-09\\
-26.10673828125	-1.52175148419276e-09\\
-26.086744140625	-1.44175422047441e-09\\
-26.06675	-1.40697868472708e-09\\
-26.046755859375	-1.38245802922152e-09\\
-26.02676171875	-1.31623031292422e-09\\
-26.006767578125	-1.22080575617605e-09\\
-25.9867734375	-1.32503503415385e-09\\
-25.966779296875	-1.24741251842106e-09\\
-25.94678515625	-1.33376967178627e-09\\
-25.926791015625	-1.24031232143463e-09\\
-25.906796875	-1.32517412037101e-09\\
-25.886802734375	-1.24311230216897e-09\\
-25.86680859375	-1.24626402892301e-09\\
-25.846814453125	-1.144160367668e-09\\
-25.8268203125	-1.06897618576516e-09\\
-25.806826171875	-1.06075313141266e-09\\
-25.78683203125	-9.71843078229393e-10\\
-25.766837890625	-1.05403730084735e-09\\
-25.74684375	-1.05460032471326e-09\\
-25.726849609375	-1.01545085733452e-09\\
-25.70685546875	-1.04567998681539e-09\\
-25.686861328125	-9.52424050811929e-10\\
-25.6668671875	-9.07135975067843e-10\\
-25.646873046875	-8.2520559009081e-10\\
-25.62687890625	-5.27936909030867e-10\\
-25.606884765625	-3.26363685178652e-10\\
-25.586890625	-3.72953935341414e-10\\
-25.566896484375	-2.1534153475318e-10\\
-25.54690234375	-2.32825997493264e-10\\
-25.526908203125	-2.21071469041236e-10\\
-25.5069140625	-1.70309137400427e-10\\
-25.486919921875	-1.19667153396572e-10\\
-25.46692578125	-2.14232319561384e-11\\
-25.446931640625	1.05613118153312e-10\\
-25.4269375	2.74628717593594e-10\\
-25.406943359375	3.7936660379236e-10\\
-25.38694921875	5.19895599936688e-10\\
-25.366955078125	5.72459138151403e-10\\
-25.3469609375	6.16847388385539e-10\\
-25.326966796875	5.6279111799668e-10\\
-25.30697265625	6.18969874443865e-10\\
-25.286978515625	5.36820303777757e-10\\
-25.266984375	5.40943979780211e-10\\
-25.246990234375	5.52685547162075e-10\\
-25.22699609375	4.68248280143961e-10\\
-25.207001953125	5.99989010148338e-10\\
-25.1870078125	6.06523221394137e-10\\
-25.167013671875	6.09097402853687e-10\\
-25.14701953125	7.39058278591184e-10\\
-25.127025390625	6.24318078877684e-10\\
-25.10703125	5.80716903465069e-10\\
-25.087037109375	6.39363397953698e-10\\
-25.06704296875	7.01473937160628e-10\\
-25.047048828125	7.76918159888615e-10\\
-25.0270546875	7.2391543125017e-10\\
-25.007060546875	8.39631536574247e-10\\
-24.98706640625	8.55438207635576e-10\\
-24.967072265625	1.00655366265253e-09\\
-24.947078125	9.51429824415873e-10\\
-24.927083984375	9.81325193690232e-10\\
-24.90708984375	9.37636736403679e-10\\
-24.887095703125	9.33055386132151e-10\\
-24.8671015625	7.97587848006513e-10\\
-24.847107421875	8.74084978700524e-10\\
-24.82711328125	8.74750052285004e-10\\
-24.807119140625	9.30095663664793e-10\\
-24.787125	1.06796304707965e-09\\
-24.767130859375	1.08147504381182e-09\\
-24.74713671875	1.1190546625238e-09\\
-24.727142578125	1.11738957872577e-09\\
-24.7071484375	1.18408694338405e-09\\
-24.687154296875	1.14970842101209e-09\\
-24.66716015625	1.16150538174096e-09\\
-24.647166015625	1.12247291363372e-09\\
-24.627171875	1.24128716049672e-09\\
-24.607177734375	1.25883665632957e-09\\
-24.58718359375	1.24915869817479e-09\\
-24.567189453125	1.29535875546352e-09\\
-24.5471953125	1.28213138003375e-09\\
-24.527201171875	1.33224827138524e-09\\
-24.50720703125	1.26909087051464e-09\\
-24.487212890625	1.32938585826039e-09\\
-24.46721875	1.32236542671673e-09\\
-24.447224609375	1.32569773437603e-09\\
-24.42723046875	1.47107970871504e-09\\
-24.407236328125	1.35437643308068e-09\\
-24.3872421875	1.60466995089428e-09\\
-24.367248046875	1.56636559856566e-09\\
-24.34725390625	1.58775392742963e-09\\
-24.327259765625	1.57113134826446e-09\\
-24.307265625	1.50426551494047e-09\\
-24.287271484375	1.56709469949167e-09\\
-24.26727734375	1.49246247710104e-09\\
-24.247283203125	1.42807096388072e-09\\
-24.2272890625	1.28862643334915e-09\\
-24.207294921875	1.37372710602665e-09\\
-24.18730078125	1.16124815922166e-09\\
-24.167306640625	1.19219606707195e-09\\
-24.1473125	1.27103272957274e-09\\
-24.127318359375	1.28285244575792e-09\\
-24.10732421875	1.17905853265614e-09\\
-24.087330078125	1.1151261463228e-09\\
-24.0673359375	1.08356593847266e-09\\
-24.047341796875	1.01174823497156e-09\\
-24.02734765625	9.39712707019912e-10\\
-24.007353515625	9.82922336017776e-10\\
-23.987359375	9.90924204339681e-10\\
-23.967365234375	8.92348980827855e-10\\
-23.94737109375	8.62716269225874e-10\\
-23.927376953125	8.33609930516313e-10\\
-23.9073828125	7.92737953975226e-10\\
-23.887388671875	7.39543293166196e-10\\
-23.86739453125	6.94349713244603e-10\\
-23.847400390625	7.00182848614142e-10\\
-23.82740625	6.06508099439782e-10\\
-23.807412109375	4.75146560195677e-10\\
-23.78741796875	4.58475183187204e-10\\
-23.767423828125	4.99616806500103e-10\\
-23.7474296875	3.70582016076038e-10\\
-23.727435546875	4.91813790393692e-10\\
-23.70744140625	5.55046010239738e-10\\
-23.687447265625	5.25633578423703e-10\\
-23.667453125	6.04503709896669e-10\\
-23.647458984375	7.42006461088898e-10\\
-23.62746484375	7.84208682069523e-10\\
-23.607470703125	8.61559242328815e-10\\
-23.5874765625	8.45371673728079e-10\\
-23.567482421875	6.98816915688287e-10\\
-23.54748828125	7.57161913730376e-10\\
-23.527494140625	5.75275740634772e-10\\
-23.5075	5.18004285337518e-10\\
-23.487505859375	5.3070399304208e-10\\
-23.46751171875	3.9654609730833e-10\\
-23.447517578125	5.26030191074934e-10\\
-23.4275234375	4.70638459156439e-10\\
-23.407529296875	6.14234703361086e-10\\
-23.38753515625	5.44180247099742e-10\\
-23.367541015625	5.20622644340699e-10\\
-23.347546875	4.33037242633333e-10\\
-23.327552734375	3.30810911192546e-10\\
-23.30755859375	3.24013006212012e-10\\
-23.287564453125	1.36064122117556e-10\\
-23.2675703125	8.68698733636112e-11\\
-23.247576171875	6.81785369095397e-11\\
-23.22758203125	-2.39218108567829e-11\\
-23.207587890625	6.50600435086587e-12\\
-23.18759375	-3.25735240614424e-11\\
-23.167599609375	-7.67105266898549e-11\\
-23.14760546875	-2.54927047816392e-10\\
-23.127611328125	-3.99896260720256e-10\\
-23.1076171875	-4.11011954655687e-10\\
-23.087623046875	-5.03232983788245e-10\\
-23.06762890625	-7.74816157574177e-10\\
-23.047634765625	-8.59145482064458e-10\\
-23.027640625	-9.10149539480305e-10\\
-23.007646484375	-9.65727542108078e-10\\
-22.98765234375	-1.03535099711557e-09\\
-22.967658203125	-1.05660329996714e-09\\
-22.9476640625	-1.0555560137177e-09\\
-22.927669921875	-1.17152564769918e-09\\
-22.90767578125	-1.31195377361466e-09\\
-22.887681640625	-1.30632736906425e-09\\
-22.8676875	-1.60026308390132e-09\\
-22.847693359375	-1.56064341616405e-09\\
-22.82769921875	-1.64950762197737e-09\\
-22.807705078125	-1.67832485411641e-09\\
-22.7877109375	-1.69260870530205e-09\\
-22.767716796875	-1.75255353724131e-09\\
-22.74772265625	-1.71457640763254e-09\\
-22.727728515625	-1.6629647502472e-09\\
-22.707734375	-1.71097721494442e-09\\
-22.687740234375	-1.66506836239431e-09\\
-22.66774609375	-1.67719082433282e-09\\
-22.647751953125	-1.67635274187394e-09\\
-22.6277578125	-1.63389566517355e-09\\
-22.607763671875	-1.59918030919823e-09\\
-22.58776953125	-1.63225432760813e-09\\
-22.567775390625	-1.64696686216297e-09\\
-22.54778125	-1.67945015482734e-09\\
-22.527787109375	-1.69716550856149e-09\\
-22.50779296875	-1.7595738719762e-09\\
-22.487798828125	-1.76319960769779e-09\\
-22.4678046875	-1.93910947417648e-09\\
-22.447810546875	-1.9671255453211e-09\\
-22.42781640625	-2.12074092278846e-09\\
-22.407822265625	-2.18228013528015e-09\\
-22.387828125	-2.13819751548037e-09\\
-22.367833984375	-2.04257181790408e-09\\
-22.34783984375	-2.08471932519424e-09\\
-22.327845703125	-1.99625732586351e-09\\
-22.3078515625	-2.06957566431766e-09\\
-22.287857421875	-2.10277447676409e-09\\
-22.26786328125	-2.07081950372856e-09\\
-22.247869140625	-2.1035358813679e-09\\
-22.227875	-2.16965266411459e-09\\
-22.207880859375	-2.26110614054965e-09\\
-22.18788671875	-2.20277633323222e-09\\
-22.167892578125	-2.28589120733368e-09\\
-22.1478984375	-2.20449736204042e-09\\
-22.127904296875	-2.25068336027892e-09\\
-22.10791015625	-2.28655659935448e-09\\
-22.087916015625	-2.45556832153143e-09\\
-22.067921875	-2.54115334858359e-09\\
-22.047927734375	-2.48500649028331e-09\\
-22.02793359375	-2.5020008907563e-09\\
-22.007939453125	-2.58640194971603e-09\\
-21.9879453125	-2.4405063205647e-09\\
-21.967951171875	-2.47608168695271e-09\\
-21.94795703125	-2.48127525512887e-09\\
-21.927962890625	-2.31817325458887e-09\\
-21.90796875	-2.22887014530935e-09\\
-21.887974609375	-2.43301490680579e-09\\
-21.86798046875	-2.49783665384868e-09\\
-21.847986328125	-2.43343639627724e-09\\
-21.8279921875	-2.46850758404835e-09\\
-21.807998046875	-2.46869962797765e-09\\
-21.78800390625	-2.47474927886266e-09\\
-21.768009765625	-2.48851238020723e-09\\
-21.748015625	-2.38138031103013e-09\\
-21.728021484375	-2.24697124305633e-09\\
-21.70802734375	-2.23947940095259e-09\\
-21.688033203125	-2.05858144088558e-09\\
-21.6680390625	-2.05812343419978e-09\\
-21.648044921875	-2.12267586135556e-09\\
-21.62805078125	-2.13790229510301e-09\\
-21.608056640625	-2.08551731076989e-09\\
-21.5880625	-2.13777790868887e-09\\
-21.568068359375	-1.98181632176789e-09\\
-21.54807421875	-2.04792174468987e-09\\
-21.528080078125	-1.96675863614308e-09\\
-21.5080859375	-1.9420000908632e-09\\
-21.488091796875	-1.90627008840413e-09\\
-21.46809765625	-1.90718311289926e-09\\
-21.448103515625	-1.74888584493519e-09\\
-21.428109375	-1.86062199897838e-09\\
-21.408115234375	-1.86442850696283e-09\\
-21.38812109375	-1.8009287531006e-09\\
-21.368126953125	-1.76140644735215e-09\\
-21.3481328125	-1.70508930874351e-09\\
-21.328138671875	-1.67239942199127e-09\\
-21.30814453125	-1.5193158035741e-09\\
-21.288150390625	-1.52900790531506e-09\\
-21.26815625	-1.40062424384638e-09\\
-21.248162109375	-1.38275410116213e-09\\
-21.22816796875	-1.41521768997573e-09\\
-21.208173828125	-1.45656760102205e-09\\
-21.1881796875	-1.44998655065178e-09\\
-21.168185546875	-1.45731326090233e-09\\
-21.14819140625	-1.59808214131953e-09\\
-21.128197265625	-1.57568948947253e-09\\
-21.108203125	-1.42381290080382e-09\\
-21.088208984375	-1.3757363651157e-09\\
-21.06821484375	-1.08349486113756e-09\\
-21.048220703125	-9.63744328876422e-10\\
-21.0282265625	-8.61672966726009e-10\\
-21.008232421875	-6.96454585814736e-10\\
-20.98823828125	-7.89534621308956e-10\\
-20.968244140625	-6.25683446114634e-10\\
-20.94825	-8.34288502845767e-10\\
-20.928255859375	-8.10165385853645e-10\\
-20.90826171875	-7.51248024995403e-10\\
-20.888267578125	-6.66795862374888e-10\\
-20.8682734375	-6.65293820722337e-10\\
-20.848279296875	-3.63343484946403e-10\\
-20.82828515625	-3.65216090565399e-10\\
-20.808291015625	-1.17898821838639e-10\\
-20.788296875	4.24345588018592e-11\\
-20.768302734375	1.47879131766649e-10\\
-20.74830859375	1.58959962613329e-10\\
-20.728314453125	1.10780683370334e-10\\
-20.7083203125	9.72277569531434e-11\\
-20.688326171875	1.21048619782296e-10\\
-20.66833203125	2.37448167319704e-10\\
-20.648337890625	3.63931946236405e-10\\
-20.62834375	4.97413488443996e-10\\
-20.608349609375	7.25902261687892e-10\\
-20.58835546875	8.10674216167929e-10\\
-20.568361328125	1.11734788321996e-09\\
-20.5483671875	1.20770870233496e-09\\
-20.528373046875	1.30665875118757e-09\\
-20.50837890625	1.33506074974061e-09\\
-20.488384765625	1.4833120412871e-09\\
-20.468390625	1.54954434238471e-09\\
-20.448396484375	1.51114633298191e-09\\
-20.42840234375	1.79174877490567e-09\\
-20.408408203125	1.96757712125931e-09\\
-20.3884140625	2.05416245857176e-09\\
-20.368419921875	2.37047667202952e-09\\
-20.34842578125	2.29404220121742e-09\\
-20.328431640625	2.43347930330988e-09\\
-20.3084375	2.40968629407773e-09\\
-20.288443359375	2.412358835281e-09\\
-20.26844921875	2.5190106121062e-09\\
-20.248455078125	2.37876093397431e-09\\
-20.2284609375	2.3773126391429e-09\\
-20.208466796875	2.48256714485596e-09\\
-20.18847265625	2.51759919896621e-09\\
-20.168478515625	2.59517502249014e-09\\
-20.148484375	2.67653277900999e-09\\
-20.128490234375	2.6456752371618e-09\\
-20.10849609375	2.66233396724105e-09\\
-20.088501953125	2.62092733892006e-09\\
-20.0685078125	2.664955572789e-09\\
-20.048513671875	2.71419485898188e-09\\
-20.02851953125	2.75708852104296e-09\\
-20.008525390625	2.81225457496767e-09\\
-19.98853125	2.74593334910025e-09\\
-19.968537109375	3.02924253282569e-09\\
-19.94854296875	3.02580174055755e-09\\
-19.928548828125	3.1851927471991e-09\\
-19.9085546875	3.24918277618442e-09\\
-19.888560546875	3.24522094090591e-09\\
-19.86856640625	3.28731570255094e-09\\
-19.848572265625	3.29053708891414e-09\\
-19.828578125	3.41313167092666e-09\\
-19.808583984375	3.41688740096796e-09\\
-19.78858984375	3.40483782741716e-09\\
-19.768595703125	3.58143233049011e-09\\
-19.7486015625	3.54104238616916e-09\\
-19.728607421875	3.62276660139628e-09\\
-19.70861328125	3.70178174241429e-09\\
-19.688619140625	3.56324391036146e-09\\
-19.668625	3.60716871784238e-09\\
-19.648630859375	3.44340540366013e-09\\
-19.62863671875	3.56920034959173e-09\\
-19.608642578125	3.58164232136195e-09\\
-19.5886484375	3.63617610327014e-09\\
-19.568654296875	3.66438450593532e-09\\
-19.54866015625	3.85934564211149e-09\\
-19.528666015625	3.94043284395349e-09\\
-19.508671875	4.0386394516595e-09\\
-19.488677734375	4.03245503103454e-09\\
-19.46868359375	4.00601040956566e-09\\
-19.448689453125	3.8613828726607e-09\\
-19.4286953125	3.91171218109699e-09\\
-19.408701171875	3.85072191350645e-09\\
-19.38870703125	3.83855610147892e-09\\
-19.368712890625	3.79989016938687e-09\\
-19.34871875	3.78025100415877e-09\\
-19.328724609375	3.96170583519443e-09\\
-19.30873046875	3.94556562260368e-09\\
-19.288736328125	3.92699277991324e-09\\
-19.2687421875	4.10437071313978e-09\\
-19.248748046875	3.97422813937297e-09\\
-19.22875390625	3.92347570229744e-09\\
-19.208759765625	3.9678820748993e-09\\
-19.188765625	3.69596197687525e-09\\
-19.168771484375	3.63976305594401e-09\\
-19.14877734375	3.61756652565804e-09\\
-19.128783203125	3.46754323819987e-09\\
-19.1087890625	3.55453842974756e-09\\
-19.088794921875	3.52602089912802e-09\\
-19.06880078125	3.35588963803254e-09\\
-19.048806640625	3.52991210103881e-09\\
-19.0288125	3.39041784279623e-09\\
-19.008818359375	3.51831719603013e-09\\
-18.98882421875	3.42280356396993e-09\\
-18.968830078125	3.38014402024511e-09\\
-18.9488359375	3.28423687520934e-09\\
-18.928841796875	3.27864809340126e-09\\
-18.90884765625	3.27487811748128e-09\\
-18.888853515625	3.27690427542031e-09\\
-18.868859375	3.29969877150787e-09\\
-18.848865234375	3.3620105869131e-09\\
-18.82887109375	3.25633315186531e-09\\
-18.808876953125	3.2326111028278e-09\\
-18.7888828125	3.31502948281929e-09\\
-18.768888671875	3.24391322614441e-09\\
-18.74889453125	3.3177499793828e-09\\
-18.728900390625	3.16091418058907e-09\\
-18.70890625	3.15191497776751e-09\\
-18.688912109375	3.25355921871478e-09\\
-18.66891796875	3.15667284634127e-09\\
-18.648923828125	3.266635053976e-09\\
-18.6289296875	3.33647355516109e-09\\
-18.608935546875	3.26324779048175e-09\\
-18.58894140625	3.29051814117197e-09\\
-18.568947265625	3.25125520070428e-09\\
-18.548953125	3.22051268723068e-09\\
-18.528958984375	3.23046128381792e-09\\
-18.50896484375	3.08320122579008e-09\\
-18.488970703125	3.08800000889763e-09\\
-18.4689765625	2.96668097595598e-09\\
-18.448982421875	2.83768996108022e-09\\
-18.42898828125	2.80753615116401e-09\\
-18.408994140625	2.73961175745083e-09\\
-18.389	2.65542289827426e-09\\
-18.369005859375	2.68621285037818e-09\\
-18.34901171875	2.46472688641567e-09\\
-18.329017578125	2.40548487453456e-09\\
-18.3090234375	2.30672714693584e-09\\
-18.289029296875	2.17772774082309e-09\\
-18.26903515625	2.0966059360087e-09\\
-18.249041015625	1.98235889429563e-09\\
-18.229046875	1.84217960568286e-09\\
-18.209052734375	1.76226234720678e-09\\
-18.18905859375	1.67310673971131e-09\\
-18.169064453125	1.51097534622666e-09\\
-18.1490703125	1.47022204806205e-09\\
-18.129076171875	1.37226038025075e-09\\
-18.10908203125	1.1513339690094e-09\\
-18.089087890625	1.22557805729278e-09\\
-18.06909375	9.31998045002514e-10\\
-18.049099609375	9.05544326581114e-10\\
-18.02910546875	8.29922740781285e-10\\
-18.009111328125	8.23677758038201e-10\\
-17.9891171875	8.83479644494127e-10\\
-17.969123046875	9.79489764080432e-10\\
-17.94912890625	9.76833357145317e-10\\
-17.929134765625	7.68288728907874e-10\\
-17.909140625	5.14011494580212e-10\\
-17.889146484375	3.83681315335744e-10\\
-17.86915234375	6.19893558086396e-11\\
-17.849158203125	1.48273996502167e-10\\
-17.8291640625	-6.55780412872273e-11\\
-17.809169921875	-4.30926610617141e-11\\
-17.78917578125	-5.2361351863979e-11\\
-17.769181640625	-3.60976603607936e-11\\
-17.7491875	1.41442307818771e-10\\
-17.729193359375	1.56998968775663e-10\\
-17.70919921875	8.97050591504522e-11\\
-17.689205078125	1.3135856852758e-10\\
-17.6692109375	-6.23801210107759e-11\\
-17.649216796875	-1.32131179836399e-10\\
-17.62922265625	-1.92123407212217e-10\\
-17.609228515625	-3.85328779007712e-10\\
-17.589234375	-4.17474400898581e-10\\
-17.569240234375	-5.32376831419557e-10\\
-17.54924609375	-4.54453849945912e-10\\
-17.529251953125	-4.50017314756806e-10\\
-17.5092578125	-6.19201178113022e-10\\
-17.489263671875	-4.48254301530029e-10\\
-17.46926953125	-6.3623445528256e-10\\
-17.449275390625	-6.27565874312954e-10\\
-17.42928125	-6.48978289892506e-10\\
-17.409287109375	-8.45029366928833e-10\\
-17.38929296875	-9.09831943647049e-10\\
-17.369298828125	-1.04491686811075e-09\\
-17.3493046875	-9.63036246308734e-10\\
-17.329310546875	-1.20500561457058e-09\\
-17.30931640625	-1.13644664840511e-09\\
-17.289322265625	-1.4925813067997e-09\\
-17.269328125	-1.55052321983388e-09\\
-17.249333984375	-1.63630221947979e-09\\
-17.22933984375	-1.6902363405336e-09\\
-17.209345703125	-1.79135188806849e-09\\
-17.1893515625	-1.68870424226112e-09\\
-17.169357421875	-1.82685888857241e-09\\
-17.14936328125	-1.76151027041843e-09\\
-17.129369140625	-1.68104566939462e-09\\
-17.109375	-1.7214368351417e-09\\
-17.089380859375	-1.8687600709947e-09\\
-17.06938671875	-1.97626326805701e-09\\
-17.049392578125	-2.17878914202135e-09\\
-17.0293984375	-2.24665969976691e-09\\
-17.009404296875	-2.29995364759513e-09\\
-16.98941015625	-2.48687983402354e-09\\
-16.969416015625	-2.41267337262053e-09\\
-16.949421875	-2.49283504583741e-09\\
-16.929427734375	-2.54271641151325e-09\\
-16.90943359375	-2.55789315157226e-09\\
-16.889439453125	-2.61258071360486e-09\\
-16.8694453125	-2.82165963748671e-09\\
-16.849451171875	-2.9509602520004e-09\\
-16.82945703125	-2.96555809611469e-09\\
-16.809462890625	-3.01305965055236e-09\\
-16.78946875	-2.99501484390461e-09\\
-16.769474609375	-3.01318139043214e-09\\
-16.74948046875	-2.93675767639355e-09\\
-16.729486328125	-2.78740604049624e-09\\
-16.7094921875	-2.78508905869639e-09\\
-16.689498046875	-2.75828499526949e-09\\
-16.66950390625	-2.8245442408607e-09\\
-16.649509765625	-2.93230959184765e-09\\
-16.629515625	-2.86720764041023e-09\\
-16.609521484375	-2.9881685341527e-09\\
-16.58952734375	-2.95393429508409e-09\\
-16.569533203125	-2.76456555416184e-09\\
-16.5495390625	-2.77122886814086e-09\\
-16.529544921875	-2.5545799072642e-09\\
-16.50955078125	-2.61011429367642e-09\\
-16.489556640625	-2.52490594857042e-09\\
-16.4695625	-2.56376151090866e-09\\
-16.449568359375	-2.57419675913504e-09\\
-16.42957421875	-2.60685162845298e-09\\
-16.409580078125	-2.629428742205e-09\\
-16.3895859375	-2.64091865564466e-09\\
-16.369591796875	-2.79462151919162e-09\\
-16.34959765625	-2.76099533300518e-09\\
-16.329603515625	-2.8327388502839e-09\\
-16.309609375	-2.72949377257971e-09\\
-16.289615234375	-2.75691895870997e-09\\
-16.26962109375	-2.78834481441871e-09\\
-16.249626953125	-2.77774091346472e-09\\
-16.2296328125	-2.6957737568611e-09\\
-16.209638671875	-2.88317383567535e-09\\
-16.18964453125	-2.91773444809253e-09\\
-16.169650390625	-2.96040703977022e-09\\
-16.14965625	-3.11340211557313e-09\\
-16.129662109375	-3.09518984141926e-09\\
-16.10966796875	-3.09541844187315e-09\\
-16.089673828125	-3.15319530536207e-09\\
-16.0696796875	-3.20649241253009e-09\\
-16.049685546875	-3.19585027585334e-09\\
-16.02969140625	-3.25238302729426e-09\\
-16.009697265625	-3.10279581536865e-09\\
-15.989703125	-3.10229290818353e-09\\
-15.969708984375	-3.08054283432654e-09\\
-15.94971484375	-2.97065872139744e-09\\
-15.929720703125	-2.8783792459904e-09\\
-15.9097265625	-2.80122799025175e-09\\
-15.889732421875	-2.76218734205923e-09\\
-15.86973828125	-2.77547855106542e-09\\
-15.849744140625	-2.59672281464524e-09\\
-15.82975	-2.64397494039232e-09\\
-15.809755859375	-2.49119222254365e-09\\
-15.78976171875	-2.33131951229153e-09\\
-15.769767578125	-2.16439982520636e-09\\
-15.7497734375	-2.21693692450056e-09\\
-15.729779296875	-1.99037967527665e-09\\
-15.70978515625	-2.04279560838283e-09\\
-15.689791015625	-1.92806827234399e-09\\
-15.669796875	-1.87253064556271e-09\\
-15.649802734375	-1.73784421019239e-09\\
-15.62980859375	-1.68316837964164e-09\\
-15.609814453125	-1.56395033779816e-09\\
-15.5898203125	-1.5143057263873e-09\\
-15.569826171875	-1.35635312731873e-09\\
-15.54983203125	-1.38591653951875e-09\\
-15.529837890625	-1.28438086717522e-09\\
-15.50984375	-1.20860357352229e-09\\
-15.489849609375	-1.23718748797897e-09\\
-15.46985546875	-1.30835108287705e-09\\
-15.449861328125	-1.07343047843081e-09\\
-15.4298671875	-1.17144845237536e-09\\
-15.409873046875	-1.046278048458e-09\\
-15.38987890625	-1.05081124423518e-09\\
-15.369884765625	-8.08630805522706e-10\\
-15.349890625	-8.59053014755483e-10\\
-15.329896484375	-7.96252827288303e-10\\
-15.30990234375	-7.60679999379408e-10\\
-15.289908203125	-7.06386526692073e-10\\
-15.2699140625	-7.69958832258609e-10\\
-15.249919921875	-7.88684339879069e-10\\
-15.22992578125	-7.19260616680976e-10\\
-15.209931640625	-6.92998853871459e-10\\
-15.1899375	-7.36105155284738e-10\\
-15.169943359375	-5.802733224147e-10\\
-15.14994921875	-6.13670637814103e-10\\
-15.129955078125	-6.4666266099011e-10\\
-15.1099609375	-5.07573521472895e-10\\
-15.089966796875	-5.31110077914233e-10\\
-15.06997265625	-5.39758838132082e-10\\
-15.049978515625	-5.00222172692514e-10\\
-15.029984375	-5.68780754998931e-10\\
-15.009990234375	-4.21147423290633e-10\\
-14.98999609375	-4.63321593461232e-10\\
-14.970001953125	-1.54790006342087e-10\\
-14.9500078125	-8.77001344106807e-11\\
-14.930013671875	-3.13139466923289e-11\\
-14.91001953125	2.23660011963727e-10\\
-14.890025390625	2.49374003283653e-10\\
-14.87003125	3.73603072480197e-10\\
-14.850037109375	2.81760917822329e-10\\
-14.83004296875	3.44656460494962e-10\\
-14.810048828125	2.794590148917e-10\\
-14.7900546875	5.08371635232165e-10\\
-14.770060546875	4.05127603106808e-10\\
-14.75006640625	7.70515969677217e-10\\
-14.730072265625	9.34417748373338e-10\\
-14.710078125	1.10492081586418e-09\\
-14.690083984375	1.17527948095143e-09\\
-14.67008984375	1.35804792702684e-09\\
-14.650095703125	1.33901580679515e-09\\
-14.6301015625	1.2086176757111e-09\\
-14.610107421875	1.14974668222384e-09\\
-14.59011328125	1.26919240668354e-09\\
-14.570119140625	1.13415419141139e-09\\
-14.550125	1.21130651744195e-09\\
-14.530130859375	1.3231804079116e-09\\
-14.51013671875	1.4401163512405e-09\\
-14.490142578125	1.62904799031031e-09\\
-14.4701484375	1.58679614128869e-09\\
-14.450154296875	1.82991199156218e-09\\
-14.43016015625	1.79006348792424e-09\\
-14.410166015625	1.84387604596825e-09\\
-14.390171875	1.91194663964994e-09\\
-14.370177734375	1.9453423768631e-09\\
-14.35018359375	1.98473854483374e-09\\
-14.330189453125	2.02251359696552e-09\\
-14.3101953125	2.07934946502586e-09\\
-14.290201171875	2.03629543728397e-09\\
-14.27020703125	2.1275178432154e-09\\
-14.250212890625	2.0389492889197e-09\\
-14.23021875	2.05681776459938e-09\\
-14.210224609375	2.02686668158317e-09\\
-14.19023046875	2.04629731014573e-09\\
-14.170236328125	2.00108724279893e-09\\
-14.1502421875	1.90036550471551e-09\\
-14.130248046875	2.05809373133012e-09\\
-14.11025390625	2.05720862557693e-09\\
-14.090259765625	1.98849907515029e-09\\
-14.070265625	2.05003025133436e-09\\
-14.050271484375	2.13542342496882e-09\\
-14.03027734375	1.98022494868172e-09\\
-14.010283203125	2.1222366678769e-09\\
-13.9902890625	1.88063051361344e-09\\
-13.970294921875	1.9629634111483e-09\\
-13.95030078125	1.84046890728535e-09\\
-13.930306640625	1.87703330902428e-09\\
-13.9103125	1.91611142022429e-09\\
-13.890318359375	1.92803142175268e-09\\
-13.87032421875	1.7963752482818e-09\\
-13.850330078125	1.73332866508099e-09\\
-13.8303359375	1.90218819869672e-09\\
-13.810341796875	1.83714933860912e-09\\
-13.79034765625	1.92676474978284e-09\\
-13.770353515625	1.95628633076841e-09\\
-13.750359375	2.00918589114603e-09\\
-13.730365234375	2.08212749477353e-09\\
-13.71037109375	2.17610215911838e-09\\
-13.690376953125	2.1930287659093e-09\\
-13.6703828125	2.18560583297288e-09\\
-13.650388671875	2.18626580793489e-09\\
-13.63039453125	2.0731711077326e-09\\
-13.610400390625	2.1908922929032e-09\\
-13.59040625	2.1566399055075e-09\\
-13.570412109375	2.31128166684209e-09\\
-13.55041796875	2.31351484188806e-09\\
-13.530423828125	2.35877280555723e-09\\
-13.5104296875	2.36664851723729e-09\\
-13.490435546875	2.41524795356066e-09\\
-13.47044140625	2.44559412691162e-09\\
-13.450447265625	2.33402222007372e-09\\
-13.430453125	2.27639158981931e-09\\
-13.410458984375	2.26961482961606e-09\\
-13.39046484375	2.13810758201458e-09\\
-13.370470703125	2.1006825708691e-09\\
-13.3504765625	2.07530599551916e-09\\
-13.330482421875	2.02437806227161e-09\\
-13.31048828125	1.96181165405695e-09\\
-13.290494140625	1.8829354046623e-09\\
-13.2705	1.68454322645051e-09\\
-13.250505859375	1.63946380600257e-09\\
-13.23051171875	1.44412255649537e-09\\
-13.210517578125	1.45307103685188e-09\\
-13.1905234375	1.41374620992929e-09\\
-13.170529296875	1.38631722040165e-09\\
-13.15053515625	1.35584659160598e-09\\
-13.130541015625	1.33410186788806e-09\\
-13.110546875	1.28517794805605e-09\\
-13.090552734375	1.30314658250945e-09\\
-13.07055859375	1.16041289282456e-09\\
-13.050564453125	1.13556930087085e-09\\
-13.0305703125	8.4961512899674e-10\\
-13.010576171875	8.97733241222425e-10\\
-12.99058203125	9.80397553621197e-10\\
-12.970587890625	8.77113960837589e-10\\
-12.95059375	8.34170335501216e-10\\
-12.930599609375	9.39735871910536e-10\\
-12.91060546875	8.47654997058793e-10\\
-12.890611328125	8.4151063346522e-10\\
-12.8706171875	7.43926927743731e-10\\
-12.850623046875	8.71195918430746e-10\\
-12.83062890625	5.89539463347713e-10\\
-12.810634765625	7.11775652458989e-10\\
-12.790640625	6.03055430842106e-10\\
-12.770646484375	7.03077199950726e-10\\
-12.75065234375	5.30917170869934e-10\\
-12.730658203125	6.54498441515539e-10\\
-12.7106640625	6.04508799478252e-10\\
-12.690669921875	5.28471109926207e-10\\
-12.67067578125	5.69355314207509e-10\\
-12.650681640625	5.7365774962195e-10\\
-12.6306875	4.55538340806426e-10\\
-12.610693359375	4.97604167141277e-10\\
-12.59069921875	4.00164251064382e-10\\
-12.570705078125	3.91009227625486e-10\\
-12.5507109375	4.48706555806057e-10\\
-12.530716796875	4.03244757398756e-10\\
-12.51072265625	4.22837974938764e-10\\
-12.490728515625	4.1100080327623e-10\\
-12.470734375	3.99714091296442e-10\\
-12.450740234375	4.03556714922751e-10\\
-12.43074609375	2.86073539754458e-10\\
-12.410751953125	1.90484178291388e-10\\
-12.3907578125	1.7949014448335e-10\\
-12.370763671875	-2.44739182462166e-12\\
-12.35076953125	5.21028381137941e-11\\
-12.330775390625	-8.63543422757048e-11\\
-12.31078125	4.75280241695442e-11\\
-12.290787109375	-8.74228575100185e-12\\
-12.27079296875	1.75416001820753e-10\\
-12.250798828125	3.05354297773214e-11\\
-12.2308046875	1.14673673191819e-10\\
-12.210810546875	-8.05380412231654e-11\\
-12.19081640625	-8.07227653083369e-11\\
-12.170822265625	-1.35562207332109e-10\\
-12.150828125	-2.25273808751406e-10\\
-12.130833984375	-1.37586933444555e-10\\
-12.11083984375	-2.86755725472097e-10\\
-12.090845703125	-3.09588090003175e-10\\
-12.0708515625	-3.50605084901365e-10\\
-12.050857421875	-2.84406671795079e-10\\
-12.03086328125	-3.10555128486233e-10\\
-12.010869140625	-3.70657358338745e-10\\
-11.990875	-3.08158756970502e-10\\
-11.970880859375	-3.3701480066583e-10\\
-11.95088671875	-4.42675936521615e-10\\
-11.930892578125	-3.62632852770126e-10\\
-11.9108984375	-4.96443369201886e-10\\
-11.890904296875	-5.22589699954748e-10\\
-11.87091015625	-6.23395216277196e-10\\
-11.850916015625	-7.04227085008257e-10\\
-11.830921875	-8.22528142728799e-10\\
-11.810927734375	-9.18818134908206e-10\\
-11.79093359375	-9.82766448995222e-10\\
-11.770939453125	-9.83853399212876e-10\\
-11.7509453125	-8.69894282376886e-10\\
-11.730951171875	-8.91525692269653e-10\\
-11.71095703125	-8.59310528007107e-10\\
-11.690962890625	-7.08886950382436e-10\\
-11.67096875	-7.65256206826781e-10\\
-11.650974609375	-6.81766971918882e-10\\
-11.63098046875	-9.12065368738514e-10\\
-11.610986328125	-7.38551551695288e-10\\
-11.5909921875	-8.1810879781857e-10\\
-11.570998046875	-7.54507206624244e-10\\
-11.55100390625	-7.4111802916128e-10\\
-11.531009765625	-6.57579656416306e-10\\
-11.511015625	-7.89048344077717e-10\\
-11.491021484375	-5.89464210686039e-10\\
-11.47102734375	-7.34540956426283e-10\\
-11.451033203125	-6.8631978039855e-10\\
-11.4310390625	-7.24100063594498e-10\\
-11.411044921875	-6.79087497791468e-10\\
-11.39105078125	-6.31664200849294e-10\\
-11.371056640625	-6.65622312763327e-10\\
-11.3510625	-5.78544749775048e-10\\
-11.331068359375	-6.18346420849484e-10\\
-11.31107421875	-5.43411461989433e-10\\
-11.291080078125	-5.85942915553383e-10\\
-11.2710859375	-5.55780963627091e-10\\
-11.251091796875	-6.78559547120911e-10\\
-11.23109765625	-6.97881967276574e-10\\
-11.211103515625	-7.32654679886691e-10\\
-11.191109375	-7.62246321127996e-10\\
-11.171115234375	-8.98060275685784e-10\\
-11.15112109375	-9.01808609576543e-10\\
-11.131126953125	-8.5775275255489e-10\\
-11.1111328125	-9.38308078477138e-10\\
-11.091138671875	-8.63542750423399e-10\\
-11.07114453125	-9.73651393190118e-10\\
-11.051150390625	-8.74695525104865e-10\\
-11.03115625	-8.8810633465152e-10\\
-11.011162109375	-8.56511504601447e-10\\
-10.99116796875	-8.62697247096231e-10\\
-10.971173828125	-8.07346310801134e-10\\
-10.9511796875	-8.81405736156105e-10\\
-10.931185546875	-9.02528655889489e-10\\
-10.91119140625	-8.46259006137629e-10\\
-10.891197265625	-8.16478533840177e-10\\
-10.871203125	-7.94111485934393e-10\\
-10.851208984375	-8.07185340037711e-10\\
-10.83121484375	-7.50277719494682e-10\\
-10.811220703125	-6.46716999370545e-10\\
-10.7912265625	-6.93668915410102e-10\\
-10.771232421875	-5.29336818693187e-10\\
-10.75123828125	-4.86371792658288e-10\\
-10.731244140625	-2.72235307408222e-10\\
-10.71125	-2.17788418630274e-10\\
-10.691255859375	-5.66970851435383e-13\\
-10.67126171875	4.85895441182225e-11\\
-10.651267578125	6.93490055132711e-11\\
-10.6312734375	1.53783325131657e-10\\
-10.611279296875	2.39206016724091e-10\\
-10.59128515625	2.68743120001299e-10\\
-10.571291015625	2.58873498005625e-10\\
-10.551296875	1.77155621755458e-10\\
-10.531302734375	3.86165479338147e-10\\
-10.51130859375	2.98945509273318e-10\\
-10.491314453125	3.80940076844078e-10\\
-10.4713203125	5.65968571377836e-10\\
-10.451326171875	4.92960723367573e-10\\
-10.43133203125	6.02999740847149e-10\\
-10.411337890625	7.33822507072657e-10\\
-10.39134375	5.88776844812021e-10\\
-10.371349609375	7.815503371436e-10\\
-10.35135546875	6.52647898215948e-10\\
-10.331361328125	8.94953129225957e-10\\
-10.3113671875	6.25532704530998e-10\\
-10.291373046875	8.4053866923119e-10\\
-10.27137890625	7.63473280990501e-10\\
-10.251384765625	8.40087276087184e-10\\
-10.231390625	7.02008970588736e-10\\
-10.211396484375	7.69727834570808e-10\\
-10.19140234375	7.6645606932401e-10\\
-10.171408203125	7.83635133159951e-10\\
-10.1514140625	7.42687418683664e-10\\
-10.131419921875	9.36453179513273e-10\\
-10.11142578125	8.96438117403655e-10\\
-10.091431640625	9.95180968564978e-10\\
-10.0714375	1.06368452869541e-09\\
-10.051443359375	1.09081689495425e-09\\
-10.03144921875	1.23235325981323e-09\\
-10.011455078125	1.18643376341208e-09\\
-9.9914609375	1.21832268966128e-09\\
-9.971466796875	1.25008925291074e-09\\
-9.95147265625	1.24314908095912e-09\\
-9.931478515625	1.40556150389173e-09\\
-9.911484375	1.43093639263761e-09\\
-9.891490234375	1.42556483807301e-09\\
-9.87149609375	1.51311500758281e-09\\
-9.851501953125	1.5409288178438e-09\\
-9.8315078125	1.54179629393498e-09\\
-9.81151367187501	1.51245822669731e-09\\
-9.79151953125	1.68221748439682e-09\\
-9.771525390625	1.55327707655274e-09\\
-9.75153125	1.77380509900831e-09\\
-9.731537109375	1.53919112172524e-09\\
-9.71154296875	1.78568129233843e-09\\
-9.691548828125	1.8439710111613e-09\\
-9.6715546875	1.74486950338666e-09\\
-9.651560546875	1.78748957400236e-09\\
-9.63156640625	1.78620182443287e-09\\
-9.611572265625	1.69899567428459e-09\\
-9.591578125	1.67742987791908e-09\\
-9.571583984375	1.59990362101076e-09\\
-9.55158984375	1.60780916895588e-09\\
-9.531595703125	1.64540049731859e-09\\
-9.5116015625	1.66124879351631e-09\\
-9.491607421875	1.67034858086462e-09\\
-9.47161328125	1.83020696692356e-09\\
-9.451619140625	1.9070803847715e-09\\
-9.431625	1.82304360464053e-09\\
-9.411630859375	2.01953365309392e-09\\
-9.39163671875	1.90804262604076e-09\\
-9.371642578125	1.8892475990896e-09\\
-9.3516484375	1.87714420183594e-09\\
-9.331654296875	1.84125606617625e-09\\
-9.31166015625	1.89959922156222e-09\\
-9.291666015625	1.88413904694208e-09\\
-9.271671875	1.82399739412262e-09\\
-9.251677734375	1.88933889306387e-09\\
-9.23168359375	2.04808838215423e-09\\
-9.211689453125	1.97216386306027e-09\\
-9.1916953125	1.89087666086951e-09\\
-9.171701171875	2.08011613149141e-09\\
-9.15170703125	1.84638861332361e-09\\
-9.131712890625	1.778081327585e-09\\
-9.11171875	1.52683731240732e-09\\
-9.091724609375	1.66650625792405e-09\\
-9.07173046875	1.52705383583135e-09\\
-9.051736328125	1.49424741899654e-09\\
-9.0317421875	1.40454666920866e-09\\
-9.011748046875	1.48935134687763e-09\\
-8.99175390625	1.18956769652566e-09\\
-8.971759765625	1.24625305443259e-09\\
-8.951765625	1.05948645322618e-09\\
-8.931771484375	1.12429635525464e-09\\
-8.91177734375	9.72991166752004e-10\\
-8.891783203125	1.07117031853999e-09\\
-8.8717890625	1.01842874134396e-09\\
-8.851794921875	9.33413161145846e-10\\
-8.83180078125	9.06551205920604e-10\\
-8.811806640625	8.51007296941664e-10\\
-8.7918125	8.99073809918429e-10\\
-8.771818359375	7.90509897731167e-10\\
-8.75182421875	7.45689810273446e-10\\
-8.731830078125	7.63408704942743e-10\\
-8.7118359375	8.02641258802085e-10\\
-8.691841796875	8.08852268800093e-10\\
-8.67184765625	8.70051162585617e-10\\
-8.651853515625	8.58814335462263e-10\\
-8.631859375	9.42359461509826e-10\\
-8.611865234375	9.09066420994493e-10\\
-8.59187109375	8.07852007187029e-10\\
-8.571876953125	8.57003508926015e-10\\
-8.5518828125	8.21230279559409e-10\\
-8.531888671875	7.60839746281066e-10\\
-8.51189453125	6.99012041754413e-10\\
-8.491900390625	6.76343913038905e-10\\
-8.47190625	6.12518585385233e-10\\
-8.451912109375	6.72659088738004e-10\\
-8.43191796875	6.33264248123472e-10\\
-8.411923828125	6.68050731271259e-10\\
-8.3919296875	5.60372151474973e-10\\
-8.371935546875	4.65062431751053e-10\\
-8.35194140625	4.22088719611794e-10\\
-8.331947265625	3.21207385858034e-10\\
-8.311953125	2.29745997332396e-10\\
-8.291958984375	1.2682574551392e-10\\
-8.27196484375	3.32357378311725e-11\\
-8.251970703125	-7.98147861791574e-11\\
-8.2319765625	-2.26048983369435e-10\\
-8.211982421875	-2.55044374583463e-10\\
-8.19198828125	-4.9203172229799e-10\\
-8.171994140625	-5.42797538529652e-10\\
-8.152	-6.66474831815071e-10\\
-8.132005859375	-6.63500832330942e-10\\
-8.11201171875	-8.65081304785116e-10\\
-8.092017578125	-9.69474283603819e-10\\
-8.0720234375	-9.11754475807914e-10\\
-8.052029296875	-1.03068272291276e-09\\
-8.03203515625	-1.06242248176379e-09\\
-8.012041015625	-1.04235061473286e-09\\
-7.992046875	-9.54438623587131e-10\\
-7.972052734375	-1.00167052395691e-09\\
-7.95205859375	-9.32262592083746e-10\\
-7.932064453125	-9.81619195194859e-10\\
-7.9120703125	-1.00565009205182e-09\\
-7.892076171875	-9.17638535209953e-10\\
-7.87208203125	-1.18821400568657e-09\\
-7.852087890625	-1.0008418085231e-09\\
-7.83209375	-1.32309132415747e-09\\
-7.812099609375	-1.10267036676411e-09\\
-7.79210546875	-1.33025910414437e-09\\
-7.772111328125	-1.27427313795959e-09\\
-7.75211718750001	-1.31140164201943e-09\\
-7.732123046875	-1.14392052135076e-09\\
-7.71212890625	-1.17848331201645e-09\\
-7.692134765625	-1.10333036276272e-09\\
-7.672140625	-1.28270872717659e-09\\
-7.652146484375	-1.14704613720914e-09\\
-7.63215234375	-1.40076749901202e-09\\
-7.612158203125	-1.42282396948402e-09\\
-7.5921640625	-1.59212163242398e-09\\
-7.572169921875	-1.66339134044718e-09\\
-7.55217578125	-1.74059740132663e-09\\
-7.53218164062501	-1.8106847186988e-09\\
-7.5121875	-1.72472591355727e-09\\
-7.492193359375	-1.83276363905811e-09\\
-7.47219921875	-1.76953384364726e-09\\
-7.452205078125	-1.82591015630582e-09\\
-7.4322109375	-1.85132407844278e-09\\
-7.412216796875	-1.91125607888281e-09\\
-7.39222265625	-2.01615595443549e-09\\
-7.372228515625	-2.17215518478259e-09\\
-7.352234375	-2.32604287333138e-09\\
-7.332240234375	-2.35505785328992e-09\\
-7.31224609375001	-2.34462678981461e-09\\
-7.292251953125	-2.26441562311876e-09\\
-7.2722578125	-2.21395453287585e-09\\
-7.252263671875	-2.23707646755669e-09\\
-7.23226953125	-2.03654595010596e-09\\
-7.212275390625	-2.30722273641718e-09\\
-7.19228125	-2.27038960088735e-09\\
-7.172287109375	-2.31640535602218e-09\\
-7.15229296875	-2.44190674002099e-09\\
-7.132298828125	-2.50735454034356e-09\\
-7.1123046875	-2.44240116116779e-09\\
-7.092310546875	-2.48145919446274e-09\\
-7.07231640625	-2.46203998783967e-09\\
-7.052322265625	-2.42336474618969e-09\\
-7.032328125	-2.37226067533827e-09\\
-7.012333984375	-2.34181845655738e-09\\
-6.99233984375	-2.20525723505565e-09\\
-6.972345703125	-2.35912603076274e-09\\
-6.9523515625	-2.37504777035543e-09\\
-6.932357421875	-2.26295957257476e-09\\
-6.91236328125	-2.45719082360718e-09\\
-6.892369140625	-2.37708234296371e-09\\
-6.872375	-2.41705365696827e-09\\
-6.852380859375	-2.43437192912325e-09\\
-6.83238671875	-2.40831992681638e-09\\
-6.812392578125	-2.40200773375451e-09\\
-6.7923984375	-2.37700611843204e-09\\
-6.772404296875	-2.31647489012394e-09\\
-6.75241015625	-2.38143033425408e-09\\
-6.732416015625	-2.47028580876311e-09\\
-6.712421875	-2.47498877344622e-09\\
-6.692427734375	-2.37754998629578e-09\\
-6.67243359375	-2.5772278032016e-09\\
-6.652439453125	-2.44059187092413e-09\\
-6.6324453125	-2.44318697638197e-09\\
-6.612451171875	-2.32581307807036e-09\\
-6.59245703125	-2.46015216514229e-09\\
-6.572462890625	-2.26922702324954e-09\\
-6.55246875	-2.30330739399985e-09\\
-6.532474609375	-2.13364956414915e-09\\
-6.51248046875	-2.20392970843039e-09\\
-6.492486328125	-2.03650020400484e-09\\
-6.4724921875	-1.99782872411752e-09\\
-6.452498046875	-1.95936133949713e-09\\
-6.43250390625	-2.06647757792668e-09\\
-6.412509765625	-1.91119859884883e-09\\
-6.392515625	-1.98748360661927e-09\\
-6.372521484375	-1.920343095456e-09\\
-6.35252734375	-1.86498776446459e-09\\
-6.332533203125	-1.88529372980305e-09\\
-6.3125390625	-1.71386121824473e-09\\
-6.292544921875	-1.72098934935217e-09\\
-6.27255078125	-1.50880789809152e-09\\
-6.252556640625	-1.51315416284168e-09\\
-6.2325625	-1.53850094049563e-09\\
-6.212568359375	-1.63985712568791e-09\\
-6.19257421875	-1.71235270018202e-09\\
-6.172580078125	-1.75534617577898e-09\\
-6.1525859375	-1.85202310930233e-09\\
-6.132591796875	-2.01102849655328e-09\\
-6.11259765625	-1.86256613779884e-09\\
-6.092603515625	-1.78209940296285e-09\\
-6.072609375	-1.77374010053444e-09\\
-6.052615234375	-1.67280765554009e-09\\
-6.03262109375	-1.63877300919494e-09\\
-6.012626953125	-1.50237905178733e-09\\
-5.9926328125	-1.65589528610496e-09\\
-5.972638671875	-1.68924892286237e-09\\
-5.95264453125	-1.80309121720548e-09\\
-5.932650390625	-1.68426643622714e-09\\
-5.91265625	-1.69514528817794e-09\\
-5.892662109375	-1.57046086304586e-09\\
-5.87266796875	-1.51306056181143e-09\\
-5.852673828125	-1.35148168205268e-09\\
-5.8326796875	-1.17144138596614e-09\\
-5.812685546875	-1.0721029566084e-09\\
-5.79269140625	-1.04137090021204e-09\\
-5.772697265625	-9.20369650360886e-10\\
-5.752703125	-8.29851544682563e-10\\
-5.732708984375	-7.90894531057839e-10\\
-5.71271484375	-7.29928237581571e-10\\
-5.692720703125	-5.24212013144936e-10\\
-5.6727265625	-4.36999434761845e-10\\
-5.652732421875	-3.52390673085975e-10\\
-5.63273828125	-2.86099883279153e-10\\
-5.612744140625	-1.59246259845907e-10\\
-5.59275	-1.13185798151966e-11\\
-5.572755859375	1.13979725627661e-11\\
-5.55276171875	1.94103365983707e-10\\
-5.532767578125	2.1384881306326e-10\\
-5.5127734375	3.41633514694243e-10\\
-5.492779296875	2.92022575724125e-10\\
-5.47278515625	4.58538619512067e-10\\
-5.452791015625	3.01392909285612e-10\\
-5.432796875	3.7794552243502e-10\\
-5.412802734375	4.52107438043135e-10\\
-5.39280859375	6.49789222521531e-10\\
-5.372814453125	8.66053346161909e-10\\
-5.3528203125	8.46254135581249e-10\\
-5.332826171875	1.27414444551474e-09\\
-5.31283203125	1.19850801521719e-09\\
-5.292837890625	1.51093022594823e-09\\
-5.27284375	1.37201853512653e-09\\
-5.25284960937501	1.56061878295155e-09\\
-5.23285546875	1.27943913983899e-09\\
-5.212861328125	1.29996773328256e-09\\
-5.1928671875	1.28960256773031e-09\\
-5.172873046875	1.34613470091983e-09\\
-5.15287890625	1.40900017058099e-09\\
-5.132884765625	1.72879933178566e-09\\
-5.112890625	1.82347856143954e-09\\
-5.092896484375	2.08735917247099e-09\\
-5.07290234375	2.20784864415368e-09\\
-5.052908203125	2.36487739691896e-09\\
-5.03291406250001	2.35662344466904e-09\\
-5.012919921875	2.35183227623078e-09\\
-4.99292578125	2.46227808671215e-09\\
-4.972931640625	2.2729192706697e-09\\
-4.9529375	2.40197731194952e-09\\
-4.932943359375	2.49836701343963e-09\\
-4.91294921875	2.6132555205453e-09\\
-4.892955078125	2.88513163596387e-09\\
-4.8729609375	2.94679541035709e-09\\
-4.852966796875	3.1564397638466e-09\\
-4.83297265625	3.06987983914951e-09\\
-4.81297851562501	3.15670305560401e-09\\
-4.792984375	3.10927873069521e-09\\
-4.772990234375	3.01396647263866e-09\\
-4.75299609375	2.95209752810954e-09\\
-4.733001953125	3.08130371797518e-09\\
-4.7130078125	3.17206811784837e-09\\
-4.693013671875	3.23256521994844e-09\\
-4.67301953125	3.42845979368117e-09\\
-4.653025390625	3.48110884650081e-09\\
-4.63303125	3.52232115382226e-09\\
-4.613037109375	3.32110654921428e-09\\
-4.59304296875	3.2729614480571e-09\\
-4.573048828125	3.25407394273058e-09\\
-4.5530546875	3.03049430010556e-09\\
-4.533060546875	2.91458775165666e-09\\
-4.51306640625	2.99192533248384e-09\\
-4.493072265625	2.91533339684382e-09\\
-4.473078125	3.11267893572802e-09\\
-4.453083984375	3.23593556451105e-09\\
-4.43308984375	3.24642681357876e-09\\
-4.413095703125	3.32583278428702e-09\\
-4.3931015625	3.15477824248811e-09\\
-4.373107421875	3.28734831239726e-09\\
-4.35311328125	3.20689123597545e-09\\
-4.333119140625	3.09630597692205e-09\\
-4.313125	3.05222713730452e-09\\
-4.293130859375	3.02886546502651e-09\\
-4.27313671875	2.87342980884461e-09\\
-4.253142578125	2.92946139509461e-09\\
-4.2331484375	2.90173274464202e-09\\
-4.213154296875	2.91400420217341e-09\\
-4.19316015625	2.86149725066731e-09\\
-4.173166015625	2.94478163677779e-09\\
-4.153171875	2.83266489201981e-09\\
-4.133177734375	2.82251747676098e-09\\
-4.11318359375	2.71290408352738e-09\\
-4.093189453125	2.787762854041e-09\\
-4.0731953125	2.65910729463146e-09\\
-4.053201171875	2.75619220767317e-09\\
-4.03320703125	2.58327088256554e-09\\
-4.013212890625	2.49263706291719e-09\\
-3.99321875	2.37252174668563e-09\\
-3.973224609375	2.31954528280159e-09\\
-3.95323046875	2.3160743135134e-09\\
-3.933236328125	2.26176700640764e-09\\
-3.9132421875	2.13546061355088e-09\\
-3.893248046875	2.21714032550857e-09\\
-3.87325390625	2.14289432042901e-09\\
-3.853259765625	2.17979519695051e-09\\
-3.833265625	2.21179821830173e-09\\
-3.813271484375	2.12258113473675e-09\\
-3.79327734375	1.97417623303533e-09\\
-3.773283203125	1.77699978046381e-09\\
-3.7532890625	1.84740403967111e-09\\
-3.733294921875	1.75511523827255e-09\\
-3.71330078125	1.73715695951154e-09\\
-3.693306640625	1.86047623236563e-09\\
-3.6733125	1.81127983964593e-09\\
-3.653318359375	1.84318437472825e-09\\
-3.63332421875	1.84930998177028e-09\\
-3.613330078125	1.86619518813006e-09\\
-3.5933359375	1.7598881573967e-09\\
-3.573341796875	1.58430073178165e-09\\
-3.55334765625	1.46869183566642e-09\\
-3.533353515625	1.48043413308367e-09\\
-3.513359375	1.32976409617452e-09\\
-3.493365234375	1.36337549297839e-09\\
-3.47337109375	1.39341207045409e-09\\
-3.453376953125	1.41688598514133e-09\\
-3.4333828125	1.3758555099551e-09\\
-3.413388671875	1.50580446454766e-09\\
-3.39339453125	1.52842826453586e-09\\
-3.373400390625	1.46437846053814e-09\\
-3.35340625	1.31784810991767e-09\\
-3.333412109375	1.23265632609417e-09\\
-3.31341796875	1.08163322891167e-09\\
-3.293423828125	9.53876104089893e-10\\
-3.2734296875	8.01038456714794e-10\\
-3.253435546875	7.37981599660848e-10\\
-3.23344140625	6.33215585376223e-10\\
-3.213447265625	5.77078004529711e-10\\
-3.193453125	4.17600231023627e-10\\
-3.173458984375	4.77761308400579e-10\\
-3.15346484375	4.56477663672484e-10\\
-3.133470703125	4.44346954360297e-10\\
-3.1134765625	4.38254259069945e-10\\
-3.093482421875	5.11868549700338e-10\\
-3.07348828125	5.29921575280523e-10\\
-3.053494140625	3.90700823782495e-10\\
-3.0335	4.62446073652264e-10\\
-3.013505859375	2.79971286379457e-10\\
-2.99351171875	2.22708475167728e-10\\
-2.973517578125	1.39898931645043e-10\\
-2.9535234375	2.29446155324716e-10\\
-2.933529296875	1.77655604444381e-10\\
-2.91353515625	1.75638617008882e-10\\
-2.893541015625	1.0017871059171e-10\\
-2.873546875	9.73431906765097e-11\\
-2.853552734375	-1.16577119782135e-13\\
-2.83355859375	-2.59809342302112e-10\\
-2.813564453125	-1.99676924039327e-11\\
-2.7935703125	-5.15823020198296e-10\\
-2.773576171875	-1.99250542540037e-10\\
-2.75358203125001	-4.71737860049301e-10\\
-2.733587890625	-3.57823871249691e-10\\
-2.71359375	-4.36523937728074e-10\\
-2.693599609375	-3.02588341269143e-10\\
-2.67360546875	-3.7307975455175e-10\\
-2.653611328125	-4.10978081126457e-10\\
-2.6336171875	-5.4200614047103e-10\\
-2.613623046875	-6.29583969804047e-10\\
-2.59362890625	-8.94858606958847e-10\\
-2.573634765625	-1.13379132423565e-09\\
-2.553640625	-1.33545584632971e-09\\
-2.53364648437501	-1.31001798513565e-09\\
-2.51365234375	-1.38929949837817e-09\\
-2.493658203125	-1.35169054406381e-09\\
-2.4736640625	-1.09435542755634e-09\\
-2.453669921875	-1.24349947495773e-09\\
-2.43367578125	-1.32155658544903e-09\\
-2.413681640625	-1.29339916742946e-09\\
-2.3936875	-1.48447663223503e-09\\
-2.373693359375	-1.5763822953682e-09\\
-2.35369921875	-1.78903276366068e-09\\
-2.333705078125	-1.77615729809415e-09\\
-2.3137109375	-1.85450355461305e-09\\
-2.293716796875	-1.72415827215561e-09\\
-2.27372265625	-1.62525850580749e-09\\
-2.253728515625	-1.61125742508234e-09\\
-2.233734375	-1.65168878548325e-09\\
-2.213740234375	-1.61589861990969e-09\\
-2.19374609375	-1.62490630521815e-09\\
-2.173751953125	-1.84655300323615e-09\\
-2.1537578125	-1.78339385110769e-09\\
-2.133763671875	-1.77638364894894e-09\\
-2.11376953125	-1.68829517431467e-09\\
-2.093775390625	-1.67970223793051e-09\\
-2.07378125	-1.66420145660538e-09\\
-2.053787109375	-1.58760908581915e-09\\
-2.03379296875	-1.46995838937878e-09\\
-2.013798828125	-1.71215574806623e-09\\
-1.9938046875	-1.65605345852421e-09\\
-1.973810546875	-1.61645490043383e-09\\
-1.95381640625	-1.68715873358743e-09\\
-1.933822265625	-1.51483963202268e-09\\
-1.913828125	-1.56860821429036e-09\\
-1.893833984375	-1.49514488875404e-09\\
-1.87383984375	-1.83633239019995e-09\\
-1.853845703125	-1.74340620065505e-09\\
-1.8338515625	-1.77002679587719e-09\\
-1.813857421875	-1.85462528039695e-09\\
-1.79386328125	-1.87385154912742e-09\\
-1.773869140625	-1.99688787643132e-09\\
-1.753875	-2.06825980107713e-09\\
-1.733880859375	-1.92565651800773e-09\\
-1.71388671875	-1.94509671674861e-09\\
-1.693892578125	-1.91780465393362e-09\\
-1.6738984375	-1.90071381547304e-09\\
-1.653904296875	-1.82047325614138e-09\\
-1.63391015625	-1.85670736725681e-09\\
-1.613916015625	-1.78410379163241e-09\\
-1.593921875	-1.82568483699151e-09\\
-1.573927734375	-1.71661704793459e-09\\
-1.55393359375	-1.89334172246569e-09\\
-1.533939453125	-1.93139504027013e-09\\
-1.5139453125	-1.79267476357905e-09\\
-1.493951171875	-1.90301264841112e-09\\
-1.47395703125	-1.82563391910799e-09\\
-1.453962890625	-1.84716555661532e-09\\
-1.43396875	-1.67561272790702e-09\\
-1.413974609375	-1.71164538500061e-09\\
-1.39398046875	-1.66617819405685e-09\\
-1.373986328125	-1.5279806060108e-09\\
-1.3539921875	-1.57786818819491e-09\\
-1.333998046875	-1.5295390609162e-09\\
-1.31400390625	-1.45142710833763e-09\\
-1.294009765625	-1.37688311332406e-09\\
-1.274015625	-1.3889692580479e-09\\
-1.254021484375	-1.51819010250428e-09\\
-1.23402734375	-1.5366291812932e-09\\
-1.214033203125	-1.44814864605419e-09\\
-1.1940390625	-1.6719892208418e-09\\
-1.174044921875	-1.58240786466707e-09\\
-1.15405078125	-1.58838350607396e-09\\
-1.134056640625	-1.48136461991714e-09\\
-1.1140625	-1.60421845769247e-09\\
-1.094068359375	-1.39035354991747e-09\\
-1.07407421875	-1.29787102478688e-09\\
-1.054080078125	-1.30934303022681e-09\\
-1.0340859375	-1.22325843177002e-09\\
-1.014091796875	-1.2114978068711e-09\\
-0.994097656249998	-1.18920802673929e-09\\
-0.974103515625004	-1.14849960480663e-09\\
-0.954109375000002	-1.29689432700449e-09\\
-0.934115234375	-1.2525700230769e-09\\
-0.914121093750005	-1.43530045526589e-09\\
-0.894126953125003	-1.3580739330184e-09\\
-0.874132812500001	-1.33037029216357e-09\\
-0.854138671874999	-1.24908420239413e-09\\
-0.834144531250004	-1.18565289140418e-09\\
-0.814150390625002	-9.65307571865525e-10\\
-0.79415625	-7.41544440257986e-10\\
-0.774162109374998	-6.36496882567609e-10\\
-0.754167968750004	-5.56699220078241e-10\\
-0.734173828125002	-3.18553259709848e-10\\
-0.7141796875	-2.29070681670669e-10\\
-0.694185546875005	-1.28855632201547e-10\\
-0.674191406250003	-7.11845666260172e-11\\
-0.654197265625001	9.09325916633877e-11\\
-0.634203124999999	4.97365467785861e-12\\
-0.614208984375004	-1.647435315345e-10\\
-0.594214843750002	-1.71995277137956e-10\\
-0.574220703125	-9.71044258561382e-11\\
-0.554226562499998	-1.33973870038834e-10\\
-0.534232421875004	-2.56420909340857e-11\\
-0.514238281250002	1.97673070623936e-10\\
-0.494244140625	3.36955006228685e-10\\
-0.474250000000005	5.23054070240691e-10\\
-0.454255859375003	5.65934124198206e-10\\
-0.434261718750001	7.04825951478714e-10\\
-0.414267578124999	6.8525831017746e-10\\
-0.394273437500004	6.44611489076195e-10\\
-0.374279296875002	8.20040154932022e-10\\
-0.35428515625	8.52503961846803e-10\\
-0.334291015624999	1.08951178969312e-09\\
-0.314296875000004	1.02475818305233e-09\\
-0.294302734375002	1.4771009712588e-09\\
-0.27430859375	1.31063967517842e-09\\
-0.254314453125005	1.61400439224291e-09\\
-0.234320312500003	1.43331464219694e-09\\
-0.214326171875001	1.55797338187249e-09\\
-0.194332031249999	1.42282988319397e-09\\
-0.174337890625004	1.63516753356147e-09\\
-0.154343750000002	1.78419688482473e-09\\
-0.134349609375001	1.96395891452989e-09\\
-0.114355468749999	2.1353505292129e-09\\
-0.0943613281250038	2.38336280203191e-09\\
-0.0743671875000018	2.63626630685588e-09\\
-0.0543730468749999	2.8572672806023e-09\\
-0.0343789062500051	3.00833497180488e-09\\
-0.0143847656250031	2.97817974380631e-09\\
0.00560937499999881	3.01176635776514e-09\\
0.0256035156250007	2.96064965288453e-09\\
0.0455976562499956	3.05424712210127e-09\\
0.0655917968749975	3.07841641923736e-09\\
0.0855859374999994	3.00575838318436e-09\\
0.105580078125001	3.32523566307239e-09\\
0.125574218749996	3.32694945770131e-09\\
0.145568359374998	3.52001615685182e-09\\
0.1655625	3.53477912574644e-09\\
0.185556640625002	3.97406414593265e-09\\
0.205550781249997	3.93273375728753e-09\\
0.225544921874999	4.03486182525824e-09\\
0.245539062500001	4.08559779286234e-09\\
0.265533203124996	4.11591959282296e-09\\
0.285527343749997	4.18217862718127e-09\\
0.305521484374999	4.18341679471968e-09\\
0.325515625000001	4.33908092143104e-09\\
0.345509765624996	4.2809187928919e-09\\
0.365503906249998	4.36246980481095e-09\\
0.385498046875	4.33238497124844e-09\\
0.405492187500002	4.5345255319201e-09\\
0.425486328124997	4.50501351278548e-09\\
0.445480468749999	4.73401853529727e-09\\
0.465474609375001	4.72471024841569e-09\\
0.485468749999995	4.85143401702039e-09\\
0.505462890624997	4.79618214839651e-09\\
0.525457031249999	4.73229501445863e-09\\
0.545451171875001	4.7848579631603e-09\\
0.565445312499996	4.66695250491337e-09\\
0.585439453124998	4.86431681507119e-09\\
0.60543359375	4.89097248624585e-09\\
0.625427734375002	5.23589036415217e-09\\
0.645421874999997	5.21969455470335e-09\\
0.665416015624999	5.63592095771524e-09\\
0.685410156250001	5.65842195895401e-09\\
0.705404296874995	5.70738957490503e-09\\
0.725398437499997	5.70655377027375e-09\\
0.745392578124999	5.81879385480999e-09\\
0.765386718750001	5.79219919200107e-09\\
0.785380859374996	5.85981100831048e-09\\
0.805374999999998	5.99826027608931e-09\\
0.825369140625	6.05178219603478e-09\\
0.845363281250002	6.28173901548362e-09\\
0.865357421874997	6.35305794661821e-09\\
0.885351562499999	6.48997334605177e-09\\
0.905345703125001	6.60966368395828e-09\\
0.925339843749995	6.44609111044729e-09\\
0.945333984374997	6.57592486367467e-09\\
0.965328124999999	6.55799819124777e-09\\
0.985322265625001	6.50481930833035e-09\\
1.00531640625	6.83133336390645e-09\\
1.025310546875	6.78058703387437e-09\\
1.0453046875	6.91484729359168e-09\\
1.065298828125	6.99855765834363e-09\\
1.08529296875	7.18579985112338e-09\\
1.105287109375	7.32987793523488e-09\\
1.12528125	7.31016609361748e-09\\
1.145275390625	7.42270975524962e-09\\
1.16526953125	7.33230490097392e-09\\
1.185263671875	7.33890170897943e-09\\
1.2052578125	7.27516804406004e-09\\
1.225251953125	7.26771543388481e-09\\
1.24524609375	7.36951280958578e-09\\
1.265240234375	7.39804267221338e-09\\
1.285234375	7.5652747072916e-09\\
1.305228515625	7.77405921297863e-09\\
1.32522265625	7.83890765650069e-09\\
1.345216796875	7.97218212592892e-09\\
1.3652109375	7.93523379070835e-09\\
1.385205078125	8.17431020376346e-09\\
1.40519921875	8.11586606025821e-09\\
1.425193359375	8.18567977236227e-09\\
1.4451875	8.28382560876331e-09\\
1.465181640625	8.08314862334759e-09\\
1.48517578125	8.27423536877177e-09\\
1.505169921875	8.26464365710352e-09\\
1.5251640625	8.30541864796059e-09\\
1.545158203125	8.47061814987612e-09\\
1.56515234375	8.50790626014209e-09\\
1.585146484375	8.65696554160399e-09\\
1.605140625	8.56936774323031e-09\\
1.625134765625	8.50304665186815e-09\\
1.64512890625	8.54326071669137e-09\\
1.665123046875	8.67412051157909e-09\\
1.6851171875	8.56890301708509e-09\\
1.705111328125	8.50199109540827e-09\\
1.72510546875	8.66800379063613e-09\\
1.745099609375	8.60816524889882e-09\\
1.76509375	8.51421016771724e-09\\
1.785087890625	8.49435056356986e-09\\
1.80508203125	8.5156285611924e-09\\
1.825076171875	8.39290998371503e-09\\
1.8450703125	8.36654530441618e-09\\
1.865064453125	8.47753173811206e-09\\
1.88505859375	8.38455650822145e-09\\
1.905052734375	8.48352520222414e-09\\
1.925046875	8.51299222560088e-09\\
1.945041015625	8.6215298717579e-09\\
1.96503515625	8.63467848936375e-09\\
1.985029296875	8.53578162578191e-09\\
2.0050234375	8.58188371426648e-09\\
2.025017578125	8.56239118788708e-09\\
2.04501171875	8.59588661068036e-09\\
2.065005859375	8.63863344989269e-09\\
2.085	8.78214743124395e-09\\
2.104994140625	8.68570798820437e-09\\
2.12498828125	8.54956986330731e-09\\
2.144982421875	8.66958949948049e-09\\
2.1649765625	8.50515514823006e-09\\
2.184970703125	8.508578010063e-09\\
2.20496484375	8.29520221656864e-09\\
2.224958984375	8.40915550798723e-09\\
2.244953125	8.31784017754673e-09\\
2.264947265625	8.47063766623719e-09\\
2.28494140625	8.25392386546394e-09\\
2.304935546875	8.4985857665626e-09\\
2.3249296875	8.10592915190369e-09\\
2.344923828125	8.11063141995379e-09\\
2.36491796875	7.92233037623639e-09\\
2.384912109375	7.69067365361849e-09\\
2.40490625	7.46910992509747e-09\\
2.424900390625	7.28450457337918e-09\\
2.44489453125	7.31927647472161e-09\\
2.464888671875	7.15503977140049e-09\\
2.4848828125	7.25632133726234e-09\\
2.504876953125	7.21819570336726e-09\\
2.52487109375	7.4369659617645e-09\\
2.544865234375	7.25697573723964e-09\\
2.564859375	7.29516308420513e-09\\
2.584853515625	7.27858128619905e-09\\
2.60484765625	7.1463739064364e-09\\
2.624841796875	6.99063973799166e-09\\
2.6448359375	6.98469668501253e-09\\
2.664830078125	6.84250111833315e-09\\
2.68482421875	6.68497300343687e-09\\
2.704818359375	6.77825810418337e-09\\
2.7248125	6.6261511465752e-09\\
2.744806640625	6.7658804497881e-09\\
2.76480078125	6.75369561232081e-09\\
2.784794921875	6.51551241178478e-09\\
2.8047890625	6.52677380329349e-09\\
2.824783203125	6.4148350410217e-09\\
2.84477734375	6.42786171530394e-09\\
2.864771484375	6.36908374848745e-09\\
2.884765625	6.30778524508441e-09\\
2.904759765625	6.21738912648365e-09\\
2.92475390625	6.33308604993886e-09\\
2.944748046875	6.08425261424321e-09\\
2.9647421875	6.29527575602524e-09\\
2.984736328125	6.26942103080538e-09\\
3.00473046875	6.28714848466713e-09\\
3.024724609375	6.27025558063031e-09\\
3.04471875	6.17300696247894e-09\\
3.064712890625	6.16066892235164e-09\\
3.08470703125	5.95976636920968e-09\\
3.104701171875	5.80757178762798e-09\\
3.1246953125	5.54287999705932e-09\\
3.144689453125	5.59521255011879e-09\\
3.16468359375	5.29786455403574e-09\\
3.184677734375	5.49993873697606e-09\\
3.204671875	5.61625715715007e-09\\
3.224666015625	5.64610090778554e-09\\
3.24466015625	5.68278350352445e-09\\
3.264654296875	5.77307429487314e-09\\
3.2846484375	5.55343619662995e-09\\
3.304642578125	5.32474357018525e-09\\
3.32463671875	5.05646988626171e-09\\
3.344630859375	4.74790001354579e-09\\
3.364625	4.63934095544614e-09\\
3.384619140625	4.41477467166602e-09\\
3.40461328125	4.61093813016468e-09\\
3.424607421875	4.73110041568583e-09\\
3.4446015625	4.75128350218794e-09\\
3.464595703125	4.88015739670886e-09\\
3.48458984375	4.89261153114672e-09\\
3.504583984375	4.58918818571584e-09\\
3.524578125	4.54316124294014e-09\\
3.544572265625	4.06197186167692e-09\\
3.56456640625	3.78145370592615e-09\\
3.584560546875	3.61327938758611e-09\\
3.6045546875	3.43172655861527e-09\\
3.624548828125	3.42761060268624e-09\\
3.64454296875	3.37332216663216e-09\\
3.664537109375	3.42369037142862e-09\\
3.68453125	3.52276619553821e-09\\
3.704525390625	3.50877756530048e-09\\
3.72451953125	3.42797573773184e-09\\
3.744513671875	3.36976030807575e-09\\
3.7645078125	3.28880433261172e-09\\
3.784501953125	3.05377754790947e-09\\
3.80449609375	2.76465187667835e-09\\
3.824490234375	2.62597241221388e-09\\
3.844484375	2.42652643589479e-09\\
3.864478515625	2.38058674306705e-09\\
3.88447265625	2.31403722293358e-09\\
3.904466796875	2.40663843576712e-09\\
3.9244609375	2.26163007861647e-09\\
3.944455078125	2.3745689274662e-09\\
3.96444921875	2.46522515321317e-09\\
3.984443359375	2.36526905976122e-09\\
4.0044375	2.3394560750006e-09\\
4.024431640625	2.31237832204446e-09\\
4.04442578125	2.04696174473751e-09\\
4.064419921875	1.92751074526605e-09\\
4.0844140625	1.74990608632764e-09\\
4.104408203125	1.73647110459799e-09\\
4.12440234375	1.83482828382412e-09\\
4.144396484375	1.76425331470654e-09\\
4.164390625	1.72269375779863e-09\\
4.184384765625	1.97328828620755e-09\\
4.20437890625	1.79246472515617e-09\\
4.224373046875	1.83324454947564e-09\\
4.2443671875	1.83177401127222e-09\\
4.264361328125	1.78439477938324e-09\\
4.28435546875	1.59321298412466e-09\\
4.304349609375	1.54852587452787e-09\\
4.32434375	1.47791849978817e-09\\
4.344337890625	1.33297566095001e-09\\
4.36433203125	1.30662054711552e-09\\
4.384326171875	1.3159634251121e-09\\
4.4043203125	1.3320859452508e-09\\
4.424314453125	1.25057552460578e-09\\
4.44430859375	1.32570436993336e-09\\
4.464302734375	1.34867606240299e-09\\
4.484296875	1.19968219521869e-09\\
4.504291015625	1.08409835666976e-09\\
4.52428515625	9.72313524362141e-10\\
4.544279296875	8.83348823870526e-10\\
4.5642734375	7.06785971014597e-10\\
4.584267578125	5.31480117415024e-10\\
4.60426171875	5.1696347232801e-10\\
4.624255859375	6.45702254702439e-10\\
4.64425	5.16354836901069e-10\\
4.664244140625	6.13954474091524e-10\\
4.68423828125	6.22361013695737e-10\\
4.704232421875	7.57908875817557e-10\\
4.7242265625	5.07401283900847e-10\\
4.744220703125	6.31942109145735e-10\\
4.76421484375	3.36667388509363e-10\\
4.784208984375	4.65213339250033e-10\\
4.804203125	1.06850207437479e-10\\
4.824197265625	2.22181955979874e-10\\
4.84419140625	2.16472161859661e-10\\
4.864185546875	3.20953902353743e-10\\
4.8841796875	3.2221019251199e-10\\
4.904173828125	2.97303571770928e-10\\
4.92416796875	3.60423980132312e-10\\
4.944162109375	3.18132494833249e-10\\
4.96415625	5.593052751789e-10\\
4.984150390625	4.09994646270021e-10\\
5.00414453125	5.89681099627005e-10\\
5.024138671875	4.89180133099639e-10\\
5.0441328125	4.16417382342124e-10\\
5.064126953125	2.65725597602058e-10\\
5.08412109375	1.56120403404205e-10\\
5.104115234375	1.27144065830257e-11\\
5.124109375	-6.78123710159194e-11\\
5.144103515625	-8.32347585463697e-11\\
5.16409765625	-1.34335357301134e-11\\
5.184091796875	2.12394978388306e-11\\
5.2040859375	2.17940358842981e-11\\
5.224080078125	1.08915844933366e-10\\
5.24407421875	1.13617459092708e-10\\
5.264068359375	-2.94057230793085e-11\\
5.2840625	-9.68253355372961e-12\\
5.304056640625	-9.96470757470002e-11\\
5.32405078125	2.83791223562129e-11\\
5.344044921875	-2.24368190016802e-10\\
5.3640390625	-1.8595775219559e-11\\
5.384033203125	-4.6181164735369e-11\\
5.40402734375	-8.74180475754726e-11\\
5.424021484375	-1.16290293964589e-10\\
5.444015625	9.34424711882407e-11\\
5.464009765625	-1.75703714138018e-10\\
5.48400390625	-7.44437193920735e-11\\
5.503998046875	-3.56822026474074e-10\\
5.5239921875	-3.05948579408841e-10\\
5.543986328125	-4.87224654559782e-10\\
5.56398046875	-5.49272490057422e-10\\
5.583974609375	-4.58620790602738e-10\\
5.60396875	-5.07794741591959e-10\\
5.623962890625	-3.88791154312776e-10\\
5.64395703125	-4.02919166157353e-10\\
5.663951171875	-3.4758761479741e-10\\
5.6839453125	-4.09102181213038e-10\\
5.703939453125	-5.70300760418054e-10\\
5.72393359375	-6.8241045993445e-10\\
5.743927734375	-8.25711406265661e-10\\
5.763921875	-8.72700097939087e-10\\
5.783916015625	-9.06102225279948e-10\\
5.80391015625	-9.6767070640168e-10\\
5.823904296875	-7.95932026334716e-10\\
5.8438984375	-6.95907975785173e-10\\
5.863892578125	-6.75920413027742e-10\\
5.88388671875	-5.87322893857931e-10\\
5.903880859375	-6.48892795517988e-10\\
5.923875	-6.72334914090539e-10\\
5.943869140625	-7.45198562200654e-10\\
5.96386328125	-7.95699811798504e-10\\
5.983857421875	-8.24478898893438e-10\\
6.0038515625	-8.91452734631571e-10\\
6.023845703125	-1.01473895071676e-09\\
6.04383984375	-7.02062372675025e-10\\
6.063833984375	-8.19843473607953e-10\\
6.083828125	-6.26543045820623e-10\\
6.103822265625	-5.8234297644784e-10\\
6.12381640625	-5.26185030549528e-10\\
6.143810546875	-5.96418595761481e-10\\
6.1638046875	-6.33324560062684e-10\\
6.183798828125	-7.61227452447464e-10\\
6.20379296875	-8.7474549692074e-10\\
6.223787109375	-8.92177441870418e-10\\
6.24378125	-8.82141438105909e-10\\
6.263775390625	-7.68397898862214e-10\\
6.28376953125	-7.50197222212509e-10\\
6.303763671875	-5.10174831698928e-10\\
6.3237578125	-4.12319253639305e-10\\
6.343751953125	-3.38031945751209e-10\\
6.36374609375	-2.7804652299903e-10\\
6.383740234375	-2.78710444334868e-10\\
6.403734375	-3.04285780284165e-10\\
6.423728515625	-3.98192972177507e-10\\
6.44372265625	-5.13547121355646e-10\\
6.463716796875	-6.58898099562478e-10\\
6.4837109375	-6.15770749161537e-10\\
6.503705078125	-7.21908284772734e-10\\
6.52369921875	-6.71084382351606e-10\\
6.543693359375	-5.53907740009293e-10\\
6.5636875	-6.18349286795557e-10\\
6.583681640625	-4.43772874319929e-10\\
6.60367578125	-4.66503840044168e-10\\
6.623669921875	-5.36176352250234e-10\\
6.6436640625	-5.50871915011876e-10\\
6.663658203125	-7.31034502535368e-10\\
6.68365234375	-9.81205195983074e-10\\
6.703646484375	-1.00305964620863e-09\\
6.723640625	-1.16800678311318e-09\\
6.743634765625	-1.2077432590235e-09\\
6.76362890625	-1.10414148725165e-09\\
6.783623046875	-1.04035241184062e-09\\
6.8036171875	-1.10066464891266e-09\\
6.823611328125	-9.33528816832026e-10\\
6.84360546875	-1.03925879291684e-09\\
6.863599609375	-1.00664513918389e-09\\
6.88359375	-1.1535746835732e-09\\
6.903587890625	-1.13188072291328e-09\\
6.92358203125	-1.23885222708475e-09\\
6.943576171875	-1.25634036876971e-09\\
6.9635703125	-1.29555387287696e-09\\
6.983564453125	-1.16419476641561e-09\\
7.00355859375	-1.16584704258243e-09\\
7.023552734375	-1.11732514786215e-09\\
7.043546875	-1.05079263023908e-09\\
7.063541015625	-9.34659469572533e-10\\
7.08353515625	-8.24746437293904e-10\\
7.103529296875	-6.86984553054923e-10\\
7.1235234375	-7.1633890172681e-10\\
7.143517578125	-7.622119937066e-10\\
7.16351171875	-8.45473625523174e-10\\
7.183505859375	-8.48494789519175e-10\\
7.2035	-9.13204747529994e-10\\
7.223494140625	-7.26063239456401e-10\\
7.24348828125	-8.77396797870207e-10\\
7.263482421875	-7.2074480698679e-10\\
7.2834765625	-8.37532052717011e-10\\
7.303470703125	-6.27444992480933e-10\\
7.32346484375	-5.82521386458821e-10\\
7.343458984375	-5.50504640406064e-10\\
7.363453125	-5.19032231217021e-10\\
7.383447265625	-5.17741205247511e-10\\
7.40344140625	-5.7508007868948e-10\\
7.423435546875	-7.22860794480742e-10\\
7.4434296875	-6.99409567044089e-10\\
7.463423828125	-7.96107396958832e-10\\
7.48341796875	-6.15355682733045e-10\\
7.503412109375	-7.82523527665709e-10\\
7.52340625	-6.91902076518022e-10\\
7.543400390625	-4.73364055925901e-10\\
7.56339453125	-5.56024754945483e-10\\
7.583388671875	-4.46830473451181e-10\\
7.6033828125	-4.62399562309305e-10\\
7.623376953125	-3.30092337362134e-10\\
7.64337109375	-5.36558761781944e-10\\
7.663365234375	-5.32176283019393e-10\\
7.683359375	-4.34107502834457e-10\\
7.703353515625	-6.10020202000709e-10\\
7.72334765625	-6.33180355055849e-10\\
7.743341796875	-5.64437942308094e-10\\
7.7633359375	-4.83645437326984e-10\\
7.783330078125	-3.50458992982641e-10\\
7.80332421875	-3.05994864636508e-10\\
7.823318359375	-3.40205935712525e-10\\
7.8433125	-2.10271756062105e-10\\
7.863306640625	-4.20907023190027e-10\\
7.88330078125	-3.60619985025104e-10\\
7.903294921875	-4.80416087857523e-10\\
7.9232890625	-5.52021427626046e-10\\
7.943283203125	-6.42381053060868e-10\\
7.96327734375	-4.46244078371652e-10\\
7.983271484375	-4.94841840695982e-10\\
8.003265625	-1.94897370006027e-10\\
8.023259765625	-1.29188109466082e-10\\
8.04325390625	1.43658260174776e-10\\
8.063248046875	8.40556618364793e-11\\
8.0832421875	1.144159746938e-10\\
8.103236328125	1.26658527422139e-10\\
8.12323046875	-3.02003026525159e-11\\
8.143224609375	-1.75644969228999e-10\\
8.16321875	-9.79988799188573e-11\\
8.183212890625	-2.01469831046615e-10\\
8.20320703125	-1.07673307541535e-10\\
8.223201171875	-5.48195829735559e-11\\
8.2431953125	1.03971752246478e-10\\
8.263189453125	2.0186792093754e-10\\
8.28318359375	3.03074996839642e-10\\
8.303177734375	3.57694790291402e-10\\
8.323171875	3.62573095810162e-10\\
8.343166015625	3.51707331124852e-10\\
8.36316015625	3.1546938637029e-10\\
8.383154296875	2.45654315616866e-10\\
8.4031484375	2.4666997513848e-10\\
8.423142578125	1.59102112361795e-10\\
8.44313671875	2.49784214936258e-10\\
8.463130859375	2.22196106990305e-10\\
8.483125	1.66073962862514e-10\\
8.503119140625	2.2356765196584e-10\\
8.52311328125	3.31120892888219e-10\\
8.543107421875	1.51876795084487e-10\\
8.5631015625	3.00476303169375e-10\\
8.583095703125	1.28736780433111e-10\\
8.60308984375	1.47744919897621e-10\\
8.623083984375	4.55073772994653e-11\\
8.643078125	2.21919666831874e-12\\
8.663072265625	-5.75602684473136e-11\\
8.68306640625	-2.1936726916882e-11\\
8.703060546875	-3.78582584227098e-11\\
8.7230546875	5.58571488782169e-11\\
8.743048828125	1.17694536522715e-10\\
8.76304296875	1.31502913022457e-10\\
8.783037109375	1.09569081137236e-10\\
8.80303125	1.02909898046372e-10\\
8.823025390625	-5.12337897978566e-12\\
8.84301953125	-1.38251566632156e-10\\
8.863013671875	-2.15615956627366e-10\\
8.8830078125	-3.13307580148502e-10\\
8.903001953125	-2.53718126036474e-10\\
8.92299609375	-3.10852060838014e-10\\
8.942990234375	-2.32679751396131e-10\\
8.962984375	-1.89811356338684e-10\\
8.982978515625	-1.65648854402088e-10\\
9.00297265625	1.55069480514295e-12\\
9.022966796875	3.09849021128154e-12\\
9.0429609375	-1.00043172132014e-10\\
9.062955078125	1.85001725870808e-11\\
9.08294921875	-1.67559930859412e-10\\
9.102943359375	-8.57558913788397e-11\\
9.1229375	-1.92857153376975e-10\\
9.142931640625	2.14221219173703e-11\\
9.16292578125	5.70840744197617e-11\\
9.182919921875	6.12051297036757e-11\\
9.2029140625	2.49751308382656e-10\\
9.222908203125	2.03711787386318e-10\\
9.24290234375	1.77212262073581e-10\\
9.262896484375	1.22095487072989e-10\\
9.282890625	2.74720616539743e-10\\
9.302884765625	1.89990426003893e-10\\
9.32287890625	9.79010384197572e-11\\
9.342873046875	2.16982352839938e-10\\
9.3628671875	2.49753561873153e-10\\
9.382861328125	3.42546553525567e-10\\
9.40285546875	3.97469983669155e-10\\
9.422849609375	5.4430488641229e-10\\
9.44284375	6.05122093782794e-10\\
9.462837890625	6.27773844597033e-10\\
9.48283203125	5.06294955027526e-10\\
9.502826171875	4.19193741479462e-10\\
9.5228203125	3.18140032120496e-10\\
9.542814453125	2.10306107823676e-10\\
9.56280859375	1.6878560459674e-10\\
9.582802734375	2.03776968595529e-10\\
9.602796875	5.92034920366801e-11\\
9.622791015625	1.55277477364115e-10\\
9.64278515625	2.23741094427599e-10\\
9.662779296875	2.99259304546546e-10\\
9.6827734375	3.44395854636005e-10\\
9.702767578125	4.17902734362123e-10\\
9.72276171875	2.41675307133908e-10\\
9.742755859375	3.78270646329319e-10\\
9.76275	1.46115713552879e-10\\
9.782744140625	1.50255549568009e-10\\
9.80273828125	9.21649583803322e-11\\
9.822732421875	9.55267244581948e-11\\
9.8427265625	7.95886665684753e-11\\
9.862720703125	1.36933235949478e-10\\
9.88271484375	2.28633546975316e-10\\
9.902708984375	3.37719477810193e-10\\
9.922703125	3.55721312679043e-10\\
9.942697265625	2.95451091130058e-10\\
9.96269140625	3.78639188186261e-10\\
9.982685546875	3.18813421726041e-10\\
10.0026796875	4.05171387597476e-10\\
10.022673828125	4.00538100064734e-10\\
10.04266796875	4.69168731655885e-10\\
10.062662109375	4.5189748442547e-10\\
10.08265625	3.37253318055188e-10\\
10.102650390625	3.03468250058037e-10\\
10.12264453125	1.83540686578862e-10\\
10.142638671875	3.1286248431921e-10\\
10.1626328125	2.10885061567163e-10\\
10.182626953125	3.58216812576962e-10\\
10.20262109375	4.05389186444468e-10\\
10.222615234375	3.48429615574268e-10\\
10.242609375	3.61735392650644e-10\\
10.262603515625	4.57043364208171e-10\\
10.28259765625	3.5952107858295e-10\\
10.302591796875	3.34756231126814e-10\\
10.3225859375	3.38389100449998e-10\\
10.342580078125	3.60856784040335e-10\\
10.36257421875	3.66222907750026e-10\\
10.382568359375	3.26857526871318e-10\\
10.4025625	4.73157851936561e-10\\
10.422556640625	4.08944126486727e-10\\
10.44255078125	4.10499970392692e-10\\
10.462544921875	2.94353835904058e-10\\
10.4825390625	3.61473392391902e-10\\
10.502533203125	1.9864899812827e-10\\
10.52252734375	2.61838981315026e-10\\
10.542521484375	-2.55723390135975e-11\\
10.562515625	3.90718331822436e-11\\
10.582509765625	-8.92423627821253e-12\\
10.60250390625	-4.56811325799024e-11\\
10.622498046875	2.5034831097899e-11\\
10.6424921875	1.73110144313347e-10\\
10.662486328125	-1.80765060756552e-11\\
10.68248046875	2.0085681965887e-10\\
10.702474609375	1.61970906798907e-10\\
10.72246875	1.40847798889355e-10\\
10.742462890625	4.36643240971082e-11\\
10.76245703125	-1.58214578569261e-11\\
10.782451171875	-6.39820264785781e-11\\
10.8024453125	-1.57016530853663e-10\\
10.822439453125	-2.99136643821768e-10\\
10.84243359375	-1.41615561480442e-10\\
10.862427734375	-1.32970959604997e-10\\
10.882421875	-2.19934324693995e-10\\
10.902416015625	-1.25866519580564e-10\\
10.92241015625	-1.22081031088955e-10\\
10.942404296875	-2.24483321521234e-10\\
10.9623984375	-1.65310905033009e-10\\
10.982392578125	-8.58342848223501e-11\\
11.00238671875	-2.31420594409035e-10\\
11.022380859375	-2.45683446202332e-10\\
11.042375	-1.89073465809829e-10\\
11.062369140625	-3.18123112132924e-10\\
11.08236328125	-2.13949202863383e-10\\
11.102357421875	-2.83780941032272e-10\\
11.1223515625	-2.36957271574236e-10\\
11.142345703125	-2.40118987678699e-10\\
11.16233984375	-4.00742305491725e-10\\
11.182333984375	-3.43791537339429e-10\\
11.202328125	-4.07406092777266e-10\\
11.222322265625	-4.80560001455744e-10\\
11.24231640625	-4.54732394565579e-10\\
11.262310546875	-4.91621353347881e-10\\
11.2823046875	-5.47535245981806e-10\\
11.302298828125	-6.03177198498129e-10\\
11.32229296875	-5.30029648902872e-10\\
11.342287109375	-4.85137726162804e-10\\
11.36228125	-5.7845173277839e-10\\
11.382275390625	-4.95346853264279e-10\\
11.40226953125	-4.43538226715106e-10\\
11.422263671875	-4.95545676787647e-10\\
11.4422578125	-4.95771500832292e-10\\
11.462251953125	-5.51129950022212e-10\\
11.48224609375	-5.63836349932615e-10\\
11.502240234375	-7.18065312442335e-10\\
11.522234375	-6.40816678145751e-10\\
11.542228515625	-6.58642767391368e-10\\
11.56222265625	-7.89024114803985e-10\\
11.582216796875	-7.9681617637098e-10\\
11.6022109375	-8.84930644918549e-10\\
11.622205078125	-8.79597520798274e-10\\
11.64219921875	-9.26904849594964e-10\\
11.662193359375	-9.02671000988506e-10\\
11.6821875	-9.97572836001381e-10\\
11.702181640625	-1.10588661470949e-09\\
11.72217578125	-1.00278262405358e-09\\
11.742169921875	-1.19270957953616e-09\\
11.7621640625	-1.18445639398157e-09\\
11.782158203125	-1.34160178996194e-09\\
11.80215234375	-1.25691085516667e-09\\
11.822146484375	-1.33012601128346e-09\\
11.842140625	-1.37087746560596e-09\\
11.862134765625	-1.3612821585237e-09\\
11.88212890625	-1.38966603482307e-09\\
11.902123046875	-1.31719051322325e-09\\
11.9221171875	-1.38321617194247e-09\\
11.942111328125	-1.38261200449706e-09\\
11.96210546875	-1.44479577374202e-09\\
11.982099609375	-1.45566289820096e-09\\
12.00209375	-1.4934292930291e-09\\
12.022087890625	-1.46786055824763e-09\\
12.04208203125	-1.33192761375721e-09\\
12.062076171875	-1.21968216815296e-09\\
12.0820703125	-1.24319741619231e-09\\
12.102064453125	-1.05058346773158e-09\\
12.12205859375	-1.03648475774172e-09\\
12.142052734375	-1.10164826864087e-09\\
12.162046875	-1.12182041774374e-09\\
12.182041015625	-1.22017701422529e-09\\
12.20203515625	-1.2188071039856e-09\\
12.222029296875	-1.16495052814512e-09\\
12.2420234375	-1.15562191704002e-09\\
12.262017578125	-9.75553538609286e-10\\
12.28201171875	-9.21420368496751e-10\\
12.302005859375	-8.26737261027369e-10\\
12.322	-8.49494436689467e-10\\
12.341994140625	-6.67733080129369e-10\\
12.36198828125	-7.39005410180583e-10\\
12.381982421875	-7.99687154551096e-10\\
12.4019765625	-8.66625294833234e-10\\
12.421970703125	-8.76655947553402e-10\\
12.44196484375	-8.36831244668606e-10\\
12.461958984375	-1.02270566865145e-09\\
12.481953125	-9.37095084490056e-10\\
12.501947265625	-8.97981792794244e-10\\
12.52194140625	-9.5904686973636e-10\\
12.541935546875	-1.04979382487875e-09\\
12.5619296875	-9.76178938465755e-10\\
12.581923828125	-9.61598842673286e-10\\
12.60191796875	-8.39940936822352e-10\\
12.621912109375	-7.59553024756895e-10\\
12.64190625	-6.8182853211615e-10\\
12.661900390625	-6.18332201044528e-10\\
12.68189453125	-7.44181245507366e-10\\
12.701888671875	-7.58012198223324e-10\\
12.7218828125	-7.60616185712228e-10\\
12.741876953125	-7.48844922660996e-10\\
12.76187109375	-9.54220234458437e-10\\
12.781865234375	-8.54935077125031e-10\\
12.801859375	-7.90667866664766e-10\\
12.821853515625	-7.85668083174849e-10\\
12.84184765625	-7.60893353051773e-10\\
12.861841796875	-6.14429225592418e-10\\
12.8818359375	-5.34658102670847e-10\\
12.901830078125	-6.15006241877357e-10\\
12.92182421875	-5.08365769738872e-10\\
12.941818359375	-6.24011375757567e-10\\
12.9618125	-4.66522163016834e-10\\
12.981806640625	-5.16684349258175e-10\\
13.00180078125	-4.83491934422502e-10\\
13.021794921875	-4.24458233316452e-10\\
13.0417890625	-1.92123373534637e-10\\
13.061783203125	-1.99648576828866e-10\\
13.08177734375	-1.01887042206476e-10\\
13.101771484375	-1.63990160949557e-10\\
13.121765625	-1.89279129283979e-10\\
13.141759765625	-2.12288195043888e-10\\
13.16175390625	-2.54809972711715e-10\\
13.181748046875	-3.69558083459425e-10\\
13.2017421875	-3.29785691453146e-10\\
13.221736328125	-3.31389113471897e-10\\
13.24173046875	-2.86976543712987e-10\\
13.261724609375	-2.70568783273863e-10\\
13.28171875	-9.08150987590826e-11\\
13.301712890625	-4.43282876980236e-11\\
13.32170703125	-5.01487709733156e-11\\
13.341701171875	1.17066503301564e-11\\
13.3616953125	9.21531969825708e-11\\
13.381689453125	-4.39582183153762e-11\\
13.40168359375	-3.66867417466399e-11\\
13.421677734375	-2.00150849654324e-10\\
13.441671875	-2.57475041587605e-10\\
13.461666015625	-3.97756452260596e-10\\
13.48166015625	-3.10813675886119e-10\\
13.501654296875	-3.22678457264066e-10\\
13.5216484375	-3.18833932609442e-10\\
13.541642578125	-2.04920820455658e-10\\
13.56163671875	-5.74833044882269e-11\\
13.581630859375	-2.48944005535228e-11\\
13.601625	5.01831942105855e-11\\
13.621619140625	-6.68045611799597e-11\\
13.64161328125	1.76406136717198e-12\\
13.661607421875	-1.27902824711034e-10\\
13.6816015625	-1.96521276870514e-10\\
13.701595703125	-2.50780691922517e-10\\
13.72158984375	-3.0155623746091e-10\\
13.741583984375	-3.53935062745147e-10\\
13.761578125	-2.40614696911449e-10\\
13.781572265625	-2.65633734087938e-10\\
13.80156640625	-1.82158519115668e-10\\
13.821560546875	-2.11288376602966e-10\\
13.8415546875	-1.59841244489022e-10\\
13.861548828125	-1.56652320077334e-10\\
13.88154296875	-2.35769848711412e-10\\
13.901537109375	-1.22082204381845e-10\\
13.92153125	-1.43969820142178e-10\\
13.941525390625	-2.46965592330983e-10\\
13.96151953125	-6.5968307889037e-11\\
13.981513671875	-7.04195484487428e-11\\
14.0015078125	-8.93601119208443e-11\\
14.021501953125	-4.24272533689112e-11\\
14.04149609375	-7.42536040353956e-11\\
14.061490234375	2.69191735041818e-11\\
14.081484375	1.27061020495145e-10\\
14.101478515625	8.98320076344851e-11\\
14.12147265625	7.9874708133426e-11\\
14.141466796875	7.50596095182756e-11\\
14.1614609375	4.30595553539204e-11\\
14.181455078125	1.65492341999892e-10\\
14.20144921875	1.80433349159795e-10\\
14.221443359375	7.25988891919714e-11\\
14.2414375	2.73134966339316e-10\\
14.261431640625	2.33651606817574e-10\\
14.28142578125	3.49460294510515e-10\\
14.301419921875	2.55680551593289e-10\\
14.3214140625	4.77971881199457e-10\\
14.341408203125	3.92506407987569e-10\\
14.36140234375	3.48212411365719e-10\\
14.381396484375	3.9049702510879e-10\\
14.401390625	4.15017940001783e-10\\
14.421384765625	5.29637644640042e-10\\
14.44137890625	4.96238744777706e-10\\
14.461373046875	6.56531612264884e-10\\
14.4813671875	8.21361915359368e-10\\
14.501361328125	8.37824328458142e-10\\
14.52135546875	8.48719434219759e-10\\
14.541349609375	7.65960461170205e-10\\
14.56134375	6.70596789062847e-10\\
14.581337890625	6.29033235585805e-10\\
14.60133203125	5.30617643021481e-10\\
14.621326171875	5.65046951153412e-10\\
14.6413203125	5.34817984718375e-10\\
14.661314453125	4.75155928204267e-10\\
14.68130859375	5.4886942812274e-10\\
14.701302734375	6.39339066588278e-10\\
14.721296875	6.22589790832237e-10\\
14.741291015625	6.38529733543349e-10\\
14.76128515625	5.8426085458218e-10\\
14.781279296875	7.09263728850483e-10\\
14.8012734375	5.07142588820702e-10\\
14.821267578125	5.02835021595284e-10\\
14.84126171875	4.81886420221912e-10\\
14.861255859375	4.99726383358661e-10\\
14.88125	6.10431011555269e-10\\
14.901244140625	6.63040307981046e-10\\
14.92123828125	6.99616998660864e-10\\
14.941232421875	7.38903181341506e-10\\
14.9612265625	8.75500334663732e-10\\
14.981220703125	8.13736405714969e-10\\
15.00121484375	7.98579110249969e-10\\
15.021208984375	9.22215421626415e-10\\
15.041203125	8.63356761787903e-10\\
15.061197265625	9.80959417234143e-10\\
15.08119140625	9.94927880422755e-10\\
15.101185546875	1.01510587232491e-09\\
15.1211796875	8.99008959167106e-10\\
15.141173828125	9.66340984284226e-10\\
15.16116796875	9.83811648538048e-10\\
15.181162109375	9.13323369220309e-10\\
15.20115625	9.70614990336358e-10\\
15.221150390625	9.71359713497697e-10\\
15.24114453125	1.11640907065591e-09\\
15.261138671875	1.15778627885633e-09\\
15.2811328125	1.18563609378028e-09\\
15.301126953125	1.18927981381262e-09\\
15.32112109375	1.18741196135795e-09\\
15.341115234375	1.11607430097788e-09\\
15.361109375	1.01057525084216e-09\\
15.381103515625	1.11930519182015e-09\\
15.40109765625	1.09168709708819e-09\\
15.421091796875	1.1001762093717e-09\\
15.4410859375	1.31255320542856e-09\\
15.461080078125	1.1273291146392e-09\\
15.48107421875	1.2160926199012e-09\\
15.501068359375	1.16793904433698e-09\\
15.5210625	1.11477410893997e-09\\
15.541056640625	9.7532684866775e-10\\
15.56105078125	1.02736997894273e-09\\
15.581044921875	9.79003238791992e-10\\
15.6010390625	1.07888291493962e-09\\
15.621033203125	1.06689590583567e-09\\
15.64102734375	1.01889330834011e-09\\
15.661021484375	1.12098938158714e-09\\
15.681015625	1.16435794969369e-09\\
15.701009765625	1.05284598222689e-09\\
15.72100390625	1.100223203047e-09\\
15.740998046875	1.1205830341205e-09\\
15.7609921875	1.05006056181738e-09\\
15.780986328125	1.01251188253168e-09\\
15.80098046875	1.13208825127568e-09\\
15.820974609375	1.11234257447704e-09\\
15.84096875	1.06463117653943e-09\\
15.860962890625	1.12492597877824e-09\\
15.88095703125	1.29849318396673e-09\\
15.900951171875	1.31993556524939e-09\\
15.9209453125	1.38441578007585e-09\\
15.940939453125	1.42724991592273e-09\\
15.96093359375	1.47937879301331e-09\\
15.980927734375	1.44615860516656e-09\\
16.000921875	1.43674436826363e-09\\
16.020916015625	1.36557349292152e-09\\
16.04091015625	1.3934933452796e-09\\
16.060904296875	1.15453544292523e-09\\
16.0808984375	1.16672398446503e-09\\
16.100892578125	1.20403168484481e-09\\
16.12088671875	1.11715719332982e-09\\
16.140880859375	1.06532491124055e-09\\
16.160875	1.26263876026598e-09\\
16.180869140625	1.20396057938116e-09\\
16.20086328125	1.31999329175509e-09\\
16.220857421875	1.31202028049851e-09\\
16.2408515625	1.36328255293621e-09\\
16.260845703125	1.29259113834729e-09\\
16.28083984375	1.34654329436693e-09\\
16.300833984375	1.31178478727946e-09\\
16.320828125	1.25474937202652e-09\\
16.340822265625	1.08086272441805e-09\\
16.36081640625	1.16446794299765e-09\\
16.380810546875	1.18006523549465e-09\\
16.4008046875	1.01687284009402e-09\\
16.420798828125	1.07368713320662e-09\\
16.44079296875	1.17920879809117e-09\\
16.460787109375	1.16786603799338e-09\\
16.48078125	1.19765611640102e-09\\
16.500775390625	1.2404661381153e-09\\
16.52076953125	1.29300005715405e-09\\
16.540763671875	1.25631367062783e-09\\
16.5607578125	1.31540556625025e-09\\
16.580751953125	1.223025951781e-09\\
16.60074609375	1.27035721258018e-09\\
16.620740234375	1.23840516339852e-09\\
16.640734375	1.23432438536944e-09\\
16.660728515625	1.17762424195592e-09\\
16.68072265625	1.03238919349447e-09\\
16.700716796875	9.62970467985531e-10\\
16.7207109375	1.01640915116563e-09\\
16.740705078125	8.4404659814402e-10\\
16.76069921875	9.23890285540814e-10\\
16.780693359375	8.07175016595679e-10\\
16.8006875	1.00779518921576e-09\\
16.820681640625	8.30603897565982e-10\\
16.84067578125	9.44678567901399e-10\\
16.860669921875	8.96902997084825e-10\\
16.8806640625	7.3889333380963e-10\\
16.900658203125	6.23798938108588e-10\\
16.92065234375	4.47475410106721e-10\\
16.940646484375	4.84768997988611e-10\\
16.960640625	1.8735640490425e-10\\
16.980634765625	1.87880298949709e-10\\
17.00062890625	1.20854454978e-10\\
17.020623046875	2.10834149556666e-10\\
17.0406171875	2.15415818921634e-10\\
17.060611328125	3.20317926291025e-10\\
17.08060546875	4.00918392305313e-10\\
17.100599609375	4.92657219886169e-10\\
17.12059375	4.8256073194393e-10\\
17.140587890625	5.21045430075721e-10\\
17.16058203125	5.14261755911565e-10\\
17.180576171875	4.04206934793047e-10\\
17.2005703125	3.46431218787839e-10\\
17.220564453125	3.44997012990813e-10\\
17.24055859375	2.27114047563832e-10\\
17.260552734375	3.28902698554785e-10\\
17.280546875	1.78422154921583e-10\\
17.300541015625	3.44692415918816e-10\\
17.32053515625	3.23088472362069e-10\\
17.340529296875	2.89603559931668e-10\\
17.3605234375	3.18307684438513e-10\\
17.380517578125	3.19697688240067e-10\\
17.40051171875	3.63486782377418e-10\\
17.420505859375	3.24579966431406e-10\\
17.4405	3.2647502043999e-10\\
17.460494140625	3.55059559384675e-10\\
17.48048828125	2.33605256235906e-10\\
17.500482421875	1.85349220095127e-10\\
17.5204765625	7.34496637016473e-11\\
17.540470703125	9.45796105631848e-11\\
17.56046484375	3.18722154440534e-11\\
17.580458984375	-3.67991548931008e-11\\
17.600453125	-9.40635477628119e-12\\
17.620447265625	1.35610729248455e-10\\
17.64044140625	1.27281291916715e-10\\
17.660435546875	2.869450836489e-10\\
17.6804296875	3.45696313981409e-10\\
17.700423828125	2.53106631445468e-10\\
17.72041796875	2.67141291943972e-10\\
17.740412109375	2.14698860004276e-10\\
17.76040625	1.45652937916882e-10\\
17.780400390625	1.16269651593632e-10\\
17.80039453125	1.72731464886008e-11\\
17.820388671875	6.97594836710307e-12\\
17.8403828125	9.4638938855307e-11\\
17.860376953125	1.90298086673551e-12\\
17.88037109375	-4.5140911567918e-11\\
17.900365234375	-1.39766633819336e-11\\
17.920359375	-4.37012547706957e-11\\
17.940353515625	-1.31094169875142e-10\\
17.96034765625	1.44712413649167e-11\\
17.980341796875	-1.61626521076008e-10\\
18.0003359375	-8.84281781949395e-11\\
18.020330078125	-1.02215734137975e-10\\
18.04032421875	-2.52453601454931e-10\\
18.060318359375	-2.42649814899251e-10\\
18.0803125	-2.07484431422773e-10\\
18.100306640625	-3.42243299370803e-10\\
18.12030078125	-2.12595405557109e-10\\
18.140294921875	-2.83950598428059e-10\\
18.1602890625	-1.17488324681489e-10\\
18.180283203125	-2.77522281913266e-10\\
18.20027734375	-1.01507282859411e-10\\
18.220271484375	-3.32471225458192e-10\\
18.240265625	-3.19314399476243e-10\\
18.260259765625	-4.28089097157427e-10\\
18.28025390625	-5.43526338079493e-10\\
18.300248046875	-5.48315161685996e-10\\
18.3202421875	-6.80525700168362e-10\\
18.340236328125	-6.43019812974448e-10\\
18.36023046875	-6.86806842440934e-10\\
18.380224609375	-7.39114786945627e-10\\
18.40021875	-6.77517463568384e-10\\
18.420212890625	-6.70394585896595e-10\\
18.44020703125	-7.5721806196238e-10\\
18.460201171875	-6.53742108632262e-10\\
18.4801953125	-7.68629629912845e-10\\
18.500189453125	-8.72309362236773e-10\\
18.52018359375	-8.54021163563619e-10\\
18.540177734375	-9.41731592591504e-10\\
18.560171875	-9.23108917251682e-10\\
18.580166015625	-8.52724646687819e-10\\
18.60016015625	-8.97088594881123e-10\\
18.620154296875	-7.9154671148575e-10\\
18.6401484375	-7.45118229757546e-10\\
18.660142578125	-8.84337729587638e-10\\
18.68013671875	-8.04381734440829e-10\\
18.700130859375	-9.33856688002189e-10\\
18.720125	-1.06536191243512e-09\\
18.740119140625	-1.15946325151634e-09\\
18.76011328125	-1.16043134254582e-09\\
18.780107421875	-1.22703430803295e-09\\
18.8001015625	-1.06485761039003e-09\\
18.820095703125	-1.0680907178357e-09\\
18.84008984375	-1.02998841567966e-09\\
18.860083984375	-1.13587988249132e-09\\
18.880078125	-9.75811954271847e-10\\
18.900072265625	-9.27704152054956e-10\\
18.92006640625	-8.28277656310839e-10\\
18.940060546875	-8.72029337199825e-10\\
18.9600546875	-8.83163368406011e-10\\
18.980048828125	-9.41124748353772e-10\\
19.00004296875	-1.08870763761299e-09\\
19.020037109375	-1.12105851172731e-09\\
19.04003125	-1.28059502177536e-09\\
19.060025390625	-1.32286484315129e-09\\
19.08001953125	-1.2645390876408e-09\\
19.100013671875	-1.32202077389767e-09\\
19.1200078125	-1.2560491468056e-09\\
19.140001953125	-1.30179809027985e-09\\
19.15999609375	-1.13460242672224e-09\\
19.179990234375	-1.04162608931307e-09\\
19.199984375	-9.88232900929917e-10\\
19.219978515625	-1.06705767243956e-09\\
19.23997265625	-1.05063136342827e-09\\
19.259966796875	-1.19150290998246e-09\\
19.2799609375	-1.22862864861699e-09\\
19.299955078125	-1.36361517325836e-09\\
19.31994921875	-1.31749095685464e-09\\
19.339943359375	-1.41973539978503e-09\\
19.3599375	-1.39386268177576e-09\\
19.379931640625	-1.26958197582597e-09\\
19.39992578125	-1.21626729604101e-09\\
19.419919921875	-1.07226050275907e-09\\
19.4399140625	-1.12962205202145e-09\\
19.459908203125	-9.90385778668079e-10\\
19.47990234375	-1.07459958358757e-09\\
19.499896484375	-1.0165262096088e-09\\
19.519890625	-1.12264894952743e-09\\
19.539884765625	-1.0030772504972e-09\\
19.55987890625	-1.12408617007506e-09\\
19.579873046875	-1.09893323967694e-09\\
19.5998671875	-1.08659233352714e-09\\
19.619861328125	-1.08231832744086e-09\\
19.63985546875	-1.05283250739424e-09\\
19.659849609375	-1.12692710009612e-09\\
19.67984375	-1.25653879121559e-09\\
19.699837890625	-1.21892388985378e-09\\
19.71983203125	-1.20675182855313e-09\\
19.739826171875	-1.20940250600216e-09\\
19.7598203125	-1.17375846733822e-09\\
19.779814453125	-1.10478764596623e-09\\
19.79980859375	-1.15862013629876e-09\\
19.819802734375	-9.99209742436806e-10\\
19.839796875	-1.05606933024475e-09\\
19.859791015625	-1.01581784282351e-09\\
19.87978515625	-9.67575433264284e-10\\
19.899779296875	-1.06558686730674e-09\\
19.9197734375	-9.94359556062614e-10\\
19.939767578125	-8.58376824458686e-10\\
19.95976171875	-8.9054871560617e-10\\
19.979755859375	-6.92821686674822e-10\\
19.99975	-6.39306651646957e-10\\
20.019744140625	-4.83384802690938e-10\\
20.03973828125	-4.70484588378852e-10\\
20.059732421875	-3.427840507848e-10\\
20.0797265625	-4.05410923052593e-10\\
20.099720703125	-3.71383554941807e-10\\
20.11971484375	-4.85649309026861e-10\\
20.139708984375	-3.47226435293208e-10\\
20.159703125	-5.19409420807652e-10\\
20.179697265625	-5.12121512891246e-10\\
20.19969140625	-5.01783465127975e-10\\
20.219685546875	-5.40569963073591e-10\\
20.2396796875	-3.95478522036682e-10\\
20.259673828125	-3.48871150759126e-10\\
20.27966796875	-3.16331750172953e-10\\
20.299662109375	-3.8948849412642e-10\\
20.31965625	-3.36563362601415e-10\\
20.339650390625	-4.01626096686647e-10\\
20.35964453125	-3.38278248222122e-10\\
20.379638671875	-5.10529800961264e-10\\
20.3996328125	-5.06896885049811e-10\\
20.419626953125	-4.68209465653363e-10\\
20.43962109375	-5.42271877925134e-10\\
20.459615234375	-5.87880202759731e-10\\
20.479609375	-3.7180122010385e-10\\
20.499603515625	-5.13626573003999e-10\\
20.51959765625	-3.99913065302272e-10\\
20.539591796875	-3.20265045069752e-10\\
20.5595859375	-3.39169165615705e-10\\
20.579580078125	-3.22841048519104e-10\\
20.59957421875	-4.13031376332507e-10\\
20.619568359375	-4.10308193844153e-10\\
20.6395625	-3.33912972531646e-10\\
20.659556640625	-5.63667610119076e-10\\
20.67955078125	-3.61707861135168e-10\\
20.699544921875	-5.29729293432269e-10\\
20.7195390625	-3.8766215721196e-10\\
20.739533203125	-4.74653413794566e-10\\
20.75952734375	-3.61012008550797e-10\\
20.779521484375	-3.04440537350445e-10\\
20.799515625	-3.5863214066348e-10\\
20.819509765625	-3.00306202743825e-10\\
20.83950390625	-2.24588627467198e-10\\
20.859498046875	-1.39109756308239e-10\\
20.8794921875	-1.96618016498989e-10\\
20.899486328125	-1.49106030408237e-10\\
20.91948046875	-2.57374587701317e-10\\
20.939474609375	-2.20356874505176e-10\\
20.95946875	-2.81064030051827e-10\\
20.979462890625	-2.33331883250619e-10\\
20.99945703125	-2.54145202561457e-10\\
21.019451171875	-3.7902737597315e-10\\
21.0394453125	-2.76254410771825e-10\\
21.059439453125	-3.03589893097223e-10\\
21.07943359375	-2.45657655040132e-10\\
21.099427734375	-2.98315450170916e-10\\
21.119421875	-2.29468711092843e-10\\
21.139416015625	-3.01283812944253e-10\\
21.15941015625	-2.702271114768e-10\\
21.179404296875	-3.1100289601944e-10\\
21.1993984375	-2.53903476214352e-10\\
21.219392578125	-2.47292671227627e-10\\
21.23938671875	-2.3953377719941e-10\\
21.259380859375	-1.81938234647571e-10\\
21.279375	-1.42665972274005e-10\\
21.299369140625	-1.69832320608752e-10\\
21.31936328125	-2.15277598276919e-10\\
21.339357421875	-4.20974872318921e-10\\
21.3593515625	-3.8893493283462e-10\\
21.379345703125	-5.20056387793931e-10\\
21.39933984375	-3.87191629297329e-10\\
21.419333984375	-4.72363895211973e-10\\
21.439328125	-4.30829085168635e-10\\
21.459322265625	-4.5057432724703e-10\\
21.47931640625	-4.56827096361578e-10\\
21.499310546875	-3.67866368693878e-10\\
21.5193046875	-3.29030982627396e-10\\
21.539298828125	-4.00614334836268e-10\\
21.55929296875	-3.80332247666174e-10\\
21.579287109375	-3.10149064517699e-10\\
21.59928125	-3.03197802508813e-10\\
21.619275390625	-3.67578652525859e-10\\
21.63926953125	-2.83390598864781e-10\\
21.659263671875	-4.8536239259656e-10\\
21.6792578125	-4.01547645171007e-10\\
21.699251953125	-4.92245484777495e-10\\
21.71924609375	-4.79445588413992e-10\\
21.739240234375	-5.1273754484584e-10\\
21.759234375	-5.24790673289194e-10\\
21.779228515625	-5.66886827995347e-10\\
21.79922265625	-5.2926143245608e-10\\
21.819216796875	-4.10366427309446e-10\\
21.8392109375	-4.65957285750428e-10\\
21.859205078125	-5.36559468003075e-10\\
21.87919921875	-5.92067567210349e-10\\
21.899193359375	-5.05921015821247e-10\\
21.9191875	-6.82813790310144e-10\\
21.939181640625	-6.20517196709961e-10\\
21.95917578125	-8.87959728204264e-10\\
21.979169921875	-7.02191054418812e-10\\
21.9991640625	-9.46579957405063e-10\\
22.019158203125	-8.92267650377762e-10\\
22.03915234375	-8.68204232459207e-10\\
22.059146484375	-7.66891861157643e-10\\
22.079140625	-8.81024122284085e-10\\
22.099134765625	-7.69712697376731e-10\\
22.11912890625	-7.15938420571902e-10\\
22.139123046875	-7.09806102122158e-10\\
22.1591171875	-6.55029466555829e-10\\
22.179111328125	-4.76878352616201e-10\\
22.19910546875	-5.60109014322595e-10\\
22.219099609375	-5.25207683941932e-10\\
22.23909375	-6.53635854232017e-10\\
22.259087890625	-6.15470378421893e-10\\
22.27908203125	-6.56085604398785e-10\\
22.299076171875	-6.28568845094099e-10\\
22.3190703125	-6.70635447701532e-10\\
22.339064453125	-6.67024913538162e-10\\
22.35905859375	-5.20528354248725e-10\\
22.379052734375	-5.70603026013016e-10\\
22.399046875	-4.39782413660179e-10\\
22.419041015625	-5.79163754522176e-10\\
22.43903515625	-6.35560497509286e-10\\
22.459029296875	-6.18321099582244e-10\\
22.4790234375	-7.99478253231476e-10\\
22.499017578125	-8.10546545962895e-10\\
22.51901171875	-1.02998570607954e-09\\
22.539005859375	-9.27979855473836e-10\\
22.559	-1.00835344325007e-09\\
22.578994140625	-9.61523807491841e-10\\
22.59898828125	-1.04011098435291e-09\\
22.618982421875	-8.12998794565535e-10\\
22.6389765625	-9.41200196545925e-10\\
22.658970703125	-7.42058413774944e-10\\
22.67896484375	-8.30115775465355e-10\\
22.698958984375	-7.6617210211059e-10\\
22.718953125	-6.79147231252073e-10\\
22.738947265625	-7.99536602131699e-10\\
22.75894140625	-7.25899336999127e-10\\
22.778935546875	-9.66717632929952e-10\\
22.7989296875	-8.99967332678167e-10\\
22.818923828125	-9.90527770798557e-10\\
22.83891796875	-8.1350998276748e-10\\
22.858912109375	-8.38435615109137e-10\\
22.87890625	-7.2668905543069e-10\\
22.898900390625	-6.48202453319869e-10\\
22.91889453125	-5.91771865057594e-10\\
22.938888671875	-4.78660030267812e-10\\
22.9588828125	-4.26652165362445e-10\\
22.978876953125	-4.85131922105149e-10\\
22.99887109375	-3.43233626011768e-10\\
23.018865234375	-5.32447972960625e-10\\
23.038859375	-4.35885733195729e-10\\
23.058853515625	-3.40084945898064e-10\\
23.07884765625	-3.53335687042172e-10\\
23.098841796875	-3.01976298949162e-10\\
23.1188359375	-3.91905653056296e-10\\
23.138830078125	-4.63413315947969e-10\\
23.15882421875	-2.90992827721549e-10\\
23.178818359375	-4.24573232068173e-10\\
23.1988125	-1.7324510304125e-10\\
23.218806640625	-2.70458693589613e-10\\
23.23880078125	-3.2261500757286e-11\\
23.258794921875	-1.38908492905677e-10\\
23.2787890625	6.57672460516708e-12\\
23.298783203125	-3.26419915271994e-11\\
23.31877734375	-1.004206699228e-10\\
23.338771484375	-1.24106180317776e-10\\
23.358765625	-2.16823024776273e-10\\
23.378759765625	-2.04350164331911e-10\\
23.39875390625	-2.56345765061125e-10\\
23.418748046875	-2.71212444266365e-10\\
23.4387421875	-2.73129664578179e-10\\
23.458736328125	-2.40488306208577e-10\\
23.47873046875	-9.73203220854479e-11\\
23.498724609375	-8.7093586730446e-11\\
23.51871875	6.87196165142079e-11\\
23.538712890625	1.57980065467431e-11\\
23.55870703125	1.14146457725935e-10\\
23.578701171875	-8.82446771246146e-11\\
23.5986953125	1.03051126590128e-10\\
23.618689453125	-1.24205619122156e-11\\
23.63868359375	-2.45585647573236e-11\\
23.658677734375	-7.92691176067832e-11\\
23.678671875	-1.14213357682479e-10\\
23.698666015625	-6.43047871730264e-11\\
23.71866015625	-8.24614460889469e-11\\
23.738654296875	4.90325570117e-12\\
23.7586484375	1.437411523714e-10\\
23.778642578125	1.19428428984157e-10\\
23.79863671875	1.59896632230572e-10\\
23.818630859375	9.51533041620045e-11\\
23.838625	2.28869796656557e-10\\
23.858619140625	1.5891791240521e-10\\
23.87861328125	7.42055168319244e-11\\
23.898607421875	4.71213411385189e-11\\
23.9186015625	2.21952584040431e-10\\
23.938595703125	1.54320187907336e-10\\
23.95858984375	1.57345133966979e-10\\
23.978583984375	1.35066724540396e-10\\
23.998578125	1.27272396526816e-10\\
24.018572265625	-8.61968221211412e-11\\
24.03856640625	8.8386839955211e-11\\
24.058560546875	3.61671807838852e-11\\
24.0785546875	1.60611210691425e-10\\
24.098548828125	1.30021238288996e-10\\
24.11854296875	2.76433088526917e-10\\
24.138537109375	2.70447827178052e-10\\
24.15853125	3.10281878779669e-10\\
24.178525390625	2.9931449830886e-10\\
24.19851953125	4.09542848751652e-10\\
24.218513671875	4.98726258024532e-10\\
24.2385078125	4.03920265697752e-10\\
24.258501953125	3.91587782219448e-10\\
24.27849609375	4.45654853572814e-10\\
24.298490234375	4.65572762224563e-10\\
24.318484375	3.81953129938019e-10\\
24.338478515625	3.88986069921045e-10\\
24.35847265625	4.27337166524363e-10\\
24.378466796875	3.5172009368443e-10\\
24.3984609375	3.54212134900099e-10\\
24.418455078125	4.41793385378201e-10\\
24.43844921875	4.10200658921009e-10\\
24.458443359375	5.82849831096568e-10\\
24.4784375	4.33592660066658e-10\\
24.498431640625	6.82165009895919e-10\\
24.51842578125	6.81696143776194e-10\\
24.538419921875	7.73588110051653e-10\\
24.5584140625	7.72072996931004e-10\\
24.578408203125	9.07138212890484e-10\\
24.59840234375	7.28950165984756e-10\\
24.618396484375	7.38299634492522e-10\\
24.638390625	6.27678393069293e-10\\
24.658384765625	5.55862468933323e-10\\
24.67837890625	4.4299446017186e-10\\
24.698373046875	5.27533632758032e-10\\
24.7183671875	4.72128364297875e-10\\
24.738361328125	5.99326561328247e-10\\
24.75835546875	6.10133929123164e-10\\
24.778349609375	7.48050791699666e-10\\
24.79834375	6.6907358203228e-10\\
24.818337890625	7.79212640691362e-10\\
24.83833203125	6.93889619657074e-10\\
24.858326171875	6.24751797660929e-10\\
24.8783203125	6.16159478337871e-10\\
24.898314453125	5.20114227035562e-10\\
24.91830859375	5.40073114959958e-10\\
24.938302734375	6.28502434503452e-10\\
24.958296875	5.12552959552369e-10\\
24.978291015625	7.09043522486933e-10\\
24.99828515625	6.48426745142355e-10\\
25.018279296875	8.78359279879358e-10\\
25.0382734375	8.98251329268818e-10\\
25.058267578125	9.16009287491889e-10\\
25.07826171875	9.98046057623773e-10\\
25.098255859375	9.41082432015294e-10\\
25.11825	9.70602727243865e-10\\
25.138244140625	1.01886189325595e-09\\
25.15823828125	8.1751023382574e-10\\
25.178232421875	9.27221493917671e-10\\
25.1982265625	7.54641294642113e-10\\
25.218220703125	8.90827003274191e-10\\
25.23821484375	9.32497352157896e-10\\
25.258208984375	8.68022291688128e-10\\
25.278203125	1.07449367501467e-09\\
25.298197265625	9.50371230476271e-10\\
25.31819140625	9.81750665079898e-10\\
25.338185546875	9.99770888317948e-10\\
25.3581796875	1.00486550204973e-09\\
25.378173828125	9.73549700242785e-10\\
25.39816796875	9.48885725203681e-10\\
25.418162109375	7.63269708941352e-10\\
25.43815625	7.99952496660451e-10\\
25.458150390625	7.70383382426389e-10\\
25.47814453125	7.44628278418915e-10\\
25.498138671875	7.02962840456273e-10\\
25.5181328125	7.0486029824547e-10\\
25.538126953125	5.60699284115683e-10\\
25.55812109375	5.54606691419233e-10\\
25.578115234375	4.79780584813743e-10\\
25.598109375	5.17948350751424e-10\\
25.618103515625	4.99275874735369e-10\\
25.63809765625	5.63848869539662e-10\\
25.658091796875	5.803851254791e-10\\
25.6780859375	7.20835930993131e-10\\
25.698080078125	6.49776696247632e-10\\
25.71807421875	7.96171442936276e-10\\
25.738068359375	6.51655310429186e-10\\
25.7580625	7.00478595638525e-10\\
25.778056640625	5.06075652305496e-10\\
25.79805078125	5.0006816497902e-10\\
25.818044921875	4.10254544352487e-10\\
25.8380390625	2.49185461461042e-10\\
25.858033203125	4.31890835078325e-10\\
25.87802734375	2.62637336747619e-10\\
25.898021484375	4.63116999212872e-10\\
25.918015625	5.23948820647727e-10\\
25.938009765625	6.1053548781379e-10\\
25.95800390625	6.42190065483467e-10\\
25.977998046875	6.21147118866095e-10\\
25.9979921875	7.1677846558566e-10\\
26.017986328125	4.78303253174483e-10\\
26.03798046875	5.14585243669614e-10\\
26.057974609375	3.59786977394961e-10\\
26.07796875	3.22342293125052e-10\\
26.097962890625	2.18980121547961e-10\\
26.11795703125	3.39366433774223e-10\\
26.137951171875	3.69901444042743e-10\\
26.1579453125	3.74521774478389e-10\\
26.177939453125	4.55481456308168e-10\\
26.19793359375	4.4675908539037e-10\\
26.217927734375	4.79464937472314e-10\\
26.237921875	2.97651914174058e-10\\
26.257916015625	2.18240221094846e-10\\
26.27791015625	2.48343614847961e-10\\
26.297904296875	2.05796685206365e-10\\
26.3178984375	1.07051727931286e-10\\
26.337892578125	7.09360372756395e-11\\
26.35788671875	8.07481472924698e-11\\
26.377880859375	8.04764868415074e-11\\
26.397875	9.93560110063952e-11\\
26.417869140625	-3.43652882523258e-13\\
26.43786328125	9.90318003167218e-11\\
26.457857421875	1.75792466448786e-11\\
26.4778515625	9.35947125473231e-11\\
26.497845703125	7.4323104727477e-11\\
26.51783984375	8.27763299967878e-11\\
26.537833984375	2.87226561547878e-11\\
26.557828125	1.51218160920994e-10\\
26.577822265625	6.04227328436206e-11\\
26.59781640625	6.55442124480165e-11\\
26.617810546875	-4.44195076793769e-11\\
26.6378046875	9.48776607286549e-12\\
26.657798828125	-1.45189412603524e-10\\
26.67779296875	-8.85491183871268e-11\\
26.697787109375	-2.85866999100018e-10\\
26.71778125	-3.27108416865812e-10\\
26.737775390625	-2.29492213909291e-10\\
26.75776953125	-3.29147187082741e-10\\
26.777763671875	-2.76278638589733e-10\\
26.7977578125	-2.2537758900907e-10\\
26.817751953125	-2.0796613622935e-10\\
26.83774609375	-2.43204893641226e-10\\
26.857740234375	-2.26224206373298e-10\\
26.877734375	-1.65103195487607e-10\\
26.897728515625	-2.14436454770955e-10\\
26.91772265625	-1.81306564163981e-10\\
26.937716796875	-8.75212396232773e-11\\
26.9577109375	-2.09184872961265e-10\\
26.977705078125	-1.16579404416822e-10\\
26.99769921875	-3.13139243369167e-10\\
27.017693359375	-2.52392370153854e-10\\
27.0376875	-5.10263053653323e-10\\
27.057681640625	-4.76584285817957e-10\\
27.07767578125	-6.02357196141648e-10\\
27.097669921875	-4.97463329621295e-10\\
27.1176640625	-6.99621891244722e-10\\
27.137658203125	-5.07469264099055e-10\\
27.15765234375	-5.38317510464248e-10\\
27.177646484375	-4.23112868687542e-10\\
27.197640625	-5.01765091485712e-10\\
27.217634765625	-4.40305940602069e-10\\
27.23762890625	-5.52836252722973e-10\\
27.257623046875	-5.37514968031525e-10\\
27.2776171875	-6.06644997492401e-10\\
27.297611328125	-6.10677593945071e-10\\
27.31760546875	-7.01760731597566e-10\\
27.337599609375	-6.4473263411707e-10\\
27.35759375	-7.23690480040552e-10\\
27.377587890625	-7.55399688584103e-10\\
27.39758203125	-6.81602509144078e-10\\
27.417576171875	-8.02214067999349e-10\\
27.4375703125	-7.50325474978224e-10\\
27.457564453125	-7.68741080985662e-10\\
27.47755859375	-7.82635617725695e-10\\
27.497552734375	-7.1130546437988e-10\\
27.517546875	-8.17599052647081e-10\\
27.537541015625	-8.21255534925993e-10\\
27.55753515625	-8.38056605511416e-10\\
27.577529296875	-9.73322831796618e-10\\
27.5975234375	-9.55661275179954e-10\\
27.617517578125	-1.06086696106847e-09\\
27.63751171875	-9.89105586792294e-10\\
27.657505859375	-8.88442137398689e-10\\
27.6775	-8.13004059675903e-10\\
27.697494140625	-6.09736630457303e-10\\
27.71748828125	-5.78435989706138e-10\\
27.737482421875	-5.99026453039678e-10\\
27.7574765625	-5.83242253675926e-10\\
27.777470703125	-7.15776736726882e-10\\
27.79746484375	-6.38761809488092e-10\\
27.817458984375	-7.99143559976995e-10\\
27.837453125	-7.43090949317892e-10\\
27.857447265625	-7.32758188744625e-10\\
27.87744140625	-6.98473271525188e-10\\
27.897435546875	-6.75498219725115e-10\\
27.9174296875	-5.34747980165704e-10\\
27.937423828125	-4.40960272909243e-10\\
27.95741796875	-3.55381043615075e-10\\
27.977412109375	-1.95560030421674e-10\\
27.99740625	-1.51968380671047e-10\\
28.017400390625	-1.43832529343067e-10\\
28.03739453125	-1.6042533400462e-10\\
28.057388671875	-1.68659413155911e-10\\
28.0773828125	-1.59291457626186e-10\\
28.097376953125	-1.91338682414468e-10\\
28.11737109375	-1.59395718538944e-10\\
28.137365234375	-2.35364478117103e-10\\
28.157359375	-2.35266117495649e-10\\
28.177353515625	-2.35894891366521e-10\\
28.19734765625	-7.67918452876749e-11\\
28.217341796875	-1.82637771286904e-10\\
28.2373359375	-9.38142918107954e-11\\
28.257330078125	-1.25812485855036e-10\\
28.27732421875	3.14101790423283e-11\\
28.297318359375	-9.70086792847505e-11\\
28.3173125	-5.07853260400566e-11\\
28.337306640625	5.16800285550871e-11\\
28.35730078125	6.07044817960388e-11\\
28.377294921875	1.12396807509249e-10\\
28.3972890625	-3.38544792180753e-12\\
28.417283203125	-9.18035627946383e-12\\
28.43727734375	-5.62484727399921e-11\\
28.457271484375	-2.28422196629479e-10\\
28.477265625	-2.68223858273085e-10\\
28.497259765625	-2.61347736461727e-10\\
28.51725390625	-1.55463778110868e-10\\
28.537248046875	-2.18589955704088e-10\\
28.5572421875	-9.55792213409165e-11\\
28.577236328125	-1.32173705726693e-10\\
28.59723046875	1.43017133710798e-11\\
28.617224609375	-4.58664138733427e-12\\
28.63721875	3.32484004779447e-11\\
28.657212890625	9.36938609063408e-11\\
28.67720703125	-3.68942064613973e-11\\
28.697201171875	-5.48508505029093e-11\\
28.7171953125	-1.89949105313603e-10\\
28.737189453125	-5.84336235844704e-11\\
28.75718359375	-1.82341010779265e-10\\
28.777177734375	-1.21680059701666e-11\\
28.797171875	-1.33393160746482e-11\\
28.817166015625	8.03135681091843e-11\\
28.83716015625	1.2782265307939e-10\\
28.857154296875	9.80624828180544e-11\\
28.8771484375	1.87698075078719e-10\\
28.897142578125	2.21409689037192e-10\\
28.91713671875	1.51584745650801e-10\\
28.937130859375	1.92346300853387e-10\\
28.957125	1.78180914476127e-10\\
28.977119140625	5.96708491114881e-11\\
28.99711328125	1.03717354363745e-10\\
29.017107421875	7.39775688323184e-11\\
29.0371015625	1.32752728436889e-10\\
29.057095703125	3.43974828309223e-11\\
29.07708984375	1.15028812992414e-10\\
29.097083984375	3.60446994954592e-11\\
29.117078125	1.05095965951189e-10\\
29.137072265625	1.4539459767889e-10\\
29.15706640625	1.91351047791226e-10\\
29.177060546875	1.38138219741455e-10\\
29.1970546875	3.03480421591164e-10\\
29.217048828125	2.52412331381548e-10\\
29.23704296875	2.40887503632007e-10\\
29.257037109375	2.18304374378211e-10\\
29.27703125	2.14356212987657e-10\\
29.297025390625	2.40813780379277e-10\\
29.31701953125	2.69359789677706e-10\\
29.337013671875	3.22106541547974e-10\\
29.3570078125	1.4323654844147e-10\\
29.377001953125	1.27198302795945e-10\\
29.39699609375	1.71471652449733e-10\\
29.416990234375	6.65631620769532e-12\\
29.436984375	6.14899814355186e-12\\
29.456978515625	6.88044989540108e-11\\
29.47697265625	-1.03343742789396e-10\\
29.496966796875	5.59554013538669e-11\\
29.5169609375	-1.06046026732464e-10\\
29.536955078125	1.24325983739005e-10\\
29.55694921875	1.14319651241375e-10\\
29.576943359375	2.24061717843147e-10\\
29.5969375	1.77146566949285e-10\\
29.616931640625	3.12157944652156e-10\\
29.63692578125	1.81297564566719e-10\\
29.656919921875	2.11311988241177e-10\\
29.6769140625	8.92198158546902e-11\\
29.696908203125	1.44841687842522e-10\\
29.71690234375	9.72976183658775e-11\\
29.736896484375	1.93416543155414e-10\\
29.756890625	1.36119250958366e-10\\
29.776884765625	2.14497328394295e-10\\
29.79687890625	1.64125302016532e-10\\
29.816873046875	3.67100851556749e-10\\
29.8368671875	2.95934188819521e-10\\
29.856861328125	3.14849378122616e-10\\
29.87685546875	2.11261725638913e-10\\
29.896849609375	3.79779713485862e-10\\
29.91684375	3.28870255139945e-10\\
29.936837890625	3.42486680694641e-10\\
29.95683203125	4.76411226230979e-10\\
29.976826171875	4.52211429878727e-10\\
29.9968203125	5.12904846277457e-10\\
30.016814453125	5.28099164831332e-10\\
30.03680859375	6.00858403503292e-10\\
30.056802734375	6.33459663574539e-10\\
30.076796875	6.08486079833291e-10\\
30.096791015625	6.07906704570451e-10\\
30.11678515625	7.5685780389997e-10\\
30.136779296875	6.4183493798245e-10\\
30.1567734375	6.7397721704144e-10\\
30.176767578125	6.84735865957622e-10\\
30.19676171875	5.76288308147239e-10\\
30.216755859375	6.1240133733735e-10\\
30.23675	5.66158307732893e-10\\
30.256744140625	5.09667070347982e-10\\
30.27673828125	6.43532567393089e-10\\
30.296732421875	5.51349824790717e-10\\
30.3167265625	7.47362052003757e-10\\
30.336720703125	6.59457800547797e-10\\
30.35671484375	7.58333616861881e-10\\
30.376708984375	7.37135491132274e-10\\
30.396703125	7.43386442596139e-10\\
30.416697265625	6.35234360656767e-10\\
30.43669140625	6.03286700676837e-10\\
30.456685546875	4.68661718412103e-10\\
30.4766796875	3.72269760501767e-10\\
30.496673828125	2.50920166582984e-10\\
30.51666796875	1.932739635156e-10\\
30.536662109375	9.39443724325469e-11\\
30.55665625	1.51277711917074e-10\\
30.576650390625	1.52628429650137e-10\\
30.59664453125	2.53476502224167e-10\\
30.616638671875	2.46059296204722e-10\\
30.6366328125	3.5801542536809e-10\\
30.656626953125	4.15002934498809e-10\\
30.67662109375	4.75207659314321e-10\\
30.696615234375	3.20298094455617e-10\\
30.716609375	3.85023816405238e-10\\
30.736603515625	2.65398807231643e-10\\
30.75659765625	2.4133421441385e-10\\
30.776591796875	1.83374340750042e-10\\
30.7965859375	2.27773226281723e-10\\
30.816580078125	1.89111849553952e-10\\
30.83657421875	2.02680880802556e-10\\
30.856568359375	2.28155014984945e-10\\
30.8765625	2.79346644295696e-10\\
30.896556640625	2.86677483687053e-10\\
30.91655078125	1.85707184806979e-10\\
30.936544921875	2.84561282544839e-10\\
30.9565390625	3.39360678272469e-10\\
30.976533203125	3.47965745647999e-10\\
30.99652734375	3.14276370525991e-10\\
31.016521484375	3.24120452606796e-10\\
31.036515625	3.48130778249414e-10\\
31.056509765625	1.45427760332436e-10\\
31.07650390625	2.64236306002265e-10\\
31.096498046875	1.4866322958671e-10\\
31.1164921875	1.29491137718803e-10\\
31.136486328125	1.08056925206719e-10\\
31.15648046875	1.99840447947953e-11\\
31.176474609375	1.03698102930657e-10\\
31.19646875	1.46006975960957e-10\\
31.216462890625	1.91830505913358e-10\\
31.23645703125	1.37355126876442e-10\\
31.256451171875	2.81934415662497e-10\\
31.2764453125	-1.76534394674162e-11\\
31.296439453125	9.77803408464365e-11\\
31.31643359375	2.58751270093038e-11\\
31.336427734375	-2.44847135069112e-11\\
31.356421875	-1.784992652201e-10\\
31.376416015625	-1.8078539499891e-10\\
31.39641015625	-2.55531336520638e-10\\
31.416404296875	-1.75116526823419e-10\\
31.4363984375	-3.23295433882965e-10\\
31.456392578125	-2.73140049955274e-10\\
31.47638671875	-2.81509789273907e-10\\
31.496380859375	-3.0427539616015e-10\\
31.516375	-1.80526097254254e-10\\
31.536369140625	-2.19474611272323e-10\\
31.55636328125	-1.75269637320158e-10\\
31.576357421875	-2.41667775919412e-10\\
31.5963515625	-1.88094569038621e-10\\
31.616345703125	-3.02656049079697e-10\\
31.63633984375	-4.54882341951595e-10\\
31.656333984375	-4.90575298573186e-10\\
31.676328125	-5.24672544919099e-10\\
31.696322265625	-5.93399061686609e-10\\
31.71631640625	-5.48777366043497e-10\\
31.736310546875	-5.85314923295517e-10\\
31.7563046875	-6.22112761899321e-10\\
31.776298828125	-6.77803463701027e-10\\
31.79629296875	-6.7775123174659e-10\\
31.816287109375	-7.01387104414863e-10\\
31.83628125	-7.7253193050741e-10\\
31.856275390625	-6.43371160684022e-10\\
31.87626953125	-6.86159889623791e-10\\
31.896263671875	-7.18445515113151e-10\\
31.9162578125	-5.63120291225131e-10\\
31.936251953125	-6.97652931842589e-10\\
31.95624609375	-6.81509208358252e-10\\
31.976240234375	-6.48641958537532e-10\\
31.996234375	-7.94656646742451e-10\\
32.016228515625	-6.60231475721671e-10\\
32.03622265625	-8.23043673625094e-10\\
32.056216796875	-8.33152470303552e-10\\
32.0762109375	-8.76636751715155e-10\\
32.096205078125	-9.54253272544057e-10\\
32.11619921875	-1.00127839929619e-09\\
32.136193359375	-9.0684469934704e-10\\
32.1561875	-9.95946169245626e-10\\
32.176181640625	-8.68103827818797e-10\\
32.19617578125	-9.06055674592667e-10\\
32.216169921875	-8.13196556670209e-10\\
32.2361640625	-9.10455705668877e-10\\
32.256158203125	-7.7648331716285e-10\\
32.27615234375	-9.21836830474385e-10\\
32.296146484375	-9.18205827308789e-10\\
32.316140625	-1.04072596010167e-09\\
32.336134765625	-9.87462861965641e-10\\
32.35612890625	-1.08169181457559e-09\\
32.376123046875	-1.14766402571951e-09\\
32.3961171875	-1.20246980177866e-09\\
32.416111328125	-1.19967354978607e-09\\
32.43610546875	-1.2151657374735e-09\\
32.456099609375	-1.25872306325666e-09\\
32.47609375	-1.27560210306003e-09\\
32.496087890625	-1.29473807948984e-09\\
32.51608203125	-1.30922263343215e-09\\
32.536076171875	-1.29565741474546e-09\\
32.5560703125	-1.25416051795482e-09\\
32.576064453125	-1.41494668023893e-09\\
32.59605859375	-1.2976422254944e-09\\
32.616052734375	-1.53969484852656e-09\\
32.636046875	-1.46211258981997e-09\\
32.656041015625	-1.55023993284591e-09\\
32.67603515625	-1.58038258957466e-09\\
32.696029296875	-1.42714138825123e-09\\
32.7160234375	-1.57734438815015e-09\\
32.736017578125	-1.45253048465239e-09\\
32.75601171875	-1.45143360819286e-09\\
32.776005859375	-1.52797990046428e-09\\
32.796	-1.48317822240108e-09\\
32.815994140625	-1.49548565128776e-09\\
32.83598828125	-1.35374824942572e-09\\
32.855982421875	-1.43847284058072e-09\\
32.8759765625	-1.30855778446848e-09\\
32.895970703125	-1.44801837349452e-09\\
32.91596484375	-1.34005429585315e-09\\
32.935958984375	-1.32573740037562e-09\\
32.955953125	-1.22782598023572e-09\\
32.975947265625	-1.19624049338619e-09\\
32.99594140625	-1.0483772931143e-09\\
33.015935546875	-1.10668006779337e-09\\
33.0359296875	-9.68791712201792e-10\\
33.055923828125	-9.63989969806638e-10\\
33.07591796875	-1.08773151018227e-09\\
33.095912109375	-1.10433831627779e-09\\
33.11590625	-1.07132570268079e-09\\
33.135900390625	-1.18801891942966e-09\\
33.15589453125	-1.21464169020357e-09\\
33.175888671875	-1.15055904573915e-09\\
33.1958828125	-1.08384063789143e-09\\
33.215876953125	-1.08884070975277e-09\\
33.23587109375	-1.01095615765861e-09\\
33.255865234375	-1.04773505230227e-09\\
33.275859375	-9.85338950950446e-10\\
33.295853515625	-1.02782651752895e-09\\
33.31584765625	-9.97524804176345e-10\\
33.335841796875	-1.03105399413366e-09\\
33.3558359375	-1.01213284077665e-09\\
33.375830078125	-9.07443411072095e-10\\
33.39582421875	-8.25709742332402e-10\\
33.415818359375	-7.72825687566723e-10\\
33.4358125	-7.94036792491226e-10\\
33.455806640625	-8.15012114109209e-10\\
33.47580078125	-7.56454241568101e-10\\
33.495794921875	-8.06913240128769e-10\\
33.5157890625	-8.82819690428883e-10\\
33.535783203125	-9.18325255647335e-10\\
33.55577734375	-8.16025414811107e-10\\
33.575771484375	-8.81285635774488e-10\\
33.595765625	-7.82265913196221e-10\\
33.615759765625	-7.44862680119384e-10\\
33.63575390625	-7.01964435787469e-10\\
33.655748046875	-6.9678995137575e-10\\
33.6757421875	-6.8143790298305e-10\\
33.695736328125	-7.4330765313635e-10\\
33.71573046875	-7.01487094151921e-10\\
33.735724609375	-7.35686318990587e-10\\
33.75571875	-8.04409350918933e-10\\
33.775712890625	-6.51046687017876e-10\\
33.79570703125	-7.68693858181767e-10\\
33.815701171875	-5.91263869707581e-10\\
33.8356953125	-5.80462795688783e-10\\
33.855689453125	-5.3906726340355e-10\\
33.87568359375	-5.16227420618721e-10\\
33.895677734375	-4.59882998285983e-10\\
33.915671875	-4.47984040693247e-10\\
33.935666015625	-3.19041215242997e-10\\
33.95566015625	-2.84736740779882e-10\\
33.975654296875	-3.07276385047731e-10\\
33.9956484375	-3.36015007335061e-10\\
34.015642578125	-4.16671084955365e-10\\
34.03563671875	-4.55062022143912e-10\\
34.055630859375	-4.74437910889308e-10\\
34.075625	-4.72881663489086e-10\\
34.095619140625	-5.25174113191407e-10\\
34.11561328125	-4.22834049936188e-10\\
34.135607421875	-4.38705718411642e-10\\
34.1556015625	-3.24522062079796e-10\\
34.175595703125	-3.23486361214767e-10\\
34.19558984375	-2.51801119721769e-10\\
34.215583984375	-2.57760098319818e-10\\
34.235578125	-8.76345015553601e-11\\
34.255572265625	-2.48273028211821e-10\\
34.27556640625	-1.80669621673187e-10\\
34.295560546875	-1.0688165370869e-10\\
34.3155546875	-1.00389029502412e-10\\
34.335548828125	-1.57054899661928e-10\\
34.35554296875	-2.11830387977522e-11\\
34.375537109375	2.52564354297164e-11\\
34.39553125	3.99747827987882e-11\\
34.415525390625	-4.72470747888902e-11\\
34.43551953125	2.46327682447726e-11\\
34.455513671875	-4.6637594579076e-11\\
34.4755078125	-1.05888916834554e-10\\
34.495501953125	3.93741845264751e-11\\
34.51549609375	1.35074058062633e-11\\
34.535490234375	2.5305482504571e-10\\
34.555484375	2.06160506273724e-10\\
34.575478515625	4.62859306592873e-10\\
34.59547265625	5.04515583080736e-10\\
34.615466796875	4.98975619021007e-10\\
34.6354609375	4.75531962171913e-10\\
34.655455078125	5.26170518024587e-10\\
34.67544921875	3.33599639490636e-10\\
34.695443359375	3.55010569583206e-10\\
34.7154375	2.96372650198382e-10\\
34.735431640625	4.22161866172782e-10\\
34.75542578125	3.73228365910319e-10\\
34.775419921875	5.23119908481178e-10\\
34.7954140625	6.01440646826627e-10\\
34.815408203125	6.99395981217322e-10\\
34.83540234375	7.50541861816896e-10\\
34.855396484375	8.70549924672156e-10\\
34.875390625	8.61570587590715e-10\\
34.895384765625	8.99237224716615e-10\\
34.91537890625	8.70485775054131e-10\\
34.935373046875	8.55305575260817e-10\\
34.9553671875	8.41701567889755e-10\\
34.975361328125	8.22410583397828e-10\\
34.99535546875	8.96563482577408e-10\\
35.015349609375	9.79862490005962e-10\\
35.03534375	1.03924506337505e-09\\
35.055337890625	1.03455636043966e-09\\
35.07533203125	1.18821371888254e-09\\
35.095326171875	1.11411369452132e-09\\
35.1153203125	1.21369826853015e-09\\
35.135314453125	1.16618339704892e-09\\
35.15530859375	1.21594076711517e-09\\
35.175302734375	1.1950081207614e-09\\
35.195296875	1.2000874064842e-09\\
35.215291015625	1.3256646217334e-09\\
35.23528515625	1.25267334671656e-09\\
35.255279296875	1.28386774506772e-09\\
35.2752734375	1.37501988658335e-09\\
35.295267578125	1.37608155152057e-09\\
35.31526171875	1.40262312711026e-09\\
35.335255859375	1.38351152966665e-09\\
35.35525	1.44270979771153e-09\\
35.375244140625	1.34753083618956e-09\\
35.39523828125	1.34006711018949e-09\\
35.415232421875	1.2682007159916e-09\\
35.4352265625	1.32385967229302e-09\\
35.455220703125	1.21627620993238e-09\\
35.47521484375	1.27212859299541e-09\\
35.495208984375	1.09457306124448e-09\\
35.515203125	1.26032950691858e-09\\
35.535197265625	1.14713686915846e-09\\
35.55519140625	1.1844375242203e-09\\
35.575185546875	1.22719030006041e-09\\
35.5951796875	1.27165364372544e-09\\
35.615173828125	1.22853597139437e-09\\
35.63516796875	1.31988841178582e-09\\
35.655162109375	1.28111397865397e-09\\
35.67515625	1.2848525239469e-09\\
35.695150390625	1.25107162267814e-09\\
35.71514453125	1.35727402488706e-09\\
35.735138671875	1.28245085801454e-09\\
35.7551328125	1.3725591484502e-09\\
35.775126953125	1.37707202011807e-09\\
35.79512109375	1.43753317401242e-09\\
35.815115234375	1.36230050935543e-09\\
35.835109375	1.48515234542669e-09\\
35.855103515625	1.34300807375037e-09\\
35.87509765625	1.25796903350719e-09\\
35.895091796875	1.30032994435909e-09\\
35.9150859375	1.24672644052049e-09\\
35.935080078125	1.30895286420906e-09\\
35.95507421875	1.23526326892028e-09\\
35.975068359375	1.29467085795275e-09\\
35.9950625	1.40030383944594e-09\\
36.015056640625	1.42813790463004e-09\\
36.03505078125	1.4689243697953e-09\\
36.055044921875	1.43774066488124e-09\\
36.0750390625	1.47996616618273e-09\\
36.095033203125	1.37065184173781e-09\\
36.11502734375	1.31848372157918e-09\\
36.135021484375	1.2675969522058e-09\\
36.155015625	1.25665160222102e-09\\
36.175009765625	1.19656043095116e-09\\
36.19500390625	1.26625167932771e-09\\
36.214998046875	1.25638559617474e-09\\
36.2349921875	1.30698981167812e-09\\
36.254986328125	1.34652790959788e-09\\
36.27498046875	1.29248102618537e-09\\
36.294974609375	1.39395737579335e-09\\
36.31496875	1.2559282567896e-09\\
36.334962890625	1.28361124347871e-09\\
36.35495703125	1.19089946628336e-09\\
36.374951171875	1.18796927634973e-09\\
36.3949453125	1.06174574036629e-09\\
36.414939453125	1.01958712127715e-09\\
36.43493359375	8.77189225569824e-10\\
36.454927734375	8.06225394545389e-10\\
36.474921875	7.01591125974952e-10\\
36.494916015625	6.61890686264909e-10\\
36.51491015625	8.39514114858511e-10\\
36.534904296875	7.94746727127439e-10\\
36.5548984375	8.9493022197951e-10\\
36.574892578125	9.19448131795338e-10\\
36.59488671875	1.04745356271712e-09\\
36.614880859375	9.4338707017105e-10\\
36.634875	9.21414785406438e-10\\
36.654869140625	7.47583051318383e-10\\
36.67486328125	6.74942622182456e-10\\
36.694857421875	5.55786288846661e-10\\
36.7148515625	6.3380645813657e-10\\
36.734845703125	5.10990186000358e-10\\
36.75483984375	5.89894031199811e-10\\
36.774833984375	5.34863526308534e-10\\
36.794828125	4.99125752136567e-10\\
36.814822265625	5.50621228375683e-10\\
36.83481640625	4.71293229137519e-10\\
36.854810546875	4.42028085412743e-10\\
36.8748046875	3.93684565822416e-10\\
36.894798828125	3.83394725347744e-10\\
36.91479296875	5.0867099684663e-10\\
36.934787109375	4.89972723304907e-10\\
36.95478125	5.14191320306781e-10\\
36.974775390625	5.33082751195668e-10\\
36.99476953125	4.72651479986296e-10\\
37.014763671875	3.7693131766247e-10\\
37.0347578125	1.54621868319265e-10\\
37.054751953125	2.71541389678658e-10\\
37.07474609375	8.78332026889973e-11\\
37.094740234375	1.48907379940865e-10\\
37.114734375	3.16881010571933e-11\\
37.134728515625	1.09919630039666e-10\\
37.15472265625	5.51664264714024e-11\\
37.174716796875	1.21552256396454e-10\\
37.1947109375	1.55898033994015e-10\\
37.214705078125	2.46735333945155e-10\\
37.23469921875	2.14659082339881e-10\\
37.254693359375	2.77342670579367e-10\\
37.2746875	1.80340803904316e-10\\
37.294681640625	2.21536150994604e-10\\
37.31467578125	1.07348271476807e-10\\
37.334669921875	1.20984416571347e-10\\
37.3546640625	3.41000444133905e-11\\
37.374658203125	-3.93082201583128e-11\\
37.39465234375	-3.59326848715511e-11\\
37.414646484375	3.38451794075811e-11\\
37.434640625	2.19391729422397e-11\\
37.454634765625	1.58295722045201e-11\\
37.47462890625	-1.00155700630814e-13\\
37.494623046875	-1.08519656646119e-12\\
37.5146171875	-3.49947914146004e-11\\
37.534611328125	-6.71027678618253e-11\\
37.55460546875	-4.99958297167133e-11\\
37.574599609375	-2.21940430429563e-10\\
37.59459375	-4.68634631466133e-11\\
37.614587890625	-1.58877476432236e-10\\
37.63458203125	-1.64949763088231e-10\\
37.654576171875	-1.49322385497588e-10\\
37.6745703125	-1.80526099581921e-10\\
37.694564453125	-2.94481608576409e-10\\
37.71455859375	-3.85864398352934e-10\\
37.734552734375	-3.54012852254023e-10\\
37.754546875	-3.93837243328435e-10\\
37.774541015625	-4.19948537594456e-10\\
37.79453515625	-3.97481163458345e-10\\
37.814529296875	-4.04886688545674e-10\\
37.8345234375	-3.14525011897033e-10\\
37.854517578125	-3.48682204284248e-10\\
};
\addplot [color=mycolor1,solid]
  table[row sep=crcr]{%
37.854517578125	-3.48682204284248e-10\\
37.87451171875	-1.52228302109132e-10\\
37.894505859375	-2.66432995661225e-10\\
37.9145	-2.63250399327523e-10\\
37.934494140625	-1.85672648257793e-10\\
37.95448828125	-1.3643845619084e-10\\
37.974482421875	-1.52435184712923e-10\\
37.9944765625	-6.95695547283366e-11\\
38.014470703125	-1.1803189671861e-10\\
38.03446484375	-5.68376669130866e-13\\
38.054458984375	-7.83982475017505e-11\\
38.074453125	-7.49005074838422e-11\\
38.094447265625	-2.36487308329265e-11\\
38.11444140625	-1.76311605696616e-10\\
38.134435546875	-1.21312763031284e-10\\
38.1544296875	-6.03332610003422e-11\\
38.174423828125	-1.75510417619406e-10\\
38.19441796875	-1.37466696852242e-10\\
38.214412109375	-1.47088122647725e-10\\
38.23440625	-1.46113663190398e-10\\
38.254400390625	-2.15010539769531e-10\\
38.27439453125	-1.17172432331863e-10\\
38.294388671875	-2.03166564690931e-10\\
38.3143828125	-1.44655021163029e-10\\
38.334376953125	-2.76605160529541e-10\\
38.35437109375	-7.52893262836563e-11\\
38.374365234375	-5.52381401257428e-12\\
38.394359375	9.59529236992806e-11\\
38.414353515625	9.83159240150961e-11\\
38.43434765625	1.67374808556225e-10\\
38.454341796875	1.2004194789963e-10\\
38.4743359375	7.6974420150146e-11\\
38.494330078125	8.32807089032188e-13\\
38.51432421875	-8.97535862628537e-11\\
38.534318359375	-2.05010213846639e-10\\
38.5543125	-2.6516904883061e-10\\
38.574306640625	-2.11704578413167e-10\\
38.59430078125	-1.89152590864453e-10\\
38.614294921875	-1.37915132801793e-10\\
38.6342890625	-7.49863946471219e-11\\
38.654283203125	-1.63932688594337e-12\\
38.67427734375	1.217361225685e-11\\
38.694271484375	1.45066350488148e-11\\
38.714265625	-6.94019867402271e-11\\
38.734259765625	-3.04344019664899e-11\\
38.75425390625	-1.61291071122131e-10\\
38.774248046875	-1.91218919118155e-10\\
38.7942421875	-1.43009908407417e-10\\
38.814236328125	-4.28005948497378e-11\\
38.83423046875	-9.95161207376076e-11\\
38.854224609375	-5.73915707182656e-11\\
38.87421875	-3.03195653327358e-11\\
38.894212890625	1.09538375228639e-10\\
38.91420703125	2.07325311933031e-10\\
38.934201171875	2.70394019999711e-10\\
38.9541953125	3.52758850091536e-10\\
38.974189453125	3.97254292457195e-10\\
38.99418359375	4.67930983545809e-10\\
39.014177734375	2.59435261796583e-10\\
39.034171875	2.57565023184724e-10\\
39.054166015625	5.32121341047746e-11\\
39.07416015625	5.41461759315112e-11\\
39.094154296875	-1.97769943013338e-10\\
39.1141484375	-1.39825804373422e-11\\
39.134142578125	2.06756382271976e-11\\
39.15413671875	1.79384373677745e-10\\
39.174130859375	2.18327434121021e-10\\
39.194125	3.22376769771793e-10\\
39.214119140625	2.88419529597463e-10\\
39.23411328125	2.85354956268737e-10\\
39.254107421875	2.26491227933473e-10\\
39.2741015625	2.35725588972615e-10\\
39.294095703125	2.04400923828435e-10\\
39.31408984375	1.67960669734905e-10\\
39.334083984375	2.40074542901772e-10\\
39.354078125	2.35872056089041e-10\\
39.374072265625	3.22176049805513e-10\\
39.39406640625	2.87144834146957e-10\\
39.414060546875	2.83635815565199e-10\\
39.4340546875	2.71341183151412e-10\\
39.454048828125	2.35298851150092e-10\\
39.47404296875	2.44599702007127e-10\\
39.494037109375	2.41815577926603e-10\\
39.51403125	2.32345143239948e-10\\
39.534025390625	3.52692915805401e-10\\
39.55401953125	1.91842625865413e-10\\
39.574013671875	4.35443782450453e-10\\
39.5940078125	2.89548291681391e-10\\
39.614001953125	4.96306488854303e-10\\
39.63399609375	3.66843541170368e-10\\
39.653990234375	4.22012119677438e-10\\
39.673984375	3.09302049587237e-10\\
39.693978515625	3.64342915107336e-10\\
39.71397265625	1.35918748780131e-10\\
39.733966796875	1.51528899646148e-10\\
39.7539609375	6.69311808951309e-11\\
39.773955078125	1.24345859390562e-10\\
39.79394921875	6.51487305466155e-11\\
39.813943359375	2.07348083687481e-10\\
39.8339375	1.96326341086916e-10\\
39.853931640625	2.59928187691905e-10\\
39.87392578125	2.86428681095556e-10\\
39.893919921875	2.23174049876152e-10\\
39.9139140625	2.3607395458152e-10\\
39.933908203125	1.81125181051536e-10\\
39.95390234375	2.02386725162235e-10\\
39.973896484375	1.57069652653376e-10\\
39.993890625	1.70377580013409e-10\\
40.013884765625	2.09893429770395e-10\\
40.03387890625	1.69957365266015e-10\\
40.053873046875	3.28370502662351e-10\\
40.0738671875	2.46877288059832e-10\\
40.093861328125	2.85539483178211e-10\\
40.11385546875	3.10573501575859e-10\\
40.133849609375	2.69658781731143e-10\\
40.15384375	2.99944472527759e-10\\
40.173837890625	2.98027520041182e-10\\
40.19383203125	3.76075373312989e-10\\
40.213826171875	4.59914086497198e-10\\
40.2338203125	4.67867054093542e-10\\
40.253814453125	5.09307658777699e-10\\
40.27380859375	5.21993578247966e-10\\
40.293802734375	5.09415930791181e-10\\
40.313796875	4.96679557351266e-10\\
40.333791015625	3.81795252214807e-10\\
40.35378515625	4.74917139519638e-10\\
40.373779296875	3.65433657985639e-10\\
40.3937734375	4.66993676069594e-10\\
40.413767578125	4.14508721339844e-10\\
40.43376171875	4.19606688089662e-10\\
40.453755859375	4.02800227043819e-10\\
40.47375	3.87430579679791e-10\\
40.493744140625	3.49086552571869e-10\\
40.51373828125	2.9382443515059e-10\\
40.533732421875	2.10587145062205e-10\\
40.5537265625	1.98809188636532e-10\\
40.573720703125	1.34782700013641e-10\\
40.59371484375	7.99732298234148e-11\\
40.613708984375	1.79789443416773e-10\\
40.633703125	7.25202791786481e-11\\
40.653697265625	1.71665778732291e-10\\
40.67369140625	1.80022982980499e-10\\
40.693685546875	2.20777410552486e-10\\
40.7136796875	1.71926671540119e-10\\
40.733673828125	1.99800954811949e-10\\
40.75366796875	2.0894403037598e-10\\
40.773662109375	1.47542487593801e-10\\
40.79365625	2.00716514087049e-10\\
40.813650390625	1.1705703350894e-10\\
40.83364453125	1.66595018163291e-10\\
40.853638671875	6.64016373553358e-11\\
40.8736328125	1.02501970503982e-10\\
40.893626953125	2.46917168415991e-11\\
40.91362109375	-5.93189143596085e-11\\
40.933615234375	-7.97752149845483e-11\\
40.953609375	-5.61904806066818e-11\\
40.973603515625	-2.0445461218511e-12\\
40.99359765625	1.5848404602103e-11\\
41.013591796875	1.01238611332288e-10\\
41.0335859375	1.75084447931828e-10\\
41.053580078125	1.43141396833462e-10\\
41.07357421875	1.67955368404742e-10\\
41.093568359375	1.96862659678205e-10\\
41.1135625	1.25280898936069e-10\\
41.133556640625	1.14273589669803e-11\\
41.15355078125	7.06165419484456e-12\\
41.173544921875	-3.84889896469618e-11\\
41.1935390625	-2.89204032292564e-11\\
41.213533203125	2.57369685447639e-11\\
41.23352734375	-2.73269096525757e-11\\
41.253521484375	9.91023916446599e-11\\
41.273515625	5.70340967639169e-11\\
41.293509765625	1.36479996784628e-11\\
41.31350390625	4.3626209782693e-12\\
41.333498046875	3.79567197774365e-11\\
41.3534921875	6.43827394351837e-12\\
41.373486328125	-2.19242444695874e-11\\
41.39348046875	-2.17164661037081e-11\\
41.413474609375	-1.66413005336441e-10\\
41.43346875	-1.46520322755524e-10\\
41.453462890625	-2.0821813770161e-10\\
41.47345703125	-2.19824506777776e-10\\
41.493451171875	-3.48708091987202e-10\\
41.5134453125	-1.88717096145184e-10\\
41.533439453125	-2.39284240699213e-10\\
41.55343359375	-1.72726980259325e-10\\
41.573427734375	-1.43934560643001e-10\\
41.593421875	-2.78279129498272e-11\\
41.613416015625	-5.99138147510461e-11\\
41.63341015625	-5.4301456799118e-11\\
41.653404296875	-7.13937953205144e-11\\
41.6733984375	-1.04000987596129e-10\\
41.693392578125	-2.31837332860606e-10\\
41.71338671875	-2.02926129592719e-10\\
41.733380859375	-2.49499794162717e-10\\
41.753375	-2.33716680887462e-10\\
41.773369140625	-3.20293137232371e-10\\
41.79336328125	-2.74517334475945e-10\\
41.813357421875	-2.84253559617128e-10\\
41.8333515625	-3.10356988695799e-10\\
41.853345703125	-3.33777935908801e-10\\
41.87333984375	-3.54202167563467e-10\\
41.893333984375	-3.17368673698298e-10\\
41.913328125	-3.56619831437921e-10\\
41.933322265625	-3.03385566627605e-10\\
41.95331640625	-3.1258088004247e-10\\
41.973310546875	-3.6768994798193e-10\\
41.9933046875	-3.83003774980813e-10\\
42.013298828125	-3.11163607104599e-10\\
42.03329296875	-3.52265558311371e-10\\
42.053287109375	-3.05530124574991e-10\\
42.07328125	-4.76965767719823e-10\\
42.093275390625	-3.61787940730868e-10\\
42.11326953125	-5.40341942286891e-10\\
42.133263671875	-4.39515961808364e-10\\
42.1532578125	-4.81176357237277e-10\\
42.173251953125	-5.2555287056404e-10\\
42.19324609375	-5.10939235890979e-10\\
42.213240234375	-3.95912605030658e-10\\
42.233234375	-3.98868374852091e-10\\
42.253228515625	-2.77737378546977e-10\\
42.27322265625	-2.30399768067463e-10\\
42.293216796875	-2.36586529108973e-10\\
42.3132109375	-3.06584999867553e-10\\
42.333205078125	-2.41655736908132e-10\\
42.35319921875	-2.83829384945311e-10\\
42.373193359375	-3.58426928774281e-10\\
42.3931875	-3.09351469682723e-10\\
42.413181640625	-3.5351107516001e-10\\
42.43317578125	-2.63077824323532e-10\\
42.453169921875	-2.79535057411275e-10\\
42.4731640625	-2.83582075774813e-10\\
42.493158203125	-3.60465598455588e-10\\
42.51315234375	-3.21867918781681e-10\\
42.533146484375	-3.72725856899899e-10\\
42.553140625	-3.63877416468441e-10\\
42.573134765625	-3.37657991862322e-10\\
42.59312890625	-3.99596486522768e-10\\
42.613123046875	-4.61473013540979e-10\\
42.6331171875	-4.03967323658651e-10\\
42.653111328125	-4.47065252230526e-10\\
42.67310546875	-4.40006939085256e-10\\
42.693099609375	-4.18497580392724e-10\\
42.71309375	-4.81262851985064e-10\\
42.733087890625	-4.94965166349014e-10\\
42.75308203125	-4.92175474399382e-10\\
42.773076171875	-5.77604420436395e-10\\
42.7930703125	-5.60764427886112e-10\\
42.813064453125	-5.80074453318437e-10\\
42.83305859375	-5.74575359065755e-10\\
42.853052734375	-5.87069176449354e-10\\
42.873046875	-4.72085435732246e-10\\
42.893041015625	-5.51864914113963e-10\\
42.91303515625	-5.25070517655892e-10\\
42.933029296875	-6.03883255468131e-10\\
42.9530234375	-6.08934071540719e-10\\
42.973017578125	-5.48482596808179e-10\\
42.99301171875	-4.53405556520439e-10\\
43.013005859375	-3.86894326659168e-10\\
43.033	-3.30701970523128e-10\\
43.052994140625	-2.59854889690764e-10\\
43.07298828125	-2.44723708889956e-10\\
43.092982421875	-2.53823644613521e-10\\
43.1129765625	-3.324839787526e-10\\
43.132970703125	-2.76467651722659e-10\\
43.15296484375	-3.75532949457023e-10\\
43.172958984375	-4.38360269206743e-10\\
43.192953125	-3.7579538665403e-10\\
43.212947265625	-3.89725788930294e-10\\
43.23294140625	-4.09291121451557e-10\\
43.252935546875	-3.06838817896644e-10\\
43.2729296875	-2.58937200965645e-10\\
43.292923828125	-3.55876653907471e-10\\
43.31291796875	-2.40437913282101e-10\\
43.332912109375	-2.1423466151683e-10\\
43.35290625	-2.19973358576112e-10\\
43.372900390625	-1.65774974625695e-10\\
43.39289453125	-5.56914711850641e-11\\
43.412888671875	-9.6617911423194e-11\\
43.4328828125	-1.58362986326299e-11\\
43.452876953125	-6.614023918357e-11\\
43.47287109375	-7.18618899699147e-11\\
43.492865234375	7.84166736542153e-13\\
43.512859375	-7.25201041044493e-11\\
43.532853515625	-1.53243588347667e-10\\
43.55284765625	-1.16434762048652e-10\\
43.572841796875	-2.20723661898546e-10\\
43.5928359375	-1.46084711954286e-10\\
43.612830078125	-1.86682472458416e-10\\
43.63282421875	-1.03638025672949e-10\\
43.652818359375	-1.3278689820118e-10\\
43.6728125	-1.60513256587215e-11\\
43.692806640625	-1.04527287997595e-10\\
43.71280078125	-7.3628676547644e-12\\
43.732794921875	2.17240400460627e-11\\
43.7527890625	-1.40743008890469e-11\\
43.772783203125	4.70138654236e-11\\
43.79277734375	2.95431538389755e-11\\
43.812771484375	3.88850895299968e-11\\
43.832765625	3.01888274235239e-12\\
43.852759765625	8.92206742146631e-11\\
43.87275390625	-7.0745651929423e-11\\
43.892748046875	4.02936323170382e-11\\
43.9127421875	7.2762122817248e-11\\
43.932736328125	6.82527708228258e-11\\
43.95273046875	1.35098159651337e-10\\
43.972724609375	2.43233142140759e-10\\
43.99271875	2.75740105275989e-10\\
44.012712890625	2.48499017189839e-10\\
44.03270703125	2.56257175548467e-10\\
44.052701171875	2.37683951983172e-10\\
44.0726953125	2.08327129976959e-10\\
44.092689453125	4.49593153064855e-11\\
44.11268359375	8.36467115730823e-11\\
44.132677734375	-2.97491347344792e-11\\
44.152671875	1.04734800238855e-11\\
44.172666015625	2.17826677075048e-11\\
44.19266015625	5.56829668262493e-12\\
44.212654296875	9.555262860783e-11\\
44.2326484375	1.58421694758689e-10\\
44.252642578125	1.38197802541832e-10\\
44.27263671875	2.04811525939641e-10\\
44.292630859375	1.36299481737266e-10\\
44.312625	1.87584787619575e-10\\
44.332619140625	1.60238381158504e-10\\
44.35261328125	2.00912882208202e-10\\
44.372607421875	1.49778102650031e-10\\
44.3926015625	7.88263059408758e-11\\
44.412595703125	5.91659844307327e-11\\
44.43258984375	1.58567661478583e-10\\
44.452583984375	1.01670171281441e-10\\
44.472578125	1.60798196623745e-10\\
44.492572265625	1.18955450234765e-10\\
44.51256640625	2.25715818523818e-10\\
44.532560546875	2.08605902610753e-10\\
44.5525546875	1.89266905422709e-10\\
44.572548828125	1.87968031096963e-10\\
44.59254296875	1.31245833668322e-10\\
44.612537109375	2.3785776594999e-10\\
44.63253125	1.6709281428982e-10\\
44.652525390625	3.13874260722569e-10\\
44.67251953125	2.91989297212206e-10\\
44.692513671875	2.35652949614758e-10\\
44.7125078125	2.088987592346e-10\\
44.732501953125	1.83581004180126e-10\\
44.75249609375	9.02878186479531e-11\\
44.772490234375	6.91563769184239e-11\\
44.792484375	1.19585237914934e-10\\
44.812478515625	1.19471716738671e-10\\
44.83247265625	4.52291994628149e-11\\
44.852466796875	1.44421633026944e-10\\
44.8724609375	1.1957739428749e-10\\
44.892455078125	2.14338927373981e-10\\
44.91244921875	1.99501626150649e-10\\
44.932443359375	1.1148139980523e-10\\
44.9524375	1.23902748585235e-10\\
44.972431640625	1.61469944208392e-10\\
44.99242578125	1.34732028113367e-10\\
45.012419921875	8.68810286279905e-11\\
45.0324140625	2.15215678394476e-10\\
45.052408203125	2.35249981288116e-10\\
45.07240234375	1.88056643959522e-10\\
45.092396484375	3.05523648478939e-10\\
45.112390625	3.10053759151632e-10\\
45.132384765625	3.03503039870434e-10\\
45.15237890625	3.77772054506843e-10\\
45.172373046875	2.69685119616672e-10\\
45.1923671875	3.35688661527044e-10\\
45.212361328125	3.78248222584199e-10\\
45.23235546875	3.49002601511836e-10\\
45.252349609375	3.7585391909582e-10\\
45.27234375	3.95553346912366e-10\\
45.292337890625	3.4236989005996e-10\\
45.31233203125	4.27977625418128e-10\\
45.332326171875	4.30458779394572e-10\\
45.3523203125	4.88076702340266e-10\\
45.372314453125	4.37869295887711e-10\\
45.39230859375	4.63510398271938e-10\\
45.412302734375	4.34908427371248e-10\\
45.432296875	5.2411807803667e-10\\
45.452291015625	4.36675972035666e-10\\
45.47228515625	4.48291507548809e-10\\
45.492279296875	3.51956185333652e-10\\
45.5122734375	2.45705110722655e-10\\
45.532267578125	2.634883759985e-10\\
45.55226171875	2.02419498344616e-10\\
45.572255859375	1.21222856357872e-10\\
45.59225	1.35779883200947e-10\\
45.612244140625	2.27580998225544e-10\\
45.63223828125	2.13152742146759e-10\\
45.652232421875	2.54451301949255e-10\\
45.6722265625	2.96222455623468e-10\\
45.692220703125	2.42326945549468e-10\\
45.71221484375	3.89185824377986e-10\\
45.732208984375	3.9134096736034e-10\\
45.752203125	3.25700244950895e-10\\
45.772197265625	3.40292073479802e-10\\
45.79219140625	3.04617967407146e-10\\
45.812185546875	2.28232610089214e-10\\
45.8321796875	2.97833943360271e-10\\
45.852173828125	2.26607644634777e-10\\
45.87216796875	2.80900132612581e-10\\
45.892162109375	2.06880370717678e-10\\
45.91215625	2.22223324885418e-10\\
45.932150390625	1.86510721886328e-10\\
45.95214453125	1.66593259498134e-10\\
45.972138671875	1.59856516680341e-10\\
45.9921328125	1.59549794174452e-10\\
46.012126953125	1.95586886512844e-10\\
46.03212109375	1.94688968579877e-10\\
46.052115234375	2.82589260094777e-10\\
46.072109375	2.89670441622017e-10\\
46.092103515625	2.26979275365397e-10\\
46.11209765625	3.77569495933243e-10\\
46.132091796875	2.57191709202289e-10\\
46.1520859375	2.75495351998896e-10\\
46.172080078125	1.80164784903452e-10\\
46.19207421875	2.92106195717884e-10\\
46.212068359375	2.31083195573037e-10\\
46.2320625	1.81028164788386e-10\\
46.252056640625	2.03851211665773e-10\\
46.27205078125	1.83445660734569e-10\\
46.292044921875	2.0813490168263e-10\\
46.3120390625	2.04134422178694e-10\\
46.332033203125	2.00011866042975e-10\\
46.35202734375	1.86800748205817e-10\\
46.372021484375	2.49879986703122e-10\\
46.392015625	2.41321133974786e-10\\
46.412009765625	1.44836931161822e-10\\
46.43200390625	1.60945454226011e-10\\
46.451998046875	1.39977591842718e-10\\
46.4719921875	1.81832709074762e-11\\
46.491986328125	8.07128647221482e-11\\
46.51198046875	-5.59275717667582e-11\\
46.531974609375	-1.83132111063923e-11\\
46.55196875	-5.78427009761172e-11\\
46.571962890625	-4.21344848788954e-11\\
46.59195703125	1.80188875726724e-11\\
46.611951171875	6.76631954742611e-11\\
46.6319453125	1.52100346601022e-10\\
46.651939453125	2.48287865854591e-10\\
46.67193359375	1.48358443251892e-10\\
46.691927734375	1.94496948137671e-10\\
46.711921875	8.20791614052468e-11\\
46.731916015625	5.02973549473215e-11\\
46.75191015625	5.98580503986627e-11\\
46.771904296875	-6.48276700577976e-11\\
46.7918984375	-4.84726781335438e-11\\
46.811892578125	-9.41654787116466e-11\\
46.83188671875	-4.51895023811583e-11\\
46.851880859375	-4.69681571951756e-11\\
46.871875	4.96322720395787e-12\\
46.891869140625	-6.72614447808908e-11\\
46.91186328125	-1.46031719713494e-11\\
46.931857421875	-8.81281012450825e-12\\
46.9518515625	-1.23599280303943e-11\\
46.971845703125	-5.88452211613927e-11\\
46.99183984375	-4.35005178506619e-11\\
47.011833984375	-1.10019899445654e-11\\
47.031828125	-7.44561493462616e-11\\
47.051822265625	3.22346111244778e-12\\
47.07181640625	-3.11977110689977e-11\\
47.091810546875	6.7436712636911e-14\\
47.1118046875	-6.12375662185415e-11\\
47.131798828125	-2.96285995318399e-11\\
47.15179296875	-1.80678175203067e-10\\
47.171787109375	-1.28775806001332e-10\\
47.19178125	-1.36911290387161e-10\\
47.211775390625	-1.28966728105077e-10\\
47.23176953125	-4.62598291216285e-11\\
47.251763671875	-3.66984096393413e-11\\
47.2717578125	8.0660391398332e-11\\
47.291751953125	-2.61785551545542e-11\\
47.31174609375	5.03926291334175e-11\\
47.331740234375	5.40056787886274e-11\\
47.351734375	-1.0945419067672e-10\\
47.371728515625	-5.62488528140436e-11\\
47.39172265625	-1.12871718231749e-10\\
47.411716796875	-8.03095675689923e-11\\
47.4317109375	-1.15421495285507e-10\\
47.451705078125	-4.13094118777238e-11\\
47.47169921875	-4.13989023553326e-11\\
47.491693359375	4.61008044456946e-11\\
47.5116875	1.42281676103253e-11\\
47.531681640625	2.09253201936647e-11\\
47.55167578125	-2.0711421936463e-11\\
47.571669921875	5.18651522091432e-11\\
47.5916640625	-8.57461251021925e-11\\
47.611658203125	-1.31128686496197e-10\\
47.63165234375	-1.83225355100771e-10\\
47.651646484375	-2.12933041028451e-10\\
47.671640625	-1.80835665759734e-10\\
47.691634765625	-2.02520747573767e-10\\
47.71162890625	-2.23451138053055e-10\\
47.731623046875	-1.89409247143636e-10\\
47.7516171875	-1.09475458948308e-10\\
47.771611328125	-2.48427339392265e-10\\
47.79160546875	-9.94012533955513e-11\\
47.811599609375	-1.62520078016661e-10\\
47.83159375	-1.23628393224054e-10\\
47.851587890625	-2.29788429437355e-10\\
47.87158203125	-1.91715891032605e-10\\
47.891576171875	-2.86056227934322e-10\\
47.9115703125	-2.72829400816188e-10\\
47.931564453125	-3.23849946894365e-10\\
47.95155859375	-1.88853769520992e-10\\
47.971552734375	-1.68300285288701e-10\\
47.991546875	-1.3686880556587e-10\\
48.011541015625	-3.12843885898815e-11\\
48.03153515625	-9.38518792314764e-12\\
48.051529296875	-8.95795894187546e-12\\
48.0715234375	2.32012784085559e-11\\
48.091517578125	4.84598894026426e-11\\
48.11151171875	3.9020553335143e-11\\
48.131505859375	-1.25059814496459e-11\\
48.1515	5.72346550117857e-11\\
48.171494140625	9.35471452848812e-11\\
48.19148828125	7.8053894968812e-11\\
48.211482421875	1.74964070638308e-11\\
48.2314765625	1.95641934969665e-11\\
48.251470703125	1.77693254537474e-11\\
48.27146484375	-4.24498511700818e-11\\
48.291458984375	-1.19062349182944e-11\\
48.311453125	1.01819559197531e-10\\
48.331447265625	-7.87556162847898e-12\\
48.35144140625	7.31203572016946e-11\\
48.371435546875	1.18856446184432e-10\\
48.3914296875	1.0085189945461e-10\\
48.411423828125	1.10351874614145e-10\\
48.43141796875	1.85173339529232e-10\\
48.451412109375	1.69134183927514e-10\\
48.47140625	1.55179150357193e-10\\
48.491400390625	1.88257128350252e-10\\
48.51139453125	1.70877043151897e-10\\
48.531388671875	2.45671661429806e-10\\
48.5513828125	1.29765998043579e-10\\
48.571376953125	1.5621251513562e-10\\
48.59137109375	1.46536066950491e-10\\
48.611365234375	1.23805017815716e-10\\
48.631359375	1.81917031203301e-10\\
48.651353515625	1.87590420134323e-10\\
48.67134765625	2.03365697491688e-10\\
48.691341796875	2.2342170909186e-10\\
48.7113359375	2.14176244510544e-10\\
48.731330078125	1.19747833278071e-10\\
48.75132421875	1.80522822244243e-10\\
48.771318359375	1.58549669688037e-10\\
48.7913125	2.04588605512029e-10\\
48.811306640625	2.49813550107701e-10\\
48.83130078125	2.03241091698001e-10\\
48.851294921875	2.16748278165412e-10\\
48.8712890625	2.66064224491563e-10\\
48.891283203125	2.85002851417565e-10\\
48.91127734375	2.82975428195033e-10\\
48.931271484375	3.25906251816544e-10\\
48.951265625	3.89384405606442e-10\\
48.971259765625	3.23413684949649e-10\\
48.99125390625	2.65869070979554e-10\\
49.011248046875	3.94101876801085e-10\\
49.0312421875	3.55411640263033e-10\\
49.051236328125	3.41097163453227e-10\\
49.07123046875	3.17754554294811e-10\\
49.091224609375	3.51337186591293e-10\\
49.11121875	3.04945760992563e-10\\
49.131212890625	3.56907092530895e-10\\
49.15120703125	3.88769050541584e-10\\
49.171201171875	3.29778780280127e-10\\
49.1911953125	3.00430451779725e-10\\
49.211189453125	3.33092199200385e-10\\
49.23118359375	3.6045518310647e-10\\
49.251177734375	3.43919357729711e-10\\
49.271171875	3.74514005279016e-10\\
49.291166015625	3.6747595574055e-10\\
49.31116015625	3.92381082373128e-10\\
49.331154296875	3.61495577505018e-10\\
49.3511484375	4.10981409808433e-10\\
49.371142578125	3.48655551997654e-10\\
49.39113671875	3.6219581375033e-10\\
49.411130859375	3.49153066378379e-10\\
49.431125	3.7771485025465e-10\\
49.451119140625	3.19786867244592e-10\\
49.47111328125	3.47024614465538e-10\\
49.491107421875	3.67712293164607e-10\\
49.5111015625	3.43637425401673e-10\\
49.531095703125	4.23557005030221e-10\\
49.55108984375	3.57987614351249e-10\\
49.571083984375	4.19419108407184e-10\\
49.591078125	4.4536679514922e-10\\
49.611072265625	4.55055698690296e-10\\
49.63106640625	4.69999305968722e-10\\
49.651060546875	4.03102869227385e-10\\
49.6710546875	3.77374533722974e-10\\
49.691048828125	4.14052087678891e-10\\
49.71104296875	3.760659190804e-10\\
49.731037109375	3.5821667387442e-10\\
49.75103125	3.57944788726702e-10\\
49.771025390625	3.89658991884625e-10\\
49.79101953125	3.97045807632574e-10\\
49.811013671875	3.27768045092233e-10\\
49.8310078125	4.16760843733014e-10\\
49.851001953125	4.07940825348848e-10\\
49.87099609375	3.55213714174121e-10\\
49.890990234375	4.04587946562126e-10\\
49.910984375	3.08640347955894e-10\\
49.930978515625	3.47094424470562e-10\\
49.95097265625	2.52864371256758e-10\\
49.970966796875	2.98432516742807e-10\\
49.9909609375	2.08944529624567e-10\\
50.010955078125	2.3235218504928e-10\\
50.03094921875	2.30409091199086e-10\\
50.050943359375	2.19008654730987e-10\\
50.0709375	2.42668515955488e-10\\
50.090931640625	2.87134474378701e-10\\
50.11092578125	2.75510089959477e-10\\
50.130919921875	3.07286443843935e-10\\
50.1509140625	3.82511767860229e-10\\
50.170908203125	4.52087710028853e-10\\
50.19090234375	3.20068274674374e-10\\
50.210896484375	4.35166467793256e-10\\
50.230890625	4.12709373815384e-10\\
50.250884765625	3.67162694535411e-10\\
50.27087890625	3.74858575375178e-10\\
50.290873046875	3.07494368306292e-10\\
50.3108671875	3.44268457072532e-10\\
50.330861328125	2.94865662641845e-10\\
50.35085546875	3.19778170366801e-10\\
50.370849609375	3.60311403731699e-10\\
50.39084375	4.71095277638123e-10\\
50.410837890625	4.27355288640001e-10\\
50.43083203125	5.23686913502206e-10\\
50.450826171875	4.2639308455474e-10\\
50.4708203125	4.79117096032324e-10\\
50.490814453125	3.81960587245164e-10\\
50.51080859375	3.4021520787416e-10\\
50.530802734375	2.83683711269426e-10\\
50.550796875	2.22248258604375e-10\\
50.570791015625	2.44873784930605e-10\\
50.59078515625	2.628093624543e-10\\
50.610779296875	3.0617743101428e-10\\
50.6307734375	3.25650883980055e-10\\
50.650767578125	3.45357439533242e-10\\
50.67076171875	3.57016062851651e-10\\
50.690755859375	3.17558840725608e-10\\
50.71075	3.11018366100905e-10\\
50.730744140625	2.36071050140896e-10\\
50.75073828125	2.79045883149263e-10\\
50.770732421875	3.18253969111127e-10\\
50.7907265625	2.46630395926449e-10\\
50.810720703125	3.09812821936055e-10\\
50.83071484375	2.52866442720596e-10\\
50.850708984375	2.19855380404944e-10\\
50.870703125	2.35648674511309e-10\\
50.890697265625	1.80101709264477e-10\\
50.91069140625	2.15276165540124e-10\\
50.930685546875	1.35607344951586e-10\\
50.9506796875	1.52819373054934e-10\\
50.970673828125	1.56854235798608e-10\\
50.99066796875	1.73085655605305e-10\\
51.010662109375	1.33880614256135e-10\\
51.03065625	4.26813607296275e-11\\
51.050650390625	1.021674364405e-10\\
51.07064453125	2.78606609968045e-11\\
51.090638671875	-1.12015840527641e-11\\
51.1106328125	-7.91230076496463e-11\\
51.130626953125	-5.64407671148981e-11\\
51.15062109375	-1.10819181774938e-11\\
51.170615234375	-7.91142311671032e-11\\
51.190609375	-6.2216925958255e-11\\
51.210603515625	-8.22754188489818e-11\\
51.23059765625	-1.13101196019508e-10\\
51.250591796875	-1.2591176042863e-10\\
51.2705859375	-2.39557031196336e-11\\
51.290580078125	-1.16396692163179e-10\\
51.31057421875	-1.06777006141555e-10\\
51.330568359375	-1.59927230756713e-10\\
51.3505625	-1.80858574741722e-10\\
51.370556640625	-2.15979398053405e-10\\
51.39055078125	-1.63810816905799e-10\\
51.410544921875	-1.35590480387997e-10\\
51.4305390625	-1.88353258220867e-10\\
51.450533203125	-2.08690580031139e-10\\
51.47052734375	-1.36405566011649e-10\\
51.490521484375	-1.41624963593369e-10\\
51.510515625	-1.86563021193912e-10\\
51.530509765625	-1.53893186293653e-10\\
51.55050390625	-2.00977125884514e-10\\
51.570498046875	-2.33186865395061e-10\\
51.5904921875	-2.81596890994332e-10\\
51.610486328125	-2.27813956742382e-10\\
51.63048046875	-2.3356066202157e-10\\
51.650474609375	-1.85815217103905e-10\\
51.67046875	-2.22048237772169e-10\\
51.690462890625	-1.67584521627358e-10\\
51.71045703125	-1.98873460626086e-10\\
51.730451171875	-2.13016868638834e-10\\
51.7504453125	-2.39398078224775e-10\\
51.770439453125	-2.20615699607075e-10\\
51.79043359375	-2.84230892833876e-10\\
51.810427734375	-2.6949681087925e-10\\
51.830421875	-3.06462452669624e-10\\
51.850416015625	-3.52584454957873e-10\\
51.87041015625	-3.04425931482167e-10\\
51.890404296875	-2.72870345165396e-10\\
51.9103984375	-2.92674842792614e-10\\
51.930392578125	-2.82093788371439e-10\\
51.95038671875	-2.68541380954925e-10\\
51.970380859375	-2.800460036926e-10\\
51.990375	-3.15898357629108e-10\\
52.010369140625	-3.48294102052811e-10\\
52.03036328125	-3.21979413302034e-10\\
52.050357421875	-4.27761495662838e-10\\
52.0703515625	-4.54076104657318e-10\\
52.090345703125	-3.51432049662886e-10\\
52.11033984375	-4.4597530205213e-10\\
52.130333984375	-4.29143611368915e-10\\
52.150328125	-3.50807910036608e-10\\
52.170322265625	-3.96580781205074e-10\\
52.19031640625	-4.47552604112901e-10\\
52.210310546875	-3.86339444567134e-10\\
52.2303046875	-4.30331586243332e-10\\
52.250298828125	-4.01940959631437e-10\\
52.27029296875	-3.98527562013302e-10\\
52.290287109375	-4.96640211203404e-10\\
52.31028125	-4.31944803008063e-10\\
52.330275390625	-4.74827872385497e-10\\
52.35026953125	-4.38449510722294e-10\\
52.370263671875	-5.12994737075112e-10\\
52.3902578125	-4.87194430359879e-10\\
52.410251953125	-4.99178134773111e-10\\
52.43024609375	-5.30618589106456e-10\\
52.450240234375	-4.64773924686653e-10\\
52.470234375	-5.21992605825855e-10\\
52.490228515625	-4.24388949317174e-10\\
52.51022265625	-4.65465446350298e-10\\
52.530216796875	-4.24979253160958e-10\\
52.5502109375	-4.17977771115464e-10\\
52.570205078125	-4.33468726283921e-10\\
52.59019921875	-4.56315032042127e-10\\
52.610193359375	-4.77730077383514e-10\\
52.6301875	-4.98781671023613e-10\\
52.650181640625	-4.66532256224649e-10\\
52.67017578125	-5.24901417921296e-10\\
52.690169921875	-4.69227097459303e-10\\
52.7101640625	-5.30427491748174e-10\\
52.730158203125	-4.81276384040058e-10\\
52.75015234375	-4.96888258740086e-10\\
52.770146484375	-4.80552274732601e-10\\
52.790140625	-5.11524478671445e-10\\
52.810134765625	-5.05920603911818e-10\\
52.83012890625	-5.07143719314894e-10\\
52.850123046875	-5.24469191347812e-10\\
52.8701171875	-4.77395529700253e-10\\
52.890111328125	-5.84297560439913e-10\\
52.91010546875	-5.08925429520052e-10\\
52.930099609375	-5.65033035217148e-10\\
52.95009375	-4.61476300782432e-10\\
52.970087890625	-5.20616753864027e-10\\
52.99008203125	-4.76381445259141e-10\\
53.010076171875	-4.62929481930626e-10\\
53.0300703125	-3.84516369823874e-10\\
53.050064453125	-4.17694355064877e-10\\
53.07005859375	-3.57014626161362e-10\\
53.090052734375	-3.66031932903765e-10\\
53.110046875	-3.64911984930008e-10\\
53.130041015625	-3.83162081534514e-10\\
53.15003515625	-3.70486066402882e-10\\
53.170029296875	-3.76195592168523e-10\\
53.1900234375	-3.71611104258618e-10\\
53.210017578125	-3.32760605419202e-10\\
53.23001171875	-3.41289329343084e-10\\
53.250005859375	-3.15358251389318e-10\\
53.27	-2.51688124181781e-10\\
53.289994140625	-3.06003472878986e-10\\
53.30998828125	-3.2908808883905e-10\\
53.329982421875	-2.24098680406805e-10\\
53.3499765625	-3.12348357667503e-10\\
53.369970703125	-2.50533964512557e-10\\
53.38996484375	-2.52778670467824e-10\\
53.409958984375	-2.45596418245063e-10\\
53.429953125	-2.19203793268919e-10\\
53.449947265625	-2.67679130654856e-10\\
53.46994140625	-2.53729820348221e-10\\
53.489935546875	-2.34706552019738e-10\\
53.5099296875	-2.1121122102702e-10\\
53.529923828125	-1.01369639321321e-10\\
53.54991796875	-1.15192392306449e-10\\
53.569912109375	-1.38549511815835e-10\\
53.58990625	-1.45472515708392e-10\\
53.609900390625	-1.15105021003534e-10\\
53.62989453125	-7.7498801390491e-11\\
};
\addlegendentry{$\text{train 5 -\textgreater{} Heimdal}$};

\end{axis}
\end{tikzpicture}%
	\label{fig:train5}
\end{subfigure}
\qquad
\begin{subfigure}[t]{0.45\textwidth}
	\centering
	% This file was created by matlab2tikz.
%
%The latest updates can be retrieved from
%  http://www.mathworks.com/matlabcentral/fileexchange/22022-matlab2tikz-matlab2tikz
%where you can also make suggestions and rate matlab2tikz.
%
\definecolor{mycolor1}{rgb}{0.00000,0.44700,0.74100}%
%
\begin{tikzpicture}

\begin{axis}[%
width=\textwidth,
height=\textwidth,
at={(0\figurewidth,0\figureheight)},
scale only axis,
xmin=-60,
xmax=60,
ymin=-2e-09,
ymax=1.2e-08,
axis background/.style={fill=white},
% title style={font=\bfseries},
% title={Influencelines for train 8, middle sensor},
legend style={legend cell align=left,align=left,draw=white!15!black}
]
\addplot [color=mycolor1,solid,forget plot]
  table[row sep=crcr]{%
-45.1653759765625	8.45071704492093e-11\\
-45.1452265625	1.47186402046743e-10\\
-45.1250771484375	1.66537649863325e-10\\
-45.104927734375	1.89168744488152e-10\\
-45.0847783203125	2.51825909432672e-10\\
-45.06462890625	1.85140700118071e-10\\
-45.0444794921875	1.60974487947595e-10\\
-45.024330078125	1.96776212132971e-10\\
-45.0041806640625	1.07660351893289e-10\\
-44.98403125	1.04467819996769e-10\\
-44.9638818359375	6.45725450081922e-11\\
-44.943732421875	4.05755900340877e-11\\
-44.9235830078125	-4.72656635694992e-12\\
-44.90343359375	6.25518182974636e-11\\
-44.8832841796875	1.26766336467024e-10\\
-44.863134765625	2.34843363093219e-10\\
-44.8429853515625	2.26711671938427e-10\\
-44.8228359375	3.53291595053458e-10\\
-44.8026865234375	3.20931226924431e-10\\
-44.782537109375	3.51496266578038e-10\\
-44.7623876953125	3.34590382489976e-10\\
-44.74223828125	1.10858527702809e-10\\
-44.7220888671875	1.88966267643341e-10\\
-44.701939453125	1.17333568570331e-10\\
-44.6817900390625	1.24254817895901e-10\\
-44.661640625	1.120500433092e-10\\
-44.6414912109375	1.33403976391458e-10\\
-44.621341796875	1.75097188285274e-10\\
-44.6011923828125	2.25601881605018e-10\\
-44.58104296875	2.66689427133213e-10\\
-44.5608935546875	3.09691606054273e-10\\
-44.540744140625	2.61335101040893e-10\\
-44.5205947265625	2.64434175091679e-10\\
-44.5004453125	1.99922961692525e-10\\
-44.4802958984375	1.84576385776907e-10\\
-44.460146484375	1.89862142437024e-10\\
-44.4399970703125	1.49824669366176e-10\\
-44.41984765625	2.16388628248071e-10\\
-44.3996982421875	1.92810996885641e-10\\
-44.379548828125	3.01344889499769e-10\\
-44.3593994140625	2.79694625761555e-10\\
-44.33925	2.56869307121692e-10\\
-44.3191005859375	3.3681499559406e-10\\
-44.298951171875	3.33285724890725e-10\\
-44.2788017578125	3.73417248672543e-10\\
-44.25865234375	3.98023612073178e-10\\
-44.2385029296875	3.39531182462425e-10\\
-44.218353515625	4.14087514950173e-10\\
-44.1982041015625	4.10562980772119e-10\\
-44.1780546875	3.92123265837225e-10\\
-44.1579052734375	3.84016754961744e-10\\
-44.137755859375	4.05451430056021e-10\\
-44.1176064453125	3.74549302974662e-10\\
-44.09745703125	3.28666754250349e-10\\
-44.0773076171875	3.24667805742834e-10\\
-44.057158203125	2.96102964526397e-10\\
-44.0370087890625	3.06166926992747e-10\\
-44.016859375	2.37722427008296e-10\\
-43.9967099609375	2.47227237896423e-10\\
-43.976560546875	2.87873189527022e-10\\
-43.9564111328125	3.63731566510949e-10\\
-43.93626171875	3.70302606907852e-10\\
-43.9161123046875	3.07924070925648e-10\\
-43.895962890625	2.70713732140356e-10\\
-43.8758134765625	2.4564644160087e-10\\
-43.8556640625	1.78124907526322e-10\\
-43.8355146484375	1.49452014138433e-10\\
-43.815365234375	1.12530586963397e-10\\
-43.7952158203125	1.11295888407545e-10\\
-43.77506640625	9.86461774800864e-11\\
-43.7549169921875	1.68386464469345e-10\\
-43.734767578125	2.4942932926659e-10\\
-43.7146181640625	3.70869528264839e-10\\
-43.69446875	3.85159281944497e-10\\
-43.6743193359375	2.57886080464655e-10\\
-43.654169921875	2.9191411598409e-10\\
-43.6340205078125	2.54179869151856e-10\\
-43.61387109375	1.80977788398939e-10\\
-43.5937216796875	1.13516269106395e-10\\
-43.573572265625	1.26467709892426e-10\\
-43.5534228515625	1.82686228943838e-10\\
-43.5332734375	2.22860394517438e-10\\
-43.5131240234375	3.66385417422612e-10\\
-43.492974609375	4.34295626885672e-10\\
-43.4728251953125	5.15558954869194e-10\\
-43.45267578125	5.52055707911911e-10\\
-43.4325263671875	5.03581272386895e-10\\
-43.412376953125	5.16541230590738e-10\\
-43.3922275390625	4.53525220440843e-10\\
-43.372078125	3.72409591934362e-10\\
-43.3519287109375	3.5689514469571e-10\\
-43.331779296875	3.05825497601182e-10\\
-43.3116298828125	3.03244983197268e-10\\
-43.29148046875	3.75300675076262e-10\\
-43.2713310546875	4.96129589596963e-10\\
-43.251181640625	4.77648559795403e-10\\
-43.2310322265625	5.47211486284802e-10\\
-43.2108828125	5.329399761301e-10\\
-43.1907333984375	4.99307190755159e-10\\
-43.170583984375	5.01740048970684e-10\\
-43.1504345703125	4.00304593491811e-10\\
-43.13028515625	3.31766472313022e-10\\
-43.1101357421875	3.028239910383e-10\\
-43.089986328125	2.45927514525047e-10\\
-43.0698369140625	3.17538944272444e-10\\
-43.0496875	3.00506931834751e-10\\
-43.0295380859375	2.18062294376546e-10\\
-43.009388671875	3.27746233289089e-10\\
-42.9892392578125	3.58008669030622e-10\\
-42.96908984375	3.56320374151984e-10\\
-42.9489404296875	3.49896839028505e-10\\
-42.928791015625	3.19947755930646e-10\\
-42.9086416015625	3.30748031748884e-10\\
-42.8884921875	2.48746029881314e-10\\
-42.8683427734375	2.84291593741262e-10\\
-42.848193359375	2.74472334507741e-10\\
-42.8280439453125	3.28615614044435e-10\\
-42.80789453125	3.14554926487969e-10\\
-42.7877451171875	3.52706575851935e-10\\
-42.767595703125	2.80215753022282e-10\\
-42.7474462890625	2.42832960419557e-10\\
-42.727296875	1.82939478855698e-10\\
-42.7071474609375	1.29810047902454e-10\\
-42.686998046875	7.35418427088892e-11\\
-42.6668486328125	1.37000639274501e-11\\
-42.64669921875	4.74950083467495e-11\\
-42.6265498046875	1.04061557464502e-10\\
-42.606400390625	1.39615702503384e-10\\
-42.5862509765625	8.72575979367904e-11\\
-42.5661015625	1.13986901682991e-10\\
-42.5459521484375	1.66912105717008e-10\\
-42.525802734375	1.27405316275178e-10\\
-42.5056533203125	5.11708308894323e-11\\
-42.48550390625	5.18762546547704e-11\\
-42.4653544921875	-3.08236903313104e-11\\
-42.445205078125	-3.25988479936393e-11\\
-42.4250556640625	-7.91700641654964e-11\\
-42.40490625	-4.40813165819488e-11\\
-42.3847568359375	-1.77853792709322e-11\\
-42.364607421875	-3.15291105511583e-11\\
-42.3444580078125	1.27473021494965e-12\\
-42.32430859375	1.24430585703477e-11\\
-42.3041591796875	6.23388089846842e-12\\
-42.284009765625	3.83086455196708e-11\\
-42.2638603515625	6.23732620629207e-12\\
-42.2437109375	-4.1733587097987e-11\\
-42.2235615234375	-6.82278278663436e-11\\
-42.203412109375	-1.43940528166327e-10\\
-42.1832626953125	-1.11079541582209e-10\\
-42.16311328125	-8.58980878129485e-11\\
-42.1429638671875	-1.06199496531501e-10\\
-42.122814453125	-5.12576502868089e-11\\
-42.1026650390625	-5.9035404435504e-11\\
-42.082515625	-8.41759418032227e-11\\
-42.0623662109375	-7.15507713948961e-11\\
-42.042216796875	1.6963411021588e-11\\
-42.0220673828125	-3.50171813998166e-11\\
-42.00191796875	-1.20097564510386e-10\\
-41.9817685546875	-1.00369633711301e-10\\
-41.961619140625	-1.00672467013208e-10\\
-41.9414697265625	-1.30306351173618e-10\\
-41.9213203125	-1.1149445259239e-10\\
-41.9011708984375	-9.08683711606058e-11\\
-41.881021484375	-8.22787639769717e-11\\
-41.8608720703125	-1.2021823063252e-10\\
-41.84072265625	-6.84178223979833e-11\\
-41.8205732421875	-1.46145052843903e-10\\
-41.800423828125	-1.25108789909495e-10\\
-41.7802744140625	-5.4521572528833e-11\\
-41.760125	-9.79746596629107e-11\\
-41.7399755859375	-5.71895223899277e-11\\
-41.719826171875	-1.60689919819299e-10\\
-41.6996767578125	-2.11583337962421e-10\\
-41.67952734375	-1.45118446444819e-10\\
-41.6593779296875	-2.60817455742576e-10\\
-41.639228515625	-2.39987604030519e-10\\
-41.6190791015625	-3.31621680366851e-10\\
-41.5989296875	-2.8809082227865e-10\\
-41.5787802734375	-3.03962329154317e-10\\
-41.558630859375	-2.8960501712943e-10\\
-41.5384814453125	-2.53678137439546e-10\\
-41.51833203125	-2.72755629400946e-10\\
-41.4981826171875	-1.99071776675999e-10\\
-41.478033203125	-2.66077858505019e-10\\
-41.4578837890625	-2.92454135728307e-10\\
-41.437734375	-2.26776445485712e-10\\
-41.4175849609375	-1.71996510456505e-10\\
-41.397435546875	-2.38883457098826e-10\\
-41.3772861328125	-3.15872412618553e-10\\
-41.35713671875	-2.78732158229254e-10\\
-41.3369873046875	-3.46880198498753e-10\\
-41.316837890625	-3.09969360265575e-10\\
-41.2966884765625	-2.8241588208718e-10\\
-41.2765390625	-2.83783194261095e-10\\
-41.2563896484375	-2.27813248084786e-10\\
-41.236240234375	-2.43028386979129e-10\\
-41.2160908203125	-3.0645204552716e-10\\
-41.19594140625	-2.27341515020874e-10\\
-41.1757919921875	-3.5687281657363e-10\\
-41.155642578125	-2.7738280847184e-10\\
-41.1354931640625	-3.10382772352612e-10\\
-41.11534375	-3.42105989934424e-10\\
-41.0951943359375	-2.93391243140955e-10\\
-41.075044921875	-3.05574798786699e-10\\
-41.0548955078125	-3.75981931614062e-10\\
-41.03474609375	-2.53005262219133e-10\\
-41.0145966796875	-2.64289205022848e-10\\
-40.994447265625	-3.12822052973208e-10\\
-40.9742978515625	-2.49673425365842e-10\\
-40.9541484375	-2.91927478875861e-10\\
-40.9339990234375	-3.20971717867444e-10\\
-40.913849609375	-3.61159014213956e-10\\
-40.8937001953125	-4.24865327049043e-10\\
-40.87355078125	-4.40893371399578e-10\\
-40.8534013671875	-4.25137553904161e-10\\
-40.833251953125	-4.39388776476773e-10\\
-40.8131025390625	-3.54846751266613e-10\\
-40.792953125	-3.57052456077831e-10\\
-40.7728037109375	-3.11749598989538e-10\\
-40.752654296875	-2.96019323645466e-10\\
-40.7325048828125	-2.8295560459537e-10\\
-40.71235546875	-3.24462992123191e-10\\
-40.6922060546875	-3.539137646497e-10\\
-40.672056640625	-3.71895473644621e-10\\
-40.6519072265625	-3.66517611827136e-10\\
-40.6317578125	-3.77807248889677e-10\\
-40.6116083984375	-4.23156319336844e-10\\
-40.591458984375	-3.33200007256496e-10\\
-40.5713095703125	-3.04201765253259e-10\\
-40.55116015625	-3.47362521292898e-10\\
-40.5310107421875	-2.13218462555475e-10\\
-40.510861328125	-2.47799265853614e-10\\
-40.4907119140625	-3.49267300928087e-10\\
-40.4705625	-2.37048040500289e-10\\
-40.4504130859375	-2.84160495359225e-10\\
-40.430263671875	-3.54668890547381e-10\\
-40.4101142578125	-3.86054454104878e-10\\
-40.38996484375	-3.35518865728542e-10\\
-40.3698154296875	-4.30467528545092e-10\\
-40.349666015625	-3.53348145239417e-10\\
-40.3295166015625	-3.20049591911552e-10\\
-40.3093671875	-1.91799451487675e-10\\
-40.2892177734375	-2.07584312434791e-10\\
-40.269068359375	-2.96278619644099e-10\\
-40.2489189453125	-1.95990157765772e-10\\
-40.22876953125	-2.75436147898873e-10\\
-40.2086201171875	-2.61643387373477e-10\\
-40.188470703125	-3.44389786016329e-10\\
-40.1683212890625	-2.05916983114685e-10\\
-40.148171875	-2.4507132318369e-10\\
-40.1280224609375	-2.38216080192966e-10\\
-40.107873046875	-1.27938418580172e-10\\
-40.0877236328125	-7.43888055562643e-11\\
-40.06757421875	-1.09776901360383e-10\\
-40.0474248046875	-9.13680514744758e-11\\
-40.027275390625	-2.06745345958897e-11\\
-40.0071259765625	2.35290414302332e-12\\
-39.9869765625	-3.12855655079708e-11\\
-39.9668271484375	-1.88580387741714e-11\\
-39.946677734375	3.34897320973204e-11\\
-39.9265283203125	3.10874046942056e-11\\
-39.90637890625	-2.19263597150404e-11\\
-39.8862294921875	-2.37118492952963e-11\\
-39.866080078125	-1.21817514968979e-11\\
-39.8459306640625	1.69536808572227e-11\\
-39.82578125	6.96510362456965e-11\\
-39.8056318359375	1.48120042071378e-10\\
-39.785482421875	1.70267124431307e-10\\
-39.7653330078125	1.86836717439551e-10\\
-39.74518359375	1.03000912194164e-10\\
-39.7250341796875	7.8934776499808e-11\\
-39.704884765625	8.71912686437777e-11\\
-39.6847353515625	5.09577686275623e-11\\
-39.6645859375	1.19449809734938e-10\\
-39.6444365234375	1.0282845168955e-10\\
-39.624287109375	1.13165843420284e-10\\
-39.6041376953125	8.40113239150128e-11\\
-39.58398828125	1.46310466113093e-10\\
-39.5638388671875	1.06518400206331e-10\\
-39.543689453125	1.2417566415594e-10\\
-39.5235400390625	1.23680633907791e-10\\
-39.503390625	1.89193830311941e-10\\
-39.4832412109375	1.20356027565979e-10\\
-39.463091796875	1.52200237114705e-10\\
-39.4429423828125	1.4462676252214e-10\\
-39.42279296875	1.20310247510465e-10\\
-39.4026435546875	1.16553357448759e-10\\
-39.382494140625	1.25724803102885e-10\\
-39.3623447265625	2.01570414786787e-10\\
-39.3421953125	1.49614288495762e-10\\
-39.3220458984375	1.31509806671003e-10\\
-39.301896484375	1.51001465918215e-10\\
-39.2817470703125	2.10224130652202e-10\\
-39.26159765625	2.13231774474774e-10\\
-39.2414482421875	2.22718430658176e-10\\
-39.221298828125	2.12938319404898e-10\\
-39.2011494140625	2.55204968997117e-10\\
-39.181	1.42981114229063e-10\\
-39.1608505859375	2.66119596656742e-10\\
-39.140701171875	2.84158237367409e-10\\
-39.1205517578125	3.23177659545226e-10\\
-39.10040234375	2.96981716957021e-10\\
-39.0802529296875	3.54347392768925e-10\\
-39.060103515625	3.3788638034364e-10\\
-39.0399541015625	3.14397449940228e-10\\
-39.0198046875	4.08166664164557e-10\\
-38.9996552734375	3.68323815716061e-10\\
-38.979505859375	3.62233890395841e-10\\
-38.9593564453125	3.43234128746621e-10\\
-38.93920703125	4.12778452286625e-10\\
-38.9190576171875	4.10434836427291e-10\\
-38.898908203125	5.06268110291126e-10\\
-38.8787587890625	4.56361904611914e-10\\
-38.858609375	5.28217105965641e-10\\
-38.8384599609375	4.07202602704635e-10\\
-38.818310546875	4.33448970801117e-10\\
-38.7981611328125	3.33579017652303e-10\\
-38.77801171875	3.99654831267416e-10\\
-38.7578623046875	3.37970955942681e-10\\
-38.737712890625	3.8692720010138e-10\\
-38.7175634765625	3.69910847491894e-10\\
-38.6974140625	4.42622275628115e-10\\
-38.6772646484375	4.97497891986403e-10\\
-38.657115234375	4.61917714023834e-10\\
-38.6369658203125	4.07542865770421e-10\\
-38.61681640625	4.17347666726089e-10\\
-38.5966669921875	3.34251805250722e-10\\
-38.576517578125	3.33267746557604e-10\\
-38.5563681640625	3.64492810361604e-10\\
-38.53621875	3.72141328209962e-10\\
-38.5160693359375	3.43147685784677e-10\\
-38.495919921875	4.86866479701063e-10\\
-38.4757705078125	4.49107207575412e-10\\
-38.45562109375	5.34896538643991e-10\\
-38.4354716796875	6.0664056406698e-10\\
-38.415322265625	4.56788141102212e-10\\
-38.3951728515625	3.99084460035095e-10\\
-38.3750234375	3.88966835749001e-10\\
-38.3548740234375	3.58276647808498e-10\\
-38.334724609375	3.84498836419529e-10\\
-38.3145751953125	3.40023037498424e-10\\
-38.29442578125	3.69357549490894e-10\\
-38.2742763671875	4.49563067558909e-10\\
-38.254126953125	4.44822235097409e-10\\
-38.2339775390625	4.92722433622869e-10\\
-38.213828125	5.20305942602462e-10\\
-38.1936787109375	4.6120243167384e-10\\
-38.173529296875	3.6915425160187e-10\\
-38.1533798828125	3.3906328858504e-10\\
-38.13323046875	2.8831947308937e-10\\
-38.1130810546875	3.53453467501004e-10\\
-38.092931640625	2.38147758826311e-10\\
-38.0727822265625	3.13455500848482e-10\\
-38.0526328125	3.57709514551976e-10\\
-38.0324833984375	3.92986389181146e-10\\
-38.012333984375	4.16832732961038e-10\\
-37.9921845703125	4.18035947083047e-10\\
-37.97203515625	1.85513628946487e-10\\
-37.9518857421875	2.75668175296938e-10\\
-37.931736328125	2.50368147259035e-10\\
-37.9115869140625	2.65019497849782e-10\\
-37.8914375	2.44337320972701e-10\\
-37.8712880859375	2.5611669803033e-10\\
-37.851138671875	2.50143330261398e-10\\
-37.8309892578125	3.7020112330243e-10\\
-37.81083984375	3.85744268227185e-10\\
-37.7906904296875	3.41554125329755e-10\\
-37.770541015625	3.53974457624128e-10\\
-37.7503916015625	2.12242065265684e-10\\
-37.7302421875	1.74221489950698e-10\\
-37.7100927734375	1.62173633189577e-10\\
-37.689943359375	1.21238374767504e-10\\
-37.6697939453125	6.17459880865736e-11\\
-37.64964453125	1.24845371399906e-10\\
-37.6294951171875	1.60753513382174e-10\\
-37.609345703125	2.11312769212862e-10\\
-37.5891962890625	2.53760163524158e-10\\
-37.569046875	2.31869307056868e-10\\
-37.5488974609375	2.95066034982466e-10\\
-37.528748046875	1.52143697751197e-10\\
-37.5085986328125	9.60445523698819e-11\\
-37.48844921875	7.95323031954592e-11\\
-37.4682998046875	2.58955794179104e-11\\
-37.448150390625	-9.57615507741424e-11\\
-37.4280009765625	-1.23556017154178e-10\\
-37.4078515625	-1.16989445950445e-10\\
-37.3877021484375	-7.74225304991874e-11\\
-37.367552734375	-2.52840660436308e-11\\
-37.3474033203125	-7.36195680458553e-11\\
-37.32725390625	-4.58634145224955e-12\\
-37.3071044921875	-7.0287300264927e-11\\
-37.286955078125	-5.83922836708437e-11\\
-37.2668056640625	-1.16762947403622e-10\\
-37.24665625	-1.76390570979185e-10\\
-37.2265068359375	-1.91018634755896e-10\\
-37.206357421875	-1.47781230858945e-10\\
-37.1862080078125	-2.05566119956639e-10\\
-37.16605859375	-1.3813970970571e-10\\
-37.1459091796875	-2.05887204738469e-10\\
-37.125759765625	-1.19910722113508e-10\\
-37.1056103515625	-9.09357280293361e-11\\
-37.0854609375	-1.26910823278236e-10\\
-37.0653115234375	-1.23930178708643e-10\\
-37.045162109375	-1.60610433652702e-10\\
-37.0250126953125	-1.2539120499826e-10\\
-37.00486328125	-2.0953696091343e-10\\
-36.9847138671875	-2.14267967618891e-10\\
-36.964564453125	-2.29890634847864e-10\\
-36.9444150390625	-2.83173482565027e-10\\
-36.924265625	-3.73174710652255e-10\\
-36.9041162109375	-2.45257518116384e-10\\
-36.883966796875	-3.1844001381099e-10\\
-36.8638173828125	-3.64823258097997e-10\\
-36.84366796875	-2.31604326347603e-10\\
-36.8235185546875	-2.63003019601731e-10\\
-36.803369140625	-3.32766459857455e-10\\
-36.7832197265625	-3.05557924680684e-10\\
-36.7630703125	-2.72059567778909e-10\\
-36.7429208984375	-3.60825193938205e-10\\
-36.722771484375	-3.4507892728853e-10\\
-36.7026220703125	-3.68670784423006e-10\\
-36.68247265625	-3.85179294659965e-10\\
-36.6623232421875	-4.2075992612794e-10\\
-36.642173828125	-3.91763593034071e-10\\
-36.6220244140625	-3.11190635898247e-10\\
-36.601875	-4.14352918101326e-10\\
-36.5817255859375	-4.27310086915522e-10\\
-36.561576171875	-4.23359039590705e-10\\
-36.5414267578125	-4.52641539883995e-10\\
-36.52127734375	-4.88658011317105e-10\\
-36.5011279296875	-4.72584222433935e-10\\
-36.480978515625	-4.64123270973071e-10\\
-36.4608291015625	-4.66009776674485e-10\\
-36.4406796875	-5.0248351515004e-10\\
-36.4205302734375	-4.21168062211283e-10\\
-36.400380859375	-4.54112243074265e-10\\
-36.3802314453125	-5.35822288125432e-10\\
-36.36008203125	-6.02813410721471e-10\\
-36.3399326171875	-5.71459548362459e-10\\
-36.319783203125	-6.27297988411162e-10\\
-36.2996337890625	-6.09578382800366e-10\\
-36.279484375	-5.32802037592354e-10\\
-36.2593349609375	-4.95261050675149e-10\\
-36.239185546875	-4.95785482414829e-10\\
-36.2190361328125	-3.87865799660882e-10\\
-36.19888671875	-4.4828398223117e-10\\
-36.1787373046875	-4.2784674273163e-10\\
-36.158587890625	-5.03160294648753e-10\\
-36.1384384765625	-6.25433856785078e-10\\
-36.1182890625	-6.62818393116765e-10\\
-36.0981396484375	-6.07425021049149e-10\\
-36.077990234375	-6.46503112725529e-10\\
-36.0578408203125	-5.48106651359939e-10\\
-36.03769140625	-5.61172235509216e-10\\
-36.0175419921875	-5.22931174006698e-10\\
-35.997392578125	-4.8791646794321e-10\\
-35.9772431640625	-4.03045956238178e-10\\
-35.95709375	-4.91157891086813e-10\\
-35.9369443359375	-5.11570567235526e-10\\
-35.916794921875	-5.58646628240315e-10\\
-35.8966455078125	-6.56373555161862e-10\\
-35.87649609375	-7.11415154699367e-10\\
-35.8563466796875	-8.01348306718201e-10\\
-35.836197265625	-7.17125879029735e-10\\
-35.8160478515625	-6.17283935775429e-10\\
-35.7958984375	-6.20002642822935e-10\\
-35.7757490234375	-5.3891783012323e-10\\
-35.755599609375	-4.39400502813209e-10\\
-35.7354501953125	-4.48115698722041e-10\\
-35.71530078125	-4.92512359914353e-10\\
-35.6951513671875	-5.26624945810048e-10\\
-35.675001953125	-5.50965984048777e-10\\
-35.6548525390625	-5.46083368666888e-10\\
-35.634703125	-6.72064782084104e-10\\
-35.6145537109375	-4.81007823875804e-10\\
-35.594404296875	-4.97228714143445e-10\\
-35.5742548828125	-5.15862335898206e-10\\
-35.55410546875	-4.51056108573085e-10\\
-35.5339560546875	-4.01687977352936e-10\\
-35.513806640625	-4.22226474889249e-10\\
-35.4936572265625	-4.92749813678858e-10\\
-35.4735078125	-4.2550810337838e-10\\
-35.4533583984375	-4.95546308161018e-10\\
-35.433208984375	-4.57595907787351e-10\\
-35.4130595703125	-3.55117793760708e-10\\
-35.39291015625	-3.74874306248711e-10\\
-35.3727607421875	-3.32240723228091e-10\\
-35.352611328125	-2.17159744881258e-10\\
-35.3324619140625	-2.79822165273226e-10\\
-35.3123125	-2.47268480726679e-10\\
-35.2921630859375	-1.47404604589299e-10\\
-35.272013671875	-3.09945571603633e-10\\
-35.2518642578125	-3.26045390474208e-10\\
-35.23171484375	-3.24762342039988e-10\\
-35.2115654296875	-2.66172919086767e-10\\
-35.191416015625	-1.65368580025121e-10\\
-35.1712666015625	-1.49424221633605e-10\\
-35.1511171875	-1.16524654548054e-10\\
-35.1309677734375	-2.31527348404749e-11\\
-35.110818359375	4.4928335039214e-11\\
-35.0906689453125	-4.33615421569865e-11\\
-35.07051953125	3.32818812699984e-11\\
-35.0503701171875	-1.25341381113758e-10\\
-35.030220703125	-9.94789969890685e-11\\
-35.0100712890625	-5.34831212398412e-11\\
-34.989921875	-1.3978166246004e-10\\
-34.9697724609375	-6.76122048146978e-11\\
-34.949623046875	-8.62889017343632e-11\\
-34.9294736328125	-2.41806200069764e-11\\
-34.90932421875	4.9687113686835e-11\\
-34.8891748046875	1.51144167069511e-10\\
-34.869025390625	1.22017532796676e-10\\
-34.8488759765625	1.65078240257204e-10\\
-34.8287265625	1.84036269113898e-10\\
-34.8085771484375	7.46715989681544e-11\\
-34.788427734375	1.57138744709713e-10\\
-34.7682783203125	1.21515038087188e-10\\
-34.74812890625	2.4104791016929e-10\\
-34.7279794921875	1.4414166967577e-10\\
-34.707830078125	1.3814143272814e-10\\
-34.6876806640625	1.83858874280822e-10\\
-34.66753125	2.41779437314929e-10\\
-34.6473818359375	2.51259814708783e-10\\
-34.627232421875	3.40270163799107e-10\\
-34.6070830078125	2.90069105405359e-10\\
-34.58693359375	2.78498854600137e-10\\
-34.5667841796875	2.66918527875214e-10\\
-34.546634765625	2.64263316081027e-10\\
-34.5264853515625	2.9080033080136e-10\\
-34.5063359375	1.92037190157532e-10\\
-34.4861865234375	2.76028787467887e-10\\
-34.466037109375	2.03082848705794e-10\\
-34.4458876953125	2.49780827828785e-10\\
-34.42573828125	3.57619792564682e-10\\
-34.4055888671875	3.36853438523902e-10\\
-34.385439453125	3.23239681962815e-10\\
-34.3652900390625	4.65036586860955e-10\\
-34.345140625	3.94644238440784e-10\\
-34.3249912109375	4.55918345223526e-10\\
-34.304841796875	4.69205403822081e-10\\
-34.2846923828125	4.3445904684478e-10\\
-34.26454296875	4.27548944578574e-10\\
-34.2443935546875	4.15112335578268e-10\\
-34.224244140625	4.95958363313845e-10\\
-34.2040947265625	4.73646817328286e-10\\
-34.1839453125	4.43388550429409e-10\\
-34.1637958984375	5.1855678897591e-10\\
-34.143646484375	6.41990657585918e-10\\
-34.1234970703125	5.06732015563703e-10\\
-34.10334765625	6.01238914930195e-10\\
-34.0831982421875	5.60760844420515e-10\\
-34.063048828125	5.57198130951925e-10\\
-34.0428994140625	5.87878849998064e-10\\
-34.02275	5.5129321256992e-10\\
-34.0026005859375	5.30525333322101e-10\\
-33.982451171875	4.99915453414278e-10\\
-33.9623017578125	5.48444222467402e-10\\
-33.94215234375	5.69483897346804e-10\\
-33.9220029296875	5.65429009031269e-10\\
-33.901853515625	5.74661892125626e-10\\
-33.8817041015625	6.52052860891272e-10\\
-33.8615546875	6.92162788503604e-10\\
-33.8414052734375	7.5447147542435e-10\\
-33.821255859375	7.51206358468145e-10\\
-33.8011064453125	7.49774776035306e-10\\
-33.78095703125	7.29662475705791e-10\\
-33.7608076171875	6.97519276974136e-10\\
-33.740658203125	6.67863593966969e-10\\
-33.7205087890625	6.17228800600587e-10\\
-33.700359375	5.88621602620691e-10\\
-33.6802099609375	6.06984866406849e-10\\
-33.660060546875	6.46101696851198e-10\\
-33.6399111328125	6.95373958173113e-10\\
-33.61976171875	7.12409231836329e-10\\
-33.5996123046875	7.77015164368383e-10\\
-33.579462890625	7.78760938959249e-10\\
-33.5593134765625	7.41210167769652e-10\\
-33.5391640625	7.05829499189214e-10\\
-33.5190146484375	6.87066076207728e-10\\
-33.498865234375	6.45228139644093e-10\\
-33.4787158203125	5.56083134104143e-10\\
-33.45856640625	6.21058312736806e-10\\
-33.4384169921875	6.10593640858765e-10\\
-33.418267578125	6.09938449119371e-10\\
-33.3981181640625	7.27255652957677e-10\\
-33.37796875	6.90127291604422e-10\\
-33.3578193359375	7.37076954226847e-10\\
-33.337669921875	7.76918087195517e-10\\
-33.3175205078125	7.09967531483223e-10\\
-33.29737109375	6.48321340589756e-10\\
-33.2772216796875	6.73765801499977e-10\\
-33.257072265625	5.68927381424858e-10\\
-33.2369228515625	5.83694507224617e-10\\
-33.2167734375	5.16168898181936e-10\\
-33.1966240234375	5.64959534415381e-10\\
-33.176474609375	5.28911430115763e-10\\
-33.1563251953125	5.55898140595314e-10\\
-33.13617578125	6.43039741949583e-10\\
-33.1160263671875	5.62203214234417e-10\\
-33.095876953125	5.2497744925814e-10\\
-33.0757275390625	6.18885049901935e-10\\
-33.055578125	4.96768904565229e-10\\
-33.0354287109375	4.23411086168401e-10\\
-33.015279296875	4.33943086472785e-10\\
-32.9951298828125	4.11764897710419e-10\\
-32.97498046875	4.51929151448416e-10\\
-32.9548310546875	4.51639294962729e-10\\
-32.934681640625	5.57504586309584e-10\\
-32.9145322265625	5.06240936278716e-10\\
-32.8943828125	5.14006737855009e-10\\
-32.8742333984375	4.96651355573715e-10\\
-32.854083984375	3.9696616903296e-10\\
-32.8339345703125	3.45709537090029e-10\\
-32.81378515625	3.53805224981308e-10\\
-32.7936357421875	1.89530032711433e-10\\
-32.773486328125	2.18637680400501e-10\\
-32.7533369140625	1.61007048595768e-10\\
-32.7331875	1.44799916231359e-10\\
-32.7130380859375	2.23946432351925e-10\\
-32.692888671875	2.07866092700248e-10\\
-32.6727392578125	2.75616278443632e-10\\
-32.65258984375	3.09399439558013e-10\\
-32.6324404296875	2.14702194638836e-10\\
-32.612291015625	1.86322132825259e-10\\
-32.5921416015625	1.36059164938141e-10\\
-32.5719921875	2.76779539846367e-11\\
-32.5518427734375	-3.88838259319611e-11\\
-32.531693359375	-3.08840170791109e-11\\
-32.5115439453125	-1.60711650538188e-10\\
-32.49139453125	-1.13105560063989e-10\\
-32.4712451171875	-4.86857355056706e-11\\
-32.451095703125	-7.84671164119504e-11\\
-32.4309462890625	-5.71100084929907e-11\\
-32.410796875	-9.7931062413366e-11\\
-32.3906474609375	-1.18936484107165e-10\\
-32.370498046875	-1.46035981532495e-10\\
-32.3503486328125	-1.85327845716646e-10\\
-32.33019921875	-1.86433193022185e-10\\
-32.3100498046875	-2.23520686705789e-10\\
-32.289900390625	-2.54166332844111e-10\\
-32.2697509765625	-3.17660906129197e-10\\
-32.2496015625	-3.52652597466332e-10\\
-32.2294521484375	-3.56343051578905e-10\\
-32.209302734375	-3.30774979067609e-10\\
-32.1891533203125	-4.07380536801674e-10\\
-32.16900390625	-3.43443269402258e-10\\
-32.1488544921875	-4.29196574957825e-10\\
-32.128705078125	-4.29649993020825e-10\\
-32.1085556640625	-4.33430600171442e-10\\
-32.08840625	-5.38424318814769e-10\\
-32.0682568359375	-4.91637243334262e-10\\
-32.048107421875	-5.16098043127941e-10\\
-32.0279580078125	-5.08347264229301e-10\\
-32.00780859375	-4.40335734330213e-10\\
-31.9876591796875	-5.02114335900297e-10\\
-31.967509765625	-4.92794493257257e-10\\
-31.9473603515625	-4.75920739447978e-10\\
-31.9272109375	-5.20720950463764e-10\\
-31.9070615234375	-6.03969825988775e-10\\
-31.886912109375	-5.6559890662825e-10\\
-31.8667626953125	-6.30846534933967e-10\\
-31.84661328125	-6.41784763338837e-10\\
-31.8264638671875	-5.40209031404193e-10\\
-31.806314453125	-6.0765269968728e-10\\
-31.7861650390625	-6.1465350805007e-10\\
-31.766015625	-6.47231373328406e-10\\
-31.7458662109375	-6.40313227937552e-10\\
-31.725716796875	-6.37210822297159e-10\\
-31.7055673828125	-6.95424689872047e-10\\
-31.68541796875	-7.00759125680395e-10\\
-31.6652685546875	-7.96294372829391e-10\\
-31.645119140625	-7.4003546731401e-10\\
-31.6249697265625	-6.96666514260915e-10\\
-31.6048203125	-7.27579278982012e-10\\
-31.5846708984375	-7.25748065359744e-10\\
-31.564521484375	-7.16236741082908e-10\\
-31.5443720703125	-7.91328151797169e-10\\
-31.52422265625	-8.22522116199243e-10\\
-31.5040732421875	-7.82119564083096e-10\\
-31.483923828125	-7.94628011161043e-10\\
-31.4637744140625	-7.65111652093523e-10\\
-31.443625	-6.92139921841939e-10\\
-31.4234755859375	-7.24079706002772e-10\\
-31.403326171875	-6.21063752024602e-10\\
-31.3831767578125	-7.37838833704541e-10\\
-31.36302734375	-6.89765231414774e-10\\
-31.3428779296875	-7.19007330410991e-10\\
-31.322728515625	-7.78576396095887e-10\\
-31.3025791015625	-7.9513365303004e-10\\
-31.2824296875	-8.80686407238901e-10\\
-31.2622802734375	-8.75309869854885e-10\\
-31.242130859375	-8.73745022710186e-10\\
-31.2219814453125	-8.26135458718705e-10\\
-31.20183203125	-7.10629100858646e-10\\
-31.1816826171875	-7.69304313294411e-10\\
-31.161533203125	-7.37615860725114e-10\\
-31.1413837890625	-7.34440736195477e-10\\
-31.121234375	-8.26067061733188e-10\\
-31.1010849609375	-9.09181184211258e-10\\
-31.080935546875	-8.98914643076416e-10\\
-31.0607861328125	-9.79238514956434e-10\\
-31.04063671875	-9.47533854975833e-10\\
-31.0204873046875	-9.22307984189087e-10\\
-31.000337890625	-8.32929947472171e-10\\
-30.9801884765625	-8.52649349141438e-10\\
-30.9600390625	-8.29958538043629e-10\\
-30.9398896484375	-8.08869778188192e-10\\
-30.919740234375	-8.20715594438939e-10\\
-30.8995908203125	-8.94589035220822e-10\\
-30.87944140625	-9.39227705796861e-10\\
-30.8592919921875	-1.08258341399366e-09\\
-30.839142578125	-1.04845425675398e-09\\
-30.8189931640625	-1.10576651363653e-09\\
-30.79884375	-1.09719892670101e-09\\
-30.7786943359375	-1.04664994190342e-09\\
-30.758544921875	-1.01532591373769e-09\\
-30.7383955078125	-8.7559123113101e-10\\
-30.71824609375	-9.33502005267488e-10\\
-30.6980966796875	-8.53570288778153e-10\\
-30.677947265625	-8.71736836294679e-10\\
-30.6577978515625	-8.37954944125762e-10\\
-30.6376484375	-9.43481480018474e-10\\
-30.6174990234375	-9.29965367770359e-10\\
-30.597349609375	-9.1729409351016e-10\\
-30.5772001953125	-8.84223968943065e-10\\
-30.55705078125	-8.39946942679565e-10\\
-30.5369013671875	-7.05230461076858e-10\\
-30.516751953125	-6.88275220375361e-10\\
-30.4966025390625	-7.22312548942399e-10\\
-30.476453125	-6.78724214632007e-10\\
-30.4563037109375	-5.96876173389396e-10\\
-30.436154296875	-6.02773403965726e-10\\
-30.4160048828125	-6.09934640532276e-10\\
-30.39585546875	-6.2301004181665e-10\\
-30.3757060546875	-6.51563326608471e-10\\
-30.355556640625	-5.74479684181941e-10\\
-30.3354072265625	-5.31449180421039e-10\\
-30.3152578125	-5.71646291034615e-10\\
-30.2951083984375	-4.33355909058951e-10\\
-30.274958984375	-4.27937022078791e-10\\
-30.2548095703125	-4.23842215793787e-10\\
-30.23466015625	-3.08453422730445e-10\\
-30.2145107421875	-4.46905885598805e-10\\
-30.194361328125	-4.13686906434258e-10\\
-30.1742119140625	-3.62782089766111e-10\\
-30.1540625	-3.26916399702423e-10\\
-30.1339130859375	-3.40576896799072e-10\\
-30.113763671875	-2.21975831980668e-10\\
-30.0936142578125	-2.38120649853214e-10\\
-30.07346484375	-2.08743879068482e-10\\
-30.0533154296875	-1.31505534934951e-10\\
-30.033166015625	-7.13120633154084e-11\\
-30.0130166015625	-5.14245081903101e-11\\
-29.9928671875	-8.75273005217232e-11\\
-29.9727177734375	-8.77041598534916e-11\\
-29.952568359375	-8.020399863212e-11\\
-29.9324189453125	-8.6326921513141e-11\\
-29.91226953125	-8.25925758710123e-11\\
-29.8921201171875	-7.82295959407536e-11\\
-29.871970703125	-5.24583220218841e-11\\
-29.8518212890625	1.286553499326e-12\\
-29.831671875	2.30424559605468e-11\\
-29.8115224609375	5.09998527224752e-11\\
-29.791373046875	5.03538237458951e-11\\
-29.7712236328125	1.70846062718908e-11\\
-29.75107421875	7.52317317491147e-11\\
-29.7309248046875	8.24049501541745e-11\\
-29.710775390625	5.22926385744651e-11\\
-29.6906259765625	1.44510184542837e-10\\
-29.6704765625	2.01195553353335e-10\\
-29.6503271484375	2.61636808462241e-10\\
-29.630177734375	3.0104589230239e-10\\
-29.6100283203125	2.80741131348938e-10\\
-29.58987890625	3.49385750817531e-10\\
-29.5697294921875	3.11921122652723e-10\\
-29.549580078125	2.46352963429888e-10\\
-29.5294306640625	2.46503724839975e-10\\
-29.50928125	1.5690832626037e-10\\
-29.4891318359375	2.47636446643861e-10\\
-29.468982421875	2.86910675076836e-10\\
-29.4488330078125	2.81044567342142e-10\\
-29.42868359375	3.35291572243942e-10\\
-29.4085341796875	4.15311021797601e-10\\
-29.388384765625	4.43472875314317e-10\\
-29.3682353515625	4.82979663674578e-10\\
-29.3480859375	5.53032313679154e-10\\
-29.3279365234375	4.85019741593348e-10\\
-29.307787109375	4.78074315512698e-10\\
-29.2876376953125	4.63835629552618e-10\\
-29.26748828125	5.30204528214694e-10\\
-29.2473388671875	5.47323076607027e-10\\
-29.227189453125	5.35820869218366e-10\\
-29.2070400390625	5.1406916305995e-10\\
-29.186890625	5.14760352608826e-10\\
-29.1667412109375	5.05745310170751e-10\\
-29.146591796875	6.28050593373857e-10\\
-29.1264423828125	6.49758565561987e-10\\
-29.10629296875	6.61357394289303e-10\\
-29.0861435546875	6.25832587611115e-10\\
-29.065994140625	6.80935303488384e-10\\
-29.0458447265625	6.15795831999358e-10\\
-29.0256953125	6.32026877168491e-10\\
-29.0055458984375	6.40213630356829e-10\\
-28.985396484375	6.10395810790866e-10\\
-28.9652470703125	6.46028866473367e-10\\
-28.94509765625	5.80932584283426e-10\\
-28.9249482421875	6.0735713281298e-10\\
-28.904798828125	6.31632099780142e-10\\
-28.8846494140625	6.18126155451346e-10\\
-28.8645	6.50594867742967e-10\\
-28.8443505859375	6.48666691443725e-10\\
-28.824201171875	6.70994731817419e-10\\
-28.8040517578125	7.0281314607472e-10\\
-28.78390234375	7.45002513027948e-10\\
-28.7637529296875	7.04358190125405e-10\\
-28.743603515625	6.60183538693489e-10\\
-28.7234541015625	6.27000642237445e-10\\
-28.7033046875	6.25775349961348e-10\\
-28.6831552734375	5.13889658242967e-10\\
-28.663005859375	5.89264848907228e-10\\
-28.6428564453125	4.52726597922866e-10\\
-28.62270703125	4.48318639499798e-10\\
-28.6025576171875	5.94095115826446e-10\\
-28.582408203125	5.48107207969749e-10\\
-28.5622587890625	6.63404415310843e-10\\
-28.542109375	7.06137730574603e-10\\
-28.5219599609375	6.56758038873424e-10\\
-28.501810546875	6.54597946242742e-10\\
-28.4816611328125	6.51126966664166e-10\\
-28.46151171875	7.17690158536946e-10\\
-28.4413623046875	7.03062084296037e-10\\
-28.421212890625	6.25492038235308e-10\\
-28.4010634765625	6.32425001545337e-10\\
-28.3809140625	5.63575523028053e-10\\
-28.3607646484375	5.97275252876149e-10\\
-28.340615234375	5.04501538811551e-10\\
-28.3204658203125	4.8548080271018e-10\\
-28.30031640625	4.56265729925698e-10\\
-28.2801669921875	4.46908797930112e-10\\
-28.260017578125	4.44803749659648e-10\\
-28.2398681640625	5.32740386547229e-10\\
-28.21971875	5.95705432763588e-10\\
-28.1995693359375	5.45921141738671e-10\\
-28.179419921875	5.27888879596925e-10\\
-28.1592705078125	4.66119496099315e-10\\
-28.13912109375	5.19104847840288e-10\\
-28.1189716796875	4.66825882545978e-10\\
-28.098822265625	4.12898515674715e-10\\
-28.0786728515625	3.0429280416688e-10\\
-28.0585234375	3.90444601414727e-10\\
-28.0383740234375	3.63625400354666e-10\\
-28.018224609375	3.56925563910419e-10\\
-27.9980751953125	3.49911192755136e-10\\
-27.97792578125	3.33021602053508e-10\\
-27.9577763671875	3.719858958451e-10\\
-27.937626953125	3.4695455136108e-10\\
-27.9174775390625	3.01082558542468e-10\\
-27.897328125	2.99736900532082e-10\\
-27.8771787109375	2.3312774208249e-10\\
-27.857029296875	1.72218198184387e-10\\
-27.8368798828125	2.26566463749944e-10\\
-27.81673046875	1.93854645209371e-10\\
-27.7965810546875	1.97697530551903e-10\\
-27.776431640625	1.05387554266354e-10\\
-27.7562822265625	1.69276999076046e-10\\
-27.7361328125	8.47971288484425e-11\\
-27.7159833984375	5.31083892281575e-11\\
-27.695833984375	9.6100826089725e-12\\
-27.6756845703125	-5.98199352094232e-11\\
-27.65553515625	-7.91286215171928e-11\\
-27.6353857421875	4.64629797085267e-12\\
-27.615236328125	-3.01726796139037e-11\\
-27.5950869140625	-1.57177896173676e-10\\
-27.5749375	-5.21671074368004e-11\\
-27.5547880859375	-7.13710350849734e-11\\
-27.534638671875	-4.86176619255877e-11\\
-27.5144892578125	-9.51060696331599e-11\\
-27.49433984375	-9.72849963669179e-11\\
-27.4741904296875	-2.33373935510791e-10\\
-27.454041015625	-3.23058018683507e-10\\
-27.4338916015625	-4.17292733165494e-10\\
-27.4137421875	-4.16756661077823e-10\\
-27.3935927734375	-3.76793929891587e-10\\
-27.373443359375	-3.59231018832744e-10\\
-27.3532939453125	-3.70505762169908e-10\\
-27.33314453125	-3.0021848305402e-10\\
-27.3129951171875	-2.62322571753922e-10\\
-27.292845703125	-2.88239620071753e-10\\
-27.2726962890625	-3.42459120939096e-10\\
-27.252546875	-4.06361203129097e-10\\
-27.2323974609375	-6.13938757094917e-10\\
-27.212248046875	-6.81745153657533e-10\\
-27.1920986328125	-7.80231141302627e-10\\
-27.17194921875	-7.53347021726505e-10\\
-27.1517998046875	-6.98905638296935e-10\\
-27.131650390625	-6.84160785202786e-10\\
-27.1115009765625	-5.26439796178945e-10\\
-27.0913515625	-4.59718946871801e-10\\
-27.0712021484375	-5.51550740160077e-10\\
-27.051052734375	-5.29163312656365e-10\\
-27.0309033203125	-6.17333894308348e-10\\
-27.01075390625	-7.21820378224868e-10\\
-26.9906044921875	-8.33624273814058e-10\\
-26.970455078125	-8.65297421064993e-10\\
-26.9503056640625	-8.96618536915733e-10\\
-26.93015625	-9.25801775836673e-10\\
-26.9100068359375	-9.26980124426054e-10\\
-26.889857421875	-8.18193553958544e-10\\
-26.8697080078125	-8.14530201985943e-10\\
-26.84955859375	-9.17813183461643e-10\\
-26.8294091796875	-7.9323024618637e-10\\
-26.809259765625	-8.61061692016806e-10\\
-26.7891103515625	-8.43864823794651e-10\\
-26.7689609375	-8.99359297343186e-10\\
-26.7488115234375	-9.55290686384182e-10\\
-26.728662109375	-1.00045290689921e-09\\
-26.7085126953125	-9.88461695763871e-10\\
-26.68836328125	-9.68809158289675e-10\\
-26.6682138671875	-9.8860688821792e-10\\
-26.648064453125	-9.87592340191046e-10\\
-26.6279150390625	-8.77435265518271e-10\\
-26.607765625	-8.87367887283068e-10\\
-26.5876162109375	-9.4938090323076e-10\\
-26.567466796875	-9.07221468514353e-10\\
-26.5473173828125	-9.81296443706973e-10\\
-26.52716796875	-9.42652410750864e-10\\
-26.5070185546875	-9.60049191278727e-10\\
-26.486869140625	-1.00320267419179e-09\\
-26.4667197265625	-9.5898440554817e-10\\
-26.4465703125	-9.86140935371648e-10\\
-26.4264208984375	-9.02484611604511e-10\\
-26.406271484375	-8.57935225389224e-10\\
-26.3861220703125	-8.34047481738166e-10\\
-26.36597265625	-7.64873066107819e-10\\
-26.3458232421875	-9.14153440912429e-10\\
-26.325673828125	-8.75252192596651e-10\\
-26.3055244140625	-9.37609362425539e-10\\
-26.285375	-1.02913672684707e-09\\
-26.2652255859375	-9.78549488530355e-10\\
-26.245076171875	-1.12161193670813e-09\\
-26.2249267578125	-1.03856492132381e-09\\
-26.20477734375	-9.02592648522906e-10\\
-26.1846279296875	-7.73809876637848e-10\\
-26.164478515625	-6.57514802324396e-10\\
-26.1443291015625	-6.18759056967378e-10\\
-26.1241796875	-5.78197759878716e-10\\
-26.1040302734375	-7.41850144872791e-10\\
-26.083880859375	-7.88801180583631e-10\\
-26.0637314453125	-7.99236274684805e-10\\
-26.04358203125	-9.8349768957892e-10\\
-26.0234326171875	-1.02263968871247e-09\\
-26.003283203125	-9.81558540384951e-10\\
-25.9831337890625	-9.30757012239513e-10\\
-25.962984375	-8.29023705478457e-10\\
-25.9428349609375	-6.35285325464256e-10\\
-25.922685546875	-5.8222248581312e-10\\
-25.9025361328125	-4.56144932808255e-10\\
-25.88238671875	-5.46952636423724e-10\\
-25.8622373046875	-4.75483859744923e-10\\
-25.842087890625	-5.7874856032367e-10\\
-25.8219384765625	-6.70487167054997e-10\\
-25.8017890625	-7.28414889517997e-10\\
-25.7816396484375	-8.58342890693521e-10\\
-25.761490234375	-8.46914415298277e-10\\
-25.7413408203125	-7.20976372038954e-10\\
-25.72119140625	-6.88230609079533e-10\\
-25.7010419921875	-5.10124372854822e-10\\
-25.680892578125	-4.49576800656374e-10\\
-25.6607431640625	-3.78862272715589e-10\\
-25.64059375	-4.04506620995385e-10\\
-25.6204443359375	-4.11703326585681e-10\\
-25.600294921875	-4.11228137532308e-10\\
-25.5801455078125	-4.85235396097136e-10\\
-25.55999609375	-5.72499646983868e-10\\
-25.5398466796875	-5.39697138178846e-10\\
-25.519697265625	-5.14296008330187e-10\\
-25.4995478515625	-4.30591206330103e-10\\
-25.4793984375	-3.4899294535569e-10\\
-25.4592490234375	-2.53897525919507e-10\\
-25.439099609375	-2.15640850072392e-10\\
-25.4189501953125	-2.42339296233497e-10\\
-25.39880078125	-2.55115719995551e-10\\
-25.3786513671875	-2.61820385414309e-10\\
-25.358501953125	-1.91601112573658e-10\\
-25.3383525390625	-3.27884595467914e-10\\
-25.318203125	-2.04612523239736e-10\\
-25.2980537109375	-2.01113754511404e-10\\
-25.277904296875	-1.16542296043218e-10\\
-25.2577548828125	-8.31909551350722e-11\\
-25.23760546875	-2.3316018055161e-11\\
-25.2174560546875	1.77867410602872e-11\\
-25.197306640625	9.6730440794428e-11\\
-25.1771572265625	1.30946704751215e-10\\
-25.1570078125	1.24116564900479e-10\\
-25.1368583984375	1.49528768344428e-10\\
-25.116708984375	2.00817850350528e-10\\
-25.0965595703125	2.41371829352218e-10\\
-25.07641015625	2.50586043364858e-10\\
-25.0562607421875	2.94481196799793e-10\\
-25.036111328125	3.81711190482501e-10\\
-25.0159619140625	4.54516460282757e-10\\
-24.9958125	4.3518767284731e-10\\
-24.9756630859375	5.16696427236782e-10\\
-24.955513671875	5.90400478563624e-10\\
-24.9353642578125	5.50539440723632e-10\\
-24.91521484375	6.87010432384422e-10\\
-24.8950654296875	6.25737491537863e-10\\
-24.874916015625	7.89635436409015e-10\\
-24.8547666015625	8.25715740160503e-10\\
-24.8346171875	8.14434253415454e-10\\
-24.8144677734375	8.10379136584346e-10\\
-24.794318359375	8.12840224845506e-10\\
-24.7741689453125	8.01221902580002e-10\\
-24.75401953125	6.98898460736056e-10\\
-24.7338701171875	7.87279415130256e-10\\
-24.713720703125	8.34890310853267e-10\\
-24.6935712890625	8.21733774666995e-10\\
-24.673421875	8.87558733014377e-10\\
-24.6532724609375	9.45336295387691e-10\\
-24.633123046875	9.90109597811151e-10\\
-24.6129736328125	9.81568445478875e-10\\
-24.59282421875	9.91073566019442e-10\\
-24.5726748046875	9.56818462844565e-10\\
-24.552525390625	9.98064138931443e-10\\
-24.5323759765625	1.00661451761898e-09\\
-24.5122265625	9.83980508080294e-10\\
-24.4920771484375	1.01794021062828e-09\\
-24.471927734375	9.67632280585002e-10\\
-24.4517783203125	1.06956247052656e-09\\
-24.43162890625	1.07880946120524e-09\\
-24.4114794921875	1.15876004384863e-09\\
-24.391330078125	1.17296539209481e-09\\
-24.3711806640625	1.23535266044511e-09\\
-24.35103125	1.37508892849322e-09\\
-24.3308818359375	1.31516092807291e-09\\
-24.310732421875	1.3247870366044e-09\\
-24.2905830078125	1.39441047843164e-09\\
-24.27043359375	1.35944209442327e-09\\
-24.2502841796875	1.38240152844611e-09\\
-24.230134765625	1.33603879469921e-09\\
-24.2099853515625	1.32314122303789e-09\\
-24.1898359375	1.30280297181578e-09\\
-24.1696865234375	1.3311537096882e-09\\
-24.149537109375	1.41069945992454e-09\\
-24.1293876953125	1.34942590363562e-09\\
-24.10923828125	1.46952983337411e-09\\
-24.0890888671875	1.57409793895687e-09\\
-24.068939453125	1.60953233593913e-09\\
-24.0487900390625	1.76285003508289e-09\\
-24.028640625	1.6739196274369e-09\\
-24.0084912109375	1.5864646859213e-09\\
-23.988341796875	1.59787329873927e-09\\
-23.9681923828125	1.45114819200424e-09\\
-23.94804296875	1.36987162443244e-09\\
-23.9278935546875	1.3503007316713e-09\\
-23.907744140625	1.31173322718765e-09\\
-23.8875947265625	1.2242035826842e-09\\
-23.8674453125	1.33849234295465e-09\\
-23.8472958984375	1.43958382851836e-09\\
-23.827146484375	1.41329394592586e-09\\
-23.8069970703125	1.46262646586612e-09\\
-23.78684765625	1.4654508598051e-09\\
-23.7666982421875	1.43724401831738e-09\\
-23.746548828125	1.42914459372655e-09\\
-23.7263994140625	1.47756068917275e-09\\
-23.70625	1.26364124763868e-09\\
-23.6861005859375	1.27440754234419e-09\\
-23.665951171875	1.09604735126109e-09\\
-23.6458017578125	1.05251837008782e-09\\
-23.62565234375	1.08852553517727e-09\\
-23.6055029296875	1.10069932522581e-09\\
-23.585353515625	1.12568781714916e-09\\
-23.5652041015625	1.24550858102218e-09\\
-23.5450546875	1.28936010107195e-09\\
-23.5249052734375	1.39316205813058e-09\\
-23.504755859375	1.31858954542594e-09\\
-23.4846064453125	1.1692070471065e-09\\
-23.46445703125	1.11566287594915e-09\\
-23.4443076171875	9.55012426421099e-10\\
-23.424158203125	1.0590307549243e-09\\
-23.4040087890625	9.00649698046819e-10\\
-23.383859375	1.07326581790567e-09\\
-23.3637099609375	1.16261504687099e-09\\
-23.343560546875	1.18623242436563e-09\\
-23.3234111328125	1.20149366378611e-09\\
-23.30326171875	1.28087840853967e-09\\
-23.2831123046875	1.20545055664576e-09\\
-23.262962890625	1.16294664706244e-09\\
-23.2428134765625	1.06315010715315e-09\\
-23.2226640625	9.1961509720711e-10\\
-23.2025146484375	9.22045916836768e-10\\
-23.182365234375	9.97461997316799e-10\\
-23.1622158203125	8.65527685088514e-10\\
-23.14206640625	1.02854855927996e-09\\
-23.1219169921875	1.04794882226588e-09\\
-23.101767578125	1.19147444911992e-09\\
-23.0816181640625	1.17891792131645e-09\\
-23.06146875	1.30973686522142e-09\\
-23.0413193359375	1.25616050021316e-09\\
-23.021169921875	1.23711161037825e-09\\
-23.0010205078125	1.0733606028326e-09\\
-22.98087109375	9.13486376222322e-10\\
-22.9607216796875	9.19174582282614e-10\\
-22.940572265625	7.75990151827987e-10\\
-22.9204228515625	8.63740407450987e-10\\
-22.9002734375	9.10711340720391e-10\\
-22.8801240234375	9.75901111417992e-10\\
-22.859974609375	1.05551646636686e-09\\
-22.8398251953125	9.54148392959217e-10\\
-22.81967578125	1.00012137616421e-09\\
-22.7995263671875	8.25102827967987e-10\\
-22.779376953125	7.98659862842245e-10\\
-22.7592275390625	5.92252857862991e-10\\
-22.739078125	5.3655375672352e-10\\
-22.7189287109375	5.04074848632875e-10\\
-22.698779296875	5.0578439996194e-10\\
-22.6786298828125	5.07598989177772e-10\\
-22.65848046875	5.43475986156089e-10\\
-22.6383310546875	4.53264493403909e-10\\
-22.618181640625	4.28801675915433e-10\\
-22.5980322265625	4.3092066523451e-10\\
-22.5778828125	2.65853972491097e-10\\
-22.5577333984375	1.50880549384367e-10\\
-22.537583984375	1.23125064236861e-10\\
-22.5174345703125	1.9453484508218e-11\\
-22.49728515625	3.46063271594175e-11\\
-22.4771357421875	9.63288405868209e-11\\
-22.456986328125	-2.7547852919308e-11\\
-22.4368369140625	1.23634084924692e-11\\
-22.4166875	-2.10892545626346e-11\\
-22.3965380859375	-3.42008619837429e-11\\
-22.376388671875	-5.39040670938846e-11\\
-22.3562392578125	-6.36355455766861e-11\\
-22.33608984375	-3.65683824161063e-11\\
-22.3159404296875	-7.96449651902297e-11\\
-22.295791015625	3.70842385929304e-12\\
-22.2756416015625	-9.21378232536823e-11\\
-22.2554921875	7.19424399551061e-11\\
-22.2353427734375	-4.19428518577039e-11\\
-22.215193359375	-4.16208378603984e-11\\
-22.1950439453125	-1.16690270007926e-10\\
-22.17489453125	-9.80931784094891e-11\\
-22.1547451171875	-1.50843449802301e-10\\
-22.134595703125	-1.70380606300871e-10\\
-22.1144462890625	-1.62283224831423e-10\\
-22.094296875	-3.13577837085103e-10\\
-22.0741474609375	-3.31370509153555e-10\\
-22.053998046875	-3.11179722391452e-10\\
-22.0338486328125	-4.07174976219238e-10\\
-22.01369921875	-3.15943237783098e-10\\
-21.9935498046875	-3.33303360945344e-10\\
-21.973400390625	-3.05144084729215e-10\\
-21.9532509765625	-2.99799270503006e-10\\
-21.9331015625	-3.46571822259933e-10\\
-21.9129521484375	-4.91688898816696e-10\\
-21.892802734375	-5.41499733820913e-10\\
-21.8726533203125	-6.84913499017016e-10\\
-21.85250390625	-8.01260585083849e-10\\
-21.8323544921875	-9.08690264642314e-10\\
-21.812205078125	-9.43738263020803e-10\\
-21.7920556640625	-1.00523163400898e-09\\
-21.77190625	-9.27899329528342e-10\\
-21.7517568359375	-8.58196485109931e-10\\
-21.731607421875	-7.87080887722236e-10\\
-21.7114580078125	-7.54409397573002e-10\\
-21.69130859375	-7.5542993801648e-10\\
-21.6711591796875	-7.62936303005442e-10\\
-21.651009765625	-9.82075710991384e-10\\
-21.6308603515625	-1.03545576761357e-09\\
-21.6107109375	-1.15382402547728e-09\\
-21.5905615234375	-1.19459326056356e-09\\
-21.570412109375	-1.18548193553969e-09\\
-21.5502626953125	-1.12147727533728e-09\\
-21.53011328125	-1.10450806440728e-09\\
-21.5099638671875	-9.92879730664281e-10\\
-21.489814453125	-9.48247983682542e-10\\
-21.4696650390625	-7.06533527177314e-10\\
-21.449515625	-7.13111333416811e-10\\
-21.4293662109375	-7.49948293563093e-10\\
-21.409216796875	-8.53953225966978e-10\\
-21.3890673828125	-9.24802223395835e-10\\
-21.36891796875	-9.80475423452244e-10\\
-21.3487685546875	-1.05048870362017e-09\\
-21.328619140625	-1.01728277383251e-09\\
-21.3084697265625	-1.19672935393595e-09\\
-21.2883203125	-1.08859182691526e-09\\
-21.2681708984375	-1.04615668043e-09\\
-21.248021484375	-1.05843467010326e-09\\
-21.2278720703125	-1.06140697875054e-09\\
-21.20772265625	-9.46078954196293e-10\\
-21.1875732421875	-1.02784305399485e-09\\
-21.167423828125	-9.55415331502625e-10\\
-21.1472744140625	-7.52528799931744e-10\\
-21.127125	-6.72435433762146e-10\\
-21.1069755859375	-7.83207618078484e-10\\
-21.086826171875	-7.52379456615419e-10\\
-21.0666767578125	-8.43854567716554e-10\\
-21.04652734375	-9.10692881564997e-10\\
-21.0263779296875	-1.00798988962674e-09\\
-21.006228515625	-1.11900823462262e-09\\
-20.9860791015625	-1.18252011954058e-09\\
-20.9659296875	-1.13974144166038e-09\\
-20.9457802734375	-1.03429949554205e-09\\
-20.925630859375	-9.66686247731513e-10\\
-20.9054814453125	-8.50565428326684e-10\\
-20.88533203125	-7.83745310627193e-10\\
-20.8651826171875	-7.18274780704071e-10\\
-20.845033203125	-5.30868225109189e-10\\
-20.8248837890625	-6.91592907853428e-10\\
-20.804734375	-5.76252262447607e-10\\
-20.7845849609375	-7.06109703192784e-10\\
-20.764435546875	-8.21827157792482e-10\\
-20.7442861328125	-9.10770350710301e-10\\
-20.72413671875	-9.66704393228107e-10\\
-20.7039873046875	-1.04707190047964e-09\\
-20.683837890625	-1.05915045730476e-09\\
-20.6636884765625	-1.00182726616007e-09\\
-20.6435390625	-9.73108001324406e-10\\
-20.6233896484375	-8.46128348892079e-10\\
-20.603240234375	-9.47488316615665e-10\\
-20.5830908203125	-7.53645268926811e-10\\
-20.56294140625	-6.86106301232083e-10\\
-20.5427919921875	-7.99605937707696e-10\\
-20.522642578125	-7.36086293163375e-10\\
-20.5024931640625	-7.27717496633869e-10\\
-20.48234375	-6.42407779290096e-10\\
-20.4621943359375	-7.36884027416652e-10\\
-20.442044921875	-6.92519620822899e-10\\
-20.4218955078125	-7.4324305142353e-10\\
-20.40174609375	-7.35317756433465e-10\\
-20.3815966796875	-6.17997559698559e-10\\
-20.361447265625	-6.92816962729517e-10\\
-20.3412978515625	-6.86230825722959e-10\\
-20.3211484375	-6.65708373990626e-10\\
-20.3009990234375	-4.8434214627379e-10\\
-20.280849609375	-5.17509326216498e-10\\
-20.2607001953125	-5.42465170025438e-10\\
-20.24055078125	-5.45097683690809e-10\\
-20.2204013671875	-4.90686180311539e-10\\
-20.200251953125	-5.81137104135074e-10\\
-20.1801025390625	-4.96373314933251e-10\\
-20.159953125	-6.1558337745802e-10\\
-20.1398037109375	-6.53655899622509e-10\\
-20.119654296875	-6.37700446905398e-10\\
-20.0995048828125	-5.75848901021198e-10\\
-20.07935546875	-4.66118317147112e-10\\
-20.0592060546875	-4.32216142446853e-10\\
-20.039056640625	-3.6152737241298e-10\\
-20.0189072265625	-2.88774028370514e-10\\
-19.9987578125	-2.10821663618144e-10\\
-19.9786083984375	-7.82042975138185e-11\\
-19.958458984375	-1.14904611443791e-11\\
-19.9383095703125	1.68592363395308e-12\\
-19.91816015625	2.67978969699912e-11\\
-19.8980107421875	1.2107141065005e-10\\
-19.877861328125	1.99819091065352e-10\\
-19.8577119140625	2.36770072657713e-10\\
-19.8375625	2.15593994629169e-10\\
-19.8174130859375	1.75741659999621e-10\\
-19.797263671875	1.29034071422785e-10\\
-19.7771142578125	1.29827791087039e-10\\
-19.75696484375	1.75046411445219e-10\\
-19.7368154296875	3.34421870696345e-10\\
-19.716666015625	2.99387584550757e-10\\
-19.6965166015625	4.14855451095434e-10\\
-19.6763671875	5.70376598456175e-10\\
-19.6562177734375	6.34043620228336e-10\\
-19.636068359375	7.69999927448405e-10\\
-19.6159189453125	6.84105799823144e-10\\
-19.59576953125	6.47637068275776e-10\\
-19.5756201171875	5.74427977813305e-10\\
-19.555470703125	4.91431954140971e-10\\
-19.5353212890625	4.83654384691234e-10\\
-19.515171875	5.00489063058726e-10\\
-19.4950224609375	3.87436008110955e-10\\
-19.474873046875	5.15167559633504e-10\\
-19.4547236328125	5.28525623677158e-10\\
-19.43457421875	6.16645942631987e-10\\
-19.4144248046875	7.97011516153748e-10\\
-19.394275390625	7.6716746785012e-10\\
-19.3741259765625	7.96004956110153e-10\\
-19.3539765625	7.52637349414876e-10\\
-19.3338271484375	4.75249389787891e-10\\
-19.313677734375	4.65189629886961e-10\\
-19.2935283203125	4.34061668085775e-10\\
-19.27337890625	2.74105623920711e-10\\
-19.2532294921875	4.07390566495882e-10\\
-19.233080078125	3.93889396006692e-10\\
-19.2129306640625	5.87551453225165e-10\\
-19.19278125	7.1881271913055e-10\\
-19.1726318359375	7.05917383152413e-10\\
-19.152482421875	7.25176525448925e-10\\
-19.1323330078125	7.69651957378599e-10\\
-19.11218359375	5.81146678656855e-10\\
-19.0920341796875	5.53213266906521e-10\\
-19.071884765625	5.34076183396614e-10\\
-19.0517353515625	5.86161531546212e-10\\
-19.0315859375	6.31862021533591e-10\\
-19.0114365234375	7.48592639252816e-10\\
-18.991287109375	9.65659198698888e-10\\
-18.9711376953125	9.14790229361026e-10\\
-18.95098828125	1.01755983288967e-09\\
-18.9308388671875	9.50682174732246e-10\\
-18.910689453125	9.31515344666561e-10\\
-18.8905400390625	8.17545921764882e-10\\
-18.870390625	7.36396887847227e-10\\
-18.8502412109375	7.19415444872813e-10\\
-18.830091796875	7.91041777244755e-10\\
-18.8099423828125	7.81375805977618e-10\\
-18.78979296875	9.61515153488358e-10\\
-18.7696435546875	9.722084312165e-10\\
-18.749494140625	1.11867070607195e-09\\
-18.7293447265625	1.1312888591405e-09\\
-18.7091953125	1.00836115537274e-09\\
-18.6890458984375	8.48444082639874e-10\\
-18.668896484375	7.91304884084884e-10\\
-18.6487470703125	6.29674188873176e-10\\
-18.62859765625	5.85438132978748e-10\\
-18.6084482421875	6.01221875537274e-10\\
-18.588298828125	6.5203463181818e-10\\
-18.5681494140625	8.4526936470161e-10\\
-18.548	1.01278326506147e-09\\
-18.5278505859375	1.15007831854233e-09\\
-18.507701171875	1.21209138425444e-09\\
-18.4875517578125	1.2871908029793e-09\\
-18.46740234375	1.09742630863867e-09\\
-18.4472529296875	1.09067773784957e-09\\
-18.427103515625	9.07046921933302e-10\\
-18.4069541015625	7.65322921174385e-10\\
-18.3868046875	6.88660334627423e-10\\
-18.3666552734375	6.18329494765135e-10\\
-18.346505859375	7.29426326673959e-10\\
-18.3263564453125	7.25551204483134e-10\\
-18.30620703125	8.38176002620391e-10\\
-18.2860576171875	9.89030600587022e-10\\
-18.265908203125	1.03813617137277e-09\\
-18.2457587890625	1.09170557354302e-09\\
-18.225609375	1.07760397609775e-09\\
-18.2054599609375	9.15306662448801e-10\\
-18.185310546875	8.50458017874562e-10\\
-18.1651611328125	6.92561105229468e-10\\
-18.14501171875	6.6744969869322e-10\\
-18.1248623046875	6.26352602332428e-10\\
-18.104712890625	7.12725932778312e-10\\
-18.0845634765625	8.27583705627044e-10\\
-18.0644140625	8.88364491817135e-10\\
-18.0442646484375	7.61385434672759e-10\\
-18.024115234375	9.17310090176172e-10\\
-18.0039658203125	8.21890877228509e-10\\
-17.98381640625	8.57629338043397e-10\\
-17.9636669921875	8.42437488532071e-10\\
-17.943517578125	9.58436153862441e-10\\
-17.9233681640625	7.75688551163589e-10\\
-17.90321875	7.52126460308735e-10\\
-17.8830693359375	7.38276962349764e-10\\
-17.862919921875	5.56913967897467e-10\\
-17.8427705078125	5.99018096160858e-10\\
-17.82262109375	6.57987325496031e-10\\
-17.8024716796875	5.21441093848825e-10\\
-17.782322265625	6.00047325936706e-10\\
-17.7621728515625	5.99475312965157e-10\\
-17.7420234375	5.78729147326128e-10\\
-17.7218740234375	6.26708174782653e-10\\
-17.701724609375	5.61697646218499e-10\\
-17.6815751953125	5.65800121090067e-10\\
-17.66142578125	4.94883776499911e-10\\
-17.6412763671875	4.09746807438063e-10\\
-17.621126953125	3.66994494084762e-10\\
-17.6009775390625	3.86843427739059e-10\\
-17.580828125	3.78016467509036e-10\\
-17.5606787109375	3.91069996017961e-10\\
-17.540529296875	3.40948429574364e-10\\
-17.5203798828125	4.36394954770892e-10\\
-17.50023046875	3.76810466162761e-10\\
-17.4800810546875	3.3600308642791e-10\\
-17.459931640625	2.1885261762312e-10\\
-17.4397822265625	1.52851506993026e-10\\
-17.4196328125	7.25961418086041e-11\\
-17.3994833984375	8.72925358657768e-11\\
-17.379333984375	4.29052014194948e-11\\
-17.3591845703125	-1.13799714538757e-10\\
-17.33903515625	-3.83274200150091e-12\\
-17.3188857421875	-1.05873851042815e-10\\
-17.298736328125	-6.9687943480993e-11\\
-17.2785869140625	-1.45902562598771e-10\\
-17.2584375	-1.60888271442502e-10\\
-17.2382880859375	-2.89324421150407e-10\\
-17.218138671875	-2.11115410077386e-10\\
-17.1979892578125	-2.24111702218596e-10\\
-17.17783984375	-1.54703996691278e-10\\
-17.1576904296875	-2.21547163745723e-10\\
-17.137541015625	-2.57275189256513e-10\\
-17.1173916015625	-2.26816405564126e-10\\
-17.0972421875	-1.85043584811662e-10\\
-17.0770927734375	-3.13491866287683e-10\\
-17.056943359375	-2.81176404702819e-10\\
-17.0367939453125	-4.13410387937241e-10\\
-17.01664453125	-3.03256912265721e-10\\
-16.9964951171875	-2.6379067364639e-10\\
-16.976345703125	-2.800361529706e-10\\
-16.9561962890625	-2.77620909375217e-10\\
-16.936046875	-2.24949159125089e-10\\
-16.9158974609375	-3.07963600626119e-10\\
-16.895748046875	-3.4133854351256e-10\\
-16.8755986328125	-3.80046925679888e-10\\
-16.85544921875	-4.47042447342692e-10\\
-16.8352998046875	-4.82017766286861e-10\\
-16.815150390625	-5.12842326497603e-10\\
-16.7950009765625	-4.63999069965389e-10\\
-16.7748515625	-4.36485357968663e-10\\
-16.7547021484375	-3.82935664250822e-10\\
-16.734552734375	-4.61902499536123e-10\\
-16.7144033203125	-3.5456441813964e-10\\
-16.69425390625	-3.96889952665076e-10\\
-16.6741044921875	-3.61723833697815e-10\\
-16.653955078125	-5.48777867651687e-10\\
-16.6338056640625	-5.28893661944472e-10\\
-16.61365625	-6.59730311186978e-10\\
-16.5935068359375	-6.00810938120683e-10\\
-16.573357421875	-6.65477021066714e-10\\
-16.5532080078125	-5.84136030080054e-10\\
-16.53305859375	-5.714142551447e-10\\
-16.5129091796875	-5.42720523478931e-10\\
-16.492759765625	-6.55454794639933e-10\\
-16.4726103515625	-6.18819448601885e-10\\
-16.4524609375	-7.3210519567489e-10\\
-16.4323115234375	-7.43902364822919e-10\\
-16.412162109375	-9.15402163756514e-10\\
-16.3920126953125	-9.06205774327059e-10\\
-16.37186328125	-7.22186986630333e-10\\
-16.3517138671875	-6.20867709078956e-10\\
-16.331564453125	-6.73457062243025e-10\\
-16.3114150390625	-5.05787027672192e-10\\
-16.291265625	-5.24629059682862e-10\\
-16.2711162109375	-5.53707552064656e-10\\
-16.250966796875	-4.90809219624623e-10\\
-16.2308173828125	-5.49839194545115e-10\\
-16.21066796875	-6.38846436651802e-10\\
-16.1905185546875	-7.09365056677997e-10\\
-16.170369140625	-7.12886400404884e-10\\
-16.1502197265625	-7.12220641990121e-10\\
-16.1300703125	-6.14713018529116e-10\\
-16.1099208984375	-6.01512537279672e-10\\
-16.089771484375	-3.44506296542195e-10\\
-16.0696220703125	-2.75296627909443e-10\\
-16.04947265625	-2.93296465369593e-10\\
-16.0293232421875	-2.9995265425655e-10\\
-16.009173828125	-3.65252350599234e-10\\
-15.9890244140625	-4.34744804214561e-10\\
-15.968875	-6.88414040351566e-10\\
-15.9487255859375	-7.03986549499021e-10\\
-15.928576171875	-6.98011279267381e-10\\
-15.9084267578125	-5.54240732010963e-10\\
-15.88827734375	-6.448152453399e-10\\
-15.8681279296875	-5.41011999963936e-10\\
-15.847978515625	-5.34899580184689e-10\\
-15.8278291015625	-5.40055726429779e-10\\
-15.8076796875	-4.69812263667946e-10\\
-15.7875302734375	-5.54888205735416e-10\\
-15.767380859375	-5.03889597261952e-10\\
-15.7472314453125	-5.86860064570087e-10\\
-15.72708203125	-5.35213972410584e-10\\
-15.7069326171875	-5.91880726179422e-10\\
-15.686783203125	-5.13339853232414e-10\\
-15.6666337890625	-6.18986349216106e-10\\
-15.646484375	-5.60733207147842e-10\\
-15.6263349609375	-5.50096923159647e-10\\
-15.606185546875	-5.15431370441876e-10\\
-15.5860361328125	-5.2174513032015e-10\\
-15.56588671875	-4.44707871977572e-10\\
-15.5457373046875	-3.2839517521744e-10\\
-15.525587890625	-3.85726505454452e-10\\
-15.5054384765625	-3.65374599432106e-10\\
-15.4852890625	-3.43210381153522e-10\\
-15.4651396484375	-3.78080022567299e-10\\
-15.444990234375	-4.09208726172903e-10\\
-15.4248408203125	-3.15971987808508e-10\\
-15.40469140625	-3.19093059351204e-10\\
-15.3845419921875	-3.18580466661762e-10\\
-15.364392578125	-3.92497472808438e-10\\
-15.3442431640625	-2.81225683158226e-10\\
-15.32409375	-4.11416421416792e-10\\
-15.3039443359375	-2.27418557389332e-10\\
-15.283794921875	-2.33153470291281e-10\\
-15.2636455078125	-1.78513308985213e-10\\
-15.24349609375	-4.77689980397013e-11\\
-15.2233466796875	-8.66891278412442e-12\\
-15.203197265625	-7.76499228314291e-12\\
-15.1830478515625	-5.21111368290403e-12\\
-15.1628984375	-1.60990295386848e-11\\
-15.1427490234375	-1.3812078151613e-11\\
-15.122599609375	-6.31975264528482e-13\\
-15.1024501953125	-1.54172184665485e-10\\
-15.08230078125	-1.08733166686743e-11\\
-15.0621513671875	-4.66628681263867e-12\\
-15.042001953125	5.11599185450558e-11\\
-15.0218525390625	2.67938094337589e-11\\
-15.001703125	1.38633686895177e-10\\
-14.9815537109375	1.61148620462202e-10\\
-14.961404296875	1.58398090324937e-10\\
-14.9412548828125	1.8575680441025e-10\\
-14.92110546875	1.41442022715862e-10\\
-14.9009560546875	1.40865594253763e-10\\
-14.880806640625	1.5952891175747e-10\\
-14.8606572265625	1.11783668077903e-10\\
-14.8405078125	1.25445173715065e-10\\
-14.8203583984375	4.73124399632025e-11\\
-14.800208984375	1.03586284718181e-10\\
-14.7800595703125	1.92889221750891e-10\\
-14.75991015625	2.27162925588057e-10\\
-14.7397607421875	3.33990932458079e-10\\
-14.719611328125	4.23059081527282e-10\\
-14.6994619140625	3.85187781159673e-10\\
-14.6793125	4.5810825743376e-10\\
-14.6591630859375	4.60358903237752e-10\\
-14.639013671875	4.42875129046472e-10\\
-14.6188642578125	5.21767507966805e-10\\
-14.59871484375	4.85869406128447e-10\\
-14.5785654296875	4.60873213263577e-10\\
-14.558416015625	5.55901228220738e-10\\
-14.5382666015625	5.68111733433527e-10\\
-14.5181171875	6.74678632627018e-10\\
-14.4979677734375	6.77992341915501e-10\\
-14.477818359375	7.66281321818208e-10\\
-14.4576689453125	7.61532867242579e-10\\
-14.43751953125	6.98406287625422e-10\\
-14.4173701171875	7.71607158423309e-10\\
-14.397220703125	7.56580788255177e-10\\
-14.3770712890625	7.20322664954625e-10\\
-14.356921875	6.81207841570948e-10\\
-14.3367724609375	8.72500215187082e-10\\
-14.316623046875	9.10245866034927e-10\\
-14.2964736328125	1.11463679290826e-09\\
-14.27632421875	1.0702608278375e-09\\
-14.2561748046875	1.23326071433992e-09\\
-14.236025390625	1.24749063604681e-09\\
-14.2158759765625	1.26373189345477e-09\\
-14.1957265625	1.20175536501298e-09\\
-14.1755771484375	1.06884728440036e-09\\
-14.155427734375	1.07720738855107e-09\\
-14.1352783203125	9.79577145033131e-10\\
-14.11512890625	1.06315317173588e-09\\
-14.0949794921875	1.01997407425041e-09\\
-14.074830078125	1.14317863399494e-09\\
-14.0546806640625	1.20070472046916e-09\\
-14.03453125	1.06293706888909e-09\\
-14.0143818359375	1.17688708133971e-09\\
-13.994232421875	1.22870542819658e-09\\
-13.9740830078125	1.14115537752392e-09\\
-13.95393359375	1.18616478927808e-09\\
-13.9337841796875	1.05219695177189e-09\\
-13.913634765625	1.17790356491472e-09\\
-13.8934853515625	1.14099343992265e-09\\
-13.8733359375	1.12171759506363e-09\\
-13.8531865234375	1.18938728676722e-09\\
-13.833037109375	1.12507677099183e-09\\
-13.8128876953125	1.10025403714044e-09\\
-13.79273828125	1.12475815682923e-09\\
-13.7725888671875	1.07860690504649e-09\\
-13.752439453125	1.16677791293742e-09\\
-13.7322900390625	1.17976104766477e-09\\
-13.712140625	1.11992888948251e-09\\
-13.6919912109375	1.23478145524671e-09\\
-13.671841796875	1.11676096824683e-09\\
-13.6516923828125	1.0842773506647e-09\\
-13.63154296875	1.04445826136727e-09\\
-13.6113935546875	9.82224775317274e-10\\
-13.591244140625	1.04566391696837e-09\\
-13.5710947265625	1.04917026515801e-09\\
-13.5509453125	1.12749022945847e-09\\
-13.5307958984375	9.63355408224271e-10\\
-13.510646484375	9.08500210482478e-10\\
-13.4904970703125	9.27321578272646e-10\\
-13.47034765625	9.58929398759467e-10\\
-13.4501982421875	1.06037243894235e-09\\
-13.430048828125	9.96445920069245e-10\\
-13.4098994140625	1.08314802275566e-09\\
-13.38975	1.153355177002e-09\\
-13.3696005859375	1.07677265693955e-09\\
-13.349451171875	1.05013167362605e-09\\
-13.3293017578125	1.11266910655212e-09\\
-13.30915234375	1.07487813977971e-09\\
-13.2890029296875	1.01993680058186e-09\\
-13.268853515625	1.09150293469238e-09\\
-13.2487041015625	1.05519506698989e-09\\
-13.2285546875	1.2828764577506e-09\\
-13.2084052734375	1.19084976992117e-09\\
-13.188255859375	1.31281520839583e-09\\
-13.1681064453125	1.14996825597176e-09\\
-13.14795703125	1.07605732931184e-09\\
-13.1278076171875	9.51410112327252e-10\\
-13.107658203125	9.73055479837677e-10\\
-13.0875087890625	8.4916185823506e-10\\
-13.067359375	9.27408296484715e-10\\
-13.0472099609375	9.02470931324225e-10\\
-13.027060546875	9.52718305066237e-10\\
-13.0069111328125	9.62973161218931e-10\\
-12.98676171875	8.7883180152196e-10\\
-12.9666123046875	9.05462410178405e-10\\
-12.946462890625	8.73781370181132e-10\\
-12.9263134765625	8.82413984967221e-10\\
-12.9061640625	8.92312940882179e-10\\
-12.8860146484375	8.82051175270995e-10\\
-12.865865234375	8.83949757934311e-10\\
-12.8457158203125	7.79978141621218e-10\\
-12.82556640625	7.79394718585604e-10\\
-12.8054169921875	8.2478327026993e-10\\
-12.785267578125	6.47089163942566e-10\\
-12.7651181640625	7.47366161667601e-10\\
-12.74496875	7.11444523727777e-10\\
-12.7248193359375	5.29051010653091e-10\\
-12.704669921875	6.75784426681767e-10\\
-12.6845205078125	6.36403661740751e-10\\
-12.66437109375	6.32805843521247e-10\\
-12.6442216796875	5.69556569319569e-10\\
-12.624072265625	5.01404365072065e-10\\
-12.6039228515625	5.59981287301586e-10\\
-12.5837734375	3.95704241147941e-10\\
-12.5636240234375	3.82143136121511e-10\\
-12.543474609375	4.89768923934767e-10\\
-12.5233251953125	3.71600593617065e-10\\
-12.50317578125	4.34733658653983e-10\\
-12.4830263671875	4.49207517057734e-10\\
-12.462876953125	4.46507590369098e-10\\
-12.4427275390625	3.7904752439527e-10\\
-12.422578125	2.83634656975625e-10\\
-12.4024287109375	2.35126851533883e-10\\
-12.382279296875	2.93454894205371e-10\\
-12.3621298828125	2.45037805674523e-10\\
-12.34198046875	3.39000352220311e-10\\
-12.3218310546875	3.34112123655344e-10\\
-12.301681640625	2.95334374395844e-10\\
-12.2815322265625	2.86268997686516e-10\\
-12.2613828125	2.92944295440549e-10\\
-12.2412333984375	2.91173257881554e-10\\
-12.221083984375	7.64662733770234e-11\\
-12.2009345703125	1.12441690904708e-10\\
-12.18078515625	-2.61020462700394e-12\\
-12.1606357421875	-6.80491280008393e-11\\
-12.140486328125	6.33700625627646e-11\\
-12.1203369140625	1.86530430856289e-10\\
-12.1001875	2.03484575728467e-11\\
-12.0800380859375	1.3383049770034e-10\\
-12.059888671875	4.17885556674694e-11\\
-12.0397392578125	2.81498653159361e-11\\
-12.01958984375	-1.34130885232248e-10\\
-11.9994404296875	-1.99052988619967e-10\\
-11.979291015625	-2.61356801203416e-10\\
-11.9591416015625	-3.41334214424187e-10\\
-11.9389921875	-4.47594408834142e-10\\
-11.9188427734375	-2.18023154967928e-10\\
-11.898693359375	-2.09566070040972e-10\\
-11.8785439453125	-1.37148905799281e-10\\
-11.85839453125	1.45464783921906e-11\\
-11.8382451171875	-8.85001450448442e-12\\
-11.818095703125	9.67529249795233e-12\\
-11.7979462890625	-2.92852141583826e-11\\
-11.777796875	-1.76863029380192e-10\\
-11.7576474609375	-3.51139175030918e-10\\
-11.737498046875	-4.95304965625076e-10\\
-11.7173486328125	-4.97265255217547e-10\\
-11.69719921875	-6.65578578387949e-10\\
-11.6770498046875	-4.60982903647499e-10\\
-11.656900390625	-4.862638076167e-10\\
-11.6367509765625	-4.11428284267804e-10\\
-11.6166015625	-3.46459176146202e-10\\
-11.5964521484375	-3.46190157516604e-10\\
-11.576302734375	-3.80055766952206e-10\\
-11.5561533203125	-4.77203109045716e-10\\
-11.53600390625	-6.03790610637178e-10\\
-11.5158544921875	-6.53098482236709e-10\\
-11.495705078125	-6.81564706440048e-10\\
-11.4755556640625	-6.43205004432352e-10\\
-11.45540625	-6.72980167955282e-10\\
-11.4352568359375	-6.92922028665436e-10\\
-11.415107421875	-5.89660389175218e-10\\
-11.3949580078125	-5.71597850136477e-10\\
-11.37480859375	-4.46338530427105e-10\\
-11.3546591796875	-5.66061081872931e-10\\
-11.334509765625	-5.29317760345403e-10\\
-11.3143603515625	-6.85639796502379e-10\\
-11.2942109375	-6.94358266661006e-10\\
-11.2740615234375	-6.76137908922367e-10\\
-11.253912109375	-7.15840538169536e-10\\
-11.2337626953125	-8.24184677159887e-10\\
-11.21361328125	-5.99836072257856e-10\\
-11.1934638671875	-6.76044967153043e-10\\
-11.173314453125	-5.86378270666073e-10\\
-11.1531650390625	-4.89247271629298e-10\\
-11.133015625	-6.04918063160689e-10\\
-11.1128662109375	-6.56986293646306e-10\\
-11.092716796875	-6.45001773641269e-10\\
-11.0725673828125	-7.26538783657787e-10\\
-11.05241796875	-7.42112146725898e-10\\
-11.0322685546875	-8.0398384217232e-10\\
-11.012119140625	-7.96354674589061e-10\\
-10.9919697265625	-7.86830969790149e-10\\
-10.9718203125	-6.97272160822857e-10\\
-10.9516708984375	-5.9437141469284e-10\\
-10.931521484375	-4.43015187045578e-10\\
-10.9113720703125	-5.15122045740802e-10\\
-10.89122265625	-5.57306318563766e-10\\
-10.8710732421875	-5.516879969021e-10\\
-10.850923828125	-5.94027513317723e-10\\
-10.8307744140625	-6.86394885426429e-10\\
-10.810625	-7.22622693163223e-10\\
-10.7904755859375	-7.31160716427614e-10\\
-10.770326171875	-6.74615338660852e-10\\
-10.7501767578125	-6.59196823480168e-10\\
-10.73002734375	-4.56652521211461e-10\\
-10.7098779296875	-3.92526869266509e-10\\
-10.689728515625	-3.33119092811792e-10\\
-10.6695791015625	-5.20770310966973e-10\\
-10.6494296875	-4.74848742580801e-10\\
-10.6292802734375	-5.81605871246938e-10\\
-10.609130859375	-5.54135993012654e-10\\
-10.5889814453125	-6.8090750229731e-10\\
-10.56883203125	-5.83184230751104e-10\\
-10.5486826171875	-6.31511139540817e-10\\
-10.528533203125	-4.99775098684614e-10\\
-10.5083837890625	-3.13472476581788e-10\\
-10.488234375	-3.17181329700021e-10\\
-10.4680849609375	-1.6943311306286e-10\\
-10.447935546875	-3.26226962779017e-10\\
-10.4277861328125	-2.30877597125394e-10\\
-10.40763671875	-3.23724271820756e-10\\
-10.3874873046875	-3.9800018304559e-10\\
-10.367337890625	-4.53939640432401e-10\\
-10.3471884765625	-4.23955777551463e-10\\
-10.3270390625	-4.61754733662952e-10\\
-10.3068896484375	-4.74060828236369e-10\\
-10.286740234375	-3.70013476198259e-10\\
-10.2665908203125	-2.73898132295191e-10\\
-10.24644140625	-3.15382136941566e-10\\
-10.2262919921875	-2.20711415250284e-10\\
-10.206142578125	-1.8387079595673e-10\\
-10.1859931640625	-1.54363479830808e-10\\
-10.16584375	-1.17401299414987e-10\\
-10.1456943359375	-1.55828327671514e-10\\
-10.125544921875	-1.9708464196356e-10\\
-10.1053955078125	-2.94007488223453e-10\\
-10.08524609375	-2.54421949400691e-10\\
-10.0650966796875	-1.1104084685014e-10\\
-10.044947265625	-1.80089870489337e-10\\
-10.0247978515625	-1.24707743722275e-10\\
-10.0046484375	-9.83437596177126e-11\\
-9.98449902343749	-8.22794291446418e-11\\
-9.964349609375	-9.86001434252439e-11\\
-9.9442001953125	-8.48896027064034e-11\\
-9.92405078125	-3.48294719168156e-11\\
-9.90390136718749	-3.9976372071443e-11\\
-9.88375195312499	1.12154841952906e-10\\
-9.8636025390625	9.99732103743118e-11\\
-9.843453125	8.93147181480957e-11\\
-9.82330371093749	3.4772254926957e-11\\
-9.80315429687499	4.98138596189295e-11\\
-9.7830048828125	6.72501433794137e-11\\
-9.76285546875	7.5795754109002e-12\\
-9.7427060546875	2.18128728388151e-12\\
-9.72255664062499	4.80736111083313e-11\\
-9.70240722656249	3.02669583166979e-11\\
-9.6822578125	1.19704589961742e-10\\
-9.6621083984375	1.8119062759371e-10\\
-9.64195898437499	3.14130262601067e-10\\
-9.62180957031249	2.44436871551798e-10\\
-9.60166015625	3.11141655427589e-10\\
-9.5815107421875	3.26672760410382e-10\\
-9.561361328125	1.35807988931693e-10\\
-9.54121191406249	3.14512878050203e-10\\
-9.52106249999999	3.95884529323539e-10\\
-9.5009130859375	3.60031833799195e-10\\
-9.480763671875	4.78447172970301e-10\\
-9.46061425781249	4.26898780789514e-10\\
-9.44046484374999	4.04391923586848e-10\\
-9.4203154296875	3.27134026292767e-10\\
-9.400166015625	2.64589550530608e-10\\
-9.3800166015625	2.9113572684085e-10\\
-9.35986718749999	2.44008148990884e-10\\
-9.33971777343749	3.0764242494189e-10\\
-9.319568359375	2.86951074649468e-10\\
-9.2994189453125	3.56356170941486e-10\\
-9.27926953124999	5.24673833927329e-10\\
-9.25912011718749	5.89439797820934e-10\\
-9.238970703125	6.41010741294612e-10\\
-9.2188212890625	6.67546232528137e-10\\
-9.198671875	6.92944623760749e-10\\
-9.17852246093749	7.26524672915758e-10\\
-9.158373046875	7.79912543982399e-10\\
-9.1382236328125	7.0294537513355e-10\\
-9.11807421875	7.42607625343884e-10\\
-9.09792480468749	6.88707943881208e-10\\
-9.07777539062499	8.5473423626485e-10\\
-9.0576259765625	9.25412980126355e-10\\
-9.0374765625	9.99391227257909e-10\\
-9.0173271484375	9.62790064711417e-10\\
-8.99717773437499	9.48738514538543e-10\\
-8.9770283203125	9.50490344165498e-10\\
-8.95687890625	8.94183487875994e-10\\
-8.9367294921875	8.99145484738123e-10\\
-8.91658007812499	9.66340667772964e-10\\
-8.89643066406249	8.59091653743201e-10\\
-8.87628125	8.5520320771012e-10\\
-8.8561318359375	8.72512573046723e-10\\
-8.83598242187499	9.00489748047973e-10\\
-8.81583300781249	8.72182891826032e-10\\
-8.79568359375	9.34787449080135e-10\\
-8.7755341796875	9.46183012455672e-10\\
-8.755384765625	9.01792649444217e-10\\
-8.73523535156249	1.07799213298324e-09\\
-8.71508593749999	1.04916037136524e-09\\
-8.6949365234375	1.05692371745196e-09\\
-8.674787109375	1.10575334717128e-09\\
-8.65463769531249	9.85790139754341e-10\\
-8.63448828124999	1.02013171371162e-09\\
-8.6143388671875	9.73006834816585e-10\\
-8.594189453125	9.33696350297527e-10\\
-8.5740400390625	9.61301835469911e-10\\
-8.55389062499999	9.21787674980536e-10\\
-8.53374121093749	8.2188512614158e-10\\
-8.513591796875	8.93824818476319e-10\\
-8.4934423828125	8.50273773727962e-10\\
-8.47329296874999	8.35943025959801e-10\\
-8.45314355468749	7.68095993065882e-10\\
-8.432994140625	7.69697355691217e-10\\
-8.4128447265625	8.19713366300925e-10\\
-8.3926953125	8.32158599443722e-10\\
-8.37254589843749	8.18371269066642e-10\\
-8.35239648437499	8.07665126871417e-10\\
-8.3322470703125	7.64280415101491e-10\\
-8.31209765625	7.96451992999045e-10\\
-8.29194824218749	8.39286284795262e-10\\
-8.27179882812499	9.66617023041718e-10\\
-8.2516494140625	1.0332041638171e-09\\
-8.2315	1.01238908185286e-09\\
-8.2113505859375	8.50682056020502e-10\\
-8.19120117187499	9.32159240993766e-10\\
-8.17105175781249	8.76441736725922e-10\\
-8.15090234375	8.66912920347509e-10\\
-8.1307529296875	8.80940617860734e-10\\
-8.11060351562499	8.17692094464867e-10\\
-8.09045410156249	8.52795121325842e-10\\
-8.0703046875	7.44898262465349e-10\\
-8.0501552734375	7.68571131711658e-10\\
-8.030005859375	8.61494346620343e-10\\
-8.00985644531249	7.60171676582582e-10\\
-7.98970703125	8.10981940150797e-10\\
-7.9695576171875	8.03684089956262e-10\\
-7.949408203125	6.23989875413703e-10\\
-7.92925878906249	6.38682040774337e-10\\
-7.90910937499999	6.6442672190532e-10\\
-7.8889599609375	6.41340331093249e-10\\
-7.868810546875	5.94448688776608e-10\\
-7.8486611328125	5.37449463270686e-10\\
-7.82851171874999	6.01962469990305e-10\\
-7.8083623046875	6.30813729078789e-10\\
-7.788212890625	5.80688853292416e-10\\
-7.7680634765625	5.28015651441822e-10\\
-7.74791406249999	4.85956990118226e-10\\
-7.72776464843749	5.32555095327966e-10\\
-7.707615234375	4.45198250162567e-10\\
-7.6874658203125	5.39974684158549e-10\\
-7.66731640625	5.43249771326547e-10\\
-7.64716699218749	4.22027806392934e-10\\
-7.627017578125	4.07399769932351e-10\\
-7.6068681640625	3.67186532016555e-10\\
-7.58671875	3.39556444040001e-10\\
-7.56656933593749	1.65390069804197e-10\\
-7.54641992187499	9.66913845426165e-11\\
-7.5262705078125	2.0226621838681e-10\\
-7.50612109375	1.00138428586771e-10\\
-7.48597167968749	2.04078277775314e-10\\
-7.46582226562499	1.40910288383059e-10\\
-7.4456728515625	2.58678616974195e-10\\
-7.4255234375	2.02383190138728e-10\\
-7.4053740234375	3.12016764347642e-10\\
-7.38522460937499	5.46553150754694e-11\\
-7.36507519531249	5.93141893923854e-11\\
-7.34492578125	-1.14658016482016e-11\\
-7.3247763671875	-7.0818419797189e-11\\
-7.30462695312499	-6.17395772591515e-11\\
-7.28447753906249	-9.97113756789089e-12\\
-7.264328125	-4.35745870574009e-11\\
-7.2441787109375	9.57632943592492e-11\\
-7.224029296875	-6.53874307979932e-11\\
-7.20387988281249	-2.67445338359765e-12\\
-7.18373046874999	-2.40949570328146e-11\\
-7.1635810546875	-2.19453638599421e-10\\
-7.143431640625	-2.52343696589552e-10\\
-7.12328222656249	-3.32011396325573e-10\\
-7.10313281249999	-2.53729912089102e-10\\
-7.0829833984375	-3.03287511443893e-10\\
-7.062833984375	-1.48205572263451e-10\\
-7.0426845703125	-4.2872457391014e-11\\
-7.02253515624999	-1.07245988622148e-10\\
-7.00238574218749	-5.2799748046187e-11\\
-6.982236328125	-2.66527568662095e-11\\
-6.9620869140625	-1.08275045929261e-10\\
-6.94193749999999	-1.45096684161521e-10\\
-6.92178808593749	-2.85842839288645e-10\\
-6.901638671875	-3.10777147808849e-10\\
-6.8814892578125	-3.73665964342089e-10\\
-6.86133984375	-3.38059475207244e-10\\
-6.84119042968749	-3.58954900986898e-10\\
-6.821041015625	-2.07613148717395e-10\\
-6.8008916015625	-3.10528012810539e-10\\
-6.7807421875	-1.02014638972414e-10\\
-6.76059277343749	-1.87971795539786e-10\\
-6.74044335937499	-5.09709702533492e-11\\
-6.7202939453125	-1.7155735133313e-10\\
-6.70014453125	-2.30195527880699e-10\\
-6.6799951171875	-3.57417598596642e-10\\
-6.65984570312499	-5.39092032790802e-10\\
-6.6396962890625	-6.33234548909977e-10\\
-6.619546875	-7.70273569611764e-10\\
-6.5993974609375	-6.69468323293479e-10\\
-6.57924804687499	-7.94153053864854e-10\\
-6.55909863281249	-8.88209980283928e-10\\
-6.53894921875	-9.28794949859935e-10\\
-6.5187998046875	-8.18913679174485e-10\\
-6.498650390625	-8.06538071442788e-10\\
-6.47850097656249	-7.89689837301307e-10\\
-6.4583515625	-8.18104174413747e-10\\
-6.4382021484375	-7.72713360498415e-10\\
-6.418052734375	-8.44015521697987e-10\\
-6.39790332031249	-8.41862644241417e-10\\
-6.37775390624999	-9.33250712729678e-10\\
-6.3576044921875	-9.65814642541236e-10\\
-6.337455078125	-9.80901542883246e-10\\
-6.31730566406249	-9.87009539302234e-10\\
-6.29715624999999	-1.00658173536148e-09\\
-6.2770068359375	-1.05349604370716e-09\\
-6.256857421875	-9.83564719788021e-10\\
-6.2367080078125	-9.5977591249665e-10\\
-6.21655859374999	-1.03661248625405e-09\\
-6.19640917968749	-1.03928458728135e-09\\
-6.176259765625	-1.07926060219288e-09\\
-6.1561103515625	-1.14930294117664e-09\\
-6.13596093749999	-1.14552272257672e-09\\
-6.11581152343749	-1.02735562114098e-09\\
-6.095662109375	-9.98953452612487e-10\\
-6.0755126953125	-9.33362324806767e-10\\
-6.05536328125	-8.69610413041136e-10\\
-6.03521386718749	-8.07156796492285e-10\\
-6.01506445312499	-7.95767729334196e-10\\
-5.9949150390625	-8.99391180437681e-10\\
-5.974765625	-9.35023561771882e-10\\
-5.95461621093749	-9.202273862261e-10\\
-5.93446679687499	-9.20629667988135e-10\\
-5.9143173828125	-1.02041525958454e-09\\
-5.89416796875	-9.32470228464872e-10\\
-5.8740185546875	-1.02794371345429e-09\\
-5.85386914062499	-1.01611070769491e-09\\
-5.83371972656249	-8.52069436519839e-10\\
-5.8135703125	-9.10431907761471e-10\\
-5.7934208984375	-8.61828014812652e-10\\
-5.77327148437499	-9.19108673046319e-10\\
-5.75312207031249	-9.30935606416427e-10\\
-5.73297265625	-9.79851623714549e-10\\
-5.7128232421875	-9.66091013377635e-10\\
-5.692673828125	-9.93235257471209e-10\\
-5.67252441406249	-1.09469272819813e-09\\
-5.652375	-8.8348862523174e-10\\
-5.6322255859375	-9.6491721157765e-10\\
-5.612076171875	-9.12284685215441e-10\\
-5.59192675781249	-7.30566246177052e-10\\
-5.57177734374999	-7.32321634667777e-10\\
-5.5516279296875	-8.03533682034421e-10\\
-5.531478515625	-7.36703804548195e-10\\
-5.5113291015625	-8.46093175712411e-10\\
-5.49117968749999	-8.79921949029147e-10\\
-5.4710302734375	-9.53554791534036e-10\\
-5.450880859375	-9.18058913258659e-10\\
-5.4307314453125	-7.31541011025866e-10\\
-5.41058203124999	-7.28041487753499e-10\\
-5.39043261718749	-6.53664923094349e-10\\
-5.370283203125	-4.57193489115792e-10\\
-5.3501337890625	-4.51114373599329e-10\\
-5.329984375	-4.42635873324765e-10\\
-5.30983496093749	-5.03100060965839e-10\\
-5.289685546875	-4.60802319721603e-10\\
-5.2695361328125	-5.98285660029964e-10\\
-5.24938671875	-6.75744283859183e-10\\
-5.22923730468749	-6.76051448097738e-10\\
-5.20908789062499	-6.98641412217464e-10\\
-5.1889384765625	-5.66894085960953e-10\\
-5.1687890625	-5.20156518102605e-10\\
-5.1486396484375	-4.10167863917266e-10\\
-5.12849023437499	-3.07734616081671e-10\\
-5.1083408203125	-3.35830948558105e-10\\
-5.08819140625	-1.97327394352936e-10\\
-5.0680419921875	-2.25754653106421e-10\\
-5.04789257812499	-1.96560484621126e-10\\
-5.02774316406249	-2.34729737223383e-10\\
-5.00759375	-2.92898234143119e-10\\
-4.9874443359375	-2.78898894605213e-10\\
-4.96729492187499	-2.35836236168028e-10\\
-4.94714550781249	-3.54165844196569e-10\\
-4.92699609375	-1.86765668311038e-10\\
-4.9068466796875	-1.75545620541577e-10\\
-4.886697265625	1.35931478035213e-10\\
-4.86654785156249	1.65443042679606e-10\\
-4.84639843749999	2.21888697400807e-10\\
-4.8262490234375	1.28730063025832e-10\\
-4.806099609375	1.81860769373058e-10\\
-4.78595019531249	1.54283124940878e-11\\
-4.76580078124999	-5.3634662049444e-11\\
-4.7456513671875	-1.1907433496072e-10\\
-4.725501953125	-1.27980271081094e-10\\
-4.7053525390625	-1.57670165265951e-11\\
-4.68520312499999	6.114272918188e-11\\
-4.66505371093749	2.23802469597627e-10\\
-4.644904296875	4.40569002781853e-10\\
-4.6247548828125	5.46962052298115e-10\\
-4.60460546874999	5.51303046137914e-10\\
-4.58445605468749	5.14995971490003e-10\\
-4.564306640625	4.04796841879693e-10\\
-4.5441572265625	2.52786988631869e-10\\
-4.5240078125	2.271845379308e-10\\
-4.50385839843749	2.75661130768756e-10\\
-4.483708984375	2.76624010504251e-10\\
-4.4635595703125	4.12863016048456e-10\\
-4.44341015625	5.99137408857439e-10\\
-4.42326074218749	6.98327778995142e-10\\
-4.40311132812499	8.66700050205352e-10\\
-4.3829619140625	7.17600150476285e-10\\
-4.3628125	6.02828820758954e-10\\
-4.3426630859375	4.40000244761401e-10\\
-4.32251367187499	4.2622328353419e-10\\
-4.3023642578125	3.67575402831319e-10\\
-4.28221484375	3.57562321979995e-10\\
-4.2620654296875	5.60681926621042e-10\\
-4.24191601562499	6.20666828597697e-10\\
-4.22176660156249	7.26939527712664e-10\\
-4.2016171875	1.03972454625854e-09\\
-4.1814677734375	1.23608953961149e-09\\
-4.161318359375	1.328878296411e-09\\
-4.14116894531249	1.30740013843167e-09\\
-4.12101953125	1.29026457875691e-09\\
-4.1008701171875	1.27961821990359e-09\\
-4.080720703125	1.26979164996534e-09\\
-4.06057128906249	1.14389656312128e-09\\
-4.04042187499999	1.2732893234271e-09\\
-4.0202724609375	1.20875500138227e-09\\
-4.000123046875	1.23081434502829e-09\\
-3.9799736328125	1.33709059834412e-09\\
-3.95982421874999	1.42624382377303e-09\\
-3.9396748046875	1.44729368743038e-09\\
-3.919525390625	1.51471827233407e-09\\
-3.8993759765625	1.59148500644956e-09\\
-3.87922656249999	1.51830240249494e-09\\
-3.85907714843749	1.61413663143843e-09\\
-3.838927734375	1.54408433174977e-09\\
-3.8187783203125	1.47893925106714e-09\\
-3.79862890624999	1.54114633653067e-09\\
-3.77847949218749	1.41468347126905e-09\\
-3.758330078125	1.5524317944053e-09\\
-3.7381806640625	1.58051263541375e-09\\
-3.71803125	1.70423262108342e-09\\
-3.69788183593749	1.77850840207081e-09\\
-3.67773242187499	1.76822308069628e-09\\
-3.6575830078125	1.82228790203124e-09\\
-3.63743359375	1.85063034392027e-09\\
-3.61728417968749	1.68003316861579e-09\\
-3.59713476562499	1.5885487146884e-09\\
-3.5769853515625	1.52792377764072e-09\\
-3.5568359375	1.43684411980211e-09\\
-3.5366865234375	1.50419704164605e-09\\
-3.51653710937499	1.55353302392513e-09\\
-3.49638769531249	1.56218471614333e-09\\
-3.47623828125	1.63766021317948e-09\\
-3.4560888671875	1.54470250242169e-09\\
-3.43593945312499	1.57135050186338e-09\\
-3.41579003906249	1.62114215439037e-09\\
-3.395640625	1.45750987414169e-09\\
-3.3754912109375	1.49810721721607e-09\\
-3.355341796875	1.59155452971967e-09\\
-3.33519238281249	1.43906941512399e-09\\
-3.31504296874999	1.61224534827414e-09\\
-3.2948935546875	1.67037592638933e-09\\
-3.274744140625	1.64075073850775e-09\\
-3.25459472656249	1.68142093031504e-09\\
-3.23444531249999	1.75828557869223e-09\\
-3.2142958984375	1.59316843028999e-09\\
-3.194146484375	1.57055449998493e-09\\
-3.1739970703125	1.57037041273201e-09\\
-3.15384765624999	1.52234738955348e-09\\
-3.1336982421875	1.41383457793648e-09\\
-3.113548828125	1.71972773523917e-09\\
-3.0933994140625	1.62611053699799e-09\\
-3.07324999999999	1.78949558711203e-09\\
-3.05310058593749	1.82236537558251e-09\\
-3.032951171875	1.81183162016405e-09\\
-3.0128017578125	1.81571693114034e-09\\
-2.99265234375	1.81291644904206e-09\\
-2.97250292968749	1.7464750482019e-09\\
-2.952353515625	1.68695054652406e-09\\
-2.9322041015625	1.4855404830249e-09\\
-2.9120546875	1.45602578140766e-09\\
-2.89190527343749	1.44247923131897e-09\\
-2.87175585937499	1.46489156158878e-09\\
-2.8516064453125	1.67373949071883e-09\\
-2.83145703125	1.78346534989267e-09\\
-2.8113076171875	1.87775944105148e-09\\
-2.79115820312499	1.83290704301935e-09\\
-2.7710087890625	1.8813022540238e-09\\
-2.750859375	1.93629546284617e-09\\
-2.7307099609375	1.73274449824517e-09\\
-2.71056054687499	1.61397523078855e-09\\
-2.69041113281249	1.57839366939063e-09\\
-2.67026171875	1.38288579503195e-09\\
-2.6501123046875	1.56864609334645e-09\\
-2.629962890625	1.4470917088442e-09\\
-2.60981347656249	1.56366817551656e-09\\
-2.5896640625	1.54552420609669e-09\\
-2.5695146484375	1.46060356254422e-09\\
-2.549365234375	1.51282419008678e-09\\
-2.52921582031249	1.4851843749604e-09\\
-2.50906640624999	1.35003705738024e-09\\
-2.4889169921875	1.33312357443657e-09\\
-2.468767578125	1.19726058470768e-09\\
-2.44861816406249	1.2829841085597e-09\\
-2.42846874999999	1.21707336524295e-09\\
-2.4083193359375	1.13032355913226e-09\\
-2.388169921875	1.25092397099253e-09\\
-2.3680205078125	1.07182424939499e-09\\
-2.34787109374999	1.06777838860879e-09\\
-2.32772167968749	1.03359970195164e-09\\
-2.307572265625	1.05177955005612e-09\\
-2.2874228515625	9.72698746416079e-10\\
-2.26727343749999	1.08700518223507e-09\\
-2.24712402343749	1.11387526784685e-09\\
-2.226974609375	1.06945585861821e-09\\
-2.2068251953125	1.07374214247232e-09\\
-2.18667578125	1.09783833059703e-09\\
-2.16652636718749	1.02100285795678e-09\\
-2.14637695312499	8.17671932793303e-10\\
-2.1262275390625	7.66931071545141e-10\\
-2.106078125	7.02825323643878e-10\\
-2.08592871093749	6.28282879658703e-10\\
-2.06577929687499	6.44999756216189e-10\\
-2.0456298828125	7.00479775432975e-10\\
-2.02548046875	7.82022194246294e-10\\
-2.0053310546875	7.11127624355587e-10\\
-1.98518164062499	8.20492352962122e-10\\
-1.9650322265625	7.59065102785725e-10\\
-1.9448828125	6.62934978226197e-10\\
-1.9247333984375	6.08328279535488e-10\\
-1.90458398437499	4.60640202297988e-10\\
-1.88443457031249	5.51927730945958e-10\\
-1.86428515625	4.81661869109162e-10\\
-1.8441357421875	6.31316513640664e-10\\
-1.823986328125	6.61902933734606e-10\\
-1.80383691406249	5.27204197140194e-10\\
-1.7836875	5.38549164339305e-10\\
-1.7635380859375	4.63074509785951e-10\\
-1.743388671875	3.16471436617507e-10\\
-1.72323925781249	4.18405612591521e-10\\
-1.70308984374999	3.7220501338156e-10\\
-1.6829404296875	2.19701453894569e-10\\
-1.662791015625	1.55585238797822e-10\\
-1.6426416015625	2.10002031617482e-10\\
-1.62249218749999	4.02200784346034e-11\\
-1.6023427734375	1.29877178973864e-11\\
-1.582193359375	-4.50423971175513e-11\\
-1.5620439453125	-1.18668600549861e-11\\
-1.54189453124999	-2.35014100399208e-10\\
-1.52174511718749	-1.16027063746185e-10\\
-1.501595703125	-2.03637499085472e-10\\
-1.4814462890625	-2.66592741707504e-10\\
-1.461296875	-3.12885404367769e-10\\
-1.44114746093749	-3.86691880622314e-10\\
-1.420998046875	-2.77959922416677e-10\\
-1.4008486328125	-2.95281930950679e-10\\
-1.38069921875	-2.46897311720711e-10\\
-1.36054980468749	-8.79710286885871e-11\\
-1.34040039062499	-1.58702879541427e-10\\
-1.3202509765625	-6.6125055934191e-11\\
-1.3001015625	-1.99360889914488e-10\\
-1.27995214843749	-2.59707375299919e-10\\
-1.25980273437499	-2.3361716577045e-10\\
-1.2396533203125	-4.2230132702971e-10\\
-1.21950390625	-3.9146964212669e-10\\
-1.1993544921875	-5.44963092013456e-10\\
-1.17920507812499	-4.75368069967095e-10\\
-1.15905566406249	-5.06185875182135e-10\\
-1.13890625	-4.58068162246615e-10\\
-1.1187568359375	-3.64357585475997e-10\\
-1.09860742187499	-3.24177603318743e-10\\
-1.07845800781249	-2.94982942648331e-10\\
-1.05830859375	-1.91442013437085e-10\\
-1.0381591796875	-3.90041367840596e-10\\
-1.018009765625	-3.05936056247912e-10\\
-0.997860351562494	-3.39611449022176e-10\\
-0.977710937499992	-2.74632386227888e-10\\
-0.957561523437498	-1.65676425417933e-10\\
-0.937412109374996	1.40709776105871e-11\\
-0.917262695312495	4.26867958526909e-11\\
-0.897113281249993	6.35173124431793e-11\\
-0.876963867187499	-2.14121641614234e-11\\
-0.856814453124997	-1.2547026063333e-10\\
-0.836665039062495	-6.75405851887087e-11\\
-0.816515624999994	-2.13482602205779e-10\\
-0.796366210937499	-2.8682087150117e-10\\
-0.776216796874998	-2.00807413111698e-10\\
-0.756067382812496	-1.95723708154093e-10\\
-0.735917968749995	-1.23664198947574e-10\\
-0.715768554687493	-2.96728631934473e-11\\
-0.695619140624999	-5.79796787402065e-11\\
-0.675469726562497	-4.40834248087479e-11\\
-0.655320312499995	-9.57476415024677e-11\\
-0.635170898437494	-1.10909215165506e-10\\
-0.615021484374999	-1.85827548542168e-11\\
-0.594872070312498	-1.92756381686429e-10\\
-0.574722656249996	-1.49521326060482e-10\\
-0.554573242187494	-2.57009285149451e-10\\
-0.534423828124993	-1.63408777941241e-10\\
-0.514274414062498	-1.45899689781626e-10\\
-0.494124999999997	1.33605189844405e-11\\
-0.473975585937495	-1.04396069185672e-11\\
-0.453826171874994	1.19498276235254e-10\\
-0.433676757812499	9.58835898322727e-11\\
-0.413527343749998	2.33313328472888e-10\\
-0.393377929687496	2.18909157471499e-10\\
-0.373228515624994	1.90923654735109e-10\\
-0.353079101562493	2.65182604196714e-10\\
-0.332929687499998	1.95544022084891e-10\\
-0.312780273437497	2.28604276752021e-10\\
-0.292630859374995	2.06274687357738e-10\\
-0.272481445312494	3.42814281534392e-10\\
-0.252332031249999	3.28913277617362e-10\\
-0.232182617187497	5.46653305584173e-10\\
-0.212033203124996	5.39122683543538e-10\\
-0.191883789062494	6.42633288094734e-10\\
-0.171734374999993	7.53082457081632e-10\\
-0.151584960937498	7.23918006696213e-10\\
-0.131435546874997	8.5017120936188e-10\\
-0.111286132812495	1.06226332591699e-09\\
-0.0911367187499934	9.86221648408654e-10\\
-0.0709873046874989	1.22848854082204e-09\\
-0.0508378906249973	1.33148870881326e-09\\
-0.0306884765624957	1.35578036310216e-09\\
-0.0105390624999941	1.54200385179823e-09\\
0.00961035156250745	1.56000051830543e-09\\
0.0297597656250019	1.67414723639338e-09\\
0.0499091796875035	1.5511281723354e-09\\
0.0700585937500051	1.68050034329208e-09\\
0.0902080078125067	1.75029500455436e-09\\
0.110357421875001	1.78166306985253e-09\\
0.130506835937503	1.83101932351234e-09\\
0.150656250000004	2.21110549084786e-09\\
0.170805664062506	2.156661231232e-09\\
0.190955078125008	2.38322343905032e-09\\
0.211104492187502	2.47446561797509e-09\\
0.231253906250004	2.49151900846188e-09\\
0.251403320312505	2.39241982808285e-09\\
0.271552734375007	2.32754186273854e-09\\
0.291702148437501	2.22295465129632e-09\\
0.311851562500003	2.27375197047017e-09\\
0.332000976562504	2.34687196359386e-09\\
0.352150390625006	2.59776894575209e-09\\
0.372299804687501	2.8623952705491e-09\\
0.392449218750002	3.01654458566355e-09\\
0.412598632812504	3.30820267637101e-09\\
0.432748046875005	3.4110339109575e-09\\
0.452897460937507	3.550783531058e-09\\
0.473046875000001	3.4384925366409e-09\\
0.493196289062503	3.31202511356265e-09\\
0.513345703125005	3.3626169355237e-09\\
0.533495117187506	3.21634457469051e-09\\
0.553644531250001	3.26778773503061e-09\\
0.573793945312502	3.32754517489435e-09\\
0.593943359375004	3.57225976661555e-09\\
0.614092773437505	3.9021058685081e-09\\
0.634242187500007	4.19487003635436e-09\\
0.654391601562502	4.4364699326425e-09\\
0.674541015625003	4.55813628305397e-09\\
0.694690429687505	4.59288668937402e-09\\
0.714839843750006	4.43491670235142e-09\\
0.734989257812501	4.27546855979165e-09\\
0.755138671875002	4.29960249212859e-09\\
0.775288085937504	4.2344897424106e-09\\
0.795437500000006	4.12204366433163e-09\\
0.815586914062507	4.44013765178621e-09\\
0.835736328125002	4.67621826350142e-09\\
0.855885742187503	5.01149620815272e-09\\
0.876035156250005	5.40340097007332e-09\\
0.896184570312506	5.78765891605383e-09\\
0.916333984375001	5.98720065906477e-09\\
0.936483398437503	5.97717188788629e-09\\
0.956632812500004	6.02611962893976e-09\\
0.976782226562506	5.90412250337475e-09\\
0.996931640625007	5.81539108618653e-09\\
1.0170810546875	5.70826820436324e-09\\
1.03723046875	5.79808743080016e-09\\
1.0573798828125	5.85397810357418e-09\\
1.07752929687501	5.93021294591822e-09\\
1.0976787109375	6.33322361433322e-09\\
1.117828125	6.58215360491828e-09\\
1.1379775390625	6.77870253135203e-09\\
1.15812695312501	6.99292022550427e-09\\
1.17827636718751	7.18548709802176e-09\\
1.19842578125	7.04413049100717e-09\\
1.2185751953125	7.11761946806205e-09\\
1.23872460937501	7.0906113459861e-09\\
1.25887402343751	6.98609269952518e-09\\
1.2790234375	7.16189796142393e-09\\
1.2991728515625	7.20019843746348e-09\\
1.319322265625	7.37848194705596e-09\\
1.33947167968751	7.71776231077976e-09\\
1.35962109375001	7.89350238550395e-09\\
1.3797705078125	8.04002306110861e-09\\
1.399919921875	8.23249897550073e-09\\
1.42006933593751	8.27233377893329e-09\\
1.44021875000001	8.15536828420209e-09\\
1.4603681640625	8.19401064681429e-09\\
1.480517578125	8.21505643994722e-09\\
1.5006669921875	8.11225307028831e-09\\
1.52081640625001	8.21742183456577e-09\\
1.5409658203125	8.32607351291082e-09\\
1.561115234375	8.4637909244545e-09\\
1.5812646484375	8.61635409053648e-09\\
1.60141406250001	8.83903619942022e-09\\
1.62156347656251	8.93796043454151e-09\\
1.641712890625	8.99211908932398e-09\\
1.6618623046875	9.03117405008076e-09\\
1.68201171875	9.0419327026283e-09\\
1.70216113281251	8.99568123960409e-09\\
1.722310546875	9.07180351179288e-09\\
1.7424599609375	8.9953929862216e-09\\
1.762609375	9.11596368034871e-09\\
1.78275878906251	9.22494552889802e-09\\
1.80290820312501	9.31075737360143e-09\\
1.8230576171875	9.63405401855023e-09\\
1.84320703125	9.78933680893489e-09\\
1.8633564453125	9.94749093034798e-09\\
1.88350585937501	1.00014200092094e-08\\
1.9036552734375	1.00414987997147e-08\\
1.9238046875	1.00430214562393e-08\\
1.9439541015625	1.01720458264704e-08\\
1.96410351562501	1.01379221525682e-08\\
1.98425292968751	1.03074924941822e-08\\
2.00440234375	1.03063781965528e-08\\
2.0245517578125	1.04578388932108e-08\\
2.044701171875	1.05616559859584e-08\\
2.06485058593751	1.06943987191332e-08\\
2.085	1.0848599343748e-08\\
2.1051494140625	1.07385378359329e-08\\
2.125298828125	1.07415744994033e-08\\
2.14544824218751	1.06088042474197e-08\\
2.16559765625001	1.05091982590077e-08\\
2.1857470703125	1.04475246210945e-08\\
2.205896484375	1.04063699834371e-08\\
2.2260458984375	1.03804127946742e-08\\
2.24619531250001	1.04458100920759e-08\\
2.2663447265625	1.0479151128519e-08\\
2.286494140625	1.05521304744716e-08\\
2.3066435546875	1.07177351352969e-08\\
2.32679296875001	1.05298331946567e-08\\
2.34694238281251	1.0481958137593e-08\\
2.367091796875	1.01266293232368e-08\\
2.3872412109375	1.00282005771348e-08\\
2.40739062500001	9.78453545396906e-09\\
2.42754003906251	9.66285800335848e-09\\
2.447689453125	9.46708420666629e-09\\
2.4678388671875	9.4401057090082e-09\\
2.48798828125	9.43398499027808e-09\\
2.50813769531251	9.46445175865456e-09\\
2.52828710937501	9.52560307862594e-09\\
2.5484365234375	9.5024299457279e-09\\
2.5685859375	9.44840380774678e-09\\
2.58873535156251	8.97049997176723e-09\\
2.60888476562501	8.8961718287187e-09\\
2.6290341796875	8.46834272284425e-09\\
2.64918359375	8.27657669940332e-09\\
2.6693330078125	7.90925336851795e-09\\
2.68948242187501	7.82876451807146e-09\\
2.7096318359375	7.81218299279192e-09\\
2.72978125	7.76360971775548e-09\\
2.7499306640625	8.02846882774384e-09\\
2.77008007812501	8.36496708300924e-09\\
2.79022949218751	8.32402415088705e-09\\
2.81037890625	8.47738711453094e-09\\
2.8305283203125	8.36492502119965e-09\\
2.850677734375	8.09693628124677e-09\\
2.87082714843751	7.84386659759845e-09\\
2.8909765625	7.6157756356447e-09\\
2.9111259765625	7.34607426037784e-09\\
2.931275390625	7.17778013357298e-09\\
2.95142480468751	7.15480294792186e-09\\
2.97157421875001	7.11401314819584e-09\\
2.9917236328125	7.44299455584962e-09\\
3.011873046875	7.44567855840661e-09\\
3.0320224609375	7.60801136034911e-09\\
3.05217187500001	7.70930674297667e-09\\
3.0723212890625	7.59474004877991e-09\\
3.092470703125	7.3872456768514e-09\\
3.1126201171875	7.20264509307946e-09\\
3.13276953125001	6.89087415186804e-09\\
3.15291894531251	6.67532458045803e-09\\
3.173068359375	6.37293530516409e-09\\
3.1932177734375	6.28690964180283e-09\\
3.2133671875	6.50896745541196e-09\\
3.23351660156251	6.46951707370189e-09\\
3.253666015625	6.63706108019447e-09\\
3.2738154296875	6.72440911434006e-09\\
3.29396484375	6.66021755060615e-09\\
3.31411425781251	6.63709272007297e-09\\
3.33426367187501	6.43640836907281e-09\\
3.3544130859375	6.29342609849076e-09\\
3.3745625	6.17715052314638e-09\\
3.3947119140625	5.90219290221853e-09\\
3.41486132812501	5.86484903288669e-09\\
3.4350107421875	5.92173563426419e-09\\
3.45516015625	5.78035992507519e-09\\
3.4753095703125	5.85767143517118e-09\\
3.49545898437501	5.78171558887907e-09\\
3.51560839843751	5.70798603035018e-09\\
3.5357578125	5.63355233847071e-09\\
3.5559072265625	5.5169886692835e-09\\
3.576056640625	5.28902281473891e-09\\
3.59620605468751	5.22373927758921e-09\\
3.61635546875	4.93932521205499e-09\\
3.6365048828125	4.96557056729371e-09\\
3.656654296875	4.90335681951696e-09\\
3.67680371093751	4.86006248197808e-09\\
3.69695312500001	4.64194156016894e-09\\
3.7171025390625	4.73078508249443e-09\\
3.737251953125	4.58188892835089e-09\\
3.75740136718751	4.43299780804553e-09\\
3.77755078125001	4.36292907928103e-09\\
3.7977001953125	4.20363792383944e-09\\
3.817849609375	4.02535446956928e-09\\
3.8379990234375	3.98959202257011e-09\\
3.85814843750001	4.01123983159533e-09\\
3.8782978515625	4.04518211991099e-09\\
3.898447265625	4.01274702109404e-09\\
3.9185966796875	4.01931919447776e-09\\
3.93874609375001	3.92919510604204e-09\\
3.95889550781251	3.84421649604763e-09\\
3.979044921875	3.77827392390878e-09\\
3.9991943359375	3.76853531120059e-09\\
4.01934375	3.69578839268768e-09\\
4.03949316406251	3.51775878447939e-09\\
4.059642578125	3.57510960402306e-09\\
4.0797919921875	3.49364780628298e-09\\
4.09994140625	3.33269513223524e-09\\
4.12009082031251	3.43192066807862e-09\\
4.14024023437501	3.28223257605143e-09\\
4.1603896484375	3.14368896775458e-09\\
4.1805390625	3.0174527780217e-09\\
4.2006884765625	3.02970780976985e-09\\
4.22083789062501	2.95050651233033e-09\\
4.2409873046875	2.73602776483383e-09\\
4.26113671875	2.83203581575082e-09\\
4.2812861328125	2.77412562946647e-09\\
4.30143554687501	2.6282046174569e-09\\
4.32158496093751	2.6484300929967e-09\\
4.341734375	2.7419124491441e-09\\
4.3618837890625	2.49493421537927e-09\\
4.382033203125	2.45121440983658e-09\\
4.40218261718751	2.41696522213943e-09\\
4.42233203125	2.25746299418944e-09\\
4.4424814453125	1.95877735166305e-09\\
4.462630859375	2.03295706348706e-09\\
4.48278027343751	1.81079709661706e-09\\
4.50292968750001	1.71561989842445e-09\\
4.5230791015625	1.62152906181042e-09\\
4.543228515625	1.64807475157757e-09\\
4.5633779296875	1.57334582944191e-09\\
4.58352734375001	1.58875333656605e-09\\
4.6036767578125	1.49462104748235e-09\\
4.623826171875	1.43094958017104e-09\\
4.6439755859375	1.19674989316264e-09\\
4.66412500000001	1.19348356410342e-09\\
4.68427441406251	1.08695457260779e-09\\
4.704423828125	1.0964593047153e-09\\
4.7245732421875	9.9866066899357e-10\\
4.74472265625	1.00645255405174e-09\\
4.76487207031251	1.01703245483283e-09\\
4.785021484375	1.07907903718498e-09\\
4.8051708984375	1.12621772566202e-09\\
4.8253203125	1.15465968929692e-09\\
4.84546972656251	1.20159777879335e-09\\
4.86561914062501	1.22554610362799e-09\\
4.8857685546875	1.14352815306665e-09\\
4.90591796875	1.03260409655679e-09\\
4.92606738281251	9.3098267767322e-10\\
4.94621679687501	8.66402505276789e-10\\
4.9663662109375	8.24505142021257e-10\\
4.986515625	7.66108240578084e-10\\
5.0066650390625	8.22428083874049e-10\\
5.02681445312501	8.00569319504638e-10\\
5.0469638671875	7.23650195968981e-10\\
5.06711328125	7.43616997342851e-10\\
5.0872626953125	7.16904634413937e-10\\
5.10741210937501	6.91350708816647e-10\\
5.12756152343751	5.75188314001121e-10\\
5.1477109375	5.60600889728445e-10\\
5.1678603515625	5.7637909788143e-10\\
5.188009765625	4.73629438404683e-10\\
5.20815917968751	4.46325516495453e-10\\
5.22830859375	3.36389487409343e-10\\
5.2484580078125	3.83772471806453e-10\\
5.268607421875	2.58938750676534e-10\\
5.28875683593751	2.91933754529544e-10\\
5.30890625000001	2.67590207939276e-10\\
5.3290556640625	1.10401098118512e-10\\
5.349205078125	2.05138353403815e-10\\
5.3693544921875	1.12544745013367e-10\\
5.38950390625001	1.54909819507509e-10\\
5.4096533203125	3.52703082673063e-11\\
5.429802734375	6.76868293922431e-11\\
5.4499521484375	-6.70510497838294e-11\\
5.47010156250001	-1.40652328743099e-10\\
5.49025097656251	-1.81992697021646e-10\\
5.510400390625	-2.38525752337551e-10\\
5.5305498046875	-3.50765827598333e-10\\
5.55069921875	-4.85085103953607e-10\\
5.57084863281251	-4.87852050232594e-10\\
5.590998046875	-5.18197667632187e-10\\
5.6111474609375	-5.06551331308841e-10\\
5.631296875	-5.61387461509877e-10\\
5.65144628906251	-5.19879316152007e-10\\
5.67159570312501	-5.08953657629934e-10\\
5.6917451171875	-6.64033997786495e-10\\
5.71189453125	-7.97657326687803e-10\\
5.7320439453125	-8.64568330420595e-10\\
5.75219335937501	-7.9824521976666e-10\\
5.7723427734375	-9.6623866379358e-10\\
5.7924921875	-9.41318358799248e-10\\
5.8126416015625	-1.03739466709185e-09\\
5.83279101562501	-8.92468666504786e-10\\
5.85294042968751	-8.2017053528151e-10\\
5.87308984375	-7.78970617161475e-10\\
5.8932392578125	-7.20309034488646e-10\\
5.913388671875	-7.32961012710654e-10\\
5.93353808593751	-7.80957305451123e-10\\
5.9536875	-7.02076687164033e-10\\
5.9738369140625	-8.78249071513595e-10\\
5.993986328125	-1.02067071508687e-09\\
6.01413574218751	-1.02004772146257e-09\\
6.03428515625001	-1.18401652199418e-09\\
6.0544345703125	-1.09253770704596e-09\\
6.074583984375	-1.08246583847477e-09\\
6.0947333984375	-1.08933288773148e-09\\
6.11488281250001	-1.00650473426698e-09\\
6.1350322265625	-9.69653317717401e-10\\
6.155181640625	-1.01920860586711e-09\\
6.1753310546875	-8.89170903833844e-10\\
6.19548046875001	-9.95242832222079e-10\\
6.2156298828125	-1.00465714045752e-09\\
6.235779296875	-1.12499062078219e-09\\
6.2559287109375	-1.02327694484113e-09\\
6.27607812500001	-1.13149633257791e-09\\
6.29622753906251	-1.11924333912403e-09\\
6.316376953125	-1.02095102485594e-09\\
6.3365263671875	-1.04821858824297e-09\\
6.35667578125	-1.06356923082205e-09\\
6.37682519531251	-9.72729972752365e-10\\
6.396974609375	-1.06434516833066e-09\\
6.4171240234375	-1.0074365822772e-09\\
6.4372734375	-1.21196492721549e-09\\
6.45742285156251	-1.17313038191276e-09\\
6.47757226562501	-1.16818054089085e-09\\
6.4977216796875	-1.21603887193848e-09\\
6.51787109375	-1.09861526617397e-09\\
6.5380205078125	-1.13546248125858e-09\\
6.55816992187501	-1.14444178346273e-09\\
6.5783193359375	-1.18741993364816e-09\\
6.59846875	-1.26151059446033e-09\\
6.6186181640625	-1.28174803623827e-09\\
6.63876757812501	-1.35908604863785e-09\\
6.65891699218751	-1.37817682037096e-09\\
6.67906640625	-1.42133237222744e-09\\
6.6992158203125	-1.38776145369706e-09\\
6.719365234375	-1.40766631311704e-09\\
6.73951464843751	-1.35467489125969e-09\\
6.7596640625	-1.2385520371547e-09\\
6.7798134765625	-1.28763161594668e-09\\
6.799962890625	-1.2112561927121e-09\\
6.82011230468751	-1.11174730821593e-09\\
6.84026171875001	-1.25639710155138e-09\\
6.8604111328125	-1.12540064797978e-09\\
6.880560546875	-1.1522282484205e-09\\
6.9007099609375	-1.29831446472327e-09\\
6.92085937500001	-1.30734899509832e-09\\
6.9410087890625	-1.1543268983683e-09\\
6.961158203125	-1.21450840467793e-09\\
6.9813076171875	-1.20695633284813e-09\\
7.00145703125001	-1.22690400089825e-09\\
7.02160644531251	-1.19000302030757e-09\\
7.041755859375	-9.87882344923473e-10\\
7.0619052734375	-1.04181568083581e-09\\
7.0820546875	-9.00303132524545e-10\\
7.10220410156251	-9.90605808179939e-10\\
7.122353515625	-1.02551117633645e-09\\
7.1425029296875	-1.05524265890962e-09\\
7.16265234375	-1.03310465022587e-09\\
7.18280175781251	-1.06604099673689e-09\\
7.20295117187501	-1.11512143416373e-09\\
7.2231005859375	-1.16607227785057e-09\\
7.24325	-1.04145677767455e-09\\
7.2633994140625	-9.7115031896659e-10\\
7.28354882812501	-8.79086675571935e-10\\
7.3036982421875	-8.75277441365796e-10\\
7.32384765625	-9.68778273934532e-10\\
7.3439970703125	-9.05960398299049e-10\\
7.36414648437501	-1.10495596331995e-09\\
7.3842958984375	-1.11188002752382e-09\\
7.4044453125	-1.18560459374991e-09\\
7.4245947265625	-1.09730706908409e-09\\
7.44474414062501	-1.24878932695219e-09\\
7.46489355468751	-1.19518254230086e-09\\
7.48504296875	-1.0892429508738e-09\\
7.5051923828125	-1.12157906044996e-09\\
7.525341796875	-1.04276679486158e-09\\
7.54549121093751	-1.08681863072568e-09\\
7.565640625	-1.10031519473315e-09\\
7.5857900390625	-1.10416154574144e-09\\
7.605939453125	-1.11704382262996e-09\\
7.62608886718751	-1.15019475110762e-09\\
7.64623828125001	-1.11930200621125e-09\\
7.6663876953125	-1.1680984386846e-09\\
7.686537109375	-1.03287618582683e-09\\
7.7066865234375	-1.09622013188494e-09\\
7.72683593750001	-9.375906160101e-10\\
7.7469853515625	-1.10396700906959e-09\\
7.767134765625	-9.87524280573618e-10\\
7.7872841796875	-1.07430175024708e-09\\
7.80743359375001	-9.98751559368171e-10\\
7.82758300781251	-1.12172510813596e-09\\
7.847732421875	-1.08943396008254e-09\\
7.8678818359375	-1.0968043703922e-09\\
7.88803125	-1.08734561341074e-09\\
7.90818066406251	-9.83885535340862e-10\\
7.928330078125	-1.01437586354188e-09\\
7.9484794921875	-9.85465154800382e-10\\
7.96862890625	-8.04276066721668e-10\\
7.98877832031251	-8.30040293417784e-10\\
8.00892773437501	-8.38269291468735e-10\\
8.0290771484375	-7.57840253696713e-10\\
8.0492265625	-7.79219676215784e-10\\
8.0693759765625	-8.35231203189068e-10\\
8.08952539062501	-8.04532769624994e-10\\
8.1096748046875	-7.3802558186643e-10\\
8.12982421875	-7.69393628742061e-10\\
8.1499736328125	-6.8287097547345e-10\\
8.17012304687501	-6.19364345342397e-10\\
8.19027246093751	-6.52137114957733e-10\\
8.210421875	-5.63785812597005e-10\\
8.2305712890625	-6.36177799317873e-10\\
8.250720703125	-6.30463821575563e-10\\
8.27087011718751	-5.83074657620475e-10\\
8.29101953125	-5.47876519414119e-10\\
8.3111689453125	-5.93442247102709e-10\\
8.331318359375	-4.97944272734412e-10\\
8.35146777343751	-5.99487484587365e-10\\
8.37161718750001	-5.47104417546601e-10\\
8.3917666015625	-5.79322244909808e-10\\
8.411916015625	-5.68703353180045e-10\\
8.4320654296875	-5.36974041599036e-10\\
8.45221484375001	-5.95333999830253e-10\\
8.4723642578125	-5.50061409816423e-10\\
8.492513671875	-5.05773247553592e-10\\
8.5126630859375	-4.07409474414524e-10\\
8.53281250000001	-4.67540214371947e-10\\
8.5529619140625	-3.25769498122597e-10\\
8.573111328125	-3.47706887138264e-10\\
8.5932607421875	-2.55733197678741e-10\\
8.61341015625	-2.81591952361076e-10\\
8.63355957031251	-2.05766565575329e-10\\
8.653708984375	-1.68524216258924e-10\\
8.6738583984375	-1.89764948560578e-10\\
8.6940078125	-2.01645917955787e-10\\
8.71415722656251	-1.47652073535791e-10\\
8.734306640625	-1.6766319193121e-10\\
8.7544560546875	-5.8251918969402e-11\\
8.77460546875	-2.18135032856518e-10\\
8.79475488281251	-6.40059204924393e-11\\
8.81490429687501	-1.03897991457977e-10\\
8.8350537109375	-2.23250052611152e-10\\
8.855203125	-1.36756122386354e-10\\
8.8753525390625	-1.31707003466414e-10\\
8.89550195312501	-1.17474820882244e-10\\
8.9156513671875	-8.59870147830064e-11\\
8.93580078125	-7.38731358574507e-13\\
8.9559501953125	1.49477245232236e-11\\
8.97609960937501	-2.17080152585615e-11\\
8.99624902343751	-3.74727014679845e-11\\
9.0163984375	-4.83956063130116e-11\\
9.0365478515625	-1.40115683478132e-10\\
9.056697265625	-2.20383848685721e-10\\
9.07684667968751	-3.66256697412479e-10\\
9.09699609375	-2.04585509448135e-10\\
9.1171455078125	-3.06277672356856e-10\\
9.137294921875	-1.72135019801541e-10\\
9.15744433593751	-1.48497970704281e-10\\
9.17759375000001	-2.3153463978671e-11\\
9.1977431640625	-1.25876142088276e-11\\
9.217892578125	4.0341136880463e-11\\
9.2380419921875	-5.97900280182659e-11\\
9.25819140625001	-7.40070392252687e-11\\
9.2783408203125	-1.11007840610866e-10\\
9.298490234375	-7.95445033474825e-11\\
9.3186396484375	-1.09713017175421e-10\\
9.33878906250001	-7.4647357107795e-11\\
9.35893847656251	5.55149719213577e-11\\
9.379087890625	-4.41296216735574e-11\\
9.3992373046875	-3.54652865838437e-11\\
9.41938671875	1.23222494625995e-10\\
9.43953613281251	4.34291132520653e-12\\
9.459685546875	-2.96999425800648e-11\\
9.4798349609375	3.7232553483532e-11\\
9.499984375	-4.23758377146125e-11\\
9.52013378906251	-3.07577923093189e-11\\
9.54028320312501	-3.62115688636809e-11\\
9.5604326171875	-1.24248670737011e-10\\
9.58058203125	-4.88258273513245e-11\\
9.6007314453125	-1.70413779777661e-11\\
9.62088085937501	7.55741983036035e-11\\
9.6410302734375	8.67848102033139e-11\\
9.6611796875	-5.50434804131447e-11\\
9.6813291015625	4.94917846337955e-11\\
9.70147851562501	-1.03797285007752e-10\\
9.7216279296875	-7.03419468816357e-11\\
9.74177734375	-1.24505004407596e-10\\
9.7619267578125	-2.32630332378154e-10\\
9.782076171875	-3.63829530098836e-10\\
9.80222558593751	-3.66271539823765e-10\\
9.822375	-3.70852769270093e-10\\
9.8425244140625	-1.72310477223712e-10\\
9.862673828125	-1.07899435225538e-10\\
9.88282324218751	-4.91311079254905e-11\\
9.90297265625	4.79392041391778e-11\\
9.9231220703125	6.32773906567707e-11\\
9.943271484375	2.43886112318637e-10\\
9.96342089843751	2.24445194163162e-10\\
9.98357031250001	1.64065012805467e-10\\
10.0037197265625	-3.3104161961917e-11\\
10.023869140625	4.13412280701553e-11\\
10.0440185546875	-1.26324182651422e-10\\
10.06416796875	-4.1845499964601e-11\\
10.0843173828125	-8.34514639390356e-11\\
10.104466796875	-1.02306370653239e-10\\
10.1246162109375	-1.2922463668323e-10\\
10.144765625	-1.82759472135552e-12\\
10.1649150390625	-2.9103684066064e-11\\
10.185064453125	4.53540594857951e-13\\
10.2052138671875	-9.60438670054248e-11\\
10.22536328125	-3.15517084089716e-11\\
10.2455126953125	-9.10203426950919e-11\\
10.265662109375	-5.94576826923778e-12\\
10.2858115234375	-3.75443245558971e-11\\
10.3059609375	-3.44946684374074e-11\\
10.3261103515625	-1.92168310159402e-11\\
10.346259765625	1.16611212621735e-10\\
10.3664091796875	6.18603519214318e-11\\
10.38655859375	1.12542446717473e-10\\
10.4067080078125	8.27616697779349e-11\\
10.426857421875	-6.08368506251656e-11\\
10.4470068359375	4.76315764416944e-12\\
10.46715625	-6.82949879944901e-11\\
10.4873056640625	-1.19253250220753e-10\\
10.507455078125	1.57794893726244e-11\\
10.5276044921875	-5.45149123164114e-12\\
10.54775390625	3.19125577922758e-11\\
10.5679033203125	8.69021247855355e-11\\
10.588052734375	6.38076059246e-11\\
10.6082021484375	8.42876279059793e-11\\
10.6283515625	5.39690076353179e-11\\
10.6485009765625	-2.24427514810294e-11\\
10.668650390625	-4.21863778263924e-11\\
10.6887998046875	-1.60860994862911e-10\\
10.70894921875	-1.26308183271076e-10\\
10.7290986328125	-1.47162453336992e-10\\
10.749248046875	-1.82604047967817e-10\\
10.7693974609375	-2.13567180685563e-10\\
10.789546875	-1.8564325526842e-10\\
10.8096962890625	-1.1078483296908e-10\\
10.829845703125	-7.26521057322517e-11\\
10.8499951171875	-1.91313436530159e-10\\
10.87014453125	-1.34497593183432e-10\\
10.8902939453125	-2.42211893944298e-10\\
10.910443359375	-2.3238831584232e-10\\
10.9305927734375	-3.3707564317252e-10\\
10.9507421875	-2.32207645703108e-10\\
10.9708916015625	-1.41170339693255e-10\\
10.991041015625	-2.1112441741327e-10\\
11.0111904296875	-8.19941414910284e-11\\
11.03133984375	9.49550790655514e-12\\
11.0514892578125	2.18804726251717e-11\\
11.071638671875	-5.58997702177591e-11\\
11.0917880859375	-2.7173960561967e-11\\
11.1119375	-2.87124411547578e-10\\
11.1320869140625	-3.23516385261365e-10\\
11.152236328125	-4.96068203184205e-10\\
11.1723857421875	-4.89789319342284e-10\\
11.19253515625	-5.06900193526091e-10\\
11.2126845703125	-5.17691108628892e-10\\
11.232833984375	-4.31235467528804e-10\\
11.2529833984375	-3.69902657571375e-10\\
11.2731328125	-2.78038938822226e-10\\
11.2932822265625	-1.44773845779976e-10\\
11.313431640625	-1.98979601978543e-10\\
11.3335810546875	-7.28444840200849e-11\\
11.35373046875	-1.77410761413662e-10\\
11.3738798828125	-1.61016971078352e-10\\
11.394029296875	-1.19746430081465e-10\\
11.4141787109375	-2.82683349033038e-10\\
11.434328125	-2.67040791722135e-10\\
11.4544775390625	-2.62675458695115e-10\\
11.474626953125	-1.66810841328507e-10\\
11.4947763671875	-1.04400750418024e-10\\
11.51492578125	-1.56921067498967e-10\\
11.5350751953125	-4.43465191147332e-12\\
11.555224609375	-3.4079302397716e-11\\
11.5753740234375	-1.88225786964982e-11\\
11.5955234375	-7.27213580081108e-11\\
11.6156728515625	-1.16034599393812e-10\\
11.635822265625	-2.42927153754246e-10\\
11.6559716796875	-2.44093725171666e-10\\
11.67612109375	-2.84258709977497e-10\\
11.6962705078125	-3.98483192823779e-10\\
11.716419921875	-3.26562042377816e-10\\
11.7365693359375	-3.6476081749316e-10\\
11.75671875	-3.48507331112475e-10\\
11.7768681640625	-3.05294365599199e-10\\
11.797017578125	-4.36161447038697e-10\\
11.8171669921875	-3.06692070075182e-10\\
11.83731640625	-3.82485188777108e-10\\
11.8574658203125	-4.37862291779621e-10\\
11.877615234375	-4.11917263507305e-10\\
11.8977646484375	-3.97757488782628e-10\\
11.9179140625	-3.55624127995473e-10\\
11.9380634765625	-4.39653128969375e-10\\
11.958212890625	-3.63132368583319e-10\\
11.9783623046875	-4.24795856460991e-10\\
11.99851171875	-4.9509654978805e-10\\
12.0186611328125	-3.7973632347296e-10\\
12.038810546875	-4.76541519893743e-10\\
12.0589599609375	-5.42100905428001e-10\\
12.079109375	-4.01886717135638e-10\\
12.0992587890625	-4.50139623516829e-10\\
12.119408203125	-3.37130383756413e-10\\
12.1395576171875	-2.57698072246021e-10\\
12.15970703125	-2.7739210494547e-10\\
12.1798564453125	-2.34256545858236e-10\\
12.200005859375	-1.9196087495968e-10\\
12.2201552734375	-2.61134090357343e-10\\
12.2403046875	-3.38878545998837e-10\\
12.2604541015625	-4.14571426212128e-10\\
12.280603515625	-3.95770133096383e-10\\
12.3007529296875	-3.97524100965402e-10\\
12.32090234375	-3.60114468822069e-10\\
12.3410517578125	-3.26268709864348e-10\\
12.361201171875	-3.11497284004418e-10\\
12.3813505859375	-2.95831357325157e-10\\
12.4015	-3.43740415971473e-10\\
12.4216494140625	-4.09082322280232e-10\\
12.441798828125	-4.54180239895885e-10\\
12.4619482421875	-4.08008036364044e-10\\
12.48209765625	-3.89400211688852e-10\\
12.5022470703125	-3.86969696984219e-10\\
12.522396484375	-3.62534490642142e-10\\
12.5425458984375	-3.14540187587971e-10\\
12.5626953125	-3.47622716266327e-10\\
12.5828447265625	-4.04426665799873e-10\\
12.602994140625	-4.50923852035705e-10\\
12.6231435546875	-4.13129170937459e-10\\
12.64329296875	-4.48497601832813e-10\\
12.6634423828125	-4.09568557759013e-10\\
12.683591796875	-4.39521228272104e-10\\
12.7037412109375	-4.36836793115631e-10\\
12.723890625	-3.73655405462209e-10\\
12.7440400390625	-3.72210753149829e-10\\
12.764189453125	-3.46837729166476e-10\\
12.7843388671875	-4.60870882287728e-10\\
12.80448828125	-4.50013951911467e-10\\
12.8246376953125	-5.17583226219494e-10\\
12.844787109375	-5.20066160317178e-10\\
12.8649365234375	-4.69000449529926e-10\\
12.8850859375	-4.81067236051691e-10\\
12.9052353515625	-5.01668028654386e-10\\
12.925384765625	-4.57996007563446e-10\\
12.9455341796875	-4.17069670874844e-10\\
12.96568359375	-4.03271890034314e-10\\
12.9858330078125	-4.14815411083268e-10\\
13.005982421875	-4.62426863545993e-10\\
13.0261318359375	-4.15116461000048e-10\\
13.04628125	-3.60824560270422e-10\\
13.0664306640625	-3.41225954238166e-10\\
13.086580078125	-3.71443731382264e-10\\
13.1067294921875	-4.5342091523859e-10\\
13.12687890625	-3.63531519679192e-10\\
13.1470283203125	-2.84620496652187e-10\\
13.167177734375	-3.00931776697367e-10\\
13.1873271484375	-2.62602911219724e-10\\
13.2074765625	-3.38334252617099e-10\\
13.2276259765625	-3.82041858588835e-10\\
13.247775390625	-3.24338052735744e-10\\
13.2679248046875	-3.58825432157763e-10\\
13.28807421875	-3.14931807699526e-10\\
13.3082236328125	-2.48964667486196e-10\\
13.328373046875	-1.9948149865888e-10\\
13.3485224609375	-2.47837332711746e-10\\
13.368671875	-1.60029068895596e-10\\
13.3888212890625	-1.762013652465e-10\\
13.408970703125	-1.49385324944624e-10\\
13.4291201171875	-2.62234272392094e-10\\
13.44926953125	-2.54459776735258e-10\\
13.4694189453125	-4.09659085393415e-10\\
13.489568359375	-4.00362455465899e-10\\
13.5097177734375	-4.72301960394494e-10\\
13.5298671875	-5.12004931627666e-10\\
13.5500166015625	-5.58667388172608e-10\\
13.570166015625	-4.11519605758797e-10\\
13.5903154296875	-3.50703204903249e-10\\
13.61046484375	-3.7931224004855e-10\\
13.6306142578125	-3.14474318023003e-10\\
13.650763671875	-3.10974132219149e-10\\
13.6709130859375	-2.51365386569674e-10\\
13.6910625	-2.6699563803782e-10\\
13.7112119140625	-2.95407033470052e-10\\
13.731361328125	-2.48195960231374e-10\\
13.7515107421875	-3.38651562314688e-10\\
13.77166015625	-3.36782128288128e-10\\
13.7918095703125	-2.87050629826728e-10\\
13.811958984375	-3.27174911605502e-10\\
13.8321083984375	-2.7421309996662e-10\\
13.8522578125	-3.51367015769315e-10\\
13.8724072265625	-3.46766521405508e-10\\
13.892556640625	-3.18058063848855e-10\\
13.9127060546875	-3.71692872447913e-10\\
13.93285546875	-3.19018211972922e-10\\
13.9530048828125	-3.61431554756374e-10\\
13.973154296875	-3.20763294275433e-10\\
13.9933037109375	-3.17260457161651e-10\\
14.013453125	-2.7856751499737e-10\\
14.0336025390625	-3.48168443909056e-10\\
14.053751953125	-2.82089864110635e-10\\
14.0739013671875	-2.78475421328141e-10\\
14.09405078125	-1.63847163183394e-10\\
14.1142001953125	-2.63380465080139e-10\\
14.134349609375	-1.81805290258593e-10\\
14.1544990234375	-2.10714307764331e-10\\
14.1746484375	-2.3836705693179e-10\\
14.1947978515625	-2.18682480178587e-10\\
14.214947265625	-1.77151264773819e-10\\
14.2350966796875	-3.02028504625371e-10\\
14.25524609375	-2.06102468852601e-10\\
14.2753955078125	-2.581379454869e-10\\
14.295544921875	-2.5042985826528e-10\\
14.3156943359375	-1.57880523828518e-10\\
14.33584375	-2.26719460936872e-10\\
14.3559931640625	-6.59543921669226e-11\\
14.376142578125	-1.12080827343738e-10\\
14.3962919921875	-1.95209617459629e-10\\
14.41644140625	-1.09682264662197e-10\\
14.4365908203125	-1.45174735508371e-10\\
14.456740234375	-1.52635164786309e-10\\
14.4768896484375	-2.25940967330633e-10\\
14.4970390625	-9.98437273829368e-11\\
14.5171884765625	-1.86368044746705e-10\\
14.537337890625	-1.6284398493916e-10\\
14.5574873046875	-2.2314844876568e-10\\
14.57763671875	-1.07314046104888e-10\\
14.5977861328125	-1.52394508621006e-10\\
14.617935546875	-1.80001207091841e-10\\
14.6380849609375	-1.95092018303935e-10\\
14.658234375	-1.70216547594692e-10\\
14.6783837890625	-9.39168455355079e-11\\
14.698533203125	5.94892030901254e-11\\
14.7186826171875	9.41665490617768e-11\\
14.73883203125	1.64344014704499e-10\\
14.7589814453125	1.28441616659939e-10\\
14.779130859375	6.10308216842403e-11\\
14.7992802734375	1.18151288390066e-11\\
14.8194296875	6.49491446736342e-12\\
14.8395791015625	-1.74528015830592e-10\\
14.859728515625	-3.35158144939929e-10\\
14.8798779296875	-2.57074762725456e-10\\
14.90002734375	-1.56503892676411e-10\\
14.9201767578125	-4.53779550943492e-12\\
14.940326171875	1.53722369726265e-10\\
14.9604755859375	5.04534431550087e-11\\
14.980625	2.66562305897387e-10\\
15.0007744140625	1.01232056608616e-10\\
15.020923828125	1.14943166257856e-10\\
15.0410732421875	7.32237492328962e-11\\
15.06122265625	1.57919937390605e-11\\
15.0813720703125	-2.52020528950256e-13\\
15.101521484375	-5.21557341486415e-12\\
15.1216708984375	-4.63630268066207e-11\\
15.1418203125	-2.78306044897146e-11\\
15.1619697265625	-2.64897553656847e-11\\
15.182119140625	-9.81677282304933e-12\\
15.2022685546875	5.62100192048825e-11\\
15.22241796875	4.83022483703135e-11\\
15.2425673828125	3.86579217148814e-11\\
15.262716796875	8.20760933513432e-11\\
15.2828662109375	7.69089429249167e-11\\
15.303015625	2.32154688686757e-11\\
15.3231650390625	7.74099556392106e-11\\
15.343314453125	-4.09142343489049e-11\\
15.3634638671875	-6.69560062158953e-11\\
15.38361328125	-1.73632239479082e-10\\
15.4037626953125	-1.30306496683057e-10\\
15.423912109375	-5.70446182673168e-11\\
15.4440615234375	-3.15318076812649e-11\\
15.4642109375	1.02074540939936e-10\\
15.4843603515625	5.17560231330152e-11\\
15.504509765625	8.8317343576122e-11\\
15.5246591796875	2.21799063141863e-11\\
15.54480859375	7.06763024402591e-12\\
15.5649580078125	6.60865725449422e-12\\
15.585107421875	-7.17871850251546e-12\\
15.6052568359375	-6.61752200447132e-11\\
15.62540625	1.03688094236991e-10\\
15.6455556640625	1.39136842322815e-10\\
15.665705078125	2.526375150329e-10\\
15.6858544921875	2.51896929059074e-10\\
15.70600390625	2.95863954824026e-10\\
15.7261533203125	1.80606233265079e-10\\
15.746302734375	1.073725493056e-10\\
15.7664521484375	4.10612292109626e-11\\
15.7866015625	-2.05340093075686e-11\\
15.8067509765625	-6.82511179978091e-11\\
15.826900390625	8.68755511827226e-11\\
15.8470498046875	1.18016190755988e-10\\
15.86719921875	2.34446482706198e-10\\
15.8873486328125	2.2613510812919e-10\\
15.907498046875	3.88361994393842e-10\\
15.9276474609375	2.68903729298085e-10\\
15.947796875	3.48261999723803e-10\\
15.9679462890625	2.4204665249822e-10\\
15.988095703125	2.64564552726533e-10\\
16.0082451171875	2.20353267357525e-10\\
16.02839453125	2.09814336470111e-10\\
16.0485439453125	2.63945101205479e-10\\
16.068693359375	2.76340765939918e-10\\
16.0888427734375	2.63967986353759e-10\\
16.1089921875	2.69969012171155e-10\\
16.1291416015625	2.77815643328758e-10\\
16.149291015625	1.64096946152456e-10\\
16.1694404296875	1.94531434971215e-10\\
16.18958984375	-4.38998059096395e-11\\
16.2097392578125	-7.15950837408676e-11\\
16.229888671875	-4.48629446553213e-13\\
16.2500380859375	-1.86083573596977e-10\\
16.2701875	-8.33680559128766e-11\\
16.2903369140625	-1.33409331348815e-10\\
16.310486328125	-3.87586263382376e-11\\
16.3306357421875	7.26580878589156e-11\\
16.35078515625	8.05761214461792e-11\\
16.3709345703125	1.77485239874107e-10\\
16.391083984375	9.4080268669805e-12\\
16.4112333984375	8.6048557120382e-11\\
16.4313828125	-6.58166864253769e-11\\
16.4515322265625	-2.24535698273066e-11\\
16.471681640625	-1.2609309186964e-10\\
16.4918310546875	-2.01121792214945e-10\\
16.51198046875	-1.68879276508245e-10\\
16.5321298828125	-8.05667970890558e-11\\
16.552279296875	-1.51911783443657e-11\\
16.5724287109375	6.00403664277016e-11\\
16.592578125	-1.45331449054945e-10\\
16.6127275390625	-8.95027803573227e-11\\
16.632876953125	-3.25219405284298e-11\\
16.6530263671875	-1.61926762943576e-10\\
16.67317578125	-6.10561796807414e-11\\
16.6933251953125	5.48899776824492e-11\\
16.713474609375	-8.30003928713363e-11\\
16.7336240234375	-3.08705650267899e-11\\
16.7537734375	-5.57165636582876e-11\\
16.7739228515625	-5.25647385025322e-11\\
16.794072265625	-1.13307021714721e-10\\
16.8142216796875	-9.42684712370743e-11\\
16.83437109375	-1.42573129078544e-10\\
16.8545205078125	-1.49991615226398e-10\\
16.874669921875	-1.51145784068041e-10\\
16.8948193359375	-1.43934488268179e-10\\
16.91496875	-4.82767983731566e-11\\
16.9351181640625	-1.55822609211313e-10\\
16.955267578125	-1.21210745565194e-10\\
16.9754169921875	-1.20933218081369e-10\\
16.99556640625	-1.80983201841617e-10\\
17.0157158203125	-1.90035909379637e-10\\
17.035865234375	-1.48925711799037e-10\\
17.0560146484375	-1.53800178586581e-10\\
17.0761640625	-1.82941187960373e-10\\
17.0963134765625	-1.89230329005855e-10\\
17.116462890625	-1.89894378750569e-10\\
17.1366123046875	-2.61192013245106e-10\\
17.15676171875	-2.95630346849904e-10\\
17.1769111328125	-3.58081723579687e-10\\
17.197060546875	-4.46404890265308e-10\\
17.2172099609375	-4.67907948783701e-10\\
17.237359375	-3.95545207940932e-10\\
17.2575087890625	-3.9216109105208e-10\\
17.277658203125	-2.96469533150766e-10\\
17.2978076171875	-3.82096075944299e-10\\
17.31795703125	-3.47841179355061e-10\\
17.3381064453125	-3.20616155761441e-10\\
17.358255859375	-3.01166949641362e-10\\
17.3784052734375	-2.81988709603358e-10\\
17.3985546875	-2.83182356233003e-10\\
17.4187041015625	-3.34807235133185e-10\\
17.438853515625	-5.14341768040927e-10\\
17.4590029296875	-4.1480972390469e-10\\
17.47915234375	-3.1488004314238e-10\\
17.4993017578125	-2.91057490888138e-10\\
17.519451171875	-2.60773237578938e-10\\
17.5396005859375	-2.08369524055745e-10\\
17.55975	-1.784869184275e-10\\
17.5798994140625	-1.87683845758016e-10\\
17.600048828125	-2.28097259302735e-10\\
17.6201982421875	-2.16009947750269e-10\\
17.64034765625	-2.791943951744e-10\\
17.6604970703125	-3.46652014443424e-10\\
17.680646484375	-3.14838956968106e-10\\
17.7007958984375	-3.73549210643314e-10\\
17.7209453125	-3.50287576841799e-10\\
17.7410947265625	-3.23776397247912e-10\\
17.761244140625	-1.38428421090136e-10\\
17.7813935546875	-1.51334544768673e-10\\
17.80154296875	-3.43138608810748e-11\\
17.8216923828125	-2.14687220764563e-11\\
17.841841796875	-1.19439059921671e-11\\
17.8619912109375	2.81096371292375e-11\\
17.882140625	-2.75571482096492e-11\\
17.9022900390625	-5.95671627610989e-11\\
17.922439453125	-9.32437002716517e-11\\
17.9425888671875	-1.39462095544056e-10\\
17.96273828125	-1.97936434099781e-10\\
17.9828876953125	-2.44032359147744e-10\\
18.003037109375	-1.8652720183621e-10\\
18.0231865234375	-1.95961579704787e-10\\
18.0433359375	-1.61801506487061e-10\\
18.0634853515625	-3.68149255817654e-10\\
18.083634765625	-4.25296849805458e-10\\
18.1037841796875	-4.68808502623404e-10\\
18.12393359375	-5.90466204021941e-10\\
18.1440830078125	-5.95295163307574e-10\\
18.164232421875	-4.99224971434662e-10\\
18.1843818359375	-4.97462912970573e-10\\
18.20453125	-4.61806722105399e-10\\
18.2246806640625	-3.98560119263079e-10\\
18.244830078125	-2.85437764770288e-10\\
18.2649794921875	-3.52106859844404e-10\\
18.28512890625	-3.79310598468621e-10\\
18.3052783203125	-3.98707858760148e-10\\
18.325427734375	-4.6598785971213e-10\\
18.3455771484375	-6.26433464647677e-10\\
18.3657265625	-5.19510142782995e-10\\
18.3858759765625	-5.6095811350169e-10\\
18.406025390625	-5.47018011271318e-10\\
18.4261748046875	-5.07123912313299e-10\\
18.44632421875	-3.51555436134089e-10\\
18.4664736328125	-3.81034523578246e-10\\
18.486623046875	-1.80123856864789e-10\\
18.5067724609375	-2.84882375043962e-10\\
18.526921875	-3.58252193368058e-10\\
18.5470712890625	-2.83513190601461e-10\\
18.567220703125	-5.05302898695391e-10\\
18.5873701171875	-4.57158909466318e-10\\
18.60751953125	-4.93951609036748e-10\\
18.6276689453125	-4.62998005972349e-10\\
18.647818359375	-4.04180950496136e-10\\
18.6679677734375	-3.1360806434857e-10\\
18.6881171875	-2.92399255677931e-10\\
18.7082666015625	-2.64461633991763e-10\\
18.728416015625	-1.67120680868724e-10\\
18.7485654296875	-9.01495486643727e-11\\
18.76871484375	-1.05985994902684e-10\\
18.7888642578125	-2.47155753680238e-10\\
18.809013671875	-3.2108503247571e-10\\
18.8291630859375	-3.64690846545067e-10\\
18.8493125	-4.8000744752957e-10\\
18.8694619140625	-4.9941213502039e-10\\
18.889611328125	-5.12618836102719e-10\\
18.9097607421875	-4.12268655441004e-10\\
18.92991015625	-3.57469786103374e-10\\
18.9500595703125	-2.4120279395753e-10\\
18.970208984375	-3.07137959194923e-10\\
18.9903583984375	-2.43119865437976e-10\\
19.0105078125	-2.54222704860214e-10\\
19.0306572265625	-1.87428671521405e-10\\
19.050806640625	-2.61796873721964e-10\\
19.0709560546875	-2.66465297666151e-10\\
19.09110546875	-1.72862777968567e-10\\
19.1112548828125	-1.50028017414765e-10\\
19.131404296875	-9.01373836102737e-11\\
19.1515537109375	-1.4873387908534e-11\\
19.171703125	1.14349852579383e-10\\
19.1918525390625	1.24193425731173e-10\\
19.212001953125	2.25116845344207e-10\\
19.2321513671875	2.83300344682304e-10\\
19.25230078125	1.0427737705754e-10\\
19.2724501953125	-5.58022665638457e-11\\
19.292599609375	-2.18819591885747e-10\\
19.3127490234375	-2.51572204976915e-10\\
19.3328984375	-3.0292639222117e-10\\
19.3530478515625	-1.21573170152292e-10\\
19.373197265625	-5.40643369093341e-11\\
19.3933466796875	5.59384186431625e-11\\
19.41349609375	2.44887442003839e-10\\
19.4336455078125	2.69977858892281e-10\\
19.453794921875	3.08388121303437e-10\\
19.4739443359375	1.82313598458233e-10\\
19.49409375	8.90934696883197e-11\\
19.5142431640625	4.89378998394869e-11\\
19.534392578125	-7.7975951331869e-11\\
19.5545419921875	-7.0646024565099e-11\\
19.57469140625	-4.44944731811127e-11\\
19.5948408203125	-1.12125926633754e-10\\
19.614990234375	-1.08301396537053e-11\\
19.6351396484375	4.2229973743176e-11\\
19.6552890625	9.86868923183584e-11\\
19.6754384765625	1.45751517452331e-10\\
19.695587890625	5.55471108979621e-11\\
19.7157373046875	-3.09925256657386e-11\\
19.73588671875	-1.56972616205227e-11\\
19.7560361328125	-1.92430097158854e-10\\
19.776185546875	-1.36632057953253e-10\\
19.7963349609375	-1.33290202870518e-10\\
19.816484375	-1.3671674873267e-10\\
19.8366337890625	-1.31828782033711e-10\\
19.856783203125	-5.99515386025188e-11\\
19.8769326171875	-3.21945839238307e-11\\
19.89708203125	-1.04028295222426e-10\\
19.9172314453125	-1.77238482720819e-10\\
19.937380859375	-1.8650460988747e-10\\
19.9575302734375	-1.84718918075158e-10\\
19.9776796875	-3.31974389758551e-10\\
19.9978291015625	-2.94847134973904e-10\\
20.017978515625	-2.39132228377315e-10\\
20.0381279296875	-3.45272189790096e-10\\
20.05827734375	-4.01043849234567e-10\\
20.0784267578125	-2.59465997293011e-10\\
20.098576171875	-2.12511797579562e-10\\
20.1187255859375	-2.19169256254837e-10\\
20.138875	-2.64524653219248e-10\\
20.1590244140625	-2.64642253313586e-10\\
20.179173828125	-2.77007308488959e-10\\
20.1993232421875	-2.55214853076757e-10\\
20.21947265625	-3.38117609504679e-10\\
20.2396220703125	-2.58496188334086e-10\\
20.259771484375	-4.14912544690689e-10\\
20.2799208984375	-4.96981469943073e-10\\
20.3000703125	-3.9111320409616e-10\\
20.3202197265625	-4.62749610249792e-10\\
20.340369140625	-4.35171763561487e-10\\
20.3605185546875	-3.61922422206399e-10\\
20.38066796875	-3.82333577903343e-10\\
20.4008173828125	-2.42371688425666e-10\\
20.420966796875	-3.82845076301057e-10\\
20.4411162109375	-2.753473691074e-10\\
20.461265625	-2.9847628741294e-10\\
20.4814150390625	-3.65032005921316e-10\\
20.501564453125	-4.0629763265435e-10\\
20.5217138671875	-3.8481567012133e-10\\
20.54186328125	-4.58918429286373e-10\\
20.5620126953125	-4.02125065711882e-10\\
20.582162109375	-3.55041522787695e-10\\
20.6023115234375	-2.07947468189335e-10\\
20.6224609375	-1.79794772559522e-10\\
20.6426103515625	-9.39042004277396e-11\\
20.662759765625	-4.18040945238317e-11\\
20.6829091796875	-8.7873843079231e-11\\
20.70305859375	-1.42532277658577e-10\\
20.7232080078125	-1.8329640285624e-10\\
20.743357421875	-2.93930883248825e-10\\
20.7635068359375	-3.93093543396166e-10\\
20.78365625	-3.43552216653934e-10\\
20.8038056640625	-3.61467499103037e-10\\
20.823955078125	-4.48796208370225e-10\\
20.8441044921875	-2.9628353267302e-10\\
20.86425390625	-3.19311485803685e-10\\
20.8844033203125	-2.5730305759026e-10\\
20.904552734375	-2.22938666869053e-10\\
20.9247021484375	-5.94711263047202e-11\\
20.9448515625	5.95906536893821e-11\\
20.9650009765625	7.74750073638537e-11\\
20.985150390625	8.6836998417703e-11\\
21.0052998046875	-5.25224192890075e-11\\
21.02544921875	-7.10329074325645e-11\\
21.0455986328125	-1.06079553056289e-10\\
21.065748046875	-2.12110867637414e-10\\
21.0858974609375	-1.78079917385417e-10\\
21.106046875	-2.45078773187437e-10\\
21.1261962890625	-2.20952163250404e-10\\
21.146345703125	-1.89173092911422e-10\\
21.1664951171875	-1.60310853596134e-10\\
21.18664453125	-1.35119421880199e-10\\
21.2067939453125	-6.94064398111674e-11\\
21.226943359375	-2.10795949066605e-10\\
21.2470927734375	-2.31612076153104e-10\\
21.2672421875	-4.23302289893039e-10\\
21.2873916015625	-5.19693394287178e-10\\
21.307541015625	-5.25821907082597e-10\\
21.3276904296875	-5.14263258086313e-10\\
21.34783984375	-4.70989223537359e-10\\
21.3679892578125	-4.22274546084666e-10\\
21.388138671875	-3.53101869979343e-10\\
21.4082880859375	-2.9653301825304e-10\\
21.4284375	-2.56634706555516e-10\\
21.4485869140625	-3.07170013770494e-10\\
21.468736328125	-1.71263860862319e-10\\
21.4888857421875	-2.66702992343303e-10\\
21.50903515625	-3.0240730079214e-10\\
21.5291845703125	-3.27988364687637e-10\\
21.549333984375	-2.91224805982983e-10\\
21.5694833984375	-3.84309758065917e-10\\
21.5896328125	-2.95221134004932e-10\\
21.6097822265625	-3.83646922412838e-10\\
21.629931640625	-3.38772837345586e-10\\
21.6500810546875	-4.18577277618074e-10\\
21.67023046875	-5.60500589767233e-10\\
21.6903798828125	-5.21452142562836e-10\\
21.710529296875	-4.28057674299189e-10\\
21.7306787109375	-4.77473853241608e-10\\
21.750828125	-3.64984532567556e-10\\
21.7709775390625	-5.01689920594525e-10\\
21.791126953125	-4.56726142997867e-10\\
21.8112763671875	-4.56116897768888e-10\\
21.83142578125	-4.63891539258489e-10\\
21.8515751953125	-4.52011728998057e-10\\
21.871724609375	-5.22634584976946e-10\\
21.8918740234375	-4.13460744207989e-10\\
21.9120234375	-3.41305301801301e-10\\
21.9321728515625	-3.03933931906878e-10\\
21.952322265625	-2.14545550540821e-10\\
21.9724716796875	-1.85205942553331e-10\\
21.99262109375	-2.69304934944497e-10\\
22.0127705078125	-3.05127969893539e-10\\
22.032919921875	-4.41741833772907e-10\\
22.0530693359375	-5.06061714129559e-10\\
22.07321875	-6.70301785466327e-10\\
22.0933681640625	-6.78655878458971e-10\\
22.113517578125	-5.94933073460861e-10\\
22.1336669921875	-5.19251924180135e-10\\
22.15381640625	-3.63120747756132e-10\\
22.1739658203125	-2.74574185451407e-10\\
22.194115234375	-7.89420136457316e-11\\
22.2142646484375	9.2876624797105e-11\\
22.2344140625	5.89629463739528e-11\\
22.2545634765625	-3.72536191617624e-11\\
22.274712890625	-1.34253823774676e-10\\
22.2948623046875	-3.45152533593926e-10\\
22.31501171875	-4.83466538003188e-10\\
22.3351611328125	-4.19936123117132e-10\\
22.355310546875	-4.37502826015173e-10\\
22.3754599609375	-2.1735157412922e-10\\
22.395609375	-1.72143005557542e-10\\
22.4157587890625	6.84594184623828e-11\\
22.435908203125	1.88010109106657e-10\\
22.4560576171875	2.04049617330049e-10\\
22.47620703125	2.27687403229491e-10\\
22.4963564453125	1.21675212031493e-10\\
22.516505859375	1.21098334871432e-11\\
22.5366552734375	-3.06400737365247e-11\\
22.5568046875	-1.60545091702904e-11\\
22.5769541015625	-7.26320473246592e-11\\
22.597103515625	-3.56577829847218e-11\\
22.6172529296875	4.21895967800568e-11\\
22.63740234375	1.42637905129425e-10\\
22.6575517578125	1.68149187464812e-10\\
22.677701171875	1.37625957917721e-10\\
22.6978505859375	1.38673942967301e-10\\
22.718	1.30935751500604e-10\\
22.7381494140625	1.19120143693249e-10\\
22.758298828125	6.95337567416919e-11\\
22.7784482421875	1.63360138332754e-10\\
22.79859765625	1.38382587221706e-10\\
22.8187470703125	6.02172650075542e-11\\
22.838896484375	2.67386398184551e-10\\
22.8590458984375	2.06386638554576e-10\\
22.8791953125	3.2072140082762e-10\\
22.8993447265625	2.331562602714e-10\\
22.919494140625	1.82952095649557e-10\\
22.9396435546875	2.41436806617873e-10\\
22.95979296875	1.68906553207116e-10\\
22.9799423828125	2.35271172952567e-10\\
23.000091796875	2.0495940087515e-10\\
23.0202412109375	1.38109852831445e-10\\
23.040390625	9.13161923082709e-11\\
23.0605400390625	8.9130586618614e-11\\
23.080689453125	9.50915353671866e-11\\
23.1008388671875	1.55454799042646e-10\\
23.12098828125	7.74725087574087e-11\\
23.1411376953125	1.34461942229782e-10\\
23.161287109375	4.75072987533216e-11\\
23.1814365234375	1.55007262530758e-10\\
23.2015859375	2.20966886356195e-10\\
23.2217353515625	2.46995025405267e-10\\
23.241884765625	9.35224365423303e-11\\
23.2620341796875	1.86416656119242e-10\\
23.28218359375	2.68403489535197e-10\\
23.3023330078125	1.55860343369339e-10\\
23.322482421875	2.68582378567515e-10\\
23.3426318359375	2.55351202704643e-10\\
23.36278125	2.59161699327703e-10\\
23.3829306640625	2.30418470203174e-10\\
23.403080078125	2.36015795371779e-10\\
23.4232294921875	1.31086169340017e-10\\
23.44337890625	1.34932901325285e-10\\
23.4635283203125	1.12057067403978e-10\\
23.483677734375	2.2091403799346e-10\\
23.5038271484375	2.10053749285672e-10\\
23.5239765625	2.6159658664382e-10\\
23.5441259765625	2.98560623533273e-10\\
23.564275390625	3.32717641515253e-10\\
23.5844248046875	3.09946455731221e-10\\
23.60457421875	2.84537031600745e-10\\
23.6247236328125	2.26039339319448e-10\\
23.644873046875	1.97589989232712e-10\\
23.6650224609375	1.92294358608345e-10\\
23.685171875	2.28342456996681e-10\\
23.7053212890625	1.9360051196696e-10\\
23.725470703125	3.18030970738134e-10\\
23.7456201171875	3.81262820324736e-10\\
23.76576953125	4.04781616405973e-10\\
23.7859189453125	4.96684915780497e-10\\
23.806068359375	4.16429802071018e-10\\
23.8262177734375	3.84668573991701e-10\\
23.8463671875	2.83877784972181e-10\\
23.8665166015625	2.48246742964142e-10\\
23.886666015625	2.75654329338989e-10\\
23.9068154296875	1.48902115576404e-10\\
23.92696484375	1.93872286090894e-10\\
23.9471142578125	2.72373585234092e-10\\
23.967263671875	2.97321368700767e-10\\
23.9874130859375	4.28196980957806e-10\\
24.0075625	4.34277703442418e-10\\
24.0277119140625	3.28001197679856e-10\\
24.047861328125	2.7558825503254e-10\\
24.0680107421875	2.76611804015434e-10\\
24.08816015625	2.26120452119502e-10\\
24.1083095703125	2.40161842878314e-10\\
24.128458984375	1.69450988926326e-10\\
24.1486083984375	1.08133243202842e-10\\
24.1687578125	1.64850176592101e-10\\
24.1889072265625	2.9872932125971e-10\\
24.209056640625	2.70185685188364e-10\\
24.2292060546875	3.06667778893843e-10\\
24.24935546875	3.19505706708359e-10\\
24.2695048828125	3.15825063990731e-10\\
24.289654296875	2.30184791469642e-10\\
24.3098037109375	2.01274600913986e-10\\
24.329953125	1.6038030717018e-10\\
24.3501025390625	2.24404918166502e-10\\
24.370251953125	2.25796497612182e-10\\
24.3904013671875	2.02320575877332e-10\\
24.41055078125	2.61246151680984e-10\\
24.4307001953125	2.78554881097923e-10\\
24.450849609375	2.45460410209769e-10\\
24.4709990234375	2.43179655909415e-10\\
24.4911484375	8.49696899104869e-11\\
24.5112978515625	6.59770341687149e-11\\
24.531447265625	5.31122316619713e-11\\
24.5515966796875	6.87441644465412e-12\\
24.57174609375	1.0814155185305e-10\\
24.5918955078125	1.24738781745059e-10\\
24.612044921875	2.07948642440131e-10\\
24.6321943359375	1.76369159077076e-10\\
24.65234375	2.82338363799029e-10\\
24.6724931640625	2.0460466077932e-10\\
24.692642578125	1.17419742217932e-10\\
24.7127919921875	2.46861279662984e-11\\
24.73294140625	-4.8807358317399e-11\\
24.7530908203125	-1.33278408608761e-10\\
24.773240234375	-1.12878398512967e-10\\
24.7933896484375	-1.69345451132346e-10\\
24.8135390625	1.33201812881564e-11\\
24.8336884765625	5.05616450884778e-12\\
24.853837890625	1.27922266839e-10\\
24.8739873046875	1.28914762194709e-10\\
24.89413671875	1.40248684327618e-10\\
24.9142861328125	-7.09874147172022e-12\\
24.934435546875	-1.20292098790365e-10\\
24.9545849609375	-2.35775265284645e-10\\
24.974734375	-3.71160687262343e-10\\
24.9948837890625	-5.62853245751289e-10\\
25.015033203125	-4.46897512545796e-10\\
25.0351826171875	-3.21080054760104e-10\\
25.05533203125	-2.27114356486241e-10\\
25.0754814453125	-1.94468488885091e-10\\
25.095630859375	-1.14452875387632e-10\\
25.1157802734375	-3.41167466052115e-11\\
25.1359296875	-7.25280379235509e-11\\
25.1560791015625	-1.51293549581312e-10\\
25.176228515625	-1.22634578168494e-10\\
25.1963779296875	-2.55774892076165e-10\\
25.21652734375	-2.70162039103706e-10\\
25.2366767578125	-3.35956642656619e-10\\
25.256826171875	-3.38894352314892e-10\\
25.2769755859375	-2.10312483479689e-10\\
25.297125	-3.01085405249448e-10\\
25.3172744140625	-1.11760462986446e-10\\
25.337423828125	-9.57866715675104e-11\\
25.3575732421875	-2.10376315884341e-10\\
25.37772265625	-1.11960885044217e-10\\
25.3978720703125	-3.02847579321563e-10\\
25.418021484375	-2.98308291435267e-10\\
25.4381708984375	-3.16219276562098e-10\\
25.4583203125	-3.72987618287543e-10\\
25.4784697265625	-2.93591871556601e-10\\
25.498619140625	-2.95946851567571e-10\\
25.5187685546875	-2.73760048662989e-10\\
25.53891796875	-1.86554198719498e-10\\
25.5590673828125	-2.24132716334608e-10\\
25.579216796875	-1.92525661244787e-10\\
25.5993662109375	-1.84319730244792e-10\\
25.619515625	-1.49218760666811e-10\\
25.6396650390625	-1.56631022973247e-10\\
25.659814453125	-1.59180494692287e-10\\
25.6799638671875	-1.74268056740461e-10\\
25.70011328125	-1.54915454702913e-10\\
25.7202626953125	-1.05014932066663e-10\\
25.740412109375	-1.38842766374243e-10\\
25.7605615234375	-8.35901564317299e-11\\
25.7807109375	-1.96217051988353e-10\\
25.8008603515625	-1.44473843139582e-10\\
25.821009765625	-1.79281839280997e-10\\
25.8411591796875	-2.16052454805941e-10\\
25.86130859375	-2.94961875443831e-10\\
25.8814580078125	-4.29974683746028e-10\\
25.901607421875	-3.58724109883043e-10\\
25.9217568359375	-3.14293110496184e-10\\
25.94190625	-2.11504490799316e-10\\
25.9620556640625	-1.74820588117899e-10\\
25.982205078125	-1.80583998859805e-10\\
26.0023544921875	-2.07756773778413e-10\\
26.02250390625	-9.79383745535507e-11\\
26.0426533203125	-1.86471321753716e-10\\
26.062802734375	-2.42933529852648e-10\\
26.0829521484375	-3.55585320303197e-10\\
26.1031015625	-3.85709759700943e-10\\
26.1232509765625	-4.30513706584067e-10\\
26.143400390625	-2.74122852693033e-10\\
26.1635498046875	-2.56189163786877e-10\\
26.18369921875	-1.93864482376911e-10\\
26.2038486328125	-1.11520862250917e-10\\
26.223998046875	-1.42232491365418e-10\\
26.2441474609375	-7.27032676503768e-11\\
26.264296875	-5.07956123626472e-11\\
26.2844462890625	-2.02176797930009e-10\\
26.304595703125	-2.95588819970538e-10\\
26.3247451171875	-3.38005431074895e-10\\
26.34489453125	-4.44536315750838e-10\\
26.3650439453125	-4.01801187446405e-10\\
26.385193359375	-3.62792683474493e-10\\
26.4053427734375	-3.23493344608127e-10\\
26.4254921875	-2.88961278000097e-10\\
26.4456416015625	-2.4855788946404e-10\\
26.465791015625	-2.41890824118724e-10\\
26.4859404296875	-1.81915057716032e-10\\
26.50608984375	-2.53923879821699e-10\\
26.5262392578125	-2.70284213393392e-10\\
26.546388671875	-3.58207338310938e-10\\
26.5665380859375	-2.44563919739719e-10\\
26.5866875	-3.17560396352845e-10\\
26.6068369140625	-3.33156495638213e-10\\
26.626986328125	-2.9754653807045e-10\\
26.6471357421875	-2.62636271637281e-10\\
26.66728515625	-2.4018410708685e-10\\
26.6874345703125	-1.44813618644804e-10\\
26.707583984375	-1.02251384668396e-10\\
26.7277333984375	-2.48627133960963e-10\\
26.7478828125	-2.05416139434719e-10\\
26.7680322265625	-2.33039804857336e-10\\
26.788181640625	-2.40795537801914e-10\\
26.8083310546875	-2.16905367707575e-10\\
26.82848046875	-2.72667985222734e-10\\
26.8486298828125	-2.29851066369729e-10\\
26.868779296875	-1.90293368389179e-10\\
26.8889287109375	-2.11907324306256e-10\\
26.909078125	-5.98129334649867e-11\\
26.9292275390625	-2.60099092952369e-10\\
26.949376953125	-1.77976359668072e-10\\
26.9695263671875	-1.3376531760693e-10\\
26.98967578125	-2.37657670003302e-10\\
27.0098251953125	-1.67528984807783e-10\\
27.029974609375	-1.78270596459595e-10\\
27.0501240234375	-1.10771093479726e-10\\
27.0702734375	-9.17209064764692e-11\\
27.0904228515625	-9.11147058099252e-11\\
27.110572265625	-7.20756456410763e-11\\
27.1307216796875	-1.04244283356529e-11\\
27.15087109375	-3.28567024760406e-11\\
27.1710205078125	-9.91078540736815e-11\\
27.191169921875	-2.06304811430742e-11\\
27.2113193359375	-1.78927033667317e-10\\
27.23146875	1.87999630328541e-11\\
27.2516181640625	-1.35655164513126e-11\\
27.271767578125	-2.67520096265891e-12\\
27.2919169921875	4.66677930717582e-11\\
27.31206640625	1.10191156599843e-10\\
27.3322158203125	7.96673825834398e-11\\
27.352365234375	1.67303180360894e-10\\
27.3725146484375	1.14792698862538e-10\\
27.3926640625	5.62692311243301e-11\\
27.4128134765625	2.55744269845763e-11\\
27.432962890625	3.72230514766449e-11\\
27.4531123046875	-9.43724238362841e-12\\
27.47326171875	-1.16405743117456e-11\\
27.4934111328125	4.99351482152291e-11\\
27.513560546875	1.47917520673822e-10\\
27.5337099609375	6.02994355662401e-11\\
27.553859375	1.42634772042514e-10\\
27.5740087890625	1.43914294862124e-10\\
27.594158203125	2.05674061024699e-10\\
27.6143076171875	2.24626693808576e-10\\
27.63445703125	1.06796994402189e-10\\
27.6546064453125	1.5230569609271e-10\\
27.674755859375	3.62649831628478e-11\\
27.6949052734375	6.70557007277367e-11\\
27.7150546875	2.45602092099475e-11\\
27.7352041015625	1.39780805489773e-10\\
27.755353515625	9.01437361528582e-11\\
27.7755029296875	1.40628960719573e-10\\
27.79565234375	1.01987588957186e-10\\
27.8158017578125	1.92940875620974e-10\\
27.835951171875	9.01180784555819e-11\\
27.8561005859375	5.39938974733799e-11\\
27.87625	1.28645762900556e-10\\
27.8963994140625	8.07583588394994e-11\\
27.916548828125	1.52458586517128e-10\\
27.9366982421875	2.08563619743104e-10\\
27.95684765625	1.57449053499591e-10\\
27.9769970703125	1.68703445614127e-10\\
27.997146484375	2.17844570834394e-10\\
28.0172958984375	2.01898120867739e-10\\
28.0374453125	1.51735066028779e-10\\
28.0575947265625	1.00827268560041e-10\\
28.077744140625	1.47794991749108e-10\\
28.0978935546875	1.78258305753072e-10\\
28.11804296875	3.21838866996009e-10\\
28.1381923828125	2.02259275070307e-10\\
28.158341796875	3.06585143135513e-10\\
28.1784912109375	2.00078427427243e-10\\
28.198640625	2.56507163324774e-10\\
28.2187900390625	2.17418554425354e-10\\
28.238939453125	2.52489354341867e-10\\
28.2590888671875	1.56603284759759e-10\\
28.27923828125	1.34363887359646e-10\\
28.2993876953125	1.87427767943946e-10\\
28.319537109375	1.40534743920737e-10\\
28.3396865234375	2.20934038745161e-10\\
28.3598359375	3.16332429873982e-10\\
28.3799853515625	2.56597495951374e-10\\
28.400134765625	2.53483934151073e-10\\
28.4202841796875	2.70889622560502e-10\\
28.44043359375	3.23405156610034e-10\\
28.4605830078125	2.79175406611201e-10\\
28.480732421875	2.64536094981627e-10\\
28.5008818359375	1.85532562396093e-10\\
28.52103125	1.88868841418613e-10\\
28.5411806640625	2.24483641262292e-10\\
28.561330078125	2.54821361820004e-10\\
28.5814794921875	2.84860757210435e-10\\
28.60162890625	3.29859164551382e-10\\
28.6217783203125	3.32713932659276e-10\\
28.641927734375	3.29727054681978e-10\\
28.6620771484375	2.00493490353572e-10\\
28.6822265625	2.54230564294158e-10\\
28.7023759765625	2.63162420013579e-10\\
28.722525390625	2.28197081140841e-10\\
28.7426748046875	3.16560329302525e-10\\
28.76282421875	2.90645635003567e-10\\
28.7829736328125	2.76193556225296e-10\\
28.803123046875	3.41861681090282e-10\\
28.8232724609375	5.10580267225357e-10\\
28.843421875	4.20898648308319e-10\\
28.8635712890625	5.04442572315423e-10\\
28.883720703125	3.92430195921554e-10\\
28.9038701171875	4.44497407725032e-10\\
28.92401953125	4.6463630131833e-10\\
28.9441689453125	4.74190682693021e-10\\
28.964318359375	6.05014869331657e-10\\
28.9844677734375	4.66223014653682e-10\\
29.0046171875	4.34110927547164e-10\\
29.0247666015625	3.83718248111037e-10\\
29.044916015625	4.3313212841259e-10\\
29.0650654296875	3.76677099040403e-10\\
29.08521484375	4.38640973422112e-10\\
29.1053642578125	4.21506764114391e-10\\
29.125513671875	4.41197941659035e-10\\
29.1456630859375	5.09053013519328e-10\\
29.1658125	5.61715463713062e-10\\
29.1859619140625	5.51034306634493e-10\\
29.206111328125	5.12222210978508e-10\\
29.2262607421875	4.25546410312161e-10\\
29.24641015625	4.47769780481975e-10\\
29.2665595703125	3.93175521793446e-10\\
29.286708984375	4.2135830894663e-10\\
29.3068583984375	3.62027660059199e-10\\
29.3270078125	4.35473289789673e-10\\
29.3471572265625	4.47057870011416e-10\\
29.367306640625	4.69172641903475e-10\\
29.3874560546875	4.99689658705962e-10\\
29.40760546875	5.14765910296206e-10\\
29.4277548828125	3.46303763853954e-10\\
29.447904296875	4.22535402192609e-10\\
29.4680537109375	3.00126755548228e-10\\
29.488203125	2.88762051802038e-10\\
29.5083525390625	2.61714285494751e-10\\
29.528501953125	2.84831876622616e-10\\
29.5486513671875	3.54218276482002e-10\\
29.56880078125	3.24791355040295e-10\\
29.5889501953125	3.2235438541625e-10\\
29.609099609375	2.87163884946451e-10\\
29.6292490234375	2.31437519963714e-10\\
29.6493984375	1.32305062889363e-10\\
29.6695478515625	9.51340019342269e-11\\
29.689697265625	1.46645809628163e-10\\
29.7098466796875	1.32668375299752e-10\\
29.72999609375	9.6486298931482e-11\\
29.7501455078125	1.95873916927354e-10\\
29.770294921875	2.11730214914494e-10\\
29.7904443359375	1.5947470657669e-10\\
29.81059375	1.63884439004922e-10\\
29.8307431640625	1.31080423168888e-10\\
29.850892578125	1.50565091358725e-10\\
29.8710419921875	2.62495485155743e-11\\
29.89119140625	3.46903014894796e-11\\
29.9113408203125	-1.21335450518068e-12\\
29.931490234375	1.13132865813786e-10\\
29.9516396484375	1.51439394005448e-10\\
29.9717890625	1.79286993777659e-10\\
29.9919384765625	2.88158975135627e-10\\
30.012087890625	2.21615637036004e-10\\
30.0322373046875	2.91123834022165e-10\\
30.05238671875	1.41891309025647e-10\\
30.0725361328125	1.85755517287648e-10\\
30.092685546875	1.42204772847938e-10\\
30.1128349609375	6.6379534976707e-11\\
30.132984375	-4.85329753448352e-12\\
30.1531337890625	8.21585732032713e-11\\
30.173283203125	9.4245563139862e-11\\
30.1934326171875	1.75579773890091e-10\\
30.21358203125	2.75625268023603e-10\\
30.2337314453125	2.54762393876996e-10\\
30.253880859375	3.30163926165762e-10\\
30.2740302734375	2.98168214228373e-10\\
30.2941796875	2.47120595933946e-10\\
30.3143291015625	1.37241066476833e-10\\
30.334478515625	7.48458840509639e-11\\
30.3546279296875	1.54287123654911e-10\\
30.37477734375	1.85812654690688e-10\\
30.3949267578125	2.03573837131396e-10\\
30.415076171875	2.80465385174919e-10\\
30.4352255859375	3.28978137322108e-10\\
30.455375	2.19429988882181e-10\\
30.4755244140625	2.56363414149956e-10\\
30.495673828125	1.88915804162294e-10\\
30.5158232421875	9.47293988679164e-11\\
30.53597265625	1.26943719126219e-11\\
30.5561220703125	6.44331920988049e-11\\
30.576271484375	8.04581744563063e-12\\
30.5964208984375	6.12994258382357e-11\\
30.6165703125	2.51575043913861e-11\\
30.6367197265625	5.00290442286259e-11\\
30.656869140625	-2.90331039156016e-11\\
30.6770185546875	4.80474956200391e-11\\
30.69716796875	1.35108570191697e-11\\
30.7173173828125	-1.93321896810504e-11\\
30.737466796875	1.94328688172961e-11\\
30.7576162109375	-2.1807498747619e-11\\
30.777765625	1.44077512096587e-13\\
30.7979150390625	8.10743885060685e-11\\
30.818064453125	9.61418357671459e-11\\
30.8382138671875	1.50365619725992e-10\\
30.85836328125	5.30288190444397e-11\\
30.8785126953125	5.50430440083554e-11\\
30.898662109375	3.18521377369382e-11\\
30.9188115234375	-3.27504931220863e-11\\
30.9389609375	-3.40436840639609e-12\\
30.9591103515625	-6.54871271348663e-11\\
30.979259765625	-2.36432144497632e-11\\
30.9994091796875	-2.20233017407717e-11\\
31.01955859375	-9.01407699856202e-12\\
31.0397080078125	8.93211342363362e-11\\
31.059857421875	-7.02119719856068e-12\\
31.0800068359375	7.93684511905989e-11\\
31.10015625	5.80676651338372e-11\\
31.1203056640625	-2.40071267152858e-11\\
31.140455078125	5.07944123228441e-11\\
31.1606044921875	-3.00309700699463e-11\\
31.18075390625	-2.28404109517619e-11\\
31.2009033203125	-4.02441324716802e-11\\
31.221052734375	-1.24362769072907e-10\\
31.2412021484375	-1.74077962600032e-10\\
31.2613515625	-1.25283810868337e-10\\
31.2815009765625	-1.37683200773168e-10\\
31.301650390625	-3.20572050442401e-11\\
31.3217998046875	-1.209748272338e-10\\
31.34194921875	-2.34210209743468e-12\\
31.3620986328125	-2.76808113187095e-11\\
31.382248046875	-7.6815543587896e-11\\
31.4023974609375	-3.0798777057205e-11\\
31.422546875	-1.41574639002163e-10\\
31.4426962890625	-1.20414198589477e-10\\
31.462845703125	-1.66469936103916e-10\\
31.4829951171875	-2.68207433538738e-10\\
31.50314453125	-3.18863276316063e-10\\
31.5232939453125	-2.73721799616417e-10\\
31.543443359375	-2.35557522143662e-10\\
31.5635927734375	-1.83906805523547e-10\\
31.5837421875	-2.29463611660537e-10\\
31.6038916015625	-8.98452565820821e-11\\
31.624041015625	-9.06268886015925e-11\\
31.6441904296875	-4.83370735562749e-11\\
31.66433984375	-1.6590550339535e-10\\
31.6844892578125	-2.05372434448956e-10\\
31.704638671875	-3.21230501549091e-10\\
31.7247880859375	-4.0143512371825e-10\\
31.7449375	-4.30577216480025e-10\\
31.7650869140625	-3.98039075036507e-10\\
31.785236328125	-3.54773330433725e-10\\
31.8053857421875	-3.20698947805144e-10\\
31.82553515625	-2.29468663241011e-10\\
31.8456845703125	-1.75232502940151e-10\\
31.865833984375	-5.46358094040444e-11\\
31.8859833984375	-9.39217656030894e-12\\
31.9061328125	-9.34541244986921e-11\\
31.9262822265625	-9.20978323707699e-11\\
31.946431640625	-2.3579015431542e-10\\
31.9665810546875	-2.65886903782596e-10\\
31.98673046875	-2.10829031417385e-10\\
32.0068798828125	-2.09325855054474e-10\\
32.027029296875	-1.39324194622299e-10\\
32.0471787109375	-3.77279521011255e-11\\
32.067328125	-2.91345775711388e-11\\
32.0874775390625	-4.24190783463293e-11\\
32.107626953125	-2.96592827895472e-11\\
32.1277763671875	-1.11417134367628e-10\\
32.14792578125	-2.29276259077286e-10\\
32.1680751953125	-2.75743657476948e-10\\
32.188224609375	-3.42711267166408e-10\\
32.2083740234375	-2.32433241237609e-10\\
32.2285234375	-1.18567714559297e-10\\
32.2486728515625	-1.00138380582576e-10\\
32.268822265625	-1.63516262776898e-11\\
32.2889716796875	6.00289442818402e-11\\
32.30912109375	6.23008854861239e-11\\
32.3292705078125	-3.50109129313473e-11\\
32.349419921875	-8.53532208745183e-11\\
32.3695693359375	-1.05746970331857e-10\\
32.38971875	-1.11627440060134e-10\\
32.4098681640625	-1.46098785094782e-10\\
32.430017578125	-1.14917870240613e-10\\
32.4501669921875	-1.06950366632449e-10\\
32.47031640625	-3.66573826474381e-11\\
32.4904658203125	-6.84529957301121e-11\\
32.510615234375	-1.43913074353412e-12\\
32.5307646484375	1.599634934146e-11\\
32.5509140625	-5.90255466360576e-11\\
32.5710634765625	2.77218288470586e-11\\
32.591212890625	-5.87710842429826e-11\\
32.6113623046875	-1.63605816663526e-11\\
32.63151171875	-3.15808943115428e-11\\
32.6516611328125	-6.45315852012945e-11\\
32.671810546875	-4.36507618267979e-11\\
32.6919599609375	5.64163716352208e-11\\
32.712109375	-1.01876187218333e-11\\
32.7322587890625	1.86694580727228e-12\\
32.752408203125	4.86810245231583e-11\\
32.7725576171875	4.17112928549581e-11\\
32.79270703125	4.73172748782518e-11\\
32.8128564453125	2.44029625928584e-12\\
32.833005859375	9.83818633799037e-12\\
32.8531552734375	-6.60677868169969e-11\\
32.8733046875	-9.28209269306619e-12\\
32.8934541015625	2.31031700585136e-11\\
32.913603515625	2.68995030888176e-11\\
32.9337529296875	5.9290167521572e-11\\
32.95390234375	1.08996580232053e-10\\
32.9740517578125	6.97219745424587e-12\\
32.994201171875	4.18489683325575e-11\\
33.0143505859375	5.44437270755331e-11\\
33.0345	4.54242885356033e-11\\
33.0546494140625	-7.612478042936e-12\\
33.074798828125	8.42835354322478e-11\\
33.0949482421875	1.10122860152522e-10\\
33.11509765625	8.92408434532358e-11\\
33.1352470703125	1.4659943976865e-10\\
33.155396484375	1.58518767811209e-10\\
33.1755458984375	1.36921970811381e-10\\
33.1956953125	1.63429515351898e-10\\
33.2158447265625	1.48189828196871e-10\\
33.235994140625	1.40980144006618e-10\\
33.2561435546875	1.93226948863449e-10\\
33.27629296875	2.28384108944143e-10\\
33.2964423828125	2.21050346756588e-10\\
33.316591796875	2.01075258899118e-10\\
33.3367412109375	1.9065913967222e-10\\
33.356890625	1.00365631100455e-10\\
33.3770400390625	1.74135692221534e-10\\
33.397189453125	1.39739715078102e-10\\
33.4173388671875	2.69760662153686e-10\\
33.43748828125	3.12966224273874e-10\\
33.4576376953125	2.51797493673882e-10\\
33.477787109375	3.65365630172001e-10\\
33.4979365234375	2.76966649326811e-10\\
33.5180859375	2.77425576186454e-10\\
33.5382353515625	1.58619137574406e-10\\
33.558384765625	1.85043230668834e-10\\
33.5785341796875	1.5242520636884e-10\\
33.59868359375	1.55378972402952e-10\\
33.6188330078125	1.6642594065372e-10\\
33.638982421875	2.66242728335374e-10\\
33.6591318359375	2.17574778706529e-10\\
33.67928125	3.69615938971725e-10\\
33.6994306640625	3.39010811659109e-10\\
33.719580078125	3.95452633522133e-10\\
33.7397294921875	2.99097251202877e-10\\
33.75987890625	2.190360254805e-10\\
33.7800283203125	2.733942927824e-10\\
33.800177734375	2.49259594404826e-10\\
33.8203271484375	1.57659915582523e-10\\
33.8404765625	2.33904185465835e-10\\
33.8606259765625	2.31385397943631e-10\\
33.880775390625	3.44238511281164e-10\\
33.9009248046875	4.15681307768612e-10\\
33.92107421875	4.37661956631997e-10\\
33.9412236328125	5.09742237931746e-10\\
33.961373046875	4.30666833218192e-10\\
33.9815224609375	3.64228587771062e-10\\
34.001671875	4.31288558230484e-10\\
34.0218212890625	3.48478626857116e-10\\
34.041970703125	3.44796297666355e-10\\
34.0621201171875	2.97305284532803e-10\\
34.08226953125	3.38807472843788e-10\\
34.1024189453125	3.29457432397852e-10\\
34.122568359375	4.13287101882673e-10\\
34.1427177734375	4.21479517872409e-10\\
34.1628671875	4.39104621409019e-10\\
34.1830166015625	4.9251311202815e-10\\
34.203166015625	4.86472406060594e-10\\
34.2233154296875	6.11437452009477e-10\\
34.24346484375	5.09378335781277e-10\\
34.2636142578125	4.46702039845395e-10\\
34.283763671875	4.117399778681e-10\\
34.3039130859375	3.35486474757982e-10\\
34.3240625	3.46749514003515e-10\\
34.3442119140625	3.21212716034827e-10\\
34.364361328125	2.55868703413473e-10\\
34.3845107421875	2.52598948231416e-10\\
34.40466015625	2.84074461773121e-10\\
34.4248095703125	3.78662074587134e-10\\
34.444958984375	3.51220913690622e-10\\
34.4651083984375	3.71608007526534e-10\\
34.4852578125	3.22796803472536e-10\\
34.5054072265625	3.00462802704556e-10\\
34.525556640625	2.02858486173228e-10\\
34.5457060546875	2.06851297853444e-10\\
34.56585546875	2.22925547113281e-10\\
34.5860048828125	2.45051017683381e-10\\
34.606154296875	3.39141753667483e-10\\
34.6263037109375	3.41929210693473e-10\\
34.646453125	3.76091474361166e-10\\
34.6666025390625	3.49622763794791e-10\\
34.686751953125	2.95753092289797e-10\\
34.7069013671875	2.61570439686382e-10\\
34.72705078125	2.5236306396587e-10\\
34.7472001953125	2.08012835122044e-10\\
34.767349609375	1.51885651343325e-10\\
34.7874990234375	1.63473481816362e-10\\
34.8076484375	1.71857120340408e-10\\
34.8277978515625	2.01957802889184e-10\\
34.847947265625	2.3804893765568e-10\\
34.8680966796875	3.06789179416815e-10\\
34.88824609375	3.17795290205647e-10\\
34.9083955078125	3.0017481382014e-10\\
34.928544921875	2.75896786251668e-10\\
34.9486943359375	2.13721572822749e-10\\
34.96884375	2.78610624642016e-10\\
34.9889931640625	1.73514690065703e-10\\
35.009142578125	2.72024835521762e-10\\
35.0292919921875	1.9068918567607e-10\\
35.04944140625	2.57175703939712e-10\\
35.0695908203125	2.27774264656773e-10\\
35.089740234375	3.00888865203074e-10\\
35.1098896484375	2.95530217187366e-10\\
35.1300390625	2.49119051104535e-10\\
35.1501884765625	2.639728264949e-10\\
35.170337890625	2.58452583201069e-10\\
35.1904873046875	2.76794446141917e-10\\
35.21063671875	1.43368175839597e-10\\
35.2307861328125	2.41989374318265e-10\\
35.250935546875	2.46347493750077e-10\\
35.2710849609375	2.82394453793899e-10\\
35.291234375	2.57906086803839e-10\\
35.3113837890625	2.97317389235872e-10\\
35.331533203125	3.12803035993477e-10\\
35.3516826171875	2.77738466552639e-10\\
35.37183203125	2.52218620025149e-10\\
35.3919814453125	2.5270519277037e-10\\
35.412130859375	2.13210377377655e-10\\
35.4322802734375	2.08548455128835e-10\\
};
\addplot [color=mycolor1,solid]
  table[row sep=crcr]{%
35.4322802734375	2.08548455128835e-10\\
35.4524296875	1.57626523016495e-10\\
35.4725791015625	1.79340721740246e-10\\
35.492728515625	1.21424515627835e-10\\
35.5128779296875	3.38458336966072e-11\\
35.53302734375	8.87858594076494e-11\\
35.5531767578125	2.95578479036608e-11\\
35.573326171875	7.25758593452401e-11\\
35.5934755859375	-6.51533084134126e-11\\
35.613625	-1.28861729312484e-10\\
35.6337744140625	-7.309099683147e-11\\
35.653923828125	-9.53687450383912e-11\\
35.6740732421875	-1.21953380393282e-10\\
35.69422265625	-1.26863852705396e-10\\
35.7143720703125	-2.05262032581983e-10\\
35.734521484375	-2.22404866735985e-10\\
35.7546708984375	-1.97638042462664e-10\\
35.7748203125	-6.33425869633105e-11\\
35.7949697265625	-1.05025466480816e-10\\
35.815119140625	-3.55572765935132e-11\\
35.8352685546875	-1.05846099919997e-10\\
35.85541796875	-1.05867630693016e-10\\
35.8755673828125	-5.53297191662778e-11\\
35.895716796875	-1.00984927368753e-10\\
35.9158662109375	-1.8016923816064e-10\\
35.936015625	-2.58428406222563e-10\\
35.9561650390625	-2.57349640617604e-10\\
35.976314453125	-2.33439141168524e-10\\
35.9964638671875	-1.60381159183842e-10\\
36.01661328125	-9.19114066000881e-11\\
36.0367626953125	-3.17950081143995e-11\\
36.056912109375	5.52857253051987e-11\\
36.0770615234375	8.80625254494245e-11\\
36.0972109375	3.80475457330424e-11\\
36.1173603515625	4.5436096679311e-11\\
36.137509765625	-3.38105301379352e-11\\
36.1576591796875	-1.10407548897508e-10\\
36.17780859375	-5.95901989942686e-11\\
36.1979580078125	-7.9355655582966e-11\\
36.218107421875	-2.64957342528049e-11\\
36.2382568359375	1.23154282315941e-10\\
36.25840625	1.59523448137755e-10\\
36.2785556640625	2.89590269167901e-10\\
36.298705078125	2.08817891482227e-10\\
36.3188544921875	1.58059491991885e-10\\
36.33900390625	1.09677742762126e-10\\
36.3591533203125	3.37401577702965e-11\\
36.379302734375	-3.92207121503601e-11\\
36.3994521484375	-3.45026768953598e-11\\
36.4196015625	-1.56306326941698e-10\\
36.4397509765625	-7.19904676813321e-11\\
36.459900390625	8.86900440222398e-12\\
36.4800498046875	1.42964833121323e-10\\
36.50019921875	1.06170958268567e-10\\
36.5203486328125	1.01663096036362e-10\\
36.540498046875	1.37332145904716e-10\\
36.5606474609375	8.43554079157311e-11\\
36.580796875	-1.37264450984043e-11\\
36.6009462890625	-1.22356616466301e-10\\
36.621095703125	-1.86134687085689e-10\\
36.6412451171875	-1.55281427855921e-10\\
36.66139453125	-1.26207100568936e-10\\
36.6815439453125	-1.33971368677509e-10\\
36.701693359375	-4.22233409660961e-12\\
36.7218427734375	2.70374560331873e-11\\
36.7419921875	-5.39913569115318e-12\\
36.7621416015625	-1.24104001263093e-11\\
36.782291015625	9.43958178116577e-11\\
36.8024404296875	-6.93333598756161e-11\\
36.82258984375	-4.06554299828148e-11\\
36.8427392578125	-6.16849284769561e-11\\
36.862888671875	-3.22092636793863e-12\\
36.8830380859375	-2.37529096645044e-11\\
36.9031875	1.66140955878475e-11\\
36.9233369140625	-5.47144621419897e-11\\
36.943486328125	4.93241159969188e-11\\
36.9636357421875	-3.3889500482967e-11\\
36.98378515625	3.70055156652263e-11\\
37.0039345703125	9.52616391298915e-11\\
37.024083984375	1.23925167608637e-10\\
37.0442333984375	1.56947557151352e-10\\
37.0643828125	1.88477440452849e-10\\
37.0845322265625	1.50743143902211e-10\\
37.104681640625	1.75285997125467e-10\\
37.1248310546875	1.57856417181465e-10\\
37.14498046875	9.24202442589652e-11\\
37.1651298828125	1.16364106803353e-10\\
37.185279296875	8.19200037609305e-11\\
37.2054287109375	2.00488070691767e-10\\
37.225578125	1.58512641870517e-10\\
37.2457275390625	2.1096218323835e-10\\
37.265876953125	2.70728300219638e-10\\
37.2860263671875	1.89801929139616e-10\\
37.30617578125	1.6552627623122e-10\\
37.3263251953125	2.19108816956396e-10\\
37.346474609375	2.11467182739736e-10\\
37.3666240234375	1.43991199342973e-10\\
37.3867734375	1.4591056213527e-10\\
37.4069228515625	1.15448627104602e-10\\
37.427072265625	1.05575457983473e-10\\
37.4472216796875	9.90702834958915e-11\\
37.46737109375	4.46230527390174e-11\\
37.4875205078125	6.37635972269817e-11\\
37.507669921875	6.09552003191262e-11\\
37.5278193359375	-3.99945441874871e-12\\
37.54796875	4.88300874964092e-11\\
37.5681181640625	4.449821855569e-11\\
37.588267578125	1.36689103626292e-10\\
37.6084169921875	7.18002124787436e-11\\
37.62856640625	9.82209742482586e-11\\
37.6487158203125	1.35619834314104e-10\\
37.668865234375	1.93108579154261e-10\\
37.6890146484375	1.26067146416366e-10\\
37.7091640625	1.36197470389372e-10\\
37.7293134765625	1.49610066622513e-10\\
37.749462890625	1.47390780776686e-10\\
37.7696123046875	2.22488901028707e-10\\
37.78976171875	1.39366364968048e-10\\
37.8099111328125	1.52097324910034e-10\\
37.830060546875	1.11859116542119e-10\\
37.8502099609375	1.46588167743713e-10\\
37.870359375	1.13676813417376e-10\\
37.8905087890625	1.71786034617043e-10\\
37.910658203125	2.14886427113065e-10\\
37.9308076171875	2.7845768068359e-10\\
37.95095703125	2.3588168403765e-10\\
37.9711064453125	2.42906718121501e-10\\
37.991255859375	2.76058917489603e-10\\
38.0114052734375	2.85672545064574e-10\\
38.0315546875	2.64414673361573e-10\\
38.0517041015625	2.3132017311694e-10\\
38.071853515625	2.44868506745926e-10\\
38.0920029296875	2.83058151361495e-10\\
38.11215234375	3.95733791978521e-10\\
38.1323017578125	4.27319651094385e-10\\
38.152451171875	4.72896426208647e-10\\
38.1726005859375	4.75068753988018e-10\\
38.19275	4.38839637599961e-10\\
38.2128994140625	4.06292834762675e-10\\
38.233048828125	4.34954705436354e-10\\
38.2531982421875	3.8006246560302e-10\\
38.27334765625	3.59806374939931e-10\\
38.2934970703125	3.98345829942918e-10\\
38.313646484375	4.01627527776401e-10\\
38.3337958984375	4.34883054312965e-10\\
38.3539453125	4.37227614413369e-10\\
38.3740947265625	4.86823216468381e-10\\
38.394244140625	4.37944812817453e-10\\
38.4143935546875	3.94936825731129e-10\\
38.43454296875	3.73826935129205e-10\\
38.4546923828125	3.06066428552616e-10\\
38.474841796875	2.83257090049238e-10\\
38.4949912109375	3.34106800744996e-10\\
38.515140625	3.37980915355997e-10\\
38.5352900390625	3.90837750110545e-10\\
38.555439453125	3.85467128216475e-10\\
38.5755888671875	3.94941601774606e-10\\
38.59573828125	4.17645098391514e-10\\
38.6158876953125	3.40441203904723e-10\\
38.636037109375	3.25772796374956e-10\\
38.6561865234375	2.26400060780186e-10\\
38.6763359375	2.39090779691946e-10\\
38.6964853515625	1.26912462873068e-10\\
38.716634765625	1.48369500453191e-10\\
38.7367841796875	1.62727780417095e-10\\
38.75693359375	1.80517948642423e-10\\
38.7770830078125	2.04093093234875e-10\\
38.797232421875	2.05172815339756e-10\\
38.8173818359375	2.31088495363127e-10\\
38.83753125	1.84102613007503e-10\\
38.8576806640625	1.26747812321966e-10\\
38.877830078125	1.51985575160895e-10\\
38.8979794921875	9.26322738132272e-11\\
38.91812890625	5.38742532687608e-11\\
38.9382783203125	1.13293093920417e-10\\
38.958427734375	1.23653199199255e-10\\
38.9785771484375	7.21360415871528e-11\\
38.9987265625	9.95026474520862e-11\\
39.0188759765625	1.37351286865272e-10\\
39.039025390625	1.9908272105609e-10\\
39.0591748046875	1.47245373263786e-10\\
39.07932421875	1.72166519254305e-10\\
39.0994736328125	2.73019979702749e-10\\
39.119623046875	2.22782101818495e-10\\
39.1397724609375	1.46501105423469e-10\\
39.159921875	2.36962425668871e-10\\
39.1800712890625	2.03956557283453e-10\\
39.200220703125	1.68740639284762e-10\\
39.2203701171875	1.69245437460216e-10\\
39.24051953125	1.33931249116785e-10\\
39.2606689453125	1.5432777929573e-10\\
39.280818359375	1.41870170519976e-10\\
39.3009677734375	1.70404026880409e-10\\
39.3211171875	2.07002975683104e-10\\
39.3412666015625	1.88345070823906e-10\\
39.361416015625	2.09138867632138e-10\\
39.3815654296875	1.50390866904942e-10\\
39.40171484375	9.910186310284e-11\\
39.4218642578125	1.64114892624607e-10\\
39.442013671875	8.06173729184155e-11\\
39.4621630859375	1.08705876269486e-10\\
39.4823125	9.01879887470997e-11\\
39.5024619140625	7.64056308739846e-12\\
39.522611328125	3.97978750731111e-11\\
39.5427607421875	4.09714697013056e-11\\
39.56291015625	3.87858933854741e-11\\
39.5830595703125	1.01400475044977e-11\\
39.603208984375	8.25020749384986e-12\\
39.6233583984375	-2.12268110052814e-11\\
39.6435078125	-3.82238746460024e-12\\
39.6636572265625	-1.1395781964179e-12\\
39.683806640625	-2.1519252361858e-11\\
39.7039560546875	-5.61938161105317e-11\\
39.72410546875	-8.08472056507464e-11\\
39.7442548828125	-9.10104471378313e-11\\
39.764404296875	-5.31181824860112e-11\\
39.7845537109375	-6.48117488310118e-11\\
39.804703125	-1.4385435392723e-10\\
39.8248525390625	-1.12070195516336e-10\\
39.845001953125	-1.41736244927888e-10\\
39.8651513671875	-1.0483836650549e-10\\
39.88530078125	-1.04805798107109e-10\\
39.9054501953125	-8.97218134789538e-11\\
39.925599609375	-1.08571010799637e-10\\
39.9457490234375	-1.35409012471024e-11\\
39.9658984375	-6.72660833474468e-12\\
39.9860478515625	-1.07810650770457e-10\\
40.006197265625	-9.67191407374938e-11\\
40.0263466796875	-1.03172506672646e-10\\
40.04649609375	-6.73475441272765e-11\\
40.0666455078125	-1.08874436303357e-10\\
40.086794921875	-1.01366227278909e-10\\
40.1069443359375	-1.67496847119756e-10\\
40.12709375	-1.78147310918662e-10\\
40.1472431640625	-1.08898008185523e-10\\
40.167392578125	-1.94851984646859e-10\\
40.1875419921875	-1.40920361823852e-10\\
40.20769140625	-1.60260450604076e-10\\
40.2278408203125	-1.19939586929248e-10\\
40.247990234375	-1.57510077248839e-10\\
40.2681396484375	-1.29583734213759e-10\\
40.2882890625	-1.75444122269765e-10\\
40.3084384765625	-1.24446813843412e-10\\
40.328587890625	-9.40460369451458e-11\\
40.3487373046875	-1.01785098734416e-10\\
40.36888671875	-8.28287136114835e-11\\
40.3890361328125	-9.60310376864314e-11\\
40.409185546875	-1.05870016700327e-10\\
40.4293349609375	-1.17582078852799e-10\\
40.449484375	-1.93827459839826e-10\\
40.4696337890625	-2.0097923435504e-10\\
40.489783203125	-2.20654899542062e-10\\
40.5099326171875	-2.23859678502846e-10\\
40.53008203125	-1.90574242980513e-10\\
40.5502314453125	-1.66581912145914e-10\\
40.570380859375	-1.71941841286094e-10\\
40.5905302734375	-1.51744803176086e-10\\
40.6106796875	-1.28764390199351e-10\\
40.6308291015625	-2.29999186500405e-10\\
40.650978515625	-1.80740254760885e-10\\
40.6711279296875	-2.44983493432655e-10\\
40.69127734375	-2.61020556858752e-10\\
40.7114267578125	-2.75241574269905e-10\\
40.731576171875	-2.08894634743258e-10\\
40.7517255859375	-2.98324183737178e-10\\
40.771875	-2.07967158865701e-10\\
40.7920244140625	-1.96648727971686e-10\\
40.812173828125	-1.6381710196449e-10\\
40.8323232421875	-1.91380841526816e-10\\
40.85247265625	-2.74865538358329e-10\\
40.8726220703125	-2.67988169153278e-10\\
40.892771484375	-3.04670231331589e-10\\
40.9129208984375	-2.62296924506142e-10\\
40.9330703125	-2.5105562506054e-10\\
40.9532197265625	-1.85763482095423e-10\\
40.973369140625	-1.49658627407388e-10\\
40.9935185546875	-1.99120967295021e-10\\
41.01366796875	-1.72785762704415e-10\\
41.0338173828125	-1.63243541706374e-10\\
41.053966796875	-1.93581451532929e-10\\
41.0741162109375	-2.40829137002706e-10\\
41.094265625	-1.86665110042834e-10\\
41.1144150390625	-2.19543766451504e-10\\
41.134564453125	-8.9479933798566e-11\\
41.1547138671875	-1.04038837831482e-10\\
41.17486328125	-5.17716448376594e-11\\
41.1950126953125	-1.20720170386787e-10\\
41.215162109375	-1.49297127932416e-11\\
41.2353115234375	-4.64781404033107e-12\\
41.2554609375	-1.14728480246081e-10\\
41.2756103515625	-6.14563580893189e-11\\
41.295759765625	-7.41300320820258e-11\\
41.3159091796875	-9.70583757486375e-11\\
41.33605859375	-5.6812758309404e-11\\
41.3562080078125	-9.65020249014508e-13\\
41.376357421875	-4.08097273008921e-11\\
41.3965068359375	-3.62015233795369e-11\\
41.41665625	-3.12104214819438e-11\\
41.4368056640625	3.40157206186405e-11\\
41.456955078125	-4.19477329162248e-11\\
41.4771044921875	3.75525725796756e-11\\
41.49725390625	2.70938534603047e-11\\
41.5174033203125	-5.83318204645277e-11\\
41.537552734375	-4.83988857543613e-11\\
41.5577021484375	-1.87218092722918e-11\\
41.5778515625	-7.62433159499659e-11\\
41.5980009765625	-6.99347071600484e-11\\
41.618150390625	-9.61469388746961e-11\\
41.6382998046875	-1.13714537161963e-10\\
41.65844921875	-1.27185048423457e-10\\
41.6785986328125	-1.51450879572409e-10\\
41.698748046875	-1.69965807050959e-10\\
41.7188974609375	-9.85646816644362e-11\\
41.739046875	-1.0479309873506e-10\\
41.7591962890625	-1.18550555672708e-10\\
41.779345703125	-5.53845339647093e-11\\
41.7994951171875	-1.45906193025374e-10\\
41.81964453125	-1.14007788186897e-10\\
41.8397939453125	-1.29686499795883e-10\\
41.859943359375	-1.38388499667249e-10\\
41.8800927734375	-1.74712236441916e-10\\
41.9002421875	-1.47816788055988e-10\\
41.9203916015625	-9.69159687156636e-11\\
41.940541015625	-1.42062731350629e-10\\
41.9606904296875	-8.78671387271628e-11\\
41.98083984375	-1.08852270703775e-10\\
42.0009892578125	-9.29395367151288e-11\\
42.021138671875	-1.31539293934229e-10\\
42.0412880859375	-9.59052461913093e-11\\
42.0614375	-5.73788393483803e-11\\
42.0815869140625	-9.67340544247148e-11\\
42.101736328125	-4.78925203642373e-11\\
42.1218857421875	-3.09517448700265e-11\\
42.14203515625	-1.16450223984637e-11\\
42.1621845703125	3.39456612472001e-13\\
42.182333984375	1.9685547401862e-11\\
42.2024833984375	5.57091818922012e-11\\
42.2226328125	5.51025636059585e-12\\
42.2427822265625	3.80331626137542e-11\\
42.262931640625	6.09713895731417e-11\\
42.2830810546875	7.24006444698854e-11\\
42.30323046875	5.19219614688455e-11\\
42.3233798828125	4.73137317260688e-11\\
42.343529296875	9.32292107835959e-11\\
42.3636787109375	1.240803779471e-10\\
42.383828125	9.47719251131267e-11\\
42.4039775390625	9.63591967178028e-11\\
42.424126953125	6.55110502350947e-11\\
42.4442763671875	9.02591388974784e-11\\
42.46442578125	1.07559924081936e-10\\
42.4845751953125	8.90337271230411e-11\\
42.504724609375	1.14263921341755e-10\\
42.5248740234375	5.89590835519849e-11\\
42.5450234375	1.02397432397758e-10\\
42.5651728515625	8.70083663291663e-11\\
42.585322265625	1.01277812638113e-10\\
42.6054716796875	1.19254900223693e-10\\
42.62562109375	1.48011797635912e-10\\
42.6457705078125	9.80861746047868e-11\\
42.665919921875	6.86287268543734e-11\\
42.6860693359375	9.24432916661401e-11\\
42.70621875	2.98532339343913e-11\\
42.7263681640625	6.09064538344779e-11\\
42.746517578125	1.15446096401105e-10\\
42.7666669921875	7.30602494233367e-11\\
42.78681640625	-2.43478133164141e-11\\
42.8069658203125	8.77308996780184e-11\\
42.827115234375	7.27799783768693e-11\\
42.8472646484375	6.16875156175629e-11\\
42.8674140625	1.09914975808249e-10\\
42.8875634765625	1.28307406419174e-10\\
42.907712890625	7.28807472474551e-11\\
42.9278623046875	2.86709606056836e-11\\
42.94801171875	6.14110455430421e-11\\
42.9681611328125	1.58818758898246e-10\\
42.988310546875	1.42940983273251e-10\\
43.0084599609375	1.86690413564733e-10\\
43.028609375	1.876665511357e-10\\
43.0487587890625	9.69472022968509e-11\\
43.068908203125	1.86524205356598e-10\\
43.0890576171875	1.57388963941436e-10\\
43.10920703125	1.09468187620841e-10\\
43.1293564453125	1.40175896377552e-10\\
43.149505859375	9.60393474479818e-11\\
43.1696552734375	9.48243493702467e-11\\
43.1898046875	1.64716905383186e-10\\
43.2099541015625	2.03456193213241e-10\\
43.230103515625	1.35186997713433e-10\\
43.2502529296875	2.03637775492343e-10\\
43.27040234375	2.15142107559072e-10\\
43.2905517578125	1.82230820281812e-10\\
43.310701171875	1.77388788041696e-10\\
43.3308505859375	1.563366975399e-10\\
43.351	8.63549812451828e-11\\
43.3711494140625	1.42047286751699e-10\\
43.391298828125	8.76583229253374e-11\\
43.4114482421875	2.35665339860954e-10\\
43.43159765625	1.5470810296236e-10\\
43.4517470703125	2.32390213744178e-10\\
43.471896484375	2.17080300562857e-10\\
43.4920458984375	1.75512518075458e-10\\
43.5121953125	1.8394184334776e-10\\
43.5323447265625	1.71059362089561e-10\\
43.552494140625	1.31075076725084e-10\\
43.5726435546875	1.25750307918657e-10\\
43.59279296875	1.28299931732649e-10\\
43.6129423828125	1.42326377931866e-10\\
43.633091796875	1.57262030705785e-10\\
43.6532412109375	7.7919687013171e-11\\
43.673390625	1.08600382893933e-10\\
43.6935400390625	6.78731284387436e-11\\
43.713689453125	7.26973222020942e-11\\
43.7338388671875	3.93959016582461e-11\\
43.75398828125	1.7071791754651e-11\\
43.7741376953125	9.44064112622782e-11\\
43.794287109375	4.09354185672134e-11\\
43.8144365234375	5.29699333077133e-11\\
43.8345859375	6.8742121610524e-11\\
43.8547353515625	-1.42752649708306e-11\\
43.874884765625	3.2722363949255e-11\\
43.8950341796875	-2.63877241098074e-11\\
43.91518359375	7.78779352098728e-12\\
43.9353330078125	-1.02716231083323e-11\\
43.955482421875	-5.1778808136632e-11\\
43.9756318359375	3.03201480036128e-11\\
43.99578125	2.47789046345349e-11\\
44.0159306640625	-3.76565269850937e-12\\
44.036080078125	1.01899703478542e-10\\
44.0562294921875	1.4001678209012e-10\\
44.07637890625	7.12506807327289e-11\\
44.0965283203125	1.22261576740247e-10\\
44.116677734375	5.17260657274605e-11\\
44.1368271484375	2.40954989084663e-11\\
44.1569765625	5.21048405536816e-11\\
44.1771259765625	3.71767312272628e-11\\
44.197275390625	1.11135355183577e-10\\
44.2174248046875	8.03349467718022e-11\\
44.23757421875	1.57992167857972e-10\\
44.2577236328125	1.61931176955509e-10\\
44.277873046875	2.40384548653091e-10\\
44.2980224609375	1.83709815892321e-10\\
44.318171875	1.28789370523169e-10\\
44.3383212890625	1.03584736585996e-10\\
44.358470703125	1.19273268015522e-10\\
44.3786201171875	5.0800568463856e-11\\
44.39876953125	2.84442892886461e-11\\
44.4189189453125	4.16163449726584e-11\\
44.439068359375	9.51094271986383e-11\\
44.4592177734375	8.40388511630424e-11\\
44.4793671875	1.40622697350757e-10\\
44.4995166015625	1.4404804267532e-10\\
44.519666015625	1.12272245134998e-10\\
44.5398154296875	1.36391046905791e-10\\
44.55996484375	6.32675651380757e-11\\
44.5801142578125	4.02602085684361e-11\\
44.600263671875	5.48641253907737e-11\\
44.6204130859375	-3.05126255471163e-11\\
44.6405625	-3.18094817232965e-11\\
44.6607119140625	2.20573556103073e-11\\
44.680861328125	-5.4632072390153e-12\\
44.7010107421875	-7.55052954921873e-12\\
44.72116015625	-3.53934944503331e-11\\
44.7413095703125	-2.77842667396711e-11\\
44.761458984375	-4.28204037108894e-11\\
44.7816083984375	-1.22969231706758e-10\\
44.8017578125	-1.25956904792055e-10\\
44.8219072265625	-5.3330445802355e-11\\
44.842056640625	-1.60917787426349e-10\\
44.8622060546875	-1.06090263338647e-10\\
44.88235546875	-5.66030510227641e-11\\
44.9025048828125	-6.53483011404463e-11\\
44.922654296875	-2.34378996349518e-11\\
44.9428037109375	-9.66737696637002e-11\\
44.962953125	-3.38488066318802e-11\\
44.9831025390625	-9.4925855225897e-11\\
45.003251953125	-1.22027443203997e-10\\
45.0234013671875	-1.30918410905025e-10\\
45.04355078125	-1.08568081550449e-10\\
45.0637001953125	-1.68605826867737e-10\\
45.083849609375	-1.14429676666146e-10\\
45.1039990234375	-8.52699557293677e-11\\
45.1241484375	-1.43560059086623e-11\\
45.1442978515625	-6.16214038718477e-11\\
45.164447265625	-6.17286135108552e-11\\
45.1845966796875	-4.32300778400579e-11\\
45.20474609375	-7.34494964292674e-11\\
45.2248955078125	-1.02097093000351e-10\\
45.245044921875	-9.7248816587423e-11\\
45.2651943359375	-1.97927863407068e-10\\
45.28534375	-1.62928187231978e-10\\
45.3054931640625	-1.62455700731737e-10\\
45.325642578125	-1.25357126829312e-10\\
45.3457919921875	-1.01343915144415e-10\\
45.36594140625	-6.94845647118238e-11\\
45.3860908203125	-4.61788764324554e-11\\
45.406240234375	-1.08016725208711e-10\\
45.4263896484375	-1.22861649989057e-10\\
45.4465390625	-1.67403389891262e-10\\
45.4666884765625	-1.52760654074667e-10\\
45.486837890625	-2.12252594176725e-10\\
45.5069873046875	-1.69303735940115e-10\\
45.52713671875	-2.18323310805463e-10\\
45.5472861328125	-1.44676398299587e-10\\
45.567435546875	-2.01157459331292e-10\\
45.5875849609375	-1.91951215040081e-10\\
45.607734375	-1.5955266517759e-10\\
45.6278837890625	-1.78744558712466e-10\\
45.648033203125	-1.92841349879459e-10\\
45.6681826171875	-1.52447360868157e-10\\
45.68833203125	-1.94096171524033e-10\\
45.7084814453125	-1.50601208394011e-10\\
45.728630859375	-2.17306076004735e-10\\
45.7487802734375	-2.29936758889897e-10\\
45.7689296875	-2.19480916226825e-10\\
45.7890791015625	-2.43277792675812e-10\\
45.809228515625	-2.6747428790563e-10\\
45.8293779296875	-2.5411147956669e-10\\
45.84952734375	-2.81543851004052e-10\\
45.8696767578125	-2.31634134076085e-10\\
45.889826171875	-2.80329532910264e-10\\
45.9099755859375	-1.77811957019947e-10\\
45.930125	-1.62444987259236e-10\\
45.9502744140625	-2.03842391319128e-10\\
45.970423828125	-1.65472469439667e-10\\
45.9905732421875	-1.90504628026224e-10\\
46.01072265625	-2.46721388714246e-10\\
46.0308720703125	-1.66288300667244e-10\\
46.051021484375	-2.22054809352955e-10\\
46.0711708984375	-1.75475911183861e-10\\
46.0913203125	-1.94775390279682e-10\\
46.1114697265625	-1.96687189700983e-10\\
46.131619140625	-1.18186330521701e-10\\
46.1517685546875	-1.016648117135e-10\\
46.17191796875	-1.5124896676621e-10\\
46.1920673828125	-1.08890610010774e-10\\
46.212216796875	-1.06437841886178e-10\\
46.2323662109375	-8.79336807448482e-11\\
46.252515625	-6.67594525418517e-11\\
46.2726650390625	-1.04435321709162e-10\\
46.292814453125	-1.04706881780073e-10\\
46.3129638671875	-7.84436737053522e-11\\
46.33311328125	-1.20093917744825e-10\\
46.3532626953125	-9.45632307382584e-11\\
46.373412109375	-3.44182383601953e-11\\
46.3935615234375	-7.58322207183939e-11\\
46.4137109375	-1.10097551693826e-11\\
46.4338603515625	-4.37984342343962e-11\\
46.454009765625	-2.77106631709055e-11\\
46.4741591796875	-6.02149523886753e-11\\
46.49430859375	-1.03166492992926e-10\\
46.5144580078125	-1.06933836111689e-10\\
46.534607421875	-6.18438274169824e-11\\
46.5547568359375	-1.85018289062091e-10\\
46.57490625	-1.44542798712725e-10\\
46.5950556640625	-1.66276356134325e-10\\
46.615205078125	-1.50273369843042e-10\\
46.6353544921875	-1.02600472658128e-10\\
46.65550390625	-1.43506303932556e-10\\
46.6756533203125	-1.31365330900711e-10\\
46.695802734375	-1.61579522080825e-10\\
46.7159521484375	-1.45249691428204e-10\\
46.7361015625	-2.03840356074285e-10\\
46.7562509765625	-2.69977871320071e-10\\
46.776400390625	-2.33931976536564e-10\\
46.7965498046875	-1.92231094405866e-10\\
46.81669921875	-1.75555361904471e-10\\
46.8368486328125	-2.5977597662615e-10\\
46.856998046875	-2.57973193958762e-10\\
46.8771474609375	-1.94896406711151e-10\\
46.897296875	-2.52964952051112e-10\\
46.9174462890625	-2.15675278481359e-10\\
46.937595703125	-1.58656752012925e-10\\
46.9577451171875	-2.51352080884651e-10\\
46.97789453125	-1.99490564930579e-10\\
46.9980439453125	-2.1336261585843e-10\\
47.018193359375	-2.38027082135093e-10\\
47.0383427734375	-2.01051368383032e-10\\
47.0584921875	-1.74499677339937e-10\\
47.0786416015625	-1.62949309935923e-10\\
47.098791015625	-1.03797101520995e-10\\
47.1189404296875	-1.59515318166277e-10\\
47.13908984375	-8.96648389413047e-11\\
47.1592392578125	-1.60428146903123e-10\\
47.179388671875	-4.65786107651292e-11\\
47.1995380859375	-7.63367012705624e-11\\
47.2196875	-4.75667069141439e-11\\
47.2398369140625	-4.78884238810894e-11\\
47.259986328125	-7.59553725144309e-11\\
47.2801357421875	-9.23731674337151e-11\\
47.30028515625	-3.34102178568611e-11\\
47.3204345703125	-5.82311986833249e-11\\
47.340583984375	-8.34370601247581e-11\\
47.3607333984375	-7.49196510869111e-12\\
47.3808828125	-1.17832097336615e-10\\
47.4010322265625	-1.63407173769872e-11\\
47.421181640625	1.95013443261289e-11\\
47.4413310546875	-3.57984378151427e-11\\
47.46148046875	-9.54812395786059e-12\\
47.4816298828125	-4.10193920629034e-11\\
47.501779296875	-1.68559635228405e-11\\
47.5219287109375	-9.45040445514424e-12\\
47.542078125	3.22421735306172e-11\\
47.5622275390625	5.07274638562573e-12\\
47.582376953125	-1.54462114509735e-11\\
47.6025263671875	2.34194830494265e-11\\
47.62267578125	-2.13880211474028e-11\\
47.6428251953125	1.36529375554271e-11\\
47.662974609375	-2.46071295478278e-11\\
47.6831240234375	-3.79463908417331e-11\\
47.7032734375	-1.87727162703869e-11\\
47.7234228515625	2.55646313160448e-11\\
47.743572265625	2.26942902662446e-11\\
47.7637216796875	3.33622913226549e-11\\
47.78387109375	6.2926372345274e-11\\
47.8040205078125	6.32754686750439e-11\\
47.824169921875	3.32984232217448e-11\\
47.8443193359375	7.60910737193999e-11\\
47.86446875	-4.97168725843029e-11\\
47.8846181640625	-6.15268827183307e-11\\
47.904767578125	-8.13201515933233e-11\\
47.9249169921875	-3.91757107584316e-11\\
47.94506640625	-3.17957606485383e-11\\
47.9652158203125	2.14171348551855e-11\\
47.985365234375	4.77204486083e-11\\
48.0055146484375	6.15821512461406e-11\\
48.0256640625	7.45009081529063e-11\\
48.0458134765625	1.09938384486081e-10\\
48.065962890625	1.11338241793036e-10\\
48.0861123046875	3.39062354109499e-11\\
48.10626171875	3.2842766800999e-11\\
48.1264111328125	-2.47338283721243e-11\\
48.146560546875	2.47771577725021e-11\\
48.1667099609375	3.9274846577808e-11\\
48.186859375	1.65343732361119e-11\\
48.2070087890625	4.21837197247046e-11\\
48.227158203125	8.58465186272964e-11\\
48.2473076171875	1.6867927913735e-10\\
48.26745703125	1.9915275793967e-10\\
48.2876064453125	1.89694908360581e-10\\
48.307755859375	1.82874188366891e-10\\
48.3279052734375	1.77783578930127e-10\\
48.3480546875	9.45722982612662e-11\\
48.3682041015625	6.85086263609739e-11\\
48.388353515625	7.29030074405203e-11\\
48.4085029296875	3.36515688937264e-11\\
48.42865234375	8.74791349304765e-11\\
48.4488017578125	1.08295317861214e-10\\
48.468951171875	1.62252488089301e-10\\
48.4891005859375	1.31148103499403e-10\\
48.50925	1.7165355622638e-10\\
48.5293994140625	1.60980139876796e-10\\
48.549548828125	9.00104375150376e-11\\
48.5696982421875	1.3145488638499e-10\\
48.58984765625	6.37082210355501e-11\\
48.6099970703125	8.28470627278599e-11\\
48.630146484375	8.12644855544378e-11\\
48.6502958984375	1.71002933121275e-10\\
48.6704453125	1.72056589351305e-10\\
48.6905947265625	2.15152277909438e-10\\
48.710744140625	1.95928683149203e-10\\
48.7308935546875	1.53299125200182e-10\\
48.75104296875	9.91282170555363e-11\\
48.7711923828125	3.26059937627526e-11\\
48.791341796875	2.51377597923369e-11\\
48.8114912109375	3.15274260854771e-11\\
48.831640625	4.29364229760951e-11\\
48.8517900390625	5.5583542034034e-11\\
48.871939453125	8.93451819488178e-11\\
48.8920888671875	1.03150067315483e-10\\
48.91223828125	1.46594583306168e-10\\
48.9323876953125	1.03160255245675e-10\\
48.952537109375	9.96042522040127e-11\\
48.9726865234375	7.54863965214571e-11\\
48.9928359375	5.8147235952189e-11\\
49.0129853515625	-1.03621518736006e-12\\
49.033134765625	-2.43114863073546e-11\\
49.0532841796875	-5.06247673666221e-11\\
49.07343359375	4.18491619413363e-11\\
49.0935830078125	4.13542140381326e-11\\
49.113732421875	1.34733023474172e-10\\
49.1338818359375	1.37757129363058e-10\\
49.15403125	1.68888306841643e-10\\
49.1741806640625	1.86025137346421e-10\\
49.194330078125	1.62527801243254e-10\\
49.2144794921875	8.27869146781047e-11\\
49.23462890625	1.32114600253853e-11\\
49.2547783203125	2.79894218573364e-11\\
49.274927734375	6.14030116296402e-12\\
49.2950771484375	8.457970435816e-12\\
49.3152265625	1.08419882457388e-10\\
49.3353759765625	1.05147917500732e-10\\
49.355525390625	1.53489380305483e-10\\
49.3756748046875	1.73210519015814e-10\\
49.39582421875	1.30613516628868e-10\\
49.4159736328125	2.26806918676774e-10\\
49.436123046875	8.46843978059755e-11\\
49.4562724609375	4.60669120557042e-11\\
49.476421875	1.17569868075338e-10\\
49.4965712890625	8.58603993402807e-11\\
49.516720703125	1.1075729620003e-10\\
49.5368701171875	8.89973924802146e-11\\
49.55701953125	1.56296280285362e-10\\
49.5771689453125	1.80930742955768e-10\\
49.597318359375	1.07456015798053e-10\\
49.6174677734375	1.28129895260447e-10\\
49.6376171875	6.98789907216101e-11\\
49.6577666015625	4.52909588884015e-11\\
49.677916015625	9.73926347685741e-11\\
49.6980654296875	8.33865821840876e-11\\
49.71821484375	1.15877639887173e-10\\
49.7383642578125	2.85931763788223e-11\\
49.758513671875	8.9385617127001e-11\\
49.7786630859375	1.54911346939441e-11\\
49.7988125	-1.48995899371993e-11\\
49.8189619140625	1.19490466670128e-11\\
49.839111328125	2.10522820626413e-11\\
49.8592607421875	-3.13007471430396e-11\\
49.87941015625	-1.17670523703764e-10\\
49.8995595703125	1.86926248768618e-11\\
49.919708984375	-3.70301285284047e-11\\
49.9398583984375	-2.45949846079226e-11\\
49.9600078125	-6.01394965171175e-11\\
49.9801572265625	-2.0298618742516e-11\\
50.000306640625	-7.11022021685221e-11\\
50.0204560546875	-3.4070470418432e-11\\
50.04060546875	-2.06962926542657e-11\\
50.0607548828125	-4.79003911180118e-12\\
50.080904296875	-5.01350432127255e-11\\
50.1010537109375	-4.2838486778643e-11\\
50.121203125	-8.02132472835215e-11\\
50.1413525390625	-4.77659274918513e-11\\
50.161501953125	-6.96760075392383e-11\\
50.1816513671875	-9.28981896374473e-11\\
50.20180078125	-8.11664362903704e-11\\
50.2219501953125	-1.13183305112339e-10\\
50.242099609375	-3.63980424134625e-11\\
50.2622490234375	-3.06438628585e-11\\
50.2823984375	-1.72312322573223e-11\\
50.3025478515625	5.34520392232698e-12\\
50.322697265625	-4.09816978554036e-11\\
50.3428466796875	-9.83725637036585e-11\\
50.36299609375	-3.56377500201986e-11\\
50.3831455078125	-1.00782503518212e-10\\
50.403294921875	-1.209534464855e-10\\
50.4234443359375	-1.43686167869833e-10\\
50.44359375	-1.37674775678574e-10\\
50.4637431640625	-8.05124445434447e-11\\
50.483892578125	-4.60273517085354e-11\\
50.5040419921875	-6.26836523249122e-11\\
50.52419140625	-6.64569481807522e-12\\
50.5443408203125	-1.17654407558972e-11\\
50.564490234375	-2.05659372574996e-11\\
50.5846396484375	-4.81352404137243e-11\\
50.6047890625	-1.98276376141764e-10\\
50.6249384765625	-1.53538354035273e-10\\
50.645087890625	-2.24887879639564e-10\\
50.6652373046875	-2.15510861148416e-10\\
50.68538671875	-1.94023052926998e-10\\
50.7055361328125	-1.86516501063604e-10\\
50.725685546875	-1.71409952618667e-10\\
50.7458349609375	-1.59669631852321e-10\\
50.765984375	-7.21786670651077e-11\\
50.7861337890625	-9.68386527917995e-11\\
50.806283203125	-1.17860522831959e-10\\
50.8264326171875	-1.73407422007028e-10\\
50.84658203125	-1.78110351660904e-10\\
50.8667314453125	-2.14348559969612e-10\\
50.886880859375	-1.82144261485547e-10\\
50.9070302734375	-2.05874478460344e-10\\
50.9271796875	-1.97224439133791e-10\\
50.9473291015625	-2.05703841126632e-10\\
50.967478515625	-9.79913004538909e-11\\
50.9876279296875	-1.57847798096694e-10\\
51.00777734375	-1.16792359720576e-10\\
51.0279267578125	-1.68636848173774e-10\\
51.048076171875	-1.77288566980353e-10\\
51.0682255859375	-1.7985137857083e-10\\
51.088375	-2.31912979205184e-10\\
51.1085244140625	-1.46270748199351e-10\\
51.128673828125	-2.01118806370061e-10\\
51.1488232421875	-1.88289751695562e-10\\
51.16897265625	-1.68705179590638e-10\\
51.1891220703125	-1.06576414715245e-10\\
51.209271484375	-2.08754156296833e-10\\
51.2294208984375	-1.70048718824457e-10\\
51.2495703125	-1.81151334792578e-10\\
51.2697197265625	-2.01788961621852e-10\\
51.289869140625	-1.65446488029716e-10\\
51.3100185546875	-1.79537091562061e-10\\
51.33016796875	-1.77637604875476e-10\\
51.3503173828125	-1.58885232611753e-10\\
51.370466796875	-1.24732019052005e-10\\
51.3906162109375	-1.3668883262377e-10\\
51.410765625	-1.15526325561755e-10\\
51.4309150390625	-1.20337928361094e-10\\
51.451064453125	-1.5866961970241e-10\\
51.4712138671875	-1.91215279235188e-10\\
51.49136328125	-1.96479151418165e-10\\
51.5115126953125	-1.73635061345664e-10\\
51.531662109375	-1.72173230277557e-10\\
51.5518115234375	-1.61093661563657e-10\\
51.5719609375	-1.30866510420798e-10\\
51.5921103515625	-1.07800992691227e-10\\
51.612259765625	-1.15740964983748e-10\\
51.6324091796875	-1.00931170142041e-10\\
51.65255859375	-1.38200529236918e-10\\
51.6727080078125	-1.87224033247745e-10\\
51.692857421875	-1.86237165721728e-10\\
51.7130068359375	-2.53163830709638e-10\\
51.73315625	-2.62403806063535e-10\\
51.7533056640625	-2.48020159344146e-10\\
51.773455078125	-2.12165944202361e-10\\
51.7936044921875	-1.82960942884525e-10\\
51.81375390625	-1.47511151745792e-10\\
51.8339033203125	-1.37055132901354e-10\\
51.854052734375	-1.2512286592591e-10\\
51.8742021484375	-2.13282969745873e-10\\
51.8943515625	-1.65935909490303e-10\\
51.9145009765625	-2.52783756613158e-10\\
51.934650390625	-2.33173816261663e-10\\
51.9547998046875	-1.9272884058452e-10\\
51.97494921875	-1.82882024319188e-10\\
51.9950986328125	-1.99320257410424e-10\\
52.015248046875	-6.96943260648816e-11\\
52.0353974609375	-1.07117734622216e-10\\
52.055546875	-1.46273038641953e-10\\
52.0756962890625	-9.56492201353996e-11\\
52.095845703125	-8.20906662474326e-11\\
52.1159951171875	-1.28891257268031e-10\\
52.13614453125	-1.35260953933929e-10\\
52.1562939453125	-9.43887307761542e-11\\
52.176443359375	-6.39006120793813e-11\\
52.1965927734375	-4.76124591218444e-11\\
52.2167421875	-1.12933155885818e-11\\
52.2368916015625	-6.60144745933914e-11\\
52.257041015625	-1.29165439787861e-10\\
52.2771904296875	-1.00027766442833e-10\\
52.29733984375	-1.452375845647e-10\\
52.3174892578125	-1.37309103200683e-10\\
52.337638671875	-1.23884341046217e-10\\
52.3577880859375	-8.38540344828025e-11\\
52.3779375	-5.39547117374363e-11\\
52.3980869140625	2.83683532667851e-11\\
52.418236328125	-6.31548103308951e-12\\
52.4383857421875	8.39314112076659e-11\\
52.45853515625	4.33491818694137e-11\\
52.4786845703125	-3.14836448095099e-11\\
52.498833984375	5.44703415141017e-12\\
52.5189833984375	-5.44628694611153e-11\\
52.5391328125	-3.78171722568164e-11\\
52.5592822265625	-4.82868940848192e-11\\
52.579431640625	-7.31213012484196e-11\\
52.5995810546875	-4.50775798402044e-11\\
52.61973046875	-6.47682914878453e-12\\
52.6398798828125	-4.31250048891074e-11\\
52.660029296875	2.19025458762684e-11\\
52.6801787109375	-2.15216341557773e-11\\
52.700328125	2.05681210014738e-11\\
52.7204775390625	-7.26514441017868e-11\\
52.740626953125	-9.12743629764658e-11\\
52.7607763671875	-1.70299710532246e-10\\
52.78092578125	-1.83069552821292e-10\\
52.8010751953125	-2.23799450295295e-10\\
52.821224609375	-1.65288265807757e-10\\
52.8413740234375	-1.60434186588593e-10\\
52.8615234375	-8.14985772129379e-11\\
52.8816728515625	-5.00481885141541e-11\\
52.901822265625	1.55188987743677e-11\\
52.9219716796875	-4.13227875489846e-11\\
52.94212109375	-2.81781611099223e-11\\
52.9622705078125	-4.70658746098127e-11\\
52.982419921875	-1.37763365557695e-10\\
53.0025693359375	-1.76743888225594e-10\\
53.02271875	-1.73791131505771e-10\\
53.0428681640625	-1.95040290105108e-10\\
53.063017578125	-2.12397574493212e-10\\
53.0831669921875	-2.15544126576098e-10\\
53.10331640625	-7.78745703157329e-11\\
53.1234658203125	-7.20143616836411e-11\\
53.143615234375	-7.53960192840946e-11\\
53.1637646484375	8.44757385710117e-12\\
53.1839140625	-7.17075620917974e-11\\
53.2040634765625	-1.38757646806171e-11\\
53.224212890625	-8.27398330678658e-11\\
53.2443623046875	-7.00555570296481e-11\\
53.26451171875	-1.32773695200544e-10\\
53.2846611328125	-1.70666257823576e-10\\
53.304810546875	-1.40927366058127e-10\\
53.3249599609375	-1.20467698310684e-10\\
53.345109375	-1.00102962064649e-10\\
53.3652587890625	-8.91320896726376e-11\\
53.385408203125	-8.72508257054589e-11\\
53.4055576171875	1.11843748810608e-11\\
53.42570703125	1.22154182825827e-12\\
53.4458564453125	-5.40238769210968e-12\\
53.466005859375	-3.17352191346575e-11\\
53.4861552734375	-1.79047036961672e-11\\
53.5063046875	-9.4105906485479e-11\\
53.5264541015625	-1.41179074600441e-10\\
53.546603515625	-1.58110613157817e-10\\
53.5667529296875	-1.44456591144783e-10\\
53.58690234375	-8.71412519693034e-11\\
53.6070517578125	-5.81752304455928e-11\\
53.627201171875	-3.32493413396788e-11\\
53.6473505859375	-5.53871960547668e-11\\
53.6675	-4.06965260004178e-11\\
53.6876494140625	-9.43630425511341e-11\\
53.707798828125	-4.88972333666884e-11\\
53.7279482421875	-1.29611834876833e-10\\
53.74809765625	-1.16711433018509e-10\\
53.7682470703125	-3.41621736827235e-11\\
53.788396484375	-3.07247517445734e-11\\
53.8085458984375	-4.97353546139106e-12\\
53.8286953125	-7.44473104803869e-12\\
53.8488447265625	3.37595141284236e-11\\
53.868994140625	3.811850161172e-12\\
53.8891435546875	-1.76111012566555e-11\\
53.90929296875	-5.89861027429436e-11\\
53.9294423828125	-6.94868385897195e-11\\
53.949591796875	-1.66951891185218e-11\\
53.9697412109375	-6.68642217077232e-11\\
53.989890625	2.53774601107085e-11\\
54.0100400390625	4.63716127449092e-11\\
54.030189453125	7.09872306767125e-11\\
54.0503388671875	5.09863936325505e-11\\
54.07048828125	7.29014048170082e-11\\
54.0906376953125	-1.58309129830203e-11\\
54.110787109375	-5.16073040683167e-13\\
54.1309365234375	1.49978451482109e-11\\
54.1510859375	-5.22491281929422e-11\\
54.1712353515625	-5.49390538950068e-11\\
54.191384765625	-5.68739633244454e-11\\
54.2115341796875	-2.77342837759312e-11\\
54.23168359375	-2.12039241575495e-11\\
54.2518330078125	-2.6628477715076e-12\\
54.271982421875	-3.06168401528676e-11\\
54.2921318359375	-1.56958258021227e-11\\
54.31228125	-7.96988757668425e-12\\
54.3324306640625	-1.4344854278763e-11\\
54.352580078125	-3.52800023073854e-11\\
54.3727294921875	-6.26781307930107e-11\\
54.39287890625	-7.83883728661531e-11\\
54.4130283203125	-7.02379011805134e-11\\
54.433177734375	-8.99348959027506e-11\\
54.4533271484375	-9.27493080022483e-11\\
54.4734765625	-8.47795517400158e-11\\
54.4936259765625	-1.00799268786761e-10\\
54.513775390625	-8.11855745262166e-11\\
54.5339248046875	-1.39861172593003e-10\\
54.55407421875	-9.28371063416029e-11\\
54.5742236328125	-9.26241787338701e-11\\
54.594373046875	-1.10279506090335e-10\\
54.6145224609375	-1.4195191509924e-10\\
54.634671875	-1.04728312379698e-10\\
54.6548212890625	-1.15709217691609e-10\\
54.674970703125	-1.06965740339314e-10\\
54.6951201171875	-1.30893075192649e-10\\
54.71526953125	-1.10817064023257e-10\\
54.7354189453125	-1.05130584587961e-10\\
54.755568359375	-6.05862455212137e-11\\
54.7757177734375	-1.35038854264852e-10\\
54.7958671875	-4.53119220537702e-11\\
54.8160166015625	-1.05744931418675e-10\\
54.836166015625	-1.10590608760721e-10\\
54.8563154296875	-9.02171497987019e-11\\
54.87646484375	-8.52346362220016e-11\\
54.8966142578125	-8.6776205718469e-11\\
54.916763671875	-1.31879516521125e-10\\
54.9369130859375	-9.97815389330837e-11\\
54.9570625	-1.1520395749655e-10\\
54.9772119140625	-8.64514059991474e-11\\
54.997361328125	-8.67308190783729e-11\\
55.0175107421875	-1.07748220425506e-10\\
55.03766015625	-6.2034431353196e-11\\
55.0578095703125	-8.25725630237247e-11\\
55.077958984375	-1.15171247433926e-10\\
55.0981083984375	-1.66905233087362e-10\\
55.1182578125	-1.32530698951587e-10\\
55.1384072265625	-1.49422809529877e-10\\
55.158556640625	-7.00260841705445e-11\\
55.1787060546875	-7.19964313081837e-11\\
};
\addlegendentry{$\text{train 8 -\textgreater{} Trondheim}$};

\end{axis}
\end{tikzpicture}%	
	\label{fig:train8}
\end{subfigure}
\caption{Influence lines found through the matrix method}\label{fig:infl_trains}
\end{figure}

\begin{figure}[H]
\centering
% This file was created by matlab2tikz.
%
%The latest updates can be retrieved from
%  http://www.mathworks.com/matlabcentral/fileexchange/22022-matlab2tikz-matlab2tikz
%where you can also make suggestions and rate matlab2tikz.
%
\definecolor{mycolor1}{rgb}{0.00000,0.44700,0.74100}%
\definecolor{mycolor2}{rgb}{0.85000,0.32500,0.09800}%
\definecolor{mycolor3}{rgb}{0.92900,0.69400,0.12500}%
\definecolor{mycolor4}{rgb}{0.49400,0.18400,0.55600}%
%
\begin{tikzpicture}

\begin{axis}[%
width=0.8\textwidth,
height=0.2\pageheight,
at={(0\figurewidth,0\figureheight)},
scale only axis,
xmin=-60,
xmax=60,
ymin=-4e-09,
ymax=1.2e-08,
axis background/.style={fill=white},
title style={font=\bfseries},
title={Influencelines for 4 trains, middle sensor},
legend style={legend cell align=left,align=left,draw=white!15!black}
]
\addplot [color=mycolor1,solid,forget plot]
  table[row sep=crcr]{%
-47.80724609375	-2.43705295696912e-11\\
-47.786748046875	-6.76412695658217e-11\\
-47.76625	-7.04825924520673e-11\\
-47.745751953125	-3.83282736349843e-11\\
-47.72525390625	-2.7624144334012e-11\\
-47.704755859375	-1.00608535823748e-10\\
-47.6842578125	-7.85422568649401e-11\\
-47.663759765625	-1.65977371633779e-11\\
-47.64326171875	-7.25506485136496e-11\\
-47.622763671875	-4.50116083969734e-11\\
-47.602265625	-7.87973899513735e-11\\
-47.581767578125	3.47683545211713e-11\\
-47.56126953125	-8.58630388578684e-11\\
-47.540771484375	-1.32798980825908e-11\\
-47.5202734375	-1.1673518179255e-10\\
-47.499775390625	-1.39964370152935e-10\\
-47.47927734375	-1.36063957271996e-10\\
-47.458779296875	-8.77792361906869e-11\\
-47.43828125	-1.85467833701982e-10\\
-47.417783203125	-1.06862863344413e-10\\
-47.39728515625	-1.23553555612857e-10\\
-47.376787109375	-1.26641307290727e-10\\
-47.3562890625	-1.51587171796695e-10\\
-47.335791015625	-1.99949146627952e-10\\
-47.31529296875	-2.27479944788248e-10\\
-47.294794921875	-2.10526185116155e-10\\
-47.274296875	-2.23811979637214e-10\\
-47.253798828125	-1.46569325791716e-10\\
-47.23330078125	-1.88224221468316e-10\\
-47.212802734375	-8.07640215896891e-11\\
-47.1923046875	-9.29015807357987e-11\\
-47.171806640625	-1.92774932787603e-11\\
-47.15130859375	-1.83646915994842e-11\\
-47.130810546875	-5.19044743822199e-11\\
-47.1103125	-7.46409089755521e-11\\
-47.089814453125	-1.23853739657538e-11\\
-47.06931640625	-4.77479755196377e-11\\
-47.048818359375	-5.44521074571839e-11\\
-47.0283203125	-2.41290617575454e-11\\
-47.007822265625	-9.02498510558779e-11\\
-46.98732421875	7.14215183875686e-11\\
-46.966826171875	6.75498386014551e-13\\
-46.946328125	9.9651447562819e-11\\
-46.925830078125	1.09079908174467e-10\\
-46.90533203125	2.21934627700921e-10\\
-46.884833984375	1.4700887264218e-10\\
-46.8643359375	1.5871594191208e-10\\
-46.843837890625	1.71152555322276e-10\\
-46.82333984375	1.52359829651946e-10\\
-46.802841796875	1.32578405295044e-10\\
-46.78234375	1.50263623436425e-10\\
-46.761845703125	2.2299218532693e-10\\
-46.74134765625	2.70153391478226e-10\\
-46.720849609375	2.14699478997746e-10\\
-46.7003515625	2.81758095389842e-10\\
-46.679853515625	2.74383331773193e-10\\
-46.65935546875	2.7258605376745e-10\\
-46.638857421875	2.79496242337175e-10\\
-46.618359375	2.08346375006837e-10\\
-46.597861328125	2.75839669378152e-10\\
-46.57736328125	2.73914883309658e-10\\
-46.556865234375	3.05739015983724e-10\\
-46.5363671875	3.55846212504023e-10\\
-46.515869140625	4.39509362008012e-10\\
-46.49537109375	4.25758082752511e-10\\
-46.474873046875	4.40967185232544e-10\\
-46.454375	3.72164124570923e-10\\
-46.433876953125	3.34052926500276e-10\\
-46.41337890625	2.83803529930271e-10\\
-46.392880859375	2.52268957719451e-10\\
-46.3723828125	1.71816623837271e-10\\
-46.351884765625	1.28855479767502e-10\\
-46.33138671875	1.8075058518828e-10\\
-46.310888671875	1.51351669063083e-10\\
-46.290390625	2.45678150334621e-10\\
-46.269892578125	2.31982190848082e-10\\
-46.24939453125	2.44894166855285e-10\\
-46.228896484375	1.76051400084553e-10\\
-46.2083984375	1.52745484473485e-10\\
-46.187900390625	7.19769331900981e-11\\
-46.16740234375	5.40941980367577e-11\\
-46.146904296875	3.16486725849255e-11\\
-46.12640625	-5.61075002970875e-11\\
-46.105908203125	-1.0915777617896e-10\\
-46.08541015625	-8.34315606761191e-11\\
-46.064912109375	-1.75003704575635e-11\\
-46.0444140625	-5.39835180888069e-11\\
-46.023916015625	-4.34590954858808e-11\\
-46.00341796875	-5.83523501914465e-11\\
-45.982919921875	-1.29297113348402e-10\\
-45.962421875	-1.66763893681193e-10\\
-45.941923828125	-1.44182958271443e-10\\
-45.92142578125	-1.2043949691265e-10\\
-45.900927734375	-1.28925402869162e-10\\
-45.8804296875	-1.01554174236212e-10\\
-45.859931640625	-9.76174684815775e-11\\
-45.83943359375	-1.87665459629072e-10\\
-45.818935546875	-6.46840604248631e-11\\
-45.7984375	-6.46215433826728e-11\\
-45.777939453125	-1.78695048235645e-11\\
-45.75744140625	-8.00601697819881e-11\\
-45.736943359375	-1.41617331056307e-10\\
-45.7164453125	-1.6515281559771e-10\\
-45.695947265625	-2.07461317237449e-10\\
-45.67544921875	-2.01781728407864e-10\\
-45.654951171875	-1.09618071835178e-10\\
-45.634453125	-1.82726911212765e-10\\
-45.613955078125	-1.1800124579516e-10\\
-45.59345703125	-6.79153772060677e-11\\
-45.572958984375	-7.11757437741872e-12\\
-45.5524609375	-7.33873348863849e-11\\
-45.531962890625	-5.90016674212971e-11\\
-45.51146484375	-9.33080522069948e-11\\
-45.490966796875	-7.34553024616156e-11\\
-45.47046875	-1.51053705972602e-10\\
-45.449970703125	-9.05887491202441e-11\\
-45.42947265625	-1.50284472306023e-10\\
-45.408974609375	-1.2377262670428e-10\\
-45.3884765625	-9.51403030751751e-11\\
-45.367978515625	-1.86224814530952e-10\\
-45.34748046875	-1.81072120112644e-10\\
-45.326982421875	-1.05417155184408e-10\\
-45.306484375	-2.22904102481793e-10\\
-45.285986328125	-1.59476250320702e-10\\
-45.26548828125	-1.56630002223744e-10\\
-45.244990234375	-1.71157241096894e-10\\
-45.2244921875	-1.66365097593585e-10\\
-45.203994140625	-1.64062780596579e-10\\
-45.18349609375	-1.563491824785e-10\\
-45.162998046875	-1.39374729909613e-10\\
-45.1425	-1.89742318953465e-10\\
-45.122001953125	-1.68236631450442e-10\\
-45.10150390625	-9.65704705358031e-11\\
-45.081005859375	-1.91174123812455e-10\\
-45.0605078125	-2.46878396182631e-11\\
-45.040009765625	-1.22270187485887e-10\\
-45.01951171875	-3.87977299426354e-12\\
-44.999013671875	-1.73987193539183e-10\\
-44.978515625	-1.69700334430719e-10\\
-44.958017578125	-1.85088629460084e-10\\
-44.93751953125	-1.59468140802528e-10\\
-44.917021484375	-2.20422310285062e-10\\
-44.8965234375	-1.12995199104762e-10\\
-44.876025390625	-2.107899984311e-10\\
-44.85552734375	-8.76504805329389e-11\\
-44.835029296875	-3.82187192024658e-11\\
-44.81453125	-3.05482864944862e-11\\
-44.794033203125	1.68000686776317e-11\\
-44.77353515625	-4.94165401719359e-11\\
-44.753037109375	-1.29589470351391e-11\\
-44.7325390625	-4.24035917476327e-11\\
-44.712041015625	-1.28176871179358e-10\\
-44.69154296875	-8.43421923102772e-11\\
-44.671044921875	-1.14937780696804e-10\\
-44.650546875	-3.04710417685558e-11\\
-44.630048828125	-6.98856382889648e-11\\
-44.60955078125	-5.70764510523244e-11\\
-44.589052734375	-1.12693122342651e-10\\
-44.5685546875	-9.93616506929683e-11\\
-44.548056640625	-9.44395414121439e-11\\
-44.52755859375	-9.7225894117133e-11\\
-44.507060546875	-1.46772625141573e-10\\
-44.4865625	-2.33699573668156e-10\\
-44.466064453125	-9.34003972203333e-11\\
-44.44556640625	-2.46039802393467e-10\\
-44.425068359375	4.27126755965892e-12\\
-44.4045703125	-1.63011305049394e-10\\
-44.384072265625	-7.34188324976539e-11\\
-44.36357421875	-1.59305427924388e-10\\
-44.343076171875	-1.77819533003364e-10\\
-44.322578125	-1.79714981509761e-10\\
-44.302080078125	-2.72134023003952e-10\\
-44.28158203125	-2.72938537888904e-10\\
-44.261083984375	-3.03388313413932e-10\\
-44.2405859375	-2.33206275092864e-10\\
-44.220087890625	-1.75288090943113e-10\\
-44.19958984375	-1.70488479663617e-10\\
-44.179091796875	-9.09146290915524e-11\\
-44.15859375	-1.03135323511149e-10\\
-44.138095703125	-1.22172661001084e-10\\
-44.11759765625	-1.56081986923805e-10\\
-44.097099609375	-1.9802340748749e-10\\
-44.0766015625	-2.39919008866864e-10\\
-44.056103515625	-3.18560589002798e-10\\
-44.03560546875	-3.02990842537389e-10\\
-44.015107421875	-2.78918187952124e-10\\
-43.994609375	-1.70697398629842e-10\\
-43.974111328125	-1.66534380971977e-10\\
-43.95361328125	-6.09628435884452e-11\\
-43.933115234375	-6.96378691265152e-11\\
-43.9126171875	-7.52410762374544e-12\\
-43.892119140625	-8.0219256201838e-11\\
-43.87162109375	-1.44315185760575e-10\\
-43.851123046875	-1.3898157068663e-10\\
-43.830625	-2.34835487599975e-10\\
-43.810126953125	-1.68042379145465e-10\\
-43.78962890625	-1.35855969321296e-10\\
-43.769130859375	5.70750900806057e-13\\
-43.7486328125	-1.85034701455156e-11\\
-43.728134765625	8.64467195996923e-11\\
-43.70763671875	1.0078720211488e-10\\
-43.687138671875	1.16908422623544e-10\\
-43.666640625	1.71506272163838e-10\\
-43.646142578125	1.32735743925711e-10\\
-43.62564453125	8.63688373143254e-11\\
-43.605146484375	2.57114693314608e-11\\
-43.5846484375	-2.7381603287378e-12\\
-43.564150390625	7.54909378089495e-11\\
-43.54365234375	1.32891752286754e-10\\
-43.523154296875	1.82533303417523e-10\\
-43.50265625	2.06770772389354e-10\\
-43.482158203125	2.81100634251449e-10\\
-43.46166015625	3.08297977862899e-10\\
-43.441162109375	2.78436386588071e-10\\
-43.4206640625	2.94568365950705e-10\\
-43.400166015625	1.94874511308612e-10\\
-43.37966796875	1.8745003429931e-10\\
-43.359169921875	2.0061308530159e-10\\
-43.338671875	2.06198649449925e-10\\
-43.318173828125	1.98306043892118e-10\\
-43.29767578125	2.16869062280162e-10\\
-43.277177734375	2.3088982364316e-10\\
-43.2566796875	2.52906333417691e-10\\
-43.236181640625	2.68816703719254e-10\\
-43.21568359375	2.65799174064098e-10\\
-43.195185546875	2.40287300485368e-10\\
-43.1746875	1.8417325835846e-10\\
-43.154189453125	1.66636956134297e-10\\
-43.13369140625	1.6727734462647e-10\\
-43.113193359375	6.15908862884959e-11\\
-43.0926953125	9.73327337426314e-11\\
-43.072197265625	8.15084703932963e-11\\
-43.05169921875	1.62606131777735e-10\\
-43.031201171875	1.08379035999676e-10\\
-43.010703125	1.14180815683523e-10\\
-42.990205078125	3.91864815561855e-11\\
-42.96970703125	1.46081002220508e-10\\
-42.949208984375	1.7394406816843e-11\\
-42.9287109375	4.55719257741036e-11\\
-42.908212890625	-3.76196153493767e-13\\
-42.88771484375	4.35629402214319e-11\\
-42.867216796875	5.67052060158337e-11\\
-42.84671875	4.13493123813237e-11\\
-42.826220703125	7.54871935615087e-12\\
-42.80572265625	9.33175441274144e-11\\
-42.785224609375	1.45681971101056e-11\\
-42.7647265625	1.20339088425569e-10\\
-42.744228515625	9.04768114592159e-11\\
-42.72373046875	8.89286618839751e-11\\
-42.703232421875	1.03925477447608e-10\\
-42.682734375	5.08391860636421e-11\\
-42.662236328125	6.54387144727438e-11\\
-42.64173828125	7.53689990009393e-11\\
-42.621240234375	2.8206375104362e-11\\
-42.6007421875	5.58297575512182e-11\\
-42.580244140625	2.15739194839052e-12\\
-42.55974609375	1.26109199482771e-11\\
-42.539248046875	1.35420195272907e-10\\
-42.51875	3.39957741156595e-12\\
-42.498251953125	1.25784078763694e-10\\
-42.47775390625	9.80499872428331e-11\\
-42.457255859375	2.69573311528824e-11\\
-42.4367578125	7.23136684203001e-11\\
-42.416259765625	6.01680008155737e-11\\
-42.39576171875	1.39809933853925e-12\\
-42.375263671875	5.71143024505259e-11\\
-42.354765625	-3.43409202083281e-11\\
-42.334267578125	2.09896716523753e-11\\
-42.31376953125	-2.87462247597426e-11\\
-42.293271484375	-8.91518796749338e-13\\
-42.2727734375	-3.84275750202111e-11\\
-42.252275390625	-5.20943997980071e-11\\
-42.23177734375	-4.92540564323501e-11\\
-42.211279296875	-5.5706163563988e-11\\
-42.19078125	-5.35155124842659e-11\\
-42.170283203125	-2.83890736503817e-11\\
-42.14978515625	-1.03148114022762e-10\\
-42.129287109375	-1.0860558919277e-10\\
-42.1087890625	-1.08129026809343e-10\\
-42.088291015625	1.77221875292916e-11\\
-42.06779296875	-1.26119770102478e-12\\
-42.047294921875	-3.76705009100405e-11\\
-42.026796875	1.06926391923895e-11\\
-42.006298828125	-7.42925435364923e-11\\
-41.98580078125	-5.13703905510795e-11\\
-41.965302734375	-3.25046739199112e-11\\
-41.9448046875	-4.20190866368884e-12\\
-41.924306640625	-1.58658101088329e-10\\
-41.90380859375	4.75681217750274e-11\\
-41.883310546875	-6.67471845017705e-11\\
-41.8628125	8.46665029290403e-11\\
-41.842314453125	-2.35447447951291e-11\\
-41.82181640625	6.7649945274316e-11\\
-41.801318359375	8.30193779385366e-11\\
-41.7808203125	5.6640075342587e-11\\
-41.760322265625	3.33445886559711e-11\\
-41.73982421875	7.58547285317205e-11\\
-41.719326171875	-5.2081191782355e-13\\
-41.698828125	1.79813039143014e-11\\
-41.678330078125	-7.97218355350971e-11\\
-41.65783203125	-1.71498492880976e-12\\
-41.637333984375	4.30316222311015e-11\\
-41.6168359375	1.79853284036644e-11\\
-41.596337890625	1.02288976026178e-10\\
-41.57583984375	7.90099835714467e-11\\
-41.555341796875	1.09755366617671e-10\\
-41.53484375	3.57430270705511e-11\\
-41.514345703125	5.09797893862575e-11\\
-41.49384765625	6.41001100426856e-11\\
-41.473349609375	5.2091677930509e-11\\
-41.4528515625	-4.33839479396308e-11\\
-41.432353515625	6.18830578460675e-11\\
-41.41185546875	2.99304031884978e-11\\
-41.391357421875	8.75789863327373e-11\\
-41.370859375	1.03226518296032e-11\\
-41.350361328125	9.61375134575839e-11\\
-41.32986328125	2.75150975292479e-12\\
-41.309365234375	-1.24915572559457e-10\\
-41.2888671875	-4.17575866966775e-11\\
-41.268369140625	-2.02576675465645e-10\\
-41.24787109375	-1.50380253509052e-10\\
-41.227373046875	-1.89730257515556e-10\\
-41.206875	-7.97710461634346e-11\\
-41.186376953125	-7.08171131077433e-11\\
-41.16587890625	-4.25903933771955e-11\\
-41.145380859375	-7.80892761450968e-12\\
-41.1248828125	-4.46946000914061e-11\\
-41.104384765625	-1.25617467850915e-10\\
-41.08388671875	-1.66789954766834e-10\\
-41.063388671875	-2.47912598965621e-10\\
-41.042890625	-2.63493745114281e-10\\
-41.022392578125	-2.82896400312868e-10\\
-41.00189453125	-3.52791446911246e-10\\
-40.981396484375	-2.94500592688265e-10\\
-40.9608984375	-2.64647396812159e-10\\
-40.940400390625	-3.21536075939014e-10\\
-40.91990234375	-3.36802635411836e-10\\
-40.899404296875	-2.95711988841558e-10\\
-40.87890625	-3.41320973356843e-10\\
-40.858408203125	-3.55923144003864e-10\\
-40.83791015625	-3.25161034026954e-10\\
-40.817412109375	-3.70911263303419e-10\\
-40.7969140625	-4.0770496526226e-10\\
-40.776416015625	-4.32316836471414e-10\\
-40.75591796875	-4.10399255614968e-10\\
-40.735419921875	-3.62887128240708e-10\\
-40.714921875	-4.31972132421927e-10\\
-40.694423828125	-4.03745660512117e-10\\
-40.67392578125	-4.39697567013521e-10\\
-40.653427734375	-5.01210382019694e-10\\
-40.6329296875	-4.24424389015381e-10\\
-40.612431640625	-3.92063202450775e-10\\
-40.59193359375	-4.18081749102953e-10\\
-40.571435546875	-3.82183157403138e-10\\
-40.5509375	-4.03079842989865e-10\\
-40.530439453125	-3.78149580736954e-10\\
-40.50994140625	-4.11677449689777e-10\\
-40.489443359375	-3.03049809743088e-10\\
-40.4689453125	-3.75777156540767e-10\\
-40.448447265625	-2.75633593037601e-10\\
-40.42794921875	-2.95071297716759e-10\\
-40.407451171875	-2.12140161628078e-10\\
-40.386953125	-2.53772938188991e-10\\
-40.366455078125	-3.12261522334413e-10\\
-40.34595703125	-3.24855092604323e-10\\
-40.325458984375	-2.85532505901919e-10\\
-40.3049609375	-3.62693563660593e-10\\
-40.284462890625	-2.7109594833159e-10\\
-40.26396484375	-3.13059533106582e-10\\
-40.243466796875	-1.74240081913453e-10\\
-40.22296875	-3.26289586191516e-10\\
-40.202470703125	-2.32820826917063e-10\\
-40.18197265625	-2.36584242895274e-10\\
-40.161474609375	-2.53560670292011e-10\\
-40.1409765625	-2.64483111797478e-10\\
-40.120478515625	-1.95470718142144e-10\\
-40.09998046875	-3.32677866245067e-10\\
-40.079482421875	-2.53365671693715e-10\\
-40.058984375	-3.11851886815887e-10\\
-40.038486328125	-2.79258280182164e-10\\
-40.01798828125	-2.35231403978289e-10\\
-39.997490234375	-3.78926361986378e-10\\
-39.9769921875	-1.55229338754134e-10\\
-39.956494140625	-2.99302467522856e-10\\
-39.93599609375	-2.65193456385683e-10\\
-39.915498046875	-3.07173667070403e-10\\
-39.895	-2.29104825832662e-10\\
-39.874501953125	-4.31940482921727e-10\\
-39.85400390625	-2.22436885320928e-10\\
-39.833505859375	-2.79517333366643e-10\\
-39.8130078125	-2.24130750732491e-10\\
-39.792509765625	-3.33032635995573e-10\\
-39.77201171875	-1.62850608037389e-10\\
-39.751513671875	-2.23820331046041e-10\\
-39.731015625	-2.00734533576301e-10\\
-39.710517578125	-1.44227870592141e-10\\
-39.69001953125	-2.06071801756148e-10\\
-39.669521484375	-1.95836249556781e-10\\
-39.6490234375	-1.18203839682934e-10\\
-39.628525390625	-2.40733771544017e-10\\
-39.60802734375	-1.17208049006199e-10\\
-39.587529296875	-1.35573729416346e-10\\
-39.56703125	-1.54825605041532e-10\\
-39.546533203125	-1.45010173380374e-10\\
-39.52603515625	-1.59993842461513e-10\\
-39.505537109375	-1.61673471374552e-10\\
-39.4850390625	-2.11800170570594e-10\\
-39.464541015625	-1.76435921487928e-10\\
-39.44404296875	-2.33714851042441e-10\\
-39.423544921875	-1.67677295735727e-10\\
-39.403046875	-2.63606005887881e-10\\
-39.382548828125	-1.09659661857086e-10\\
-39.36205078125	-2.45529971139778e-10\\
-39.341552734375	-1.058519445589e-10\\
-39.3210546875	-2.6993415704536e-10\\
-39.300556640625	-1.40795064053159e-10\\
-39.28005859375	-2.5194984561036e-10\\
-39.259560546875	-1.87427295278064e-10\\
-39.2390625	-2.92081362933305e-10\\
-39.218564453125	-2.2382510208881e-10\\
-39.19806640625	-1.98119320707778e-10\\
-39.177568359375	-2.16523555187087e-10\\
-39.1570703125	-2.72650814014248e-10\\
-39.136572265625	-1.68844305463519e-10\\
-39.11607421875	-3.23355866927323e-10\\
-39.095576171875	-2.84826596388668e-10\\
-39.075078125	-2.17152369276605e-10\\
-39.054580078125	-3.2591998859372e-10\\
-39.03408203125	-4.15541508334372e-10\\
-39.013583984375	-3.76668961893966e-10\\
-38.9930859375	-3.30463414435603e-10\\
-38.972587890625	-3.53906602252981e-10\\
-38.95208984375	-4.26992944754937e-10\\
-38.931591796875	-3.29165363839534e-10\\
-38.91109375	-2.68291440969643e-10\\
-38.890595703125	-2.94868089891037e-10\\
-38.87009765625	-3.03386586425663e-10\\
-38.849599609375	-2.89148678046843e-10\\
-38.8291015625	-2.54350810334442e-10\\
-38.808603515625	-3.67395972880495e-10\\
-38.78810546875	-3.91992065445774e-10\\
-38.767607421875	-2.4100141470735e-10\\
-38.747109375	-2.7749074488315e-10\\
-38.726611328125	-1.88497229204873e-10\\
-38.70611328125	-1.45173420263556e-10\\
-38.685615234375	-1.33376099860491e-10\\
-38.6651171875	-1.1408134554201e-10\\
-38.644619140625	-1.24482749625956e-10\\
-38.62412109375	-9.46822070124405e-11\\
-38.603623046875	-1.42529562991262e-10\\
-38.583125	-1.4505874667946e-10\\
-38.562626953125	-4.71287670044929e-11\\
-38.54212890625	-5.20207274483341e-11\\
-38.521630859375	4.09046061365757e-11\\
-38.5011328125	1.03973183844875e-10\\
-38.480634765625	4.83483378492824e-12\\
-38.46013671875	1.03130033143908e-11\\
-38.439638671875	3.43873234409207e-11\\
-38.419140625	3.82894236308931e-11\\
-38.398642578125	3.61975536991747e-11\\
-38.37814453125	4.62878889040539e-11\\
-38.357646484375	-6.41379367529644e-12\\
-38.3371484375	9.39218509269218e-11\\
-38.316650390625	4.5249524002794e-11\\
-38.29615234375	1.49009344884079e-10\\
-38.275654296875	1.30867716383921e-10\\
-38.25515625	1.12657448154008e-10\\
-38.234658203125	7.03260423884635e-11\\
-38.21416015625	1.74945804583649e-10\\
-38.193662109375	9.96276844408104e-11\\
-38.1731640625	1.73000321616676e-10\\
-38.152666015625	1.62412321808815e-10\\
-38.13216796875	2.000380261462e-10\\
-38.111669921875	2.73716424475172e-10\\
-38.091171875	1.59200418596048e-10\\
-38.070673828125	2.00447051638319e-10\\
-38.05017578125	2.46203050656437e-10\\
-38.029677734375	1.95704788119266e-10\\
-38.0091796875	2.76695474769099e-10\\
-37.988681640625	2.18047552143217e-10\\
-37.96818359375	2.03887597507391e-10\\
-37.947685546875	1.91039321568445e-10\\
-37.9271875	2.01059618406939e-10\\
-37.906689453125	2.46010398766634e-10\\
-37.88619140625	1.40099572281507e-10\\
-37.865693359375	7.43503627724476e-11\\
-37.8451953125	8.19113790421817e-11\\
-37.824697265625	2.94607871421857e-11\\
-37.80419921875	1.51391714390144e-10\\
-37.783701171875	2.28626569909449e-11\\
-37.763203125	2.20947119880333e-10\\
-37.742705078125	5.89082437733519e-11\\
-37.72220703125	1.70222704926209e-10\\
-37.701708984375	6.43257177504404e-11\\
-37.6812109375	9.38790831845613e-11\\
-37.660712890625	-1.12665107289164e-11\\
-37.64021484375	4.79102567554217e-11\\
-37.619716796875	3.69467692633099e-11\\
-37.59921875	3.83253847573362e-12\\
-37.578720703125	6.72817370958088e-11\\
-37.55822265625	6.91338882450983e-11\\
-37.537724609375	1.13804584050741e-10\\
-37.5172265625	1.00262224977274e-10\\
-37.496728515625	1.06231211609229e-10\\
-37.47623046875	9.05038699209426e-11\\
-37.455732421875	8.6286042273127e-11\\
-37.435234375	-7.32388560086488e-11\\
-37.414736328125	7.18150019763915e-11\\
-37.39423828125	3.82488202463592e-11\\
-37.373740234375	1.82615888533702e-10\\
-37.3532421875	7.97574641086942e-11\\
-37.332744140625	2.44497498147001e-10\\
-37.31224609375	4.83904764405608e-11\\
-37.291748046875	1.85358296499497e-10\\
-37.27125	-5.84344410718559e-11\\
-37.250751953125	7.79092009841314e-11\\
-37.23025390625	-1.212969547652e-10\\
-37.209755859375	-7.06471257683655e-11\\
-37.1892578125	-1.31013336558734e-10\\
-37.168759765625	-1.84557624472317e-10\\
-37.14826171875	-4.93202579718116e-11\\
-37.127763671875	-2.72558161937925e-11\\
-37.107265625	-3.5776874428964e-11\\
-37.086767578125	-5.82211246964659e-11\\
-37.06626953125	-2.63002285144156e-11\\
-37.045771484375	-8.85701180598952e-11\\
-37.0252734375	-1.2387264314426e-10\\
-37.004775390625	-9.43784941405918e-11\\
-36.98427734375	-1.10110662572702e-10\\
-36.963779296875	-1.19727470092679e-10\\
-36.94328125	-8.38830441259712e-11\\
-36.922783203125	2.12514487474505e-12\\
-36.90228515625	4.09425612094497e-11\\
-36.881787109375	2.36991738061953e-11\\
-36.8612890625	5.84432531413368e-11\\
-36.840791015625	-6.15344297603022e-11\\
-36.82029296875	1.37824706232304e-10\\
-36.799794921875	-1.019797725903e-10\\
-36.779296875	3.2159058768942e-11\\
-36.758798828125	-1.23334065489452e-10\\
-36.73830078125	3.71828071017845e-11\\
-36.717802734375	1.4488686105761e-11\\
-36.6973046875	8.72153538969797e-12\\
-36.676806640625	1.06200929836358e-10\\
-36.65630859375	1.49726397764548e-10\\
-36.635810546875	6.49241859726044e-11\\
-36.6153125	2.24434346369326e-10\\
-36.594814453125	1.00849352120549e-10\\
-36.57431640625	1.34220096381132e-10\\
-36.553818359375	6.29185357731046e-11\\
-36.5333203125	1.42468298133508e-10\\
-36.512822265625	5.52756820168553e-11\\
-36.49232421875	1.48478601153346e-10\\
-36.471826171875	1.29929756013825e-10\\
-36.451328125	2.68652546861895e-10\\
-36.430830078125	1.63235664867152e-10\\
-36.41033203125	1.87650593474265e-10\\
-36.389833984375	1.82276762943576e-11\\
-36.3693359375	1.1965557255204e-10\\
-36.348837890625	2.1424736303768e-11\\
-36.32833984375	7.59518321805785e-11\\
-36.307841796875	3.421353543801e-11\\
-36.28734375	-3.89194636686715e-11\\
-36.266845703125	3.46588507445628e-11\\
-36.24634765625	6.3063979072795e-11\\
-36.225849609375	8.57933086034676e-12\\
-36.2053515625	2.45972379558824e-11\\
-36.184853515625	-5.45482939957858e-11\\
-36.16435546875	-3.05470128903915e-11\\
-36.143857421875	-4.36426381847847e-12\\
-36.123359375	-1.21167789022449e-10\\
-36.102861328125	-7.47399030211622e-11\\
-36.08236328125	-1.78838283627482e-10\\
-36.061865234375	-1.31358210136702e-10\\
-36.0413671875	-1.32648472043892e-10\\
-36.020869140625	-1.85368525708294e-10\\
-36.00037109375	-1.93756360082739e-10\\
-35.979873046875	-1.89446413158324e-10\\
-35.959375	-3.13946865600641e-10\\
-35.938876953125	-2.10746681090881e-10\\
-35.91837890625	-3.04570619071647e-10\\
-35.897880859375	-2.60110681163038e-10\\
-35.8773828125	-2.45977444659216e-10\\
-35.856884765625	-2.27142467602347e-10\\
-35.83638671875	-2.72160203091543e-10\\
-35.815888671875	-1.76843962197611e-10\\
-35.795390625	-2.19242361712812e-10\\
-35.774892578125	-1.45054786593115e-10\\
-35.75439453125	-3.15904488040781e-10\\
-35.733896484375	-2.2244734216784e-10\\
-35.7133984375	-2.64372481762778e-10\\
-35.692900390625	-2.34569401662901e-10\\
-35.67240234375	-2.44030415911244e-10\\
-35.651904296875	-1.38751304165167e-10\\
-35.63140625	-2.41352971979471e-10\\
-35.610908203125	-2.82729133405877e-10\\
-35.59041015625	-1.97052494142862e-10\\
-35.569912109375	-2.65861880468044e-10\\
-35.5494140625	-2.35757909917697e-10\\
-35.528916015625	-2.94079380066253e-10\\
-35.50841796875	-3.6824099875944e-10\\
-35.487919921875	-2.87991961853027e-10\\
-35.467421875	-3.22129357013582e-10\\
-35.446923828125	-2.96561671131761e-10\\
-35.42642578125	-3.46221187438232e-10\\
-35.405927734375	-3.47725828853588e-10\\
-35.3854296875	-3.30027573591559e-10\\
-35.364931640625	-1.73731284123401e-10\\
-35.34443359375	-2.2244846189882e-10\\
-35.323935546875	-1.3696888385376e-10\\
-35.3034375	-2.27281404675606e-10\\
-35.282939453125	-1.36813509408624e-10\\
-35.26244140625	-2.77858960865139e-10\\
-35.241943359375	-8.80690603536957e-11\\
-35.2214453125	-2.65461955741111e-10\\
-35.200947265625	-1.68241019058176e-10\\
-35.18044921875	-3.22660094506431e-10\\
-35.159951171875	-2.29349846743522e-10\\
-35.139453125	-3.19987671575588e-10\\
-35.118955078125	-2.1322181320087e-10\\
-35.09845703125	-2.84919654364432e-10\\
-35.077958984375	-1.54620530829645e-10\\
-35.0574609375	-1.87558125755801e-10\\
-35.036962890625	-1.68784672592815e-10\\
-35.01646484375	-1.67637237930322e-10\\
-34.995966796875	-2.29276656316151e-10\\
-34.97546875	-2.05317415452382e-10\\
-34.954970703125	-2.08408669519605e-10\\
-34.93447265625	-1.56402916505216e-10\\
-34.913974609375	-2.7248123319837e-10\\
-34.8934765625	-1.90715752166324e-10\\
-34.872978515625	-2.64549420315457e-10\\
-34.85248046875	-1.1005499935472e-10\\
-34.831982421875	-1.8915989678251e-10\\
-34.811484375	-1.15919103174151e-10\\
-34.790986328125	-8.02725039962858e-11\\
-34.77048828125	3.35815758696776e-12\\
-34.749990234375	-3.80005475644415e-11\\
-34.7294921875	1.17638321855294e-11\\
-34.708994140625	1.05857647792467e-11\\
-34.68849609375	1.01696273460718e-10\\
-34.667998046875	-3.60314018279977e-11\\
-34.6475	4.69420065244623e-11\\
-34.627001953125	1.12810805637331e-10\\
-34.60650390625	9.01424204601594e-11\\
-34.586005859375	1.27445728900914e-10\\
-34.5655078125	7.14019116154892e-11\\
-34.545009765625	1.80761493287762e-10\\
-34.52451171875	1.82046135889404e-10\\
-34.504013671875	2.4382118006815e-10\\
-34.483515625	2.3291571031761e-10\\
-34.463017578125	1.91613471082963e-10\\
-34.44251953125	2.10283566342071e-10\\
-34.422021484375	1.8747382751256e-10\\
-34.4015234375	1.40351325728888e-10\\
-34.381025390625	2.38483463547974e-10\\
-34.36052734375	9.85793181454672e-11\\
-34.340029296875	2.21674745698262e-10\\
-34.31953125	8.21320997704624e-11\\
-34.299033203125	1.69674901260055e-10\\
-34.27853515625	1.31037196265216e-12\\
-34.258037109375	4.47829753485436e-11\\
-34.2375390625	-1.17343873801697e-10\\
-34.217041015625	-5.31599541020613e-11\\
-34.19654296875	-1.29266027056518e-10\\
-34.176044921875	-1.23763339429829e-11\\
-34.155546875	-4.45441571997517e-11\\
-34.135048828125	-1.43784132530104e-10\\
-34.11455078125	-1.12372965999297e-10\\
-34.094052734375	-1.90697208604662e-11\\
-34.0735546875	-1.91233040540656e-10\\
-34.053056640625	-1.4394844140838e-10\\
-34.03255859375	-3.03711649215377e-10\\
-34.012060546875	-2.33192413387561e-10\\
-33.9915625	-2.17612319505937e-10\\
-33.971064453125	-2.30030716728244e-10\\
-33.95056640625	-2.11091935515531e-10\\
-33.930068359375	-1.63547356198702e-10\\
-33.9095703125	-2.0003572298937e-10\\
-33.889072265625	1.79010192067642e-11\\
-33.86857421875	-8.44236816717907e-11\\
-33.848076171875	2.41548725372791e-11\\
-33.827578125	-1.451206047699e-10\\
-33.807080078125	-9.38109569454244e-11\\
-33.78658203125	-8.66613773231233e-11\\
-33.766083984375	-1.2993098653522e-10\\
-33.7455859375	-3.32939928262442e-11\\
-33.725087890625	1.32564627487682e-11\\
-33.70458984375	1.15287461193302e-11\\
-33.684091796875	3.16895539212481e-11\\
-33.66359375	6.58842948530677e-11\\
-33.643095703125	7.20805593406148e-11\\
-33.62259765625	1.80442932618295e-11\\
-33.602099609375	-1.1726651752777e-10\\
-33.5816015625	-2.05172145649521e-11\\
-33.561103515625	1.50813802969478e-11\\
-33.54060546875	3.48016281098005e-11\\
-33.520107421875	1.18386116794343e-10\\
-33.499609375	1.92521177868274e-10\\
-33.479111328125	2.46483411518674e-10\\
-33.45861328125	2.49863887876151e-10\\
-33.438115234375	2.64599990303974e-10\\
-33.4176171875	3.92123099158466e-10\\
-33.397119140625	2.33359805254272e-10\\
-33.37662109375	2.58425101183085e-10\\
-33.356123046875	1.78073847927076e-10\\
-33.335625	2.25166699017868e-10\\
-33.315126953125	1.36589852197979e-10\\
-33.29462890625	1.51322890307245e-10\\
-33.274130859375	1.0698245697143e-10\\
-33.2536328125	2.132185615783e-10\\
-33.233134765625	2.01887837192646e-10\\
-33.21263671875	2.84470907312481e-10\\
-33.192138671875	2.35904334382018e-10\\
-33.171640625	2.28917352151971e-10\\
-33.151142578125	1.68237623609121e-10\\
-33.13064453125	1.76692630549994e-10\\
-33.110146484375	1.69203201801323e-10\\
-33.0896484375	1.76351573602819e-10\\
-33.069150390625	1.51192198750007e-10\\
-33.04865234375	2.32551563648404e-10\\
-33.028154296875	2.18804460550951e-10\\
-33.00765625	2.47626317775727e-10\\
-32.987158203125	2.76949779597622e-10\\
-32.96666015625	2.86177238469753e-10\\
-32.946162109375	1.67783366967882e-10\\
-32.9256640625	1.85294298971715e-10\\
-32.905166015625	1.21145439813666e-10\\
-32.88466796875	2.04202406645455e-10\\
-32.864169921875	1.06394505446589e-10\\
-32.843671875	2.43256343494502e-10\\
-32.823173828125	1.94688989477166e-10\\
-32.80267578125	2.50657737136572e-10\\
-32.782177734375	1.50183773399713e-10\\
-32.7616796875	2.14929264756756e-10\\
-32.741181640625	1.25886972029337e-10\\
-32.72068359375	1.91857563362932e-10\\
-32.700185546875	9.39655711516493e-11\\
-32.6796875	2.21548225385039e-10\\
-32.659189453125	8.10567332998721e-11\\
-32.63869140625	1.97828269279178e-10\\
-32.618193359375	1.09768444992132e-10\\
-32.5976953125	2.83641474833096e-10\\
-32.577197265625	1.36833704563114e-10\\
-32.55669921875	2.66470369785803e-10\\
-32.536201171875	7.63067005612642e-11\\
-32.515703125	1.34500045611448e-10\\
-32.495205078125	8.53008970778084e-11\\
-32.47470703125	3.88868729879023e-11\\
-32.454208984375	8.16072469724982e-11\\
-32.4337109375	2.2036576484755e-11\\
-32.413212890625	7.80654326920302e-11\\
-32.39271484375	-3.11572156874141e-11\\
-32.372216796875	1.70934279069003e-10\\
-32.35171875	-2.85030334222485e-12\\
-32.331220703125	1.54271922417094e-10\\
-32.31072265625	-6.1201312581544e-11\\
-32.290224609375	2.30647919978706e-11\\
-32.2697265625	-1.28691780956718e-10\\
-32.249228515625	-7.44026541461635e-11\\
-32.22873046875	-2.33946413779677e-10\\
-32.208232421875	-2.00365224654428e-10\\
-32.187734375	-2.07012084400355e-10\\
-32.167236328125	-3.08065071061691e-10\\
-32.14673828125	-3.2923293162217e-10\\
-32.126240234375	-3.05024166107625e-10\\
-32.1057421875	-3.18057421924298e-10\\
-32.085244140625	-3.6248964005213e-10\\
-32.06474609375	-2.93390451057643e-10\\
-32.044248046875	-4.2341561187878e-10\\
-32.02375	-4.23715313243362e-10\\
-32.003251953125	-5.06210503935136e-10\\
-31.98275390625	-5.30546865275647e-10\\
-31.962255859375	-5.47588901843848e-10\\
-31.9417578125	-5.81019816827201e-10\\
-31.921259765625	-5.94841189445576e-10\\
-31.90076171875	-5.59996909621933e-10\\
-31.880263671875	-6.43438408633824e-10\\
-31.859765625	-4.92462302341826e-10\\
-31.839267578125	-6.16928984391516e-10\\
-31.81876953125	-3.77264983205633e-10\\
-31.798271484375	-4.91722395618548e-10\\
-31.7777734375	-2.95471999425628e-10\\
-31.757275390625	-4.22805014321533e-10\\
-31.73677734375	-2.49312334246659e-10\\
-31.716279296875	-3.24782416030547e-10\\
-31.69578125	-1.72536215398124e-10\\
-31.675283203125	-3.59465194175305e-10\\
-31.65478515625	-2.52611592345664e-10\\
-31.634287109375	-2.94052545272836e-10\\
-31.6137890625	-1.99942513210539e-10\\
-31.593291015625	-2.78231513664007e-10\\
-31.57279296875	-1.37055977445849e-10\\
-31.552294921875	-1.90398891933584e-10\\
-31.531796875	-1.49192812745467e-10\\
-31.511298828125	-1.47733674267556e-10\\
-31.49080078125	-3.29329519324962e-11\\
-31.470302734375	-1.84050936607243e-10\\
-31.4498046875	-1.12259430589829e-10\\
-31.429306640625	-1.34167611714569e-10\\
-31.40880859375	-1.63093687549816e-10\\
-31.388310546875	-2.3720690610579e-10\\
-31.3678125	-1.20792052370813e-10\\
-31.347314453125	-3.08392290891971e-10\\
-31.32681640625	-1.56646901572918e-10\\
-31.306318359375	-2.01766509344413e-10\\
-31.2858203125	-1.7072909800821e-10\\
-31.265322265625	-1.7784208648308e-10\\
-31.24482421875	-2.1317930276112e-10\\
-31.224326171875	-2.58112018997327e-10\\
-31.203828125	-2.86220653895345e-10\\
-31.183330078125	-3.92372042033462e-10\\
-31.16283203125	-3.67306136980351e-10\\
-31.142333984375	-3.33641411889008e-10\\
-31.1218359375	-2.6014788230901e-10\\
-31.101337890625	-2.40356523147226e-10\\
-31.08083984375	-2.96380845767929e-10\\
-31.060341796875	-2.46275562999253e-10\\
-31.03984375	-3.59734436378868e-10\\
-31.019345703125	-2.99128532137434e-10\\
-30.99884765625	-3.22804478134041e-10\\
-30.978349609375	-4.2360450605639e-10\\
-30.9578515625	-4.44236035864364e-10\\
-30.937353515625	-5.78690558472403e-10\\
-30.91685546875	-5.6138601141847e-10\\
-30.896357421875	-5.43100574569932e-10\\
-30.875859375	-6.41766084862431e-10\\
-30.855361328125	-4.65045147091411e-10\\
-30.83486328125	-4.44513415492352e-10\\
-30.814365234375	-3.92473271732638e-10\\
-30.7938671875	-4.09989779841488e-10\\
-30.773369140625	-3.74040894078568e-10\\
-30.75287109375	-4.53415752285521e-10\\
-30.732373046875	-3.66659183530585e-10\\
-30.711875	-5.33251310745361e-10\\
-30.691376953125	-5.39351115960254e-10\\
-30.67087890625	-6.01175376546962e-10\\
-30.650380859375	-5.97111200560555e-10\\
-30.6298828125	-5.47716573853284e-10\\
-30.609384765625	-5.2321029719581e-10\\
-30.58888671875	-5.13879191152601e-10\\
-30.568388671875	-4.52918471821387e-10\\
-30.547890625	-4.61064392564769e-10\\
-30.527392578125	-4.16204047868864e-10\\
-30.50689453125	-4.51602654039996e-10\\
-30.486396484375	-4.83309350291469e-10\\
-30.4658984375	-5.40480676181337e-10\\
-30.445400390625	-6.1240291802484e-10\\
-30.42490234375	-6.9056551361534e-10\\
-30.404404296875	-5.87917001825043e-10\\
-30.38390625	-6.85396890488235e-10\\
-30.363408203125	-6.11096519873672e-10\\
-30.34291015625	-6.60744014741398e-10\\
-30.322412109375	-5.41745360567067e-10\\
-30.3019140625	-6.43126551553997e-10\\
-30.281416015625	-4.71974244081527e-10\\
-30.26091796875	-4.78161569860462e-10\\
-30.240419921875	-5.08157623839926e-10\\
-30.219921875	-5.398599257645e-10\\
-30.199423828125	-5.25976707357913e-10\\
-30.17892578125	-7.09746623814949e-10\\
-30.158427734375	-5.66006378465421e-10\\
-30.1379296875	-7.386472246946e-10\\
-30.117431640625	-5.97451457357162e-10\\
-30.09693359375	-6.76122427412786e-10\\
-30.076435546875	-6.43207344825355e-10\\
-30.0559375	-6.33724225825745e-10\\
-30.035439453125	-4.99624313643317e-10\\
-30.01494140625	-6.94868661500714e-10\\
-29.994443359375	-6.15971068556158e-10\\
-29.9739453125	-6.70398013307514e-10\\
-29.953447265625	-8.2191411970249e-10\\
-29.93294921875	-7.10587769169451e-10\\
-29.912451171875	-8.22347206885721e-10\\
-29.891953125	-5.89459813338398e-10\\
-29.871455078125	-8.03939222241849e-10\\
-29.85095703125	-5.79338260989873e-10\\
-29.830458984375	-6.73121525878508e-10\\
-29.8099609375	-5.29587817419377e-10\\
-29.789462890625	-6.66625333128542e-10\\
-29.76896484375	-5.43093037658006e-10\\
-29.748466796875	-6.65176270958881e-10\\
-29.72796875	-5.08644124530748e-10\\
-29.707470703125	-5.7688430968257e-10\\
-29.68697265625	-5.04945763904081e-10\\
-29.666474609375	-4.81788006758791e-10\\
-29.6459765625	-3.63061058343069e-10\\
-29.625478515625	-3.25093328380624e-10\\
-29.60498046875	-1.72785702956999e-10\\
-29.584482421875	-1.31288840481602e-10\\
-29.563984375	-9.67179461755092e-11\\
-29.543486328125	-1.24703786805788e-10\\
-29.52298828125	-1.6551457359748e-10\\
-29.502490234375	-1.74441011826686e-11\\
-29.4819921875	-2.62498089193254e-11\\
-29.461494140625	9.63971093953089e-11\\
-29.44099609375	5.53762791164291e-11\\
-29.420498046875	1.16970632131207e-10\\
-29.4	2.03674039102718e-10\\
-29.379501953125	2.14407497326207e-10\\
-29.35900390625	1.62892183562076e-10\\
-29.338505859375	2.65368431708031e-10\\
-29.3180078125	1.13864317534071e-10\\
-29.297509765625	2.37187254305105e-10\\
-29.27701171875	8.08348712407892e-11\\
-29.256513671875	1.59844583958699e-10\\
-29.236015625	1.09628545337205e-10\\
-29.215517578125	1.74825477734886e-10\\
-29.19501953125	1.05878692152172e-10\\
-29.174521484375	3.03531976236528e-10\\
-29.1540234375	1.05831251292924e-10\\
-29.133525390625	1.98919002915899e-10\\
-29.11302734375	4.00004880522366e-11\\
-29.092529296875	-6.03547658398e-12\\
-29.07203125	-8.271192994848e-11\\
-29.051533203125	-5.6832958419947e-11\\
-29.03103515625	-1.12767891878213e-10\\
-29.010537109375	-9.80605128115017e-11\\
-28.9900390625	-1.80665865413098e-10\\
-28.969541015625	-1.3348406408401e-10\\
-28.94904296875	-1.48348869323676e-10\\
-28.928544921875	-7.89883045249923e-11\\
-28.908046875	-1.38629759545825e-10\\
-28.887548828125	-1.5707125970648e-10\\
-28.86705078125	-2.30947743073988e-10\\
-28.846552734375	-1.38067239811052e-10\\
-28.8260546875	-2.91833799222184e-10\\
-28.805556640625	-1.20503155860695e-10\\
-28.78505859375	-2.05707527312886e-10\\
-28.764560546875	-3.95877339096604e-11\\
-28.7440625	-9.50352341733775e-11\\
-28.723564453125	-1.02308670982197e-10\\
-28.70306640625	-1.08361157550089e-10\\
-28.682568359375	1.30220776082879e-11\\
-28.6620703125	-3.62605524171018e-11\\
-28.641572265625	3.08225403563653e-11\\
-28.62107421875	7.25884321330888e-11\\
-28.600576171875	6.23901725022546e-11\\
-28.580078125	7.85144799853516e-11\\
-28.559580078125	-1.5022789712931e-11\\
-28.53908203125	4.41228276978601e-11\\
-28.518583984375	-5.36149673750907e-11\\
-28.4980859375	-1.82438886398194e-11\\
-28.477587890625	9.79966958210713e-12\\
-28.45708984375	1.63025208292688e-10\\
-28.436591796875	1.92612438206156e-10\\
-28.41609375	2.167741293193e-10\\
-28.395595703125	3.28203802220666e-10\\
-28.37509765625	3.44320731897049e-10\\
-28.354599609375	2.10497358365128e-10\\
-28.3341015625	3.15752964519593e-10\\
-28.313603515625	1.75792115772226e-10\\
-28.29310546875	9.87682029592843e-11\\
-28.272607421875	4.28532093573201e-11\\
-28.252109375	5.68336427805671e-11\\
-28.231611328125	4.10728885136771e-11\\
-28.21111328125	3.09786633703127e-10\\
-28.190615234375	2.02135334886325e-10\\
-28.1701171875	3.68011409974797e-10\\
-28.149619140625	3.73556409909117e-10\\
-28.12912109375	3.68424759131772e-10\\
-28.108623046875	3.33497756916911e-10\\
-28.088125	3.41319863511745e-10\\
-28.067626953125	2.43350446703982e-10\\
-28.04712890625	1.93641100503023e-10\\
-28.026630859375	8.16713825613025e-11\\
-28.0061328125	2.44102805773459e-10\\
-27.985634765625	3.43031577115334e-10\\
-27.96513671875	3.96390785619135e-10\\
-27.944638671875	5.25598240259149e-10\\
-27.924140625	5.66915962461263e-10\\
-27.903642578125	3.87294225925344e-10\\
-27.88314453125	4.64329692584739e-10\\
-27.862646484375	3.77824068698341e-10\\
-27.8421484375	3.58822991543999e-10\\
-27.821650390625	2.44413461044083e-10\\
-27.80115234375	3.05624048168844e-10\\
-27.780654296875	3.14902747105997e-10\\
-27.76015625	3.87427289966702e-10\\
-27.739658203125	3.25468265471835e-10\\
-27.71916015625	3.84690905644364e-10\\
-27.698662109375	3.62692516407118e-10\\
-27.6781640625	4.95441701982264e-10\\
-27.657666015625	4.08646117788393e-10\\
-27.63716796875	4.93570992632801e-10\\
-27.616669921875	2.87585596638182e-10\\
-27.596171875	4.24473830303618e-10\\
-27.575673828125	3.14146641158087e-10\\
-27.55517578125	4.16804996648024e-10\\
-27.534677734375	3.31228159579342e-10\\
-27.5141796875	3.53864040336441e-10\\
-27.493681640625	3.80176515974373e-10\\
-27.47318359375	4.3455484971258e-10\\
-27.452685546875	5.00312676323037e-10\\
-27.4321875	4.31263093352119e-10\\
-27.411689453125	5.51775132350384e-10\\
-27.39119140625	3.82395165017079e-10\\
-27.370693359375	4.25972591410735e-10\\
-27.3501953125	3.11864148222014e-10\\
-27.329697265625	4.91986054395607e-10\\
-27.30919921875	2.34549310702442e-10\\
-27.288701171875	4.30409903316231e-10\\
-27.268203125	2.41458670941011e-10\\
-27.247705078125	3.80058041424944e-10\\
-27.22720703125	2.57497233398809e-10\\
-27.206708984375	3.15041460193005e-10\\
-27.1862109375	1.78613261211203e-10\\
-27.165712890625	2.22218780147741e-10\\
-27.14521484375	1.41275906659502e-10\\
-27.124716796875	5.23228013707437e-11\\
-27.10421875	5.50512914704859e-11\\
-27.083720703125	5.77374376288308e-13\\
-27.06322265625	-1.43505694365446e-10\\
-27.042724609375	-1.27880964916839e-10\\
-27.0222265625	-2.07298536785748e-10\\
-27.001728515625	-1.85180934364783e-10\\
-26.98123046875	-1.96571105588254e-10\\
-26.960732421875	-2.32933806927177e-10\\
-26.940234375	-2.27863748813029e-10\\
-26.919736328125	-3.64130847777452e-10\\
-26.89923828125	-4.20927804864054e-10\\
-26.878740234375	-5.29064101665882e-10\\
-26.8582421875	-5.47644604732973e-10\\
-26.837744140625	-6.55520232803403e-10\\
-26.81724609375	-4.74444588300903e-10\\
-26.796748046875	-6.22542857582195e-10\\
-26.77625	-3.600075668467e-10\\
-26.755751953125	-4.67281823535086e-10\\
-26.73525390625	-2.26310585744298e-10\\
-26.714755859375	-3.89829618151522e-10\\
-26.6942578125	-1.77248013824442e-10\\
-26.673759765625	-3.53405854241671e-10\\
-26.65326171875	-2.95863910602781e-10\\
-26.632763671875	-5.23094188727653e-10\\
-26.612265625	-3.7715456637735e-10\\
-26.591767578125	-5.32332668129498e-10\\
-26.57126953125	-4.33783743315569e-10\\
-26.550771484375	-4.38126440193259e-10\\
-26.5302734375	-2.35454321730349e-10\\
-26.509775390625	-2.2138745253265e-10\\
-26.48927734375	-1.58635343314882e-10\\
-26.468779296875	-1.20443727840783e-10\\
-26.44828125	-2.95954455263914e-11\\
-26.427783203125	-1.69656910461383e-10\\
-26.40728515625	-1.49640448149117e-10\\
-26.386787109375	-7.71708789709696e-11\\
-26.3662890625	-1.97336397813891e-10\\
-26.345791015625	-1.98963610175176e-10\\
-26.32529296875	-1.60901641565084e-10\\
-26.304794921875	-1.55368358245746e-10\\
-26.284296875	-4.49082121857232e-11\\
-26.263798828125	-2.33367307264461e-10\\
-26.24330078125	-1.31445902528433e-11\\
-26.222802734375	-1.25387998546557e-10\\
-26.2023046875	-1.26628094086963e-10\\
-26.181806640625	-1.97587739069607e-10\\
-26.16130859375	-1.55093920548396e-10\\
-26.140810546875	-2.34958575223599e-10\\
-26.1203125	-2.44436527481089e-10\\
-26.099814453125	-2.11852166625443e-10\\
-26.07931640625	-1.85114070898158e-10\\
-26.058818359375	-1.53456012311295e-10\\
-26.0383203125	-1.93708471760091e-10\\
-26.017822265625	-1.73814999117219e-10\\
-25.99732421875	-1.73632647692179e-10\\
-25.976826171875	-1.90947052169913e-10\\
-25.956328125	-2.86382183801622e-10\\
-25.935830078125	-4.39261309617766e-10\\
-25.91533203125	-4.80917808632753e-10\\
-25.894833984375	-5.81911272481082e-10\\
-25.8743359375	-5.49797181590928e-10\\
-25.853837890625	-6.06338612467341e-10\\
-25.83333984375	-6.47662212101812e-10\\
-25.812841796875	-5.34738971738871e-10\\
-25.79234375	-5.7497744240267e-10\\
-25.771845703125	-5.63671676454569e-10\\
-25.75134765625	-6.16792576046291e-10\\
-25.730849609375	-5.68694401316383e-10\\
-25.7103515625	-7.21905238973256e-10\\
-25.689853515625	-5.81144045090617e-10\\
-25.66935546875	-7.96104755847739e-10\\
-25.648857421875	-6.80409016494201e-10\\
-25.628359375	-6.28111759757319e-10\\
-25.607861328125	-7.65591341472487e-10\\
-25.58736328125	-6.52607505190426e-10\\
-25.566865234375	-6.95695847988677e-10\\
-25.5463671875	-7.23974665084902e-10\\
-25.525869140625	-5.91443870986594e-10\\
-25.50537109375	-6.9585384111776e-10\\
-25.484873046875	-6.38227837283618e-10\\
-25.464375	-7.26888388987428e-10\\
-25.443876953125	-8.37359129403379e-10\\
-25.42337890625	-8.76180885738629e-10\\
-25.402880859375	-8.29863655757017e-10\\
-25.3823828125	-8.70030285371956e-10\\
-25.361884765625	-7.14211369196505e-10\\
-25.34138671875	-7.75759186397001e-10\\
-25.320888671875	-6.13954280005404e-10\\
-25.300390625	-6.01735023921947e-10\\
-25.279892578125	-5.19907426235139e-10\\
-25.25939453125	-7.41643216034341e-10\\
-25.238896484375	-6.16389463340787e-10\\
-25.2183984375	-8.58942832840968e-10\\
-25.197900390625	-7.86465433812048e-10\\
-25.17740234375	-7.79604706534199e-10\\
-25.156904296875	-7.49200659315848e-10\\
-25.13640625	-7.57767369138055e-10\\
-25.115908203125	-4.95482024229156e-10\\
-25.09541015625	-4.74136970004202e-10\\
-25.074912109375	-4.06120510865629e-10\\
-25.0544140625	-5.10360331108694e-10\\
-25.033916015625	-3.97520790569756e-10\\
-25.01341796875	-5.70809312798279e-10\\
-24.992919921875	-5.84543385213371e-10\\
-24.972421875	-6.90630947470857e-10\\
-24.951923828125	-5.75314441182521e-10\\
-24.93142578125	-6.33000686191616e-10\\
-24.910927734375	-7.45671598587667e-10\\
-24.8904296875	-5.69710022077677e-10\\
-24.869931640625	-4.97626568851973e-10\\
-24.84943359375	-3.46485845282948e-10\\
-24.828935546875	-4.37802139098719e-10\\
-24.8084375	-2.19652844761408e-10\\
-24.787939453125	-3.50807675775231e-10\\
-24.76744140625	-1.62823692629748e-10\\
-24.746943359375	-3.36093661706271e-10\\
-24.7264453125	-1.40781420427282e-10\\
-24.705947265625	-3.35451746652683e-10\\
-24.68544921875	-2.01976077971607e-10\\
-24.664951171875	-2.86580453061014e-10\\
-24.644453125	-1.17007653542985e-10\\
-24.623955078125	-1.02237102083155e-10\\
-24.60345703125	6.49557993842721e-11\\
-24.582958984375	9.25344116449014e-11\\
-24.5624609375	2.68848960003735e-10\\
-24.541962890625	3.23359949727907e-10\\
-24.52146484375	3.75101311602251e-10\\
-24.500966796875	4.02466918064426e-10\\
-24.48046875	4.05680923281675e-10\\
-24.459970703125	3.89282341787163e-10\\
-24.43947265625	4.03887360594605e-10\\
-24.418974609375	5.08250497553544e-10\\
-24.3984765625	5.90721690758441e-10\\
-24.377978515625	6.93615558584366e-10\\
-24.35748046875	7.49983199602069e-10\\
-24.336982421875	8.36571099285064e-10\\
-24.316484375	7.38390269507214e-10\\
-24.295986328125	8.97847255473588e-10\\
-24.27548828125	6.82568806358489e-10\\
-24.254990234375	8.39956993424498e-10\\
-24.2344921875	6.96029006303244e-10\\
-24.213994140625	8.12854421338418e-10\\
-24.19349609375	6.53381135543278e-10\\
-24.172998046875	9.01569718591716e-10\\
-24.1525	7.61135736354055e-10\\
-24.132001953125	1.07220631823358e-09\\
-24.11150390625	8.89424069299075e-10\\
-24.091005859375	1.07553830183878e-09\\
-24.0705078125	9.22634108201924e-10\\
-24.050009765625	8.15926940368085e-10\\
-24.02951171875	6.81466226697773e-10\\
-24.009013671875	7.12056480407344e-10\\
-23.988515625	6.39895974410481e-10\\
-23.968017578125	5.47673817163969e-10\\
-23.94751953125	4.23565522260601e-10\\
-23.927021484375	5.58781535971922e-10\\
-23.9065234375	5.99504678636822e-10\\
-23.886025390625	4.98647427719573e-10\\
-23.86552734375	6.40258424417758e-10\\
-23.845029296875	5.54460385815783e-10\\
-23.82453125	5.9447203152916e-10\\
-23.804033203125	4.98904793856494e-10\\
-23.78353515625	4.69069927312811e-10\\
-23.763037109375	6.0544640868016e-10\\
-23.7425390625	4.22411529760707e-10\\
-23.722041015625	6.24112345735304e-10\\
-23.70154296875	4.65250925728572e-10\\
-23.681044921875	5.86474132095106e-10\\
-23.660546875	6.09340044751481e-10\\
-23.640048828125	6.38282756026853e-10\\
-23.61955078125	6.01771267576185e-10\\
-23.599052734375	7.23275644487022e-10\\
-23.5785546875	6.00459830798433e-10\\
-23.558056640625	6.01200120191943e-10\\
-23.53755859375	6.39634258712568e-10\\
-23.517060546875	6.16641418681675e-10\\
-23.4965625	6.20260813390563e-10\\
-23.476064453125	7.11897058309213e-10\\
-23.45556640625	6.8077142825278e-10\\
-23.435068359375	8.68923769818716e-10\\
-23.4145703125	8.54639270151203e-10\\
-23.394072265625	1.00586372581475e-09\\
-23.37357421875	1.12235458302105e-09\\
-23.353076171875	1.18081058692039e-09\\
-23.332578125	1.15103786784728e-09\\
-23.312080078125	1.18180382462626e-09\\
-23.29158203125	1.35514223971541e-09\\
-23.271083984375	1.21656138907517e-09\\
-23.2505859375	1.3262019803161e-09\\
-23.230087890625	1.25796293080688e-09\\
-23.20958984375	1.4151803592418e-09\\
-23.189091796875	1.36451853439134e-09\\
-23.16859375	1.5817325858206e-09\\
-23.148095703125	1.43105758011692e-09\\
-23.12759765625	1.49659232148965e-09\\
-23.107099609375	1.53528826160855e-09\\
-23.0866015625	1.26347450727743e-09\\
-23.066103515625	1.38197828920227e-09\\
-23.04560546875	1.29087431556004e-09\\
-23.025107421875	1.07104133758633e-09\\
-23.004609375	1.18623168001136e-09\\
-22.984111328125	1.10963200352015e-09\\
-22.96361328125	1.20127925651425e-09\\
-22.943115234375	1.2515735054182e-09\\
-22.9226171875	1.22896489651992e-09\\
-22.902119140625	1.18179266364472e-09\\
-22.88162109375	1.16825755967319e-09\\
-22.861123046875	1.02561111921729e-09\\
-22.840625	1.02938820900196e-09\\
-22.820126953125	8.65947516388412e-10\\
-22.79962890625	6.99621434461686e-10\\
-22.779130859375	5.01661474421144e-10\\
-22.7586328125	7.10899764636231e-10\\
-22.738134765625	5.87545204790211e-10\\
-22.71763671875	7.71161740384642e-10\\
-22.697138671875	6.50538206859006e-10\\
-22.676640625	7.49067403675316e-10\\
-22.656142578125	5.71097464322075e-10\\
-22.63564453125	6.16731232834048e-10\\
-22.615146484375	3.48862774904497e-10\\
-22.5946484375	3.85870189923397e-10\\
-22.574150390625	2.00406985452281e-10\\
-22.55365234375	2.84330438040096e-10\\
-22.533154296875	3.30169467887547e-10\\
-22.51265625	4.23104418981409e-10\\
-22.492158203125	3.56658183593776e-10\\
-22.47166015625	5.73660872446669e-10\\
-22.451162109375	4.45605421699668e-10\\
-22.4306640625	3.61299493731502e-10\\
-22.410166015625	3.79545205943015e-10\\
-22.38966796875	-7.72121597461469e-13\\
-22.369169921875	1.11409378854824e-10\\
-22.348671875	-1.18975975132174e-10\\
-22.328173828125	1.55162563634641e-10\\
-22.30767578125	1.84209163975101e-11\\
-22.287177734375	2.87253706225225e-10\\
-22.2666796875	1.55776072565827e-10\\
-22.246181640625	5.95276015079022e-10\\
-22.22568359375	2.87163001982804e-10\\
-22.205185546875	5.48355410915693e-10\\
-22.1846875	2.86694633565218e-10\\
-22.164189453125	2.89264925495114e-10\\
-22.14369140625	8.66741824072562e-11\\
-22.123193359375	1.22413538672947e-10\\
-22.1026953125	-8.29972384102758e-11\\
-22.082197265625	2.17925189936873e-11\\
-22.06169921875	2.49450491684089e-11\\
-22.041201171875	8.19839081660152e-11\\
-22.020703125	4.30624543010721e-11\\
-22.000205078125	1.37487688265725e-10\\
-21.97970703125	2.01482231601381e-10\\
-21.959208984375	3.16826280669173e-10\\
-21.9387109375	2.01914194215592e-10\\
-21.918212890625	1.13424679144872e-10\\
-21.89771484375	9.81874366026376e-11\\
-21.877216796875	-4.72448477542831e-11\\
-21.85671875	-1.19295515794117e-12\\
-21.836220703125	-1.20992114038462e-10\\
-21.81572265625	-8.67374506601598e-12\\
-21.795224609375	-1.32687959484098e-10\\
-21.7747265625	2.15677436017177e-11\\
-21.754228515625	-1.83755497603358e-10\\
-21.73373046875	1.53950867737631e-11\\
-21.713232421875	-1.86436422606618e-10\\
-21.692734375	-1.27229729092916e-10\\
-21.672236328125	-3.56981026785066e-10\\
-21.65173828125	-3.09478736301434e-10\\
-21.631240234375	-5.45184946402655e-10\\
-21.6107421875	-5.26666527853294e-10\\
-21.590244140625	-5.50342675111319e-10\\
-21.56974609375	-4.33889185723131e-10\\
-21.549248046875	-4.66156895457161e-10\\
-21.52875	-2.96145547884648e-10\\
-21.508251953125	-2.42247840619871e-10\\
-21.48775390625	-2.25222455784124e-10\\
-21.467255859375	-1.00247355288605e-10\\
-21.4467578125	-2.10787564724746e-10\\
-21.426259765625	-2.49325887039036e-10\\
-21.40576171875	-3.85463366915432e-10\\
-21.385263671875	-2.27821991826621e-10\\
-21.364765625	-2.74808958973623e-10\\
-21.344267578125	-1.94811559385095e-10\\
-21.32376953125	-1.79518640501692e-10\\
-21.303271484375	3.98236437966376e-11\\
-21.2827734375	9.0450604091216e-11\\
-21.262275390625	1.36835796038806e-11\\
-21.24177734375	2.40335925203355e-10\\
-21.221279296875	6.22227275918352e-11\\
-21.20078125	2.37198488374225e-10\\
-21.180283203125	7.20296015248876e-11\\
-21.15978515625	2.27040928487136e-10\\
-21.139287109375	1.27016134387216e-10\\
-21.1187890625	1.99737891701956e-10\\
-21.098291015625	1.05251654811962e-10\\
-21.07779296875	2.30354468497827e-10\\
-21.057294921875	1.63181377449173e-10\\
-21.036796875	1.26966113477056e-10\\
-21.016298828125	7.27444474885078e-11\\
-20.99580078125	1.10048046440004e-10\\
-20.975302734375	-1.38588394323927e-10\\
-20.9548046875	-3.34480255599101e-11\\
-20.934306640625	-2.2767749634487e-10\\
-20.91380859375	-3.12296619345648e-10\\
-20.893310546875	-3.6307264498644e-10\\
-20.8728125	-4.19200264277836e-10\\
-20.852314453125	-3.61974409123057e-10\\
-20.83181640625	-4.18587524682248e-10\\
-20.811318359375	-3.51730243147511e-10\\
-20.7908203125	-5.5830454845349e-10\\
-20.770322265625	-2.65679417859927e-10\\
-20.74982421875	-4.87301500600753e-10\\
-20.729326171875	-3.57185462631221e-10\\
-20.708828125	-4.96609841275818e-10\\
-20.688330078125	-4.17843888292328e-10\\
-20.66783203125	-6.69446975200703e-10\\
-20.647333984375	-6.52002437301948e-10\\
-20.6268359375	-6.94421688937825e-10\\
-20.606337890625	-7.59701191752861e-10\\
-20.58583984375	-5.21471139463458e-10\\
-20.565341796875	-6.09229244394644e-10\\
-20.54484375	-6.00654459503299e-10\\
-20.524345703125	-4.98478678685929e-10\\
-20.50384765625	-6.55352099659392e-10\\
-20.483349609375	-4.33263476481658e-10\\
-20.4628515625	-7.20257423005953e-10\\
-20.442353515625	-5.48260840205078e-10\\
-20.42185546875	-6.90733149684236e-10\\
-20.401357421875	-6.6670700210634e-10\\
-20.380859375	-6.42100727280871e-10\\
-20.360361328125	-4.75130708070476e-10\\
-20.33986328125	-5.64182912849423e-10\\
-20.319365234375	-4.5622163027748e-10\\
-20.2988671875	-4.38230521944817e-10\\
-20.278369140625	-3.7631378905255e-10\\
-20.25787109375	-6.15894393025248e-10\\
-20.237373046875	-4.58186172619211e-10\\
-20.216875	-4.28867003422204e-10\\
-20.196376953125	-4.88890854523688e-10\\
-20.17587890625	-5.19416236882332e-10\\
-20.155380859375	-3.10882172381399e-10\\
-20.1348828125	-2.59664998989225e-10\\
-20.114384765625	-3.55280649994725e-10\\
-20.09388671875	-4.6015211555691e-10\\
-20.073388671875	-3.62248666868513e-10\\
-20.052890625	-4.78855838942855e-10\\
-20.032392578125	-6.64825021707195e-10\\
-20.01189453125	-4.56193931024715e-10\\
-19.991396484375	-5.02265207883117e-10\\
-19.9708984375	-6.06972366124695e-10\\
-19.950400390625	-4.14176426313625e-10\\
-19.92990234375	-3.45602510438655e-10\\
-19.909404296875	-4.03824111780111e-10\\
-19.88890625	-7.83532069994269e-11\\
-19.868408203125	-4.52597455416137e-10\\
-19.84791015625	-1.53499505044951e-10\\
-19.827412109375	-6.21174081974946e-10\\
-19.8069140625	-4.42878189148015e-10\\
-19.786416015625	-5.9679271314183e-10\\
-19.76591796875	-4.88238419274499e-10\\
-19.745419921875	-7.78890567379068e-10\\
-19.724921875	-3.83511906695978e-10\\
-19.704423828125	-4.92260817213037e-10\\
-19.68392578125	-2.1082185778566e-10\\
-19.663427734375	-3.38490560738268e-10\\
-19.6429296875	-1.70557343701868e-10\\
-19.622431640625	-3.66844528630623e-10\\
-19.60193359375	-3.53448468529959e-10\\
-19.581435546875	-4.29405324263619e-10\\
-19.5609375	-4.41201127651865e-10\\
-19.540439453125	-6.19872782139401e-10\\
-19.51994140625	-4.11570802999955e-10\\
-19.499443359375	-4.30027741145084e-10\\
-19.4789453125	-2.76915719104535e-10\\
-19.458447265625	-1.34645267721603e-10\\
-19.43794921875	-4.52250595600916e-11\\
-19.417451171875	2.95592286852425e-12\\
-19.396953125	-1.22788226140305e-10\\
-19.376455078125	-6.39758648629864e-11\\
-19.35595703125	-2.01703437216587e-10\\
-19.335458984375	-2.32310997863054e-10\\
-19.3149609375	-3.35283568284347e-10\\
-19.294462890625	-2.75064202412166e-10\\
-19.27396484375	-3.27248188288309e-10\\
-19.253466796875	-2.59585903387177e-11\\
-19.23296875	-1.54686500337408e-10\\
-19.212470703125	6.74184371473896e-11\\
-19.19197265625	3.087568294331e-11\\
-19.171474609375	2.11194312634093e-10\\
-19.1509765625	2.98109090541633e-11\\
-19.130478515625	2.18575056875788e-10\\
-19.10998046875	7.93872191473836e-11\\
-19.089482421875	1.09724066328554e-10\\
-19.068984375	3.35092533241994e-11\\
-19.048486328125	1.54172591812023e-10\\
-19.02798828125	1.62427293393061e-10\\
-19.007490234375	1.58704315730214e-10\\
-18.9869921875	1.71195250525395e-10\\
-18.966494140625	4.88492746021333e-10\\
-18.94599609375	4.12231832038702e-10\\
-18.925498046875	4.96577508323199e-10\\
-18.905	5.03688476601446e-10\\
-18.884501953125	4.3499344630222e-10\\
-18.86400390625	3.45078669949011e-10\\
-18.843505859375	2.7123906749519e-10\\
-18.8230078125	2.691278010345e-10\\
-18.802509765625	1.85598721166023e-10\\
-18.78201171875	2.84644097245409e-10\\
-18.761513671875	2.16226450120528e-10\\
-18.741015625	1.65775077671695e-10\\
-18.720517578125	4.67933480537548e-10\\
-18.70001953125	2.67699650606872e-10\\
-18.679521484375	4.78822207646066e-10\\
-18.6590234375	2.59112781070615e-10\\
-18.638525390625	3.13680492608768e-10\\
-18.61802734375	1.25006601461509e-10\\
-18.597529296875	1.87565886095819e-10\\
-18.57703125	3.37271907073364e-11\\
-18.556533203125	8.08711594577989e-11\\
-18.53603515625	1.63795641198648e-10\\
-18.515537109375	3.10466582064776e-10\\
-18.4950390625	2.99252792757893e-10\\
-18.474541015625	3.94330310288136e-10\\
-18.45404296875	2.64584804357188e-10\\
-18.433544921875	3.95324935462703e-10\\
-18.413046875	4.48042579779106e-10\\
-18.392548828125	3.82306140838143e-10\\
-18.37205078125	4.96203741704122e-10\\
-18.351552734375	4.43336531245234e-10\\
-18.3310546875	6.58527783266348e-10\\
-18.310556640625	6.36975763307709e-10\\
-18.29005859375	6.7848667182672e-10\\
-18.269560546875	6.34409227106023e-10\\
-18.2490625	7.17648965295585e-10\\
-18.228564453125	5.07666008115745e-10\\
-18.20806640625	5.74959684956857e-10\\
-18.187568359375	4.16900620189304e-10\\
-18.1670703125	6.30490400734994e-10\\
-18.146572265625	4.39023667087291e-10\\
-18.12607421875	6.17731418028974e-10\\
-18.105576171875	7.08461795972264e-10\\
-18.085078125	4.40450009046096e-10\\
-18.064580078125	6.53310332517895e-10\\
-18.04408203125	5.49472992329849e-10\\
-18.023583984375	5.00995445967862e-10\\
-18.0030859375	4.401300043287e-10\\
-17.982587890625	3.52902387068413e-10\\
-17.96208984375	5.70701302832612e-10\\
-17.941591796875	3.38784434600776e-10\\
-17.92109375	5.20869691623229e-10\\
-17.900595703125	5.68639843476592e-10\\
-17.88009765625	5.89162472353604e-10\\
-17.859599609375	5.46680201597061e-10\\
-17.8391015625	6.31897515070049e-10\\
-17.818603515625	4.90662942106221e-10\\
-17.79810546875	5.68742803682158e-10\\
-17.777607421875	3.69823558033397e-10\\
-17.757109375	5.57473980335295e-10\\
-17.736611328125	4.77754698160173e-10\\
-17.71611328125	4.45548639393748e-10\\
-17.695615234375	5.8266613778143e-10\\
-17.6751171875	5.53398540487918e-10\\
-17.654619140625	4.73835922987189e-10\\
-17.63412109375	5.79447677148987e-10\\
-17.613623046875	5.88841458550918e-10\\
-17.593125	6.4332509907213e-10\\
-17.572626953125	5.44750706306448e-10\\
-17.55212890625	5.8719609752202e-10\\
-17.531630859375	7.55423865104352e-10\\
-17.5111328125	4.82931251639621e-10\\
-17.490634765625	5.65334258916957e-10\\
-17.47013671875	5.82865121212922e-10\\
-17.449638671875	4.99458923645612e-10\\
-17.429140625	5.00223667174132e-10\\
-17.408642578125	5.91347682032056e-10\\
-17.38814453125	4.93118944058737e-10\\
-17.367646484375	7.03644462443943e-10\\
-17.3471484375	3.72324639161051e-10\\
-17.326650390625	8.01735414001389e-10\\
-17.30615234375	3.85718592240272e-10\\
-17.285654296875	5.26757573971792e-10\\
-17.26515625	2.40386976661874e-10\\
-17.244658203125	4.17311276101137e-10\\
-17.22416015625	2.60584978380589e-10\\
-17.203662109375	4.90038051949943e-10\\
-17.1831640625	4.84227998462701e-10\\
-17.162666015625	5.91471191416091e-10\\
-17.14216796875	5.07979465320138e-10\\
-17.121669921875	6.02486449875216e-10\\
-17.101171875	5.21518850413121e-10\\
-17.080673828125	5.21893223101474e-10\\
-17.06017578125	3.00990953295536e-10\\
-17.039677734375	3.21586959014066e-10\\
-17.0191796875	7.35430906478745e-11\\
-16.998681640625	1.06231118417232e-10\\
-16.97818359375	-8.21578790763479e-11\\
-16.957685546875	7.14037851634502e-11\\
-16.9371875	-3.50092779072766e-11\\
-16.916689453125	4.09131871608851e-11\\
-16.89619140625	1.28396570111802e-10\\
-16.875693359375	2.01057238302713e-11\\
-16.8551953125	1.99834925462187e-10\\
-16.834697265625	9.02999292144782e-11\\
-16.81419921875	1.8012133224753e-10\\
-16.793701171875	1.13247844381186e-11\\
-16.773203125	7.08069155736232e-11\\
-16.752705078125	-8.87564436431616e-11\\
-16.73220703125	5.80995164339363e-11\\
-16.711708984375	-1.03799627125621e-10\\
-16.6912109375	8.7336783478503e-11\\
-16.670712890625	5.4701430094425e-11\\
-16.65021484375	1.97950193285162e-10\\
-16.629716796875	-6.15119368922555e-12\\
-16.60921875	5.97914096992572e-12\\
-16.588720703125	-8.40284493335758e-11\\
-16.56822265625	-9.67819009960329e-11\\
-16.547724609375	-1.2191543882873e-10\\
-16.5272265625	-6.36203285083026e-11\\
-16.506728515625	2.16879138737656e-11\\
-16.48623046875	6.24054704816145e-11\\
-16.465732421875	7.86463733036229e-11\\
-16.445234375	2.29616626180711e-10\\
-16.424736328125	1.47764061509949e-10\\
-16.40423828125	1.47097176636292e-10\\
-16.383740234375	7.50731585238211e-11\\
-16.3632421875	-9.68046620287378e-11\\
-16.342744140625	-1.48312596905251e-10\\
-16.32224609375	-2.49885529348674e-10\\
-16.301748046875	-2.58107110724502e-10\\
-16.28125	-3.37844858754234e-10\\
-16.260751953125	-1.82794748106494e-10\\
-16.24025390625	-1.359678273824e-10\\
-16.219755859375	-1.19197886554537e-10\\
-16.1992578125	1.06308200266305e-10\\
-16.178759765625	-6.63431444266498e-11\\
-16.15826171875	1.61468437990808e-10\\
-16.137763671875	-1.6652934736593e-10\\
-16.117265625	-8.62141367305741e-12\\
-16.096767578125	-1.54690380014298e-10\\
-16.07626953125	-1.19482141048705e-10\\
-16.055771484375	-2.89030268581768e-10\\
-16.0352734375	-1.61704267269441e-10\\
-16.014775390625	-1.1468830276172e-10\\
-15.99427734375	3.04889891468247e-11\\
-15.973779296875	-9.81143439549222e-11\\
-15.95328125	1.2130508020859e-10\\
-15.932783203125	2.11269571542038e-12\\
-15.91228515625	-6.34749485260219e-11\\
-15.891787109375	-1.64574249639964e-10\\
-15.8712890625	-2.74083583702414e-10\\
-15.850791015625	-3.16910930514231e-10\\
-15.83029296875	-3.82510783502444e-10\\
-15.809794921875	-3.26087961099828e-10\\
-15.789296875	-3.04841863483253e-10\\
-15.768798828125	-2.89616185296031e-10\\
-15.74830078125	-1.85198679877158e-10\\
-15.727802734375	-4.51348929247416e-11\\
-15.7073046875	-1.09830497664289e-10\\
-15.686806640625	1.06138036801671e-11\\
-15.66630859375	-1.50245883795484e-10\\
-15.645810546875	-5.35372395906459e-11\\
-15.6253125	-2.06267444370417e-10\\
-15.604814453125	-2.14821978275596e-10\\
-15.58431640625	-1.9555963118339e-10\\
-15.563818359375	-2.60001466070002e-10\\
-15.5433203125	-8.03897340714489e-11\\
-15.522822265625	-9.7860069796105e-11\\
-15.50232421875	7.18769064424708e-11\\
-15.481826171875	1.83483603077833e-10\\
-15.461328125	1.22997064001745e-10\\
-15.440830078125	2.8794632992825e-10\\
-15.42033203125	-4.98772334415226e-11\\
-15.399833984375	9.3772230237427e-11\\
-15.3793359375	8.38008548549892e-12\\
-15.358837890625	-3.36938540406795e-12\\
-15.33833984375	-3.17689075083169e-11\\
-15.317841796875	1.44823082947062e-10\\
-15.29734375	5.28541305698829e-12\\
-15.276845703125	4.25639977361009e-10\\
-15.25634765625	2.43115604019945e-10\\
-15.235849609375	3.95350782377115e-10\\
-15.2153515625	2.84802342583976e-10\\
-15.194853515625	1.90338567617542e-10\\
-15.17435546875	1.4442612125575e-10\\
-15.153857421875	1.30399400838855e-10\\
-15.133359375	-1.06258947564266e-10\\
-15.112861328125	7.03563436324994e-11\\
-15.09236328125	-9.17832448167472e-11\\
-15.071865234375	7.17237320823661e-11\\
-15.0513671875	8.42402917524766e-11\\
-15.030869140625	1.06851693554461e-10\\
-15.01037109375	2.57996750934297e-10\\
-14.989873046875	1.04390592039009e-10\\
-14.969375	1.96971518635594e-10\\
-14.948876953125	1.15091845233419e-10\\
-14.92837890625	8.7801213999375e-11\\
-14.907880859375	5.07011318059707e-11\\
-14.8873828125	8.45005489096614e-11\\
-14.866884765625	-3.66022812028269e-11\\
-14.84638671875	2.42731277838262e-10\\
-14.825888671875	-2.4292131513509e-10\\
-14.805390625	2.70991987265763e-10\\
-14.784892578125	3.13572263079618e-11\\
-14.76439453125	2.92861311665146e-10\\
-14.743896484375	8.21646508793061e-11\\
-14.7233984375	2.55285989776894e-10\\
-14.702900390625	6.24058121755039e-11\\
-14.68240234375	9.42906834454754e-11\\
-14.661904296875	1.01752443751093e-10\\
-14.64140625	5.31823529187901e-11\\
-14.620908203125	-1.09440023753915e-10\\
-14.60041015625	6.81614755959235e-12\\
-14.579912109375	-7.13266913482088e-11\\
-14.5594140625	8.45727208739464e-11\\
-14.538916015625	-1.76219790026245e-11\\
-14.51841796875	9.54092569512309e-11\\
-14.497919921875	1.53233748677204e-10\\
-14.477421875	2.55498963066794e-10\\
-14.456923828125	1.81265350664723e-10\\
-14.43642578125	4.18489057245982e-10\\
-14.415927734375	3.03998263618409e-10\\
-14.3954296875	2.18715402972475e-10\\
-14.374931640625	3.30568616605731e-10\\
-14.35443359375	2.00741841482495e-10\\
-14.333935546875	3.68021622039631e-10\\
-14.3134375	2.43627099475029e-10\\
-14.292939453125	4.67975193135769e-10\\
-14.27244140625	5.04431990259051e-10\\
-14.251943359375	6.61230140168964e-10\\
-14.2314453125	5.15495763206153e-10\\
-14.210947265625	7.02860652856827e-10\\
-14.19044921875	5.07830066050008e-10\\
-14.169951171875	4.60988572692618e-10\\
-14.149453125	4.4830545636057e-10\\
-14.128955078125	5.62027405020231e-10\\
-14.10845703125	5.93218778465468e-10\\
-14.087958984375	6.41050470110124e-10\\
-14.0674609375	6.15523287309213e-10\\
-14.046962890625	7.73798408552128e-10\\
-14.02646484375	6.22364407259101e-10\\
-14.005966796875	6.66197847475358e-10\\
-13.98546875	5.54093918080119e-10\\
-13.964970703125	4.37142437034519e-10\\
-13.94447265625	3.22824787536884e-10\\
-13.923974609375	4.89104239564983e-10\\
-13.9034765625	4.64951956507172e-10\\
-13.882978515625	5.20543127982029e-10\\
-13.86248046875	5.35785740992794e-10\\
-13.841982421875	6.43729219013809e-10\\
-13.821484375	6.53860496869405e-10\\
-13.800986328125	6.6498801624706e-10\\
-13.78048828125	6.82941313374977e-10\\
-13.759990234375	6.22739896650391e-10\\
-13.7394921875	4.31479902997638e-10\\
-13.718994140625	4.81545028952045e-10\\
-13.69849609375	4.20300428789258e-10\\
-13.677998046875	6.11032381214322e-10\\
-13.6575	4.81415872472837e-10\\
-13.637001953125	6.45779730115818e-10\\
-13.61650390625	6.66925401762635e-10\\
-13.596005859375	7.70245081775245e-10\\
-13.5755078125	7.15289292109522e-10\\
-13.555009765625	8.67875144006907e-10\\
-13.53451171875	7.28309758736403e-10\\
-13.514013671875	8.04750071023161e-10\\
-13.493515625	7.0822294576134e-10\\
-13.473017578125	9.30266376879687e-10\\
-13.45251953125	7.44598699456481e-10\\
-13.432021484375	1.02238708595704e-09\\
-13.4115234375	8.05824567246562e-10\\
-13.391025390625	1.02706634114355e-09\\
-13.37052734375	1.03187537059361e-09\\
-13.350029296875	9.40555907000349e-10\\
-13.32953125	1.17732847657142e-09\\
-13.309033203125	1.10459279116377e-09\\
-13.28853515625	1.28952190321563e-09\\
-13.268037109375	1.19288277663978e-09\\
-13.2475390625	1.25039240694262e-09\\
-13.227041015625	9.73275647281332e-10\\
-13.20654296875	1.15106770262687e-09\\
-13.186044921875	9.00100302534656e-10\\
-13.165546875	1.05092500790879e-09\\
-13.145048828125	1.00312026106892e-09\\
-13.12455078125	1.03012080358687e-09\\
-13.104052734375	9.03522057446915e-10\\
-13.0835546875	1.06715190826725e-09\\
-13.063056640625	1.05630372676595e-09\\
-13.04255859375	1.04679119841893e-09\\
-13.022060546875	1.09605646291231e-09\\
-13.0015625	9.68990032652473e-10\\
-12.981064453125	6.96814503177592e-10\\
-12.96056640625	7.39049653560259e-10\\
-12.940068359375	5.49056201606935e-10\\
-12.9195703125	6.88429229377888e-10\\
-12.899072265625	4.36375601960608e-10\\
-12.87857421875	6.25052955407313e-10\\
-12.858076171875	5.11108596229046e-10\\
-12.837578125	6.30064476233422e-10\\
-12.817080078125	4.18745970506768e-10\\
-12.79658203125	5.2760410598539e-10\\
-12.776083984375	2.28689079655389e-10\\
-12.7555859375	1.99686300333567e-10\\
-12.735087890625	-5.39516194808691e-12\\
-12.71458984375	-2.0335959102382e-11\\
-12.694091796875	-1.1366039096144e-10\\
-12.67359375	-9.5738920488898e-11\\
-12.653095703125	5.44442313822749e-11\\
-12.63259765625	2.56855099075657e-10\\
-12.612099609375	2.1040200917164e-10\\
-12.5916015625	3.87107786239177e-10\\
-12.571103515625	3.47246242114796e-10\\
-12.55060546875	2.42362495798662e-10\\
-12.530107421875	1.42531303823123e-10\\
-12.509609375	1.04709988246203e-10\\
-12.489111328125	2.85269616916911e-10\\
-12.46861328125	7.93151170984271e-11\\
-12.448115234375	1.99382468290585e-10\\
-12.4276171875	2.66400078539524e-10\\
-12.407119140625	2.38187424180777e-10\\
-12.38662109375	4.64692637444187e-10\\
-12.366123046875	4.92180448507715e-10\\
-12.345625	4.58594108708522e-10\\
-12.325126953125	6.1982395223982e-10\\
-12.30462890625	2.1489470808861e-10\\
-12.284130859375	5.30114861199252e-10\\
-12.2636328125	1.76927011890079e-10\\
-12.243134765625	2.74076690113271e-10\\
-12.22263671875	2.55763397012883e-10\\
-12.202138671875	5.68253417011112e-10\\
-12.181640625	4.6971349602222e-10\\
-12.161142578125	6.59104372033895e-10\\
-12.14064453125	6.2566381661595e-10\\
-12.120146484375	6.47951198080932e-10\\
-12.0996484375	4.60081055633185e-10\\
-12.079150390625	4.23940220322456e-10\\
-12.05865234375	3.71645960390612e-10\\
-12.038154296875	3.34459432866057e-10\\
-12.01765625	2.75028629381425e-10\\
-11.997158203125	2.8946796495186e-10\\
-11.97666015625	2.24163247406074e-10\\
-11.956162109375	4.23961227468538e-10\\
-11.9356640625	2.55929335854664e-10\\
-11.915166015625	3.57105653975263e-10\\
-11.89466796875	2.53859223136934e-10\\
-11.874169921875	5.18554377348625e-11\\
-11.853671875	9.70328788378568e-11\\
-11.833173828125	-6.02361634310501e-11\\
-11.81267578125	-3.84250051987571e-12\\
-11.792177734375	-5.41162495299309e-11\\
-11.7716796875	4.32962995855892e-11\\
-11.751181640625	-5.98755401026837e-11\\
-11.73068359375	-3.25750792231027e-11\\
-11.710185546875	-2.54642851182468e-10\\
-11.6896875	-1.04504117115151e-10\\
-11.669189453125	-2.66492238897123e-10\\
-11.64869140625	-2.69216547413668e-10\\
-11.628193359375	-3.64855265290189e-10\\
-11.6076953125	-4.34847423412214e-10\\
-11.587197265625	-5.45024849143284e-10\\
-11.56669921875	-4.75264015955469e-10\\
-11.546201171875	-6.86702079479991e-10\\
-11.525703125	-4.94856760695555e-10\\
-11.505205078125	-6.7624805187296e-10\\
-11.48470703125	-5.49220385420559e-10\\
-11.464208984375	-5.52448843242609e-10\\
-11.4437109375	-4.0792022548467e-10\\
-11.423212890625	-3.86371711068883e-10\\
-11.40271484375	-4.29158098318848e-10\\
-11.382216796875	-4.41509700244943e-10\\
-11.36171875	-4.65198219105057e-10\\
-11.341220703125	-4.52646101281391e-10\\
-11.32072265625	-4.1109644735833e-10\\
-11.300224609375	-5.67335271635662e-10\\
-11.2797265625	-3.83272456511323e-10\\
-11.259228515625	-5.13794978652337e-10\\
-11.23873046875	-3.91719826078308e-10\\
-11.218232421875	-3.14672582546268e-10\\
-11.197734375	-3.45544203115448e-10\\
-11.177236328125	-4.09218771062912e-10\\
-11.15673828125	-2.6938059016665e-10\\
-11.136240234375	-4.87908519600198e-10\\
-11.1157421875	-5.07646593652678e-10\\
-11.095244140625	-4.66920185809571e-10\\
-11.07474609375	-4.27182485711551e-10\\
-11.054248046875	-4.51898927013824e-10\\
-11.03375	-3.22304547866978e-10\\
-11.013251953125	-3.63837432062471e-10\\
-10.99275390625	-2.57096639873676e-10\\
-10.972255859375	-4.99965600514627e-10\\
-10.9517578125	-3.8650855854401e-10\\
-10.931259765625	-6.04819218162677e-10\\
-10.91076171875	-5.7684092832263e-10\\
-10.890263671875	-8.44120735383971e-10\\
-10.869765625	-7.42041897737203e-10\\
-10.849267578125	-6.84595007063824e-10\\
-10.82876953125	-8.54605977144939e-10\\
-10.808271484375	-7.25546076538979e-10\\
-10.7877734375	-7.99772826814706e-10\\
-10.767275390625	-6.88466857345722e-10\\
-10.74677734375	-8.94064353084852e-10\\
-10.726279296875	-7.32306215146421e-10\\
-10.70578125	-1.08753751664264e-09\\
-10.685283203125	-8.67659472612799e-10\\
-10.66478515625	-1.11771949898361e-09\\
-10.644287109375	-9.0906162613803e-10\\
-10.6237890625	-1.08699313918373e-09\\
-10.603291015625	-8.65677411909356e-10\\
-10.58279296875	-9.84971462860995e-10\\
-10.562294921875	-8.46412476146786e-10\\
-10.541796875	-8.78355549064327e-10\\
-10.521298828125	-7.76711867568271e-10\\
-10.50080078125	-7.32310642239602e-10\\
-10.480302734375	-7.58635609322785e-10\\
-10.4598046875	-8.15591162784838e-10\\
-10.439306640625	-6.30320663140605e-10\\
-10.41880859375	-7.93140934600699e-10\\
-10.398310546875	-5.5312393554451e-10\\
-10.3778125	-6.88536709362916e-10\\
-10.357314453125	-3.91914128656051e-10\\
-10.33681640625	-4.86529050660575e-10\\
-10.316318359375	-3.33530371744083e-10\\
-10.2958203125	-3.39785945452984e-10\\
-10.275322265625	-1.58985177778639e-10\\
-10.25482421875	-3.18564852604073e-10\\
-10.234326171875	-1.39068610870038e-10\\
-10.213828125	-1.91451867528988e-10\\
-10.193330078125	-1.36822176870253e-10\\
-10.17283203125	-1.56033265645045e-10\\
-10.152333984375	-7.23171972108433e-11\\
-10.1318359375	-2.37557237001565e-11\\
-10.111337890625	7.74521529262106e-11\\
-10.09083984375	2.17897939186593e-10\\
-10.070341796875	2.79314038650657e-10\\
-10.04984375	2.80330996455691e-10\\
-10.029345703125	2.22349789700561e-10\\
-10.00884765625	1.91586728661935e-10\\
-9.988349609375	1.02341174455618e-10\\
-9.9678515625	1.61329610959759e-10\\
-9.947353515625	-4.65090089510319e-11\\
-9.92685546875	7.52957172585548e-12\\
-9.906357421875	-1.07807068582007e-11\\
-9.885859375	9.96940543551757e-11\\
-9.865361328125	2.59818590500426e-11\\
-9.84486328125	6.72572759205931e-11\\
-9.824365234375	-1.02090387929436e-10\\
-9.8038671875	8.33211054403545e-11\\
-9.783369140625	-2.31502433206019e-10\\
-9.76287109375	-3.97068384494519e-11\\
-9.742373046875	-2.28353494171397e-10\\
-9.721875	-1.91918082561459e-10\\
-9.701376953125	-2.85579083159768e-10\\
-9.68087890625	-2.48846260341317e-10\\
-9.660380859375	-5.35038171537669e-10\\
-9.6398828125	-3.28821810029095e-10\\
-9.619384765625	-4.57423109962618e-10\\
-9.59888671874999	-4.11818243945463e-10\\
-9.578388671875	-3.55841407114723e-10\\
-9.557890625	-2.43995411586441e-10\\
-9.537392578125	-2.40038853861329e-10\\
-9.51689453125	-1.28479526401187e-10\\
-9.496396484375	-1.61296761058371e-10\\
-9.4758984375	-1.29854267055858e-11\\
-9.455400390625	-6.27634084847859e-11\\
-9.43490234375	8.55505716144461e-11\\
-9.414404296875	7.63811225715e-11\\
-9.39390625	1.47544621778943e-10\\
-9.373408203125	1.28974248501922e-10\\
-9.35291015625	1.56478349889255e-10\\
-9.332412109375	2.28248079200116e-10\\
-9.3119140625	2.62984580497849e-10\\
-9.291416015625	4.05766787918155e-10\\
-9.27091796875	2.72296890953593e-10\\
-9.250419921875	5.43164890418465e-10\\
-9.229921875	4.58312080598786e-10\\
-9.209423828125	6.02004436477737e-10\\
-9.18892578125	4.8736004222081e-10\\
-9.168427734375	7.3519227280215e-10\\
-9.1479296875	6.14018478202594e-10\\
-9.127431640625	7.35695759132607e-10\\
-9.10693359375	7.97367210993302e-10\\
-9.08643554687499	8.49941526439105e-10\\
-9.0659375	7.70780451112769e-10\\
-9.045439453125	1.00246218741324e-09\\
-9.02494140625	9.07858701631056e-10\\
-9.004443359375	1.07888024203886e-09\\
-8.9839453125	9.05333204140399e-10\\
-8.963447265625	9.62915754272714e-10\\
-8.94294921875	9.71258475183571e-10\\
-8.922451171875	8.7905545416805e-10\\
-8.901953125	8.00222164316454e-10\\
-8.881455078125	8.39268071268083e-10\\
-8.86095703125	8.04750435145821e-10\\
-8.840458984375	8.01327731542885e-10\\
-8.8199609375	8.01317801180262e-10\\
-8.799462890625	9.28389222558504e-10\\
-8.77896484375	7.65036349113096e-10\\
-8.758466796875	8.73155534065985e-10\\
-8.73796875	8.73626579397157e-10\\
-8.717470703125	8.07313385323772e-10\\
-8.69697265625	7.984679497852e-10\\
-8.676474609375	6.90852269696108e-10\\
-8.6559765625	5.85960185785261e-10\\
-8.635478515625	7.32856703678014e-10\\
-8.61498046875	5.18100615573356e-10\\
-8.594482421875	6.32503490420374e-10\\
-8.57398437499999	4.4499934103263e-10\\
-8.553486328125	5.23314548582792e-10\\
-8.53298828125	3.86803784447075e-10\\
-8.512490234375	7.14304770462647e-10\\
-8.4919921875	5.53707599966885e-10\\
-8.471494140625	8.27473757876535e-10\\
-8.45099609375	6.54653439525593e-10\\
-8.430498046875	8.62310199632642e-10\\
-8.41	7.48390911949779e-10\\
-8.389501953125	8.4367027190488e-10\\
-8.36900390625	7.91339848911266e-10\\
-8.348505859375	8.25451585605575e-10\\
-8.3280078125	8.04396481863844e-10\\
-8.307509765625	7.54248094811668e-10\\
-8.28701171875	9.77129039530997e-10\\
-8.266513671875	8.31267418367988e-10\\
-8.246015625	1.01082153828963e-09\\
-8.225517578125	1.04398490911047e-09\\
-8.20501953125	1.16240356947433e-09\\
-8.184521484375	1.02820662496859e-09\\
-8.1640234375	1.25582246915372e-09\\
-8.143525390625	1.09165437612866e-09\\
-8.12302734375	1.12145413519963e-09\\
-8.102529296875	9.74650556601287e-10\\
-8.08203125	1.02642662083595e-09\\
-8.06153320312499	8.70354569572979e-10\\
-8.04103515625	9.79121518753457e-10\\
-8.020537109375	9.19810036011483e-10\\
-8.0000390625	9.49775332556563e-10\\
-7.979541015625	8.43030088547839e-10\\
-7.95904296875	8.67881192419683e-10\\
-7.938544921875	7.77197105074764e-10\\
-7.918046875	8.8182986837865e-10\\
-7.897548828125	6.13122914264146e-10\\
-7.87705078125	7.96845738968603e-10\\
-7.856552734375	5.45792319036893e-10\\
-7.8360546875	6.50670941319376e-10\\
-7.815556640625	4.90705889905715e-10\\
-7.79505859375	5.36804970237031e-10\\
-7.774560546875	2.8842868101223e-10\\
-7.7540625	3.6798217191162e-10\\
-7.733564453125	9.44846573833556e-11\\
-7.71306640625	7.8936168187752e-11\\
-7.692568359375	-7.37744376694435e-11\\
-7.6720703125	-7.55974731841149e-11\\
-7.651572265625	-2.89249284656874e-10\\
-7.63107421875	-1.9365083552534e-10\\
-7.610576171875	-2.86485720098316e-10\\
-7.590078125	-3.45593560198815e-10\\
-7.569580078125	-3.33447533877999e-10\\
-7.54908203125	-3.16602887408609e-10\\
-7.528583984375	-3.81575478222558e-10\\
-7.5080859375	-4.90512577111926e-10\\
-7.487587890625	-3.12912445908629e-10\\
-7.46708984375	-3.61069889768422e-10\\
-7.446591796875	-2.16416883763984e-10\\
-7.42609375	-3.38870388510258e-10\\
-7.405595703125	-1.33333604937324e-10\\
-7.38509765625	-2.58651624754622e-10\\
-7.364599609375	-8.02167361441271e-11\\
-7.3441015625	-1.40711020666976e-10\\
-7.323603515625	-7.80043104964093e-13\\
-7.30310546875	-1.58704534331598e-10\\
-7.282607421875	5.82834571408853e-11\\
-7.262109375	-7.36934215720313e-11\\
-7.241611328125	1.20396640038555e-10\\
-7.22111328125	-9.28031122237544e-11\\
-7.200615234375	1.52914838923133e-10\\
-7.1801171875	1.1939675773742e-10\\
-7.159619140625	4.02559340042399e-10\\
-7.13912109375	3.16546359174488e-10\\
-7.118623046875	4.83977760318237e-10\\
-7.098125	4.08431903797268e-10\\
-7.077626953125	2.58410195229469e-10\\
-7.05712890625	2.92894636552069e-10\\
-7.036630859375	1.68949635468452e-10\\
-7.0161328125	8.65248642362789e-11\\
-6.995634765625	1.32800197665651e-10\\
-6.97513671875	-3.01028901029293e-11\\
-6.954638671875	6.47293703788239e-11\\
-6.934140625	-3.46355423584991e-11\\
-6.913642578125	3.61653263447785e-11\\
-6.89314453125	-1.13465576432954e-10\\
-6.872646484375	-4.64493453779212e-11\\
-6.8521484375	-1.68028295385345e-10\\
-6.831650390625	-2.55705807943732e-10\\
-6.81115234375	-3.35194405034549e-10\\
-6.790654296875	-5.64300865849918e-10\\
-6.77015625	-4.87502642927795e-10\\
-6.749658203125	-7.62012879417194e-10\\
-6.72916015625	-5.62915236252066e-10\\
-6.708662109375	-8.29763497516241e-10\\
-6.6881640625	-7.47094054535625e-10\\
-6.667666015625	-9.59553375761097e-10\\
-6.64716796875	-8.28308318215791e-10\\
-6.626669921875	-9.62876974059933e-10\\
-6.606171875	-1.03697559287658e-09\\
-6.585673828125	-1.01866573160189e-09\\
-6.56517578125	-1.07452284621559e-09\\
-6.544677734375	-1.14267563674454e-09\\
-6.5241796875	-9.52793133142204e-10\\
-6.503681640625	-9.96280194648043e-10\\
-6.48318359375	-8.87743971959575e-10\\
-6.462685546875	-9.95742783193313e-10\\
-6.4421875	-1.03422926509207e-09\\
-6.421689453125	-8.9510349171321e-10\\
-6.40119140625	-8.5065927139132e-10\\
-6.380693359375	-9.58028023071129e-10\\
-6.3601953125	-8.79463876241131e-10\\
-6.339697265625	-9.60744920168854e-10\\
-6.31919921875	-8.06359631174275e-10\\
-6.298701171875	-8.3372262135806e-10\\
-6.278203125	-5.76010183143074e-10\\
-6.257705078125	-7.03836872259382e-10\\
-6.23720703125	-5.60678641626345e-10\\
-6.216708984375	-5.05674726904924e-10\\
-6.1962109375	-4.34951664928956e-10\\
-6.175712890625	-4.66409051783173e-10\\
-6.15521484375	-3.5625172984711e-10\\
-6.134716796875	-5.08296889128616e-10\\
-6.11421875	-3.15273849692107e-10\\
-6.093720703125	-6.01788264238487e-10\\
-6.07322265625	-2.8548855518811e-10\\
-6.052724609375	-4.57433886930738e-10\\
-6.0322265625	-3.08379378401967e-10\\
-6.011728515625	-4.58285577828416e-10\\
-5.99123046875	-2.3407814262699e-10\\
-5.970732421875	-4.04738041901235e-10\\
-5.950234375	-3.0355500561368e-10\\
-5.929736328125	-5.61556339976678e-10\\
-5.90923828125	-3.74805814120983e-10\\
-5.888740234375	-4.79386903848477e-10\\
-5.8682421875	-5.43273707655453e-10\\
-5.847744140625	-5.67316492455445e-10\\
-5.82724609375	-5.44169618990842e-10\\
-5.806748046875	-6.13701541953216e-10\\
-5.78625	-6.26114951826238e-10\\
-5.765751953125	-5.90087619057342e-10\\
-5.74525390625	-6.51268897742752e-10\\
-5.724755859375	-6.65282135472834e-10\\
-5.7042578125	-7.435901745062e-10\\
-5.683759765625	-7.6560837565037e-10\\
-5.66326171875	-9.47773369813823e-10\\
-5.642763671875	-8.02922796966388e-10\\
-5.622265625	-9.47400613957018e-10\\
-5.601767578125	-7.24276128370257e-10\\
-5.58126953125	-7.90502277573128e-10\\
-5.560771484375	-6.27519383137209e-10\\
-5.5402734375	-6.06224519673127e-10\\
-5.519775390625	-5.18105155657613e-10\\
-5.49927734375	-5.1091178982551e-10\\
-5.478779296875	-5.44553752289555e-10\\
-5.45828125	-6.81578233743399e-10\\
-5.437783203125	-6.83912646146095e-10\\
-5.41728515625	-8.44919478708107e-10\\
-5.396787109375	-6.14217143914544e-10\\
-5.3762890625	-7.49791803856879e-10\\
-5.355791015625	-3.84931415897712e-10\\
-5.33529296875	-5.59444899365873e-10\\
-5.314794921875	-3.00418732626404e-10\\
-5.294296875	-3.06577411731791e-10\\
-5.273798828125	-8.21028442807684e-11\\
-5.25330078125	-2.00797369648673e-10\\
-5.232802734375	5.51608646526036e-11\\
-5.2123046875	-7.89279475093295e-11\\
-5.191806640625	1.91536439685071e-10\\
-5.17130859375	7.22461461247724e-11\\
-5.150810546875	2.6847501858841e-10\\
-5.1303125	3.04822015627813e-10\\
-5.109814453125	4.11560484997503e-10\\
-5.08931640625	5.48591262743555e-10\\
-5.068818359375	4.3173397785767e-10\\
-5.0483203125	4.99236442592958e-10\\
-5.027822265625	5.45221039464052e-10\\
-5.00732421875	5.4003018475784e-10\\
-4.986826171875	4.63717756403874e-10\\
-4.966328125	4.25571581241464e-10\\
-4.945830078125	3.80503313550078e-10\\
-4.92533203125	4.99368200186195e-10\\
-4.904833984375	3.06101574780234e-10\\
-4.8843359375	4.46188428156914e-10\\
-4.863837890625	3.36510326548588e-10\\
-4.84333984375	3.72896501824568e-10\\
-4.822841796875	2.08911410437943e-10\\
-4.80234375	2.86004406513046e-10\\
-4.781845703125	1.79088733439901e-10\\
-4.76134765625	2.0770289565661e-10\\
-4.740849609375	1.8136397183906e-11\\
-4.7203515625	2.64070025331699e-10\\
-4.699853515625	-5.13146005260651e-11\\
-4.67935546875	7.48997747157924e-11\\
-4.658857421875	-1.77075484176266e-10\\
-4.638359375	-6.89497550656449e-11\\
-4.617861328125	-2.95595610786983e-10\\
-4.59736328125	-2.1552763516666e-10\\
-4.576865234375	-2.05451310770285e-10\\
-4.5563671875	-2.33714369697295e-10\\
-4.535869140625	-1.64872150156113e-10\\
-4.51537109375	-5.50263378764235e-11\\
-4.494873046875	-2.35166264599557e-11\\
-4.474375	1.85534118656782e-10\\
-4.453876953125	9.6083728036268e-11\\
-4.43337890625	2.30817173640356e-10\\
-4.412880859375	2.90676455460602e-10\\
-4.3923828125	3.24197478885077e-10\\
-4.371884765625	3.1542376205214e-10\\
-4.35138671875	3.89467029142733e-10\\
-4.330888671875	4.06893422894002e-10\\
-4.310390625	3.57228364149754e-10\\
-4.289892578125	5.84345303622357e-10\\
-4.26939453125	4.64325822864031e-10\\
-4.248896484375	6.93063791101784e-10\\
-4.2283984375	6.92487722665849e-10\\
-4.207900390625	9.87221780638242e-10\\
-4.18740234375	8.31870126016867e-10\\
-4.166904296875	1.13532805638022e-09\\
-4.14640625	1.02004033832483e-09\\
-4.125908203125	1.14778625464378e-09\\
-4.10541015625	1.09008751416762e-09\\
-4.084912109375	8.65764417403688e-10\\
-4.0644140625	8.76661217511556e-10\\
-4.043916015625	9.8394828833703e-10\\
-4.02341796875	9.07166165502578e-10\\
-4.002919921875	8.9375593784642e-10\\
-3.982421875	9.37810168632346e-10\\
-3.961923828125	1.10244916219982e-09\\
-3.94142578125	1.11932915721332e-09\\
-3.920927734375	1.20070534842449e-09\\
-3.9004296875	1.11395806410006e-09\\
-3.879931640625	1.06134287025103e-09\\
-3.85943359375	8.86775575903146e-10\\
-3.838935546875	9.65427891159564e-10\\
-3.8184375	7.48545474698534e-10\\
-3.797939453125	8.34096084643774e-10\\
-3.77744140625	8.15154218402539e-10\\
-3.756943359375	1.0016872646818e-09\\
-3.7364453125	8.04744146800655e-10\\
-3.715947265625	9.6313938340832e-10\\
-3.69544921875	9.86072385627043e-10\\
-3.674951171875	9.69480203428302e-10\\
-3.654453125	8.47349196113653e-10\\
-3.633955078125	9.92351742035558e-10\\
-3.61345703125	7.24806787677068e-10\\
-3.592958984375	9.17491811276429e-10\\
-3.5724609375	7.09826877145562e-10\\
-3.551962890625	8.17406980020286e-10\\
-3.53146484375	7.57007675538109e-10\\
-3.510966796875	8.26752328144678e-10\\
-3.49046875	6.51047706026917e-10\\
-3.469970703125	9.51674269890005e-10\\
-3.44947265625	7.80248823016637e-10\\
-3.428974609375	1.0050114661463e-09\\
-3.4084765625	8.98481813767114e-10\\
-3.387978515625	9.42237615967583e-10\\
-3.36748046875	1.00232561102115e-09\\
-3.346982421875	1.05177758843602e-09\\
-3.326484375	9.37681619016524e-10\\
-3.305986328125	1.04445691770007e-09\\
-3.28548828125	9.91866071570608e-10\\
-3.264990234375	1.04202726279606e-09\\
-3.2444921875	1.08338825260358e-09\\
-3.223994140625	1.18246551744041e-09\\
-3.20349609375	1.30317975321904e-09\\
-3.182998046875	1.30050831397149e-09\\
-3.1625	1.51649591203193e-09\\
-3.142001953125	1.36561498322102e-09\\
-3.12150390625	1.4464456682529e-09\\
-3.101005859375	1.22463172090078e-09\\
-3.0805078125	1.40663846522184e-09\\
-3.060009765625	1.17888680409465e-09\\
-3.03951171875	1.33252694355437e-09\\
-3.019013671875	1.14315277534479e-09\\
-2.998515625	1.32016986101373e-09\\
-2.978017578125	1.18729018788279e-09\\
-2.95751953125	1.33999094623038e-09\\
-2.937021484375	1.28404173311419e-09\\
-2.9165234375	1.32153685414537e-09\\
-2.896025390625	1.04673042751207e-09\\
-2.87552734375	1.12390770522634e-09\\
-2.855029296875	8.88778307549767e-10\\
-2.83453125	1.01948566354019e-09\\
-2.814033203125	8.58938585953058e-10\\
-2.79353515625	1.01695156511326e-09\\
-2.773037109375	7.86761592748534e-10\\
-2.7525390625	9.37524086484186e-10\\
-2.732041015625	7.37036623360582e-10\\
-2.71154296875	8.01473387238725e-10\\
-2.691044921875	4.8661307701023e-10\\
-2.670546875	4.93301209827946e-10\\
-2.650048828125	3.16419702398169e-10\\
-2.62955078125	1.90029349334032e-10\\
-2.609052734375	1.65195126616002e-10\\
-2.5885546875	1.8797683952931e-10\\
-2.568056640625	2.27911264441145e-10\\
-2.54755859375	1.23299419394295e-10\\
-2.527060546875	1.78763072883509e-10\\
-2.5065625	5.54055506421351e-11\\
-2.486064453125	1.15941748004172e-10\\
-2.46556640625	7.49739342108789e-11\\
-2.445068359375	1.39417839383347e-10\\
-2.4245703125	-1.031879377496e-11\\
-2.404072265625	8.57939455784022e-11\\
-2.38357421875	3.64887342236344e-11\\
-2.363076171875	2.6179828815502e-10\\
-2.342578125	2.45859873323511e-10\\
-2.322080078125	4.1718412418047e-10\\
-2.30158203125	3.55188913492746e-10\\
-2.281083984375	4.46665125426022e-10\\
-2.2605859375	3.45609567251982e-10\\
-2.240087890625	4.5680584400814e-10\\
-2.21958984375	2.43207370063517e-10\\
-2.199091796875	5.24042873488586e-10\\
-2.17859375	2.49559480408921e-10\\
-2.158095703125	5.35554908552607e-10\\
-2.13759765625	4.18477762439159e-10\\
-2.117099609375	5.59241491836508e-10\\
-2.0966015625	4.72081887402799e-10\\
-2.076103515625	7.43507426497227e-10\\
-2.05560546875	6.99125650279793e-10\\
-2.035107421875	6.53803325058743e-10\\
-2.014609375	4.66436197817881e-10\\
-1.994111328125	3.68514141925315e-10\\
-1.97361328125	3.18738370798498e-10\\
-1.953115234375	2.54710438219728e-10\\
-1.9326171875	2.40908816831456e-10\\
-1.912119140625	2.28856112027945e-10\\
-1.89162109375	1.39108559733311e-10\\
-1.871123046875	1.8757201898121e-10\\
-1.850625	8.1037815741068e-12\\
-1.830126953125	-3.86658725012992e-11\\
-1.80962890625	8.9687712963462e-11\\
-1.789130859375	-2.00205755889345e-10\\
-1.7686328125	-7.11360450540897e-11\\
-1.748134765625	-3.46577041124643e-10\\
-1.72763671875	-2.817970520201e-10\\
-1.707138671875	-3.41501856446093e-10\\
-1.686640625	-3.42105860017513e-10\\
-1.666142578125	-4.42079592762051e-10\\
-1.64564453125	-3.02080016129961e-10\\
-1.625146484375	-4.57198310241357e-10\\
-1.6046484375	-2.36751802415647e-10\\
-1.584150390625	-2.05585450856336e-10\\
-1.56365234375	-2.63374393868697e-10\\
-1.543154296875	-4.30755361908287e-10\\
-1.52265625	-3.78194621838431e-10\\
-1.502158203125	-4.72195885002426e-10\\
-1.48166015625	-5.49683719794182e-10\\
-1.461162109375	-6.8460409221903e-10\\
-1.4406640625	-6.07871657205191e-10\\
-1.420166015625	-8.64484630707967e-10\\
-1.39966796875	-6.79348938404467e-10\\
-1.379169921875	-6.11717455471824e-10\\
-1.358671875	-3.52101224536977e-10\\
-1.338173828125	-2.61973533386822e-10\\
-1.31767578125	-1.36911446931495e-10\\
-1.297177734375	-1.99599612413978e-10\\
-1.2766796875	-1.47095341689314e-10\\
-1.256181640625	-3.03764842927679e-10\\
-1.23568359375	-2.02652480557025e-10\\
-1.215185546875	-5.3280913633961e-10\\
-1.1946875	-4.93506418735025e-10\\
-1.174189453125	-6.93017720506208e-10\\
-1.15369140625	-6.08785543094064e-10\\
-1.133193359375	-7.1436302012525e-10\\
-1.1126953125	-4.8608378513689e-10\\
-1.092197265625	-4.77747513326333e-10\\
-1.07169921875	-9.82901042482708e-11\\
-1.051201171875	-3.06685099590617e-10\\
-1.030703125	1.08962873783246e-12\\
-1.010205078125	-9.10462832187623e-11\\
-0.989707031249999	-1.25776383181246e-10\\
-0.969208984375001	-3.55073347009764e-10\\
-0.948710937499996	-2.35510086352566e-10\\
-0.928212890624998	-6.93713731782026e-10\\
-0.90771484375	-5.29719156271039e-10\\
-0.887216796875002	-7.94512544955936e-10\\
-0.866718749999997	-6.04852668047306e-10\\
-0.846220703124999	-6.46407676007171e-10\\
-0.825722656250001	-4.53843926009696e-10\\
-0.805224609374996	-4.47212432255496e-10\\
-0.784726562499998	-2.42028705586099e-10\\
-0.764228515625	-2.37605032652367e-10\\
-0.743730468750002	-2.076572229356e-10\\
-0.723232421874997	-2.70077444855668e-10\\
-0.702734374999999	-5.21437430597273e-10\\
-0.682236328125001	-6.29596848497733e-10\\
-0.661738281249995	-8.92590150928165e-10\\
-0.641240234374997	-8.62382201400737e-10\\
-0.620742187499999	-9.34911741260589e-10\\
-0.600244140625001	-8.96582856029409e-10\\
-0.579746093749996	-8.19979405327283e-10\\
-0.559248046874998	-6.200664650214e-10\\
-0.53875	-5.35434996533569e-10\\
-0.518251953125002	-2.77576294833992e-10\\
-0.497753906249997	-4.45279312585137e-10\\
-0.477255859374999	-1.87413700333495e-10\\
-0.456757812500001	-3.73807952659298e-10\\
-0.436259765624996	-3.22361100708327e-10\\
-0.415761718749998	-3.17054093345858e-10\\
-0.395263671875	-2.18611000916642e-10\\
-0.374765625000002	-2.90562337173867e-10\\
-0.354267578124997	-5.47477432202269e-12\\
-0.333769531249999	-6.41289288940673e-11\\
-0.313271484375001	1.12849827506011e-10\\
-0.292773437499996	1.44379989361341e-10\\
-0.272275390624998	3.64757891906201e-10\\
-0.25177734375	2.71161443839943e-10\\
-0.231279296875002	4.98621344574356e-10\\
-0.210781249999997	4.98431780053783e-10\\
-0.190283203124999	7.87131357505675e-10\\
-0.169785156250001	6.92243789208462e-10\\
-0.149287109374995	9.04353838880999e-10\\
-0.128789062499997	9.80652025773304e-10\\
-0.108291015624999	9.71786447610259e-10\\
-0.0877929687500014	1.06153025599499e-09\\
-0.0672949218749963	1.05314768099806e-09\\
-0.0467968749999983	1.16820901523835e-09\\
-0.0262988281250003	1.14803473517527e-09\\
-0.00580078125000227	1.22722663170255e-09\\
0.0146972656250028	1.33719038078653e-09\\
0.0351953125000009	1.3966177244681e-09\\
0.0556933593749989	1.18342428868619e-09\\
0.076191406250004	1.33712243332107e-09\\
0.096689453125002	1.37930272360571e-09\\
0.1171875	1.53816336228712e-09\\
0.137685546874998	1.42208666526754e-09\\
0.158183593750003	1.68551056056152e-09\\
0.178681640625001	1.73698752896279e-09\\
0.199179687499999	1.94758187883086e-09\\
0.219677734375004	1.86976297010165e-09\\
0.240175781250002	2.17558832854747e-09\\
0.260673828125	1.99591197775805e-09\\
0.281171874999998	2.09436854794601e-09\\
0.301669921875003	1.88337172294814e-09\\
0.322167968750001	2.07917434429279e-09\\
0.342666015624999	1.8560617377878e-09\\
0.363164062499997	1.99588652033232e-09\\
0.383662109375003	1.89271828573449e-09\\
0.404160156250001	2.03966766299496e-09\\
0.424658203124999	1.96001440754492e-09\\
0.445156250000004	2.25942363716886e-09\\
0.465654296875002	2.24943525508698e-09\\
0.48615234375	2.42453225578649e-09\\
0.506650390624998	2.4892927358175e-09\\
0.527148437500003	2.54684221006687e-09\\
0.547646484375001	2.65162538151762e-09\\
0.568144531249999	2.74414364867003e-09\\
0.588642578125004	2.79578756265129e-09\\
0.609140625000002	2.93674644729242e-09\\
0.629638671875	3.11819412247697e-09\\
0.650136718749998	3.24595224954067e-09\\
0.670634765625003	3.38729638927049e-09\\
0.691132812500001	3.53128748379507e-09\\
0.711630859374999	3.68238986190188e-09\\
0.732128906250004	3.80754193251861e-09\\
0.752626953125002	4.01020670564337e-09\\
0.773125	3.8296927560272e-09\\
0.793623046874998	4.09447869906455e-09\\
0.814121093750003	4.1524970385662e-09\\
0.834619140625001	4.1940087080238e-09\\
0.855117187499999	4.11455228615808e-09\\
0.875615234374997	4.50552881014117e-09\\
0.896113281250003	4.29527614374813e-09\\
0.916611328125001	4.53363539428854e-09\\
0.937109374999999	4.53020860456487e-09\\
0.957607421875004	4.61923539588084e-09\\
0.978105468750002	4.62665443845735e-09\\
0.998603515625	4.7613304204717e-09\\
1.0191015625	4.68856497391595e-09\\
1.039599609375	4.87362438837712e-09\\
1.06009765625	4.70827461309004e-09\\
1.080595703125	5.08732608925599e-09\\
1.10109375	5.05457458939742e-09\\
1.121591796875	5.33475158804985e-09\\
1.14208984375	5.32092067205555e-09\\
1.162587890625	5.26892953647476e-09\\
1.1830859375	5.43443839478097e-09\\
1.203583984375	5.27988832930853e-09\\
1.22408203125	5.3257577144296e-09\\
1.244580078125	5.45280865010593e-09\\
1.265078125	5.40937184497147e-09\\
1.285576171875	5.65788761975894e-09\\
1.30607421875	5.75609539810344e-09\\
1.326572265625	5.99709044034859e-09\\
1.3470703125	6.04559056392444e-09\\
1.367568359375	6.33285686232466e-09\\
1.38806640625	6.35769555752182e-09\\
1.408564453125	6.3653125092676e-09\\
1.4290625	6.2219179694431e-09\\
1.449560546875	6.42646588687519e-09\\
1.47005859375	5.97894834296338e-09\\
1.490556640625	6.23832964895638e-09\\
1.5110546875	6.2487749040555e-09\\
1.531552734375	6.46917649594991e-09\\
1.55205078125	6.53701111503975e-09\\
1.572548828125	6.97771129923239e-09\\
1.593046875	6.96302190975036e-09\\
1.613544921875	7.34972824865083e-09\\
1.63404296875	7.13951599278312e-09\\
1.654541015625	7.38602365076758e-09\\
1.6750390625	7.09794143936191e-09\\
1.695537109375	7.13866422403821e-09\\
1.71603515625	7.12868919058419e-09\\
1.736533203125	7.10883166929835e-09\\
1.75703125	7.30699265310153e-09\\
1.777529296875	7.48627638155341e-09\\
1.79802734375	7.74325068753366e-09\\
1.818525390625	8.06655751936096e-09\\
1.8390234375	8.34919298669442e-09\\
1.859521484375	8.59225922919574e-09\\
1.88001953125	8.73399952451866e-09\\
1.900517578125	8.77136144471868e-09\\
1.921015625	8.74995514653399e-09\\
1.941513671875	8.60257664001419e-09\\
1.96201171875	8.62637494807048e-09\\
1.982509765625	8.61603529470752e-09\\
2.0030078125	8.69923720590893e-09\\
2.023505859375	8.70259703400786e-09\\
2.04400390625	9.12529067297501e-09\\
2.064501953125	9.04321796860391e-09\\
2.085	9.28390490978786e-09\\
2.105498046875	9.23657151367677e-09\\
2.12599609375	9.27017733366119e-09\\
2.146494140625	9.07341558922037e-09\\
2.1669921875	9.10906190499613e-09\\
2.187490234375	8.85417142666965e-09\\
2.20798828125	8.97210045327035e-09\\
2.228486328125	8.90604663152304e-09\\
2.248984375	8.96592175687034e-09\\
2.269482421875	8.8217989396866e-09\\
2.28998046875	8.9386341114063e-09\\
2.310478515625	8.71474830037479e-09\\
2.3309765625	8.7986299035719e-09\\
2.351474609375	8.54904202961748e-09\\
2.37197265625	8.55369215734158e-09\\
2.392470703125	8.43597108410854e-09\\
2.41296875	8.46875249373771e-09\\
2.433466796875	8.39845116111748e-09\\
2.45396484375	8.46850919474696e-09\\
2.474462890625	8.34257298028888e-09\\
2.4949609375	8.47284768318775e-09\\
2.515458984375	8.38828836535971e-09\\
2.53595703125	8.44109710177638e-09\\
2.556455078125	8.47189688867044e-09\\
2.576953125	8.52031804585033e-09\\
2.597451171875	8.55913811096678e-09\\
2.61794921875	8.47372662391232e-09\\
2.638447265625	8.35433514468389e-09\\
2.6589453125	8.26940456180873e-09\\
2.679443359375	8.08396353170507e-09\\
2.69994140625	8.02698881013463e-09\\
2.720439453125	8.06292868337253e-09\\
2.7409375	7.97163781435869e-09\\
2.761435546875	8.13097143155218e-09\\
2.78193359375	8.1213230153366e-09\\
2.802431640625	8.34271277696955e-09\\
2.8229296875	8.26408578891268e-09\\
2.843427734375	8.34344124786416e-09\\
2.86392578125	8.33436497613599e-09\\
2.884423828125	8.58721797236361e-09\\
2.904921875	8.39798374006984e-09\\
2.925419921875	8.5595816444983e-09\\
2.94591796875	8.41927108074935e-09\\
2.966416015625	8.50046483827934e-09\\
2.9869140625	8.40957925303245e-09\\
3.007412109375	8.36763924257873e-09\\
3.02791015625	8.26417982853808e-09\\
3.048408203125	8.07887432748514e-09\\
3.06890625	8.06842283491618e-09\\
3.089404296875	7.90627471460151e-09\\
3.10990234375	7.78444665886819e-09\\
3.130400390625	7.65861230349911e-09\\
3.1508984375	7.41798259218486e-09\\
3.171396484375	7.62405640578575e-09\\
3.19189453125	7.39417173054285e-09\\
3.212392578125	7.44408376435693e-09\\
3.232890625	7.44006461678262e-09\\
3.253388671875	7.41030191384428e-09\\
3.27388671875	7.23274000092562e-09\\
3.294384765625	7.04090243655242e-09\\
3.3148828125	6.92935094113132e-09\\
3.335380859375	6.53779365453731e-09\\
3.35587890625	6.57737099323851e-09\\
3.376376953125	6.17323124968107e-09\\
3.396875	6.35579269839557e-09\\
3.417373046875	6.26861553411176e-09\\
3.43787109375	6.436008765307e-09\\
3.458369140625	6.38342488634272e-09\\
3.4788671875	6.37336581350718e-09\\
3.499365234375	6.33817414156578e-09\\
3.51986328125	6.17610878443545e-09\\
3.540361328125	6.11147172129584e-09\\
3.560859375	5.67901475396757e-09\\
3.581357421875	5.41236887745167e-09\\
3.60185546875	5.4181036430771e-09\\
3.622353515625	5.21641089145293e-09\\
3.6428515625	5.34545169050897e-09\\
3.663349609375	5.49204723445676e-09\\
3.68384765625	5.58991678042281e-09\\
3.704345703125	5.680106941793e-09\\
3.72484375	5.61065888311661e-09\\
3.745341796875	5.4113410605348e-09\\
3.76583984375	5.22669154596205e-09\\
3.786337890625	4.6047495991262e-09\\
3.8068359375	4.70987128226347e-09\\
3.827333984375	4.14102696812191e-09\\
3.84783203125	4.12993002562388e-09\\
3.868330078125	4.02983066083446e-09\\
3.888828125	3.96843026419929e-09\\
3.909326171875	4.0403485530143e-09\\
3.92982421875	4.33500335030249e-09\\
3.950322265625	4.293242396198e-09\\
3.9708203125	4.52839331264002e-09\\
3.991318359375	4.13215250872455e-09\\
4.01181640625	3.9836046369573e-09\\
4.032314453125	3.73845720916033e-09\\
4.0528125	3.40380075978475e-09\\
4.073310546875	3.32748999706286e-09\\
4.09380859375	3.16154764904133e-09\\
4.114306640625	3.05585749727038e-09\\
4.1348046875	3.25732685220241e-09\\
4.155302734375	3.21489657279625e-09\\
4.17580078125	3.58017415292158e-09\\
4.196298828125	3.47622178977554e-09\\
4.216796875	3.29461391459089e-09\\
4.237294921875	3.27635199898006e-09\\
4.25779296875	2.79452375251668e-09\\
4.278291015625	2.62693839183826e-09\\
4.2987890625	2.3240596375784e-09\\
4.319287109375	2.31202217049854e-09\\
4.33978515625	2.16857448576178e-09\\
4.360283203125	2.24403486114774e-09\\
4.38078125	2.37573940328071e-09\\
4.401279296875	2.45244142889788e-09\\
4.42177734375	2.36618487358155e-09\\
4.442275390625	2.20730735961876e-09\\
4.4627734375	1.94307009486558e-09\\
4.483271484375	1.83901020831133e-09\\
4.50376953125	1.36809004985015e-09\\
4.524267578125	1.38061537647319e-09\\
4.544765625	1.14081282499008e-09\\
4.565263671875	1.19757544642266e-09\\
4.58576171875	1.13294484357073e-09\\
4.606259765625	1.42596137751531e-09\\
4.6267578125	1.45058722147117e-09\\
4.647255859375	1.65024270499556e-09\\
4.66775390625	1.48621292283868e-09\\
4.688251953125	1.6177901260023e-09\\
4.70875	1.24738735192705e-09\\
4.729248046875	1.08684677200851e-09\\
4.74974609375	8.89195771133911e-10\\
4.770244140625	7.80608820770718e-10\\
4.7907421875	6.80577334609079e-10\\
4.811240234375	8.10359744074384e-10\\
4.83173828125	7.83298278323637e-10\\
4.852236328125	1.2071573453687e-09\\
4.872734375	1.12890149559986e-09\\
4.893232421875	1.49225144786763e-09\\
4.91373046875	1.32748077128098e-09\\
4.934228515625	1.50258145755399e-09\\
4.9547265625	1.1908093016396e-09\\
4.975224609375	1.08254063848763e-09\\
4.99572265625	8.89403584635476e-10\\
5.016220703125	9.17041884927047e-10\\
5.03671875	7.45052185868798e-10\\
5.057216796875	7.4675373149074e-10\\
5.07771484375	9.09514924672255e-10\\
5.098212890625	8.46315514407641e-10\\
5.1187109375	1.01576768374185e-09\\
5.139208984375	9.55112832938894e-10\\
5.15970703125	8.98596478613913e-10\\
5.180205078125	6.20572215426376e-10\\
5.200703125	4.47907591689857e-10\\
5.221201171875	1.40981312766134e-10\\
5.24169921875	6.03161961436029e-12\\
5.262197265625	-1.49916743689176e-10\\
5.2826953125	-1.63644008768049e-10\\
5.303193359375	-2.15832954264673e-10\\
5.32369140625	3.04277573804665e-11\\
5.344189453125	-9.25715076685974e-11\\
5.3646875	1.28689214031898e-10\\
5.385185546875	-3.82614634426035e-11\\
5.40568359375	-9.05470198226552e-11\\
5.426181640625	-3.94145315736702e-10\\
5.4466796875	-4.85380878718467e-10\\
5.467177734375	-8.53306026708216e-10\\
5.48767578125	-8.26064547200432e-10\\
5.508173828125	-8.85096018502397e-10\\
5.528671875	-8.10089734707505e-10\\
5.549169921875	-6.82107070375858e-10\\
5.56966796875	-5.75504287327418e-10\\
5.590166015625	-6.35366464805158e-10\\
5.6106640625	-4.59578143903484e-10\\
5.631162109375	-6.84283494022021e-10\\
5.65166015625	-3.84210421730314e-10\\
5.672158203125	-6.90031052483324e-10\\
5.69265625	-7.89956535886252e-10\\
5.713154296875	-8.85029861619654e-10\\
5.73365234375	-9.23875422070292e-10\\
5.754150390625	-9.7451777882251e-10\\
5.7746484375	-7.94932270383048e-10\\
5.795146484375	-7.22625438105233e-10\\
5.81564453125	-8.57819423102325e-10\\
5.836142578125	-6.80449502455731e-10\\
5.856640625	-8.52537974470326e-10\\
5.877138671875	-7.12604404692839e-10\\
5.89763671875	-8.82171542960335e-10\\
5.918134765625	-7.83867586071657e-10\\
5.9386328125	-9.32217800276492e-10\\
5.959130859375	-9.07342158573442e-10\\
5.97962890625	-9.04733471555279e-10\\
6.000126953125	-8.65600727774555e-10\\
6.020625	-7.4544431920695e-10\\
6.041123046875	-8.2185826497741e-10\\
6.06162109375	-7.60279309250926e-10\\
6.082119140625	-6.82729353621283e-10\\
6.1026171875	-8.6607173791418e-10\\
6.123115234375	-8.68521870489855e-10\\
6.14361328125	-8.91134067902034e-10\\
6.164111328125	-1.00144360002691e-09\\
6.184609375	-1.10210747378394e-09\\
6.205107421875	-1.07026087289885e-09\\
6.22560546875	-1.1406705760698e-09\\
6.246103515625	-1.04155068915624e-09\\
6.2666015625	-1.03387854268836e-09\\
6.287099609375	-8.78370661810964e-10\\
6.30759765625	-9.26237191339172e-10\\
6.328095703125	-7.16503110432243e-10\\
6.34859375	-8.09648452540608e-10\\
6.369091796875	-7.10596748096732e-10\\
6.38958984375	-7.86617206143421e-10\\
6.410087890625	-7.86281907190264e-10\\
6.4305859375	-8.59747528505607e-10\\
6.451083984375	-8.90894557357034e-10\\
6.47158203125	-9.68277910933152e-10\\
6.492080078125	-7.8097280136032e-10\\
6.512578125	-9.51659789034839e-10\\
6.533076171875	-8.98650108898572e-10\\
6.55357421875	-7.85504844285749e-10\\
6.574072265625	-7.58357455721696e-10\\
6.5945703125	-6.87354993076747e-10\\
6.615068359375	-6.4218026365511e-10\\
6.63556640625	-6.93984981636774e-10\\
6.656064453125	-5.04314379624613e-10\\
6.6765625	-7.9498701199182e-10\\
6.697060546875	-6.65425215781635e-10\\
6.71755859375	-6.63659682841329e-10\\
6.738056640625	-7.98629637877061e-10\\
6.7585546875	-5.55592696844725e-10\\
6.779052734375	-6.92449719760322e-10\\
6.79955078125	-6.77612559124829e-10\\
6.820048828125	-6.39423625339388e-10\\
6.840546875	-6.29406333574349e-10\\
6.861044921875	-5.53649097544684e-10\\
6.88154296875	-5.49567681198473e-10\\
6.902041015625	-4.85259643946861e-10\\
6.9225390625	-3.75129457841985e-10\\
6.943037109375	-3.79695156454076e-10\\
6.96353515625	-3.60959734000657e-10\\
6.984033203125	-3.9356574751302e-10\\
7.00453125	-3.20544110452639e-10\\
7.025029296875	-3.47515842547371e-10\\
7.04552734375	-3.71205900337296e-10\\
7.066025390625	-4.60527605610922e-10\\
7.0865234375	-3.28798862142439e-10\\
7.107021484375	-4.54523377946592e-10\\
7.12751953125	-3.50690180787128e-10\\
7.148017578125	-4.61568707057604e-10\\
7.168515625	-3.28542479283926e-10\\
7.189013671875	-4.71341412161501e-10\\
7.20951171875	-4.09224354035885e-10\\
7.230009765625	-5.53148001008677e-10\\
7.2505078125	-4.38266215009385e-10\\
7.271005859375	-4.70071109450928e-10\\
7.29150390625	-3.73946626961989e-10\\
7.312001953125	-4.01024834401882e-10\\
7.3325	-3.82584004964973e-10\\
7.352998046875	-5.75995572526883e-10\\
7.37349609375	-4.35941159782376e-10\\
7.393994140625	-7.70290446816139e-10\\
7.4144921875	-6.41596612135542e-10\\
7.434990234375	-8.22945746130532e-10\\
7.45548828125	-6.74168077677621e-10\\
7.475986328125	-8.41593451680934e-10\\
7.496484375	-6.72971410536941e-10\\
7.516982421875	-6.73517200116971e-10\\
7.53748046875	-5.59165026717644e-10\\
7.557978515625	-6.30094056862894e-10\\
7.5784765625	-6.10252720097256e-10\\
7.598974609375	-6.54034422925427e-10\\
7.61947265625	-7.19344937311713e-10\\
7.639970703125	-7.58639459248126e-10\\
7.66046875	-7.27939055701202e-10\\
7.680966796875	-7.31975121641139e-10\\
7.70146484375	-6.8659219056406e-10\\
7.721962890625	-5.77138646171264e-10\\
7.7424609375	-5.8487772824261e-10\\
7.762958984375	-3.90774485760525e-10\\
7.78345703125	-4.20667417094695e-10\\
7.803955078125	-3.55463263775712e-10\\
7.824453125	-4.00585649920655e-10\\
7.844951171875	-1.88292365635665e-10\\
7.86544921875	-4.40329650689212e-10\\
7.885947265625	-1.753296597688e-10\\
7.9064453125	-1.99054254416789e-10\\
7.926943359375	-8.57765806498247e-11\\
7.94744140625	-2.74897424763197e-11\\
7.967939453125	1.76578655776571e-12\\
7.9884375	-2.37313398384604e-11\\
8.008935546875	-4.88240022274196e-11\\
8.02943359375	-8.679735508496e-11\\
8.049931640625	-8.95293747899908e-11\\
8.0704296875	-1.78415463073371e-10\\
8.090927734375	-1.4447152821714e-10\\
8.11142578125	-8.22295910728889e-11\\
8.131923828125	1.05864376128446e-10\\
8.152421875	-1.32096918517577e-10\\
8.172919921875	8.00081855727289e-11\\
8.19341796875	2.50632524272707e-11\\
8.213916015625	1.2477426308477e-10\\
8.2344140625	-2.73566969787237e-12\\
8.254912109375	-3.5705597062514e-11\\
8.27541015625	-1.10541511171487e-10\\
8.295908203125	-2.40705916109662e-10\\
8.31640625	-1.78458946211768e-10\\
8.336904296875	-2.70161263073764e-10\\
8.35740234375	-1.31172487716089e-10\\
8.377900390625	-2.10334357246755e-10\\
8.3983984375	1.20843069579049e-11\\
8.418896484375	-2.37335309689561e-10\\
8.43939453125	-6.40466590862536e-11\\
8.459892578125	-2.45031092614203e-10\\
8.480390625	-1.86157177956013e-10\\
8.500888671875	-2.52265730414999e-10\\
8.52138671875	-2.15018832189702e-10\\
8.541884765625	-1.485951032009e-10\\
8.5623828125	-2.14319335635229e-10\\
8.582880859375	-1.98575797866773e-10\\
8.60337890625	-2.09748293201374e-10\\
8.623876953125	-2.40059242923688e-10\\
8.644375	-1.79518006362972e-10\\
8.664873046875	-2.27276668092317e-10\\
8.68537109375	-2.19144545716692e-10\\
8.705869140625	-1.46328097469573e-10\\
8.7263671875	7.02726376798214e-12\\
8.746865234375	-2.79652736943483e-12\\
8.76736328125	5.78773118590714e-11\\
8.787861328125	1.03476948694684e-11\\
8.808359375	3.69147135902988e-11\\
8.828857421875	-7.83086616552474e-11\\
8.84935546875	-1.17994863347601e-10\\
8.869853515625	-1.71592145097567e-10\\
8.8903515625	-1.74521932337284e-10\\
8.910849609375	-2.70553141320031e-10\\
8.93134765625	-1.93835542951518e-10\\
8.951845703125	-2.23946880390118e-10\\
8.97234375	-7.23680552267433e-11\\
8.992841796875	-8.37141938133673e-11\\
9.01333984375	-1.7383128027411e-11\\
9.033837890625	-1.39124367313716e-10\\
9.0543359375	-2.07571127608266e-10\\
9.074833984375	-2.46614687612051e-10\\
9.09533203125	-3.09148221713129e-10\\
9.115830078125	-3.84599334586212e-10\\
9.136328125	-4.15671785233632e-10\\
9.156826171875	-5.15679266988149e-10\\
9.17732421875	-2.867627693556e-10\\
9.197822265625	-4.0287877747079e-10\\
9.2183203125	-2.73804475117825e-10\\
9.238818359375	-3.18276200153498e-10\\
9.25931640625	-3.98475070169427e-10\\
9.279814453125	-3.80331109378599e-10\\
9.3003125	-4.29147574453722e-10\\
9.320810546875	-4.65943971030016e-10\\
9.34130859375	-5.7280213689297e-10\\
9.361806640625	-5.64829815933734e-10\\
9.3823046875	-6.21275739369124e-10\\
9.402802734375	-5.97816998230066e-10\\
9.42330078125	-4.97192036064808e-10\\
9.443798828125	-6.29099025427038e-10\\
9.464296875	-6.87858228252662e-10\\
9.484794921875	-8.00013303624737e-10\\
9.50529296875	-8.33059669222094e-10\\
9.525791015625	-8.89231691861648e-10\\
9.5462890625	-9.86567285743451e-10\\
9.566787109375	-9.19766151188266e-10\\
9.58728515625	-9.62574800972635e-10\\
9.607783203125	-8.2662958631835e-10\\
9.62828125	-9.0964910106949e-10\\
9.648779296875	-7.48402552442889e-10\\
9.66927734375	-7.77431213203237e-10\\
9.689775390625	-6.33107279974837e-10\\
9.7102734375	-7.37443383969071e-10\\
9.730771484375	-5.69510495235058e-10\\
9.75126953125	-7.50271645019416e-10\\
9.771767578125	-7.3180173278661e-10\\
9.792265625	-7.43097843943462e-10\\
9.812763671875	-7.43634227501976e-10\\
9.83326171875	-7.93705120350467e-10\\
9.853759765625	-5.6803116307381e-10\\
9.8742578125	-7.56887147843493e-10\\
9.894755859375	-3.817091403712e-10\\
9.91525390625	-6.02160015232271e-10\\
9.935751953125	-3.51463112058964e-10\\
9.95625	-5.52556936740015e-10\\
9.976748046875	-4.23846707381878e-10\\
9.99724609375	-5.9601681612216e-10\\
10.017744140625	-6.06746568191928e-10\\
10.0382421875	-5.77444418102392e-10\\
10.058740234375	-5.62634386407171e-10\\
10.07923828125	-5.85967168667177e-10\\
10.099736328125	-5.84077869082041e-10\\
10.120234375	-4.0520281145841e-10\\
10.140732421875	-3.95690112439192e-10\\
10.16123046875	-4.22801169206922e-10\\
10.181728515625	-3.88317056301213e-10\\
10.2022265625	-5.69693066276421e-10\\
10.222724609375	-6.07347910056954e-10\\
10.24322265625	-6.5123963645621e-10\\
10.263720703125	-8.61621416845137e-10\\
10.28421875	-8.25584674813395e-10\\
10.304716796875	-9.03764844364459e-10\\
10.32521484375	-8.94321804488299e-10\\
10.345712890625	-9.24518649240021e-10\\
10.3662109375	-7.85271140688681e-10\\
10.386708984375	-9.06475591062876e-10\\
10.40720703125	-7.86060937372125e-10\\
10.427705078125	-8.63558175524696e-10\\
10.448203125	-9.0221400261278e-10\\
10.468701171875	-9.75563234261104e-10\\
10.48919921875	-1.03421509726264e-09\\
10.509697265625	-1.00661592003781e-09\\
10.5301953125	-1.06614283711668e-09\\
10.550693359375	-1.13659210960897e-09\\
10.57119140625	-1.00188938691581e-09\\
10.591689453125	-1.09939611339849e-09\\
10.6121875	-1.04558305880029e-09\\
10.632685546875	-1.04490162671268e-09\\
10.65318359375	-1.00762486367332e-09\\
10.673681640625	-1.12419248698182e-09\\
10.6941796875	-1.07244840287473e-09\\
10.714677734375	-1.17581207838806e-09\\
10.73517578125	-1.11724907036203e-09\\
10.755673828125	-1.14713517628929e-09\\
10.776171875	-1.11764930987167e-09\\
10.796669921875	-1.09019783408356e-09\\
10.81716796875	-1.05398667940312e-09\\
10.837666015625	-9.98234553331049e-10\\
10.8581640625	-1.01776497670953e-09\\
10.878662109375	-9.29440879607481e-10\\
10.89916015625	-1.0198156764566e-09\\
10.919658203125	-1.01492941596151e-09\\
10.94015625	-1.09936820776231e-09\\
10.960654296875	-9.02303652237927e-10\\
10.98115234375	-1.01449325490506e-09\\
11.001650390625	-8.81886312828817e-10\\
11.0221484375	-9.38347218456425e-10\\
11.042646484375	-9.04009262406756e-10\\
11.06314453125	-6.78265554825816e-10\\
11.083642578125	-7.32110208745405e-10\\
11.104140625	-6.31098178365139e-10\\
11.124638671875	-7.05123765242174e-10\\
11.14513671875	-7.68005616070548e-10\\
11.165634765625	-7.22351649972165e-10\\
11.1861328125	-8.40887800542301e-10\\
11.206630859375	-8.18076807883256e-10\\
11.22712890625	-8.63245608028363e-10\\
11.247626953125	-7.88347059463073e-10\\
11.268125	-7.95829154043389e-10\\
11.288623046875	-7.59224680340864e-10\\
11.30912109375	-6.33420720502752e-10\\
11.329619140625	-5.59436984271356e-10\\
11.3501171875	-5.39404317956328e-10\\
11.370615234375	-4.48835412669954e-10\\
11.39111328125	-6.13111436517923e-10\\
11.411611328125	-4.95479921684735e-10\\
11.432109375	-5.30605331836171e-10\\
11.452607421875	-4.55395132692155e-10\\
11.47310546875	-6.1968728713427e-10\\
11.493603515625	-5.66469245368653e-10\\
11.5141015625	-4.71882626397429e-10\\
11.534599609375	-4.13649649501806e-10\\
11.55509765625	-3.79251756783631e-10\\
11.575595703125	-1.81453989934055e-10\\
11.59609375	-1.56204511647866e-10\\
11.616591796875	-3.72713811323209e-11\\
11.63708984375	-4.99411543183921e-11\\
11.657587890625	6.56864033261238e-12\\
11.6780859375	-1.32551622521574e-12\\
11.698583984375	-1.54661982732073e-11\\
11.71908203125	-9.46178976523838e-11\\
11.739580078125	-5.6959095744044e-11\\
11.760078125	-2.10161305353752e-10\\
11.780576171875	-8.38117950889522e-11\\
11.80107421875	-7.79605279985619e-11\\
11.821572265625	3.29956505688046e-11\\
11.8420703125	1.12020668990848e-10\\
11.862568359375	1.59436484252021e-12\\
11.88306640625	9.17947003972352e-11\\
11.903564453125	2.4613721815001e-12\\
11.9240625	-6.24336384100812e-11\\
11.944560546875	9.70395963222315e-11\\
11.96505859375	7.38219901611288e-11\\
11.985556640625	2.01313357242098e-10\\
12.0060546875	1.45432671579443e-10\\
12.026552734375	2.45186206286269e-10\\
12.04705078125	3.28518500888364e-10\\
12.067548828125	1.81120290771291e-10\\
12.088046875	3.57252930993652e-10\\
12.108544921875	2.52977301204481e-10\\
12.12904296875	3.72282179243353e-10\\
12.149541015625	1.78242615742089e-10\\
12.1700390625	2.73018833655851e-10\\
12.190537109375	8.37345942420681e-11\\
12.21103515625	2.20623104693787e-10\\
12.231533203125	1.03087056214633e-10\\
12.25203125	2.83587072670495e-10\\
12.272529296875	1.3863081424238e-10\\
12.29302734375	1.94843378721006e-10\\
12.313525390625	2.14944126052075e-10\\
12.3340234375	1.71044242420309e-10\\
12.354521484375	1.28578191632246e-10\\
12.37501953125	1.87538792714813e-10\\
12.395517578125	7.10450817124075e-11\\
12.416015625	3.47588646345259e-10\\
12.436513671875	1.13076601129067e-10\\
12.45701171875	4.18124860476336e-10\\
12.477509765625	2.8550983722287e-10\\
12.4980078125	3.56840254554716e-10\\
12.518505859375	2.63746232801342e-10\\
12.53900390625	1.97681962964962e-10\\
12.559501953125	1.81770019862754e-10\\
12.58	2.99835550952223e-10\\
12.600498046875	1.88215775851626e-10\\
12.62099609375	2.03415334487163e-10\\
12.641494140625	2.42869353364138e-10\\
12.6619921875	2.26049092133509e-10\\
12.682490234375	2.59774254436892e-10\\
12.70298828125	3.98361299467737e-10\\
12.723486328125	3.10024429670334e-10\\
12.743984375	3.53749413936281e-10\\
12.764482421875	3.91635781865105e-10\\
12.78498046875	3.76693411449078e-10\\
12.805478515625	3.76891853990341e-10\\
12.8259765625	3.67193204237849e-10\\
12.846474609375	3.58262828638741e-10\\
12.86697265625	2.28117277790182e-10\\
12.887470703125	2.23915558342083e-10\\
12.90796875	2.21916137582871e-10\\
12.928466796875	3.81085735202691e-10\\
12.94896484375	2.94948830178779e-10\\
12.969462890625	3.73936591708125e-10\\
12.9899609375	4.13224320157015e-10\\
13.010458984375	5.57113742546289e-10\\
13.03095703125	4.76729246122444e-10\\
13.051455078125	3.93417474443261e-10\\
13.071953125	3.59328649326808e-10\\
13.092451171875	4.74682428677662e-10\\
13.11294921875	2.29396017077069e-10\\
13.133447265625	4.34369421663614e-10\\
13.1539453125	3.36836460860103e-10\\
13.174443359375	4.74079764612759e-10\\
13.19494140625	4.09508819655914e-10\\
13.215439453125	5.5643989599978e-10\\
13.2359375	5.14881098167559e-10\\
13.256435546875	4.94892232193306e-10\\
13.27693359375	3.81326666023195e-10\\
13.297431640625	4.52381396237667e-10\\
13.3179296875	3.72062417100488e-10\\
13.338427734375	4.3880372619212e-10\\
13.35892578125	3.68775072383364e-10\\
13.379423828125	4.2587193816542e-10\\
13.399921875	3.90306560863822e-10\\
13.420419921875	4.11644686238501e-10\\
13.44091796875	5.17933377771843e-10\\
13.461416015625	3.99508829521914e-10\\
13.4819140625	4.43823549558467e-10\\
13.502412109375	3.05013376900086e-10\\
13.52291015625	3.23658566989454e-10\\
13.543408203125	2.36448107145216e-10\\
13.56390625	2.36596259240507e-10\\
13.584404296875	3.50106773902172e-10\\
13.60490234375	2.21179974740414e-10\\
13.625400390625	2.99169630880831e-10\\
13.6458984375	2.67637133009116e-10\\
13.666396484375	2.75081785848305e-10\\
13.68689453125	2.59705011445315e-10\\
13.707392578125	2.23240224554661e-10\\
13.727890625	2.35340102410718e-10\\
13.748388671875	2.75852099560896e-10\\
13.76888671875	2.7144607337812e-10\\
13.789384765625	2.27241275900914e-10\\
13.8098828125	2.65047450786207e-10\\
13.830380859375	2.24409911751933e-10\\
13.85087890625	2.52889366653107e-10\\
13.871376953125	1.04736151317943e-10\\
13.891875	2.37720316365321e-10\\
13.912373046875	-4.36567423849805e-12\\
13.93287109375	1.64761299609786e-10\\
13.953369140625	5.244906841232e-11\\
13.9738671875	1.66423125489314e-10\\
13.994365234375	2.16201427163994e-10\\
14.01486328125	9.65425577982131e-11\\
14.035361328125	4.76350171460156e-11\\
14.055859375	1.36300470250361e-10\\
14.076357421875	-1.68643995764078e-10\\
14.09685546875	-1.0452029859916e-10\\
14.117353515625	-2.93995557739879e-10\\
14.1378515625	-3.29379576995112e-10\\
14.158349609375	-4.78046454829873e-10\\
14.17884765625	-5.05809504244804e-10\\
14.199345703125	-3.92646925286261e-10\\
14.21984375	-3.18879594225773e-10\\
14.240341796875	-3.61400408098187e-10\\
14.26083984375	-2.19943285166813e-10\\
14.281337890625	-3.19302799429051e-10\\
14.3018359375	-1.92537974603474e-10\\
14.322333984375	-2.86235611227769e-10\\
14.34283203125	-3.34943420310418e-10\\
14.363330078125	-2.06656765552838e-10\\
14.383828125	-3.35050965803396e-10\\
14.404326171875	-2.45180178070316e-10\\
14.42482421875	-2.06685267824875e-10\\
14.445322265625	-4.4186561021384e-10\\
14.4658203125	-2.58309806406057e-10\\
14.486318359375	-4.26901298365554e-10\\
14.50681640625	-3.02857639536403e-10\\
14.527314453125	-3.72030463249806e-10\\
14.5478125	-4.50250183965688e-10\\
14.568310546875	-3.61324316434326e-10\\
14.58880859375	-5.28788686764818e-10\\
14.609306640625	-3.51924063528175e-10\\
14.6298046875	-5.06756229233726e-10\\
14.650302734375	-3.44960692819718e-10\\
14.67080078125	-4.30773186221027e-10\\
14.691298828125	-3.58262067099863e-10\\
14.711796875	-5.33964786166416e-10\\
14.732294921875	-3.54039567037224e-10\\
14.75279296875	-4.84633235977367e-10\\
14.773291015625	-3.90364373107148e-10\\
14.7937890625	-3.51663125327464e-10\\
14.814287109375	-4.29745823196505e-10\\
14.83478515625	-4.00177386667633e-10\\
14.855283203125	-4.19644088875897e-10\\
14.87578125	-5.24342968751839e-10\\
14.896279296875	-4.59792543421886e-10\\
14.91677734375	-6.46246070159334e-10\\
14.937275390625	-3.88020382791779e-10\\
14.9577734375	-5.92101335428952e-10\\
14.978271484375	-4.19487228330098e-10\\
14.99876953125	-4.99944631272212e-10\\
15.019267578125	-3.36154574525989e-10\\
15.039765625	-3.4572709778499e-10\\
15.060263671875	-3.86222888560941e-10\\
15.08076171875	-3.88427996839577e-10\\
15.101259765625	-3.82985570845654e-10\\
15.1217578125	-5.11669540752819e-10\\
15.142255859375	-5.08753362618586e-10\\
15.16275390625	-4.95276752459651e-10\\
15.183251953125	-4.64911549701075e-10\\
15.20375	-4.97919040717424e-10\\
15.224248046875	-3.01640481677908e-10\\
15.24474609375	-5.18762861906299e-10\\
15.265244140625	-3.94224019382883e-10\\
15.2857421875	-5.0094093765179e-10\\
15.306240234375	-4.76497182166054e-10\\
15.32673828125	-4.71209919467562e-10\\
15.347236328125	-4.58655254883292e-10\\
15.367734375	-4.43982728663609e-10\\
15.388232421875	-4.15155879513104e-10\\
15.40873046875	-3.68137237943719e-10\\
15.429228515625	-5.0672161133688e-10\\
15.4497265625	-4.53753623841917e-10\\
15.470224609375	-4.08605256906568e-10\\
15.49072265625	-5.09100752190598e-10\\
15.511220703125	-5.73594637619956e-10\\
15.53171875	-4.42718613059901e-10\\
15.552216796875	-4.73855994614982e-10\\
15.57271484375	-4.29746504639083e-10\\
15.593212890625	-4.08878833659376e-10\\
15.6137109375	-2.75499859361241e-10\\
15.634208984375	-3.69008723801796e-10\\
15.65470703125	-2.88687913104265e-10\\
15.675205078125	-4.52380969473552e-10\\
15.695703125	-3.59605797733222e-10\\
15.716201171875	-5.77831786421294e-10\\
15.73669921875	-5.12730141977907e-10\\
15.757197265625	-6.29726280830945e-10\\
15.7776953125	-5.00684561851372e-10\\
15.798193359375	-4.46769289148175e-10\\
15.81869140625	-4.23751405664394e-10\\
15.839189453125	-3.67794799960107e-10\\
15.8596875	-2.9081457712667e-10\\
15.880185546875	-3.20250943243279e-10\\
15.90068359375	-2.48897432434028e-10\\
15.921181640625	-2.87620444476055e-10\\
15.9416796875	-4.07522366750269e-10\\
15.962177734375	-3.52641050435094e-10\\
15.98267578125	-5.30497508690168e-10\\
16.003173828125	-3.46085828975459e-10\\
16.023671875	-3.49876829518014e-10\\
16.044169921875	-3.58255323555033e-10\\
16.06466796875	-2.5664919033727e-10\\
16.085166015625	-3.75734998465485e-10\\
16.1056640625	-1.96512139449233e-10\\
16.126162109375	-2.44671946623252e-10\\
16.14666015625	-1.18310708375958e-10\\
16.167158203125	-1.88709774924748e-10\\
16.18765625	-2.09432615192335e-10\\
16.208154296875	-2.72793058788546e-10\\
16.22865234375	-3.07517314410067e-10\\
16.249150390625	-3.21999376308072e-10\\
16.2696484375	-3.16112345206118e-10\\
16.290146484375	-3.74947358120844e-10\\
16.31064453125	-3.91309707235806e-10\\
16.331142578125	-2.99432229671846e-10\\
16.351640625	-3.59991625791401e-10\\
16.372138671875	-1.70472790870939e-10\\
16.39263671875	-3.37938947325264e-10\\
16.413134765625	-8.42447735792799e-11\\
16.4336328125	-2.15603698955679e-10\\
16.454130859375	-3.7604851354953e-11\\
16.47462890625	-5.22003084145205e-11\\
16.495126953125	-2.49384091663218e-11\\
16.515625	-5.45861854844173e-11\\
16.536123046875	2.54508369984631e-12\\
16.55662109375	-9.36116448831215e-11\\
16.577119140625	1.56408564237559e-10\\
16.5976171875	7.08811885880861e-11\\
16.618115234375	3.04032205298607e-10\\
16.63861328125	2.5394271511208e-10\\
16.659111328125	3.12526901354605e-10\\
16.679609375	3.54160029031562e-10\\
16.700107421875	2.53578793051978e-10\\
16.72060546875	1.60594757256058e-10\\
16.741103515625	1.62578620684542e-10\\
16.7616015625	-2.15722652048334e-11\\
16.782099609375	1.45435339200273e-10\\
16.80259765625	6.68697525776992e-11\\
16.823095703125	2.18448580636544e-10\\
16.84359375	2.35190834424003e-10\\
16.864091796875	1.62308359404299e-10\\
16.88458984375	3.38054105270541e-10\\
16.905087890625	3.20246060021533e-10\\
16.9255859375	2.24670364089181e-10\\
16.946083984375	1.90319075503767e-10\\
16.96658203125	-7.08180288405604e-11\\
16.987080078125	1.3658665739346e-10\\
17.007578125	3.76735696875938e-11\\
17.028076171875	5.41790442844972e-11\\
17.04857421875	2.46602126714997e-10\\
17.069072265625	2.09887144547527e-10\\
17.0895703125	3.05211156860467e-10\\
17.110068359375	2.21596404682053e-10\\
17.13056640625	3.39472943809708e-10\\
17.151064453125	1.37390598487613e-10\\
17.1715625	2.91973660809071e-10\\
17.192060546875	1.77305918139665e-10\\
17.21255859375	1.81378942539063e-10\\
17.233056640625	4.88874801212445e-11\\
17.2535546875	1.50748174601956e-10\\
17.274052734375	9.23061504852416e-11\\
17.29455078125	1.13299383531999e-10\\
17.315048828125	1.72913632602517e-10\\
17.335546875	2.40237809767725e-10\\
17.356044921875	2.12956771680085e-10\\
17.37654296875	3.23509904411306e-10\\
17.397041015625	8.96458923960107e-11\\
17.4175390625	2.02531000545626e-10\\
17.438037109375	-3.48545931742346e-11\\
17.45853515625	3.24780284138665e-10\\
17.479033203125	1.66617423500263e-11\\
17.49953125	2.40934936417741e-10\\
17.520029296875	2.01721684885601e-11\\
17.54052734375	1.93494834519057e-10\\
17.561025390625	2.21452054130802e-10\\
17.5815234375	1.6469357784707e-10\\
17.602021484375	1.22382135392837e-10\\
17.62251953125	1.42582447170451e-10\\
17.643017578125	5.26123502627935e-11\\
17.663515625	-8.36066743912191e-11\\
17.684013671875	-2.90064804521495e-11\\
17.70451171875	-5.69157898811023e-11\\
17.725009765625	-9.61905690123328e-11\\
17.7455078125	4.49936946735475e-11\\
17.766005859375	-9.98416332708416e-11\\
17.78650390625	3.04060509064636e-11\\
17.807001953125	6.93576694878922e-12\\
17.8275	-5.58688218011077e-12\\
17.847998046875	-1.0050430596587e-10\\
17.86849609375	-6.95721760356687e-11\\
17.888994140625	-1.22039851573327e-10\\
17.9094921875	-1.91120392600435e-10\\
17.929990234375	-2.25187151034808e-10\\
17.95048828125	-1.84300109720588e-10\\
17.970986328125	-1.51447244780326e-10\\
17.991484375	-4.22918552675726e-11\\
18.011982421875	-1.21544061434025e-10\\
18.03248046875	-4.52388114966252e-11\\
18.052978515625	-7.2168994435822e-11\\
18.0734765625	-8.43822174987906e-11\\
18.093974609375	-1.64888007187545e-10\\
18.11447265625	-1.10330481143813e-10\\
18.134970703125	-1.53367543860318e-10\\
18.15546875	-1.76721872284882e-10\\
18.175966796875	3.54480192017825e-11\\
18.19646484375	-1.15903627884587e-10\\
18.216962890625	1.38720335209179e-10\\
18.2374609375	2.69711918740233e-11\\
18.257958984375	1.11378867821217e-10\\
18.27845703125	-6.75649496643789e-11\\
18.298955078125	-5.2729556850635e-11\\
18.319453125	-1.04865837427067e-10\\
18.339951171875	-8.88283253323298e-11\\
18.36044921875	-2.09110806202387e-10\\
18.380947265625	-2.45391992620517e-11\\
18.4014453125	-1.30771649508463e-10\\
18.421943359375	5.8695613977463e-11\\
18.44244140625	3.27595477346518e-11\\
18.462939453125	4.86650441565336e-11\\
18.4834375	6.51398176688085e-11\\
18.503935546875	-4.99242475903443e-11\\
18.52443359375	-1.78396962921756e-10\\
18.544931640625	-1.68290713826646e-10\\
18.5654296875	-4.42844100759066e-10\\
18.585927734375	-2.75179185200718e-10\\
18.60642578125	-4.4258516000935e-10\\
18.626923828125	-3.37978865334773e-10\\
18.647421875	-3.35698407363472e-10\\
18.667919921875	-3.24443888389673e-10\\
18.68841796875	-2.55394912642731e-10\\
18.708916015625	-1.51788145065574e-10\\
18.7294140625	-2.84641451565522e-10\\
18.749912109375	-2.18764845555269e-10\\
18.77041015625	-2.77802125527049e-10\\
18.790908203125	-2.79772678755682e-10\\
18.81140625	-3.43693467327812e-10\\
18.831904296875	-4.41971889450978e-10\\
18.85240234375	-2.37502781051042e-10\\
18.872900390625	-5.04970690891291e-10\\
18.8933984375	-2.89263418406408e-10\\
18.913896484375	-5.81541566318417e-10\\
18.93439453125	-3.65683407157514e-10\\
18.954892578125	-6.53486199521588e-10\\
18.975390625	-5.69043731116779e-10\\
18.995888671875	-6.38763406972941e-10\\
19.01638671875	-5.35793382032174e-10\\
19.036884765625	-7.282621323789e-10\\
19.0573828125	-5.64537411524708e-10\\
19.077880859375	-6.8540201264858e-10\\
19.09837890625	-5.36248243076275e-10\\
19.118876953125	-6.18490249663541e-10\\
19.139375	-5.66075097540608e-10\\
19.159873046875	-5.71275523917154e-10\\
19.18037109375	-7.36794840530714e-10\\
19.200869140625	-7.86251964011595e-10\\
19.2213671875	-7.14155547074614e-10\\
19.241865234375	-8.07767549264555e-10\\
19.26236328125	-6.31550366073805e-10\\
19.282861328125	-7.35053412642963e-10\\
19.303359375	-5.33299431319852e-10\\
19.323857421875	-5.50445554960576e-10\\
19.34435546875	-6.22459234949934e-10\\
19.364853515625	-5.51728866323225e-10\\
19.3853515625	-5.47906532176527e-10\\
19.405849609375	-5.99279952234572e-10\\
19.42634765625	-6.03515422949297e-10\\
19.446845703125	-6.6435989171358e-10\\
19.46734375	-5.25363514462893e-10\\
19.487841796875	-6.91731608965259e-10\\
19.50833984375	-6.68718807622578e-10\\
19.528837890625	-6.12236751846267e-10\\
19.5493359375	-7.26423909376442e-10\\
19.569833984375	-6.75833017899098e-10\\
19.59033203125	-7.91662772328922e-10\\
19.610830078125	-5.90257706730057e-10\\
19.631328125	-7.30115139899494e-10\\
19.651826171875	-6.80603463038375e-10\\
19.67232421875	-6.49571549477911e-10\\
19.692822265625	-6.59076320272072e-10\\
19.7133203125	-8.42161645189778e-10\\
19.733818359375	-7.62219090295988e-10\\
19.75431640625	-9.89344376083591e-10\\
19.774814453125	-9.46270107076756e-10\\
19.7953125	-9.71846241310165e-10\\
19.815810546875	-8.45641241122448e-10\\
19.83630859375	-8.42816416135913e-10\\
19.856806640625	-6.13557735415297e-10\\
19.8773046875	-6.74904929295616e-10\\
19.897802734375	-4.56061302231422e-10\\
19.91830078125	-7.4392699135397e-10\\
19.938798828125	-6.17363179171471e-10\\
19.959296875	-7.94387908201143e-10\\
19.979794921875	-6.98996399389456e-10\\
20.00029296875	-8.72494853634863e-10\\
20.020791015625	-7.69718108995257e-10\\
20.0412890625	-7.96019292395519e-10\\
20.061787109375	-6.33785938381106e-10\\
20.08228515625	-5.29905148560071e-10\\
20.102783203125	-3.5332186987829e-10\\
20.12328125	-3.06329127814968e-10\\
20.143779296875	-1.99266336854594e-10\\
20.16427734375	-2.58983699592426e-10\\
20.184775390625	-2.84056920928149e-10\\
20.2052734375	-4.2125009365509e-10\\
20.225771484375	-3.66637751161525e-10\\
20.24626953125	-4.98796992153115e-10\\
20.266767578125	-3.2647586575273e-10\\
20.287265625	-5.39306387260173e-10\\
20.307763671875	-2.91573455544631e-10\\
20.32826171875	-2.99795306292815e-10\\
20.348759765625	-2.49721698245154e-10\\
20.3692578125	-3.08986002553379e-10\\
20.389755859375	-3.53426260213182e-10\\
20.41025390625	-4.52566357918842e-10\\
20.430751953125	-4.43489550710287e-10\\
20.45125	-5.89483036139839e-10\\
20.471748046875	-5.03061920796456e-10\\
20.49224609375	-3.63489417963869e-10\\
20.512744140625	-3.41972953207255e-10\\
20.5332421875	-2.47779994307119e-10\\
20.553740234375	-2.11014459830675e-10\\
20.57423828125	-2.30094084101353e-10\\
20.594736328125	-2.20627648939245e-10\\
20.615234375	-3.90250210456439e-10\\
20.635732421875	-4.32876807282879e-10\\
20.65623046875	-3.27270684414468e-10\\
20.676728515625	-5.12471340544369e-10\\
20.6972265625	-2.41253906362406e-10\\
20.717724609375	-4.37992925071553e-10\\
20.73822265625	-1.8693744547107e-10\\
20.758720703125	-3.34289297156363e-10\\
20.77921875	-8.91071650047149e-11\\
20.799716796875	-2.55114048228649e-10\\
20.82021484375	-7.21582210891823e-11\\
20.840712890625	-2.52627695795608e-10\\
20.8612109375	-2.3805192676437e-10\\
20.881708984375	-3.80531127249569e-10\\
20.90220703125	-2.71796578746042e-10\\
20.922705078125	-4.83209273122952e-10\\
20.943203125	-3.17082662423129e-10\\
20.963701171875	-3.8374360144607e-10\\
20.98419921875	-3.66340335978827e-10\\
21.004697265625	-3.54573614902675e-10\\
21.0251953125	-2.0123814682228e-10\\
21.045693359375	-1.84323444705981e-10\\
21.06619140625	-5.52643049961058e-11\\
21.086689453125	-1.61454033900982e-10\\
21.1071875	-8.4372036844102e-11\\
21.127685546875	-2.39505773439456e-10\\
21.14818359375	-1.49693353229807e-10\\
21.168681640625	-1.73698506630494e-10\\
21.1891796875	-6.68596266066393e-11\\
21.209677734375	-1.14722309053915e-10\\
21.23017578125	7.79762318621532e-12\\
21.250673828125	-2.14109726345004e-11\\
21.271171875	1.05032984596783e-10\\
21.291669921875	-3.17012129397377e-11\\
21.31216796875	5.60363188795888e-11\\
21.332666015625	3.77622192037033e-11\\
21.3531640625	-1.67804112004556e-10\\
21.373662109375	1.28073428309559e-11\\
21.39416015625	-1.34828418066901e-10\\
21.414658203125	8.54941238627627e-11\\
21.43515625	-4.06548453862277e-11\\
21.455654296875	2.10142219665545e-10\\
21.47615234375	1.63046593380796e-10\\
21.496650390625	3.16878214751422e-10\\
21.5171484375	2.49606597183901e-10\\
21.537646484375	5.72340134633373e-10\\
21.55814453125	2.4568273586133e-10\\
21.578642578125	5.23739844921714e-10\\
21.599140625	4.022462308224e-10\\
21.619638671875	5.09797815881551e-10\\
21.64013671875	5.31212921364895e-10\\
21.660634765625	7.59914325047087e-10\\
21.6811328125	8.26573420347283e-10\\
21.701630859375	9.56153688069551e-10\\
21.72212890625	8.95468430352227e-10\\
21.742626953125	9.87960136993635e-10\\
21.763125	9.08064196334997e-10\\
21.783623046875	8.66137573383318e-10\\
21.80412109375	7.21855501873442e-10\\
21.824619140625	5.43843833415959e-10\\
21.8451171875	5.05945043847289e-10\\
21.865615234375	5.22512681164398e-10\\
21.88611328125	5.26617837732023e-10\\
21.906611328125	6.09959971429245e-10\\
21.927109375	6.39825728283649e-10\\
21.947607421875	7.06996647357154e-10\\
21.96810546875	6.98278285637087e-10\\
21.988603515625	6.765969659739e-10\\
22.0091015625	6.98646320056692e-10\\
22.029599609375	6.61023795952524e-10\\
22.05009765625	7.52065399457669e-10\\
22.070595703125	5.86581088357763e-10\\
22.09109375	6.2990290057656e-10\\
22.111591796875	5.44265358981111e-10\\
22.13208984375	6.34263305562656e-10\\
22.152587890625	5.4522789867752e-10\\
22.1730859375	6.1228493819875e-10\\
22.193583984375	4.04717509552218e-10\\
22.21408203125	6.86206552917094e-10\\
22.234580078125	5.24639635255173e-10\\
22.255078125	7.19376183494094e-10\\
22.275576171875	5.91330106602818e-10\\
22.29607421875	8.41822135962362e-10\\
22.316572265625	7.45166897015703e-10\\
22.3370703125	7.28593785241849e-10\\
22.357568359375	6.09843462355978e-10\\
22.37806640625	5.59362345992014e-10\\
22.398564453125	5.06580834076723e-10\\
22.4190625	5.21966531650274e-10\\
22.439560546875	4.43430365864951e-10\\
22.46005859375	6.13484337381568e-10\\
22.480556640625	5.54978739591592e-10\\
22.5010546875	7.28001505062192e-10\\
22.521552734375	5.77699060007984e-10\\
22.54205078125	6.07941575296155e-10\\
22.562548828125	4.18454945742216e-10\\
22.583046875	4.28477267471728e-10\\
22.603544921875	5.14985906084912e-10\\
22.62404296875	4.17766340116979e-10\\
22.644541015625	5.10583492441787e-10\\
22.6650390625	5.0872566515798e-10\\
22.685537109375	4.98342031430091e-10\\
22.70603515625	4.31981963835697e-10\\
22.726533203125	4.02138476566367e-10\\
22.74703125	4.03097017789359e-10\\
22.767529296875	2.83099501327933e-10\\
22.78802734375	4.29038399234672e-10\\
22.808525390625	9.6277001086979e-11\\
22.8290234375	2.31093994554239e-10\\
22.849521484375	2.36686485382105e-10\\
22.87001953125	3.70620940782502e-10\\
22.890517578125	4.10180241531214e-10\\
22.911015625	5.0536355682005e-10\\
22.931513671875	3.69450472913712e-10\\
22.95201171875	4.06556029211907e-10\\
22.972509765625	2.4921937829402e-10\\
22.9930078125	2.8307345364422e-10\\
23.013505859375	2.31324863784659e-10\\
23.03400390625	1.55630507574637e-10\\
23.054501953125	2.24878321672714e-10\\
23.075	1.14249278089797e-10\\
23.095498046875	2.29583003733338e-10\\
23.11599609375	3.90514450337581e-10\\
23.136494140625	4.17280567886649e-10\\
23.1569921875	3.94673322971434e-10\\
23.177490234375	2.53338920962455e-10\\
23.19798828125	3.36313762607153e-11\\
23.218486328125	2.38658209381284e-10\\
23.238984375	-1.4331788057649e-10\\
23.259482421875	1.40594936464805e-10\\
23.27998046875	-1.97113098435325e-11\\
23.300478515625	1.2976133177646e-10\\
23.3209765625	3.89827132581402e-12\\
23.341474609375	1.71047982339703e-10\\
23.36197265625	2.25327886841852e-11\\
23.382470703125	2.12798852604459e-10\\
23.40296875	-9.46127421523108e-11\\
23.423466796875	1.47076360121647e-10\\
23.44396484375	2.56555595206733e-11\\
23.464462890625	1.63828219939665e-10\\
23.4849609375	2.30303149809816e-10\\
23.505458984375	2.7676553715768e-10\\
23.52595703125	2.7459683353591e-10\\
23.546455078125	3.77758111674871e-10\\
23.566953125	3.37303742528649e-10\\
23.587451171875	3.31262430177768e-10\\
23.60794921875	2.51596070556753e-10\\
23.628447265625	1.72676555317977e-10\\
23.6489453125	4.36394791918139e-11\\
23.669443359375	1.5790054389637e-10\\
23.68994140625	4.74331125531168e-12\\
23.710439453125	9.84580480326379e-11\\
23.7309375	5.73662126137104e-11\\
23.751435546875	1.10603070985466e-10\\
23.77193359375	3.28385628604666e-11\\
23.792431640625	1.22634743301761e-10\\
23.8129296875	4.29871177523813e-11\\
23.833427734375	1.29534632622051e-10\\
23.85392578125	1.74905897191843e-10\\
23.874423828125	6.16701908822457e-12\\
23.894921875	1.99244726371976e-10\\
23.915419921875	2.92045833491304e-11\\
23.93591796875	1.87469705984261e-10\\
23.956416015625	-1.33135602867434e-11\\
23.9769140625	6.46895422364243e-11\\
23.997412109375	-2.42438078309947e-10\\
24.01791015625	-2.18249783678762e-10\\
24.038408203125	-5.82470141782331e-10\\
24.05890625	-2.22139238515339e-10\\
24.079404296875	-4.88101679149231e-10\\
24.09990234375	-3.91154019218757e-10\\
24.120400390625	-3.28868824815956e-10\\
24.1408984375	-2.88200094495995e-10\\
24.161396484375	-4.85335972065871e-10\\
24.18189453125	-3.90278897304671e-10\\
24.202392578125	-5.62480997827166e-10\\
24.222890625	-5.31391324524352e-10\\
24.243388671875	-7.01559842013202e-10\\
24.26388671875	-6.42585749475949e-10\\
24.284384765625	-6.09397669095891e-10\\
24.3048828125	-6.58733334008532e-10\\
24.325380859375	-5.92883276870791e-10\\
24.34587890625	-5.47252630887131e-10\\
24.366376953125	-5.78380245830366e-10\\
24.386875	-6.00099106844465e-10\\
24.407373046875	-5.13146746917845e-10\\
24.42787109375	-4.99525958709214e-10\\
24.448369140625	-4.28414956651829e-10\\
24.4688671875	-4.21443257640516e-10\\
24.489365234375	-3.8985529197163e-10\\
24.50986328125	-3.96255386932744e-10\\
24.530361328125	-4.59090788267658e-10\\
24.550859375	-5.37910860378624e-10\\
24.571357421875	-3.83728131424435e-10\\
24.59185546875	-5.79429755552123e-10\\
24.612353515625	-5.01025958913821e-10\\
24.6328515625	-5.09027130481667e-10\\
24.653349609375	-4.64893901982166e-10\\
24.67384765625	-5.26499185022446e-10\\
24.694345703125	-4.07835654002436e-10\\
24.71484375	-4.10794983235793e-10\\
24.735341796875	-3.14192521238019e-10\\
24.75583984375	-4.34334892991408e-10\\
24.776337890625	-2.07282926624187e-10\\
24.7968359375	-3.99506375001248e-10\\
24.817333984375	-2.31212530485784e-10\\
24.83783203125	-3.34290596965861e-10\\
24.858330078125	-3.64978923178733e-10\\
24.878828125	-3.70176415783635e-10\\
24.899326171875	-3.2316056242577e-10\\
24.91982421875	-3.48937921003689e-10\\
24.940322265625	-3.26774213488282e-10\\
24.9608203125	-4.20106886921522e-10\\
24.981318359375	-3.32061292886481e-10\\
25.00181640625	-5.08856360536288e-10\\
25.022314453125	-3.27219286166944e-10\\
25.0428125	-4.38850028024209e-10\\
25.063310546875	-3.1062157706381e-10\\
25.08380859375	-2.91765623799469e-10\\
25.104306640625	-2.84354731163595e-10\\
25.1248046875	-1.53328840817091e-10\\
25.145302734375	-3.08392613586159e-10\\
25.16580078125	-3.14386981842619e-10\\
25.186298828125	-2.74446860177414e-10\\
25.206796875	-2.66677120054987e-10\\
25.227294921875	-3.16695980740026e-10\\
25.24779296875	-2.89266711379353e-10\\
25.268291015625	-2.42051888421291e-10\\
25.2887890625	-4.11718372806988e-10\\
25.309287109375	-2.03226552546652e-10\\
25.32978515625	-2.81108329113894e-10\\
25.350283203125	-1.3768144298521e-10\\
25.37078125	-2.39163554223942e-10\\
25.391279296875	-2.65217187482259e-10\\
25.41177734375	-3.15541575000909e-10\\
25.432275390625	-3.12030797182394e-10\\
25.4527734375	-3.74770585442153e-10\\
25.473271484375	-2.1012560614731e-10\\
25.49376953125	-3.80165983968758e-10\\
25.514267578125	-2.19376692250986e-10\\
25.534765625	-1.71124929925745e-10\\
25.555263671875	-1.49307652756758e-10\\
25.57576171875	-1.77155108987348e-10\\
25.596259765625	-2.0213846781684e-10\\
25.6167578125	-2.61713412520263e-10\\
25.637255859375	-2.18816804362796e-10\\
25.65775390625	-3.08791330153972e-10\\
25.678251953125	-1.57960958913256e-10\\
25.69875	-1.97585538316024e-11\\
25.719248046875	-1.28548691538603e-10\\
25.73974609375	2.42431751398775e-10\\
25.760244140625	1.18728659587543e-11\\
25.7807421875	1.32093872342616e-10\\
25.801240234375	-6.50995940686532e-11\\
25.82173828125	1.42723651161082e-10\\
25.842236328125	-5.07283896637689e-11\\
25.862734375	7.34166676094877e-11\\
25.883232421875	-9.19154952034644e-11\\
25.90373046875	1.95405009376621e-10\\
25.924228515625	5.44787468054305e-11\\
25.9447265625	2.24389127502925e-10\\
25.965224609375	1.79653168108612e-10\\
25.98572265625	2.44234426758181e-10\\
26.006220703125	2.03881419616572e-10\\
26.02671875	9.18470302079794e-11\\
26.047216796875	-2.84192066844754e-11\\
26.06771484375	-2.62828599549674e-11\\
26.088212890625	-1.24132142117405e-10\\
26.1087109375	-1.78830840322741e-10\\
26.129208984375	-2.22487650003867e-10\\
26.14970703125	-1.35677238300329e-10\\
26.170205078125	-1.38918318968919e-10\\
26.190703125	-7.49355637927679e-11\\
26.211201171875	-7.33961996817704e-11\\
26.23169921875	-1.67606639324763e-11\\
26.252197265625	3.99057823031327e-11\\
26.2726953125	1.56747047792129e-10\\
26.293193359375	2.31929448281075e-11\\
26.31369140625	2.07377344778168e-10\\
26.334189453125	-3.74469052939564e-11\\
26.3546875	3.32856885699906e-11\\
26.375185546875	-3.32431262305475e-11\\
26.39568359375	-2.06163424690003e-10\\
26.416181640625	6.81696374484487e-12\\
26.4366796875	-6.25928330684528e-11\\
26.457177734375	8.71705814036562e-11\\
26.47767578125	1.41486888088205e-10\\
26.498173828125	4.1041772736994e-10\\
26.518671875	3.27133384730172e-10\\
26.539169921875	6.16784766873367e-10\\
26.55966796875	4.2782677089809e-10\\
26.580166015625	5.93549737622659e-10\\
26.6006640625	4.6599692479834e-10\\
26.621162109375	4.33784733171275e-10\\
26.64166015625	3.82304617345079e-10\\
26.662158203125	4.23732411100152e-10\\
26.68265625	4.79081234490776e-10\\
26.703154296875	7.22655385170643e-10\\
26.72365234375	6.93116102188947e-10\\
26.744150390625	8.02722016812494e-10\\
26.7646484375	8.4168779824095e-10\\
26.785146484375	9.36115521972957e-10\\
26.80564453125	8.47203817927906e-10\\
26.826142578125	7.85998069906276e-10\\
26.846640625	7.13542762530489e-10\\
26.867138671875	6.49125445666993e-10\\
26.88763671875	7.30598921205907e-10\\
26.908134765625	5.76159146846794e-10\\
26.9286328125	6.28930618150007e-10\\
26.949130859375	5.6353870474408e-10\\
26.96962890625	6.18100551034753e-10\\
26.990126953125	5.44103544559904e-10\\
27.010625	6.35724664997006e-10\\
27.031123046875	6.90037764682327e-10\\
27.05162109375	6.13354540083378e-10\\
27.072119140625	5.48357425863482e-10\\
27.0926171875	6.20243571864988e-10\\
27.113115234375	5.21794203200005e-10\\
27.13361328125	5.50185944568044e-10\\
27.154111328125	4.59089178509767e-10\\
27.174609375	6.1660436651008e-10\\
27.195107421875	5.55518991651325e-10\\
27.21560546875	6.11188301371717e-10\\
27.236103515625	5.8061242234214e-10\\
27.2566015625	7.48498784580532e-10\\
27.277099609375	5.55707236467334e-10\\
27.29759765625	7.73617218742143e-10\\
27.318095703125	5.37207243533362e-10\\
27.33859375	6.13015974750008e-10\\
27.359091796875	4.10333877674425e-10\\
27.37958984375	4.47541142290918e-10\\
27.400087890625	3.78153874768547e-10\\
27.4205859375	3.43063929089697e-10\\
27.441083984375	4.43243997138192e-10\\
27.46158203125	4.74080780499239e-10\\
27.482080078125	4.59728551128917e-10\\
27.502578125	6.10196155727345e-10\\
27.523076171875	5.23171988660627e-10\\
27.54357421875	6.17875962158863e-10\\
27.564072265625	4.57128866513832e-10\\
27.5845703125	4.52228365859602e-10\\
27.605068359375	4.44384533937253e-10\\
27.62556640625	3.76018218344695e-10\\
27.646064453125	4.03256688688737e-10\\
27.6665625	4.41789310602089e-10\\
27.687060546875	3.25617405480597e-10\\
27.70755859375	4.45757740807183e-10\\
27.728056640625	4.02458603843773e-10\\
27.7485546875	2.84958005839881e-10\\
27.769052734375	3.46037018695617e-10\\
27.78955078125	2.89191830710231e-10\\
27.810048828125	2.46210687927352e-10\\
27.830546875	3.64383664040273e-10\\
27.851044921875	2.38419254389246e-10\\
27.87154296875	2.75065260972048e-10\\
27.892041015625	2.95135868887374e-10\\
27.9125390625	3.15174558088133e-10\\
27.933037109375	3.54081173667574e-10\\
27.95353515625	4.34359207110718e-10\\
27.974033203125	3.01839278613265e-10\\
27.99453125	4.41755745030795e-10\\
28.015029296875	2.78495916311548e-10\\
28.03552734375	2.96431410291719e-10\\
28.056025390625	3.58139284654064e-10\\
28.0765234375	2.9854374268434e-10\\
28.097021484375	2.46218259530626e-10\\
28.11751953125	2.16078246237436e-10\\
28.138017578125	1.62877083940384e-10\\
28.158515625	1.97981303090342e-10\\
28.179013671875	7.55230485288062e-11\\
28.19951171875	2.34612761968656e-11\\
28.220009765625	6.75232006932753e-11\\
28.2405078125	-2.34942309786185e-11\\
28.261005859375	2.90205726068655e-11\\
28.28150390625	1.18097397388795e-12\\
28.302001953125	6.63655425972239e-11\\
28.3225	-1.78228508716873e-11\\
28.342998046875	3.5217313291707e-11\\
28.36349609375	-6.92989255643752e-11\\
28.383994140625	2.01289075925026e-11\\
28.4044921875	-9.39146939495988e-11\\
28.424990234375	5.10863826483003e-11\\
28.44548828125	-7.84582718584257e-11\\
28.465986328125	-2.24762909830128e-12\\
28.486484375	-4.4556509708729e-11\\
28.506982421875	5.48579610986475e-11\\
28.52748046875	-1.57663162373144e-11\\
28.547978515625	2.15809324830065e-11\\
28.5684765625	-4.11969969069755e-11\\
28.588974609375	8.38342703171281e-11\\
28.60947265625	3.44337636100239e-11\\
28.629970703125	6.29025690484843e-11\\
28.65046875	3.57075094803941e-11\\
28.670966796875	2.19868954053786e-10\\
28.69146484375	1.21589639531688e-10\\
28.711962890625	1.6319991035742e-10\\
28.7324609375	1.49584409802147e-10\\
28.752958984375	5.66994463717957e-11\\
28.77345703125	3.37539964760721e-11\\
28.793955078125	1.15115206730454e-10\\
28.814453125	-4.27731112258366e-11\\
28.834951171875	1.55724949602929e-10\\
28.85544921875	7.61020674567502e-12\\
28.875947265625	6.80966099306e-11\\
28.8964453125	1.18789440480671e-10\\
28.916943359375	-1.13978573419039e-11\\
28.93744140625	2.72644543434916e-11\\
28.957939453125	-9.55069684537496e-11\\
28.9784375	-4.34687518601347e-11\\
28.998935546875	-2.16709571905945e-10\\
29.01943359375	-1.00082387038349e-10\\
29.039931640625	-2.6544463780297e-10\\
29.0604296875	-2.41592796428895e-10\\
29.080927734375	-4.27918282329105e-10\\
29.10142578125	-3.58196984125566e-10\\
29.121923828125	-3.39997599627338e-10\\
29.142421875	-5.38880428315065e-10\\
29.162919921875	-5.02733743144867e-10\\
29.18341796875	-5.0318319643342e-10\\
29.203916015625	-7.09171124243455e-10\\
29.2244140625	-5.62816490384138e-10\\
29.244912109375	-6.95326669559023e-10\\
29.26541015625	-6.02001790676231e-10\\
29.285908203125	-6.40088422179205e-10\\
29.30640625	-6.01094242019222e-10\\
29.326904296875	-6.48752273807981e-10\\
29.34740234375	-5.87685446518045e-10\\
29.367900390625	-4.67826614634035e-10\\
29.3883984375	-5.18641036261177e-10\\
29.408896484375	-4.97606660247197e-10\\
29.42939453125	-5.62208366948733e-10\\
29.449892578125	-4.50831127278171e-10\\
29.470390625	-5.57835569267453e-10\\
29.490888671875	-3.99021197197635e-10\\
29.51138671875	-4.45410074177046e-10\\
29.531884765625	-4.14227118666679e-10\\
29.5523828125	-4.37480789742352e-10\\
29.572880859375	-3.80804788324935e-10\\
29.59337890625	-4.31834666232907e-10\\
29.613876953125	-4.2351861932268e-10\\
29.634375	-4.35700953216121e-10\\
29.654873046875	-4.11515065175286e-10\\
29.67537109375	-4.00765822559101e-10\\
29.695869140625	-2.85520776814828e-10\\
29.7163671875	-3.50378639546429e-10\\
29.736865234375	-2.4334271848107e-10\\
29.75736328125	-3.44589629537624e-10\\
29.777861328125	-2.91617484758797e-10\\
29.798359375	-3.45010697337018e-10\\
29.818857421875	-2.47179469156737e-10\\
29.83935546875	-3.47265539965208e-10\\
29.859853515625	-1.73249798355864e-10\\
29.8803515625	-2.25971010794976e-10\\
29.900849609375	-1.99281223306573e-10\\
29.92134765625	-1.42416082213018e-10\\
29.941845703125	-1.19585186192439e-10\\
29.96234375	-4.94805480370404e-11\\
29.982841796875	-1.36884862769105e-10\\
30.00333984375	-1.82767408554433e-10\\
30.023837890625	-1.03112883030806e-10\\
30.0443359375	-2.57742095678571e-10\\
30.064833984375	-1.85161580030929e-10\\
30.08533203125	-2.72051868395126e-10\\
30.105830078125	-2.64527922311216e-10\\
30.126328125	-2.40147346066881e-10\\
30.146826171875	-2.42586025586696e-10\\
30.16732421875	-2.22518848743792e-10\\
30.187822265625	-1.21921963685551e-10\\
30.2083203125	-1.59080669444477e-10\\
30.228818359375	-8.46217846560165e-11\\
30.24931640625	-3.93549382757339e-11\\
30.269814453125	-1.12009412691699e-10\\
30.2903125	-3.09348793719503e-11\\
30.310810546875	-1.07822351326696e-10\\
30.33130859375	-1.66069986504409e-10\\
30.351806640625	-1.53646793867655e-10\\
30.3723046875	-1.54394416092008e-10\\
30.392802734375	-1.3130877410302e-10\\
30.41330078125	-1.52479652478292e-10\\
30.433798828125	-1.40076377770657e-10\\
30.454296875	-2.982739145821e-10\\
30.474794921875	-1.63208648648432e-10\\
30.49529296875	-2.62707836781675e-10\\
30.515791015625	-6.42771312612235e-11\\
30.5362890625	-1.61451161765488e-10\\
30.556787109375	-9.5233632529343e-11\\
30.57728515625	-5.29470346655394e-11\\
30.597783203125	3.82421456431067e-12\\
30.61828125	-1.06265209247558e-10\\
30.638779296875	5.92136482068935e-12\\
30.65927734375	-4.24433230199931e-11\\
30.679775390625	-2.47507153670762e-11\\
30.7002734375	9.53012252480302e-11\\
30.720771484375	5.17923477792729e-11\\
30.74126953125	2.21518297736077e-10\\
30.761767578125	9.75962077759815e-11\\
30.782265625	3.11589208701858e-10\\
30.802763671875	1.29012462974519e-10\\
30.82326171875	1.85341490390384e-10\\
30.843759765625	8.14488078481684e-11\\
30.8642578125	1.67346028325367e-10\\
30.884755859375	5.37373922831519e-11\\
30.90525390625	1.54796267897429e-10\\
30.925751953125	1.55469740738575e-10\\
30.94625	3.17648021922327e-10\\
30.966748046875	1.90379931713415e-10\\
30.98724609375	2.56077883398674e-10\\
31.007744140625	1.15014147201182e-10\\
31.0282421875	8.74594391239208e-11\\
31.048740234375	-2.26025176269314e-11\\
31.06923828125	1.42766350924752e-10\\
31.089736328125	6.26355260225331e-12\\
31.110234375	7.91192846065299e-11\\
31.130732421875	-4.30691731891635e-11\\
31.15123046875	-3.36800070269275e-11\\
31.171728515625	-1.15191744961598e-10\\
31.1922265625	2.74209579161823e-11\\
31.212724609375	-4.75506451841368e-11\\
31.23322265625	2.75816297333601e-11\\
31.253720703125	8.68308053790695e-12\\
31.27421875	8.51435243941174e-11\\
31.294716796875	6.14067407875521e-11\\
31.31521484375	9.26036638894266e-11\\
31.335712890625	6.34382921790393e-11\\
31.3562109375	1.09424328764148e-10\\
31.376708984375	2.1769844163799e-11\\
31.39720703125	1.44784251708922e-10\\
31.417705078125	1.23397196609741e-10\\
31.438203125	3.06105056103693e-11\\
31.458701171875	2.34951049586679e-10\\
31.47919921875	3.53229806753545e-11\\
31.499697265625	2.53718361083284e-10\\
31.5201953125	1.70228714630706e-10\\
31.540693359375	2.54611595890826e-10\\
31.56119140625	2.1590455597041e-10\\
31.581689453125	3.45142463354318e-10\\
31.6021875	2.72309425864021e-10\\
31.622685546875	4.62202556013389e-10\\
31.64318359375	5.3106047164328e-10\\
31.663681640625	5.7302426272243e-10\\
31.6841796875	5.75807992424904e-10\\
31.704677734375	5.96201868537907e-10\\
31.72517578125	5.15013441276729e-10\\
31.745673828125	6.89847141221027e-10\\
31.766171875	5.92377514412167e-10\\
31.786669921875	6.11510625595297e-10\\
31.80716796875	5.35706123914996e-10\\
31.827666015625	6.11096945168089e-10\\
31.8481640625	5.33858622131181e-10\\
31.868662109375	5.47676378819754e-10\\
31.88916015625	5.8558200660235e-10\\
31.909658203125	5.55641237924598e-10\\
31.93015625	6.10298845996029e-10\\
31.950654296875	4.42031936755586e-10\\
31.97115234375	5.64945583875764e-10\\
31.991650390625	4.68706956112017e-10\\
32.0121484375	4.73861788767281e-10\\
32.032646484375	4.12855074934825e-10\\
32.05314453125	3.73317135089282e-10\\
32.073642578125	3.97098873206206e-10\\
32.094140625	3.58724108931236e-10\\
32.114638671875	4.10247490132673e-10\\
32.13513671875	4.02333223284284e-10\\
32.155634765625	2.18232750412919e-10\\
32.1761328125	3.14409220459187e-10\\
32.196630859375	1.99727962892463e-10\\
32.21712890625	2.93192150354671e-10\\
32.237626953125	2.76265850577214e-10\\
32.258125	2.95880281517475e-10\\
32.278623046875	3.2308595996106e-10\\
32.29912109375	3.45717202091512e-10\\
32.319619140625	2.47462405564196e-10\\
32.3401171875	4.11792988155024e-10\\
32.360615234375	2.71093869457294e-10\\
32.38111328125	2.51088335116883e-10\\
32.401611328125	1.99018470266217e-10\\
32.422109375	1.18395364868549e-10\\
32.442607421875	1.67821643527461e-10\\
32.46310546875	9.10845030085825e-11\\
32.483603515625	1.75244027927519e-10\\
32.5041015625	1.38681344843986e-10\\
32.524599609375	2.5129644699134e-10\\
32.54509765625	2.47408963351904e-10\\
32.565595703125	2.73428421043083e-10\\
32.58609375	2.97389295055791e-10\\
32.606591796875	2.09151677933261e-10\\
32.62708984375	1.97863045775763e-10\\
32.647587890625	1.29337654594239e-10\\
32.6680859375	2.210904443439e-10\\
32.688583984375	1.00415329764467e-10\\
32.70908203125	2.38415544603744e-10\\
32.729580078125	1.92208559988478e-10\\
32.750078125	2.04292413871421e-10\\
32.770576171875	1.65032820823525e-10\\
32.79107421875	1.74267948436932e-10\\
32.811572265625	1.89390087744219e-10\\
32.8320703125	2.13992101624146e-10\\
32.852568359375	5.84753394089038e-11\\
32.87306640625	1.92031673364939e-10\\
32.893564453125	1.23371441603881e-10\\
32.9140625	1.9404397094593e-10\\
32.934560546875	1.90389856321203e-10\\
32.95505859375	2.31881019942429e-10\\
32.975556640625	1.37974561206182e-10\\
32.9960546875	1.71568277962398e-10\\
33.016552734375	3.3739952675882e-11\\
33.03705078125	1.44792304293392e-10\\
33.057548828125	-2.7095141867654e-11\\
33.078046875	-2.65352239315624e-11\\
33.098544921875	-1.81520411395224e-11\\
33.11904296875	7.92238587199407e-11\\
33.139541015625	2.21198562937465e-11\\
33.1600390625	8.96026652164778e-11\\
33.180537109375	9.3445419113095e-11\\
33.20103515625	6.27121889456822e-11\\
33.221533203125	7.83403150235383e-11\\
33.24203125	2.29777456740658e-12\\
33.262529296875	1.9768016100158e-11\\
33.28302734375	-5.7992407319566e-11\\
33.303525390625	-4.8280208659008e-11\\
33.3240234375	-7.3205397162418e-11\\
33.344521484375	2.73472893238518e-11\\
33.36501953125	-7.21258278607005e-11\\
33.385517578125	5.14636286984561e-11\\
33.406015625	-1.17379867451872e-10\\
33.426513671875	-7.13935339596384e-12\\
33.44701171875	-1.0335059085913e-10\\
33.467509765625	1.92452106659397e-11\\
33.4880078125	-4.95600231332081e-11\\
33.508505859375	5.73552994058749e-11\\
33.52900390625	-2.47561404381811e-11\\
33.549501953125	1.03584065250627e-10\\
33.57	6.72489563252631e-11\\
33.590498046875	8.21236454226368e-11\\
33.61099609375	7.75682424697736e-11\\
33.631494140625	1.01391958445049e-10\\
33.6519921875	3.56654065575956e-11\\
33.672490234375	7.79731000676856e-11\\
33.69298828125	9.92272847316872e-11\\
33.713486328125	1.41097188807241e-10\\
33.733984375	2.16673395326607e-10\\
33.754482421875	1.96989931829566e-10\\
33.77498046875	1.33862722066987e-10\\
33.795478515625	1.63398959776979e-10\\
33.8159765625	3.34627844875944e-11\\
33.836474609375	8.16560818227925e-11\\
33.85697265625	-9.72361287759683e-11\\
33.877470703125	2.09556925747016e-12\\
33.89796875	-3.93682985327977e-11\\
33.918466796875	1.64087478437603e-11\\
33.93896484375	5.82279062920178e-11\\
33.959462890625	2.5018665678567e-11\\
33.9799609375	1.49154992791619e-10\\
34.000458984375	-2.52911373362584e-12\\
34.02095703125	4.4918525331125e-11\\
34.041455078125	-1.91825231465246e-10\\
34.061953125	-1.19571296597105e-10\\
34.082451171875	-1.64432173485596e-10\\
34.10294921875	-2.23340534540366e-10\\
34.123447265625	-2.75987216743197e-10\\
34.1439453125	-1.37901206910193e-10\\
34.164443359375	-1.09946936646463e-10\\
34.18494140625	-5.75528718117355e-11\\
};
\addplot [color=mycolor1,solid]
  table[row sep=crcr]{%
34.18494140625	-5.75528718117355e-11\\
34.205439453125	-7.94595954501198e-11\\
34.2259375	-1.16114236170846e-10\\
34.246435546875	-2.43316791051307e-10\\
34.26693359375	-1.80749794443402e-10\\
34.287431640625	-2.568003064379e-10\\
34.3079296875	-2.46956856162327e-10\\
34.328427734375	-2.71194828169722e-10\\
34.34892578125	-1.9337291343089e-10\\
34.369423828125	-1.55661080873516e-10\\
34.389921875	-1.46820882690956e-10\\
34.410419921875	-1.4959670606709e-11\\
34.43091796875	-9.3290957246054e-11\\
34.451416015625	-7.45309055436923e-12\\
34.4719140625	-9.69797185258352e-11\\
34.492412109375	3.48463361085069e-12\\
34.51291015625	-1.28405848338154e-10\\
34.533408203125	-4.24913357112896e-11\\
34.55390625	-5.68871603505517e-11\\
34.574404296875	-9.29255879702043e-11\\
34.59490234375	5.4127782024235e-12\\
34.615400390625	-6.09250862523116e-12\\
34.6358984375	-1.23109406576595e-10\\
34.656396484375	-8.10408282055923e-11\\
34.67689453125	-4.17180549839348e-11\\
34.697392578125	3.86895598602346e-11\\
34.717890625	-6.59560112065399e-11\\
34.738388671875	8.51644907277833e-12\\
34.75888671875	3.37098109784111e-11\\
34.779384765625	8.71827218060544e-11\\
34.7998828125	-1.50366897523329e-11\\
34.820380859375	1.36769612248074e-10\\
34.84087890625	6.59625422160468e-11\\
34.861376953125	1.09808878744067e-10\\
34.881875	2.98024179391692e-12\\
34.902373046875	7.54448577285468e-11\\
34.92287109375	1.39519549526192e-11\\
34.943369140625	3.9916123348057e-11\\
34.9638671875	7.10238661445388e-11\\
34.984365234375	7.29544016367381e-11\\
35.00486328125	1.72749225190608e-10\\
35.025361328125	1.32197742802957e-10\\
35.045859375	2.68049456921444e-10\\
35.066357421875	2.64719541312855e-10\\
35.08685546875	2.44514951271649e-10\\
35.107353515625	2.71402969885012e-10\\
35.1278515625	2.25635718653754e-10\\
35.148349609375	3.111792676234e-10\\
35.16884765625	2.12113334152702e-10\\
35.189345703125	2.09997557460303e-10\\
35.20984375	1.4174080729881e-10\\
35.230341796875	1.903518925095e-10\\
35.25083984375	1.51312457528215e-10\\
35.271337890625	2.77601083918911e-10\\
35.2918359375	2.36968526769388e-10\\
35.312333984375	3.6915595684769e-10\\
35.33283203125	3.05701637946576e-10\\
35.353330078125	3.42222930995506e-10\\
35.373828125	2.75798778218699e-10\\
35.394326171875	2.95429419711242e-10\\
35.41482421875	2.51517879322627e-10\\
35.435322265625	2.85888114823244e-10\\
35.4558203125	2.07289277892886e-10\\
35.476318359375	2.30206988192524e-10\\
35.49681640625	2.5230187400934e-10\\
35.517314453125	2.88169531917897e-10\\
35.5378125	2.94137992243473e-10\\
35.558310546875	3.91245398203555e-10\\
35.57880859375	2.96761063489001e-10\\
35.599306640625	3.55944124657468e-10\\
35.6198046875	1.99237487087367e-10\\
35.640302734375	1.82190531456171e-10\\
35.66080078125	1.3494794321016e-10\\
35.681298828125	1.71516246555746e-10\\
35.701796875	1.66182822158228e-10\\
35.722294921875	1.75043255216663e-10\\
35.74279296875	2.75220727981922e-10\\
35.763291015625	3.10952592502406e-10\\
35.7837890625	3.67105774950702e-10\\
35.804287109375	3.71801794974358e-10\\
35.82478515625	4.50732034813216e-10\\
35.845283203125	3.02119705518229e-10\\
35.86578125	3.66085719867859e-10\\
35.886279296875	2.97204221872217e-10\\
35.90677734375	3.03231356290971e-10\\
35.927275390625	2.30291142949426e-10\\
35.9477734375	3.48874542228629e-10\\
35.968271484375	3.59287071872935e-10\\
35.98876953125	5.1275333174397e-10\\
36.009267578125	3.90324805880207e-10\\
36.029765625	5.42663092818397e-10\\
36.050263671875	4.51078636707967e-10\\
36.07076171875	4.04937523239895e-10\\
36.091259765625	3.61340892763098e-10\\
36.1117578125	3.02256433362687e-10\\
36.132255859375	2.48726474363359e-10\\
36.15275390625	3.0728630343768e-10\\
36.173251953125	2.11422380229301e-10\\
36.19375	2.9825752661552e-10\\
36.214248046875	2.55182831384374e-10\\
36.23474609375	3.44983509699293e-10\\
36.255244140625	3.14222213715138e-10\\
36.2757421875	3.59905598215179e-10\\
36.296240234375	3.51633584928762e-10\\
36.31673828125	4.56204360521753e-10\\
36.337236328125	3.07400667146735e-10\\
36.357734375	3.91751472857097e-10\\
36.378232421875	2.56496491165797e-10\\
36.39873046875	2.92967068531814e-10\\
36.419228515625	3.0456532282227e-10\\
36.4397265625	3.40185073217493e-10\\
36.460224609375	3.71825644891912e-10\\
36.48072265625	2.85395432295444e-10\\
36.501220703125	4.61747791698396e-10\\
36.52171875	3.7598687459387e-10\\
36.542216796875	4.87319889643525e-10\\
36.56271484375	4.24459782113819e-10\\
36.583212890625	4.07022871453142e-10\\
36.6037109375	2.621587358893e-10\\
36.624208984375	2.93486086752745e-10\\
36.64470703125	2.03043702185673e-10\\
36.665205078125	2.58861235786703e-10\\
36.685703125	3.32536846651329e-10\\
36.706201171875	3.49037250941555e-10\\
36.72669921875	3.87942926938378e-10\\
36.747197265625	4.59289103983153e-10\\
36.7676953125	4.61282979015765e-10\\
36.788193359375	5.10081834885508e-10\\
36.80869140625	4.18956851521151e-10\\
36.829189453125	4.29126282881815e-10\\
36.8496875	2.65322093187213e-10\\
36.870185546875	2.46116363808319e-10\\
36.89068359375	2.181848739282e-10\\
36.911181640625	2.51066179842794e-10\\
36.9316796875	3.6989116915771e-10\\
36.952177734375	3.07349261381518e-10\\
36.97267578125	4.1918452235743e-10\\
36.993173828125	2.93960653250167e-10\\
37.013671875	3.78761208993687e-10\\
37.034169921875	4.01896682283123e-10\\
37.05466796875	4.26686330814523e-10\\
37.075166015625	3.23528675630947e-10\\
37.0956640625	3.19582323848383e-10\\
37.116162109375	2.88378306121318e-10\\
37.13666015625	3.86159479802804e-10\\
37.157158203125	4.44851897474582e-10\\
37.17765625	4.3901282311953e-10\\
37.198154296875	3.70571903746166e-10\\
37.21865234375	4.0330684273163e-10\\
37.239150390625	2.84157553456814e-10\\
37.2596484375	2.42475257671921e-10\\
37.280146484375	2.22415563679216e-10\\
37.30064453125	2.39789534278568e-10\\
37.321142578125	1.84994253950997e-10\\
37.341640625	1.83795661017228e-10\\
37.362138671875	1.93222594433965e-10\\
37.38263671875	2.98475542643874e-10\\
37.403134765625	2.9763681049297e-10\\
37.4236328125	2.42921010522519e-10\\
37.444130859375	1.69363307855681e-10\\
37.46462890625	1.02589740026541e-10\\
37.485126953125	7.39352609800006e-11\\
37.505625	-2.69058341783165e-11\\
37.526123046875	1.64062831039552e-11\\
37.54662109375	-5.11483893903014e-11\\
37.567119140625	-1.8522511643244e-11\\
37.5876171875	7.78650095010815e-11\\
37.608115234375	-2.65868171878966e-12\\
37.62861328125	1.93416882548831e-10\\
37.649111328125	2.03880965629743e-11\\
37.669609375	1.15449938307525e-10\\
37.690107421875	7.2737581962869e-11\\
37.71060546875	1.781022196002e-11\\
37.731103515625	-3.3838485689956e-11\\
37.7516015625	-4.63066849208536e-11\\
37.772099609375	-5.35246944196631e-11\\
37.79259765625	-6.21334552945036e-11\\
37.813095703125	-1.19466587305717e-10\\
37.83359375	2.71425205868077e-11\\
37.854091796875	-8.01839055980398e-11\\
37.87458984375	-3.72485209787321e-11\\
37.895087890625	-1.10127249317621e-10\\
37.9155859375	-1.59021594132147e-11\\
37.936083984375	-3.94806338863406e-11\\
37.95658203125	2.11166944637044e-12\\
37.977080078125	-2.08587214429764e-11\\
37.997578125	-8.54399554851142e-11\\
38.018076171875	-5.50956425442025e-11\\
38.03857421875	-9.20896084691012e-11\\
38.059072265625	-1.65558293755356e-10\\
38.0795703125	-6.00464307781486e-11\\
38.100068359375	-2.07143752702539e-10\\
38.12056640625	-1.98363493765444e-10\\
38.141064453125	-1.73060923086141e-10\\
38.1615625	-1.55886981869252e-10\\
38.182060546875	-2.34744714629373e-10\\
38.20255859375	-1.52454184053102e-10\\
38.223056640625	-2.21241601158481e-10\\
38.2435546875	-2.25228868502054e-10\\
38.264052734375	-3.06189402734427e-10\\
38.28455078125	-3.26349649659669e-10\\
38.305048828125	-2.21768875123163e-10\\
38.325546875	-4.16921867389866e-10\\
38.346044921875	-2.02078712268163e-10\\
38.36654296875	-3.00413238426928e-10\\
38.387041015625	-2.67865418217704e-10\\
38.4075390625	-3.05104429528278e-10\\
38.428037109375	-2.21697099084461e-10\\
38.44853515625	-3.95519744018352e-10\\
38.469033203125	-3.27977387031333e-10\\
38.48953125	-5.21396747284785e-10\\
38.510029296875	-3.27186380436657e-10\\
38.53052734375	-4.78551890086371e-10\\
38.551025390625	-3.68825710024047e-10\\
38.5715234375	-3.16972320524783e-10\\
38.592021484375	-2.54364953782064e-10\\
38.61251953125	-3.00027433906954e-10\\
38.633017578125	-1.83884280615012e-10\\
38.653515625	-1.39556862489158e-10\\
38.674013671875	-1.2737031447026e-10\\
38.69451171875	-2.45587587831663e-10\\
38.715009765625	-1.68438022956535e-10\\
38.7355078125	-3.00993679807013e-10\\
38.756005859375	-1.64208602260338e-10\\
38.77650390625	-2.03823983317234e-10\\
38.797001953125	-1.58629262452794e-10\\
38.8175	-2.90753788575253e-10\\
38.837998046875	-2.19869230593302e-10\\
38.85849609375	-2.4941658828885e-10\\
38.878994140625	-1.97299878245399e-10\\
38.8994921875	-2.34075040298405e-10\\
38.919990234375	-2.31699152051691e-10\\
38.94048828125	-2.58331098970876e-10\\
38.960986328125	-2.64401139350374e-10\\
38.981484375	-2.66336246827307e-10\\
39.001982421875	-3.2364944546868e-10\\
39.02248046875	-2.46078773073999e-10\\
39.042978515625	-2.72340318790867e-10\\
39.0634765625	-2.49300456969992e-10\\
39.083974609375	-2.4899858846187e-10\\
39.10447265625	-2.00607327233854e-10\\
39.124970703125	-3.07335035690612e-10\\
39.14546875	-1.18305296362446e-10\\
39.165966796875	-2.48559420444406e-10\\
39.18646484375	-2.17015381798768e-10\\
39.206962890625	-2.64213667001958e-10\\
39.2274609375	-3.13195870010127e-10\\
39.247958984375	-3.08727087110463e-10\\
39.26845703125	-3.404416205022e-10\\
39.288955078125	-3.14794734046674e-10\\
39.309453125	-2.60895980321553e-10\\
39.329951171875	-2.41264240326456e-10\\
39.35044921875	-2.2135730648205e-10\\
39.370947265625	-2.10525177975199e-10\\
39.3914453125	-2.24490255455845e-10\\
39.411943359375	-2.02320305143254e-10\\
39.43244140625	-2.11122537076989e-10\\
39.452939453125	-1.95230768066167e-10\\
39.4734375	-2.09344016287526e-10\\
39.493935546875	-2.19090692929805e-10\\
39.51443359375	-3.09612948752083e-10\\
39.534931640625	-2.68051133650146e-10\\
39.5554296875	-2.93237674816043e-10\\
39.575927734375	-2.27253751150369e-10\\
39.59642578125	-2.74706422693794e-10\\
39.616923828125	-2.56416768237969e-10\\
39.637421875	-1.71108111116938e-10\\
39.657919921875	-2.38457599857638e-10\\
39.67841796875	-2.15315280149187e-10\\
39.698916015625	-1.22028706150574e-10\\
39.7194140625	-8.28457859164207e-11\\
39.739912109375	-1.20752192053312e-10\\
39.76041015625	-1.03219749218749e-10\\
39.780908203125	-1.04910753242875e-10\\
39.80140625	-1.33529778927626e-10\\
39.821904296875	-1.73985160701449e-10\\
39.84240234375	-2.68572498203055e-10\\
39.862900390625	-2.21457093054602e-10\\
39.8833984375	-2.70363966306792e-10\\
39.903896484375	-2.25563860010637e-10\\
39.92439453125	-1.09520592396739e-10\\
39.944892578125	-1.17399262194743e-10\\
39.965390625	-2.64018257481734e-12\\
39.985888671875	-3.23338500141077e-11\\
40.00638671875	4.02283113715792e-11\\
40.026884765625	-1.56027603829199e-11\\
40.0473828125	2.04348157926766e-11\\
40.067880859375	-1.27221500109594e-10\\
40.08837890625	-1.58854917664472e-10\\
40.108876953125	-1.08432697554265e-10\\
40.129375	-2.05321327583689e-10\\
40.149873046875	-5.8326888311176e-11\\
40.17037109375	-1.4311125414669e-10\\
40.190869140625	-6.1708718700139e-11\\
40.2113671875	-8.52815774073841e-12\\
40.231865234375	5.12785845311222e-11\\
40.25236328125	3.79454036311059e-11\\
40.272861328125	5.25986685899278e-11\\
40.293359375	-1.44605677981581e-10\\
40.313857421875	-2.27186001821796e-11\\
40.33435546875	-1.49661049031874e-10\\
40.354853515625	-8.86367943182108e-11\\
40.3753515625	-8.50945134390957e-13\\
40.395849609375	2.3076487545447e-11\\
40.41634765625	3.53006386342834e-11\\
40.436845703125	4.41352895759444e-11\\
40.45734375	6.31470989756102e-11\\
40.477841796875	8.27216105743207e-11\\
40.49833984375	5.21965150260106e-11\\
40.518837890625	9.90660951882899e-11\\
40.5393359375	3.69914763929722e-11\\
40.559833984375	5.65885469198262e-11\\
40.58033203125	3.2932353675376e-11\\
40.600830078125	1.43213202314541e-10\\
40.621328125	1.25320094453705e-10\\
40.641826171875	2.28118100928028e-10\\
40.66232421875	1.4928065791689e-10\\
40.682822265625	1.91672569426574e-10\\
40.7033203125	1.67934728682363e-10\\
40.723818359375	2.21086912309565e-10\\
40.74431640625	1.35133810356104e-10\\
40.764814453125	1.36076123729355e-10\\
40.7853125	2.09404425254113e-10\\
40.805810546875	1.46963106452812e-10\\
40.82630859375	2.64378264334795e-10\\
40.846806640625	1.37991770651998e-10\\
40.8673046875	2.15986678274683e-10\\
40.887802734375	1.48479063959434e-10\\
40.90830078125	2.49919558742893e-10\\
40.928798828125	1.29024217527343e-10\\
40.949296875	1.80880632663323e-10\\
40.969794921875	1.10878431579851e-10\\
40.99029296875	2.53833564587508e-10\\
41.010791015625	9.33859055833159e-11\\
41.0312890625	1.81486999667618e-10\\
41.051787109375	3.57493005756196e-11\\
41.07228515625	1.13052420813618e-10\\
41.092783203125	3.98502926420846e-11\\
41.11328125	1.19391017309399e-10\\
41.133779296875	1.58152830450607e-11\\
41.15427734375	2.24464026980912e-11\\
41.174775390625	-5.69257510855491e-12\\
41.1952734375	5.04859425705175e-11\\
41.215771484375	5.82330400675384e-11\\
41.23626953125	6.20388135950554e-11\\
41.256767578125	-1.07925732307795e-11\\
41.277265625	5.92231096722633e-11\\
41.297763671875	2.49582461060007e-11\\
41.31826171875	4.54725557054591e-11\\
41.338759765625	2.72023077753465e-11\\
41.3592578125	7.99055800925978e-11\\
41.379755859375	6.48181456838102e-11\\
41.40025390625	1.00050283326704e-10\\
41.420751953125	7.70234059497738e-11\\
41.44125	1.73332180111676e-10\\
41.461748046875	1.77110199264742e-10\\
41.48224609375	1.07259249528347e-10\\
41.502744140625	1.07322058261734e-10\\
41.5232421875	-3.44291869699968e-11\\
41.543740234375	3.99820352520186e-11\\
41.56423828125	-4.58637686133686e-11\\
41.584736328125	3.93314594235375e-11\\
41.605234375	-4.6104718727626e-11\\
41.625732421875	1.57095534583271e-10\\
41.64623046875	1.13415696991712e-10\\
41.666728515625	2.68940990158631e-10\\
41.6872265625	2.21103440375273e-10\\
41.707724609375	2.37802030268502e-10\\
41.72822265625	2.56094857072866e-10\\
41.748720703125	2.06639050340586e-10\\
41.76921875	1.92572292095281e-10\\
41.789716796875	9.46658317109378e-11\\
41.81021484375	8.36134384423502e-11\\
41.830712890625	1.07651889502289e-10\\
41.8512109375	1.91927852657393e-10\\
41.871708984375	1.75349594161567e-10\\
41.89220703125	1.82894890858852e-10\\
41.912705078125	2.03261583938018e-10\\
41.933203125	1.69756878752642e-10\\
41.953701171875	2.04242513306133e-10\\
41.97419921875	2.00750334971431e-10\\
41.994697265625	2.12273449233484e-10\\
42.0151953125	2.1361725578981e-10\\
42.035693359375	1.13094027929213e-10\\
42.05619140625	2.11723928600683e-10\\
42.076689453125	1.91409090949089e-10\\
42.0971875	2.4491794285293e-10\\
42.117685546875	2.07922381830586e-10\\
42.13818359375	1.2284194810306e-10\\
42.158681640625	1.51814602865909e-10\\
42.1791796875	1.69744975322956e-10\\
42.199677734375	1.55352943560315e-10\\
42.22017578125	7.09123665495687e-11\\
42.240673828125	2.0130163370592e-10\\
42.261171875	1.93533538312532e-10\\
42.281669921875	1.51580080820811e-10\\
42.30216796875	1.42988396680025e-10\\
42.322666015625	9.05219692425345e-11\\
42.3431640625	1.50627223965274e-10\\
42.363662109375	8.44029338446272e-11\\
42.38416015625	1.24770866973191e-10\\
42.404658203125	8.6153592290081e-11\\
42.42515625	1.07060898762442e-10\\
42.445654296875	-1.75886783295593e-11\\
42.46615234375	-1.88706052684791e-11\\
42.486650390625	-1.34057550730336e-11\\
42.5071484375	-1.54928102750331e-11\\
42.527646484375	-1.0027003606642e-11\\
42.54814453125	4.47040435657207e-11\\
42.568642578125	-2.82468715967454e-11\\
42.589140625	7.13624427380281e-11\\
42.609638671875	2.68541172129101e-11\\
42.63013671875	7.83310253000295e-11\\
42.650634765625	-1.11366172245217e-11\\
42.6711328125	7.28972701189251e-11\\
42.691630859375	-9.57169885398313e-11\\
42.71212890625	2.13706537143304e-11\\
42.732626953125	2.37268277676651e-13\\
42.753125	2.0682528315207e-11\\
42.773623046875	-2.36469643769145e-11\\
42.79412109375	-2.32282758492476e-11\\
42.814619140625	3.24367019840797e-11\\
42.8351171875	9.2766746091796e-11\\
42.855615234375	4.19179226301644e-11\\
42.87611328125	5.48091329069413e-11\\
42.896611328125	2.64872903102251e-11\\
42.917109375	-3.33543312371333e-11\\
42.937607421875	-4.11276750628748e-11\\
42.95810546875	-1.07984701479185e-10\\
42.978603515625	-1.87738079439491e-11\\
42.9991015625	-1.18230980884099e-10\\
43.019599609375	-1.12364042987377e-10\\
43.04009765625	-5.18022763284937e-11\\
43.060595703125	-9.1941850388296e-11\\
43.08109375	-1.48069070978554e-10\\
43.101591796875	-7.27197565912137e-11\\
43.12208984375	-6.01601916868838e-11\\
43.142587890625	-8.18764546785101e-11\\
43.1630859375	-1.62013360539372e-10\\
43.183583984375	-1.23946415925336e-10\\
43.20408203125	-1.78898615502626e-10\\
43.224580078125	-2.42794529309496e-10\\
43.245078125	-2.01407009888969e-10\\
43.265576171875	-2.10612956320305e-10\\
43.28607421875	-2.26786014293516e-10\\
43.306572265625	-2.2458535858278e-10\\
43.3270703125	-2.25364557624465e-10\\
43.347568359375	-1.50803715508732e-10\\
43.36806640625	-2.1192535939558e-10\\
43.388564453125	-1.01389829259009e-10\\
43.4090625	-1.81366746334468e-10\\
43.429560546875	-1.88804323586093e-10\\
43.45005859375	-2.43188607269233e-10\\
43.470556640625	-1.66090091917593e-10\\
43.4910546875	-2.97672410216324e-10\\
43.511552734375	-2.22420100799454e-10\\
43.53205078125	-2.8296167039731e-10\\
43.552548828125	-1.41655638563882e-10\\
43.573046875	-1.51911026910157e-10\\
43.593544921875	-9.92370343364719e-11\\
43.61404296875	-1.42639014354033e-10\\
43.634541015625	-3.66213202617136e-11\\
43.6550390625	-8.36725631543692e-11\\
43.675537109375	-8.23105999214769e-11\\
43.69603515625	-1.3498356231826e-10\\
43.716533203125	-1.55859209468832e-10\\
43.73703125	-1.8614871695906e-10\\
43.757529296875	-1.16206258367268e-10\\
43.77802734375	-1.98321637751626e-10\\
43.798525390625	-2.59210588633455e-11\\
43.8190234375	-1.00801712969644e-10\\
43.839521484375	-1.01593337357822e-10\\
43.86001953125	-1.15761272391411e-10\\
43.880517578125	-1.26060436682732e-10\\
43.901015625	-1.20589473832129e-10\\
43.921513671875	-1.63426460180193e-10\\
43.94201171875	-2.511180138719e-10\\
43.962509765625	-1.89270664395066e-10\\
43.9830078125	-1.01161707307371e-10\\
44.003505859375	-1.7365202095265e-10\\
44.02400390625	-1.0761455733353e-11\\
44.044501953125	-1.00700288184363e-10\\
44.065	1.07507053784045e-11\\
44.085498046875	-1.088124419816e-10\\
44.10599609375	-7.11602963086385e-11\\
44.126494140625	-2.39885063545035e-10\\
44.1469921875	-1.46418058507809e-10\\
44.167490234375	-2.27409983220743e-10\\
44.18798828125	-1.95977271892788e-10\\
44.208486328125	-1.75024255829458e-10\\
44.228984375	-2.21248990860916e-10\\
44.249482421875	-2.18078855306915e-10\\
44.26998046875	-1.77725235950268e-10\\
44.290478515625	-1.87437219120034e-10\\
44.3109765625	-1.44869149924616e-10\\
44.331474609375	-1.69139565224365e-10\\
44.35197265625	-1.77816222093239e-10\\
44.372470703125	-2.20829040438948e-10\\
44.39296875	-1.24317027362143e-10\\
44.413466796875	-1.59860546801227e-10\\
44.43396484375	-1.44082168138044e-10\\
44.454462890625	-1.92506453138776e-10\\
44.4749609375	-1.96725315320477e-10\\
44.495458984375	-2.27677223100733e-10\\
44.51595703125	-1.57462049104113e-10\\
44.536455078125	-2.12028025712068e-10\\
44.556953125	-2.21708116781584e-10\\
44.577451171875	-2.24475978500457e-10\\
44.59794921875	-2.03715442582787e-10\\
44.618447265625	-1.48761864852574e-10\\
44.6389453125	-1.3430380382003e-10\\
44.659443359375	-6.14118216023796e-11\\
44.67994140625	-5.52610063530331e-11\\
44.700439453125	-6.77230125695716e-11\\
44.7209375	-1.258474734523e-10\\
44.741435546875	-3.45525392409377e-11\\
44.76193359375	-1.45756199074149e-10\\
44.782431640625	-1.18288456709395e-10\\
44.8029296875	-1.49048190737578e-10\\
44.823427734375	1.93258604992159e-11\\
44.84392578125	-1.17312868762546e-10\\
44.864423828125	8.67200267530642e-12\\
44.884921875	-4.88585329305544e-11\\
44.905419921875	4.49822939402611e-11\\
44.92591796875	-4.04446742038492e-11\\
44.946416015625	3.9694959667268e-11\\
44.9669140625	-2.25757536193604e-11\\
44.987412109375	-6.81627951153753e-11\\
45.00791015625	6.50941517459086e-11\\
45.028408203125	1.6868831523054e-11\\
45.04890625	3.77113671742438e-11\\
45.069404296875	6.99710234650657e-11\\
45.08990234375	6.26673555768172e-11\\
45.110400390625	7.58840498455331e-11\\
45.1308984375	6.63412407758568e-11\\
45.151396484375	1.1799476139586e-10\\
45.17189453125	4.15046275973373e-11\\
45.192392578125	3.46384698292102e-11\\
45.212890625	3.24687123303396e-12\\
45.233388671875	7.62293691491699e-11\\
45.25388671875	-5.444391008877e-11\\
45.274384765625	5.59935256767759e-11\\
45.2948828125	-4.33506545941564e-11\\
45.315380859375	-2.62504360459473e-11\\
45.33587890625	-7.29474259008547e-11\\
45.356376953125	6.57405659788638e-12\\
45.376875	-1.67527932084991e-11\\
45.397373046875	4.41040765906563e-11\\
45.41787109375	3.75284827247029e-11\\
45.438369140625	5.99953689722792e-11\\
45.4588671875	1.44745880720734e-10\\
45.479365234375	1.09308919761792e-10\\
45.49986328125	1.19044783181477e-10\\
45.520361328125	1.0136921911058e-10\\
45.540859375	9.35637047205903e-11\\
45.561357421875	1.13609485476057e-10\\
45.58185546875	9.20671333541664e-11\\
45.602353515625	3.9192959764787e-11\\
45.6228515625	1.26074521760124e-10\\
45.643349609375	1.14373450094829e-10\\
45.66384765625	1.52887733498842e-10\\
45.684345703125	1.95044067561528e-10\\
45.70484375	2.12413622002681e-10\\
45.725341796875	2.20044629566342e-10\\
45.74583984375	2.40942630276033e-10\\
45.766337890625	2.07627765208865e-10\\
45.7868359375	2.24096696961749e-10\\
45.807333984375	1.56261194286495e-10\\
45.82783203125	1.62449763334991e-10\\
45.848330078125	8.82827841443662e-11\\
45.868828125	1.6053380637045e-10\\
45.889326171875	5.29755159521891e-11\\
45.90982421875	1.24591265703681e-10\\
45.930322265625	1.31709760395018e-10\\
45.9508203125	2.09031901351951e-10\\
45.971318359375	1.48685169455766e-10\\
45.99181640625	2.47315767509595e-10\\
46.012314453125	1.21599303792872e-10\\
46.0328125	2.4009803365074e-10\\
46.053310546875	5.18673342799185e-11\\
46.07380859375	1.53579077276792e-10\\
46.094306640625	5.47421435279796e-11\\
46.1148046875	8.06310819213412e-11\\
46.135302734375	5.41021831304074e-11\\
46.15580078125	1.18106433669836e-10\\
46.176298828125	7.1345878647265e-11\\
46.196796875	-2.73433897862147e-11\\
46.217294921875	6.03446560122073e-11\\
46.23779296875	8.13431787476145e-11\\
46.258291015625	1.15894295046302e-11\\
46.2787890625	7.39830137011755e-11\\
46.299287109375	-8.19612040790132e-12\\
46.31978515625	-6.44222655937907e-11\\
46.340283203125	3.54188232309531e-11\\
46.36078125	2.87853699932361e-11\\
46.381279296875	4.07455193112153e-11\\
46.40177734375	7.56426131728744e-11\\
46.422275390625	9.61625092290348e-11\\
46.4427734375	1.17492729300615e-10\\
46.463271484375	5.19521201257342e-11\\
46.48376953125	2.90819168736619e-11\\
46.504267578125	6.53500547395156e-11\\
46.524765625	2.79688140054667e-12\\
46.545263671875	5.91114296896531e-11\\
46.56576171875	-2.45126367944189e-11\\
46.586259765625	7.38813801641197e-11\\
46.6067578125	1.29748914796972e-11\\
46.627255859375	1.20166012099097e-10\\
46.64775390625	1.51764738896772e-11\\
46.668251953125	5.94972623196368e-11\\
46.68875	5.01698751172192e-11\\
46.709248046875	4.64791923651639e-11\\
46.72974609375	8.91786270752109e-11\\
46.750244140625	1.57648444594235e-11\\
46.7707421875	6.1848541036106e-11\\
46.791240234375	5.37372914789388e-11\\
46.81173828125	8.27822820284155e-11\\
46.832236328125	8.82868385787825e-11\\
46.852734375	5.40746197799336e-11\\
46.873232421875	9.73318146098264e-11\\
46.89373046875	7.33433441182103e-11\\
46.914228515625	5.82031382943679e-11\\
46.9347265625	1.28978638505283e-10\\
46.955224609375	1.13832446935874e-10\\
46.97572265625	1.58412787166458e-10\\
46.996220703125	1.52134782956638e-10\\
47.01671875	1.44058887450216e-10\\
47.037216796875	1.50610951321869e-10\\
47.05771484375	1.35527213974552e-10\\
47.078212890625	5.79351425567104e-11\\
47.0987109375	8.09440694103061e-11\\
47.119208984375	5.83370454036261e-11\\
47.13970703125	2.29703358855315e-11\\
47.160205078125	2.70806260061599e-11\\
47.180703125	-5.16211673336684e-12\\
47.201201171875	1.19269658175665e-11\\
47.22169921875	3.04689866675484e-11\\
47.242197265625	-1.14395106967426e-12\\
47.2626953125	1.72961508532174e-11\\
47.283193359375	1.53647087903169e-11\\
47.30369140625	1.58880144621969e-11\\
47.324189453125	-1.021868215252e-11\\
47.3446875	5.60913774187311e-11\\
47.365185546875	-6.74772137649656e-11\\
47.38568359375	9.70691889645131e-11\\
47.406181640625	-2.55959265142068e-11\\
47.4266796875	5.81775040346921e-11\\
47.447177734375	-1.62545086665897e-11\\
47.46767578125	3.89466679226188e-12\\
47.488173828125	1.31845746305911e-11\\
47.508671875	2.46392918132466e-11\\
47.529169921875	9.32628078848289e-12\\
47.54966796875	-1.48041802971843e-10\\
47.570166015625	-6.72579475285726e-11\\
47.5906640625	-1.30595322125554e-10\\
47.611162109375	-4.8723196987512e-11\\
47.63166015625	-3.49593220287587e-11\\
47.652158203125	-3.39325871606457e-11\\
47.67265625	-3.18917086736648e-11\\
47.693154296875	-1.06051484580709e-10\\
47.71365234375	-8.65006661361629e-11\\
47.734150390625	-1.32016114223435e-10\\
47.7546484375	-9.5899747259138e-11\\
47.775146484375	-8.97982813613813e-11\\
47.79564453125	-4.84255685080826e-11\\
47.816142578125	-4.34787239264622e-11\\
47.836640625	5.85479595750902e-12\\
47.857138671875	-1.11396771332115e-11\\
47.87763671875	1.87507153091103e-11\\
47.898134765625	-2.80877414847357e-11\\
47.9186328125	-3.83880148393069e-11\\
47.939130859375	-1.34878691975264e-10\\
47.95962890625	-7.45694364539609e-11\\
47.980126953125	-1.27668832122586e-10\\
48.000625	-1.84797779464671e-10\\
48.021123046875	-7.72135028937574e-11\\
48.04162109375	-1.09848186733697e-10\\
48.062119140625	-1.32837455288698e-10\\
48.0826171875	-1.12304370252109e-10\\
48.103115234375	-9.50253878656772e-11\\
48.12361328125	-9.15728989920973e-11\\
48.144111328125	-1.7561922765424e-10\\
48.164609375	-1.02386571436739e-10\\
48.185107421875	-1.80722639627642e-10\\
48.20560546875	-1.17480813810612e-10\\
48.226103515625	-1.27476975525618e-10\\
48.2466015625	-1.41194886885393e-10\\
48.267099609375	-1.99808583992973e-10\\
48.28759765625	-1.85023862153535e-10\\
48.308095703125	-1.51090929722363e-10\\
48.32859375	-1.65654641056142e-10\\
48.349091796875	-1.3087358661126e-10\\
48.36958984375	-1.5678209849357e-10\\
48.390087890625	-8.77425147814064e-11\\
48.4105859375	-1.11549765783423e-10\\
48.431083984375	-7.75707183977135e-11\\
48.45158203125	-1.09394418789872e-10\\
48.472080078125	-1.23231757230516e-10\\
48.492578125	-1.81315979166508e-10\\
48.513076171875	-7.75340562985404e-11\\
48.53357421875	-2.52948144989044e-10\\
48.554072265625	-4.51526314741281e-11\\
48.5745703125	-1.20633759450784e-10\\
48.595068359375	-5.1806272934429e-11\\
48.61556640625	-2.87611626499309e-11\\
48.636064453125	-5.52393577477823e-12\\
48.6565625	-5.35952481007516e-11\\
48.677060546875	2.2539697686976e-11\\
48.69755859375	-1.25142224587832e-12\\
48.718056640625	-1.87779156521568e-11\\
48.7385546875	4.6079640879458e-11\\
48.759052734375	-8.90884138560423e-11\\
48.77955078125	-7.51716758818728e-11\\
48.800048828125	-4.62021924086604e-11\\
48.820546875	-4.05137746403853e-11\\
48.841044921875	-2.11374320070119e-11\\
48.86154296875	2.64601061122954e-11\\
48.882041015625	-7.74631712164225e-11\\
48.9025390625	6.01617545349186e-11\\
48.923037109375	-2.70109611304176e-11\\
48.94353515625	-9.15342996511684e-11\\
48.964033203125	-7.64573618664986e-11\\
48.98453125	-2.27854114509722e-11\\
49.005029296875	-2.09500537072947e-11\\
49.02552734375	-3.12155411885293e-11\\
49.046025390625	-1.68373022862273e-10\\
49.0665234375	-4.55084932124417e-11\\
49.087021484375	-1.01422457848565e-10\\
49.10751953125	-3.2687693501027e-11\\
49.128017578125	-2.3131293654721e-11\\
49.148515625	-8.26878174637514e-12\\
49.169013671875	2.2579079020196e-11\\
49.18951171875	2.8572778650763e-11\\
49.210009765625	-3.14652188302493e-11\\
49.2305078125	-2.90241681027188e-11\\
49.251005859375	-1.11838098837176e-10\\
49.27150390625	-1.03982805382486e-10\\
49.292001953125	-9.47640934390056e-11\\
49.3125	-1.17056816320812e-10\\
49.332998046875	-7.72570687396058e-11\\
49.35349609375	-4.57157921355264e-11\\
49.373994140625	2.15865699243711e-11\\
49.3944921875	-2.16239735240783e-11\\
49.414990234375	-2.47831047806832e-11\\
49.43548828125	-3.82873319227194e-11\\
49.455986328125	-8.91322061010097e-11\\
49.476484375	-1.04037114266954e-10\\
49.496982421875	-5.05095394179729e-11\\
49.51748046875	-1.46827165152468e-10\\
49.537978515625	-9.30927455608045e-11\\
49.5584765625	-1.03696384706451e-10\\
49.578974609375	-3.84801431883138e-11\\
49.59947265625	-6.23919685800623e-11\\
49.619970703125	-3.86775729149781e-11\\
49.64046875	-6.68267905048368e-11\\
49.660966796875	-1.04471256512985e-11\\
49.68146484375	-9.13544671167737e-12\\
49.701962890625	-9.71022612007971e-11\\
49.7224609375	-7.80162659329991e-11\\
49.742958984375	-5.66190333503416e-11\\
49.76345703125	-1.25214182031499e-10\\
49.783955078125	-7.02276627902827e-11\\
49.804453125	-1.53116067559475e-10\\
49.824951171875	-9.42579395699554e-11\\
49.84544921875	-4.6157970616106e-11\\
49.865947265625	-2.33481399426793e-11\\
49.8864453125	-7.01894613540986e-11\\
49.906943359375	-2.06531911403914e-11\\
49.92744140625	-6.89454693301918e-11\\
49.947939453125	-8.39786072175351e-11\\
49.9684375	-1.1605530615677e-10\\
49.988935546875	-8.62218054489529e-11\\
50.00943359375	-9.91791788071079e-11\\
50.029931640625	-1.34129737254466e-10\\
50.0504296875	-1.26641918088989e-11\\
50.070927734375	-4.03022896844865e-11\\
50.09142578125	-1.04560563443436e-12\\
50.111923828125	1.93042865421022e-11\\
50.132421875	-2.43987437175172e-11\\
50.152919921875	-2.55509468505875e-11\\
50.17341796875	-1.77571969025208e-11\\
50.193916015625	7.69784447474548e-11\\
50.2144140625	-4.63412659642703e-11\\
50.234912109375	-5.0919614685414e-12\\
50.25541015625	-1.7079844589245e-11\\
50.275908203125	-2.98624349630535e-11\\
50.29640625	-8.47315369268963e-11\\
50.316904296875	-1.00012778359289e-10\\
50.33740234375	-5.38624931967526e-11\\
50.357900390625	-3.63313879214579e-11\\
50.3783984375	-4.56824369797839e-12\\
50.398896484375	-3.66091609046077e-11\\
50.41939453125	-4.63238222834574e-11\\
50.439892578125	1.79382924983552e-11\\
50.460390625	-2.40238274487812e-11\\
50.480888671875	4.56829504786647e-13\\
50.50138671875	-1.6153411652773e-12\\
50.521884765625	4.09493109753498e-11\\
50.5423828125	4.22244683393706e-11\\
50.562880859375	5.87760176372619e-11\\
50.58337890625	6.56625265541873e-11\\
50.603876953125	9.78542927354341e-11\\
50.624375	8.30843182660818e-11\\
50.644873046875	5.01320832740263e-11\\
50.66537109375	-3.81560884936703e-11\\
50.685869140625	-6.89687260511519e-11\\
50.7063671875	-1.16378305355978e-10\\
50.726865234375	-5.92599252934686e-11\\
50.74736328125	-1.01544899638181e-10\\
50.767861328125	-7.71093937099319e-11\\
50.788359375	1.11866002664004e-11\\
50.808857421875	1.87111533184297e-11\\
50.82935546875	9.61326148258298e-11\\
50.849853515625	5.60860133993163e-11\\
50.8703515625	4.08329818805701e-11\\
50.890849609375	-2.12624154734961e-11\\
50.91134765625	-9.22835705868867e-11\\
50.931845703125	-1.46914261217173e-10\\
50.95234375	-1.30889525733183e-10\\
50.972841796875	-7.76694330522683e-11\\
50.99333984375	-1.30897490753257e-10\\
51.013837890625	-1.24983162592784e-10\\
51.0343359375	3.4773362268232e-12\\
51.054833984375	-5.44708402289195e-11\\
51.07533203125	5.45585446101572e-11\\
51.095830078125	-7.2002542857968e-11\\
51.116328125	-2.89107062714039e-12\\
51.136826171875	-1.28099883059108e-10\\
51.15732421875	-1.01399641034695e-10\\
51.177822265625	-9.1861827376663e-11\\
51.1983203125	-1.24003281432888e-10\\
51.218818359375	-4.08600164358356e-11\\
51.23931640625	-1.21010577416057e-10\\
51.259814453125	-2.91850829175592e-11\\
51.2803125	-6.65365459639861e-11\\
51.300810546875	-1.09073259013443e-10\\
51.32130859375	-4.66193204546159e-11\\
51.341806640625	-1.25252772172737e-10\\
51.3623046875	-1.58049939299365e-10\\
51.382802734375	-1.65297053870873e-10\\
51.40330078125	-9.55966100411984e-11\\
51.423798828125	-3.33921857926554e-13\\
51.444296875	-1.56908892021777e-12\\
51.464794921875	-4.31284332725031e-11\\
51.48529296875	-1.25405903317124e-12\\
51.505791015625	-3.99578985666285e-11\\
51.5262890625	-1.11608279211336e-10\\
51.546787109375	2.91091583643607e-13\\
51.56728515625	-1.47699668976525e-10\\
51.587783203125	-3.00299255309512e-11\\
51.60828125	-1.94344615808158e-10\\
51.628779296875	-5.20251427805552e-11\\
51.64927734375	-1.02691358155704e-10\\
51.669775390625	-5.51404611813947e-11\\
51.6902734375	-9.54208173183757e-11\\
51.710771484375	-1.03128731560119e-10\\
51.73126953125	-1.1544209812675e-10\\
51.751767578125	-9.05216022616714e-11\\
51.772265625	-5.5940480831706e-11\\
51.792763671875	-1.05184477506129e-10\\
51.81326171875	-4.9221763218151e-11\\
51.833759765625	-8.4537559871252e-11\\
51.8542578125	-7.33299668533361e-11\\
51.874755859375	-6.94483531129779e-11\\
51.89525390625	-6.93842367008192e-11\\
51.915751953125	-1.25658845683316e-10\\
51.93625	-8.12256631035695e-11\\
51.956748046875	-6.99132239477047e-11\\
51.97724609375	-9.54361681384611e-11\\
51.997744140625	-7.95824899176667e-12\\
52.0182421875	-5.55222608923812e-11\\
52.038740234375	-9.64170435745147e-12\\
52.05923828125	-4.6639941674543e-11\\
52.079736328125	-6.41937534815664e-11\\
52.100234375	9.63799100146614e-12\\
52.120732421875	-2.35452677589479e-11\\
52.14123046875	-6.33606673794791e-12\\
52.161728515625	3.97910407053155e-11\\
52.1822265625	-3.19328262011878e-11\\
52.202724609375	-4.35023344614647e-11\\
52.22322265625	-6.19225879331052e-11\\
52.243720703125	-8.78641890398966e-11\\
52.26421875	-3.28620457844258e-11\\
52.284716796875	-1.42506414603969e-10\\
52.30521484375	7.4893536670579e-11\\
52.325712890625	-3.30797762460295e-11\\
52.3462109375	-3.73857738347679e-11\\
52.366708984375	-1.05615730843566e-10\\
52.38720703125	-6.7521350033477e-11\\
52.407705078125	-1.417688663358e-10\\
52.428203125	-9.88854241434473e-11\\
52.448701171875	-1.07340604528627e-10\\
52.46919921875	-1.03128250739052e-10\\
52.489697265625	-1.3681476075852e-10\\
52.5101953125	-1.0704377731184e-10\\
52.530693359375	-6.4962717161915e-11\\
52.55119140625	-1.2957989225118e-10\\
52.571689453125	-1.3678087233834e-10\\
52.5921875	-1.58414696831165e-10\\
52.612685546875	-1.28549503867356e-10\\
52.63318359375	-2.15139352067715e-10\\
52.653681640625	-1.16198194886789e-10\\
52.6741796875	-1.51966900587089e-10\\
52.694677734375	-1.95516033003481e-10\\
52.71517578125	-1.6887524174391e-10\\
52.735673828125	-1.8381210787103e-10\\
52.756171875	-1.89341434771534e-10\\
52.776669921875	-1.85992841212478e-10\\
52.79716796875	-1.86050428580414e-10\\
52.817666015625	-1.74882377544802e-10\\
52.8381640625	-1.65191865275694e-10\\
52.858662109375	-1.31088157114959e-10\\
52.87916015625	-1.5075532355009e-10\\
52.899658203125	-1.5670537152033e-10\\
52.92015625	-1.24950855973095e-10\\
52.940654296875	-1.48234658032606e-10\\
52.96115234375	-1.44854768298725e-10\\
52.981650390625	-1.63298586669355e-10\\
53.0021484375	-1.71671525861173e-10\\
53.022646484375	-1.87699512604301e-10\\
53.04314453125	-2.21442815226754e-10\\
53.063642578125	-1.74261603736224e-10\\
53.084140625	-1.91678261703803e-10\\
53.104638671875	-1.42334288079475e-10\\
53.12513671875	-7.97586362134813e-11\\
53.145634765625	-1.19674893114215e-10\\
53.1661328125	-6.6775179917473e-11\\
53.186630859375	-4.39243519388991e-11\\
53.20712890625	2.10908929986575e-12\\
53.227626953125	-1.37706574001105e-11\\
53.248125	1.1851710262221e-11\\
53.268623046875	-9.15824603141385e-11\\
53.28912109375	-6.66184299730606e-11\\
53.309619140625	-1.04236628334623e-10\\
53.3301171875	-1.1579488885447e-10\\
53.350615234375	-1.03531001780447e-10\\
53.37111328125	-7.52293209708208e-11\\
53.391611328125	-4.74949960525788e-11\\
53.412109375	-6.81757144577415e-11\\
53.432607421875	-2.30315506110429e-11\\
53.45310546875	1.44968318606594e-12\\
53.473603515625	-1.99584474088811e-11\\
53.4941015625	3.3729567649732e-11\\
53.514599609375	6.53652108661318e-12\\
53.53509765625	-8.34965989713188e-11\\
53.555595703125	-7.83888368737223e-11\\
53.57609375	-1.1553425540183e-10\\
53.596591796875	3.04474907751007e-11\\
53.61708984375	-1.81312262822456e-10\\
53.637587890625	-9.50614478837989e-11\\
53.6580859375	-1.83562650624492e-10\\
53.678583984375	-1.22539896912087e-10\\
53.69908203125	-9.4522753733092e-11\\
53.719580078125	-5.49438049455272e-11\\
53.740078125	-2.79011000547192e-11\\
53.760576171875	-6.58067770322339e-11\\
53.78107421875	-1.13492884370938e-11\\
53.801572265625	-5.45230344983984e-11\\
53.8220703125	-4.22597053114489e-11\\
53.842568359375	-2.81935694114078e-11\\
53.86306640625	-5.72528716624593e-11\\
53.883564453125	-1.11624764541627e-11\\
53.9040625	-1.23560412183424e-11\\
53.924560546875	-6.92318809180404e-12\\
53.94505859375	-2.45961361012635e-11\\
53.965556640625	-5.93416803642842e-11\\
53.9860546875	-1.18456228945269e-12\\
54.006552734375	-1.05590744291719e-11\\
54.02705078125	5.52299145031947e-11\\
54.047548828125	6.8092671395552e-12\\
54.068046875	7.15124108462135e-11\\
54.088544921875	-1.83086832724642e-11\\
54.10904296875	3.21728691754e-11\\
54.129541015625	-1.63953855170485e-11\\
54.1500390625	-8.08208229399595e-12\\
54.170537109375	-7.84233827549739e-11\\
54.19103515625	-2.09470354072112e-11\\
54.211533203125	-2.554007246414e-11\\
54.23203125	3.02537220079604e-11\\
54.252529296875	7.91096519075251e-11\\
54.27302734375	1.06559523877294e-10\\
54.293525390625	1.53578334368674e-10\\
54.3140234375	8.14256475056063e-11\\
54.334521484375	8.78946943594697e-11\\
54.35501953125	-1.70602549481118e-11\\
54.375517578125	-1.55668650590282e-11\\
54.396015625	-5.1707765956657e-11\\
54.416513671875	6.25500890070242e-11\\
54.43701171875	-4.65099852851707e-11\\
54.457509765625	2.50181299958153e-11\\
54.4780078125	6.48819370398152e-11\\
54.498505859375	5.20714726061857e-11\\
54.51900390625	7.69181678769116e-11\\
54.539501953125	8.4107008139838e-11\\
54.56	3.22173936525247e-11\\
54.580498046875	2.3913539086748e-11\\
54.60099609375	-3.72816788753425e-12\\
54.621494140625	-2.15451404540373e-12\\
54.6419921875	-6.69453525075142e-12\\
54.662490234375	7.72918775702171e-12\\
54.68298828125	2.08542917953015e-11\\
54.703486328125	4.11589504764856e-11\\
54.723984375	5.95708148561443e-11\\
54.744482421875	7.95053136302866e-11\\
54.76498046875	9.31779817535561e-11\\
54.785478515625	1.2833783987442e-10\\
54.8059765625	1.92291077481218e-11\\
54.826474609375	1.84582186060999e-12\\
54.84697265625	2.37169854875756e-12\\
54.867470703125	2.05523022946169e-11\\
54.88796875	6.81920321129087e-11\\
54.908466796875	1.2936369163445e-10\\
54.92896484375	1.13981432256627e-10\\
54.949462890625	1.65256190793771e-10\\
54.9699609375	1.19093887276582e-10\\
54.990458984375	1.36318858532083e-10\\
55.01095703125	1.58380505966555e-10\\
55.031455078125	1.42620342810809e-10\\
55.051953125	9.59630627102404e-11\\
55.072451171875	5.93302537454603e-11\\
55.09294921875	9.47855170345082e-11\\
55.113447265625	6.75943491491362e-11\\
55.1339453125	1.50385159448645e-10\\
55.154443359375	9.57483660501976e-11\\
55.17494140625	1.41274868870033e-10\\
55.195439453125	1.14166775158367e-10\\
55.2159375	1.65011602620577e-10\\
55.236435546875	1.63390151656176e-10\\
55.25693359375	1.4916835427779e-10\\
55.277431640625	9.16144316327495e-11\\
55.2979296875	1.66405042868376e-10\\
55.318427734375	1.42979589459619e-10\\
55.33892578125	1.49933923427886e-10\\
55.359423828125	1.13452047344039e-10\\
55.379921875	9.90673625761973e-11\\
55.400419921875	1.41247677424824e-10\\
55.42091796875	1.1055093854481e-10\\
55.441416015625	9.32638349955511e-11\\
55.4619140625	1.00134914295442e-10\\
55.482412109375	1.42002662529554e-10\\
55.50291015625	1.53497534643117e-10\\
55.523408203125	1.2900731750161e-10\\
55.54390625	1.46490555066269e-10\\
55.564404296875	1.31484422503769e-10\\
55.58490234375	1.73206031080916e-10\\
55.605400390625	1.40533813386824e-10\\
55.6258984375	9.28544763060149e-11\\
55.646396484375	8.96581839165342e-11\\
55.66689453125	-4.59728297184434e-12\\
55.687392578125	3.89985638300723e-11\\
55.707890625	2.32593050064768e-11\\
55.728388671875	4.87810933133795e-11\\
55.74888671875	7.51859853890841e-11\\
55.769384765625	1.10941956059781e-10\\
55.7898828125	3.10757395440189e-11\\
55.810380859375	8.2434776588252e-11\\
55.83087890625	5.57728635649445e-12\\
55.851376953125	-4.30796474682074e-13\\
55.871875	3.50523590484394e-11\\
55.892373046875	1.1310876178524e-11\\
55.91287109375	2.56069120752693e-11\\
55.933369140625	-1.17317737355858e-11\\
55.9538671875	-4.65903042664967e-12\\
55.974365234375	6.45881454277093e-12\\
55.99486328125	3.92136394907818e-11\\
56.015361328125	6.00688546751237e-11\\
56.035859375	3.22676322389657e-11\\
56.056357421875	-2.23146022190987e-11\\
56.07685546875	2.23774289698132e-11\\
56.097353515625	1.42327579190321e-12\\
56.1178515625	4.9983720578248e-11\\
56.138349609375	1.33712835073701e-11\\
56.15884765625	8.75826323247906e-11\\
56.179345703125	1.10661291088065e-11\\
56.19984375	9.30654838770237e-11\\
56.220341796875	2.05769861990184e-11\\
56.24083984375	-3.4542772426088e-11\\
56.261337890625	4.96501754735037e-12\\
56.2818359375	9.21344349548384e-11\\
56.302333984375	5.45649043465652e-11\\
56.32283203125	1.13290529858909e-11\\
56.343330078125	7.81615596136907e-11\\
56.363828125	8.93892166756978e-11\\
56.384326171875	4.40409699220769e-11\\
56.40482421875	6.98421902651562e-11\\
56.425322265625	7.40584546556902e-11\\
56.4458203125	3.41305281035997e-11\\
56.466318359375	1.06813525002646e-10\\
56.48681640625	1.91789882798185e-11\\
56.507314453125	1.26145719024356e-11\\
56.5278125	-6.91947060846123e-12\\
56.548310546875	3.6042341034142e-11\\
56.56880859375	-2.1463673490275e-11\\
56.589306640625	1.34687446220729e-10\\
56.6098046875	2.61214568220266e-11\\
56.630302734375	1.13989492600845e-10\\
56.65080078125	1.03663615247506e-10\\
56.671298828125	1.05666742942155e-10\\
56.691796875	7.24680811098307e-11\\
56.712294921875	1.8187558791739e-11\\
56.73279296875	1.00839053578453e-11\\
56.753291015625	-3.64482955983206e-11\\
56.7737890625	4.88851640892706e-11\\
56.794287109375	-8.09921711837282e-11\\
56.81478515625	7.1858423163015e-11\\
56.835283203125	2.79210256865845e-11\\
56.85578125	6.14594131865105e-11\\
56.876279296875	5.12621459213703e-11\\
56.89677734375	8.83830333225702e-11\\
56.917275390625	4.88093267477575e-11\\
56.9377734375	9.4456857430227e-11\\
};
\addlegendentry{$\text{train 3 -\textgreater{} Heimdal}$};

\addplot [color=mycolor2,solid,forget plot]
  table[row sep=crcr]{%
-47.518515625	5.29155774508828e-11\\
-47.4972265625	9.70779845633673e-11\\
-47.4759375	1.13581701913621e-10\\
-47.4546484375	1.58662338131455e-10\\
-47.433359375	1.82773108164428e-10\\
-47.4120703125	2.04996027506188e-10\\
-47.39078125	2.06403589235636e-10\\
-47.3694921875	1.94801093121976e-10\\
-47.348203125	1.56818868225733e-10\\
-47.3269140625	1.53724250383501e-10\\
-47.305625	1.14902042195776e-10\\
-47.2843359375	7.12401650110145e-11\\
-47.263046875	1.53418709275908e-10\\
-47.2417578125	1.65282741910374e-10\\
-47.22046875	1.89705518989793e-10\\
-47.1991796875	1.80403345475759e-10\\
-47.177890625	1.3832015278965e-10\\
-47.1566015625	1.17298784505299e-10\\
-47.1353125	-3.02218326867449e-11\\
-47.1140234375	-5.54931052131998e-11\\
-47.092734375	-6.67682499138156e-11\\
-47.0714453125	-4.53783512082528e-11\\
-47.05015625	-2.68658544979426e-11\\
-47.0288671875	-1.9760068350685e-11\\
-47.007578125	6.54025682030477e-13\\
-46.9862890625	6.35967535233481e-11\\
-46.965	2.89641661565706e-11\\
-46.9437109375	2.09384361697574e-11\\
-46.922421875	1.6442616409302e-11\\
-46.9011328125	-9.77651032178706e-11\\
-46.87984375	-1.05207854118384e-10\\
-46.8585546875	-1.42245781688098e-10\\
-46.837265625	-1.0429130162108e-10\\
-46.8159765625	-1.9498190395136e-10\\
-46.7946875	-3.0566277785989e-11\\
-46.7733984375	-8.63706329489739e-11\\
-46.752109375	-5.42821501085703e-11\\
-46.7308203125	1.89866588951538e-11\\
-46.70953125	1.9359263985507e-11\\
-46.6882421875	-3.74372045995671e-11\\
-46.666953125	-1.1689665278136e-10\\
-46.6456640625	-1.13765217326284e-10\\
-46.624375	-1.78393072620391e-10\\
-46.6030859375	-2.07271219656156e-10\\
-46.581796875	-2.20017475798494e-10\\
-46.5605078125	-1.27392710591837e-10\\
-46.53921875	-1.20266312362535e-10\\
-46.5179296875	-1.08455013535852e-10\\
-46.496640625	-1.05981915828954e-10\\
-46.4753515625	-1.11612053497822e-10\\
-46.4540625	-2.19092036616169e-10\\
-46.4327734375	-2.69017573262009e-10\\
-46.411484375	-2.47408017021783e-10\\
-46.3901953125	-3.53109104140687e-10\\
-46.36890625	-3.7775073112801e-10\\
-46.3476171875	-2.90644979485823e-10\\
-46.326328125	-2.57971466115388e-10\\
-46.3050390625	-2.70969061779388e-10\\
-46.28375	-1.72211725787536e-10\\
-46.2624609375	-1.04230300297028e-10\\
-46.241171875	-1.31316691864035e-10\\
-46.2198828125	-1.26243044738091e-10\\
-46.19859375	-1.71369368086588e-10\\
-46.1773046875	-1.46020165686051e-10\\
-46.156015625	-2.46582393097134e-10\\
-46.1347265625	-2.92089034797039e-10\\
-46.1134375	-2.02270120123085e-10\\
-46.0921484375	-2.05161973891312e-10\\
-46.070859375	-2.08119864907182e-10\\
-46.0495703125	-1.91939238495336e-10\\
-46.02828125	-1.16371086634417e-10\\
-46.0069921875	-1.42142907371269e-10\\
-45.985703125	-1.13022814035716e-10\\
-45.9644140625	6.32203723491569e-12\\
-45.943125	-2.93581125588241e-11\\
-45.9218359375	3.440104999403e-11\\
-45.900546875	7.5009429243114e-12\\
-45.8792578125	-8.03974671520022e-12\\
-45.85796875	-3.62383926170932e-11\\
-45.8366796875	-2.59382991003335e-11\\
-45.815390625	2.80527524866306e-11\\
-45.7941015625	-1.6446961048521e-11\\
-45.7728125	6.69100592718681e-11\\
-45.7515234375	5.60899103382012e-11\\
-45.730234375	8.31511859011359e-11\\
-45.7089453125	9.10764522436921e-11\\
-45.68765625	1.19735536301525e-10\\
-45.6663671875	4.01823131430636e-11\\
-45.645078125	1.0367582264347e-10\\
-45.6237890625	-7.22376717662354e-12\\
-45.6025	4.7976238847949e-11\\
-45.5812109375	-2.12610395758464e-11\\
-45.559921875	2.51689848839379e-11\\
-45.5386328125	4.76874801335447e-11\\
-45.51734375	5.45450533132569e-11\\
-45.4960546875	7.97941312994581e-11\\
-45.474765625	1.07876846260321e-10\\
-45.4534765625	8.80272935011509e-11\\
-45.4321875	1.26803140390802e-10\\
-45.4108984375	1.02790408654201e-10\\
-45.389609375	1.17605072189174e-10\\
-45.3683203125	1.251705781308e-10\\
-45.34703125	1.80463999332642e-10\\
-45.3257421875	1.55573542611982e-10\\
-45.304453125	1.63863404343057e-10\\
-45.2831640625	2.1442155910099e-10\\
-45.261875	1.7131748170767e-10\\
-45.2405859375	1.72647190662725e-10\\
-45.219296875	1.73083996777155e-10\\
-45.1980078125	8.78190306287545e-11\\
-45.17671875	1.52706733222112e-10\\
-45.1554296875	1.22800064192166e-10\\
-45.134140625	9.55279103508487e-11\\
-45.1128515625	1.15195815329857e-10\\
-45.0915625	6.13119610291091e-11\\
-45.0702734375	8.21958238588454e-11\\
-45.048984375	6.33587059262676e-12\\
-45.0276953125	1.24953123466954e-10\\
-45.00640625	4.05355181481148e-12\\
-44.9851171875	-4.39125873416276e-12\\
-44.963828125	-6.91749138037992e-11\\
-44.9425390625	-6.77489754391465e-11\\
-44.92125	-1.26156796901288e-10\\
-44.8999609375	-1.41483140699027e-10\\
-44.878671875	-1.4465699707691e-10\\
-44.8573828125	-1.42197498333525e-10\\
-44.83609375	-6.22737352799471e-11\\
-44.8148046875	-5.61961832023139e-11\\
-44.793515625	4.30880216426179e-11\\
-44.7722265625	-2.24507872082199e-11\\
-44.7509375	2.66891457076969e-11\\
-44.7296484375	-1.19209724225299e-11\\
-44.708359375	-3.79250065346803e-11\\
-44.6870703125	-8.59852523339577e-11\\
-44.66578125	-1.52738438065989e-10\\
-44.6444921875	-1.25804279322416e-10\\
-44.623203125	-1.31772209625496e-10\\
-44.6019140625	-1.01342378829151e-10\\
-44.580625	-9.3488892305982e-11\\
-44.5593359375	-3.28707272342547e-11\\
-44.538046875	1.05117832261841e-11\\
-44.5167578125	4.13295010350331e-11\\
-44.49546875	2.92972251332303e-12\\
-44.4741796875	1.17336429160387e-10\\
-44.452890625	1.02321350242939e-10\\
-44.4316015625	6.01676311630318e-11\\
-44.4103125	1.07066283261968e-10\\
-44.3890234375	6.01151290859904e-11\\
-44.367734375	5.29115751702051e-12\\
-44.3464453125	5.71218746338359e-12\\
-44.32515625	5.26768932272041e-11\\
-44.3038671875	9.19081430823166e-11\\
-44.282578125	8.04476084226438e-11\\
-44.2612890625	1.89982085281926e-10\\
-44.24	1.55514522715833e-10\\
-44.2187109375	1.86161043184817e-10\\
-44.197421875	1.89671827141288e-10\\
-44.1761328125	1.38000618109889e-10\\
-44.15484375	1.43909168464065e-10\\
-44.1335546875	8.44475372659193e-11\\
-44.112265625	1.18648926143841e-10\\
-44.0909765625	1.32252921816056e-10\\
-44.0696875	1.69977144577158e-10\\
-44.0483984375	2.55845343136266e-10\\
-44.027109375	2.08700848114071e-10\\
-44.0058203125	1.45428057496812e-10\\
-43.98453125	1.63607413790602e-10\\
-43.9632421875	7.99655163104418e-11\\
-43.941953125	3.73382628965661e-11\\
-43.9206640625	-4.53336683577573e-11\\
-43.899375	-2.37754346693145e-11\\
-43.8780859375	-3.52832983330141e-11\\
-43.856796875	-6.67679233776391e-11\\
-43.8355078125	9.17523551492325e-11\\
-43.81421875	1.68117221722803e-10\\
-43.7929296875	8.36755447219899e-11\\
-43.771640625	1.20369170423633e-10\\
-43.7503515625	1.62001482680683e-10\\
-43.7290625	1.77058438408382e-10\\
-43.7077734375	9.64743743340239e-11\\
-43.686484375	1.12740134639916e-10\\
-43.6651953125	1.50101033417881e-10\\
-43.64390625	1.37280791625206e-10\\
-43.6226171875	1.20152844634869e-10\\
-43.601328125	1.21987932302451e-10\\
-43.5800390625	7.38264073147857e-11\\
-43.55875	2.63828752641845e-11\\
-43.5374609375	8.3948667546111e-11\\
-43.516171875	1.36224868950572e-10\\
-43.4948828125	1.61495422970864e-10\\
-43.47359375	2.45246903017477e-10\\
-43.4523046875	2.90165197650069e-10\\
-43.431015625	3.46528757543712e-10\\
-43.4097265625	2.98714683426647e-10\\
-43.3884375	3.03001470008212e-10\\
-43.3671484375	2.1117724080405e-10\\
-43.345859375	2.04121240804475e-10\\
-43.3245703125	1.33803508201607e-10\\
-43.30328125	1.34984283340879e-10\\
-43.2819921875	2.63794719690174e-10\\
-43.260703125	1.72188335429436e-10\\
-43.2394140625	2.47293834149195e-10\\
-43.218125	3.06132850266276e-10\\
-43.1968359375	3.45497030047438e-10\\
-43.175546875	3.71271079266726e-10\\
-43.1542578125	2.98707960459016e-10\\
-43.13296875	2.20711247529509e-10\\
-43.1116796875	2.47504534998121e-10\\
-43.090390625	1.44885773826049e-10\\
-43.0691015625	1.5307631307675e-10\\
-43.0478125	1.25839527870767e-10\\
-43.0265234375	2.10634753597317e-10\\
-43.005234375	2.12579678802195e-10\\
-42.9839453125	2.13235720799763e-10\\
-42.96265625	2.55795973142706e-10\\
-42.9413671875	1.92930408963978e-10\\
-42.920078125	1.7638062088508e-10\\
-42.8987890625	8.19956540684781e-11\\
-42.8775	1.03451701342103e-10\\
-42.8562109375	1.07819265653327e-10\\
-42.834921875	4.9770348201761e-11\\
-42.8136328125	1.54541408335225e-10\\
-42.79234375	8.36980510743645e-11\\
-42.7710546875	9.39205752602607e-11\\
-42.749765625	1.41122658685857e-10\\
-42.7284765625	1.4148935833924e-10\\
-42.7071875	4.10206213564273e-11\\
-42.6858984375	5.14649658263423e-12\\
-42.664609375	3.72219224680721e-11\\
-42.6433203125	8.04728818845983e-12\\
-42.62203125	-3.93475583352655e-11\\
-42.6007421875	-5.26432172374498e-11\\
-42.579453125	3.77222553318786e-11\\
-42.5581640625	-1.20094584349622e-11\\
-42.536875	-2.84210709438006e-11\\
-42.5155859375	3.21532131813685e-11\\
-42.494296875	1.85941358885913e-10\\
-42.4730078125	2.17487481132048e-11\\
-42.45171875	6.75550240683597e-11\\
-42.4304296875	3.11630704716366e-11\\
-42.409140625	1.763156863249e-11\\
-42.3878515625	1.52708394130503e-11\\
-42.3665625	9.00443085748699e-11\\
-42.3452734375	1.73280195717726e-11\\
-42.323984375	9.58871075313241e-11\\
-42.3026953125	1.94712388104035e-10\\
-42.28140625	2.11153831563148e-10\\
-42.2601171875	1.31162246382295e-10\\
-42.238828125	8.36871215410136e-11\\
-42.2175390625	1.69432498055827e-10\\
-42.19625	-4.66571063437227e-12\\
-42.1749609375	-3.52303510688327e-11\\
-42.153671875	-1.19334491534037e-12\\
-42.1323828125	2.54184313982693e-11\\
-42.11109375	-5.56885008376202e-12\\
-42.0898046875	8.76158801961017e-11\\
-42.068515625	1.238329235188e-10\\
-42.0472265625	4.98447140547283e-11\\
-42.0259375	8.71902861654118e-11\\
-42.0046484375	3.13797993996978e-11\\
-41.983359375	-4.1202040289824e-12\\
-41.9620703125	-3.73060117371297e-12\\
-41.94078125	-3.81681981041761e-11\\
-41.9194921875	-2.7805318817002e-12\\
-41.898203125	-7.27381741077695e-11\\
-41.8769140625	-3.619889470589e-11\\
-41.855625	-3.23469594220853e-12\\
-41.8343359375	3.31107914230206e-11\\
-41.813046875	2.04771213994468e-12\\
-41.7917578125	5.26350910454213e-11\\
-41.77046875	7.7771069559931e-11\\
-41.7491796875	-1.73389566747472e-11\\
-41.727890625	2.59970520420876e-11\\
-41.7066015625	-7.96238450512483e-11\\
-41.6853125	-8.81609172267471e-11\\
-41.6640234375	-1.40623133029898e-10\\
-41.642734375	-1.23682304709536e-10\\
-41.6214453125	-1.49175627277957e-10\\
-41.60015625	-1.45728720625465e-10\\
-41.5788671875	-1.01887117016844e-10\\
-41.557578125	-1.54840174981589e-10\\
-41.5362890625	-1.86429132609521e-10\\
-41.515	-2.14958817634404e-10\\
-41.4937109375	-2.40077733403892e-10\\
-41.472421875	-2.05404211207946e-10\\
-41.4511328125	-2.78759787231111e-10\\
-41.42984375	-2.90485319382362e-10\\
-41.4085546875	-2.73185143007808e-10\\
-41.387265625	-2.70243198669376e-10\\
-41.3659765625	-2.48793707389448e-10\\
-41.3446875	-2.96253374518431e-10\\
-41.3233984375	-3.05773718585205e-10\\
-41.302109375	-2.24464353742882e-10\\
-41.2808203125	-3.07474051993183e-10\\
-41.25953125	-3.39831065384177e-10\\
-41.2382421875	-3.78546634616406e-10\\
-41.216953125	-3.17695545226984e-10\\
-41.1956640625	-2.79231999560322e-10\\
-41.174375	-3.05470772189679e-10\\
-41.1530859375	-2.00282628284482e-10\\
-41.131796875	-2.59617146686949e-10\\
-41.1105078125	-2.95410740694527e-10\\
-41.08921875	-3.14595683381512e-10\\
-41.0679296875	-3.41248941068124e-10\\
-41.046640625	-3.8832115102357e-10\\
-41.0253515625	-2.9013281039394e-10\\
-41.0040625	-2.93116002540581e-10\\
-40.9827734375	-1.69996532111395e-10\\
-40.961484375	-1.93832081881828e-10\\
-40.9401953125	-1.99217527341992e-10\\
-40.91890625	-1.52872458026569e-10\\
-40.8976171875	-2.20303037881916e-10\\
-40.876328125	-2.42330089325172e-10\\
-40.8550390625	-2.50963847638187e-10\\
-40.83375	-2.95249249759825e-10\\
-40.8124609375	-2.66839733020107e-10\\
-40.791171875	-3.26700041572815e-10\\
-40.7698828125	-1.9846128879472e-10\\
-40.74859375	-1.72549161849469e-10\\
-40.7273046875	-1.31564100559484e-10\\
-40.706015625	-1.82530793752101e-10\\
-40.6847265625	-2.32887121368086e-10\\
-40.6634375	-2.30643158579722e-10\\
-40.6421484375	-2.50625332882063e-10\\
-40.620859375	-2.93549328120984e-10\\
-40.5995703125	-2.43842061959478e-10\\
-40.57828125	-2.68019884379895e-10\\
-40.5569921875	-2.34229832934205e-10\\
-40.535703125	-1.25705159672674e-10\\
-40.5144140625	-1.83092800594028e-10\\
-40.493125	-1.27326120813954e-10\\
-40.4718359375	-2.72462134399536e-10\\
-40.450546875	-1.85537808199139e-10\\
-40.4292578125	-2.6056932850537e-10\\
-40.40796875	-2.49865132449155e-10\\
-40.3866796875	-3.46853174696986e-10\\
-40.365390625	-3.04320557635124e-10\\
-40.3441015625	-2.47919011858e-10\\
-40.3228125	-2.14573270805504e-10\\
-40.3015234375	-2.2637665491751e-10\\
-40.280234375	-1.0669714787636e-10\\
-40.2589453125	-2.22832295734968e-10\\
-40.23765625	-1.80522396263466e-10\\
-40.2163671875	-1.40369321963672e-10\\
-40.195078125	-2.18631227137114e-10\\
-40.1737890625	-2.17487160610847e-10\\
-40.1525	-3.43963279546435e-10\\
-40.1312109375	-1.8864581208612e-10\\
-40.109921875	-1.39172511584901e-10\\
-40.0886328125	-1.25337199939983e-10\\
-40.06734375	1.2176146032424e-11\\
-40.0460546875	-6.80057638669151e-11\\
-40.024765625	9.74363038976618e-12\\
-40.0034765625	-5.3773147288463e-11\\
-39.9821875	5.31596104896161e-11\\
-39.9608984375	-1.38877642765201e-11\\
-39.939609375	-1.1126574620152e-10\\
-39.9183203125	1.1399421278309e-11\\
-39.89703125	-6.84364171827815e-11\\
-39.8757421875	-3.33679734082112e-11\\
-39.854453125	3.92946269744669e-11\\
-39.8331640625	7.72173515895099e-11\\
-39.811875	-1.82581167250013e-11\\
-39.7905859375	2.97368885061263e-11\\
-39.769296875	-2.56413053158267e-11\\
-39.7480078125	4.23926461253369e-12\\
-39.72671875	-1.36663784539983e-10\\
-39.7054296875	-7.94664648809193e-11\\
-39.684140625	-4.90135337216291e-12\\
-39.6628515625	-1.29637411456692e-10\\
-39.6415625	5.26754416126048e-11\\
-39.6202734375	1.93218832597504e-11\\
-39.598984375	5.43665546357042e-11\\
-39.5776953125	3.93907013142097e-11\\
-39.55640625	6.35205205633448e-12\\
-39.5351171875	-4.77674328646681e-12\\
-39.513828125	3.30017429362602e-11\\
-39.4925390625	-2.06196427450563e-11\\
-39.47125	-5.87190568318182e-11\\
-39.4499609375	7.56034970892773e-11\\
-39.428671875	1.52785811280755e-10\\
-39.4073828125	1.5956795275818e-10\\
-39.38609375	1.28791348837337e-10\\
-39.3648046875	1.73787582786056e-10\\
-39.343515625	8.47950090932683e-11\\
-39.3222265625	1.16621669889096e-10\\
-39.3009375	6.87179332367048e-11\\
-39.2796484375	1.0617138061245e-10\\
-39.258359375	1.18799884709846e-10\\
-39.2370703125	4.53585126844015e-11\\
-39.21578125	1.44450072442691e-10\\
-39.1944921875	1.97017651793238e-10\\
-39.173203125	2.33987515084444e-10\\
-39.1519140625	1.60789952285013e-10\\
-39.130625	1.95125306129184e-10\\
-39.1093359375	2.11436017432279e-10\\
-39.088046875	1.24662637285224e-10\\
-39.0667578125	2.72972301169494e-10\\
-39.04546875	2.56869203161857e-10\\
-39.0241796875	2.30801199778109e-10\\
-39.002890625	2.65278162775064e-10\\
-38.9816015625	2.75642060824307e-10\\
-38.9603125	2.38736093321844e-10\\
-38.9390234375	2.79477423558068e-10\\
-38.917734375	2.24105484786927e-10\\
-38.8964453125	3.87236157069175e-10\\
-38.87515625	3.82064761794789e-10\\
-38.8538671875	4.07519821975718e-10\\
-38.832578125	3.51435072335162e-10\\
-38.8112890625	4.10248668881976e-10\\
-38.79	4.54117138784731e-10\\
-38.7687109375	4.2690279229989e-10\\
-38.747421875	4.56537129936008e-10\\
-38.7261328125	5.42011795419592e-10\\
-38.70484375	4.48842385689104e-10\\
-38.6835546875	5.23957372899945e-10\\
-38.662265625	5.09192794551758e-10\\
-38.6409765625	5.38286944026483e-10\\
-38.6196875	4.69393976469014e-10\\
-38.5983984375	4.14185808729265e-10\\
-38.577109375	4.95444871287083e-10\\
-38.5558203125	3.56082952777156e-10\\
-38.53453125	4.52066322351278e-10\\
-38.5132421875	5.21539134198918e-10\\
-38.491953125	5.11784122173516e-10\\
-38.4706640625	5.7733485729617e-10\\
-38.449375	6.03570644687967e-10\\
-38.4280859375	5.7076900370127e-10\\
-38.406796875	5.35471708024782e-10\\
-38.3855078125	4.89357649744278e-10\\
-38.36421875	4.12287227598871e-10\\
-38.3429296875	3.82099111043219e-10\\
-38.321640625	2.44160631958796e-10\\
-38.3003515625	3.18992734073135e-10\\
-38.2790625	3.24672605564117e-10\\
-38.2577734375	3.39798566389652e-10\\
-38.236484375	3.59520389435369e-10\\
-38.2151953125	3.74215584116237e-10\\
-38.19390625	4.16607754939197e-10\\
-38.1726171875	4.55019169844085e-10\\
-38.151328125	3.46392284923854e-10\\
-38.1300390625	4.30561401935982e-10\\
-38.10875	3.72915444123227e-10\\
-38.0874609375	3.08874949865166e-10\\
-38.066171875	3.1067902790119e-10\\
-38.0448828125	3.16143823079156e-10\\
-38.02359375	2.6922470401978e-10\\
-38.0023046875	2.48803515876659e-10\\
-37.981015625	2.62167149664989e-10\\
-37.9597265625	2.46916510814561e-10\\
-37.9384375	3.0425050571553e-10\\
-37.9171484375	2.59629311806858e-10\\
-37.895859375	2.89145477334797e-10\\
-37.8745703125	2.27718445216148e-10\\
-37.85328125	2.56595539472526e-10\\
-37.8319921875	2.99347225096776e-10\\
-37.810703125	2.30824644161761e-10\\
-37.7894140625	2.30556368007092e-10\\
-37.768125	2.52890979524415e-10\\
-37.7468359375	2.69474700975659e-10\\
-37.725546875	1.86122093788683e-10\\
-37.7042578125	2.39312668351404e-10\\
-37.68296875	2.56061811848873e-10\\
-37.6616796875	1.55411803861385e-10\\
-37.640390625	1.06516787646282e-10\\
-37.6191015625	1.03171408912553e-10\\
-37.5978125	4.47295927745188e-11\\
-37.5765234375	1.1956606885674e-10\\
-37.555234375	6.86537048295593e-11\\
-37.5339453125	-1.05830623170888e-11\\
-37.51265625	8.37529369727591e-11\\
-37.4913671875	4.42562414168455e-11\\
-37.470078125	3.04962737432687e-11\\
-37.4487890625	2.24430821970636e-10\\
-37.4275	9.25657259243503e-11\\
-37.4062109375	1.37257542244545e-10\\
-37.384921875	2.30884156549798e-10\\
-37.3636328125	3.67175300251269e-11\\
-37.34234375	9.24625184348515e-12\\
-37.3210546875	-1.72765307230657e-11\\
-37.299765625	-1.33544748141057e-10\\
-37.2784765625	-1.06761923111133e-10\\
-37.2571875	-9.65457949486961e-11\\
-37.2358984375	-7.21433126936781e-11\\
-37.214609375	-7.31982621710865e-11\\
-37.1933203125	-6.70786374897018e-11\\
-37.17203125	3.63950556151537e-11\\
-37.1507421875	3.42682488439649e-11\\
-37.129453125	-3.90799237036078e-11\\
-37.1081640625	-4.60180097457445e-11\\
-37.086875	-1.22631149375995e-10\\
-37.0655859375	-9.61273777085958e-11\\
-37.044296875	-1.28731530951259e-10\\
-37.0230078125	-2.23525073712723e-10\\
-37.00171875	-1.13847456297573e-10\\
-36.9804296875	-7.21439812529796e-11\\
-36.959140625	-1.10009176732864e-10\\
-36.9378515625	-1.29979555971978e-10\\
-36.9165625	-1.0589220974174e-10\\
-36.8952734375	-1.49693085530504e-10\\
-36.873984375	-2.42000553383503e-10\\
-36.8526953125	-2.68396096709724e-10\\
-36.83140625	-2.98254785112352e-10\\
-36.8101171875	-2.20684260789647e-10\\
-36.788828125	-3.13006325258258e-10\\
-36.7675390625	-2.92799545038071e-10\\
-36.74625	-2.01700287958598e-10\\
-36.7249609375	-2.3470361341714e-10\\
-36.703671875	-2.52017246345549e-10\\
-36.6823828125	-2.02897202173952e-10\\
-36.66109375	-2.62692039946825e-10\\
-36.6398046875	-2.95280919597982e-10\\
-36.618515625	-3.56603896732583e-10\\
-36.5972265625	-3.87421440146981e-10\\
-36.5759375	-4.15963757996598e-10\\
-36.5546484375	-4.56504443505084e-10\\
-36.533359375	-3.65332603891696e-10\\
-36.5120703125	-4.50791168102178e-10\\
-36.49078125	-3.56877716971125e-10\\
-36.4694921875	-4.08802003963513e-10\\
-36.448203125	-4.0228945156331e-10\\
-36.4269140625	-3.82915735162851e-10\\
-36.405625	-3.59544648723426e-10\\
-36.3843359375	-3.47420858705821e-10\\
-36.363046875	-4.4102486590448e-10\\
-36.3417578125	-3.76599348384553e-10\\
-36.32046875	-3.9912990650108e-10\\
-36.2991796875	-5.18591148742485e-10\\
-36.277890625	-4.06156805030707e-10\\
-36.2566015625	-4.73294509025581e-10\\
-36.2353125	-5.51646672922303e-10\\
-36.2140234375	-4.7380720925102e-10\\
-36.192734375	-5.35654145010592e-10\\
-36.1714453125	-4.96708494330653e-10\\
-36.15015625	-6.07474069476459e-10\\
-36.1288671875	-6.66184567054785e-10\\
-36.107578125	-6.07180578392946e-10\\
-36.0862890625	-7.09670225821992e-10\\
-36.065	-6.59421390530497e-10\\
-36.0437109375	-5.35272609735244e-10\\
-36.022421875	-5.68894688784847e-10\\
-36.0011328125	-4.86783662960249e-10\\
-35.97984375	-4.42766994951936e-10\\
-35.9585546875	-4.65376955565868e-10\\
-35.937265625	-5.16351620297922e-10\\
-35.9159765625	-6.03108793584164e-10\\
-35.8946875	-7.19686358425679e-10\\
-35.8733984375	-6.92315151982934e-10\\
-35.852109375	-7.86995651635917e-10\\
-35.8308203125	-7.25186685224764e-10\\
-35.80953125	-5.92797854076996e-10\\
-35.7882421875	-6.19713866187185e-10\\
-35.766953125	-5.35559742214777e-10\\
-35.7456640625	-4.52838515202557e-10\\
-35.724375	-4.08245346487129e-10\\
-35.7030859375	-4.01020660136568e-10\\
-35.681796875	-4.18200329383809e-10\\
-35.6605078125	-5.34527354723305e-10\\
-35.63921875	-5.65172708851425e-10\\
-35.6179296875	-5.05483876625729e-10\\
-35.596640625	-6.09968638342305e-10\\
-35.5753515625	-5.27234234423925e-10\\
-35.5540625	-4.49996523818605e-10\\
-35.5327734375	-3.71020650476315e-10\\
-35.511484375	-3.58079588932469e-10\\
-35.4901953125	-3.20905209758194e-10\\
-35.46890625	-3.47972973971048e-10\\
-35.4476171875	-3.25182900912535e-10\\
-35.426328125	-4.23994619834217e-10\\
-35.4050390625	-3.55239696049411e-10\\
-35.38375	-4.29405902130201e-10\\
-35.3624609375	-4.4732116492376e-10\\
-35.341171875	-3.73261970969635e-10\\
-35.3198828125	-3.94721020293316e-10\\
-35.29859375	-2.72031367008179e-10\\
-35.2773046875	-2.72559210854474e-10\\
-35.256015625	-2.74437140766109e-10\\
-35.2347265625	-2.32962390262655e-10\\
-35.2134375	-1.95531146324881e-10\\
-35.1921484375	-2.06139838059492e-10\\
-35.170859375	-1.82651593610662e-10\\
-35.1495703125	-1.7769095321524e-10\\
-35.12828125	-2.08660257625443e-10\\
-35.1069921875	-2.47338551233235e-10\\
-35.085703125	-1.16806354310474e-10\\
-35.0644140625	-9.40725233091553e-11\\
-35.043125	3.51106655926893e-11\\
-35.0218359375	-4.97765463171777e-11\\
-35.000546875	5.39876752790229e-11\\
-34.9792578125	9.9616852238541e-11\\
-34.95796875	7.18596274530056e-12\\
-34.9366796875	2.26744588482131e-11\\
-34.915390625	-3.97868287217583e-11\\
-34.8941015625	-1.03163189262894e-10\\
-34.8728125	-4.60712071808955e-11\\
-34.8515234375	-8.13507101867789e-11\\
-34.830234375	-2.57228530851824e-11\\
-34.8089453125	3.97151177297098e-11\\
-34.78765625	3.60166299127328e-11\\
-34.7663671875	1.3638497990009e-10\\
-34.745078125	8.50992834582307e-11\\
-34.7237890625	1.43395192966666e-10\\
-34.7025	1.28855333751822e-10\\
-34.6812109375	-2.27263351963295e-11\\
-34.659921875	8.41429665924483e-11\\
-34.6386328125	1.24578585571968e-10\\
-34.61734375	2.06434191520971e-11\\
-34.5960546875	1.44148210104813e-10\\
-34.574765625	1.14237542465225e-10\\
-34.5534765625	9.3998729946739e-11\\
-34.5321875	2.10467652451573e-10\\
-34.5108984375	1.62529977567568e-10\\
-34.489609375	1.5240078342206e-10\\
-34.4683203125	1.81540533922956e-10\\
-34.44703125	1.55602184797522e-10\\
-34.4257421875	1.83755154493279e-10\\
-34.404453125	2.45861509776006e-10\\
-34.3831640625	2.04705523326278e-10\\
-34.361875	2.68849877903463e-10\\
-34.3405859375	3.18041922400845e-10\\
-34.319296875	3.3020004471623e-10\\
-34.2980078125	3.99834632091996e-10\\
-34.27671875	4.09397455588098e-10\\
-34.2554296875	2.58513431823725e-10\\
-34.234140625	3.34789743754314e-10\\
-34.2128515625	4.03537193115129e-10\\
-34.1915625	3.79971384413225e-10\\
-34.1702734375	4.51370586948254e-10\\
-34.148984375	3.86382115990401e-10\\
-34.1276953125	4.62894444052451e-10\\
-34.10640625	4.49709008403468e-10\\
-34.0851171875	5.13911717058081e-10\\
-34.063828125	5.48445397364232e-10\\
-34.0425390625	4.52225604910409e-10\\
-34.02125	5.43797990111742e-10\\
-33.9999609375	5.47088342000215e-10\\
-33.978671875	4.71502601082631e-10\\
-33.9573828125	5.40918512129787e-10\\
-33.93609375	5.52506959914953e-10\\
-33.9148046875	5.23006298599254e-10\\
-33.893515625	5.75596223774596e-10\\
-33.8722265625	5.19173156073476e-10\\
-33.8509375	5.55682121455411e-10\\
-33.8296484375	6.00342663052827e-10\\
-33.808359375	5.93238004112519e-10\\
-33.7870703125	5.9096011053671e-10\\
-33.76578125	5.97241091673187e-10\\
-33.7444921875	5.61642562588183e-10\\
-33.723203125	6.00705067317588e-10\\
-33.7019140625	5.69066997305164e-10\\
-33.680625	6.57780295645397e-10\\
-33.6593359375	5.81095554247498e-10\\
-33.638046875	6.45486744002964e-10\\
-33.6167578125	6.27587049173702e-10\\
-33.59546875	6.13298473579627e-10\\
-33.5741796875	6.87724780452472e-10\\
-33.552890625	6.7702599812177e-10\\
-33.5316015625	6.99872517223023e-10\\
-33.5103125	6.78380213122545e-10\\
-33.4890234375	6.55766317620497e-10\\
-33.467734375	6.78639229287639e-10\\
-33.4464453125	5.7292445699656e-10\\
-33.42515625	5.91680780974767e-10\\
-33.4038671875	6.32839379263763e-10\\
-33.382578125	6.586917494866e-10\\
-33.3612890625	6.498761218253e-10\\
-33.34	7.36776708133014e-10\\
-33.3187109375	7.16082117442211e-10\\
-33.297421875	6.9050321163155e-10\\
-33.2761328125	7.2695718848839e-10\\
-33.25484375	6.29045046389763e-10\\
-33.2335546875	5.20880925485999e-10\\
-33.212265625	5.72116604134964e-10\\
-33.1909765625	5.4124961164109e-10\\
-33.1696875	5.42266212016999e-10\\
-33.1483984375	5.1066446776583e-10\\
-33.127109375	5.43867109473817e-10\\
-33.1058203125	5.63408742620249e-10\\
-33.08453125	6.06281760679858e-10\\
-33.0632421875	6.22164457743285e-10\\
-33.041953125	5.38490193859388e-10\\
-33.0206640625	5.79648332205293e-10\\
-32.999375	5.10208272733699e-10\\
-32.9780859375	4.97421725360186e-10\\
-32.956796875	3.88278000228689e-10\\
-32.9355078125	4.14137680635708e-10\\
-32.91421875	4.01852694842394e-10\\
-32.8929296875	4.25546360520216e-10\\
-32.871640625	4.31635259282382e-10\\
-32.8503515625	4.55550980544529e-10\\
-32.8290625	3.95446206079097e-10\\
-32.8077734375	4.18206939464114e-10\\
-32.786484375	3.83258086583338e-10\\
-32.7651953125	4.2375677514369e-10\\
-32.74390625	3.05705884335284e-10\\
-32.7226171875	2.46399047772104e-10\\
-32.701328125	2.63382500242591e-10\\
-32.6800390625	1.56495777257977e-10\\
-32.65875	2.49963883671324e-10\\
-32.6374609375	2.58255635556337e-10\\
-32.616171875	2.83398648828913e-10\\
-32.5948828125	2.20827458613009e-10\\
-32.57359375	2.19565242862542e-10\\
-32.5523046875	1.71158416396522e-10\\
-32.531015625	1.43553959328685e-10\\
-32.5097265625	3.25996369055984e-11\\
-32.4884375	-4.46215330772838e-11\\
-32.4671484375	-3.02891961819846e-11\\
-32.445859375	-1.65229587066257e-10\\
-32.4245703125	-4.11021732517395e-11\\
-32.40328125	-4.32717711721221e-11\\
-32.3819921875	-4.65077726542774e-11\\
-32.360703125	-1.32089014285406e-11\\
-32.3394140625	-1.76248208475082e-11\\
-32.318125	-3.49413344607201e-11\\
-32.2968359375	-6.80815931675525e-11\\
-32.275546875	-9.7079077488966e-11\\
-32.2542578125	-1.67092352234028e-10\\
-32.23296875	-1.84748631137099e-10\\
-32.2116796875	-2.9848621218436e-10\\
-32.190390625	-1.9736457731181e-10\\
-32.1691015625	-1.06412987927393e-10\\
-32.1478125	-1.97930667438566e-10\\
-32.1265234375	-1.00080111093279e-10\\
-32.105234375	-1.10051412905805e-10\\
-32.0839453125	-1.99365910429166e-10\\
-32.06265625	-2.42656738892781e-10\\
-32.0413671875	-3.49539015470653e-10\\
-32.020078125	-3.761750123355e-10\\
-31.9987890625	-4.57429310308417e-10\\
-31.9775	-3.74891623152464e-10\\
-31.9562109375	-4.69651274385058e-10\\
-31.934921875	-3.48042012439886e-10\\
-31.9136328125	-3.48360390054987e-10\\
-31.89234375	-3.50279706862328e-10\\
-31.8710546875	-3.03576276164842e-10\\
-31.849765625	-4.14951274622114e-10\\
-31.8284765625	-4.31319621843033e-10\\
-31.8071875	-5.21217547424278e-10\\
-31.7858984375	-5.54051165619874e-10\\
-31.764609375	-5.34037005314631e-10\\
-31.7433203125	-5.96008785828467e-10\\
-31.72203125	-6.32775281659699e-10\\
-31.7007421875	-5.47908027926398e-10\\
-31.679453125	-6.4716197621425e-10\\
-31.6581640625	-5.57428692820422e-10\\
-31.636875	-6.5686915836605e-10\\
-31.6155859375	-6.76336201261299e-10\\
-31.594296875	-7.11221513766795e-10\\
-31.5730078125	-7.42449658706572e-10\\
-31.55171875	-7.39221515373505e-10\\
-31.5304296875	-7.65566497098533e-10\\
-31.509140625	-7.36584550880227e-10\\
-31.4878515625	-6.85519451404085e-10\\
-31.4665625	-6.49478843837113e-10\\
-31.4452734375	-6.82928830488023e-10\\
-31.423984375	-6.61023072756211e-10\\
-31.4026953125	-6.41012130393871e-10\\
-31.38140625	-7.06318814685019e-10\\
-31.3601171875	-6.85164509973378e-10\\
-31.338828125	-8.02952677481352e-10\\
-31.3175390625	-8.0076861916106e-10\\
-31.29625	-7.63027036828856e-10\\
-31.2749609375	-8.50354861390898e-10\\
-31.253671875	-7.68548411912124e-10\\
-31.2323828125	-8.03462613742871e-10\\
-31.21109375	-8.34893826900591e-10\\
-31.1898046875	-8.38233116104744e-10\\
-31.168515625	-8.73620185868108e-10\\
-31.1472265625	-8.38730124770186e-10\\
-31.1259375	-8.71175585509518e-10\\
-31.1046484375	-8.99268935477645e-10\\
-31.083359375	-9.83419586215735e-10\\
-31.0620703125	-1.05165299636472e-09\\
-31.04078125	-1.06785410033638e-09\\
-31.0194921875	-1.13283454160014e-09\\
-30.998203125	-1.08996461167398e-09\\
-30.9769140625	-1.20137442459873e-09\\
-30.955625	-1.14912839621937e-09\\
-30.9343359375	-1.12836957230563e-09\\
-30.913046875	-1.26124605607446e-09\\
-30.8917578125	-1.15653920195177e-09\\
-30.87046875	-1.1792223637081e-09\\
-30.8491796875	-1.20934393323987e-09\\
-30.827890625	-1.20118955282119e-09\\
-30.8066015625	-1.14993349930153e-09\\
-30.7853125	-1.21251310336929e-09\\
-30.7640234375	-1.16897665161151e-09\\
-30.742734375	-1.2700503827055e-09\\
-30.7214453125	-1.21752963359724e-09\\
-30.70015625	-1.17111226693096e-09\\
-30.6788671875	-1.11424966846726e-09\\
-30.657578125	-9.92506950836915e-10\\
-30.6362890625	-9.26518601224931e-10\\
-30.615	-8.11630840133294e-10\\
-30.5937109375	-8.28996702416901e-10\\
-30.572421875	-8.74957367312044e-10\\
-30.5511328125	-8.94279758666055e-10\\
-30.52984375	-8.4940199273835e-10\\
-30.5085546875	-9.08172009936147e-10\\
-30.487265625	-8.9586320915953e-10\\
-30.4659765625	-8.10285015540627e-10\\
-30.4446875	-8.17223742656464e-10\\
-30.4233984375	-7.16243920769838e-10\\
-30.402109375	-6.79990977825052e-10\\
-30.3808203125	-5.22285725768849e-10\\
-30.35953125	-5.53175954542076e-10\\
-30.3382421875	-5.22167664065684e-10\\
-30.316953125	-6.00554149018116e-10\\
-30.2956640625	-5.84249480895324e-10\\
-30.274375	-6.27151805608503e-10\\
-30.2530859375	-6.4269203775812e-10\\
-30.231796875	-6.41356138946022e-10\\
-30.2105078125	-5.54808575715302e-10\\
-30.18921875	-5.16325407780366e-10\\
-30.1679296875	-3.79136675189515e-10\\
-30.146640625	-3.56782315495237e-10\\
-30.1253515625	-2.24379052652878e-10\\
-30.1040625	-2.52481856038633e-10\\
-30.0827734375	-2.30760362795792e-10\\
-30.061484375	-2.70038046825945e-10\\
-30.0401953125	-2.78526059173835e-10\\
-30.01890625	-1.87041403150829e-10\\
-29.9976171875	-2.00433714828296e-10\\
-29.976328125	-5.03543570571302e-11\\
-29.9550390625	2.44410131949273e-11\\
-29.93375	3.80313826987905e-11\\
-29.9124609375	1.47544792088889e-10\\
-29.891171875	2.23578628879623e-10\\
-29.8698828125	1.84133693581288e-10\\
-29.84859375	9.97850347027706e-11\\
-29.8273046875	9.61096034670938e-12\\
-29.806015625	1.67643829422165e-11\\
-29.7847265625	7.18444043843812e-11\\
-29.7634375	3.31825495897224e-11\\
-29.7421484375	1.96717737346884e-11\\
-29.720859375	1.20257583785079e-10\\
-29.6995703125	1.64189999263169e-10\\
-29.67828125	1.63133240389436e-10\\
-29.6569921875	2.51794400669931e-10\\
-29.635703125	3.45495642723297e-10\\
-29.6144140625	2.74841976946654e-10\\
-29.593125	2.27442081999254e-10\\
-29.5718359375	2.60479250962361e-10\\
-29.550546875	2.63826685453283e-10\\
-29.5292578125	2.21459486047951e-10\\
-29.50796875	2.23023854628379e-10\\
-29.4866796875	2.21593780057049e-10\\
-29.465390625	1.74309351597191e-10\\
-29.4441015625	2.21671473570382e-10\\
-29.4228125	2.0209561574652e-10\\
-29.4015234375	3.25225317856208e-10\\
-29.380234375	2.32846369300212e-10\\
-29.3589453125	2.62955242642104e-10\\
-29.33765625	3.21703562352977e-10\\
-29.3163671875	2.79145031776855e-10\\
-29.295078125	2.65831405096261e-10\\
-29.2737890625	2.38103670408574e-10\\
-29.2525	2.74565003747673e-10\\
-29.2312109375	3.18454958841506e-10\\
-29.209921875	4.049398915104e-10\\
-29.1886328125	4.54621176035527e-10\\
-29.16734375	5.5288333796369e-10\\
-29.1460546875	5.65314770459273e-10\\
-29.124765625	5.80039866113399e-10\\
-29.1034765625	6.14739592114875e-10\\
-29.0821875	6.65040219031597e-10\\
-29.0608984375	5.85037862593596e-10\\
-29.039609375	6.03853140679303e-10\\
-29.0183203125	6.54080848223514e-10\\
-28.99703125	5.08401480703892e-10\\
-28.9757421875	5.61391471193008e-10\\
-28.954453125	5.69241759779556e-10\\
-28.9331640625	5.65532628148831e-10\\
-28.911875	5.51732195528636e-10\\
-28.8905859375	5.48748600798188e-10\\
-28.869296875	5.48396291762042e-10\\
-28.8480078125	4.98571514101992e-10\\
-28.82671875	4.97563490256092e-10\\
-28.8054296875	4.69713733237431e-10\\
-28.784140625	4.46161048223059e-10\\
-28.7628515625	4.74245704324771e-10\\
-28.7415625	5.85018144809157e-10\\
-28.7202734375	5.10696823774161e-10\\
-28.698984375	4.43943673968295e-10\\
-28.6776953125	5.70290878764262e-10\\
-28.65640625	5.37587543647126e-10\\
-28.6351171875	5.67731612041439e-10\\
-28.613828125	6.54487243865201e-10\\
-28.5925390625	6.10096808017274e-10\\
-28.57125	6.7682826517197e-10\\
-28.5499609375	6.04654943616979e-10\\
-28.528671875	5.99147221332224e-10\\
-28.5073828125	6.74918719475524e-10\\
-28.48609375	6.95468053953108e-10\\
-28.4648046875	7.04843844556747e-10\\
-28.443515625	7.30176627489859e-10\\
-28.4222265625	7.71127264990927e-10\\
-28.4009375	8.24933668200854e-10\\
-28.3796484375	8.15961796913888e-10\\
-28.358359375	8.12226454759248e-10\\
-28.3370703125	7.92526905139794e-10\\
-28.31578125	8.22374683765647e-10\\
-28.2944921875	6.89473236924688e-10\\
-28.273203125	7.16883811699029e-10\\
-28.2519140625	6.18923127553242e-10\\
-28.230625	6.86384383050547e-10\\
-28.2093359375	6.79447664149441e-10\\
-28.188046875	7.82576734357629e-10\\
-28.1667578125	7.77164762643361e-10\\
-28.14546875	8.19238071415611e-10\\
-28.1241796875	8.21559830560266e-10\\
-28.102890625	8.30188382119121e-10\\
-28.0816015625	8.04043981528016e-10\\
-28.0603125	7.41280236525241e-10\\
-28.0390234375	6.4847600798052e-10\\
-28.017734375	6.52620751193711e-10\\
-27.9964453125	5.05792611178727e-10\\
-27.97515625	5.59644251180268e-10\\
-27.9538671875	4.44000505032471e-10\\
-27.932578125	4.87407550590538e-10\\
-27.9112890625	5.20916565278933e-10\\
-27.89	4.98204836490507e-10\\
-27.8687109375	5.56593341677213e-10\\
-27.847421875	5.49284109274862e-10\\
-27.8261328125	5.30851777933246e-10\\
-27.80484375	4.29109439101411e-10\\
-27.7835546875	4.59916277055856e-10\\
-27.762265625	3.122196758428e-10\\
-27.7409765625	4.05981682683579e-10\\
-27.7196875	1.99103770411897e-10\\
-27.6983984375	2.71874834452284e-10\\
-27.677109375	2.637958616845e-10\\
-27.6558203125	3.11778465339923e-10\\
-27.63453125	3.39594396778434e-10\\
-27.6132421875	2.99726970658171e-10\\
-27.591953125	1.94652228896297e-10\\
-27.5706640625	5.52548628794315e-11\\
-27.549375	3.19977681985931e-11\\
-27.5280859375	-3.62830821104444e-11\\
-27.506796875	-1.23615848089424e-10\\
-27.4855078125	-9.45447948955415e-11\\
-27.46421875	-1.2566536708817e-10\\
-27.4429296875	-4.68141349027137e-11\\
-27.421640625	-3.14380305478693e-12\\
-27.4003515625	-4.95703742695615e-12\\
-27.3790625	1.69170038223324e-11\\
-27.3577734375	2.36091184719583e-11\\
-27.336484375	-9.5531408996223e-11\\
-27.3151953125	-2.06413640818431e-10\\
-27.29390625	-3.17921510615855e-10\\
-27.2726171875	-4.12592814276901e-10\\
-27.251328125	-3.78712667043156e-10\\
-27.2300390625	-4.41835175106487e-10\\
-27.20875	-5.1518226163871e-10\\
-27.1874609375	-4.28799681303998e-10\\
-27.166171875	-4.29247192581176e-10\\
-27.1448828125	-4.85850599496122e-10\\
-27.12359375	-3.74663024417796e-10\\
-27.1023046875	-3.46596269518459e-10\\
-27.081015625	-3.82769655194011e-10\\
-27.0597265625	-3.71049549422809e-10\\
-27.0384375	-4.52341821077161e-10\\
-27.0171484375	-5.2211582754254e-10\\
-26.995859375	-5.87226265861116e-10\\
-26.9745703125	-5.48087479556326e-10\\
-26.95328125	-6.02522504923071e-10\\
-26.9319921875	-5.2108913542291e-10\\
-26.910703125	-5.73981563449005e-10\\
-26.8894140625	-4.68123256997848e-10\\
-26.868125	-4.56378103926779e-10\\
-26.8468359375	-5.37529639780062e-10\\
-26.825546875	-5.48292452445951e-10\\
-26.8042578125	-5.26789989862958e-10\\
-26.78296875	-6.39232081120132e-10\\
-26.7616796875	-6.48316270621058e-10\\
-26.740390625	-6.85435465713639e-10\\
-26.7191015625	-7.10923621769163e-10\\
-26.6978125	-7.2376449143951e-10\\
-26.6765234375	-7.4875187511791e-10\\
-26.655234375	-7.45898451085485e-10\\
-26.6339453125	-7.00794296490682e-10\\
-26.61265625	-7.24824178266011e-10\\
-26.5913671875	-8.1589558480257e-10\\
-26.570078125	-7.81062100834167e-10\\
-26.5487890625	-8.57149674824825e-10\\
-26.5275	-9.51596456623694e-10\\
-26.5062109375	-9.2686820136125e-10\\
-26.484921875	-9.55191306620192e-10\\
-26.4636328125	-9.31464862651245e-10\\
-26.44234375	-8.7056795713659e-10\\
-26.4210546875	-8.04887748161187e-10\\
-26.399765625	-7.17302767488324e-10\\
-26.3784765625	-6.90738714156679e-10\\
-26.3571875	-6.99553085869699e-10\\
-26.3358984375	-6.89252790556211e-10\\
-26.314609375	-7.63430463029352e-10\\
-26.2933203125	-7.50705371333276e-10\\
-26.27203125	-7.43280522089009e-10\\
-26.2507421875	-8.41404980967389e-10\\
-26.229453125	-7.73149387536587e-10\\
-26.2081640625	-7.21767304725698e-10\\
-26.186875	-7.28247246077409e-10\\
-26.1655859375	-7.42236963669736e-10\\
-26.144296875	-6.67268457666574e-10\\
-26.1230078125	-6.84244917190179e-10\\
-26.10171875	-7.26581508194422e-10\\
-26.0804296875	-6.78487145652583e-10\\
-26.059140625	-7.10446585815334e-10\\
-26.0378515625	-7.73212488665028e-10\\
-26.0165625	-7.11748937013257e-10\\
-25.9952734375	-7.76949597518846e-10\\
-25.973984375	-7.83854841071482e-10\\
-25.9526953125	-6.75139292527901e-10\\
-25.93140625	-7.6722626790901e-10\\
-25.9101171875	-6.62000455085208e-10\\
-25.888828125	-6.74523128477608e-10\\
-25.8675390625	-8.03578942583696e-10\\
-25.84625	-8.19429675546238e-10\\
-25.8249609375	-8.14930648812436e-10\\
-25.803671875	-1.03535800860768e-09\\
-25.7823828125	-9.55253021172239e-10\\
-25.76109375	-9.54132613535756e-10\\
-25.7398046875	-8.90269975881949e-10\\
-25.718515625	-7.31004744592501e-10\\
-25.6972265625	-6.92616053926774e-10\\
-25.6759375	-6.18713269708745e-10\\
-25.6546484375	-5.74716749131553e-10\\
-25.633359375	-6.18606326011695e-10\\
-25.6120703125	-6.30497467073756e-10\\
-25.59078125	-6.90038858108356e-10\\
-25.5694921875	-7.56457097658585e-10\\
-25.548203125	-6.53303477184721e-10\\
-25.5269140625	-6.77898375354098e-10\\
-25.505625	-5.38788793289291e-10\\
-25.4843359375	-5.33359160227034e-10\\
-25.463046875	-3.85520358599429e-10\\
-25.4417578125	-4.42610059978523e-10\\
-25.42046875	-3.50499393223929e-10\\
-25.3991796875	-2.99217959652858e-10\\
-25.377890625	-4.31370984608631e-10\\
-25.3566015625	-3.37127845773308e-10\\
-25.3353125	-3.46834892140958e-10\\
-25.3140234375	-3.47531073342987e-10\\
-25.292734375	-2.89583833774504e-10\\
-25.2714453125	-2.19374389323976e-10\\
-25.25015625	-2.30468729977303e-10\\
-25.2288671875	-1.93681102875423e-10\\
-25.207578125	-1.96364731480588e-10\\
-25.1862890625	-1.3997995438499e-10\\
-25.165	-2.73122145131845e-10\\
-25.1437109375	-1.99226107637743e-10\\
-25.122421875	-7.45912342751065e-11\\
-25.1011328125	-2.23497148891542e-12\\
-25.07984375	2.15946681264476e-12\\
-25.0585546875	1.8729636592836e-10\\
-25.037265625	1.55090762635868e-10\\
-25.0159765625	3.0419400672369e-10\\
-24.9946875	3.56597620600351e-10\\
-24.9733984375	4.13043022959436e-10\\
-24.952109375	3.57011847014916e-10\\
-24.9308203125	3.86188726943233e-10\\
-24.90953125	4.28689681119769e-10\\
-24.8882421875	4.08321077965231e-10\\
-24.866953125	4.4874549595601e-10\\
-24.8456640625	5.47531406604358e-10\\
-24.824375	5.57679507036517e-10\\
-24.8030859375	5.84871697835507e-10\\
-24.781796875	6.37518733757191e-10\\
-24.7605078125	7.28763425381935e-10\\
-24.73921875	8.1040800826891e-10\\
-24.7179296875	7.90715482549055e-10\\
-24.696640625	6.8903998036336e-10\\
-24.6753515625	8.75167802166288e-10\\
-24.6540625	8.47658057235433e-10\\
-24.6327734375	8.22558671522306e-10\\
-24.611484375	8.79142059449652e-10\\
-24.5901953125	1.0001259390438e-09\\
-24.56890625	1.0002523111013e-09\\
-24.5476171875	1.08695437428291e-09\\
-24.526328125	1.04273953843052e-09\\
-24.5050390625	1.07365321785732e-09\\
-24.48375	1.02851791350689e-09\\
-24.4624609375	9.54483312605443e-10\\
-24.441171875	1.02232891026871e-09\\
-24.4198828125	1.03324520560433e-09\\
-24.39859375	1.00141295621192e-09\\
-24.3773046875	1.16758330133628e-09\\
-24.356015625	1.22053937387838e-09\\
-24.3347265625	1.34637522873864e-09\\
-24.3134375	1.30957993756784e-09\\
-24.2921484375	1.34120784932484e-09\\
-24.270859375	1.29454346472359e-09\\
-24.2495703125	1.13718145496832e-09\\
-24.22828125	1.23707952208981e-09\\
-24.2069921875	1.1407366842662e-09\\
-24.185703125	1.20028421007333e-09\\
-24.1644140625	1.2801029400165e-09\\
-24.143125	1.30162418155656e-09\\
-24.1218359375	1.37267650399856e-09\\
-24.100546875	1.55903553497567e-09\\
-24.0792578125	1.56766617735028e-09\\
-24.05796875	1.54697967634348e-09\\
-24.0366796875	1.40084033698955e-09\\
-24.015390625	1.53650724743525e-09\\
-23.9941015625	1.39203449518779e-09\\
-23.9728125	1.40758525738629e-09\\
-23.9515234375	1.42245738562865e-09\\
-23.930234375	1.35081078419325e-09\\
-23.9089453125	1.36368569200543e-09\\
-23.88765625	1.42189020066828e-09\\
-23.8663671875	1.40600880500724e-09\\
-23.845078125	1.5288610674194e-09\\
-23.8237890625	1.42343868972545e-09\\
-23.8025	1.43734623443336e-09\\
-23.7812109375	1.3855163962777e-09\\
-23.759921875	1.36782525064711e-09\\
-23.7386328125	1.37069044889317e-09\\
-23.71734375	1.37492902011056e-09\\
-23.6960546875	1.42655144794181e-09\\
-23.674765625	1.35789340532703e-09\\
-23.6534765625	1.44033320336917e-09\\
-23.6321875	1.33055560183916e-09\\
-23.6108984375	1.36922806055125e-09\\
-23.589609375	1.38265280809895e-09\\
-23.5683203125	1.35162291933554e-09\\
-23.54703125	1.22849156343757e-09\\
-23.5257421875	1.3980789044272e-09\\
-23.504453125	1.20944933794512e-09\\
-23.4831640625	1.32520833925729e-09\\
-23.461875	1.26890109839731e-09\\
-23.4405859375	1.35137267314331e-09\\
-23.419296875	1.25460793039293e-09\\
-23.3980078125	1.28422415147233e-09\\
-23.37671875	1.26970314047686e-09\\
-23.3554296875	1.43313751150159e-09\\
-23.334140625	1.26601713406266e-09\\
-23.3128515625	1.22739860506334e-09\\
-23.2915625	1.24024368853624e-09\\
-23.2702734375	1.15411887778884e-09\\
-23.248984375	1.19976446108286e-09\\
-23.2276953125	1.07143753656135e-09\\
-23.20640625	1.11804092665073e-09\\
-23.1851171875	1.13321728373108e-09\\
-23.163828125	1.11794106978777e-09\\
-23.1425390625	1.05181198053515e-09\\
-23.12125	1.11624575874647e-09\\
-23.0999609375	1.04351158235173e-09\\
-23.078671875	1.00647760527963e-09\\
-23.0573828125	9.67822555547567e-10\\
-23.03609375	9.83220660484108e-10\\
-23.0148046875	1.09336420650328e-09\\
-22.993515625	1.15268284686659e-09\\
-22.9722265625	1.10723671260572e-09\\
-22.9509375	1.12017422594017e-09\\
-22.9296484375	1.0899988369299e-09\\
-22.908359375	9.18055598556426e-10\\
-22.8870703125	8.39052526319235e-10\\
-22.86578125	8.29520403891003e-10\\
-22.8444921875	6.48906026833202e-10\\
-22.823203125	6.48693337271668e-10\\
-22.8019140625	5.7302696784734e-10\\
-22.780625	6.84743577247674e-10\\
-22.7593359375	7.19038559494095e-10\\
-22.738046875	7.87948175738561e-10\\
-22.7167578125	8.41081180339216e-10\\
-22.69546875	8.33142087254951e-10\\
-22.6741796875	8.88175607501062e-10\\
-22.652890625	7.19937691228698e-10\\
-22.6316015625	6.50774838345136e-10\\
-22.6103125	4.24526716838169e-10\\
-22.5890234375	4.77207079385218e-10\\
-22.567734375	3.23470699475695e-10\\
-22.5464453125	3.81784289891136e-10\\
-22.52515625	3.36178084901272e-10\\
-22.5038671875	4.65912809096063e-10\\
-22.482578125	3.30805233922295e-10\\
-22.4612890625	4.32247730656263e-10\\
-22.44	4.24788313241925e-10\\
-22.4187109375	2.38321059588922e-10\\
-22.397421875	1.03014139566169e-10\\
-22.3761328125	1.2277342497008e-10\\
-22.35484375	6.60073693617214e-11\\
-22.3335546875	1.22198950582845e-10\\
-22.312265625	1.69375371461963e-10\\
-22.2909765625	2.5888179057887e-10\\
-22.2696875	2.31190916462549e-10\\
-22.2483984375	2.81572831846812e-10\\
-22.227109375	1.09775177302805e-10\\
-22.2058203125	1.11873701475947e-10\\
-22.18453125	-4.73104247185898e-12\\
-22.1632421875	-2.67775249254311e-10\\
-22.141953125	-1.63144530930947e-10\\
-22.1206640625	-3.87708727849844e-10\\
-22.099375	-3.15428769829429e-10\\
-22.0780859375	-2.44433141837238e-10\\
-22.056796875	-2.41870421317001e-10\\
-22.0355078125	-3.22853228398898e-10\\
-22.01421875	-4.27627143252774e-10\\
-21.9929296875	-3.75798694178149e-10\\
-21.971640625	-4.38732571081656e-10\\
-21.9503515625	-4.10413942096093e-10\\
-21.9290625	-4.94317122723212e-10\\
-21.9077734375	-4.31132831856275e-10\\
-21.886484375	-4.69241797771434e-10\\
-21.8651953125	-4.49262011308657e-10\\
-21.84390625	-5.50630217623499e-10\\
-21.8226171875	-6.08245213903956e-10\\
-21.801328125	-7.21122724767059e-10\\
-21.7800390625	-6.99434351648002e-10\\
-21.75875	-7.99987005095622e-10\\
-21.7374609375	-9.49134700688765e-10\\
-21.716171875	-9.03281612762082e-10\\
-21.6948828125	-8.26190435094093e-10\\
-21.67359375	-8.76642940924072e-10\\
-21.6523046875	-8.53776177657948e-10\\
-21.631015625	-9.05309143643582e-10\\
-21.6097265625	-1.01652799483514e-09\\
-21.5884375	-1.12693714679831e-09\\
-21.5671484375	-1.23497383514389e-09\\
-21.545859375	-1.25846976956425e-09\\
-21.5245703125	-1.27908163677014e-09\\
-21.50328125	-1.14137008522567e-09\\
-21.4819921875	-1.14094503852131e-09\\
-21.460703125	-1.00389124894296e-09\\
-21.4394140625	-9.99264232672339e-10\\
-21.418125	-8.41568748721211e-10\\
-21.3968359375	-9.87126703718498e-10\\
-21.375546875	-9.69780266040301e-10\\
-21.3542578125	-1.1752828636123e-09\\
-21.33296875	-1.08220220832963e-09\\
-21.3116796875	-1.21168529987795e-09\\
-21.290390625	-1.10181583134472e-09\\
-21.2691015625	-1.02047904115687e-09\\
-21.2478125	-1.11813839468647e-09\\
-21.2265234375	-1.02704198882688e-09\\
-21.205234375	-1.08092311412726e-09\\
-21.1839453125	-1.07903682712854e-09\\
-21.16265625	-1.14511344879845e-09\\
-21.1413671875	-1.09527934237017e-09\\
-21.120078125	-1.12198389121174e-09\\
-21.0987890625	-1.08117036293823e-09\\
-21.0775	-1.049982732106e-09\\
-21.0562109375	-9.27963676060425e-10\\
-21.034921875	-1.02401110268363e-09\\
-21.0136328125	-8.98169041200539e-10\\
-20.99234375	-9.5044308792914e-10\\
-20.9710546875	-8.65377795583154e-10\\
-20.949765625	-1.01809885951896e-09\\
-20.9284765625	-9.53640923707166e-10\\
-20.9071875	-8.3589795311301e-10\\
-20.8858984375	-9.89184189693864e-10\\
-20.864609375	-8.43056656073649e-10\\
-20.8433203125	-6.59310368299746e-10\\
-20.82203125	-5.45154794454723e-10\\
-20.8007421875	-4.84471011166408e-10\\
-20.779453125	-4.04341872033977e-10\\
-20.7581640625	-4.85264215546868e-10\\
-20.736875	-4.95652355573012e-10\\
-20.7155859375	-6.06145984598095e-10\\
-20.694296875	-6.36628500501164e-10\\
-20.6730078125	-6.08286475791147e-10\\
-20.65171875	-7.74238384651559e-10\\
-20.6304296875	-6.10984472842371e-10\\
-20.609140625	-4.41185763675818e-10\\
-20.5878515625	-5.16233257686901e-10\\
-20.5665625	-4.48502947024158e-10\\
-20.5452734375	-3.15488872632191e-10\\
-20.523984375	-3.02503843744751e-10\\
-20.5026953125	-4.93329500398328e-10\\
-20.48140625	-5.12887056213037e-10\\
-20.4601171875	-4.16200182725746e-10\\
-20.438828125	-5.43776149688432e-10\\
-20.4175390625	-5.03386954315628e-10\\
-20.39625	-3.9987468113889e-10\\
-20.3749609375	-2.21771968956987e-10\\
-20.353671875	-2.21691210388963e-10\\
-20.3323828125	-3.08987465000803e-10\\
-20.31109375	-2.09288960919935e-10\\
-20.2898046875	-2.85000628622538e-10\\
-20.268515625	-3.29681643302835e-10\\
-20.2472265625	-4.06289999462279e-10\\
-20.2259375	-4.35183669317589e-10\\
-20.2046484375	-4.74118670850442e-10\\
-20.183359375	-4.84129586172271e-10\\
-20.1620703125	-3.42604927717076e-10\\
-20.14078125	-1.97794739786551e-10\\
-20.1194921875	-1.75851857955901e-10\\
-20.098203125	-8.51520640255799e-11\\
-20.0769140625	-6.5514738563819e-11\\
-20.055625	-7.66902172932335e-11\\
-20.0343359375	-8.50401657502171e-11\\
-20.013046875	-2.65968303985712e-11\\
-19.9917578125	-4.01262404068836e-11\\
-19.97046875	1.17347782960796e-10\\
-19.9491796875	1.76626551108486e-10\\
-19.927890625	2.3859735142042e-10\\
-19.9066015625	3.70131202814693e-10\\
-19.8853125	3.57545998117568e-10\\
-19.8640234375	3.94649504756752e-10\\
-19.842734375	5.00823465378894e-10\\
-19.8214453125	3.77441597589265e-10\\
-19.80015625	5.27881516298416e-10\\
-19.7788671875	5.95518252012361e-10\\
-19.757578125	7.17659513634207e-10\\
-19.7362890625	9.07498044644676e-10\\
-19.715	8.66140668228895e-10\\
-19.6937109375	9.88554236739197e-10\\
-19.672421875	1.12793103670275e-09\\
-19.6511328125	1.12619346663864e-09\\
-19.62984375	1.19639703643391e-09\\
-19.6085546875	1.15402136837386e-09\\
-19.587265625	9.2442455447634e-10\\
-19.5659765625	8.96772197060197e-10\\
-19.5446875	8.41303665160482e-10\\
-19.5233984375	9.06139893917818e-10\\
-19.502109375	8.34113819252887e-10\\
-19.4808203125	9.68759773052088e-10\\
-19.45953125	1.12922321350964e-09\\
-19.4382421875	1.13812802849383e-09\\
-19.416953125	1.37583480046493e-09\\
-19.3956640625	1.44717659309542e-09\\
-19.374375	1.28594349846987e-09\\
-19.3530859375	1.14498697643923e-09\\
-19.331796875	1.09861561743903e-09\\
-19.3105078125	8.93031360296565e-10\\
-19.28921875	8.93424230796791e-10\\
-19.2679296875	7.80709079601962e-10\\
-19.246640625	7.9415499190709e-10\\
-19.2253515625	8.62107441402703e-10\\
-19.2040625	8.97737692619971e-10\\
-19.1827734375	1.0388438366699e-09\\
-19.161484375	1.15743356345498e-09\\
-19.1401953125	1.13555785850538e-09\\
-19.11890625	1.1309906896631e-09\\
-19.0976171875	1.05051481483848e-09\\
-19.076328125	1.11830353281779e-09\\
-19.0550390625	1.02496812023772e-09\\
-19.03375	1.01419860762895e-09\\
-19.0124609375	1.06407986815594e-09\\
-18.991171875	1.224320931822e-09\\
-18.9698828125	1.25642456921497e-09\\
-18.94859375	1.31576118839564e-09\\
-18.9273046875	1.30547376886108e-09\\
-18.906015625	1.35685402089294e-09\\
-18.8847265625	1.25610571044632e-09\\
-18.8634375	1.26197137018767e-09\\
-18.8421484375	1.10247750340567e-09\\
-18.820859375	1.19514561831254e-09\\
-18.7995703125	1.13608004666045e-09\\
-18.77828125	1.09919306615845e-09\\
-18.7569921875	1.26962395694027e-09\\
-18.735703125	1.21757302354823e-09\\
-18.7144140625	1.26828172356081e-09\\
-18.693125	1.15520300324398e-09\\
-18.6718359375	1.17000769019593e-09\\
-18.650546875	1.18685582363253e-09\\
-18.6292578125	1.27456050241158e-09\\
-18.60796875	1.17349376272186e-09\\
-18.5866796875	1.20007655569849e-09\\
-18.565390625	1.18488179517382e-09\\
-18.5441015625	1.22907204115384e-09\\
-18.5228125	1.26946476293798e-09\\
-18.5015234375	1.12817909750558e-09\\
-18.480234375	1.1899167700074e-09\\
-18.4589453125	1.10119097028916e-09\\
-18.43765625	1.14512985779674e-09\\
-18.4163671875	1.07784384366879e-09\\
-18.395078125	1.07671853186235e-09\\
-18.3737890625	1.13695684020698e-09\\
-18.3525	1.0681075905111e-09\\
-18.3312109375	1.02770831451261e-09\\
-18.309921875	1.01102644172405e-09\\
-18.2886328125	1.05834926177249e-09\\
-18.26734375	1.00927905637211e-09\\
-18.2460546875	1.07689962334701e-09\\
-18.224765625	9.7517207946456e-10\\
-18.2034765625	9.99955970234136e-10\\
-18.1821875	9.66360690405734e-10\\
-18.1608984375	8.21702269126991e-10\\
-18.139609375	7.8662420635489e-10\\
-18.1183203125	6.14884324073491e-10\\
-18.09703125	6.65196336278416e-10\\
-18.0757421875	7.53065078105731e-10\\
-18.054453125	6.85756662647044e-10\\
-18.0331640625	7.80148398718169e-10\\
-18.011875	8.54914725353024e-10\\
-17.9905859375	8.53782567963065e-10\\
-17.969296875	6.76874848626343e-10\\
-17.9480078125	7.38968249955265e-10\\
-17.92671875	7.33137028923198e-10\\
-17.9054296875	5.60361605995921e-10\\
-17.884140625	5.51941475474213e-10\\
-17.8628515625	6.11900999628973e-10\\
-17.8415625	5.71599004307822e-10\\
-17.8202734375	4.31616076537517e-10\\
-17.798984375	3.64577726523051e-10\\
-17.7776953125	4.01920129438468e-10\\
-17.75640625	4.76208310695231e-10\\
-17.7351171875	3.18897730068861e-10\\
-17.713828125	4.65877292442257e-10\\
-17.6925390625	4.88392046446444e-10\\
-17.67125	4.23506164497324e-10\\
-17.6499609375	3.16899399632966e-10\\
-17.628671875	3.45094976108866e-10\\
-17.6073828125	1.92492855777737e-10\\
-17.58609375	2.26249405002925e-10\\
-17.5648046875	8.61624397539014e-11\\
-17.543515625	1.13613467520218e-10\\
-17.5222265625	1.39621112584672e-10\\
-17.5009375	-4.62600496611633e-11\\
-17.4796484375	9.24426947997661e-11\\
-17.458359375	-4.71915795094077e-12\\
-17.4370703125	-2.17122256429e-11\\
-17.41578125	-7.81001278353012e-11\\
-17.3944921875	-1.49732235012685e-10\\
-17.373203125	-1.34955717433748e-10\\
-17.3519140625	-2.70558646047897e-10\\
-17.330625	-1.94599307051622e-10\\
-17.3093359375	-1.60681054440851e-10\\
-17.288046875	-2.7384881662355e-10\\
-17.2667578125	-2.26553653995702e-10\\
-17.24546875	-3.08101073840923e-10\\
-17.2241796875	-4.09858098979762e-10\\
-17.202890625	-3.20116894985819e-10\\
-17.1816015625	-6.21252455631733e-10\\
-17.1603125	-6.62033365904184e-10\\
-17.1390234375	-5.9005420678951e-10\\
-17.117734375	-7.21341019018338e-10\\
-17.0964453125	-5.7437462097959e-10\\
-17.07515625	-6.27947057462268e-10\\
-17.0538671875	-7.08803131407994e-10\\
-17.032578125	-6.64550099678514e-10\\
-17.0112890625	-7.05043546655126e-10\\
-16.99	-8.24545199061271e-10\\
-16.9687109375	-7.69086687938541e-10\\
-16.947421875	-7.09947303300517e-10\\
-16.9261328125	-7.58741843923543e-10\\
-16.90484375	-6.31113376833482e-10\\
-16.8835546875	-6.64725140704439e-10\\
-16.862265625	-7.53937125052249e-10\\
-16.8409765625	-7.93907635541819e-10\\
-16.8196875	-9.08349619773849e-10\\
-16.7983984375	-1.0057033898215e-09\\
-16.777109375	-9.88523180243989e-10\\
-16.7558203125	-9.78006591822904e-10\\
-16.73453125	-8.63441556217607e-10\\
-16.7132421875	-8.15607940336127e-10\\
-16.691953125	-7.25049096039673e-10\\
-16.6706640625	-6.20610948841503e-10\\
-16.649375	-6.6262069666402e-10\\
-16.6280859375	-7.43135916963126e-10\\
-16.606796875	-7.96140519153494e-10\\
-16.5855078125	-9.93523991844169e-10\\
-16.56421875	-1.15199556150221e-09\\
-16.5429296875	-1.11628796758074e-09\\
-16.521640625	-1.20777889489801e-09\\
-16.5003515625	-1.24140663287101e-09\\
-16.4790625	-1.05038731830423e-09\\
-16.4577734375	-1.02569223843903e-09\\
-16.436484375	-9.31656823465281e-10\\
-16.4151953125	-1.0799786940052e-09\\
-16.39390625	-9.92849620909729e-10\\
-16.3726171875	-1.03348088181218e-09\\
-16.351328125	-1.09830112537291e-09\\
-16.3300390625	-1.00457630470035e-09\\
-16.30875	-1.09031491977195e-09\\
-16.2874609375	-9.98331240053558e-10\\
-16.266171875	-9.67546764425003e-10\\
-16.2448828125	-8.60111572106286e-10\\
-16.22359375	-7.25554401320209e-10\\
-16.2023046875	-8.42795421978402e-10\\
-16.181015625	-8.05943168748983e-10\\
-16.1597265625	-9.31611008253288e-10\\
-16.1384375	-1.11008866462109e-09\\
-16.1171484375	-1.03189081608973e-09\\
-16.095859375	-1.06489989201869e-09\\
-16.0745703125	-1.03062655376873e-09\\
-16.05328125	-9.29495031506745e-10\\
-16.0319921875	-8.46064116931488e-10\\
-16.010703125	-8.99687369504853e-10\\
-15.9894140625	-7.73994802896248e-10\\
-15.968125	-6.90513105519585e-10\\
-15.9468359375	-8.92534875250168e-10\\
-15.925546875	-6.84165318062578e-10\\
-15.9042578125	-8.18212248345355e-10\\
-15.88296875	-8.87281267023622e-10\\
-15.8616796875	-8.69865027008179e-10\\
-15.840390625	-9.18047357400479e-10\\
-15.8191015625	-1.02083978155338e-09\\
-15.7978125	-8.81117809693575e-10\\
-15.7765234375	-1.03766003990306e-09\\
-15.755234375	-1.00530938614317e-09\\
-15.7339453125	-8.74692672168983e-10\\
-15.71265625	-8.09936541427901e-10\\
-15.6913671875	-6.90374287003927e-10\\
-15.670078125	-7.18548081855543e-10\\
-15.6487890625	-6.94405289294128e-10\\
-15.6275	-7.59074565959877e-10\\
-15.6062109375	-7.19151917758872e-10\\
-15.584921875	-6.58796092004666e-10\\
-15.5636328125	-5.18282459134418e-10\\
-15.54234375	-4.43420231455152e-10\\
-15.5210546875	-5.05363492883434e-10\\
-15.499765625	-3.24070557222181e-10\\
-15.4784765625	-1.98927632225897e-10\\
-15.4571875	-3.39203914587669e-10\\
-15.4358984375	-2.86198079141916e-10\\
-15.414609375	-2.44556707139157e-10\\
-15.3933203125	-3.2238311231807e-10\\
-15.37203125	-3.90030377028983e-10\\
-15.3507421875	-3.51446744466022e-10\\
-15.329453125	-2.50718546040052e-10\\
-15.3081640625	-2.82249414761725e-10\\
-15.286875	-2.39131590373905e-10\\
-15.2655859375	-1.17803730443802e-10\\
-15.244296875	-2.46282150863002e-11\\
-15.2230078125	6.27438439950771e-11\\
-15.20171875	7.32211024126401e-11\\
-15.1804296875	-3.3768003169713e-11\\
-15.159140625	-3.69963180340742e-11\\
-15.1378515625	-4.7636546243828e-11\\
-15.1165625	5.44992508401268e-11\\
-15.0952734375	-1.83531776980161e-11\\
-15.073984375	6.79821863696775e-13\\
-15.0526953125	1.54417359507076e-11\\
-15.03140625	1.15972683772e-10\\
-15.0101171875	1.78569776302372e-10\\
-14.988828125	2.53854782309771e-10\\
-14.9675390625	3.08585562394214e-10\\
-14.94625	3.82321182733261e-10\\
-14.9249609375	2.88397069843787e-10\\
-14.903671875	3.4319411417344e-10\\
-14.8823828125	3.89786785985905e-10\\
-14.86109375	2.93119351484934e-10\\
-14.8398046875	3.71821653779997e-10\\
-14.818515625	3.75759819940761e-10\\
-14.7972265625	4.77610453125512e-10\\
-14.7759375	4.5019183608782e-10\\
-14.7546484375	4.81884640640225e-10\\
-14.733359375	6.91229004777042e-10\\
-14.7120703125	7.43100424486945e-10\\
-14.69078125	7.29346932470424e-10\\
-14.6694921875	9.46793624463925e-10\\
-14.648203125	8.48506056091773e-10\\
-14.6269140625	8.36489375713661e-10\\
-14.605625	9.47560721205441e-10\\
-14.5843359375	8.87283608385325e-10\\
-14.563046875	1.06067948077917e-09\\
-14.5417578125	1.01962741248765e-09\\
-14.52046875	1.14274755902226e-09\\
-14.4991796875	1.36344669889243e-09\\
-14.477890625	1.37212028484499e-09\\
-14.4566015625	1.42165288711146e-09\\
-14.4353125	1.58049624782368e-09\\
-14.4140234375	1.50269630719893e-09\\
-14.392734375	1.3337000901952e-09\\
-14.3714453125	1.37922880119766e-09\\
-14.35015625	1.16205248571705e-09\\
-14.3288671875	1.14678173467863e-09\\
-14.307578125	1.19957186415925e-09\\
-14.2862890625	1.35751536675238e-09\\
-14.265	1.4719743113454e-09\\
-14.2437109375	1.70193265395903e-09\\
-14.222421875	1.68470823979323e-09\\
-14.2011328125	1.83078198539294e-09\\
-14.17984375	1.57875499938189e-09\\
-14.1585546875	1.48680175037113e-09\\
-14.137265625	1.41843267657346e-09\\
-14.1159765625	1.20805110680382e-09\\
-14.0946875	1.13735459134576e-09\\
-14.0733984375	1.23272998263755e-09\\
-14.052109375	1.3090034438459e-09\\
-14.0308203125	1.46208858566141e-09\\
-14.00953125	1.66977203353293e-09\\
-13.9882421875	1.67193798997111e-09\\
-13.966953125	1.75307747829341e-09\\
-13.9456640625	1.67525533178241e-09\\
-13.924375	1.56248546848171e-09\\
-13.9030859375	1.38120260020013e-09\\
-13.881796875	1.28990130604071e-09\\
-13.8605078125	1.16563670770706e-09\\
-13.83921875	1.19566080711726e-09\\
-13.8179296875	1.23135600309897e-09\\
-13.796640625	1.41448579490927e-09\\
-13.7753515625	1.33600541617025e-09\\
-13.7540625	1.59192898325533e-09\\
-13.7327734375	1.481114125982e-09\\
-13.711484375	1.54035728757812e-09\\
-13.6901953125	1.39119378135376e-09\\
-13.66890625	1.26129468032441e-09\\
-13.6476171875	1.27144179531207e-09\\
-13.626328125	1.16483990557786e-09\\
-13.6050390625	1.15827502989058e-09\\
-13.58375	1.2810240207549e-09\\
-13.5624609375	1.28691403451166e-09\\
-13.541171875	1.29557894415388e-09\\
-13.5198828125	1.45207709859566e-09\\
-13.49859375	1.37777250249992e-09\\
-13.4773046875	1.44887387166486e-09\\
-13.456015625	1.31134718578778e-09\\
-13.4347265625	1.19497204763991e-09\\
-13.4134375	1.07351570851954e-09\\
-13.3921484375	8.51608078400382e-10\\
-13.370859375	9.07398500513931e-10\\
-13.3495703125	9.95222500099036e-10\\
-13.32828125	1.01125855826747e-09\\
-13.3069921875	1.08235230918919e-09\\
-13.285703125	1.17057722844064e-09\\
-13.2644140625	1.29450043773576e-09\\
-13.243125	1.11528288368321e-09\\
-13.2218359375	1.3360619597834e-09\\
-13.200546875	1.1869839189665e-09\\
-13.1792578125	1.10277826103214e-09\\
-13.15796875	9.64534122767765e-10\\
-13.1366796875	9.0879636965519e-10\\
-13.115390625	9.00037076043728e-10\\
-13.0941015625	9.86824829490466e-10\\
-13.0728125	8.97063426166857e-10\\
-13.0515234375	1.12613731523965e-09\\
-13.030234375	1.10832942116701e-09\\
-13.0089453125	9.86096140476228e-10\\
-12.98765625	1.05583197960548e-09\\
-12.9663671875	1.0385374185443e-09\\
-12.945078125	8.83762590303873e-10\\
-12.9237890625	7.40954789080543e-10\\
-12.9025	8.12208297370656e-10\\
-12.8812109375	7.89416720203116e-10\\
-12.859921875	8.73284017224412e-10\\
-12.8386328125	8.30793274142765e-10\\
-12.81734375	9.74483695280576e-10\\
-12.7960546875	9.59544079509013e-10\\
-12.774765625	8.41510169172842e-10\\
-12.7534765625	8.24452119737253e-10\\
-12.7321875	8.06552039942643e-10\\
-12.7108984375	6.71024889465116e-10\\
-12.689609375	6.30999383170854e-10\\
-12.6683203125	4.85455434168291e-10\\
-12.64703125	4.7001144854286e-10\\
-12.6257421875	4.12849518917471e-10\\
-12.604453125	5.27224665118736e-10\\
-12.5831640625	4.62952075776215e-10\\
-12.561875	4.94301521669986e-10\\
-12.5405859375	5.19164583047921e-10\\
-12.519296875	4.52047806598441e-10\\
-12.4980078125	5.32919631844338e-10\\
-12.47671875	3.8916876083947e-10\\
-12.4554296875	3.51015237531017e-10\\
-12.434140625	2.57968677155832e-10\\
-12.4128515625	2.47560149510863e-10\\
-12.3915625	2.32471059574361e-10\\
-12.3702734375	2.37440424700981e-10\\
-12.348984375	3.05460713177751e-10\\
-12.3276953125	3.56406513578338e-10\\
-12.30640625	3.4863741179588e-10\\
-12.2851171875	3.96587475942478e-10\\
-12.263828125	2.72746632879911e-10\\
-12.2425390625	2.61118649955198e-10\\
-12.22125	5.74167304686116e-11\\
-12.1999609375	8.09910725282302e-11\\
-12.178671875	-9.50730609928324e-11\\
-12.1573828125	-1.26696028161026e-10\\
-12.13609375	-1.67109935336043e-10\\
-12.1148046875	-2.32137004981735e-10\\
-12.093515625	-1.16727990638627e-10\\
-12.0722265625	-1.91592437503699e-10\\
-12.0509375	-2.6891541198367e-10\\
-12.0296484375	-9.61633241658282e-11\\
-12.008359375	-2.80514293668848e-10\\
-11.9870703125	-3.69659702142613e-10\\
-11.96578125	-4.49777163317829e-10\\
-11.9444921875	-5.9673963948595e-10\\
-11.923203125	-7.14939047516368e-10\\
-11.9019140625	-6.6957327283713e-10\\
-11.880625	-5.90828820344736e-10\\
-11.8593359375	-5.40464899972578e-10\\
-11.838046875	-4.67469616558548e-10\\
-11.8167578125	-4.66537165634387e-10\\
-11.79546875	-3.83850525823751e-10\\
-11.7741796875	-4.83602039659827e-10\\
-11.752890625	-4.77855648841048e-10\\
-11.7316015625	-5.68400177240783e-10\\
-11.7103125	-6.54371931241227e-10\\
-11.6890234375	-8.39736333038222e-10\\
-11.667734375	-7.79615067308655e-10\\
-11.6464453125	-8.53422007529106e-10\\
-11.62515625	-9.51538709675399e-10\\
-11.6038671875	-9.4663860652614e-10\\
-11.582578125	-8.56136135593051e-10\\
-11.5612890625	-8.65587487777361e-10\\
-11.54	-9.22045012297054e-10\\
-11.5187109375	-1.01477277618545e-09\\
-11.497421875	-1.06896827213918e-09\\
-11.4761328125	-1.01544663098236e-09\\
-11.45484375	-1.20230996447005e-09\\
-11.4335546875	-1.19994092317984e-09\\
-11.412265625	-1.12998700782687e-09\\
-11.3909765625	-1.09350258513733e-09\\
-11.3696875	-1.02287089890035e-09\\
-11.3483984375	-1.04861679132692e-09\\
-11.327109375	-9.52892299033259e-10\\
-11.3058203125	-1.04436366378303e-09\\
-11.28453125	-9.06562694147553e-10\\
-11.2632421875	-9.02263080860167e-10\\
-11.241953125	-1.0322694501937e-09\\
-11.2206640625	-9.9190482257896e-10\\
-11.199375	-9.05802239734824e-10\\
-11.1780859375	-1.00140893286167e-09\\
-11.156796875	-1.07569670544116e-09\\
-11.1355078125	-1.03304429063175e-09\\
-11.11421875	-1.13219987629285e-09\\
-11.0929296875	-1.0663656322418e-09\\
-11.071640625	-9.67368516642671e-10\\
-11.0503515625	-9.01328038957114e-10\\
-11.0290625	-9.60461784450193e-10\\
-11.0077734375	-9.62462844132425e-10\\
-10.986484375	-1.0154311959053e-09\\
-10.9651953125	-1.03978928177667e-09\\
-10.94390625	-1.11301911037109e-09\\
-10.9226171875	-1.09068072259365e-09\\
-10.901328125	-1.15358576189004e-09\\
-10.8800390625	-9.99608714322299e-10\\
-10.85875	-1.10367376669175e-09\\
-10.8374609375	-8.77877467117294e-10\\
-10.816171875	-8.00367146301521e-10\\
-10.7948828125	-8.43511529269509e-10\\
-10.77359375	-8.85954761876426e-10\\
-10.7523046875	-9.21205450637779e-10\\
-10.731015625	-8.7764808182915e-10\\
-10.7097265625	-1.0220017673185e-09\\
-10.6884375	-7.96382313984008e-10\\
-10.6671484375	-8.34706681508137e-10\\
-10.645859375	-7.87011849110382e-10\\
-10.6245703125	-6.84156954915035e-10\\
-10.60328125	-7.1871094895049e-10\\
-10.5819921875	-7.53534749780331e-10\\
-10.560703125	-7.64308475501884e-10\\
-10.5394140625	-8.55583068331273e-10\\
-10.518125	-8.04714699715145e-10\\
-10.4968359375	-6.95601258865437e-10\\
-10.475546875	-7.11943932018046e-10\\
-10.4542578125	-6.54370275643979e-10\\
-10.43296875	-5.4280336434626e-10\\
-10.4116796875	-5.50211483274911e-10\\
-10.390390625	-5.38335198332093e-10\\
-10.3691015625	-5.87934321084095e-10\\
-10.3478125	-5.29845302244768e-10\\
-10.3265234375	-6.156195816352e-10\\
-10.305234375	-6.98095238837546e-10\\
-10.2839453125	-6.81348937718506e-10\\
-10.26265625	-5.55950165735194e-10\\
-10.2413671875	-4.78918548819377e-10\\
-10.220078125	-4.65411264005737e-10\\
-10.1987890625	-3.47213973138648e-10\\
-10.1775	-3.16212804739505e-10\\
-10.1562109375	-3.07896154885029e-10\\
-10.134921875	-3.08257988019203e-10\\
-10.1136328125	-2.81577199175255e-10\\
-10.09234375	-2.95710018710462e-10\\
-10.0710546875	-2.89324952423032e-10\\
-10.049765625	-2.94020761112192e-10\\
-10.0284765625	-1.95134338461519e-10\\
-10.0071875	-2.28587580729578e-10\\
-9.9858984375	-1.19263376656248e-10\\
-9.96460937499999	-4.708074687751e-11\\
-9.9433203125	-1.26605887118457e-10\\
-9.92203125	2.18829502248947e-11\\
-9.9007421875	-1.01281268456865e-10\\
-9.879453125	-2.67149579854067e-11\\
-9.85816406249999	-1.61321050021085e-10\\
-9.836875	-1.42239832449291e-10\\
-9.8155859375	-5.55261723334784e-11\\
-9.794296875	-3.53944154464422e-11\\
-9.7730078125	5.90877857193006e-12\\
-9.75171874999999	6.9271836268622e-11\\
-9.7304296875	7.96759553602951e-11\\
-9.709140625	1.70031927953081e-10\\
-9.6878515625	2.20031059820131e-10\\
-9.6665625	2.09843445314674e-10\\
-9.64527343749999	2.26540117493316e-10\\
-9.623984375	2.99753354646698e-10\\
-9.6026953125	2.50952620062246e-10\\
-9.58140625	2.48581339152606e-10\\
-9.5601171875	3.77864418154425e-10\\
-9.53882812499999	3.83514788287047e-10\\
-9.5175390625	3.41477849912814e-10\\
-9.49625	5.19608071808227e-10\\
-9.4749609375	5.96462873491265e-10\\
-9.453671875	7.0772205676476e-10\\
-9.43238281249999	7.34004095132275e-10\\
-9.41109375	8.21124255475726e-10\\
-9.3898046875	7.01934548915513e-10\\
-9.368515625	8.58415522758201e-10\\
-9.3472265625	6.546796704195e-10\\
-9.32593749999999	7.14407927145915e-10\\
-9.3046484375	7.63525032737861e-10\\
-9.283359375	7.2251713465287e-10\\
-9.2620703125	8.966768941194e-10\\
-9.24078125	8.62068682874115e-10\\
-9.21949218749999	1.0056960222749e-09\\
-9.198203125	9.86892158429021e-10\\
-9.1769140625	9.73187157058341e-10\\
-9.155625	9.46473987281571e-10\\
-9.1343359375	1.01770586219465e-09\\
-9.11304687499999	9.96957282004654e-10\\
-9.0917578125	1.01022068233528e-09\\
-9.07046875	9.59197097061761e-10\\
-9.0491796875	9.98660120463684e-10\\
-9.027890625	1.0917205278365e-09\\
-9.00660156249999	1.1654109773781e-09\\
-8.9853125	1.1743268312076e-09\\
-8.9640234375	1.33215272062388e-09\\
-8.942734375	1.3665522703259e-09\\
-8.9214453125	1.30144420818832e-09\\
-8.90015624999999	1.22308098802571e-09\\
-8.8788671875	1.28306851846488e-09\\
-8.857578125	1.09381503406578e-09\\
-8.8362890625	1.0402038333158e-09\\
-8.815	1.06907066704501e-09\\
-8.79371093749999	9.92110038324565e-10\\
-8.772421875	1.06700084132557e-09\\
-8.7511328125	1.12254263820679e-09\\
-8.72984375	1.34283005935849e-09\\
-8.7085546875	1.3455945739815e-09\\
-8.68726562499999	1.37430230297746e-09\\
-8.6659765625	1.45422032160038e-09\\
-8.6446875	1.42913438195897e-09\\
-8.6233984375	1.23053775142471e-09\\
-8.602109375	1.31837094253211e-09\\
-8.58082031249999	1.08026429791279e-09\\
-8.55953125	1.07170314529461e-09\\
-8.5382421875	9.83966725321416e-10\\
-8.516953125	1.10732014744053e-09\\
-8.4956640625	1.04264762639724e-09\\
-8.47437499999999	1.09966290182847e-09\\
-8.4530859375	1.23794421104502e-09\\
-8.431796875	1.24379867393159e-09\\
-8.4105078125	1.27023854914256e-09\\
-8.38921875	1.20648049394504e-09\\
-8.36792968749999	1.13697225364788e-09\\
-8.346640625	9.58407150296346e-10\\
-8.3253515625	8.82232230547668e-10\\
-8.3040625	9.09930433421833e-10\\
-8.2827734375	1.03727437932942e-09\\
-8.26148437499999	1.06706898694184e-09\\
-8.2401953125	1.04046937550912e-09\\
-8.21890625	1.07995703954292e-09\\
-8.1976171875	1.1364360545858e-09\\
-8.176328125	9.03116668666717e-10\\
-8.15503906249999	1.04193881496791e-09\\
-8.13375	8.95682713393341e-10\\
-8.1124609375	7.45299812240618e-10\\
-8.091171875	7.08618176242894e-10\\
-8.0698828125	6.33276392485898e-10\\
-8.04859374999999	6.07683098456227e-10\\
-8.0273046875	6.92893009154731e-10\\
-8.006015625	7.52490209271775e-10\\
-7.9847265625	6.42352744151973e-10\\
-7.9634375	8.29838216873756e-10\\
-7.94214843749999	7.74530178303546e-10\\
-7.920859375	7.26234354127294e-10\\
-7.8995703125	7.28222190506501e-10\\
-7.87828125	6.74009110288232e-10\\
-7.8569921875	5.0304666557328e-10\\
-7.83570312499999	5.12039460470822e-10\\
-7.8144140625	4.5224860312784e-10\\
-7.793125	4.35334842826986e-10\\
-7.7718359375	4.93984652619434e-10\\
-7.750546875	4.73650145935852e-10\\
-7.72925781249999	4.43007304329691e-10\\
-7.70796875	5.11428982831449e-10\\
-7.6866796875	4.28145280761605e-10\\
-7.665390625	3.81173317978375e-10\\
-7.6441015625	3.34625911866268e-10\\
-7.62281249999999	3.15270456208015e-10\\
-7.6015234375	1.98598346823362e-10\\
-7.580234375	1.68815558846939e-10\\
-7.5589453125	1.36469748587617e-10\\
-7.53765625	6.2006717419707e-11\\
-7.51636718749999	7.92096997253999e-11\\
-7.495078125	1.82775307434544e-10\\
-7.4737890625	2.10796804558713e-10\\
-7.4525	1.59392328581519e-10\\
-7.4312109375	1.14201259425887e-10\\
-7.40992187499999	3.74212273027816e-11\\
-7.3886328125	4.65427188941287e-11\\
-7.36734375	-1.33705339717194e-10\\
-7.3460546875	-1.83580781737012e-10\\
-7.324765625	-1.78078204338814e-10\\
-7.30347656249999	-2.21549224496709e-10\\
-7.2821875	-2.48880900792317e-10\\
-7.2608984375	-9.62984369337344e-11\\
-7.239609375	-6.45067901968666e-11\\
-7.2183203125	-1.15516522514323e-10\\
-7.19703124999999	-1.66453071690114e-10\\
-7.1757421875	-1.83611904305096e-10\\
-7.154453125	-2.06013124616401e-10\\
-7.1331640625	-3.79531581021478e-10\\
-7.111875	-4.158364540135e-10\\
-7.09058593749999	-3.16389228105064e-10\\
-7.069296875	-2.97620543114324e-10\\
-7.0480078125	-2.78894640730302e-10\\
-7.02671875	-2.52145667141744e-10\\
-7.0054296875	-2.08734485806748e-10\\
-6.98414062499999	-3.70878752532857e-10\\
-6.9628515625	-4.79186181957664e-10\\
-6.9415625	-4.31857339840987e-10\\
-6.9202734375	-5.71595050621388e-10\\
-6.898984375	-4.89148056066812e-10\\
-6.87769531249999	-6.02074260331454e-10\\
-6.85640625	-5.46729459917458e-10\\
-6.8351171875	-3.4131600778061e-10\\
-6.813828125	-3.90986123168957e-10\\
-6.7925390625	-2.83047446935183e-10\\
-6.77124999999999	-2.2377689156715e-10\\
-6.7499609375	-3.60257539128449e-10\\
-6.728671875	-4.61496866675905e-10\\
-6.7073828125	-6.47527204942939e-10\\
-6.68609375	-6.62805002707684e-10\\
-6.66480468749999	-7.52104547388479e-10\\
-6.643515625	-7.73561690812686e-10\\
-6.6222265625	-7.35919259139098e-10\\
-6.6009375	-5.57664476713952e-10\\
-6.5796484375	-5.53192377512442e-10\\
-6.55835937499999	-5.6086999606289e-10\\
-6.5370703125	-6.04273059635795e-10\\
-6.51578125	-8.38613830535684e-10\\
-6.4944921875	-8.78523899744953e-10\\
-6.473203125	-1.00974753822389e-09\\
-6.45191406249999	-1.15812873253964e-09\\
-6.430625	-1.10495672032533e-09\\
-6.4093359375	-1.04066825150339e-09\\
-6.388046875	-8.6871489755663e-10\\
-6.3667578125	-7.47725877705578e-10\\
-6.34546874999999	-5.61994804148273e-10\\
-6.3241796875	-5.03044724517167e-10\\
-6.302890625	-5.41612644745638e-10\\
-6.2816015625	-6.71071903727668e-10\\
-6.2603125	-8.88566038916699e-10\\
-6.23902343749999	-9.94103785487125e-10\\
-6.217734375	-1.18106142214703e-09\\
-6.1964453125	-1.13643869245944e-09\\
-6.17515625	-1.2055836180405e-09\\
-6.1538671875	-1.11936020833967e-09\\
-6.13257812499999	-1.00666154733727e-09\\
-6.1112890625	-1.01306363243527e-09\\
-6.09	-9.98198886647924e-10\\
-6.0687109375	-9.65588510086379e-10\\
-6.047421875	-1.01082505489096e-09\\
-6.02613281249999	-1.18364540479081e-09\\
-6.00484375	-1.25113135101292e-09\\
-5.9835546875	-1.20371828882224e-09\\
-5.962265625	-1.28561735493354e-09\\
-5.9409765625	-1.28968452717737e-09\\
-5.91968749999999	-1.25606006991218e-09\\
-5.8983984375	-1.27274502817345e-09\\
-5.877109375	-1.22726639937821e-09\\
-5.8558203125	-1.25445653818715e-09\\
-5.83453125	-1.24014795241964e-09\\
-5.81324218749999	-1.08938011575619e-09\\
-5.791953125	-1.34749476222607e-09\\
-5.7706640625	-1.3336484882047e-09\\
-5.749375	-1.42948656049344e-09\\
-5.7280859375	-1.24358878111473e-09\\
-5.70679687499999	-1.19351848214461e-09\\
-5.6855078125	-1.13231211617192e-09\\
-5.66421875	-9.56425618812447e-10\\
-5.6429296875	-8.96542520317592e-10\\
-5.621640625	-7.5227788883321e-10\\
-5.60035156249999	-7.5466593301428e-10\\
-5.5790625	-7.73124676066316e-10\\
-5.5577734375	-8.45622348528341e-10\\
-5.536484375	-9.01285141082782e-10\\
-5.5151953125	-9.85011844909367e-10\\
-5.49390624999999	-9.78117767844926e-10\\
-5.4726171875	-9.75826412799458e-10\\
-5.451328125	-9.78059572321944e-10\\
-5.4300390625	-8.95921345541262e-10\\
-5.40875	-9.35143927832717e-10\\
-5.38746093749999	-8.53182848597567e-10\\
-5.366171875	-7.10687416376263e-10\\
-5.3448828125	-6.99608528459319e-10\\
-5.32359375	-7.21991510097816e-10\\
-5.3023046875	-7.37713721677392e-10\\
-5.28101562499999	-7.3731895837551e-10\\
-5.2597265625	-7.31593023497229e-10\\
-5.2384375	-7.75026170141391e-10\\
-5.2171484375	-7.38295495155563e-10\\
-5.195859375	-7.60051278405139e-10\\
-5.17457031249999	-6.24537356180218e-10\\
-5.15328125	-4.5688207042121e-10\\
-5.1319921875	-4.12308001134659e-10\\
-5.110703125	-3.69526544372217e-10\\
-5.0894140625	-2.79771648399612e-10\\
-5.06812499999999	-3.38472392094737e-10\\
-5.0468359375	-4.88806017596834e-10\\
-5.025546875	-5.49136529027866e-10\\
-5.0042578125	-5.04739463824323e-10\\
-4.98296875	-6.42322858505204e-10\\
-4.96167968749999	-4.85452214068252e-10\\
-4.940390625	-3.5952137779614e-10\\
-4.9191015625	-2.41450140991367e-10\\
-4.8978125	-1.11192649151217e-10\\
-4.8765234375	-1.10859498522826e-11\\
-4.85523437499999	1.29030148343892e-10\\
-4.8339453125	6.01669189946011e-11\\
-4.81265625	-6.4588709287777e-12\\
-4.7913671875	-3.98264572435302e-13\\
-4.770078125	-1.36098925160132e-10\\
-4.74878906249999	-1.47847725458109e-10\\
-4.7275	-4.08677116594368e-11\\
-4.7062109375	2.62145751771183e-11\\
-4.684921875	1.91433547011119e-10\\
-4.6636328125	1.64004351382855e-10\\
-4.64234374999999	3.72758263818626e-10\\
-4.6210546875	4.96162013567855e-10\\
-4.599765625	4.75706942570772e-10\\
-4.5784765625	5.21760214449851e-10\\
-4.5571875	5.12430944679519e-10\\
-4.53589843749999	4.7255347023517e-10\\
-4.514609375	4.3047455821075e-10\\
-4.4933203125	4.09905657803013e-10\\
-4.47203125	5.01283477740528e-10\\
-4.4507421875	6.11678472265368e-10\\
-4.42945312499999	5.81554313249846e-10\\
-4.4081640625	7.79160226218718e-10\\
-4.386875	7.54343667578448e-10\\
-4.3655859375	7.48285366170112e-10\\
-4.344296875	6.68303794712086e-10\\
-4.32300781249999	6.08761138837556e-10\\
-4.30171875	6.07737035963235e-10\\
-4.2804296875	5.05766118269159e-10\\
-4.259140625	5.32416531246934e-10\\
-4.2378515625	6.58639680241238e-10\\
-4.21656249999999	7.78839971238476e-10\\
-4.1952734375	8.18711500249374e-10\\
-4.173984375	9.44236555142841e-10\\
-4.1526953125	1.06908449258044e-09\\
-4.13140625	1.07373185074823e-09\\
-4.11011718749999	8.95067684170688e-10\\
-4.088828125	9.24140768517609e-10\\
-4.0675390625	8.7092752799082e-10\\
-4.04625	8.12712787935434e-10\\
-4.0249609375	9.82708758543256e-10\\
-4.00367187499999	1.20626569743739e-09\\
-3.9823828125	1.17969841487679e-09\\
-3.96109375	1.22942609591602e-09\\
-3.9398046875	1.41317260171039e-09\\
-3.918515625	1.47573472242442e-09\\
-3.89722656249999	1.48772665039882e-09\\
-3.8759375	1.40964027536141e-09\\
-3.8546484375	1.51276837376369e-09\\
-3.833359375	1.37460725600085e-09\\
-3.8120703125	1.44842040956903e-09\\
-3.79078124999999	1.46638879780688e-09\\
-3.7694921875	1.55282210579425e-09\\
-3.748203125	1.67749721139723e-09\\
-3.7269140625	1.74050767749176e-09\\
-3.705625	1.80366579050176e-09\\
-3.68433593749999	1.81787642799359e-09\\
-3.663046875	1.82848909654935e-09\\
-3.6417578125	1.81135052921969e-09\\
-3.62046875	1.82777400496118e-09\\
-3.5991796875	1.80522328865544e-09\\
-3.57789062499999	1.82253793514363e-09\\
-3.5566015625	1.76457933262651e-09\\
-3.5353125	1.92235819567355e-09\\
-3.5140234375	1.92130572126463e-09\\
-3.492734375	1.84983695211217e-09\\
-3.47144531249999	1.8155467434936e-09\\
-3.45015625	1.81054987631068e-09\\
-3.4288671875	1.67928753054351e-09\\
-3.407578125	1.82502739831935e-09\\
-3.3862890625	1.85645066111662e-09\\
-3.36499999999999	1.80840880883684e-09\\
-3.3437109375	1.83596101070333e-09\\
-3.322421875	1.84347509099767e-09\\
-3.3011328125	1.70038034028087e-09\\
-3.27984375	1.71320953102651e-09\\
-3.25855468749999	1.63335303398749e-09\\
-3.237265625	1.54608292981106e-09\\
-3.2159765625	1.58375923586625e-09\\
-3.1946875	1.59847060247276e-09\\
-3.1733984375	1.57003543883165e-09\\
-3.15210937499999	1.45915586815736e-09\\
-3.1308203125	1.42694676355096e-09\\
-3.10953125	1.33261748944926e-09\\
-3.0882421875	1.53445571176724e-09\\
-3.066953125	1.51677998029302e-09\\
-3.04566406249999	1.41259563088024e-09\\
-3.024375	1.64008051278615e-09\\
-3.0030859375	1.62471315235662e-09\\
-2.981796875	1.6362651386774e-09\\
-2.9605078125	1.56047221570432e-09\\
-2.93921874999999	1.61312944515454e-09\\
-2.9179296875	1.39532540024809e-09\\
-2.896640625	1.50698856946656e-09\\
-2.8753515625	1.39046716204569e-09\\
-2.8540625	1.59196439087534e-09\\
-2.83277343749999	1.64678806192468e-09\\
-2.811484375	1.61578790536579e-09\\
-2.7901953125	1.80008287164458e-09\\
-2.76890625	1.7124212969284e-09\\
-2.7476171875	1.82047708374293e-09\\
-2.72632812499999	1.74460570991845e-09\\
-2.7050390625	1.72234183215769e-09\\
-2.68375	1.76027769183256e-09\\
-2.6624609375	1.5173425341708e-09\\
-2.641171875	1.63179526124613e-09\\
-2.61988281249999	1.63145531884013e-09\\
-2.59859375	1.76325321483849e-09\\
-2.5773046875	1.5912038236953e-09\\
-2.556015625	1.59186551761003e-09\\
-2.5347265625	1.66014436423286e-09\\
-2.51343749999999	1.35448621862187e-09\\
-2.4921484375	1.29220971808578e-09\\
-2.470859375	1.25769625538306e-09\\
-2.4495703125	1.14438704016381e-09\\
-2.42828125	1.02805079183181e-09\\
-2.40699218749999	1.11519596576804e-09\\
-2.385703125	1.09721679986882e-09\\
-2.3644140625	1.07482689601558e-09\\
-2.343125	1.05269755237097e-09\\
-2.3218359375	1.14500598851407e-09\\
-2.30054687499999	8.89444208290613e-10\\
-2.2792578125	9.04136981636834e-10\\
-2.25796875	7.57399556259211e-10\\
-2.2366796875	7.20445733061443e-10\\
-2.215390625	6.4187092544335e-10\\
-2.19410156249999	6.06264154191585e-10\\
-2.1728125	7.24022710915097e-10\\
-2.1515234375	5.04739413512345e-10\\
-2.130234375	5.25487609936641e-10\\
-2.1089453125	3.04151914339804e-10\\
-2.08765624999999	2.5148740005559e-10\\
-2.0663671875	1.24472993395214e-10\\
-2.045078125	2.70049146839023e-10\\
-2.0237890625	1.69451570349131e-10\\
-2.0025	2.86431793914573e-10\\
-1.98121093749999	3.98766203843643e-10\\
-1.959921875	3.51615922296071e-10\\
-1.9386328125	4.02155793445059e-10\\
-1.91734375	2.74701538019859e-10\\
-1.8960546875	3.77165053055109e-10\\
-1.87476562499999	1.17456835445735e-10\\
-1.8534765625	4.66058550116955e-11\\
-1.8321875	2.19269432787142e-11\\
-1.8108984375	1.27027489644534e-10\\
-1.789609375	1.68238297199375e-12\\
-1.76832031249999	2.54989033965589e-10\\
-1.74703125	2.18418038549861e-10\\
-1.7257421875	2.86707082179468e-10\\
-1.704453125	3.04931070655957e-10\\
-1.6831640625	3.88648015437645e-10\\
-1.66187499999999	2.10823308412001e-10\\
-1.6405859375	3.0156152586071e-10\\
-1.619296875	3.76862682954422e-10\\
-1.5980078125	2.94036397118999e-10\\
-1.57671875	7.49721327232663e-11\\
-1.55542968749999	1.28078056180863e-10\\
-1.534140625	1.96037266229951e-11\\
-1.5128515625	-1.58143912293171e-10\\
-1.4915625	-1.77134503439737e-10\\
-1.4702734375	-1.30706431539197e-10\\
-1.44898437499999	-2.84906750758388e-10\\
-1.4276953125	-3.90908580228779e-10\\
-1.40640625	-3.70797148387089e-10\\
-1.3851171875	-4.35142625805482e-10\\
-1.363828125	-5.36196756873695e-10\\
-1.34253906249999	-6.46180041700959e-10\\
-1.32125	-5.39591237658471e-10\\
-1.2999609375	-4.10755812706336e-10\\
-1.278671875	-4.11612632896822e-10\\
-1.2573828125	-3.79083442374607e-10\\
-1.23609374999999	-4.40143650522839e-10\\
-1.2148046875	-4.77930720408191e-10\\
-1.193515625	-7.49884369736789e-10\\
-1.1722265625	-7.48255125744435e-10\\
-1.1509375	-7.66771126404274e-10\\
-1.12964843749999	-7.88777872501105e-10\\
-1.108359375	-7.55898279431772e-10\\
-1.0870703125	-6.46670205893926e-10\\
-1.06578125	-4.4990299811293e-10\\
-1.0444921875	-4.1095903594589e-10\\
-1.02320312499999	-3.37435464665737e-10\\
-1.0019140625	-2.39174513231379e-10\\
-0.980624999999996	-3.30527330574838e-10\\
-0.959335937500001	-5.54773079093903e-10\\
-0.938046874999998	-4.92404169640213e-10\\
-0.916757812499995	-5.69886091003184e-10\\
-0.895468749999999	-6.41932945537864e-10\\
-0.874179687499996	-5.76633946340881e-10\\
-0.852890625000001	-3.79495222657904e-10\\
-0.831601562499998	-4.18223504751345e-10\\
-0.810312499999995	-1.46855279031556e-10\\
-0.789023437499999	-5.97016422421557e-11\\
-0.767734374999996	-5.51386300252021e-12\\
-0.746445312500001	-5.61750021611966e-11\\
-0.725156249999998	-1.53838144047269e-10\\
-0.703867187499995	-3.1759181609178e-10\\
-0.682578124999999	-3.41295545809179e-10\\
-0.661289062499996	-2.0557455308315e-10\\
-0.640000000000001	-2.19439532172439e-10\\
-0.618710937499998	-1.63730074239787e-11\\
-0.597421874999995	4.79476671144098e-11\\
-0.576132812499999	3.72067332603516e-11\\
-0.554843749999996	3.5057746043647e-10\\
-0.533554687500001	2.39498896397493e-10\\
-0.512265624999998	1.21576881381203e-10\\
-0.490976562499995	2.18557298134623e-10\\
-0.469687499999999	-8.05749056171173e-11\\
-0.448398437499996	-1.8175532834768e-10\\
-0.427109375000001	-1.31744212369158e-10\\
-0.405820312499998	-9.11427911946386e-11\\
-0.384531249999995	-6.6191713147101e-11\\
-0.363242187499999	1.54681888096111e-10\\
-0.341953124999996	2.63247716583206e-10\\
-0.320664062500001	4.04273530079893e-10\\
-0.299374999999998	3.85339904474149e-10\\
-0.278085937499995	4.87174092953813e-10\\
-0.256796874999999	4.11006023272987e-10\\
-0.235507812499996	2.35852266929631e-10\\
-0.214218750000001	1.98243326338886e-10\\
-0.192929687499998	7.84639496073747e-11\\
-0.171640624999995	6.59663383809524e-11\\
-0.150351562499999	2.10802606522308e-10\\
-0.129062499999996	3.0716020747084e-10\\
-0.107773437500001	7.87135795132734e-10\\
-0.0864843749999977	8.23785227908103e-10\\
-0.0651953124999949	1.13204904851568e-09\\
-0.0439062499999991	1.25043783179802e-09\\
-0.0226171874999963	1.24680578329839e-09\\
-0.00132812500000057	1.22364818061891e-09\\
0.0199609375000023	1.31322221586539e-09\\
0.0412500000000051	1.20717427573132e-09\\
0.0625390625000009	1.34239718823273e-09\\
0.0838281250000037	1.44338979512521e-09\\
0.105117187499999	1.6058598812599e-09\\
0.126406250000002	1.78608277576464e-09\\
0.147695312500005	2.02534827384568e-09\\
0.168984375000001	2.14964312083621e-09\\
0.190273437500004	2.29555083800449e-09\\
0.211562499999999	2.30426139963212e-09\\
0.232851562500002	2.2201616659809e-09\\
0.254140625000005	2.41306697852907e-09\\
0.275429687500001	2.41760151515823e-09\\
0.296718750000004	2.32763740065648e-09\\
0.318007812499999	2.46309457515716e-09\\
0.339296875000002	2.72880665693602e-09\\
0.360585937500005	2.88304234956547e-09\\
0.381875000000001	2.97982791042881e-09\\
0.403164062500004	3.29929012690009e-09\\
0.424453124999999	3.36781096858981e-09\\
0.445742187500002	3.47348717724293e-09\\
0.467031250000005	3.63488067967484e-09\\
0.488320312500001	3.69032821369129e-09\\
0.509609375000004	3.56559011477187e-09\\
0.530898437499999	3.61910686847574e-09\\
0.552187500000002	3.71236570987807e-09\\
0.573476562500005	3.94124248789241e-09\\
0.594765625000001	3.9019214223769e-09\\
0.616054687500004	4.06506088760242e-09\\
0.637343749999999	4.31312403524096e-09\\
0.658632812500002	4.46986668936171e-09\\
0.679921875000005	4.6324149230293e-09\\
0.701210937500001	4.82680532065087e-09\\
0.722500000000004	4.82658878019181e-09\\
0.743789062499999	4.66641386637561e-09\\
0.765078125000002	4.4848390749276e-09\\
0.786367187500005	4.55290940724249e-09\\
0.807656250000001	4.72618701910768e-09\\
0.828945312500004	4.82657567705054e-09\\
0.850234374999999	5.11270500750173e-09\\
0.871523437500002	5.18292573527442e-09\\
0.892812500000005	5.47421659399586e-09\\
0.914101562500001	5.64006034652268e-09\\
0.935390625000004	5.64394464282397e-09\\
0.956679687499999	5.70326075473248e-09\\
0.977968750000002	5.77680802598876e-09\\
0.999257812500005	5.83323992898653e-09\\
1.020546875	5.81981682145955e-09\\
1.0418359375	5.97079247065765e-09\\
1.063125	6.29083630276289e-09\\
1.0844140625	6.46840896317255e-09\\
1.10570312500001	6.72008605942225e-09\\
1.1269921875	6.90582015745282e-09\\
1.14828125	7.16970482893648e-09\\
1.1695703125	7.2600721703507e-09\\
1.190859375	7.26727874275482e-09\\
1.21214843750001	7.34343491453929e-09\\
1.2334375	7.4105640094491e-09\\
1.2547265625	7.29528494272607e-09\\
1.276015625	7.48241581598291e-09\\
1.2973046875	7.74538559906703e-09\\
1.31859375000001	8.05463235197611e-09\\
1.3398828125	8.25037717748588e-09\\
1.361171875	8.45975433337933e-09\\
1.3824609375	8.72851146788335e-09\\
1.40375	8.62670642827232e-09\\
1.42503906250001	8.61791774808329e-09\\
1.446328125	8.54520893820689e-09\\
1.4676171875	8.30175911931206e-09\\
1.48890625	8.35902382477191e-09\\
1.5101953125	8.47081582456608e-09\\
1.53148437500001	8.64767183125331e-09\\
1.5527734375	8.92017041660261e-09\\
1.5740625	9.08733867592781e-09\\
1.5953515625	9.14732322445159e-09\\
1.616640625	9.28042380879954e-09\\
1.63792968750001	9.26058682391058e-09\\
1.65921875	9.16518857164958e-09\\
1.6805078125	9.01695949948663e-09\\
1.701796875	9.03245408463458e-09\\
1.7230859375	8.99395293099547e-09\\
1.74437500000001	8.99478966804249e-09\\
1.7656640625	9.05146869805717e-09\\
1.786953125	9.41235450264718e-09\\
1.8082421875	9.4213581131748e-09\\
1.82953125	9.6253565410628e-09\\
1.85082031250001	9.79145228248793e-09\\
1.872109375	9.89819949254585e-09\\
1.8933984375	9.79931001876697e-09\\
1.9146875	9.87903048173558e-09\\
1.9359765625	9.84301019891909e-09\\
1.95726562500001	9.95395224794586e-09\\
1.9785546875	9.94588695728193e-09\\
1.99984375	1.01469068298287e-08\\
2.0211328125	1.04267687609398e-08\\
2.042421875	1.0584665280597e-08\\
2.06371093750001	1.05871646665979e-08\\
2.085	1.07624940924368e-08\\
2.1062890625	1.06572234919156e-08\\
2.127578125	1.0604533983569e-08\\
2.1488671875	1.06850961624546e-08\\
2.17015625000001	1.06343533402357e-08\\
2.1914453125	1.04825565268033e-08\\
2.212734375	1.04062400396568e-08\\
2.2340234375	1.04157314835442e-08\\
2.2553125	1.03246354234297e-08\\
2.27660156250001	1.03984454207877e-08\\
2.297890625	1.04143399003729e-08\\
2.3191796875	1.0569346746494e-08\\
2.34046875	1.0482371185402e-08\\
2.3617578125	1.04383808972258e-08\\
2.38304687500001	1.04748680949448e-08\\
2.4043359375	1.02527732446447e-08\\
2.425625	1.01172522674915e-08\\
2.4469140625	9.95521172563912e-09\\
2.468203125	9.86700396624595e-09\\
2.48949218750001	9.58757134102981e-09\\
2.51078125	9.54026951309969e-09\\
2.5320703125	9.44495788757271e-09\\
2.553359375	9.45362443484199e-09\\
2.5746484375	9.33943813199935e-09\\
2.59593750000001	9.35025992458235e-09\\
2.6172265625	9.30338038581899e-09\\
2.638515625	9.02906111901811e-09\\
2.6598046875	8.76442854148943e-09\\
2.68109375	8.54493790367713e-09\\
2.70238281250001	8.30707901189641e-09\\
2.723671875	8.24224412347915e-09\\
2.7449609375	8.20242366168809e-09\\
2.76625	8.12132429794035e-09\\
2.7875390625	8.33985094216705e-09\\
2.80882812500001	8.39291845448158e-09\\
2.8301171875	8.35494668485552e-09\\
2.85140625	8.40438942885432e-09\\
2.8726953125	8.28028604620518e-09\\
2.893984375	7.96188057347489e-09\\
2.91527343750001	7.74932627493004e-09\\
2.9365625	7.77577327423483e-09\\
2.9578515625	7.32953412251082e-09\\
2.979140625	7.48016842852473e-09\\
3.0004296875	7.33419665272864e-09\\
3.02171875000001	7.49055699883374e-09\\
3.0430078125	7.5296784105918e-09\\
3.064296875	7.46861482994417e-09\\
3.0855859375	7.41558292803734e-09\\
3.106875	7.41744817491438e-09\\
3.12816406250001	7.03092029309568e-09\\
3.149453125	7.07824132423564e-09\\
3.1707421875	6.70490600999289e-09\\
3.19203125	6.65170501046307e-09\\
3.2133203125	6.56911451097396e-09\\
3.23460937500001	6.53678171450089e-09\\
3.2558984375	6.64379464822998e-09\\
3.2771875	6.5068803602224e-09\\
3.2984765625	6.53558626446122e-09\\
3.319765625	6.44339108324199e-09\\
3.34105468750001	6.20247200408768e-09\\
3.36234375	6.10825601849275e-09\\
3.3836328125	5.91334048614688e-09\\
3.404921875	5.85825347422229e-09\\
3.4262109375	5.77802899495929e-09\\
3.44750000000001	5.82273198706913e-09\\
3.4687890625	5.95139808820145e-09\\
3.490078125	5.85650001006726e-09\\
3.5113671875	5.75702136989376e-09\\
3.53265625	5.64945592294988e-09\\
3.55394531250001	5.30164979005266e-09\\
3.575234375	5.18086560426371e-09\\
3.5965234375	4.86058938454839e-09\\
3.6178125	4.60027205656275e-09\\
3.6391015625	4.68522006452926e-09\\
3.66039062500001	4.61654536493522e-09\\
3.6816796875	4.53266575224511e-09\\
3.70296875	4.6704453483534e-09\\
3.7242578125	4.53061619357131e-09\\
3.745546875	4.49900213620132e-09\\
3.76683593750001	4.31178583537266e-09\\
3.788125	4.24509154734172e-09\\
3.8094140625	4.05317419685669e-09\\
3.830703125	3.86811753137983e-09\\
3.8519921875	3.95488005233078e-09\\
3.87328125000001	3.8637486416036e-09\\
3.8945703125	3.7833215736707e-09\\
3.915859375	3.77241751332764e-09\\
3.9371484375	3.70705397859944e-09\\
3.9584375	3.87271227331049e-09\\
3.97972656250001	3.7023157587295e-09\\
4.001015625	3.63462691371873e-09\\
4.0223046875	3.59221604284941e-09\\
4.04359375	3.34465082393283e-09\\
4.0648828125	3.30326464206178e-09\\
4.08617187500001	3.24722852521653e-09\\
4.1074609375	3.18707900229719e-09\\
4.12875	3.01914280563568e-09\\
4.1500390625	3.11661462875558e-09\\
4.171328125	2.99546361499426e-09\\
4.19261718750001	2.93016204333544e-09\\
4.21390625	2.85484937216706e-09\\
4.2351953125	2.84984256944474e-09\\
4.256484375	2.80043536034751e-09\\
4.2777734375	2.63517126088383e-09\\
4.29906250000001	2.80161672804547e-09\\
4.3203515625	2.59516335081181e-09\\
4.341640625	2.34603690116168e-09\\
4.3629296875	2.36975894903329e-09\\
4.38421875	2.33289659389552e-09\\
4.40550781250001	2.3133776766102e-09\\
4.426796875	2.31816721606215e-09\\
4.4480859375	2.31749361140023e-09\\
4.469375	2.24467704596003e-09\\
4.4906640625	2.15978521452495e-09\\
4.51195312500001	2.15875566008327e-09\\
4.5332421875	1.98153472035361e-09\\
4.55453125	2.03291532509828e-09\\
4.5758203125	1.99461752854768e-09\\
4.597109375	1.78551016896533e-09\\
4.61839843750001	1.80652480242999e-09\\
4.6396875	1.89150564739088e-09\\
4.6609765625	1.7919114917216e-09\\
4.682265625	1.85966132236782e-09\\
4.7035546875	1.82790496667996e-09\\
4.72484375000001	1.73799733816931e-09\\
4.7461328125	1.63713054439346e-09\\
4.767421875	1.52779485745923e-09\\
4.7887109375	1.47980405613648e-09\\
4.81	1.55439724957492e-09\\
4.83128906250001	1.53705195002111e-09\\
4.852578125	1.61483487241959e-09\\
4.8738671875	1.63495614466983e-09\\
4.89515625	1.5877731582875e-09\\
4.9164453125	1.65901796107733e-09\\
4.93773437500001	1.52037264697695e-09\\
4.9590234375	1.40454599443869e-09\\
4.9803125	1.2501001006808e-09\\
5.0016015625	1.13931343726869e-09\\
5.022890625	9.6193471725368e-10\\
5.04417968750001	1.00585108871357e-09\\
5.06546875	1.004894131504e-09\\
5.0867578125	8.44863958703184e-10\\
5.108046875	9.28326048434986e-10\\
5.1293359375	8.67566081856928e-10\\
5.15062500000001	8.42037094220152e-10\\
5.1719140625	9.50640477944436e-10\\
5.193203125	8.17698811626887e-10\\
5.2144921875	7.74263860927498e-10\\
5.23578125	6.32252976488371e-10\\
5.25707031250001	5.3871930250551e-10\\
5.278359375	3.74721739694766e-10\\
5.2996484375	2.9391657194705e-10\\
5.3209375	3.35606356667371e-10\\
5.3422265625	3.83343900370114e-10\\
5.36351562500001	3.68221271585108e-10\\
5.3848046875	3.59298303736028e-10\\
5.40609375	3.23262010265857e-10\\
5.4273828125	3.04520240731312e-10\\
5.448671875	1.27415766131478e-10\\
5.46996093750001	1.91717033721582e-10\\
5.49125	2.05202436217875e-10\\
5.5125390625	1.10503548150688e-10\\
5.533828125	1.38692149376063e-10\\
5.5551171875	3.98168325941102e-11\\
5.57640625000001	2.38227119624034e-11\\
5.5976953125	1.52701171612431e-11\\
5.618984375	5.19430374804421e-13\\
5.6402734375	-7.78044475236458e-12\\
5.6615625	-6.13925006320952e-11\\
5.68285156250001	-1.95432512601751e-10\\
5.704140625	-1.59019379074968e-10\\
5.7254296875	-1.3367844151617e-10\\
5.74671875	-2.00702124104351e-10\\
5.7680078125	-2.3022099502824e-10\\
5.78929687500001	-2.09913400526756e-10\\
5.8105859375	-3.32776440333848e-10\\
5.831875	-3.19188434058311e-10\\
5.8531640625	-3.64471034393461e-10\\
5.874453125	-3.00796851149608e-10\\
5.89574218750001	-3.23800491892953e-10\\
5.91703125	-3.01550765988131e-10\\
5.9383203125	-3.31676386394291e-10\\
5.959609375	-4.83865075918597e-10\\
5.9808984375	-5.96262007339958e-10\\
6.00218750000001	-5.96852642757413e-10\\
6.0234765625	-5.72286023186325e-10\\
6.044765625	-5.50868393498955e-10\\
6.0660546875	-5.35528010063086e-10\\
6.08734375	-5.33378267549645e-10\\
6.10863281250001	-3.46711973757412e-10\\
6.129921875	-5.09688839969489e-10\\
6.1512109375	-5.55296992495807e-10\\
6.1725	-5.60986532141015e-10\\
6.1937890625	-5.47741059102542e-10\\
6.21507812500001	-6.78122278188669e-10\\
6.2363671875	-6.12211021424814e-10\\
6.25765625	-6.52090078220101e-10\\
6.2789453125	-6.10010232471506e-10\\
6.300234375	-6.64963945546327e-10\\
6.32152343750001	-6.60666448138612e-10\\
6.3428125	-6.44782697608175e-10\\
6.3641015625	-5.65901674722992e-10\\
6.385390625	-6.51673802956901e-10\\
6.4066796875	-6.10034008830768e-10\\
6.42796875000001	-7.32289154389893e-10\\
6.4492578125	-7.72404658616114e-10\\
6.470546875	-7.72604806929654e-10\\
6.4918359375	-7.64651079526152e-10\\
6.513125	-8.28254095530477e-10\\
6.53441406250001	-8.71016212640934e-10\\
6.555703125	-7.72491751531994e-10\\
6.5769921875	-8.89192925983801e-10\\
6.59828125	-8.17020951652923e-10\\
6.6195703125	-8.07964277397114e-10\\
6.64085937500001	-9.59075012370451e-10\\
6.6621484375	-9.48967236032755e-10\\
6.6834375	-1.00778868605194e-09\\
6.7047265625	-1.04519899978012e-09\\
6.726015625	-9.83808761755933e-10\\
6.74730468750001	-1.13170410140301e-09\\
6.76859375	-9.77302990975808e-10\\
6.7898828125	-9.25144138508468e-10\\
6.811171875	-1.17848963652442e-09\\
6.8324609375	-9.99833368720122e-10\\
6.85375000000001	-9.09441819615379e-10\\
6.8750390625	-9.79266366707477e-10\\
6.896328125	-1.00586324188137e-09\\
6.9176171875	-1.03198920549771e-09\\
6.93890625	-1.05199401708359e-09\\
6.96019531250001	-1.15956327183832e-09\\
6.981484375	-1.27708788603633e-09\\
7.0027734375	-1.20074580033133e-09\\
7.0240625	-1.26023707361009e-09\\
7.0453515625	-1.25216842459278e-09\\
7.06664062500001	-1.18725735923099e-09\\
7.0879296875	-1.08158875806369e-09\\
7.10921875	-1.00525246300693e-09\\
7.1305078125	-1.03541416306988e-09\\
7.151796875	-1.11197645690686e-09\\
7.17308593750001	-1.10338191693558e-09\\
7.194375	-1.25195850912675e-09\\
7.2156640625	-1.40260956321932e-09\\
7.236953125	-1.34450663872473e-09\\
7.2582421875	-1.42391925206937e-09\\
7.27953125000001	-1.48353353988998e-09\\
7.3008203125	-1.32441729094526e-09\\
7.322109375	-1.20716472406604e-09\\
7.3433984375	-1.20679851657713e-09\\
7.3646875	-1.2011439866436e-09\\
7.38597656250001	-1.28361290830275e-09\\
7.407265625	-1.23553275927447e-09\\
7.4285546875	-1.25122739491835e-09\\
7.44984375	-1.33917567613268e-09\\
7.4711328125	-1.35383668838628e-09\\
7.49242187500001	-1.32048747422829e-09\\
7.5137109375	-1.45619019250694e-09\\
7.535	-1.42229620632129e-09\\
7.5562890625	-1.3262819328894e-09\\
7.577578125	-1.28344211649741e-09\\
7.59886718750001	-1.29726308411989e-09\\
7.62015625	-1.30566040073466e-09\\
7.6414453125	-1.26067836341525e-09\\
7.662734375	-1.22399219660496e-09\\
7.6840234375	-1.23872333979874e-09\\
7.70531250000001	-1.15608027200378e-09\\
7.7266015625	-1.21447163707919e-09\\
7.747890625	-1.16009258474003e-09\\
7.7691796875	-1.24648568913233e-09\\
7.79046875	-1.17419222394484e-09\\
7.81175781250001	-1.16287236824621e-09\\
7.833046875	-1.23047184191076e-09\\
7.8543359375	-1.13302458286421e-09\\
7.875625	-1.14369209161942e-09\\
7.8969140625	-1.15493293897555e-09\\
7.91820312500001	-1.13442967614157e-09\\
7.9394921875	-1.00060899266691e-09\\
7.96078125	-9.55093000552412e-10\\
7.9820703125	-9.83671737571411e-10\\
8.003359375	-1.05004314417157e-09\\
8.02464843750001	-1.07111974567176e-09\\
8.0459375	-1.08114041612568e-09\\
8.0672265625	-1.15630926314727e-09\\
8.088515625	-1.01929625972762e-09\\
8.1098046875	-1.1150995092899e-09\\
8.13109375000001	-9.69857070813847e-10\\
8.1523828125	-1.02171216132267e-09\\
8.173671875	-8.38304531083713e-10\\
8.1949609375	-8.97374811709888e-10\\
8.21625	-8.82883931975991e-10\\
8.23753906250001	-7.85540250903798e-10\\
8.258828125	-8.46324996395935e-10\\
8.2801171875	-8.71568820069388e-10\\
8.30140625	-8.46725475868304e-10\\
8.3226953125	-7.85731789499805e-10\\
8.34398437500001	-7.49575838136859e-10\\
8.3652734375	-7.17837161086142e-10\\
8.3865625	-5.64562381947194e-10\\
8.4078515625	-4.17950115757459e-10\\
8.429140625	-4.93480808703649e-10\\
8.45042968750001	-3.12196055080307e-10\\
8.47171875	-2.98788498210051e-10\\
8.4930078125	-3.61198916226366e-10\\
8.514296875	-3.41516516364186e-10\\
8.5355859375	-4.31092477999344e-10\\
8.55687500000001	-4.71117528319969e-10\\
8.5781640625	-5.92345524167907e-10\\
8.599453125	-5.6725003536521e-10\\
8.6207421875	-4.91407390511621e-10\\
8.64203125	-4.503233610321e-10\\
8.66332031250001	-3.46066451649242e-10\\
8.684609375	-3.55532756289861e-10\\
8.7058984375	-2.33921829907511e-10\\
8.7271875	-1.87690351380962e-10\\
8.7484765625	-2.23538042350144e-10\\
8.76976562500001	-2.16679415298367e-10\\
8.7910546875	-2.91009626060774e-10\\
8.81234375	-1.99482945114489e-10\\
8.8336328125	-1.44832149444455e-10\\
8.854921875	-1.30005212982793e-10\\
8.87621093750001	-1.08155321407208e-10\\
8.8975	-1.2178385913095e-10\\
8.9187890625	-6.26677252611206e-11\\
8.940078125	-3.10086831814846e-11\\
8.9613671875	-8.38469712704667e-11\\
8.98265625000001	-1.19321597428893e-10\\
9.0039453125	-2.89455975102947e-10\\
9.025234375	-3.08736393260528e-10\\
9.0465234375	-3.83848152828894e-10\\
9.0678125	-3.80050191373888e-10\\
9.08910156250001	-2.70429990510156e-10\\
9.110390625	-3.33833854396713e-10\\
9.1316796875	-2.44886105925507e-10\\
9.15296875	-1.61197176878186e-10\\
9.1742578125	-6.47649576340738e-11\\
9.19554687500001	-3.70058612031352e-11\\
9.2168359375	4.14679926200903e-12\\
9.238125	-8.84850702405015e-11\\
9.2594140625	-1.90862439841837e-10\\
9.280703125	-1.97549566526562e-10\\
9.30199218750001	-2.71982612802796e-10\\
9.32328125	-1.81280290964877e-10\\
9.3445703125	-1.73279195268677e-10\\
9.365859375	-8.00814827768084e-11\\
9.3871484375	-1.41083875027049e-10\\
9.40843750000001	-8.24106487094064e-11\\
9.4297265625	3.05722622132675e-11\\
9.451015625	-7.71109665246187e-11\\
9.4723046875	-1.37451717250184e-11\\
9.49359375	1.29995259299e-10\\
9.51488281250001	9.27091555652741e-11\\
9.536171875	1.1640384874491e-10\\
9.5574609375	-4.27957241501612e-11\\
9.57875	3.7125928071636e-11\\
9.6000390625	-5.67480950532055e-11\\
9.62132812500001	-1.89091223506858e-10\\
9.6426171875	-4.19417820673361e-12\\
9.66390625	-1.22294218376806e-10\\
9.6851953125	-5.07047885764507e-11\\
9.706484375	-6.07055069492344e-11\\
9.72777343750001	4.53425284341584e-13\\
9.7490625	-5.26022334838168e-13\\
9.7703515625	-4.28968991962944e-11\\
9.791640625	2.57346065052814e-11\\
9.8129296875	6.37634015993814e-12\\
9.83421875000001	3.38890389655368e-12\\
9.8555078125	6.59316779514623e-11\\
9.876796875	-2.78948397782764e-11\\
9.8980859375	8.08534663933094e-11\\
9.919375	1.07909609695429e-10\\
9.94066406250001	6.58049051503751e-11\\
9.961953125	9.09305095974248e-11\\
9.9832421875	7.6868186549333e-11\\
10.00453125	1.35302861250249e-10\\
10.0258203125	1.70650318403028e-10\\
10.047109375	1.25713566449366e-10\\
10.0683984375	1.20374162879705e-10\\
10.0896875	1.72042183124634e-10\\
10.1109765625	1.23341042218904e-10\\
10.132265625	-2.59013877213621e-11\\
10.1535546875	-1.09064012860261e-12\\
10.17484375	-1.05824794436827e-10\\
10.1961328125	-1.35082783157943e-10\\
10.217421875	-1.19020417130257e-10\\
10.2387109375	-2.4763228812977e-12\\
10.26	-1.24891933987864e-10\\
10.2812890625	-4.78244378447873e-11\\
10.302578125	-6.24712478023861e-12\\
10.3238671875	-3.8572719382075e-11\\
10.34515625	-3.43916684755471e-11\\
10.3664453125	-9.95691785522889e-11\\
10.387734375	-3.98597153747035e-11\\
10.4090234375	2.20146211439176e-11\\
10.4303125	-8.64601446348053e-11\\
10.4516015625	-1.04899259901681e-11\\
10.472890625	9.80495420531599e-11\\
10.4941796875	1.74274280485583e-10\\
10.51546875	1.96120645871174e-10\\
10.5367578125	2.28354020576045e-10\\
10.558046875	2.57820943591359e-10\\
10.5793359375	1.79942584036134e-10\\
10.600625	1.60500732196897e-10\\
10.6219140625	1.86017164958094e-10\\
10.643203125	4.25450840315391e-11\\
10.6644921875	3.93942789345694e-11\\
10.68578125	9.01053595041899e-11\\
10.7070703125	1.72950016829067e-10\\
10.728359375	3.00665722580611e-10\\
10.7496484375	2.422980495875e-10\\
10.7709375	2.34220276592748e-10\\
10.7922265625	9.50002900348288e-11\\
10.813515625	1.3185795957747e-10\\
10.8348046875	1.97387168542033e-11\\
10.85609375	4.65220858939073e-11\\
10.8773828125	-5.90627099121476e-11\\
10.898671875	-1.17491514604944e-10\\
10.9199609375	-3.4308830989113e-12\\
10.94125	4.75154202227198e-11\\
10.9625390625	-1.18149327370882e-10\\
10.983828125	-5.51620517027774e-11\\
11.0051171875	-2.96752385531754e-11\\
11.02640625	-1.03521183584411e-10\\
11.0476953125	1.09384078541303e-10\\
11.068984375	1.22111644231348e-11\\
11.0902734375	8.48164338417279e-11\\
11.1115625	1.13951193762679e-10\\
11.1328515625	6.71649985183037e-11\\
11.154140625	4.36914482702579e-11\\
11.1754296875	1.09047990517093e-11\\
11.19671875	1.17233253688741e-10\\
11.2180078125	-8.45848444925557e-11\\
11.239296875	4.24352556864367e-13\\
11.2605859375	-5.3515226153302e-11\\
11.281875	-2.61538726949776e-11\\
11.3031640625	9.5624082923557e-11\\
11.324453125	-6.92039177303902e-11\\
11.3457421875	-5.5137451355313e-11\\
11.36703125	-1.88344065110531e-11\\
11.3883203125	5.22896993269277e-11\\
11.409609375	-2.23395977508261e-11\\
11.4308984375	-4.34391552061175e-11\\
11.4521875	1.71996351539535e-10\\
11.4734765625	1.3247633926408e-10\\
11.494765625	1.53661291446237e-10\\
11.5160546875	2.80663366044888e-10\\
11.53734375	1.83687138393999e-10\\
11.5586328125	1.66883132393231e-10\\
11.579921875	5.07182060439446e-11\\
11.6012109375	8.75097842644948e-11\\
11.6225	1.24057314470624e-12\\
11.6437890625	-8.76086132349139e-11\\
11.665078125	-1.32674934633557e-10\\
11.6863671875	-1.02774334653549e-10\\
11.70765625	-9.83892636683269e-11\\
11.7289453125	-1.30844778156222e-10\\
11.750234375	-1.24330759085616e-11\\
11.7715234375	-9.04374850850066e-11\\
11.7928125	-1.4754799381566e-10\\
11.8141015625	-5.63610406623563e-11\\
11.835390625	-1.5901845579669e-10\\
11.8566796875	-1.4947748064446e-10\\
11.87796875	-3.63440854348432e-10\\
11.8992578125	-2.37195479016527e-10\\
11.920546875	-3.92530206840603e-10\\
11.9418359375	-3.60285310465766e-10\\
11.963125	-2.8727983962048e-10\\
11.9844140625	-5.13434343416386e-10\\
12.005703125	-5.517187674059e-10\\
12.0269921875	-5.05942343621475e-10\\
12.04828125	-4.32516140048428e-10\\
12.0695703125	-3.13730207126299e-10\\
12.090859375	-3.59127648159307e-10\\
12.1121484375	-2.20169485605384e-10\\
12.1334375	-1.88027777673931e-10\\
12.1547265625	-3.2914755480592e-10\\
12.176015625	-2.0281101806971e-10\\
12.1973046875	-4.66356645960648e-10\\
12.21859375	-3.94138467927281e-10\\
12.2398828125	-4.92913146468672e-10\\
12.261171875	-5.242768802315e-10\\
12.2824609375	-4.73750258351175e-10\\
12.30375	-4.57752468190797e-10\\
12.3250390625	-3.91690558459091e-10\\
12.346328125	-3.31314835569714e-10\\
12.3676171875	-3.01175169748171e-10\\
12.38890625	-3.15819112762353e-10\\
12.4101953125	-3.16799819660377e-10\\
12.431484375	-3.45969033016317e-10\\
12.4527734375	-4.72012844217697e-10\\
12.4740625	-4.96918990790883e-10\\
12.4953515625	-4.48595983417308e-10\\
12.516640625	-3.78780744756647e-10\\
12.5379296875	-2.16260241760121e-10\\
12.55921875	-2.5686060499958e-10\\
12.5805078125	-2.04041872353979e-10\\
12.601796875	-2.59046352573633e-10\\
12.6230859375	-2.97452433156197e-10\\
12.644375	-3.54064522134218e-10\\
12.6656640625	-4.21790529894945e-10\\
12.686953125	-5.29567238117714e-10\\
12.7082421875	-5.95742803192647e-10\\
12.72953125	-5.92137891465855e-10\\
12.7508203125	-6.16933042042466e-10\\
12.772109375	-5.09404628583847e-10\\
12.7933984375	-5.83022152929477e-10\\
12.8146875	-4.38173023523214e-10\\
12.8359765625	-5.11457569679817e-10\\
12.857265625	-5.70267590841068e-10\\
12.8785546875	-7.25480121984621e-10\\
12.89984375	-7.9812286999294e-10\\
12.9211328125	-8.06802194709488e-10\\
12.942421875	-7.70066066147046e-10\\
12.9637109375	-7.97463507516998e-10\\
12.985	-7.20897583624782e-10\\
13.0062890625	-6.55974131475669e-10\\
13.027578125	-4.88038129919684e-10\\
13.0488671875	-4.27489095703259e-10\\
13.07015625	-4.04926605688215e-10\\
13.0914453125	-4.97636197223205e-10\\
13.112734375	-5.73759559722577e-10\\
13.1340234375	-4.55276194015306e-10\\
13.1553125	-5.48247148341516e-10\\
13.1766015625	-5.38189845505534e-10\\
13.197890625	-5.02584779681438e-10\\
13.2191796875	-5.25059126989043e-10\\
13.24046875	-3.96125046014795e-10\\
13.2617578125	-3.25561636067655e-10\\
13.283046875	-3.05949421694876e-10\\
13.3043359375	-2.44680463371991e-10\\
13.325625	-2.72945326416243e-10\\
13.3469140625	-1.80058627772013e-10\\
13.368203125	-2.99034633522106e-10\\
13.3894921875	-3.45381594739605e-10\\
13.41078125	-3.91683553195778e-10\\
13.4320703125	-3.71460008849123e-10\\
13.453359375	-4.28955631402314e-10\\
13.4746484375	-3.57021708540506e-10\\
13.4959375	-3.61474176349095e-10\\
13.5172265625	-3.14303572126492e-10\\
13.538515625	-2.95088808113196e-10\\
13.5598046875	-2.70174926942594e-10\\
13.58109375	-2.32904107099376e-10\\
13.6023828125	-3.69018323352589e-10\\
13.623671875	-4.55826842716484e-10\\
13.6449609375	-4.69752037865598e-10\\
13.66625	-5.92838644094353e-10\\
13.6875390625	-6.23746180565224e-10\\
13.708828125	-5.82785460467056e-10\\
13.7301171875	-4.84200438713045e-10\\
13.75140625	-5.1831692630929e-10\\
13.7726953125	-4.57977569159325e-10\\
13.793984375	-3.48752737374604e-10\\
13.8152734375	-4.56061122780693e-10\\
13.8365625	-3.79474535628032e-10\\
13.8578515625	-5.16647627086349e-10\\
13.879140625	-5.50560546174113e-10\\
13.9004296875	-5.35494736347508e-10\\
13.92171875	-5.92778749632167e-10\\
13.9430078125	-5.44689433384349e-10\\
13.964296875	-5.07992220722238e-10\\
13.9855859375	-4.41914952082631e-10\\
14.006875	-2.97402477637805e-10\\
14.0281640625	-3.81458305637532e-10\\
14.049453125	-3.03980874300844e-10\\
14.0707421875	-3.47090332352673e-10\\
14.09203125	-4.11523432213119e-10\\
14.1133203125	-4.36064680380331e-10\\
14.134609375	-4.43507879633271e-10\\
14.1558984375	-3.91884590446986e-10\\
14.1771875	-3.57244066775131e-10\\
14.1984765625	-2.63121528980186e-10\\
14.219765625	-1.67868086834682e-10\\
14.2410546875	-1.35540401773574e-10\\
14.26234375	-1.15373204848663e-10\\
14.2836328125	-6.81340392630359e-11\\
14.304921875	-1.00346431259625e-10\\
14.3262109375	-1.75491690166383e-10\\
14.3475	-2.82382506819472e-10\\
14.3687890625	-1.94309363368867e-10\\
14.390078125	-9.12285551003216e-11\\
14.4113671875	-1.75563534306954e-10\\
14.43265625	8.33764262744851e-11\\
14.4539453125	5.5926798318669e-11\\
14.475234375	1.01874341731353e-10\\
14.4965234375	2.22688421062209e-10\\
14.5178125	2.63351187314829e-10\\
14.5391015625	2.29023788693668e-10\\
14.560390625	1.08461253316711e-10\\
14.5816796875	7.08560691138354e-12\\
14.60296875	1.74529510742675e-10\\
14.6242578125	-1.20008603161493e-10\\
14.645546875	-4.59704816173716e-11\\
14.6668359375	-1.46051589546591e-11\\
14.688125	9.75635520566423e-11\\
14.7094140625	2.27364558461106e-10\\
14.730703125	3.36662810490336e-10\\
14.7519921875	2.84667700419997e-10\\
14.77328125	3.43967270950673e-10\\
14.7945703125	1.1155085549161e-10\\
14.815859375	-5.66260227399294e-12\\
14.8371484375	1.9006637861154e-11\\
14.8584375	-2.16456529549898e-10\\
14.8797265625	-2.56317555343451e-10\\
14.901015625	-2.23798918354144e-10\\
14.9223046875	1.12219388841665e-11\\
14.94359375	1.42140322122987e-10\\
14.9648828125	2.54351261060362e-10\\
14.986171875	2.79333562045571e-10\\
15.0074609375	1.7431764150459e-10\\
15.02875	5.76359760247466e-11\\
15.0500390625	4.80069544831724e-11\\
15.071328125	7.67851129790386e-11\\
15.0926171875	-4.05018039559816e-11\\
15.11390625	7.08385458045498e-11\\
15.1351953125	1.51142846349309e-10\\
15.156484375	2.12679510158101e-10\\
15.1777734375	2.25366939363363e-10\\
15.1990625	3.26591572768761e-10\\
15.2203515625	3.83940768320121e-10\\
15.241640625	2.76306599024204e-10\\
15.2629296875	2.55926671466615e-10\\
15.28421875	2.10071622442098e-10\\
15.3055078125	2.47424303911379e-10\\
15.326796875	1.68969517426654e-10\\
15.3480859375	2.5992329043687e-10\\
15.369375	1.71950728415097e-10\\
15.3906640625	1.72064030880606e-10\\
15.411953125	2.26612142396675e-10\\
15.4332421875	1.92031800590555e-10\\
15.45453125	2.31615241711561e-11\\
15.4758203125	2.16986110001423e-10\\
15.497109375	1.43057248215369e-10\\
15.5183984375	1.66336336511286e-10\\
15.5396875	3.57140319390572e-10\\
15.5609765625	4.16217070988074e-10\\
15.582265625	4.17586055362133e-10\\
15.6035546875	5.54336704188598e-10\\
15.62484375	4.72079938034469e-10\\
15.6461328125	4.00300324722439e-10\\
15.667421875	2.31319411559527e-10\\
15.6887109375	1.52315358817779e-10\\
15.71	9.05596089306523e-11\\
15.7312890625	1.87626947818077e-10\\
15.752578125	1.84209744500925e-10\\
15.7738671875	3.56478789166474e-10\\
15.79515625	3.6686211472839e-10\\
15.8164453125	4.74907518055168e-10\\
15.837734375	4.07303037990371e-10\\
15.8590234375	3.34093716803628e-10\\
15.8803125	3.42270538344757e-10\\
15.9016015625	2.22064962738579e-10\\
15.922890625	2.57054682182592e-10\\
15.9441796875	1.48418795986957e-10\\
15.96546875	1.73553816432336e-10\\
15.9867578125	1.39728471389723e-10\\
16.008046875	1.65416906182442e-10\\
16.0293359375	1.80111819881189e-10\\
16.050625	3.10978096107362e-10\\
16.0719140625	4.65969806991303e-10\\
16.093203125	4.01815619976839e-10\\
16.1144921875	5.46455753766467e-10\\
16.13578125	5.12051342133077e-10\\
16.1570703125	4.69584549173205e-10\\
16.178359375	3.87388156842054e-10\\
16.1996484375	2.18719070703806e-10\\
16.2209375	1.69309320149603e-10\\
16.2422265625	4.43304823809942e-11\\
16.263515625	3.03968303259473e-10\\
16.2848046875	1.91946305999186e-10\\
16.30609375	3.74604250345802e-10\\
16.3273828125	4.64753535274799e-10\\
16.348671875	4.77940677854782e-10\\
16.3699609375	5.85357593014778e-10\\
16.39125	4.71306895010763e-10\\
16.4125390625	2.79104694657134e-10\\
16.433828125	1.55525767355113e-10\\
16.4551171875	9.08625218024622e-11\\
16.47640625	6.17329176035726e-11\\
16.4976953125	3.0708570497865e-11\\
16.518984375	8.5534363013908e-11\\
16.5402734375	1.54733630486621e-10\\
16.5615625	2.92787389604878e-10\\
16.5828515625	2.79014155209914e-10\\
16.604140625	3.66296199500282e-10\\
16.6254296875	2.93152347668601e-10\\
16.64671875	1.00968548206186e-10\\
16.6680078125	-3.10592630948613e-11\\
16.689296875	-1.09731750330095e-10\\
16.7105859375	-1.36407143772497e-10\\
16.731875	-1.51197456335238e-10\\
16.7531640625	-1.0169708135943e-10\\
16.774453125	-1.67284705986638e-10\\
16.7957421875	4.2015022343257e-11\\
16.81703125	9.09466765045111e-11\\
16.8383203125	7.63654675330357e-11\\
16.859609375	1.53093080960898e-10\\
16.8808984375	8.55107219691084e-11\\
16.9021875	-1.51345534757112e-10\\
16.9234765625	-1.18901230711469e-10\\
16.944765625	-3.49553968420155e-10\\
16.9660546875	-3.67448700025819e-10\\
16.98734375	-4.7311788675146e-10\\
17.0086328125	-4.13889518316173e-10\\
17.029921875	-4.47372222669197e-10\\
17.0512109375	-4.76659643971706e-10\\
17.0725	-6.11025868115071e-10\\
17.0937890625	-5.71234937219846e-10\\
17.115078125	-5.97901302640179e-10\\
17.1363671875	-4.23627463012876e-10\\
17.15765625	-4.37305683135465e-10\\
17.1789453125	-6.1326233651706e-10\\
17.200234375	-6.16267650721113e-10\\
17.2215234375	-7.74760366502774e-10\\
17.2428125	-6.61655909259882e-10\\
17.2641015625	-7.46055720449024e-10\\
17.285390625	-7.49870732800646e-10\\
17.3066796875	-6.73937466873711e-10\\
17.32796875	-6.3628969900923e-10\\
17.3492578125	-7.96037132900163e-10\\
17.370546875	-6.19173882275818e-10\\
17.3918359375	-5.65836005582586e-10\\
17.413125	-7.0120887381222e-10\\
17.4344140625	-6.95600043098582e-10\\
17.455703125	-5.66645094583696e-10\\
17.4769921875	-7.43673836977604e-10\\
17.49828125	-6.79516289442867e-10\\
17.5195703125	-6.2410805128208e-10\\
17.540859375	-6.6918869042269e-10\\
17.5621484375	-7.44594675238552e-10\\
17.5834375	-8.15121001652357e-10\\
17.6047265625	-8.43429058158762e-10\\
17.626015625	-9.18801817803697e-10\\
17.6473046875	-9.56188965078235e-10\\
17.66859375	-9.77479005525853e-10\\
17.6898828125	-9.05997196731634e-10\\
17.711171875	-9.54300787446586e-10\\
17.7324609375	-1.00882564939105e-09\\
17.75375	-9.62592092533369e-10\\
17.7750390625	-9.26491137920442e-10\\
17.796328125	-1.04997565929055e-09\\
17.8176171875	-8.98467289218884e-10\\
17.83890625	-9.26840834435231e-10\\
17.8601953125	-8.56209741929705e-10\\
17.881484375	-7.48583569078873e-10\\
17.9027734375	-6.14826253585193e-10\\
17.9240625	-8.63497507350064e-10\\
17.9453515625	-5.92984012436908e-10\\
17.966640625	-7.05954343545392e-10\\
17.9879296875	-6.66821916752644e-10\\
18.00921875	-7.08589436462824e-10\\
18.0305078125	-7.80388943151508e-10\\
18.051796875	-8.2597635320783e-10\\
18.0730859375	-7.49177921470879e-10\\
18.094375	-8.58241512246201e-10\\
18.1156640625	-8.44512486582696e-10\\
18.136953125	-7.41645472387407e-10\\
18.1582421875	-7.3850220101614e-10\\
18.17953125	-6.11125181429894e-10\\
18.2008203125	-5.31245036282244e-10\\
18.222109375	-5.57084530347917e-10\\
18.2433984375	-5.13265464041755e-10\\
18.2646875	-5.79815359575449e-10\\
18.2859765625	-5.84618811565991e-10\\
18.307265625	-6.28395780175767e-10\\
18.3285546875	-6.01356187131167e-10\\
18.34984375	-6.15591635397404e-10\\
18.3711328125	-4.99137411111695e-10\\
18.392421875	-6.74370019189544e-10\\
18.4137109375	-5.53019421930544e-10\\
18.435	-5.74106694376963e-10\\
18.4562890625	-5.44623855696522e-10\\
18.477578125	-5.94543885522526e-10\\
18.4988671875	-4.70754661097458e-10\\
18.52015625	-4.82278970293155e-10\\
18.5414453125	-4.35177281307706e-10\\
18.562734375	-5.82269588011842e-10\\
18.5840234375	-4.29296574299757e-10\\
18.6053125	-4.93326830307882e-10\\
18.6266015625	-6.16276038216849e-10\\
18.647890625	-5.8314433068121e-10\\
18.6691796875	-6.24268871527337e-10\\
18.69046875	-7.33170094058089e-10\\
18.7117578125	-5.71493660306949e-10\\
18.733046875	-5.07922619035383e-10\\
18.7543359375	-4.50187934385777e-10\\
18.775625	-5.44581806852876e-10\\
18.7969140625	-3.26705296818418e-10\\
18.818203125	-4.1306067681179e-10\\
18.8394921875	-3.33513732104865e-10\\
18.86078125	-2.74502080998529e-10\\
18.8820703125	-2.50683148796377e-10\\
18.903359375	-1.92901381781594e-10\\
18.9246484375	-2.70340759570973e-10\\
18.9459375	-2.64052331234765e-10\\
18.9672265625	-2.58427720027136e-10\\
18.988515625	-2.43213911107493e-10\\
19.0098046875	-1.88806613390387e-10\\
19.03109375	-8.68882558417828e-11\\
19.0523828125	-1.99235589085621e-10\\
19.073671875	-9.96104704478081e-11\\
19.0949609375	-1.05530587686806e-11\\
19.11625	8.13275955876842e-11\\
19.1375390625	1.02357709560552e-10\\
19.158828125	2.96659719612454e-10\\
19.1801171875	1.06629281812705e-10\\
19.20140625	1.5445244933825e-10\\
19.2226953125	2.06642711353456e-10\\
19.243984375	4.84012352401262e-11\\
19.2652734375	1.19904275928943e-11\\
19.2865625	5.17046141938178e-11\\
19.3078515625	2.95211910636684e-11\\
19.329140625	7.10472957839944e-11\\
19.3504296875	1.38290285659855e-10\\
19.37171875	2.26854549595537e-10\\
19.3930078125	3.13741988094909e-10\\
19.414296875	3.11848977833485e-10\\
19.4355859375	3.14370448664859e-10\\
19.456875	4.37189228242101e-10\\
19.4781640625	3.85963578937255e-10\\
19.499453125	1.85353975648695e-10\\
19.5207421875	1.8937376778258e-10\\
19.54203125	2.98134717877828e-10\\
19.5633203125	2.33276997010504e-10\\
19.584609375	4.55562755105542e-10\\
19.6058984375	3.32450012039884e-10\\
19.6271875	4.55542442569783e-10\\
19.6484765625	3.04818373651868e-10\\
19.669765625	3.33639638848142e-10\\
19.6910546875	2.12548835333313e-10\\
19.71234375	3.02743073477739e-10\\
19.7336328125	6.56081965574911e-11\\
19.754921875	1.22905387005245e-10\\
19.7762109375	2.51431595951492e-10\\
19.7975	4.12191024696878e-10\\
19.8187890625	4.16195689631112e-10\\
19.840078125	4.8790067511346e-10\\
19.8613671875	6.56389137524163e-10\\
19.88265625	3.25566351808117e-10\\
19.9039453125	4.59237594783728e-10\\
19.925234375	3.60396986687758e-10\\
19.9465234375	3.44018356171757e-10\\
19.9678125	1.41636740501451e-10\\
19.9891015625	2.02693700643974e-10\\
20.010390625	5.71518841451582e-11\\
20.0316796875	2.73267712174378e-10\\
20.05296875	1.20479223952198e-10\\
20.0742578125	4.31524111269769e-10\\
20.095546875	3.26722021252318e-10\\
20.1168359375	3.9082837982682e-10\\
20.138125	2.82766268205197e-10\\
20.1594140625	4.53577706302583e-10\\
20.180703125	3.76632271031133e-10\\
20.2019921875	2.45500173319079e-10\\
20.22328125	4.07254291741508e-10\\
20.2445703125	3.78162984651364e-10\\
20.265859375	1.84638769265256e-10\\
20.2871484375	2.50171373341671e-10\\
20.3084375	3.0406662211279e-10\\
20.3297265625	2.02429200416844e-10\\
20.351015625	1.1400257551186e-10\\
20.3723046875	1.12960156454842e-10\\
20.39359375	1.31406624618827e-10\\
20.4148828125	2.16186256818221e-10\\
20.436171875	2.82805016645491e-10\\
20.4574609375	2.2864277794698e-10\\
20.47875	2.47132704897867e-10\\
20.5000390625	1.03128091869654e-10\\
20.521328125	-3.60283388583814e-11\\
20.5426171875	1.26190416866051e-11\\
20.56390625	-1.61434199981932e-10\\
20.5851953125	-1.35191857812447e-10\\
20.606484375	-4.38598615516162e-11\\
20.6277734375	-1.96044210819211e-11\\
20.6490625	-1.01064258194401e-11\\
20.6703515625	4.73486057092217e-11\\
20.691640625	-3.89528850367248e-11\\
20.7129296875	-7.20276784803276e-11\\
20.73421875	-1.4913421012335e-10\\
20.7555078125	-2.15359763878259e-10\\
20.776796875	-2.61076033679055e-10\\
20.7980859375	-3.66823922434238e-10\\
20.819375	-2.96664281254491e-10\\
20.8406640625	-3.14886485465672e-10\\
20.861953125	-3.70179007499176e-10\\
20.8832421875	-2.6830794747048e-10\\
20.90453125	-1.2930120136486e-10\\
20.9258203125	-2.56219028125845e-10\\
20.947109375	-7.5536301795438e-11\\
20.9683984375	-6.19389736127973e-11\\
20.9896875	-6.05708577649746e-12\\
21.0109765625	-1.92195057361671e-11\\
21.032265625	-2.91825893076134e-11\\
21.0535546875	-1.08384928428033e-10\\
21.07484375	9.15616922933472e-12\\
21.0961328125	-1.85964430984904e-10\\
21.117421875	-1.42433063039832e-10\\
21.1387109375	-1.60924984538512e-10\\
21.16	-2.35088711819873e-10\\
21.1812890625	-2.1809259017954e-10\\
21.202578125	-2.63841966513739e-10\\
21.2238671875	-2.24738692824358e-10\\
21.24515625	-1.14904321742011e-10\\
21.2664453125	-1.72238583736163e-10\\
21.287734375	-1.856641600804e-10\\
21.3090234375	-1.85384021594916e-10\\
21.3303125	-1.79327590485818e-10\\
21.3516015625	-2.32180077400624e-10\\
21.372890625	-4.15991541183478e-10\\
21.3941796875	-4.00033051655045e-10\\
21.41546875	-4.12233802975104e-10\\
21.4367578125	-4.06844533082686e-10\\
21.458046875	-4.77975875259138e-10\\
21.4793359375	-2.9799825154469e-10\\
21.500625	-2.27262998829166e-10\\
21.5219140625	-1.71230874181041e-10\\
21.543203125	-1.78694112524375e-10\\
21.5644921875	-1.91366513737701e-10\\
21.58578125	-3.47187000005788e-10\\
21.6070703125	-2.80356794908488e-10\\
21.628359375	-3.64106676720898e-10\\
21.6496484375	-5.21241779366466e-10\\
21.6709375	-5.05883492462318e-10\\
21.6922265625	-4.62588534758572e-10\\
21.713515625	-5.93486619312306e-10\\
21.7348046875	-5.70978563102335e-10\\
21.75609375	-4.73629368832601e-10\\
21.7773828125	-4.61285972762036e-10\\
21.798671875	-4.77566416451153e-10\\
21.8199609375	-5.6533550239469e-10\\
21.84125	-5.72860592207966e-10\\
21.8625390625	-5.36066738230043e-10\\
21.883828125	-5.7986793000933e-10\\
21.9051171875	-4.92903785911923e-10\\
21.92640625	-4.64447079364213e-10\\
21.9476953125	-4.1883514899022e-10\\
21.968984375	-5.63928937020578e-10\\
21.9902734375	-4.77429042790644e-10\\
22.0115625	-5.08947698695505e-10\\
22.0328515625	-4.50050257823498e-10\\
22.054140625	-5.9291141659669e-10\\
22.0754296875	-5.98063154167562e-10\\
22.09671875	-5.76056594205239e-10\\
22.1180078125	-6.33194725331882e-10\\
22.139296875	-7.85711996884814e-10\\
22.1605859375	-5.43502265530275e-10\\
22.181875	-6.1773857480778e-10\\
22.2031640625	-4.90809793080565e-10\\
22.224453125	-5.32827667060671e-10\\
22.2457421875	-3.20143054197355e-10\\
22.26703125	-3.30140022871752e-10\\
22.2883203125	-3.39647745415942e-10\\
22.309609375	-4.3454560163307e-10\\
22.3308984375	-4.19850225194381e-10\\
22.3521875	-7.06088778870711e-10\\
22.3734765625	-6.77693301274759e-10\\
22.394765625	-6.53885873283609e-10\\
22.4160546875	-7.51674244448027e-10\\
22.43734375	-6.73742480767235e-10\\
22.4586328125	-5.41192486621144e-10\\
22.479921875	-4.08165255865241e-10\\
22.5012109375	-3.85951558493764e-10\\
22.5225	-4.27813750620437e-10\\
22.5437890625	-3.8785863572258e-10\\
22.565078125	-3.94021104617788e-10\\
22.5863671875	-4.60637150589907e-10\\
22.60765625	-3.43323398820031e-10\\
22.6289453125	-4.26733411103825e-10\\
22.650234375	-3.18243998352665e-10\\
22.6715234375	-4.31500407861262e-10\\
22.6928125	-3.58119626841714e-10\\
22.7141015625	-3.21822312403562e-10\\
22.735390625	-4.54655243799452e-10\\
22.7566796875	-4.55270549932889e-10\\
22.77796875	-4.38728686103357e-10\\
22.7992578125	-3.62762604982271e-10\\
22.820546875	-3.39765548793172e-10\\
22.8418359375	-2.07713336250939e-10\\
22.863125	-1.50513742635492e-10\\
22.8844140625	-1.39939577294515e-10\\
22.905703125	-1.98287283854861e-10\\
22.9269921875	-2.67583692181e-10\\
22.94828125	-2.46800956134809e-10\\
22.9695703125	-3.0927446684322e-10\\
22.990859375	-3.9284324122616e-10\\
23.0121484375	-2.26765863299406e-10\\
23.0334375	-2.48537544340357e-10\\
23.0547265625	-2.02687642332541e-10\\
23.076015625	-6.86870002672065e-11\\
23.0973046875	-4.06174369605629e-11\\
23.11859375	-1.0014720398173e-10\\
23.1398828125	-5.98109649857359e-11\\
23.161171875	5.49959681688473e-12\\
23.1824609375	-5.91957806518117e-11\\
23.20375	-2.94625795971378e-12\\
23.2250390625	1.23213300435204e-10\\
23.246328125	1.25597024245034e-10\\
23.2676171875	1.67961563422571e-10\\
23.28890625	2.57199554607513e-10\\
23.3101953125	2.07765893692279e-10\\
23.331484375	1.09755863355264e-10\\
23.3527734375	9.87637667061732e-11\\
23.3740625	1.71021109379626e-10\\
23.3953515625	1.80191565889855e-10\\
23.416640625	2.20326729642814e-10\\
23.4379296875	3.31143434119968e-10\\
23.45921875	2.97944038860975e-10\\
23.4805078125	4.3219598716655e-10\\
23.501796875	3.6209756950386e-10\\
23.5230859375	3.87786241578554e-10\\
23.544375	3.34809776100568e-10\\
23.5656640625	3.3376643624672e-10\\
23.586953125	1.59332158145106e-10\\
23.6082421875	2.91329105284735e-10\\
23.62953125	2.32813726402835e-10\\
23.6508203125	2.60776785001253e-10\\
23.672109375	2.07409303892155e-10\\
23.6933984375	2.76210164729203e-10\\
23.7146875	3.60580767208015e-10\\
23.7359765625	3.33508268558205e-10\\
23.757265625	2.39736663215065e-10\\
23.7785546875	4.40260979599314e-10\\
23.79984375	2.9325959697965e-10\\
23.8211328125	3.20153835372755e-10\\
23.842421875	2.73542778275877e-10\\
23.8637109375	3.06247746904092e-10\\
23.885	3.03022988605213e-10\\
23.9062890625	2.74154229671186e-10\\
23.927578125	4.1136823960147e-10\\
23.9488671875	3.38337982812823e-10\\
23.97015625	3.08812835517115e-10\\
23.9914453125	4.28088804724572e-10\\
24.012734375	4.4426041155941e-10\\
24.0340234375	4.20185700594362e-10\\
24.0553125	4.18728443899401e-10\\
24.0766015625	3.45186157885172e-10\\
24.097890625	5.32419653915176e-10\\
24.1191796875	4.22565775634325e-10\\
24.14046875	4.85180437861396e-10\\
24.1617578125	5.41583834896278e-10\\
24.183046875	4.78010755720034e-10\\
24.2043359375	3.8846282098202e-10\\
24.225625	5.56650592626186e-10\\
24.2469140625	4.94791906879243e-10\\
24.268203125	4.8344813765697e-10\\
24.2894921875	5.34996063332616e-10\\
24.31078125	5.30227176630202e-10\\
24.3320703125	4.66552784811088e-10\\
24.353359375	5.5022171400521e-10\\
24.3746484375	4.78055078605746e-10\\
24.3959375	5.87134427253259e-10\\
24.4172265625	5.75823328370856e-10\\
24.438515625	6.19608762106085e-10\\
24.4598046875	5.32384480415252e-10\\
24.48109375	5.59903460894505e-10\\
24.5023828125	5.9974921490354e-10\\
24.523671875	6.04655112855267e-10\\
24.5449609375	5.77289205658126e-10\\
24.56625	5.73551808627898e-10\\
24.5875390625	6.70187679852679e-10\\
24.608828125	5.53372489176565e-10\\
24.6301171875	6.71684003171367e-10\\
24.65140625	6.14349688374733e-10\\
24.6726953125	5.82175481804222e-10\\
24.693984375	5.06864764291951e-10\\
24.7152734375	5.10227797834319e-10\\
24.7365625	5.20364346775298e-10\\
24.7578515625	4.07507859168619e-10\\
24.779140625	4.37831950460178e-10\\
24.8004296875	5.33834979621794e-10\\
24.82171875	3.88503371607542e-10\\
24.8430078125	4.01684290596833e-10\\
24.864296875	5.93727586013004e-10\\
24.8855859375	3.66059376040908e-10\\
24.906875	4.51327128380624e-10\\
24.9281640625	3.34078038701835e-10\\
24.949453125	2.46912730451081e-10\\
24.9707421875	2.44632380222796e-10\\
24.99203125	1.19564610384077e-10\\
25.0133203125	1.68797810030939e-10\\
25.034609375	2.35337568706468e-10\\
25.0558984375	9.54870065391638e-11\\
25.0771875	2.67953352710895e-10\\
25.0984765625	2.28700448281563e-10\\
25.119765625	2.45977841048379e-10\\
25.1410546875	2.17215005510536e-10\\
25.16234375	1.61390241556127e-10\\
25.1836328125	1.68203744255551e-10\\
25.204921875	6.30119779580567e-11\\
25.2262109375	4.46008672831202e-11\\
25.2475	1.72753725995211e-10\\
25.2687890625	1.30282095852213e-10\\
25.290078125	1.36756386341729e-10\\
25.3113671875	1.49259854299124e-10\\
25.33265625	2.51205081457202e-10\\
25.3539453125	8.22359722369564e-11\\
25.375234375	1.06690951137466e-10\\
25.3965234375	-1.49106271822256e-11\\
25.4178125	-7.20134469998724e-12\\
25.4391015625	-7.90845494399774e-11\\
25.460390625	-1.17720382736818e-11\\
25.4816796875	1.76214130459561e-10\\
25.50296875	1.99147663138801e-11\\
25.5242578125	1.17132759107759e-10\\
25.545546875	1.84754536393158e-10\\
25.5668359375	2.11431805238155e-10\\
25.588125	3.99656735839518e-12\\
25.6094140625	1.15776647901837e-10\\
25.630703125	4.05989834113948e-11\\
25.6519921875	-5.20254085763132e-11\\
25.67328125	-1.27320061785447e-10\\
25.6945703125	-9.8086021434328e-11\\
25.715859375	-1.42083124218279e-10\\
25.7371484375	-4.1155613112651e-11\\
25.7584375	-1.11703759308849e-10\\
25.7797265625	1.81096690225563e-11\\
25.801015625	-1.12411886609605e-12\\
25.8223046875	-1.38566179133168e-10\\
25.84359375	-6.67574520969646e-11\\
25.8648828125	-2.40487401927559e-11\\
25.886171875	-1.82276109743443e-10\\
25.9074609375	-2.17134844376538e-10\\
25.92875	-2.18329299844603e-10\\
25.9500390625	-2.9177232000055e-10\\
25.971328125	-2.2085075984623e-10\\
25.9926171875	-3.68482230629246e-10\\
26.01390625	-3.68858244381326e-10\\
26.0351953125	-4.16228340267807e-10\\
26.056484375	-4.17368891549097e-10\\
26.0777734375	-2.78083512176645e-10\\
26.0990625	-2.43845908184709e-10\\
26.1203515625	-3.70562005575528e-10\\
26.141640625	-1.81332397440185e-10\\
26.1629296875	-2.99353856476542e-10\\
26.18421875	-2.30418102255355e-10\\
26.2055078125	-1.85029341995952e-10\\
26.226796875	-2.6278257436117e-10\\
26.2480859375	-8.86861267141408e-11\\
26.269375	-1.52076616274226e-10\\
26.2906640625	-2.7071017359979e-10\\
26.311953125	-1.77164050780105e-10\\
26.3332421875	-2.51212567242407e-10\\
26.35453125	-2.77897297992131e-10\\
26.3758203125	-2.61358912613243e-10\\
26.397109375	-2.86131311029736e-10\\
26.4183984375	-2.73086062117682e-10\\
26.4396875	-3.11838371100426e-10\\
26.4609765625	-3.56250330251235e-10\\
26.482265625	-3.29049933360613e-10\\
26.5035546875	-3.69341969959457e-10\\
26.52484375	-2.07991458136311e-10\\
26.5461328125	-2.64121630627288e-10\\
26.567421875	-1.78064158432853e-10\\
26.5887109375	-2.39860443532099e-10\\
26.61	-1.95381774975363e-10\\
26.6312890625	-2.19238326845777e-10\\
26.652578125	-3.28448751978437e-10\\
26.6738671875	-2.40604666366882e-10\\
26.69515625	-3.08923772381818e-10\\
26.7164453125	-3.45191517071772e-10\\
26.737734375	-3.73631957645664e-10\\
26.7590234375	-2.85454896473827e-10\\
26.7803125	-4.00567039028297e-10\\
26.8016015625	-3.48927677098778e-10\\
26.822890625	-4.16677835202119e-10\\
26.8441796875	-2.95188429686589e-10\\
26.86546875	-3.97130120978031e-10\\
26.8867578125	-3.84980891622316e-10\\
26.908046875	-4.38255346506717e-10\\
26.9293359375	-3.6886067980668e-10\\
26.950625	-5.31949039556595e-10\\
26.9719140625	-6.29059307594569e-10\\
26.993203125	-6.06801378307022e-10\\
27.0144921875	-7.0029682158165e-10\\
27.03578125	-6.75239333415982e-10\\
27.0570703125	-6.2145904949905e-10\\
27.078359375	-5.83803097808466e-10\\
27.0996484375	-6.3713372007691e-10\\
27.1209375	-5.11591834423829e-10\\
27.1422265625	-5.73989924727654e-10\\
27.163515625	-5.58759710237203e-10\\
27.1848046875	-5.54600033627821e-10\\
27.20609375	-6.57127594353395e-10\\
27.2273828125	-7.22421205522255e-10\\
27.248671875	-6.65888513464801e-10\\
27.2699609375	-5.34040177948356e-10\\
27.29125	-4.54806869624772e-10\\
27.3125390625	-5.64318803941415e-10\\
27.333828125	-4.41348703924612e-10\\
27.3551171875	-4.3747310710682e-10\\
27.37640625	-4.4491677307415e-10\\
27.3976953125	-3.68937199730169e-10\\
27.418984375	-4.90046426132591e-10\\
27.4402734375	-5.05927337189284e-10\\
27.4615625	-4.58981708590815e-10\\
27.4828515625	-5.49060852551632e-10\\
27.504140625	-4.27452489607551e-10\\
27.5254296875	-4.28852717738239e-10\\
27.54671875	-4.51052364636549e-10\\
27.5680078125	-3.60244166791106e-10\\
27.589296875	-3.52692325242294e-10\\
27.6105859375	-3.47050859813758e-10\\
27.631875	-3.39858228825667e-10\\
27.6531640625	-4.06086612305022e-10\\
27.674453125	-4.47182395032155e-10\\
27.6957421875	-4.67863601302105e-10\\
27.71703125	-3.91609494814539e-10\\
27.7383203125	-3.55162078462899e-10\\
27.759609375	-4.80018967543599e-10\\
27.7808984375	-3.01456731252452e-10\\
27.8021875	-3.42850241500017e-10\\
27.8234765625	-2.95267804090739e-10\\
27.844765625	-3.59766463076406e-10\\
27.8660546875	-4.0993945327402e-10\\
27.88734375	-2.97463837742629e-10\\
27.9086328125	-3.10862119468335e-10\\
27.929921875	-3.13334889821135e-10\\
27.9512109375	-2.20460032063107e-10\\
27.9725	-2.91933416939467e-10\\
27.9937890625	-3.21854942274148e-10\\
28.015078125	-2.14425915006893e-10\\
28.0363671875	-2.57962071364024e-10\\
28.05765625	-2.24017751647463e-10\\
28.0789453125	-2.67121617720638e-10\\
28.100234375	-1.14955212645794e-10\\
28.1215234375	-1.79466410086259e-10\\
28.1428125	-9.47427869357416e-11\\
28.1641015625	-9.33403832582024e-11\\
28.185390625	-4.02190554468251e-11\\
28.2066796875	-7.19802163241436e-11\\
28.22796875	-7.04302385060073e-11\\
28.2492578125	1.94644113030931e-11\\
28.270546875	1.04075902073731e-11\\
28.2918359375	-6.34650592467762e-12\\
28.313125	-1.58203553687879e-11\\
28.3344140625	9.73888539479208e-11\\
28.355703125	7.06461778486352e-11\\
28.3769921875	1.14017356933918e-10\\
28.39828125	9.42788162894079e-11\\
28.4195703125	1.00745578582822e-10\\
28.440859375	1.21787934839329e-10\\
28.4621484375	1.39956662926472e-10\\
28.4834375	1.25872727281467e-10\\
28.5047265625	2.79590524344191e-10\\
28.526015625	1.4788386790205e-10\\
28.5473046875	2.55864209026841e-10\\
28.56859375	1.82021224637869e-10\\
28.5898828125	2.29654028111921e-10\\
28.611171875	2.07155165759251e-10\\
28.6324609375	2.16883444418954e-10\\
28.65375	2.08386971274226e-10\\
28.6750390625	2.5133089877906e-10\\
28.696328125	1.9755918389263e-10\\
28.7176171875	3.49555817078076e-10\\
28.73890625	2.42065825651319e-10\\
28.7601953125	2.25838417284189e-10\\
28.781484375	3.31345787033415e-10\\
28.8027734375	2.57241479151951e-10\\
28.8240625	1.91659278704981e-10\\
28.8453515625	2.47751381817953e-10\\
28.866640625	1.99948370588542e-10\\
28.8879296875	1.33778340822034e-10\\
28.90921875	1.56985153165584e-10\\
28.9305078125	2.44591013119948e-10\\
28.951796875	1.63339199836795e-10\\
28.9730859375	3.18570575202494e-10\\
28.994375	4.48725221972665e-10\\
29.0156640625	3.49468301226423e-10\\
29.036953125	3.40424348320372e-10\\
29.0582421875	3.22858720951954e-10\\
29.07953125	1.7891758936729e-10\\
29.1008203125	1.79249400069567e-10\\
29.122109375	1.84189414391685e-10\\
29.1433984375	2.94305693199139e-10\\
29.1646875	4.15736416652113e-10\\
29.1859765625	3.70226423085435e-10\\
29.207265625	5.58309166843542e-10\\
29.2285546875	6.94162839899804e-10\\
29.24984375	6.87176454969545e-10\\
29.2711328125	6.14369774360639e-10\\
29.292421875	7.23443186150778e-10\\
29.3137109375	6.26511411818267e-10\\
29.335	6.28943473806274e-10\\
29.3562890625	6.79114701437854e-10\\
29.377578125	6.72315149840238e-10\\
29.3988671875	6.52873328774492e-10\\
29.42015625	7.63536233457689e-10\\
29.4414453125	7.08616351191427e-10\\
29.462734375	8.23686625556865e-10\\
29.4840234375	7.63023249593945e-10\\
29.5053125	6.98184107261037e-10\\
29.5266015625	7.52323246083654e-10\\
29.547890625	6.58945000369631e-10\\
29.5691796875	6.16151357883989e-10\\
29.59046875	6.53305393038639e-10\\
29.6117578125	5.93760007245985e-10\\
29.633046875	6.58597658438648e-10\\
29.6543359375	7.00483504740285e-10\\
29.675625	7.63334796683438e-10\\
29.6969140625	7.48863781816072e-10\\
29.718203125	7.63111455965264e-10\\
29.7394921875	6.45133362530636e-10\\
29.76078125	6.08126683609181e-10\\
29.7820703125	4.58416958677814e-10\\
29.803359375	4.49659247037792e-10\\
29.8246484375	4.8587477817537e-10\\
29.8459375	3.30514893673433e-10\\
29.8672265625	4.04272578601008e-10\\
29.888515625	4.04228553067768e-10\\
29.9098046875	4.76942156593241e-10\\
29.93109375	5.75840281915313e-10\\
29.9523828125	5.23020940162133e-10\\
29.973671875	5.55451648002061e-10\\
29.9949609375	5.89446180576422e-10\\
30.01625	4.73823169603863e-10\\
30.0375390625	5.09620193139124e-10\\
30.058828125	5.2399870271731e-10\\
30.0801171875	3.69072279073834e-10\\
30.10140625	4.35375550299473e-10\\
30.1226953125	4.31983720085162e-10\\
30.143984375	4.65981653647796e-10\\
30.1652734375	5.79641993711298e-10\\
30.1865625	5.94157041686558e-10\\
30.2078515625	7.74371667408459e-10\\
30.229140625	6.67740563619993e-10\\
30.2504296875	6.43021134955607e-10\\
30.27171875	7.2859412960463e-10\\
30.2930078125	5.63440010530527e-10\\
30.314296875	5.35297826137718e-10\\
30.3355859375	5.28856832770611e-10\\
30.356875	6.51975986306211e-10\\
30.3781640625	6.40105137696664e-10\\
30.399453125	6.07025829450042e-10\\
30.4207421875	5.54084043504461e-10\\
30.44203125	6.57472279965824e-10\\
30.4633203125	5.41405113582194e-10\\
30.484609375	5.30821282197887e-10\\
30.5058984375	4.895431921191e-10\\
30.5271875	3.66497961264413e-10\\
30.5484765625	3.09759107021767e-10\\
30.569765625	4.48222294240312e-10\\
30.5910546875	3.46336991803764e-10\\
30.61234375	3.59632612028264e-10\\
30.6336328125	3.51958101292241e-10\\
30.654921875	3.33316096331099e-10\\
30.6762109375	3.52545177396323e-10\\
30.6975	3.04084051307655e-10\\
30.7187890625	2.48902124104581e-10\\
30.740078125	1.84728053599365e-10\\
30.7613671875	1.15727352658399e-10\\
30.78265625	1.26796514767682e-10\\
30.8039453125	1.34829661833202e-10\\
30.825234375	3.86274965917873e-11\\
30.8465234375	-2.34409902123533e-11\\
30.8678125	1.33030105700264e-10\\
30.8891015625	4.09045832100416e-11\\
30.910390625	3.60732635415184e-11\\
30.9316796875	7.55195948289473e-11\\
30.95296875	1.10242377124982e-11\\
30.9742578125	-7.47175510105421e-11\\
30.995546875	-1.16142881681247e-10\\
31.0168359375	-5.45770405406494e-12\\
31.038125	9.00536963242888e-11\\
31.0594140625	-6.03852052130937e-11\\
31.080703125	2.88282157906977e-11\\
31.1019921875	-2.77737287628367e-11\\
31.12328125	-1.15822618448963e-10\\
31.1445703125	-2.71451437567864e-11\\
31.165859375	-1.63206592895577e-10\\
31.1871484375	-8.56696575073816e-11\\
31.2084375	-2.04427447753867e-10\\
31.2297265625	-2.11143940356065e-10\\
31.251015625	-1.51496456172513e-10\\
31.2723046875	-4.74738477796713e-11\\
31.29359375	-8.28097451670382e-11\\
31.3148828125	1.4498790321883e-12\\
31.336171875	5.70717122780795e-11\\
31.3574609375	4.77062317458265e-11\\
31.37875	4.21876452889772e-11\\
31.4000390625	-9.51377539983313e-11\\
31.421328125	-2.52509150373814e-11\\
31.4426171875	-8.13529999112821e-11\\
31.46390625	-1.2849444267684e-10\\
31.4851953125	-1.06383042272568e-10\\
31.506484375	-9.58684796621911e-11\\
31.5277734375	-3.54305177776258e-11\\
31.5490625	-1.11662835206352e-10\\
31.5703515625	-1.64528605970653e-10\\
31.591640625	-1.06681707207736e-10\\
31.6129296875	-1.80380998011698e-10\\
31.63421875	-1.72167973830673e-10\\
31.6555078125	-3.03732564640437e-10\\
31.676796875	-2.30863201659417e-10\\
31.6980859375	-2.99327674908605e-10\\
31.719375	-3.80753506318839e-10\\
31.7406640625	-3.37480292462656e-10\\
31.761953125	-3.12775224495374e-10\\
31.7832421875	-3.79954328666279e-10\\
31.80453125	-4.61590771216684e-10\\
31.8258203125	-4.1865486261432e-10\\
31.847109375	-5.41321951786698e-10\\
31.8683984375	-5.40768858275724e-10\\
31.8896875	-3.44328597230182e-10\\
31.9109765625	-4.81587827014558e-10\\
31.932265625	-3.83329757837285e-10\\
31.9535546875	-3.44701308748028e-10\\
31.97484375	-3.99041524775324e-10\\
31.9961328125	-3.41182173381774e-10\\
32.017421875	-4.27724435087828e-10\\
32.0387109375	-4.8234805890724e-10\\
32.06	-6.08622475448094e-10\\
32.0812890625	-6.066091818091e-10\\
32.102578125	-6.10938033463399e-10\\
32.1238671875	-6.05846502538458e-10\\
32.14515625	-5.98315603504085e-10\\
32.1664453125	-6.1883016346063e-10\\
32.187734375	-5.19193585385833e-10\\
32.2090234375	-6.5051580333068e-10\\
32.2303125	-5.07242091885877e-10\\
32.2516015625	-5.55369646778163e-10\\
32.272890625	-5.9333334195827e-10\\
32.2941796875	-5.55516771860169e-10\\
32.31546875	-4.937839093415e-10\\
32.3367578125	-4.86041087105805e-10\\
32.358046875	-3.62964407872819e-10\\
32.3793359375	-4.49246892124569e-10\\
32.400625	-3.98499316160214e-10\\
32.4219140625	-4.6937774961157e-10\\
32.443203125	-3.92726721975597e-10\\
32.4644921875	-4.06308292031266e-10\\
32.48578125	-4.63615554152411e-10\\
32.5070703125	-4.02913109519368e-10\\
32.528359375	-4.06251891061027e-10\\
32.5496484375	-5.11186205350952e-10\\
32.5709375	-4.79248152835347e-10\\
32.5922265625	-5.05238297557631e-10\\
32.613515625	-4.22224531814137e-10\\
32.6348046875	-4.93971007633149e-10\\
32.65609375	-4.1890724993712e-10\\
32.6773828125	-4.43447475850846e-10\\
32.698671875	-4.92571800156843e-10\\
32.7199609375	-4.59032447321095e-10\\
32.74125	-4.38383274875566e-10\\
32.7625390625	-5.24807744765388e-10\\
32.783828125	-4.86654404442689e-10\\
32.8051171875	-4.46841252921964e-10\\
32.82640625	-5.14687940538368e-10\\
32.8476953125	-4.41460764081208e-10\\
32.868984375	-5.12532401750963e-10\\
32.8902734375	-4.41582306538466e-10\\
32.9115625	-3.02119417610294e-10\\
32.9328515625	-3.33857548161153e-10\\
32.954140625	-3.58643802377471e-10\\
32.9754296875	-3.09925602615817e-10\\
32.99671875	-3.35623216535786e-10\\
33.0180078125	-3.20919648231627e-10\\
33.039296875	-4.28625002469525e-10\\
33.0605859375	-3.74916474566712e-10\\
33.081875	-4.29204660232841e-10\\
33.1031640625	-3.05791251910703e-10\\
33.124453125	-2.87483456480301e-10\\
33.1457421875	-1.90598534906425e-10\\
33.16703125	-1.88662941655157e-10\\
33.1883203125	-1.3250278932475e-10\\
33.209609375	-3.33477836842592e-11\\
33.2308984375	-9.28293884019534e-11\\
33.2521875	-1.20558695735544e-10\\
33.2734765625	-9.17729238865862e-11\\
33.294765625	-2.32800305728636e-11\\
33.3160546875	-1.27171764552487e-11\\
33.33734375	3.90138219332115e-12\\
33.3586328125	-7.13000629153796e-12\\
33.379921875	1.12958990073535e-10\\
33.4012109375	9.81123279793846e-11\\
33.4225	1.38785906930368e-10\\
33.4437890625	1.99547009251408e-10\\
33.465078125	1.31478042888106e-10\\
33.4863671875	1.78729116536505e-10\\
33.50765625	1.73238574938948e-10\\
33.5289453125	1.65022945444541e-10\\
33.550234375	2.12936355237769e-10\\
33.5715234375	3.02370459405742e-10\\
33.5928125	2.20229123689038e-10\\
33.6141015625	3.94505485156645e-10\\
33.635390625	3.28062181986701e-10\\
33.6566796875	3.66707364810242e-10\\
33.67796875	3.96397652809237e-10\\
33.6992578125	4.12139359954121e-10\\
33.720546875	4.20615122133353e-10\\
33.7418359375	4.80666865305143e-10\\
33.763125	3.17697960017265e-10\\
33.7844140625	3.20628271068752e-10\\
33.805703125	4.02363640022926e-10\\
33.8269921875	4.11392262040223e-10\\
33.84828125	5.72862264040168e-10\\
33.8695703125	4.76463198246973e-10\\
33.890859375	5.98935095356347e-10\\
33.9121484375	6.27745983336543e-10\\
33.9334375	6.17305946277937e-10\\
33.9547265625	5.51822477489757e-10\\
33.976015625	6.63836777083851e-10\\
33.9973046875	5.42682725495723e-10\\
34.01859375	5.27702726126858e-10\\
34.0398828125	5.51318301785455e-10\\
34.061171875	6.22159118982606e-10\\
34.0824609375	6.49682885197995e-10\\
34.10375	6.48024022764948e-10\\
34.1250390625	7.25194346963017e-10\\
34.146328125	7.21610323060133e-10\\
34.1676171875	7.23735864966753e-10\\
34.18890625	6.7576163577259e-10\\
34.2101953125	7.22678144435696e-10\\
34.231484375	7.64755879187276e-10\\
34.2527734375	7.56914100487286e-10\\
34.2740625	7.63338742083156e-10\\
34.2953515625	7.95837696128556e-10\\
34.316640625	9.26154306338221e-10\\
34.3379296875	7.81039767920654e-10\\
34.35921875	8.46823270518439e-10\\
34.3805078125	7.86475896788657e-10\\
34.401796875	7.51777820061131e-10\\
34.4230859375	7.5041642785474e-10\\
34.444375	7.29665642000465e-10\\
34.4656640625	6.76413344546107e-10\\
34.486953125	8.10343747521999e-10\\
34.5082421875	7.13943169745285e-10\\
34.52953125	9.28854065976921e-10\\
34.5508203125	8.71051902416255e-10\\
34.572109375	8.60045218230235e-10\\
34.5933984375	8.54281354124551e-10\\
34.6146875	8.13371198933652e-10\\
34.6359765625	7.25790487232269e-10\\
34.657265625	7.36975192514235e-10\\
34.6785546875	6.68623529819918e-10\\
34.69984375	6.78793643174012e-10\\
34.7211328125	6.76326595916278e-10\\
34.742421875	7.700219798784e-10\\
34.7637109375	7.92975506824458e-10\\
34.785	8.10794833122018e-10\\
34.8062890625	7.25894822633616e-10\\
34.827578125	7.22007421972463e-10\\
34.8488671875	6.49186776014053e-10\\
34.87015625	6.42448242989892e-10\\
34.8914453125	6.48708348092487e-10\\
34.912734375	6.51218411513337e-10\\
34.9340234375	7.03413375282838e-10\\
34.9553125	5.84839397472278e-10\\
34.9766015625	7.19331301089756e-10\\
34.997890625	6.24580029191536e-10\\
35.0191796875	6.75052463374933e-10\\
35.04046875	7.06368319296153e-10\\
35.0617578125	7.12228978660752e-10\\
35.083046875	7.26512299129178e-10\\
35.1043359375	5.89280656928108e-10\\
35.125625	6.51488242034798e-10\\
35.1469140625	6.02357596944663e-10\\
35.168203125	6.14249137786842e-10\\
35.1894921875	6.13582536811986e-10\\
35.21078125	5.63688477805633e-10\\
35.2320703125	6.5360331931929e-10\\
35.253359375	6.54443221564278e-10\\
35.2746484375	6.72219669396964e-10\\
35.2959375	5.82282749876021e-10\\
35.3172265625	5.35401372581174e-10\\
35.338515625	4.54376061224644e-10\\
35.3598046875	3.48053128039786e-10\\
35.38109375	3.19764809058352e-10\\
35.4023828125	2.78020430791722e-10\\
35.423671875	2.78213936189253e-10\\
35.4449609375	3.57005979006238e-10\\
35.46625	2.97133362668279e-10\\
35.4875390625	3.15438364557601e-10\\
35.508828125	2.76266393679294e-10\\
35.5301171875	3.43220045712347e-10\\
35.55140625	3.50489854518312e-10\\
35.5726953125	3.03138800185101e-10\\
35.593984375	2.8920294234839e-10\\
35.6152734375	1.27769240694467e-10\\
35.6365625	7.72918404130641e-11\\
35.6578515625	3.88651041972087e-12\\
35.679140625	9.42249591534427e-11\\
35.7004296875	3.89247255247886e-11\\
35.72171875	2.43629040180961e-11\\
35.7430078125	3.94534941945335e-11\\
35.764296875	1.00773131054493e-10\\
35.7855859375	3.4619188963728e-11\\
35.806875	5.15398400846899e-11\\
35.8281640625	1.41515092234051e-11\\
35.849453125	-4.4461823790186e-11\\
35.8707421875	-9.62946766698387e-11\\
35.89203125	-1.74180195011138e-10\\
35.9133203125	-1.19235498079789e-10\\
35.934609375	-1.38042669550183e-10\\
35.9558984375	-1.69177236710774e-10\\
35.9771875	-8.51953245373013e-14\\
35.9984765625	-9.68971217641521e-11\\
36.019765625	2.18483235837652e-12\\
36.0410546875	-4.58810165322539e-11\\
36.06234375	-1.18836779714902e-10\\
36.0836328125	-1.20160298949032e-10\\
36.104921875	-5.89894170970123e-11\\
36.1262109375	-1.71003725372062e-10\\
36.1475	-8.45551630265801e-11\\
36.1687890625	-1.44138259102279e-10\\
36.190078125	-8.14748156394559e-11\\
36.2113671875	-4.94231283830485e-11\\
36.23265625	-1.7954202476303e-10\\
36.2539453125	-1.14805412807567e-10\\
36.275234375	-1.01685579294695e-10\\
36.2965234375	-1.43642405941119e-10\\
36.3178125	-1.76071800747459e-10\\
36.3391015625	-3.35244922545154e-10\\
36.360390625	-3.70074788610989e-10\\
36.3816796875	-3.18234795858478e-10\\
36.40296875	-3.49027062408566e-10\\
36.4242578125	-2.59381898855663e-10\\
36.445546875	-3.09894418242237e-10\\
36.4668359375	-2.32673362632674e-10\\
36.488125	-2.30206678376576e-10\\
36.5094140625	-1.86252627652367e-10\\
36.530703125	-1.78116217637055e-10\\
36.5519921875	-3.29719267612119e-10\\
36.57328125	-3.43360862961043e-10\\
36.5945703125	-3.46918708769332e-10\\
36.615859375	-4.78780803710584e-10\\
36.6371484375	-3.4720029404436e-10\\
36.6584375	-3.7203943780691e-10\\
36.6797265625	-3.82828177829643e-10\\
36.701015625	-3.06106619152454e-10\\
36.7223046875	-4.48669675561141e-10\\
36.74359375	-4.00938645624016e-10\\
36.7648828125	-4.39078516891502e-10\\
36.786171875	-4.07783058935124e-10\\
36.8074609375	-4.26691000081122e-10\\
36.82875	-3.31194523129717e-10\\
36.8500390625	-3.33554278396036e-10\\
36.871328125	-2.92053649399986e-10\\
36.8926171875	-3.22236193103545e-10\\
36.91390625	-3.56258548351512e-10\\
36.9351953125	-3.96716919256048e-10\\
36.956484375	-2.58469815254747e-10\\
36.9777734375	-3.94989708793007e-10\\
36.9990625	-3.1199458296725e-10\\
37.0203515625	-2.91248148619379e-10\\
37.041640625	-2.45167420192207e-10\\
37.0629296875	-2.70800570750804e-10\\
37.08421875	-1.67728099090571e-10\\
37.1055078125	-2.28150735390642e-10\\
37.126796875	-1.26282572268367e-10\\
37.1480859375	-1.88970789561184e-10\\
37.169375	-2.93955594356535e-10\\
37.1906640625	-1.80844451545027e-10\\
37.211953125	-2.35701170386711e-10\\
37.2332421875	-1.92376570612891e-10\\
37.25453125	-1.74619708619199e-10\\
37.2758203125	-1.71954472290337e-10\\
37.297109375	-2.16495328574308e-10\\
37.3183984375	-1.21407976468025e-10\\
37.3396875	-1.16838859832664e-10\\
37.3609765625	-1.42570421250399e-10\\
37.382265625	-1.33952907158364e-10\\
37.4035546875	-1.95904821467281e-10\\
37.42484375	-1.69177631780966e-10\\
37.4461328125	-1.75847357846583e-10\\
37.467421875	-1.81549305278873e-10\\
37.4887109375	-1.95013229209412e-10\\
37.51	-1.86090237115321e-10\\
37.5312890625	-2.11686421952329e-10\\
37.552578125	-7.07608071011366e-11\\
37.5738671875	-2.21554161306915e-10\\
37.59515625	-1.14952053603221e-10\\
37.6164453125	-1.47496927695395e-10\\
37.637734375	-1.63130757892225e-10\\
};
\addplot [color=mycolor2,solid]
  table[row sep=crcr]{%
37.637734375	-1.63130757892225e-10\\
37.6590234375	-7.97369229213885e-11\\
37.6803125	-6.22906475659067e-11\\
37.7016015625	-9.53500846524714e-11\\
37.722890625	-4.88999834670626e-11\\
37.7441796875	-1.58669468371519e-10\\
37.76546875	-1.39375055615476e-10\\
37.7867578125	-1.49420998803656e-10\\
37.808046875	-9.3121152061739e-11\\
37.8293359375	-8.1635628348648e-11\\
37.850625	-2.08110776668755e-11\\
37.8719140625	2.63632447277036e-11\\
37.893203125	1.42158671568583e-11\\
37.9144921875	1.65306032717471e-11\\
37.93578125	3.38142640810466e-11\\
37.9570703125	-1.19728244848437e-10\\
37.978359375	-5.03077501767871e-12\\
37.9996484375	-4.38210986774569e-12\\
38.0209375	4.67295832293269e-11\\
38.0422265625	-1.36310618358766e-11\\
38.063515625	-4.29626504572772e-11\\
38.0848046875	4.57712679004824e-11\\
38.10609375	7.86912511256589e-11\\
38.1273828125	5.13736013596336e-11\\
38.148671875	1.17706970389003e-10\\
38.1699609375	3.10306584386193e-10\\
38.19125	1.02620788676247e-10\\
38.2125390625	2.33484225897023e-10\\
38.233828125	2.7361551208306e-10\\
38.2551171875	1.23973707908288e-10\\
38.27640625	1.77557526740291e-10\\
38.2976953125	2.06557818886609e-10\\
38.318984375	2.25903569233823e-10\\
38.3402734375	2.93155285987555e-10\\
38.3615625	2.59212870324512e-10\\
38.3828515625	4.71860788314097e-10\\
38.404140625	3.81390473839802e-10\\
38.4254296875	3.29720778567367e-10\\
38.44671875	3.61784877375994e-10\\
38.4680078125	3.12965110017867e-10\\
38.489296875	2.76711900322095e-10\\
38.5105859375	1.66472549061198e-10\\
38.531875	2.63906452126571e-10\\
38.5531640625	2.85583779750312e-10\\
38.574453125	2.64677564712546e-10\\
38.5957421875	3.57276440537439e-10\\
38.61703125	2.68138889188777e-10\\
38.6383203125	3.52577792389365e-10\\
38.659609375	2.6264819258358e-10\\
38.6808984375	2.82489937890505e-10\\
38.7021875	2.36417637119918e-10\\
38.7234765625	2.134629975305e-10\\
38.744765625	2.99860903628545e-10\\
38.7660546875	2.82221305891155e-10\\
38.78734375	3.27453159355438e-10\\
38.8086328125	3.09073823441147e-10\\
38.829921875	2.96458299950243e-10\\
38.8512109375	3.23960825329706e-10\\
38.8725	3.68696620331543e-10\\
38.8937890625	3.13734832671966e-10\\
38.915078125	4.77505204829949e-10\\
38.9363671875	3.33702787507932e-10\\
38.95765625	3.94162161168435e-10\\
38.9789453125	2.79576402074906e-10\\
39.000234375	4.13088434254741e-10\\
39.0215234375	3.36547355966868e-10\\
39.0428125	3.93478802199085e-10\\
39.0641015625	3.66020665899717e-10\\
39.085390625	4.00491895445997e-10\\
39.1066796875	2.60226453798223e-10\\
39.12796875	3.98707131718856e-10\\
39.1492578125	3.14598031137782e-10\\
39.170546875	4.52995411018862e-10\\
39.1918359375	3.97196057974408e-10\\
39.213125	4.05895368463632e-10\\
39.2344140625	5.22222405737365e-10\\
39.255703125	4.81387441768832e-10\\
39.2769921875	4.4875050195245e-10\\
39.29828125	5.23349236767809e-10\\
39.3195703125	4.27992804405767e-10\\
39.340859375	3.56886042643439e-10\\
39.3621484375	3.87824400108962e-10\\
39.3834375	4.56282158232691e-10\\
39.4047265625	4.62809528270682e-10\\
39.426015625	4.82299187391743e-10\\
39.4473046875	4.649681170081e-10\\
39.46859375	3.96996992416719e-10\\
39.4898828125	4.56582976774853e-10\\
39.511171875	2.7318297280086e-10\\
39.5324609375	3.10236090569664e-10\\
39.55375	3.2587441684116e-10\\
39.5750390625	3.1294964660865e-10\\
39.596328125	2.80744978288394e-10\\
39.6176171875	3.40838426636109e-10\\
39.63890625	2.5333153915834e-10\\
39.6601953125	3.66037221743326e-10\\
39.681484375	3.00720246904267e-10\\
39.7027734375	2.34036077097141e-10\\
39.7240625	2.35522824887901e-10\\
39.7453515625	2.15531861245649e-10\\
39.766640625	2.14317401722598e-10\\
39.7879296875	2.06980350855339e-10\\
39.80921875	2.37854234365573e-10\\
39.8305078125	1.2927307656204e-10\\
39.851796875	1.63322056739532e-10\\
39.8730859375	1.09205002736899e-10\\
39.894375	1.23083162008274e-10\\
39.9156640625	9.69543542118517e-11\\
39.936953125	7.8060098091682e-11\\
39.9582421875	1.55132191058775e-10\\
39.97953125	2.08027280682556e-10\\
40.0008203125	1.70440970196775e-10\\
40.022109375	2.28549284948632e-10\\
40.0433984375	2.22990366624805e-10\\
40.0646875	6.83722405024638e-11\\
40.0859765625	1.06514138295526e-10\\
40.107265625	2.22896560262307e-11\\
40.1285546875	4.62996801050248e-11\\
40.14984375	-3.49000502103545e-11\\
40.1711328125	2.76188478136547e-11\\
40.192421875	4.32666934508108e-11\\
40.2137109375	-5.15854130815537e-12\\
40.235	2.77323319662272e-11\\
40.2562890625	9.38944989640884e-11\\
40.277578125	8.15163759751565e-11\\
40.2988671875	4.44085857769311e-11\\
40.32015625	1.21385582402326e-10\\
40.3414453125	2.32067638089357e-11\\
40.362734375	6.71366544462215e-11\\
40.3840234375	-4.35410885185747e-11\\
40.4053125	-5.84687796883608e-11\\
40.4266015625	-4.14181588284635e-11\\
40.447890625	-3.19987195708211e-11\\
40.4691796875	-3.33818323758341e-12\\
40.49046875	2.10506173650344e-11\\
40.5117578125	-6.64941279623174e-11\\
40.533046875	-8.41342875660557e-11\\
40.5543359375	-9.23127219613322e-12\\
40.575625	2.04557072658477e-11\\
40.5969140625	1.54993368914426e-11\\
40.618203125	-2.3302815546764e-11\\
40.6394921875	-5.33517384429479e-11\\
40.66078125	-1.37840642056005e-10\\
40.6820703125	-1.73146932847494e-10\\
40.703359375	-1.41396830276314e-10\\
40.7246484375	-2.46911437075672e-10\\
40.7459375	-1.89756214022714e-10\\
40.7672265625	-1.43121619236505e-10\\
40.788515625	-9.10968996174194e-11\\
40.8098046875	-1.23194532017648e-10\\
40.83109375	-1.17466365445559e-10\\
40.8523828125	-9.89985636603393e-11\\
40.873671875	-1.64275769972072e-10\\
40.8949609375	-3.02291150644738e-10\\
40.91625	-1.94845462543459e-10\\
40.9375390625	-3.2197232729281e-10\\
40.958828125	-2.9952573266637e-10\\
40.9801171875	-2.7694898901328e-10\\
41.00140625	-1.51409597796761e-10\\
41.0226953125	-2.64297767293946e-10\\
41.043984375	-2.97072257714676e-10\\
41.0652734375	-2.94726786769058e-10\\
41.0865625	-3.03256351039253e-10\\
41.1078515625	-3.57245086184701e-10\\
41.129140625	-3.5208855391659e-10\\
41.1504296875	-3.20269157340335e-10\\
41.17171875	-2.82404757852787e-10\\
41.1930078125	-3.44250193400323e-10\\
41.214296875	-2.05960743032821e-10\\
41.2355859375	-2.81225527618869e-10\\
41.256875	-2.61287768965749e-10\\
41.2781640625	-2.36575430544422e-10\\
41.299453125	-1.91720328696053e-10\\
41.3207421875	-2.43919098180219e-10\\
41.34203125	-2.21730477814154e-10\\
41.3633203125	-2.0014410237884e-10\\
41.384609375	-1.45258977235242e-10\\
41.4058984375	-1.80881423990352e-10\\
41.4271875	-2.4814342648017e-10\\
41.4484765625	-1.8470493862693e-10\\
41.469765625	-1.76891807712035e-10\\
41.4910546875	-2.05019491780577e-10\\
41.51234375	-2.09448335909649e-10\\
41.5336328125	-1.0736546521474e-10\\
41.554921875	-1.84414026142275e-10\\
41.5762109375	-6.35473963379058e-11\\
41.5975	-2.30770933640111e-10\\
41.6187890625	-1.24290802435631e-10\\
41.640078125	-1.88627146826496e-10\\
41.6613671875	-1.45768150327054e-10\\
41.68265625	-2.67095164101819e-10\\
41.7039453125	-1.77405225930672e-10\\
41.725234375	-2.69516955924327e-10\\
41.7465234375	-2.3117670936771e-10\\
41.7678125	-2.19211175549573e-10\\
41.7891015625	-2.55642505471185e-10\\
41.810390625	-2.47912976650192e-10\\
41.8316796875	-2.23092978903346e-10\\
41.85296875	-2.65997002002252e-10\\
41.8742578125	-2.11001298766304e-10\\
41.895546875	-2.00765315364186e-10\\
41.9168359375	-2.76262275781023e-10\\
41.938125	-3.08192935408068e-10\\
41.9594140625	-2.72902908793936e-10\\
41.980703125	-2.47145637079367e-10\\
42.0019921875	-3.65457734947279e-10\\
42.02328125	-2.24393649450118e-10\\
42.0445703125	-3.09800054129899e-10\\
42.065859375	-2.18106955538817e-10\\
42.0871484375	-3.00466156784025e-10\\
42.1084375	-2.77346906322737e-10\\
42.1297265625	-2.83201912583972e-10\\
42.151015625	-2.8085476158612e-10\\
42.1723046875	-2.27166423323378e-10\\
42.19359375	-1.70591443138839e-10\\
42.2148828125	-2.25285583627504e-10\\
42.236171875	-1.58211217754609e-10\\
42.2574609375	-1.73877472431099e-10\\
42.27875	-1.94320355939363e-10\\
42.3000390625	-2.53361373113642e-10\\
42.321328125	-2.99989584264006e-10\\
42.3426171875	-2.40435498013601e-10\\
42.36390625	-2.27666612239393e-10\\
42.3851953125	-1.86851314875763e-10\\
42.406484375	-1.59682191753286e-10\\
42.4277734375	-9.55125726208373e-11\\
42.4490625	-9.72570501065874e-11\\
42.4703515625	-1.15078907354809e-10\\
42.491640625	-7.00913679649101e-11\\
42.5129296875	-9.21417512620898e-11\\
42.53421875	-1.58756253726784e-10\\
42.5555078125	-1.50604309894526e-10\\
42.576796875	-1.49289809688844e-10\\
42.5980859375	-7.49946323641649e-11\\
42.619375	-1.83085158387984e-11\\
42.6406640625	-2.59399690860064e-11\\
42.661953125	5.09195873206197e-11\\
42.6832421875	2.01514865409473e-11\\
42.70453125	7.92549345559907e-11\\
42.7258203125	-3.32236030004635e-11\\
42.747109375	5.44919932695028e-11\\
42.7683984375	-1.81855339496337e-11\\
42.7896875	-1.67581512407736e-11\\
42.8109765625	-3.46950211889705e-11\\
42.832265625	5.60071337280547e-11\\
42.8535546875	1.6963044376617e-11\\
42.87484375	8.87970070582943e-11\\
42.8961328125	4.00943469026483e-11\\
42.917421875	5.71119924397235e-11\\
42.9387109375	1.51080571028799e-10\\
42.96	5.29059768277594e-11\\
42.9812890625	9.31310219355455e-11\\
43.002578125	6.04582516909351e-11\\
43.0238671875	3.74211921879328e-11\\
43.04515625	4.1628756049216e-11\\
43.0664453125	6.41096329653806e-11\\
43.087734375	6.2296653079519e-11\\
43.1090234375	2.47867698647124e-11\\
43.1303125	-5.64339233302818e-11\\
43.1516015625	-2.1059824307411e-11\\
43.172890625	-5.05236924890796e-11\\
43.1941796875	3.52738395405565e-11\\
43.21546875	1.05086275442423e-10\\
43.2367578125	1.66735122455943e-10\\
43.258046875	1.04445387158536e-10\\
43.2793359375	1.19050861573831e-10\\
43.300625	1.49274338071985e-10\\
43.3219140625	1.16800971656494e-10\\
43.343203125	1.14359045166638e-10\\
43.3644921875	9.21917207819751e-11\\
43.38578125	8.006226839491e-11\\
43.4070703125	1.14072841361535e-10\\
43.428359375	1.11330142860445e-10\\
43.4496484375	2.36718528040921e-10\\
43.4709375	2.01133590768485e-10\\
43.4922265625	2.92084036262478e-10\\
43.513515625	2.87159999688902e-10\\
43.5348046875	3.48933712562698e-10\\
43.55609375	2.77259131681684e-10\\
43.5773828125	3.84698984538053e-10\\
43.598671875	2.60227558725681e-10\\
43.6199609375	2.30726796728958e-10\\
43.64125	3.07856328978232e-10\\
43.6625390625	2.16372446667279e-10\\
43.683828125	2.85548811910464e-10\\
43.7051171875	3.28729575677689e-10\\
43.72640625	3.56024478691239e-10\\
43.7476953125	3.61490128583409e-10\\
43.768984375	3.91492795811695e-10\\
43.7902734375	3.92339930969701e-10\\
43.8115625	3.63559209389537e-10\\
43.8328515625	3.0614137318221e-10\\
43.854140625	2.91467353247698e-10\\
43.8754296875	2.56699384847458e-10\\
43.89671875	1.59906620980652e-10\\
43.9180078125	2.44778288316132e-10\\
43.939296875	2.17327783114972e-10\\
43.9605859375	2.56752340525391e-10\\
43.981875	2.4772065648573e-10\\
44.0031640625	3.78119416758209e-10\\
44.024453125	2.79664424380854e-10\\
44.0457421875	2.84596674696701e-10\\
44.06703125	2.37184992630409e-10\\
44.0883203125	2.17298746993377e-10\\
44.109609375	2.42303789924099e-10\\
44.1308984375	1.51331716226736e-10\\
44.1521875	1.88113143795726e-10\\
44.1734765625	2.13665408804574e-10\\
44.194765625	2.97489935439001e-10\\
44.2160546875	2.47116512885062e-10\\
44.23734375	3.6551161696042e-10\\
44.2586328125	2.3234105410924e-10\\
44.279921875	3.12704462387597e-10\\
44.3012109375	2.59790660095617e-10\\
44.3225	1.97892486892453e-10\\
44.3437890625	3.4612435011002e-10\\
44.365078125	2.72958387166521e-10\\
44.3863671875	2.86201935569937e-10\\
44.40765625	2.49971053927978e-10\\
44.4289453125	2.86482132703379e-10\\
44.450234375	3.15923093572811e-10\\
44.4715234375	3.81026469333829e-10\\
44.4928125	3.056579184548e-10\\
44.5141015625	3.06599622545502e-10\\
44.535390625	2.6117889061283e-10\\
44.5566796875	2.85614991604578e-10\\
44.57796875	2.43306586524304e-10\\
44.5992578125	2.68400314869954e-10\\
44.620546875	1.85254393974226e-10\\
44.6418359375	2.57046658555056e-10\\
44.663125	1.90519619617397e-10\\
44.6844140625	2.75672798044931e-10\\
44.705703125	2.61100137881529e-10\\
44.7269921875	2.68402947497765e-10\\
44.74828125	2.33819735191705e-10\\
44.7695703125	1.77328870581969e-10\\
44.790859375	2.04671631412902e-10\\
44.8121484375	1.55677811108701e-10\\
44.8334375	1.50958906158865e-10\\
44.8547265625	1.79754186926346e-10\\
44.876015625	2.72089140204209e-10\\
44.8973046875	2.07036173930595e-10\\
44.91859375	2.23817443172295e-10\\
44.9398828125	1.77237960087965e-10\\
44.961171875	1.68205783174071e-10\\
44.9824609375	1.40645789289222e-10\\
45.00375	1.40624677053702e-10\\
45.0250390625	1.5731280073018e-10\\
45.046328125	1.59067372923951e-10\\
45.0676171875	2.14348453601228e-10\\
45.08890625	2.58488452267647e-10\\
45.1101953125	1.63044224638978e-10\\
45.131484375	1.44829725617746e-10\\
45.1527734375	8.01415325577528e-11\\
45.1740625	1.09642943009563e-11\\
45.1953515625	-2.55777248381712e-11\\
45.216640625	6.05317126958304e-12\\
45.2379296875	-2.54205026821302e-11\\
45.25921875	-6.59685046036079e-11\\
45.2805078125	-3.06854634083e-11\\
45.301796875	-5.01791297287142e-11\\
45.3230859375	1.18181547773331e-11\\
45.344375	6.32452649972998e-12\\
45.3656640625	-4.41418822335738e-12\\
45.386953125	-1.79389799728878e-11\\
45.4082421875	-2.05675658020037e-11\\
45.42953125	-3.29530825597936e-11\\
45.4508203125	-6.65365644927157e-11\\
45.472109375	-2.36207002004853e-11\\
45.4933984375	-1.11531251921438e-10\\
45.5146875	-5.8837750989759e-11\\
45.5359765625	4.90972635104895e-11\\
45.557265625	-2.45170865684303e-11\\
45.5785546875	1.0207524017454e-11\\
45.59984375	-1.28404978377437e-11\\
45.6211328125	-4.10734970959147e-11\\
45.642421875	-5.54294078194594e-11\\
45.6637109375	-2.54978489970842e-11\\
45.685	-8.76936509181234e-12\\
45.7062890625	-4.12965687808031e-11\\
45.727578125	-1.48486007411895e-10\\
45.7488671875	-1.2816700017565e-10\\
45.77015625	-1.34520293752426e-10\\
45.7914453125	-1.8375468971749e-10\\
45.812734375	-1.77982941393657e-10\\
45.8340234375	-2.04090209021142e-10\\
45.8553125	-2.18756231519253e-10\\
45.8766015625	-1.89902257242416e-10\\
45.897890625	-2.18804376993779e-10\\
45.9191796875	-1.90372797065334e-10\\
45.94046875	-2.0666433308209e-10\\
45.9617578125	-2.19218469695629e-10\\
45.983046875	-1.689645984553e-10\\
46.0043359375	-2.34365809779885e-10\\
46.025625	-1.70130015881475e-10\\
46.0469140625	-2.34729632770391e-10\\
46.068203125	-1.93913578451632e-10\\
46.0894921875	-2.4798106127558e-10\\
46.11078125	-2.04703056011748e-10\\
46.1320703125	-2.37755649571451e-10\\
46.153359375	-2.40164793223722e-10\\
46.1746484375	-2.26482634461189e-10\\
46.1959375	-2.83270473296104e-10\\
46.2172265625	-1.66701778994503e-10\\
46.238515625	-2.34927706676169e-10\\
46.2598046875	-1.61442814664326e-10\\
46.28109375	-2.38227470974303e-10\\
46.3023828125	-2.71558844910216e-10\\
46.323671875	-2.57512563450225e-10\\
46.3449609375	-2.20164375745581e-10\\
46.36625	-2.55868856020625e-10\\
46.3875390625	-1.4918541648795e-10\\
46.408828125	-1.53170481187185e-10\\
46.4301171875	-1.91238177035142e-10\\
46.45140625	-1.60515817490733e-10\\
46.4726953125	-2.16064620502811e-10\\
46.493984375	-1.84799586442864e-10\\
46.5152734375	-2.49477905387068e-10\\
46.5365625	-1.75426872256769e-10\\
46.5578515625	-2.48522021504494e-10\\
46.579140625	-1.95736312245962e-10\\
46.6004296875	-1.57663349415874e-10\\
46.62171875	-2.26409171010932e-10\\
46.6430078125	-1.53012878014532e-10\\
46.664296875	-1.78700332091571e-10\\
46.6855859375	-2.55128882037008e-10\\
46.706875	-2.51580890145035e-10\\
46.7281640625	-3.45592152156353e-10\\
46.749453125	-2.91575267321658e-10\\
46.7707421875	-3.28443638580461e-10\\
46.79203125	-3.40983100702466e-10\\
46.8133203125	-3.55252073097681e-10\\
46.834609375	-3.28435179668752e-10\\
46.8558984375	-3.43611230812063e-10\\
46.8771875	-3.90498307543257e-10\\
46.8984765625	-3.04729450362546e-10\\
46.919765625	-3.71831468433809e-10\\
46.9410546875	-3.14906826014973e-10\\
46.96234375	-3.83722147812479e-10\\
46.9836328125	-3.67622030169865e-10\\
47.004921875	-3.48175123135226e-10\\
47.0262109375	-2.62387421647817e-10\\
47.0475	-2.87704520041085e-10\\
47.0687890625	-2.79849072356951e-10\\
47.090078125	-2.53592943202691e-10\\
47.1113671875	-2.48976235421811e-10\\
47.13265625	-2.61348193108227e-10\\
47.1539453125	-2.27506801168161e-10\\
47.175234375	-2.37371653884113e-10\\
47.1965234375	-2.4533483974451e-10\\
47.2178125	-1.76691603022769e-10\\
47.2391015625	-2.24352947671766e-10\\
47.260390625	-2.0309137059141e-10\\
47.2816796875	-2.22932872431004e-10\\
47.30296875	-1.9128932209511e-10\\
47.3242578125	-1.56051721689726e-10\\
47.345546875	-9.60686637621614e-11\\
47.3668359375	-1.94117175898276e-10\\
47.388125	-1.79841294664403e-10\\
47.4094140625	-2.10645140208371e-10\\
47.430703125	-2.17437766495418e-10\\
47.4519921875	-2.03971290793268e-10\\
47.47328125	-1.27497279660821e-10\\
47.4945703125	-1.25689196248482e-10\\
47.515859375	-1.66800690430005e-10\\
47.5371484375	-2.04993302722348e-10\\
47.5584375	-1.52297477069618e-10\\
47.5797265625	-2.09342559801917e-10\\
47.601015625	-2.08511979799738e-10\\
47.6223046875	-2.80238141576194e-10\\
47.64359375	-1.52279912085986e-10\\
47.6648828125	-1.64002676580756e-10\\
47.686171875	-1.05773064274242e-10\\
47.7074609375	-8.32327662145007e-11\\
47.72875	-3.09932136091722e-11\\
47.7500390625	-2.18876542001032e-11\\
47.771328125	-2.68167391968544e-11\\
47.7926171875	-6.14521411230485e-11\\
47.81390625	-1.00528557059673e-10\\
47.8351953125	-9.43855924195391e-11\\
47.856484375	-1.09219351845826e-10\\
47.8777734375	-3.68202494835339e-11\\
47.8990625	-1.29081793885371e-10\\
47.9203515625	4.59319158113165e-11\\
47.941640625	4.26619060037908e-11\\
47.9629296875	-1.19928108016847e-11\\
47.98421875	4.2364086409181e-11\\
48.0055078125	-4.6423671312738e-11\\
48.026796875	-6.3346025297926e-11\\
48.0480859375	-2.79353733889074e-13\\
48.069375	-2.16390398675656e-11\\
48.0906640625	-5.06269270998746e-12\\
48.111953125	5.98435563914766e-11\\
48.1332421875	1.47926259916502e-10\\
48.15453125	1.25550066268078e-10\\
48.1758203125	6.84336304482438e-11\\
48.197109375	6.9355726347231e-11\\
48.2183984375	6.83599519227746e-11\\
48.2396875	6.83167660170569e-11\\
48.2609765625	-1.24211578760006e-12\\
48.282265625	-3.64005100267e-11\\
48.3035546875	1.02431260406582e-10\\
48.32484375	8.13845039279955e-11\\
48.3461328125	1.21993149748496e-10\\
48.367421875	1.68121851674469e-10\\
48.3887109375	1.45779594503001e-10\\
48.41	1.38679833819596e-10\\
48.4312890625	1.23140159298158e-10\\
48.452578125	4.17248588943406e-11\\
48.4738671875	1.12387129719919e-10\\
48.49515625	1.69024985960387e-11\\
48.5164453125	8.36742579381939e-11\\
48.537734375	1.04090366072926e-10\\
48.5590234375	1.36559128318998e-10\\
48.5803125	1.84364900480462e-10\\
48.6016015625	1.28557954728763e-10\\
48.622890625	1.97850069033541e-10\\
48.6441796875	2.52893026006406e-10\\
48.66546875	2.57134123722471e-10\\
48.6867578125	2.50349099530084e-10\\
48.708046875	1.01413053235935e-10\\
48.7293359375	1.92620053678699e-10\\
48.750625	1.36418858370538e-10\\
48.7719140625	8.93122843927843e-11\\
48.793203125	2.43680871217418e-10\\
48.8144921875	2.38752138825793e-10\\
48.83578125	2.54096458192264e-10\\
48.8570703125	2.50152453090734e-10\\
48.878359375	3.15548046178141e-10\\
48.8996484375	2.6786002661726e-10\\
48.9209375	2.55859963086497e-10\\
48.9422265625	1.64549426136144e-10\\
48.963515625	2.30907142681751e-10\\
48.9848046875	1.64212160962291e-10\\
49.00609375	1.49620130035252e-10\\
49.0273828125	1.74825113919605e-10\\
49.048671875	1.66226663799521e-10\\
49.0699609375	2.9193652260084e-10\\
49.09125	1.78276692939182e-10\\
49.1125390625	3.4941147891677e-10\\
49.133828125	2.32719575180486e-10\\
49.1551171875	1.9748833410038e-10\\
49.17640625	2.84634053038023e-10\\
49.1976953125	2.21800909608748e-10\\
49.218984375	2.04957652661936e-10\\
49.2402734375	1.79107962660431e-10\\
49.2615625	1.59066007948993e-10\\
49.2828515625	2.74225632567935e-10\\
49.304140625	2.46957036814852e-10\\
49.3254296875	3.10351122450438e-10\\
49.34671875	3.79500183750468e-10\\
49.3680078125	3.2161741446056e-10\\
49.389296875	2.78180461879868e-10\\
49.4105859375	2.38962485808225e-10\\
49.431875	2.40052115660292e-10\\
49.4531640625	2.44411044040738e-10\\
49.474453125	2.38049610343122e-10\\
49.4957421875	2.81046749022433e-10\\
49.51703125	2.74801777658964e-10\\
49.5383203125	3.27385663445488e-10\\
49.559609375	3.02246791927881e-10\\
49.5808984375	3.35506115475972e-10\\
49.6021875	2.51061843251357e-10\\
49.6234765625	3.09277367272402e-10\\
49.644765625	2.53365316352147e-10\\
49.6660546875	1.91444884075003e-10\\
49.68734375	2.67292483486354e-10\\
49.7086328125	1.48025188407762e-10\\
49.729921875	1.60568073531875e-10\\
49.7512109375	1.38380176818836e-10\\
49.7725	1.90123053001622e-10\\
49.7937890625	2.10648576035502e-10\\
49.815078125	1.29383875933727e-10\\
49.8363671875	1.79890106212587e-10\\
49.85765625	1.53341233711639e-10\\
49.8789453125	1.7180238380845e-10\\
49.900234375	8.79240609392944e-11\\
49.9215234375	1.06021159245298e-10\\
49.9428125	1.04311225406721e-10\\
49.9641015625	1.24459634096145e-10\\
49.985390625	5.97089299498318e-11\\
50.0066796875	8.55035333234255e-11\\
50.02796875	1.21298276463263e-10\\
50.0492578125	7.5732678257494e-11\\
50.070546875	2.98608124731032e-11\\
50.0918359375	5.8007765042326e-11\\
50.113125	8.15124668651816e-11\\
50.1344140625	8.14313220191218e-11\\
50.155703125	8.09876363619656e-11\\
50.1769921875	7.74043783504518e-11\\
50.19828125	1.04848872721468e-10\\
50.2195703125	1.17591767185976e-11\\
50.240859375	6.90358592695009e-11\\
50.2621484375	6.52923442353459e-12\\
50.2834375	-1.49408079436833e-11\\
50.3047265625	-3.26643731998282e-11\\
50.326015625	-8.16049296753419e-11\\
50.3473046875	-5.77518456944725e-11\\
50.36859375	-9.8294455233826e-11\\
50.3898828125	-2.0414537544997e-10\\
50.411171875	-9.77745677074983e-11\\
50.4324609375	-1.99462546106917e-10\\
50.45375	-1.84369337703825e-10\\
50.4750390625	-1.97948553531878e-10\\
50.496328125	-1.83442743487062e-10\\
50.5176171875	-1.67679237416117e-10\\
50.53890625	-1.72263558470533e-10\\
50.5601953125	-1.75967394046843e-10\\
50.581484375	-7.84678111027892e-11\\
50.6027734375	-1.72161471787651e-10\\
50.6240625	-1.63066269760809e-10\\
50.6453515625	-2.36001653417905e-10\\
50.666640625	-2.84046229538639e-10\\
50.6879296875	-2.56427032392404e-10\\
50.70921875	-2.64954693714725e-10\\
50.7305078125	-2.28244907041225e-10\\
50.751796875	-2.38019334772541e-10\\
50.7730859375	-1.81451038491566e-10\\
50.794375	-2.39573507263055e-10\\
50.8156640625	-2.61929679246982e-10\\
50.836953125	-1.74266739774705e-10\\
50.8582421875	-2.21195454284719e-10\\
50.87953125	-2.20164027344202e-10\\
50.9008203125	-2.10520751111072e-10\\
50.922109375	-2.58401381160312e-10\\
50.9433984375	-2.7919373674512e-10\\
50.9646875	-2.78253698453798e-10\\
50.9859765625	-2.599452824352e-10\\
51.007265625	-2.7358290468343e-10\\
51.0285546875	-2.60252279710257e-10\\
51.04984375	-3.0994758307116e-10\\
51.0711328125	-2.79681590602945e-10\\
51.092421875	-2.670234979938e-10\\
51.1137109375	-1.88194288377554e-10\\
51.135	-2.39310747335429e-10\\
51.1562890625	-1.9669869209307e-10\\
51.177578125	-3.15088860331032e-10\\
51.1988671875	-2.17292337735983e-10\\
51.22015625	-2.37918097278018e-10\\
51.2414453125	-2.38565334973043e-10\\
51.262734375	-2.53137301471822e-10\\
51.2840234375	-2.66599667265788e-10\\
51.3053125	-2.65790852133498e-10\\
51.3266015625	-2.57895304544086e-10\\
51.347890625	-2.65196602178895e-10\\
51.3691796875	-3.09064333928095e-10\\
51.39046875	-3.52376469771193e-10\\
51.4117578125	-3.60668170976367e-10\\
51.433046875	-3.0894513900902e-10\\
51.4543359375	-3.2387619986069e-10\\
51.475625	-3.44090067681472e-10\\
51.4969140625	-2.49007502348571e-10\\
51.518203125	-2.50956239698999e-10\\
51.5394921875	-2.56806922584487e-10\\
51.56078125	-2.91086756428546e-10\\
51.5820703125	-2.66399405846179e-10\\
51.603359375	-3.16556125948015e-10\\
51.6246484375	-3.07304923501766e-10\\
51.6459375	-2.44697825127154e-10\\
51.6672265625	-2.68006816789647e-10\\
51.688515625	-3.16047752311857e-10\\
51.7098046875	-2.72554981497196e-10\\
51.73109375	-2.95315663360412e-10\\
51.7523828125	-2.72720086735172e-10\\
51.773671875	-2.37132186763865e-10\\
51.7949609375	-2.35149714240115e-10\\
51.81625	-1.86830596802056e-10\\
51.8375390625	-2.18156369653378e-10\\
51.858828125	-1.84972892965779e-10\\
51.8801171875	-1.36533775594845e-10\\
51.90140625	-1.18457847944038e-10\\
51.9226953125	-1.39106852084466e-10\\
51.943984375	-1.76874807281478e-10\\
51.9652734375	-1.88969411930988e-10\\
51.9865625	-1.91547646793971e-10\\
52.0078515625	-1.83446438053925e-10\\
52.029140625	-1.9423620484691e-10\\
52.0504296875	-2.12051981965194e-10\\
52.07171875	-2.18377484146638e-10\\
52.0930078125	-1.96283699814201e-10\\
52.114296875	-1.4679069605998e-10\\
52.1355859375	-1.87694159767384e-10\\
52.156875	-1.01555344105802e-10\\
52.1781640625	-1.33179120936437e-10\\
52.199453125	-1.72080450456383e-10\\
52.2207421875	-1.05529715333513e-10\\
52.24203125	-1.33504140657764e-10\\
52.2633203125	-7.79450823323604e-11\\
52.284609375	-1.35726431256703e-10\\
52.3058984375	-6.74193464394329e-11\\
52.3271875	-1.32103352627948e-10\\
52.3484765625	-1.20757128443627e-10\\
52.369765625	-7.86613506696919e-11\\
52.3910546875	-6.70653017839171e-12\\
52.41234375	-8.61985376231187e-11\\
52.4336328125	-8.16256286209276e-11\\
52.454921875	-4.99619499268788e-11\\
52.4762109375	-7.72216281199818e-11\\
52.4975	-1.21123129042339e-10\\
52.5187890625	-1.51202095126831e-10\\
52.540078125	-6.02357785308662e-11\\
52.5613671875	-1.32895339880103e-10\\
52.58265625	-1.18335511745284e-10\\
52.6039453125	-1.27440790035695e-10\\
52.625234375	-7.85102098067599e-11\\
52.6465234375	-5.70132222450488e-11\\
52.6678125	-1.3024492535736e-10\\
52.6891015625	-1.39036112366362e-10\\
52.710390625	-1.11102400680516e-10\\
52.7316796875	-9.43473083749975e-11\\
52.75296875	-1.08826230792132e-10\\
52.7742578125	-7.97710122191311e-11\\
52.795546875	-4.2428377575374e-11\\
52.8168359375	9.47089886882609e-13\\
52.838125	4.4924146887733e-11\\
52.8594140625	1.19200228412208e-10\\
52.880703125	4.84212713118084e-11\\
52.9019921875	3.25744565476586e-11\\
52.92328125	-1.48859353863952e-12\\
52.9445703125	1.420841367472e-11\\
52.965859375	-4.26305902150627e-12\\
52.9871484375	7.66828712354761e-11\\
53.0084375	3.88496413858881e-11\\
53.0297265625	1.65259354067318e-10\\
53.051015625	1.16995777317922e-10\\
53.0723046875	1.17330236812863e-10\\
53.09359375	1.42280603533921e-10\\
53.1148828125	1.02879078921591e-10\\
53.136171875	6.70078418432896e-11\\
53.1574609375	6.52333537814536e-11\\
53.17875	1.68466333484758e-11\\
53.2000390625	7.66801153544622e-11\\
53.221328125	9.36542028133066e-11\\
53.2426171875	1.28466930848081e-10\\
53.26390625	9.83331952499334e-11\\
53.2851953125	8.13598600377335e-11\\
53.306484375	1.5649509242295e-10\\
53.3277734375	1.20193342081847e-10\\
53.3490625	7.79129190283336e-12\\
53.3703515625	-1.11174721602284e-11\\
53.391640625	-4.20948261501183e-11\\
53.4129296875	4.35544551237123e-12\\
53.43421875	-3.71125668359601e-12\\
53.4555078125	3.14407756106331e-12\\
53.476796875	3.95937466521631e-11\\
53.4980859375	7.18732561148173e-11\\
53.519375	8.40676029928188e-11\\
53.5406640625	3.12061046407164e-11\\
53.561953125	3.55911658701554e-11\\
53.5832421875	2.13842937561875e-11\\
53.60453125	-5.76439608275325e-11\\
53.6258203125	-6.96276330932273e-11\\
53.647109375	-4.97016675377541e-11\\
53.6683984375	3.41805664056753e-12\\
53.6896875	-1.97648262726032e-12\\
53.7109765625	-3.32760329062001e-11\\
53.732265625	3.18079784138847e-11\\
53.7535546875	-5.89924343989572e-12\\
53.77484375	-4.70309899455199e-11\\
53.7961328125	-6.1342971141449e-11\\
53.817421875	-8.5982583972006e-11\\
53.8387109375	-6.35067561328459e-11\\
53.86	1.4381976220122e-11\\
53.8812890625	2.87361263174467e-11\\
53.902578125	1.09516489792638e-10\\
53.9238671875	8.57997695816678e-11\\
53.94515625	9.39927775009447e-11\\
53.9664453125	4.62075876470051e-11\\
53.987734375	1.27705487281712e-10\\
54.0090234375	6.56933515238292e-11\\
54.0303125	8.52487478068955e-11\\
54.0516015625	6.976300210215e-11\\
54.072890625	2.5890849570028e-11\\
54.0941796875	4.97242557178808e-11\\
54.11546875	4.69528078204146e-11\\
54.1367578125	6.03461833308819e-11\\
54.158046875	4.08054563683182e-11\\
54.1793359375	1.05165850332533e-10\\
54.200625	4.74696153459652e-11\\
54.2219140625	3.70962120696454e-12\\
54.243203125	1.22523153369414e-10\\
54.2644921875	3.9350299524326e-11\\
54.28578125	5.55342271594244e-11\\
54.3070703125	3.57749132935661e-11\\
54.328359375	-8.58590369007052e-12\\
54.3496484375	3.66773016322348e-11\\
54.3709375	-8.9623536327923e-13\\
54.3922265625	1.50794212121639e-11\\
54.413515625	3.08673476545425e-11\\
54.4348046875	3.97442710680819e-11\\
54.45609375	2.72127668470473e-11\\
54.4773828125	4.35769858735044e-11\\
54.498671875	5.55927197633464e-11\\
54.5199609375	2.75323333080045e-11\\
54.54125	2.53407275512783e-11\\
54.5625390625	2.39138935207223e-11\\
54.583828125	3.1484542509114e-11\\
54.6051171875	4.63965287771756e-11\\
54.62640625	2.82886401892076e-11\\
54.6476953125	5.80484476347894e-11\\
54.668984375	8.82579702251461e-11\\
54.6902734375	3.92392306633563e-11\\
54.7115625	-4.38913640105162e-12\\
54.7328515625	6.19031702924719e-11\\
54.754140625	9.9354020084183e-12\\
54.7754296875	3.58053474089429e-11\\
54.79671875	3.38859108487732e-11\\
54.8180078125	-4.87249144835964e-11\\
54.839296875	2.33213073039135e-11\\
54.8605859375	-8.89912515033651e-11\\
54.881875	-1.87347828800954e-11\\
54.9031640625	-7.07397374891286e-11\\
54.924453125	-2.59789143478251e-11\\
54.9457421875	-2.4596286129698e-11\\
54.96703125	-7.37168195655457e-11\\
54.9883203125	8.85395102947459e-11\\
55.009609375	1.40236833255344e-11\\
55.0308984375	2.68114439984891e-11\\
55.0521875	3.64359972655425e-11\\
55.0734765625	1.05059110564841e-11\\
55.094765625	1.90303961999225e-11\\
55.1160546875	6.84411542286255e-11\\
55.13734375	4.10576150751371e-11\\
55.1586328125	5.66382623082111e-11\\
55.179921875	2.88408614806518e-11\\
55.2012109375	4.6045484135292e-11\\
55.2225	4.2793621533873e-11\\
55.2437890625	2.17194958059111e-11\\
55.265078125	-3.23450617016479e-11\\
55.2863671875	3.4045877494629e-11\\
55.30765625	-9.30664564399474e-12\\
55.3289453125	-2.22350038269891e-11\\
55.350234375	-5.22825863129782e-11\\
55.3715234375	-1.71937529020843e-11\\
55.3928125	-8.13505887506194e-11\\
55.4141015625	-4.64586332150933e-11\\
55.435390625	-5.01567642444771e-11\\
55.4566796875	-2.74881634614578e-11\\
55.47796875	-4.05189008068012e-11\\
55.4992578125	-1.10284185133706e-10\\
55.520546875	-6.76513437926942e-11\\
55.5418359375	-1.38384177682421e-10\\
55.563125	-1.08488312376556e-10\\
55.5844140625	-1.07642490793638e-10\\
55.605703125	-1.12796146960985e-10\\
55.6269921875	-1.48809509906514e-10\\
55.64828125	-1.01158772285226e-10\\
55.6695703125	-1.45338132926043e-10\\
55.690859375	-2.25845114421461e-11\\
55.7121484375	-1.13244895803736e-10\\
55.7334375	-9.1734352422172e-11\\
55.7547265625	-1.10381187622671e-10\\
55.776015625	-1.02619504130089e-10\\
55.7973046875	-7.24627675088653e-11\\
55.81859375	-1.318055273422e-10\\
55.8398828125	-7.52430125006117e-11\\
55.861171875	-1.23290703940252e-10\\
55.8824609375	-7.25142780454313e-11\\
55.90375	-9.65929600030472e-11\\
55.9250390625	-3.93930858744981e-11\\
55.946328125	-7.08988394543992e-11\\
55.9676171875	-3.11841591482985e-11\\
55.98890625	-2.36416452071595e-11\\
56.0101953125	-3.77958634649023e-11\\
56.031484375	1.55470169360926e-11\\
56.0527734375	-8.05494394979234e-11\\
56.0740625	-1.47225446953159e-11\\
56.0953515625	-2.38842332676461e-11\\
56.116640625	1.42347947374852e-11\\
56.1379296875	-2.08452962165995e-11\\
56.15921875	-6.89710770795322e-12\\
56.1805078125	1.18698528221413e-11\\
56.201796875	2.57433684852129e-11\\
56.2230859375	9.35308157101708e-12\\
56.244375	-1.96596401346099e-12\\
56.2656640625	5.4199076336378e-11\\
56.286953125	1.93129764619557e-11\\
56.3082421875	3.28858045285789e-11\\
56.32953125	4.16507507226336e-11\\
56.3508203125	6.70334245725394e-11\\
56.372109375	-1.39373160436613e-11\\
56.3933984375	1.85171568119809e-12\\
56.4146875	1.5303969674034e-12\\
56.4359765625	2.35064356860054e-11\\
56.457265625	-1.90151523039946e-11\\
56.4785546875	-2.76505865989347e-11\\
56.49984375	-5.46609082796067e-11\\
56.5211328125	-5.56818240662938e-12\\
56.542421875	-1.65247488480773e-11\\
56.5637109375	-6.18056422592052e-11\\
56.585	-5.90358449675473e-11\\
56.6062890625	4.26082339526161e-12\\
56.627578125	3.15442186494099e-11\\
56.6488671875	-2.9524890434523e-11\\
56.67015625	1.61477753962055e-11\\
56.6914453125	1.11191437006492e-11\\
56.712734375	-3.92723870928728e-11\\
56.7340234375	-1.54718576948258e-11\\
56.7553125	-5.8254367775279e-11\\
56.7766015625	-8.25980869384452e-12\\
56.797890625	-6.31921630233898e-11\\
};
\addlegendentry{$\text{train 4 -\textgreater{} Trondheim}$};

\addplot [color=mycolor3,solid,forget plot]
  table[row sep=crcr]{%
-42.122044921875	-6.36711615647638e-10\\
-42.10205078125	-6.94657200635323e-10\\
-42.082056640625	-7.20491132803054e-10\\
-42.0620625	-7.0630087107732e-10\\
-42.042068359375	-7.83475183126038e-10\\
-42.02207421875	-7.28434238760549e-10\\
-42.002080078125	-7.94166577154994e-10\\
-41.9820859375	-7.71298334022731e-10\\
-41.962091796875	-8.21755087169326e-10\\
-41.94209765625	-8.8442288518353e-10\\
-41.922103515625	-9.48851713001819e-10\\
-41.902109375	-1.03313144740452e-09\\
-41.882115234375	-9.9092612001472e-10\\
-41.86212109375	-1.01436263796855e-09\\
-41.842126953125	-9.56753795601444e-10\\
-41.8221328125	-9.95512405418778e-10\\
-41.802138671875	-9.5007243766237e-10\\
-41.78214453125	-9.05966089858972e-10\\
-41.762150390625	-8.87240128310315e-10\\
-41.74215625	-8.15867557969108e-10\\
-41.722162109375	-7.62018375407651e-10\\
-41.70216796875	-8.3991060248494e-10\\
-41.682173828125	-8.05749886675556e-10\\
-41.6621796875	-8.4720355262605e-10\\
-41.642185546875	-8.96766636609685e-10\\
-41.62219140625	-8.41276264496886e-10\\
-41.602197265625	-9.24204467876266e-10\\
-41.582203125	-9.06595383810749e-10\\
-41.562208984375	-9.25131042064211e-10\\
-41.54221484375	-9.30899587609836e-10\\
-41.522220703125	-8.41160904101124e-10\\
-41.5022265625	-8.23975948374616e-10\\
-41.482232421875	-8.43204760138811e-10\\
-41.46223828125	-8.19386621144996e-10\\
-41.442244140625	-7.65363027028711e-10\\
-41.42225	-7.82557326390553e-10\\
-41.402255859375	-7.82418322618448e-10\\
-41.38226171875	-7.57148628666961e-10\\
-41.362267578125	-8.03113673220812e-10\\
-41.3422734375	-8.16470452936008e-10\\
-41.322279296875	-7.98240914392697e-10\\
-41.30228515625	-8.49023721488167e-10\\
-41.282291015625	-7.92136724910609e-10\\
-41.262296875	-7.90766229107266e-10\\
-41.242302734375	-7.58145829861446e-10\\
-41.22230859375	-6.96472422372365e-10\\
-41.202314453125	-6.46745295093059e-10\\
-41.1823203125	-7.07042489504156e-10\\
-41.162326171875	-6.88584933855046e-10\\
-41.14233203125	-7.31672779038377e-10\\
-41.122337890625	-7.8185979296303e-10\\
-41.10234375	-7.45914603611069e-10\\
-41.082349609375	-7.31033899051653e-10\\
-41.06235546875	-7.69631412824923e-10\\
-41.042361328125	-7.02577596248465e-10\\
-41.0223671875	-5.7418036024572e-10\\
-41.002373046875	-4.95507066296161e-10\\
-40.98237890625	-4.80933372310137e-10\\
-40.962384765625	-4.25552044292854e-10\\
-40.942390625	-3.98697285150774e-10\\
-40.922396484375	-5.08695097176722e-10\\
-40.90240234375	-5.35800048611741e-10\\
-40.882408203125	-5.18277313436753e-10\\
-40.8624140625	-6.23933434180161e-10\\
-40.842419921875	-5.84677936756177e-10\\
-40.82242578125	-4.81355376513295e-10\\
-40.802431640625	-4.56427772366349e-10\\
-40.7824375	-4.76599198040946e-10\\
-40.762443359375	-3.56716219608645e-10\\
-40.74244921875	-3.37434202565744e-10\\
-40.722455078125	-4.1475764834427e-10\\
-40.7024609375	-3.80657443647227e-10\\
-40.682466796875	-5.03862607136507e-10\\
-40.66247265625	-4.92855506373994e-10\\
-40.642478515625	-4.94902357053211e-10\\
-40.622484375	-4.67417132133816e-10\\
-40.602490234375	-4.32779371806154e-10\\
-40.58249609375	-3.61276295219484e-10\\
-40.562501953125	-4.16658024003546e-10\\
-40.5425078125	-2.8259088877512e-10\\
-40.522513671875	-2.6144866229099e-10\\
-40.50251953125	-3.44919510076703e-10\\
-40.482525390625	-2.90787470387237e-10\\
-40.46253125	-3.53723218657425e-10\\
-40.442537109375	-3.90553970575117e-10\\
-40.42254296875	-4.41310144748519e-10\\
-40.402548828125	-4.01015937985221e-10\\
-40.3825546875	-3.59516755777636e-10\\
-40.362560546875	-3.07927456011718e-10\\
-40.34256640625	-2.87167202557025e-10\\
-40.322572265625	-1.91903668345051e-10\\
-40.302578125	-1.73513963430295e-10\\
-40.282583984375	-2.69446389379091e-10\\
-40.26258984375	-2.52072048745655e-10\\
-40.242595703125	-2.3045510522733e-10\\
-40.2226015625	-3.17218089766929e-10\\
-40.202607421875	-3.20541223589898e-10\\
-40.18261328125	-3.80736055686952e-10\\
-40.162619140625	-2.78646662522191e-10\\
-40.142625	-3.38452453130159e-10\\
-40.122630859375	-2.99995188965512e-10\\
-40.10263671875	-2.93953745253606e-10\\
-40.082642578125	-2.54186212488497e-10\\
-40.0626484375	-3.01751413734835e-10\\
-40.042654296875	-2.34204879313374e-10\\
-40.02266015625	-2.77542852326181e-10\\
-40.002666015625	-2.81413191379824e-10\\
-39.982671875	-2.96825100610439e-10\\
-39.962677734375	-3.48621457728537e-10\\
-39.94268359375	-3.619817841221e-10\\
-39.922689453125	-3.95431763464876e-10\\
-39.9026953125	-3.97230524121624e-10\\
-39.882701171875	-3.73260857688856e-10\\
-39.86270703125	-3.07580573132698e-10\\
-39.842712890625	-2.81626615585108e-10\\
-39.82271875	-3.26422802773755e-10\\
-39.802724609375	-2.85054503232998e-10\\
-39.78273046875	-2.92205290483529e-10\\
-39.762736328125	-3.23627529913731e-10\\
-39.7427421875	-2.31840653959845e-10\\
-39.722748046875	-2.45901769965464e-10\\
-39.70275390625	-1.27759233541803e-10\\
-39.682759765625	-1.47878008364698e-10\\
-39.662765625	-1.86216959511758e-10\\
-39.642771484375	-3.26025262171294e-11\\
-39.62277734375	2.04084539821345e-10\\
-39.602783203125	6.88061995054516e-11\\
-39.5827890625	3.47170801665962e-10\\
-39.562794921875	4.65268137283909e-10\\
-39.54280078125	4.87566845504653e-10\\
-39.522806640625	5.40556275605837e-10\\
-39.5028125	6.06153535411739e-10\\
-39.482818359375	5.44659115994708e-10\\
-39.46282421875	5.91387665094854e-10\\
-39.442830078125	6.49812912393328e-10\\
-39.4228359375	7.34482758216857e-10\\
-39.402841796875	7.35284455150824e-10\\
-39.38284765625	7.4456594154429e-10\\
-39.362853515625	7.92760693925352e-10\\
-39.342859375	7.66419819629553e-10\\
-39.322865234375	7.42421039873288e-10\\
-39.30287109375	7.09890215124637e-10\\
-39.282876953125	6.984290340155e-10\\
-39.2628828125	6.41886083860263e-10\\
-39.242888671875	6.89947946159618e-10\\
-39.22289453125	6.47724194220416e-10\\
-39.202900390625	6.75532128322439e-10\\
-39.18290625	6.37392885047407e-10\\
-39.162912109375	6.14043001044907e-10\\
-39.14291796875	6.18744749306428e-10\\
-39.122923828125	5.41614326845042e-10\\
-39.1029296875	5.27786449629539e-10\\
-39.082935546875	5.37424010155363e-10\\
-39.06294140625	5.04594826456211e-10\\
-39.042947265625	4.69184463389609e-10\\
-39.022953125	4.80911941518879e-10\\
-39.002958984375	5.41770748056919e-10\\
-38.98296484375	4.98922254349057e-10\\
-38.962970703125	5.70299829756446e-10\\
-38.9429765625	4.97379033736935e-10\\
-38.922982421875	6.65336930059731e-10\\
-38.90298828125	5.47812553582003e-10\\
-38.882994140625	3.83362380709407e-10\\
-38.863	4.24622018691609e-10\\
-38.843005859375	3.57187367144976e-10\\
-38.82301171875	2.88997737902264e-10\\
-38.803017578125	4.31098262081462e-10\\
-38.7830234375	3.98574260433607e-10\\
-38.763029296875	4.83524165946537e-10\\
-38.74303515625	4.7740963916441e-10\\
-38.723041015625	6.05874220819413e-10\\
-38.703046875	5.73997309416435e-10\\
-38.683052734375	5.85764192030221e-10\\
-38.66305859375	5.10501839106894e-10\\
-38.643064453125	5.26190522620959e-10\\
-38.6230703125	5.38011039630653e-10\\
-38.603076171875	5.33814360935282e-10\\
-38.58308203125	5.58224306598678e-10\\
-38.563087890625	6.80930546824679e-10\\
-38.54309375	5.56582439988806e-10\\
-38.523099609375	6.24449420957214e-10\\
-38.50310546875	5.55095020187697e-10\\
-38.483111328125	5.51665266963308e-10\\
-38.4631171875	4.73856569134245e-10\\
-38.443123046875	4.25430485602361e-10\\
-38.42312890625	4.32126055764189e-10\\
-38.403134765625	2.97692677928374e-10\\
-38.383140625	3.98998522797533e-10\\
-38.363146484375	3.78527299967212e-10\\
-38.34315234375	3.75193289709849e-10\\
-38.323158203125	3.98642389321404e-10\\
-38.3031640625	4.52316653011879e-10\\
-38.283169921875	4.38466638635457e-10\\
-38.26317578125	3.33947724359413e-10\\
-38.243181640625	3.53038318202656e-10\\
-38.2231875	1.90378507824888e-10\\
-38.203193359375	1.39426344180589e-10\\
-38.18319921875	1.54114098469398e-10\\
-38.163205078125	1.11940233904629e-10\\
-38.1432109375	1.68455951437599e-10\\
-38.123216796875	1.89125552197029e-10\\
-38.10322265625	1.75683295516135e-10\\
-38.083228515625	1.63866840496013e-10\\
-38.063234375	1.64589846115148e-10\\
-38.043240234375	1.24361929277433e-10\\
-38.02324609375	1.02650098964767e-10\\
-38.003251953125	2.08608053891449e-11\\
-37.9832578125	2.41293902096759e-11\\
-37.963263671875	-2.06294710203101e-11\\
-37.94326953125	3.27281994880827e-11\\
-37.923275390625	6.57721486577038e-11\\
-37.90328125	6.48200187220821e-11\\
-37.883287109375	8.46739428212078e-11\\
-37.86329296875	1.97841979171167e-11\\
-37.843298828125	5.54522945011197e-12\\
-37.8233046875	-2.90763253842257e-11\\
-37.803310546875	-1.04912100526681e-10\\
-37.78331640625	-1.23191697780851e-10\\
-37.763322265625	-1.53851466749229e-10\\
-37.743328125	-1.50963895041792e-10\\
-37.723333984375	-5.85913207653046e-11\\
-37.70333984375	-8.56935121485696e-11\\
-37.683345703125	7.14087278641389e-11\\
-37.6633515625	1.39163967163659e-10\\
-37.643357421875	1.2894618990868e-10\\
-37.62336328125	7.62372250279763e-11\\
-37.603369140625	5.06456571334916e-11\\
-37.583375	-3.89942193254899e-11\\
-37.563380859375	-3.26472121184995e-11\\
-37.54338671875	-1.27864160449981e-10\\
-37.523392578125	-9.11470021442527e-11\\
-37.5033984375	-1.31060435134998e-10\\
-37.483404296875	-1.6294323085666e-10\\
-37.46341015625	-5.96017362449199e-11\\
-37.443416015625	-3.46178570804556e-11\\
-37.423421875	3.53392936783062e-11\\
-37.403427734375	6.55376038142219e-11\\
-37.38343359375	7.3609619006351e-11\\
-37.363439453125	3.74879152971326e-11\\
-37.3434453125	3.21755415230661e-11\\
-37.323451171875	4.76704045311935e-12\\
-37.30345703125	-4.76133861181266e-12\\
-37.283462890625	-2.70995437133826e-12\\
-37.26346875	-5.92343024251348e-11\\
-37.243474609375	-1.69878076175132e-10\\
-37.22348046875	-2.41003370667842e-11\\
-37.203486328125	-1.30634460312452e-10\\
-37.1834921875	-2.01635384550615e-11\\
-37.163498046875	-5.46850399869658e-11\\
-37.14350390625	-1.15514091391284e-10\\
-37.123509765625	-1.59093484988544e-10\\
-37.103515625	-2.0178755891662e-10\\
-37.083521484375	-4.34109285405005e-10\\
-37.06352734375	-4.61029337187082e-10\\
-37.043533203125	-5.851423555645e-10\\
-37.0235390625	-6.74185148990892e-10\\
-37.003544921875	-7.93009444301687e-10\\
-36.98355078125	-7.3327410157888e-10\\
-36.963556640625	-7.4054088936749e-10\\
-36.9435625	-8.49590953779038e-10\\
-36.923568359375	-7.69683093718468e-10\\
-36.90357421875	-9.38536671514739e-10\\
-36.883580078125	-9.74555843805639e-10\\
-36.8635859375	-1.03513573071809e-09\\
-36.843591796875	-1.006808528567e-09\\
-36.82359765625	-1.00089520168648e-09\\
-36.803603515625	-1.00636677653681e-09\\
-36.783609375	-1.01667115937808e-09\\
-36.763615234375	-8.72979833872775e-10\\
-36.74362109375	-9.01889249105896e-10\\
-36.723626953125	-7.55739856197368e-10\\
-36.7036328125	-8.22786449545075e-10\\
-36.683638671875	-7.71952939728002e-10\\
-36.66364453125	-8.19772774456398e-10\\
-36.643650390625	-7.11180062367246e-10\\
-36.62365625	-6.88832063317933e-10\\
-36.603662109375	-7.34575864684741e-10\\
-36.58366796875	-7.24354268009383e-10\\
-36.563673828125	-6.46433882076658e-10\\
-36.5436796875	-6.56556870286679e-10\\
-36.523685546875	-5.95415936774408e-10\\
-36.50369140625	-5.02425801634152e-10\\
-36.483697265625	-5.20792909762858e-10\\
-36.463703125	-5.46098696152927e-10\\
-36.443708984375	-4.80205453991569e-10\\
-36.42371484375	-6.33583349096295e-10\\
-36.403720703125	-5.13852066154055e-10\\
-36.3837265625	-6.28407662413542e-10\\
-36.363732421875	-6.82977755930444e-10\\
-36.34373828125	-6.31702020491335e-10\\
-36.323744140625	-6.01420432668859e-10\\
-36.30375	-6.95728976209286e-10\\
-36.283755859375	-5.2070592179746e-10\\
-36.26376171875	-6.34325832952304e-10\\
-36.243767578125	-5.02363458787421e-10\\
-36.2237734375	-6.68064999521782e-10\\
-36.203779296875	-6.32569920475819e-10\\
-36.18378515625	-7.00047576824564e-10\\
-36.163791015625	-6.68183041832438e-10\\
-36.143796875	-6.99660274511271e-10\\
-36.123802734375	-7.97418155752885e-10\\
-36.10380859375	-7.36160833960996e-10\\
-36.083814453125	-8.25212686812988e-10\\
-36.0638203125	-8.22162349406562e-10\\
-36.043826171875	-8.21615650208775e-10\\
-36.02383203125	-9.8416794491531e-10\\
-36.003837890625	-7.98656753136069e-10\\
-35.98384375	-9.63710157909908e-10\\
-35.963849609375	-8.6492386025324e-10\\
-35.94385546875	-7.93051222139389e-10\\
-35.923861328125	-8.01071593753657e-10\\
-35.9038671875	-8.28035433118467e-10\\
-35.883873046875	-7.49296375803986e-10\\
-35.86387890625	-6.89815319130866e-10\\
-35.843884765625	-6.78143989091015e-10\\
-35.823890625	-8.06952963662984e-10\\
-35.803896484375	-7.78445617463122e-10\\
-35.78390234375	-7.49770413209305e-10\\
-35.763908203125	-7.9163984215761e-10\\
-35.7439140625	-8.39923165997773e-10\\
-35.723919921875	-7.68323346476437e-10\\
-35.70392578125	-7.36111551636572e-10\\
-35.683931640625	-6.81536653902034e-10\\
-35.6639375	-6.57445805653767e-10\\
-35.643943359375	-6.07330806842391e-10\\
-35.62394921875	-5.45623973336933e-10\\
-35.603955078125	-5.5352236894806e-10\\
-35.5839609375	-5.14709205869166e-10\\
-35.563966796875	-4.3647340854112e-10\\
-35.54397265625	-4.78700505473577e-10\\
-35.523978515625	-4.58440000609489e-10\\
-35.503984375	-3.46065784647124e-10\\
-35.483990234375	-3.88565819104747e-10\\
-35.46399609375	-2.37444257897843e-10\\
-35.444001953125	-3.5186147430318e-10\\
-35.4240078125	-2.43157857336085e-10\\
-35.404013671875	-3.20935623634777e-10\\
-35.38401953125	-3.06769886991137e-10\\
-35.364025390625	-3.33137226954431e-10\\
-35.34403125	-3.36046813174787e-10\\
-35.324037109375	-2.76002161857847e-10\\
-35.30404296875	-2.22879401630464e-10\\
-35.284048828125	-2.08916761106225e-10\\
-35.2640546875	-1.93954701332309e-10\\
-35.244060546875	-1.23013587034024e-10\\
-35.22406640625	-9.10907126217708e-11\\
-35.204072265625	-7.84566289251771e-11\\
-35.184078125	-1.19103252411614e-10\\
-35.164083984375	-1.74438965246565e-10\\
-35.14408984375	-2.26757624864483e-10\\
-35.124095703125	-3.1995881494428e-10\\
-35.1041015625	-3.08231114312664e-10\\
-35.084107421875	-2.82541903959407e-10\\
-35.06411328125	-2.83603298825912e-10\\
-35.044119140625	-2.52365556486449e-10\\
-35.024125	-2.03058382856045e-10\\
-35.004130859375	-1.12175999325422e-10\\
-34.98413671875	-1.14472437201618e-10\\
-34.964142578125	-5.13305563671487e-11\\
-34.9441484375	-3.50946261433615e-11\\
-34.924154296875	2.26899623323423e-11\\
-34.90416015625	-5.44786470526566e-11\\
-34.884166015625	-1.10112932904195e-10\\
-34.864171875	-6.12461101687621e-11\\
-34.844177734375	-5.89144189903171e-11\\
-34.82418359375	-1.59210598302412e-10\\
-34.804189453125	-5.22636733456315e-11\\
-34.7841953125	-1.31404174562562e-10\\
-34.764201171875	-4.49641815502866e-11\\
-34.74420703125	-8.46883101724563e-12\\
-34.724212890625	5.20141726217625e-12\\
-34.70421875	1.48551021776595e-10\\
-34.684224609375	2.03492493147817e-10\\
-34.66423046875	2.6979607365974e-10\\
-34.644236328125	2.42470190740663e-10\\
-34.6242421875	3.4492311898939e-10\\
-34.604248046875	3.69594843702078e-10\\
-34.58425390625	3.78620541127883e-10\\
-34.564259765625	4.64785147730273e-10\\
-34.544265625	6.29183681380483e-10\\
-34.524271484375	7.08905311555129e-10\\
-34.50427734375	8.16002515128932e-10\\
-34.484283203125	8.46664867321784e-10\\
-34.4642890625	9.05443260272492e-10\\
-34.444294921875	9.59739807180887e-10\\
-34.42430078125	9.14335718973699e-10\\
-34.404306640625	1.09414724767285e-09\\
-34.3843125	9.71234844128619e-10\\
-34.364318359375	1.12314161217242e-09\\
-34.34432421875	1.17779862876251e-09\\
-34.324330078125	1.24528821568577e-09\\
-34.3043359375	1.27025718647978e-09\\
-34.284341796875	1.36592244133412e-09\\
-34.26434765625	1.30492098781711e-09\\
-34.244353515625	1.2938323833345e-09\\
-34.224359375	1.17215919593924e-09\\
-34.204365234375	1.16889182261377e-09\\
-34.18437109375	1.07300407235337e-09\\
-34.164376953125	1.03870820174055e-09\\
-34.1443828125	9.88970984730512e-10\\
-34.124388671875	1.02524846732479e-09\\
-34.10439453125	1.04520297882293e-09\\
-34.084400390625	9.51056107354644e-10\\
-34.06440625	1.04552709612731e-09\\
-34.044412109375	1.02080635639286e-09\\
-34.02441796875	1.04251782500785e-09\\
-34.004423828125	1.00690799588333e-09\\
-33.9844296875	9.67492699937697e-10\\
-33.964435546875	9.17095232004432e-10\\
-33.94444140625	8.45613670771432e-10\\
-33.924447265625	7.95514512044267e-10\\
-33.904453125	7.73372728081515e-10\\
-33.884458984375	8.39566350722159e-10\\
-33.86446484375	7.93532336162259e-10\\
-33.844470703125	8.52428377446333e-10\\
-33.8244765625	8.68277000692583e-10\\
-33.804482421875	8.86301219232487e-10\\
-33.78448828125	9.76043813669927e-10\\
-33.764494140625	1.00065225741374e-09\\
-33.7445	1.01997583885424e-09\\
-33.724505859375	1.1624126935027e-09\\
-33.70451171875	1.0971193844787e-09\\
-33.684517578125	1.13787896142892e-09\\
-33.6645234375	1.10429130209653e-09\\
-33.644529296875	1.09110156408946e-09\\
-33.62453515625	1.11719104519424e-09\\
-33.604541015625	9.83643727962862e-10\\
-33.584546875	1.04179915912629e-09\\
-33.564552734375	1.0515122634121e-09\\
-33.54455859375	1.13042974754211e-09\\
-33.524564453125	1.24525599990282e-09\\
-33.5045703125	1.20891549662815e-09\\
-33.484576171875	1.4170538140511e-09\\
-33.46458203125	1.29609563000194e-09\\
-33.444587890625	1.35791621271542e-09\\
-33.42459375	1.29633072554974e-09\\
-33.404599609375	1.34208640653426e-09\\
-33.38460546875	1.22583648986162e-09\\
-33.364611328125	1.28670284378368e-09\\
-33.3446171875	1.21240293042747e-09\\
-33.324623046875	1.21827747842329e-09\\
-33.30462890625	1.16970186502973e-09\\
-33.284634765625	1.2552608702948e-09\\
-33.264640625	1.19607942271924e-09\\
-33.244646484375	1.26021954700386e-09\\
-33.22465234375	1.19877916897132e-09\\
-33.204658203125	1.13139257048453e-09\\
-33.1846640625	1.1892101952979e-09\\
-33.164669921875	1.18452672878391e-09\\
-33.14467578125	1.15090901038289e-09\\
-33.124681640625	1.16968707540058e-09\\
-33.1046875	1.1754379075756e-09\\
-33.084693359375	1.17389014264565e-09\\
-33.06469921875	1.02319640161582e-09\\
-33.044705078125	1.11877298183936e-09\\
-33.0247109375	9.31208243725754e-10\\
-33.004716796875	8.85188901752744e-10\\
-32.98472265625	8.47928015525068e-10\\
-32.964728515625	7.53121208208723e-10\\
-32.944734375	8.03082350942345e-10\\
-32.924740234375	7.65937223380648e-10\\
-32.90474609375	7.97109891916937e-10\\
-32.884751953125	8.20615875608704e-10\\
-32.8647578125	9.08890712667531e-10\\
-32.844763671875	8.97598667859933e-10\\
-32.82476953125	8.8348510757992e-10\\
-32.804775390625	8.93117339396433e-10\\
-32.78478125	7.61641766285244e-10\\
-32.764787109375	7.51721452254994e-10\\
-32.74479296875	7.12420519826303e-10\\
-32.724798828125	6.18105667418462e-10\\
-32.7048046875	6.35719138864738e-10\\
-32.684810546875	6.43466026563616e-10\\
-32.66481640625	7.21088094980432e-10\\
-32.644822265625	7.79646288767402e-10\\
-32.624828125	8.58157399347707e-10\\
-32.604833984375	8.03751752508245e-10\\
-32.58483984375	9.00101274285527e-10\\
-32.564845703125	8.21090972323519e-10\\
-32.5448515625	8.70223366845735e-10\\
-32.524857421875	8.22077867562575e-10\\
-32.50486328125	8.49248096323022e-10\\
-32.484869140625	7.70815883529817e-10\\
-32.464875	7.83552266025623e-10\\
-32.444880859375	7.8897282782918e-10\\
-32.42488671875	7.18412762405667e-10\\
-32.404892578125	7.43733605291186e-10\\
-32.3848984375	7.0494077022902e-10\\
-32.364904296875	6.81358885600887e-10\\
-32.34491015625	6.94386059563486e-10\\
-32.324916015625	7.45840908533203e-10\\
-32.304921875	7.72908198670756e-10\\
-32.284927734375	8.60595478383634e-10\\
-32.26493359375	8.13870444834446e-10\\
-32.244939453125	8.1349446626648e-10\\
-32.2249453125	8.37282632737346e-10\\
-32.204951171875	7.41361821505677e-10\\
-32.18495703125	7.62137660454012e-10\\
-32.164962890625	6.30170092037385e-10\\
-32.14496875	5.99630623624606e-10\\
-32.124974609375	5.11591867076227e-10\\
-32.10498046875	4.17747911222808e-10\\
-32.084986328125	4.26930509278182e-10\\
-32.0649921875	2.935786806625e-10\\
-32.044998046875	3.42042612432723e-10\\
-32.02500390625	2.29516365471776e-10\\
-32.005009765625	7.98021518891244e-11\\
-31.985015625	1.21968588323325e-11\\
-31.965021484375	-7.40710644294213e-11\\
-31.94502734375	-1.36098419120869e-10\\
-31.925033203125	-1.38104513990578e-10\\
-31.9050390625	-1.82569218690943e-10\\
-31.885044921875	-1.33445544510016e-10\\
-31.86505078125	-3.0661006291902e-10\\
-31.845056640625	-1.46771885016297e-10\\
-31.8250625	-2.4771798541782e-10\\
-31.805068359375	-2.36378217306388e-10\\
-31.78507421875	-4.06999547469168e-10\\
-31.765080078125	-3.00408853455352e-10\\
-31.7450859375	-4.37745075324565e-10\\
-31.725091796875	-3.59529101770558e-10\\
-31.70509765625	-3.38128683634116e-10\\
-31.685103515625	-2.1497607810208e-10\\
-31.665109375	-2.48321311221288e-10\\
-31.645115234375	-1.80870979047837e-10\\
-31.62512109375	-5.8450859928208e-11\\
-31.605126953125	-3.92552083786562e-11\\
-31.5851328125	-5.00224164614575e-11\\
-31.565138671875	-4.93365219561847e-11\\
-31.54514453125	-1.46133395267574e-11\\
-31.525150390625	2.29079758146365e-11\\
-31.50515625	2.08377495441623e-11\\
-31.485162109375	2.24501271673569e-11\\
-31.46516796875	3.07419696690002e-11\\
-31.445173828125	2.97720489132681e-11\\
-31.4251796875	1.18494178769486e-10\\
-31.405185546875	1.25811136437289e-10\\
-31.38519140625	1.6459215894239e-10\\
-31.365197265625	1.96087466622547e-10\\
-31.345203125	1.67180530985987e-10\\
-31.325208984375	1.17053974678486e-10\\
-31.30521484375	1.19965136309802e-10\\
-31.285220703125	-8.74302640085624e-12\\
-31.2652265625	9.56932756206682e-11\\
-31.245232421875	-1.29976806167644e-12\\
-31.22523828125	-1.12358978212291e-10\\
-31.205244140625	-4.29240027488772e-11\\
-31.18525	-1.51235582211152e-10\\
-31.165255859375	-1.9293049846883e-10\\
-31.14526171875	-2.27509010726189e-10\\
-31.125267578125	-2.25395568602621e-10\\
-31.1052734375	-1.9223764536951e-10\\
-31.085279296875	-2.53909418320593e-10\\
-31.06528515625	-1.89815557221123e-10\\
-31.045291015625	-2.06153549805397e-10\\
-31.025296875	-3.22955342324114e-10\\
-31.005302734375	-2.90104053727068e-10\\
-30.98530859375	-3.98524239597648e-10\\
-30.965314453125	-4.48403485546921e-10\\
-30.9453203125	-5.52441254877496e-10\\
-30.925326171875	-4.74221767311241e-10\\
-30.90533203125	-4.87855631147499e-10\\
-30.885337890625	-4.19175433000088e-10\\
-30.86534375	-4.25093537276212e-10\\
-30.845349609375	-3.42160707149124e-10\\
-30.82535546875	-3.10582952453457e-10\\
-30.805361328125	-2.79502187571012e-10\\
-30.7853671875	-3.04784999923151e-10\\
-30.765373046875	-2.3133188222433e-10\\
-30.74537890625	-3.46942769031531e-10\\
-30.725384765625	-2.53995236952152e-10\\
-30.705390625	-3.11912160772023e-10\\
-30.685396484375	-2.53524767713174e-10\\
-30.66540234375	-1.83328186104355e-10\\
-30.645408203125	-1.37808675173704e-10\\
-30.6254140625	-1.61085001335592e-10\\
-30.605419921875	-1.51952285436096e-10\\
-30.58542578125	-1.79196299176027e-10\\
-30.565431640625	-1.29417981164011e-10\\
-30.5454375	-2.50170982577508e-10\\
-30.525443359375	-1.027747320028e-10\\
-30.50544921875	-1.43832106608575e-10\\
-30.485455078125	1.3729450994281e-12\\
-30.4654609375	1.2468331338343e-10\\
-30.445466796875	1.19572868346877e-10\\
-30.42547265625	1.67395838938434e-10\\
-30.405478515625	1.32261031768263e-10\\
-30.385484375	1.81637732380521e-10\\
-30.365490234375	1.38751087565644e-10\\
-30.34549609375	1.61335816743787e-10\\
-30.325501953125	1.08280090142531e-10\\
-30.3055078125	7.31319155017134e-11\\
-30.285513671875	1.5658470148513e-10\\
-30.26551953125	1.19994571193711e-10\\
-30.245525390625	1.90709392238417e-10\\
-30.22553125	2.21562266886704e-10\\
-30.205537109375	1.71881471568341e-10\\
-30.18554296875	2.27665575326279e-10\\
-30.165548828125	1.76565315056941e-10\\
-30.1455546875	2.09445994142043e-10\\
-30.125560546875	2.09984230718637e-10\\
-30.10556640625	1.78141816841015e-10\\
-30.085572265625	1.15727151498714e-10\\
-30.065578125	1.60992098745126e-10\\
-30.045583984375	9.26746327147188e-11\\
-30.02558984375	1.28911663435938e-10\\
-30.005595703125	5.40869242955399e-11\\
-29.9856015625	6.69115966791108e-11\\
-29.965607421875	-9.64061492386887e-12\\
-29.94561328125	1.27406326712033e-10\\
-29.925619140625	9.43975514523342e-11\\
-29.905625	9.69220061441549e-11\\
-29.885630859375	1.17349401847917e-10\\
-29.86563671875	1.6608340535419e-10\\
-29.845642578125	1.56197406951191e-10\\
-29.8256484375	1.67346141776787e-10\\
-29.805654296875	8.46857360185627e-11\\
-29.78566015625	6.36993296034095e-11\\
-29.765666015625	7.41901347356865e-11\\
-29.745671875	1.06554207796522e-10\\
-29.725677734375	1.08707262921851e-10\\
-29.70568359375	-1.04908327564295e-11\\
-29.685689453125	6.31145109936836e-11\\
-29.6656953125	4.93625746592072e-11\\
-29.645701171875	1.38354447447961e-10\\
-29.62570703125	1.8206102096181e-10\\
-29.605712890625	1.28091836655507e-10\\
-29.58571875	1.11657327678603e-10\\
-29.565724609375	1.18764509739492e-10\\
-29.54573046875	1.02075556497631e-10\\
-29.525736328125	1.38742730222867e-10\\
-29.5057421875	2.26857945315048e-10\\
-29.485748046875	1.85287144866315e-10\\
-29.46575390625	3.01918297928656e-10\\
-29.445759765625	3.33987603452128e-10\\
-29.425765625	3.56511141057519e-10\\
-29.405771484375	4.0725121168467e-10\\
-29.38577734375	3.58102141026897e-10\\
-29.365783203125	4.8144289436914e-10\\
-29.3457890625	3.54019480834865e-10\\
-29.325794921875	4.92234333478775e-10\\
-29.30580078125	4.52032909011299e-10\\
-29.285806640625	5.05387258290185e-10\\
-29.2658125	4.02658712527421e-10\\
-29.245818359375	5.18971775195941e-10\\
-29.22582421875	4.66560983664766e-10\\
-29.205830078125	4.84279402041919e-10\\
-29.1858359375	3.98067646878872e-10\\
-29.165841796875	3.77833648393142e-10\\
-29.14584765625	2.71439236289425e-10\\
-29.125853515625	2.43004783493735e-10\\
-29.105859375	2.32371720208161e-10\\
-29.085865234375	7.81831739799259e-11\\
-29.06587109375	7.35378038319564e-11\\
-29.045876953125	4.9981905465622e-12\\
-29.0258828125	-3.14488677627953e-11\\
-29.005888671875	-5.10069238475418e-11\\
-28.98589453125	-8.58856841529688e-11\\
-28.965900390625	-7.41731329143496e-11\\
-28.94590625	-1.8883900759116e-10\\
-28.925912109375	-2.43712944637695e-10\\
-28.90591796875	-2.00003587385647e-10\\
-28.885923828125	-3.04800210067253e-10\\
-28.8659296875	-4.0758017079981e-10\\
-28.845935546875	-3.55289383207815e-10\\
-28.82594140625	-4.21193886377486e-10\\
-28.805947265625	-4.98499802629319e-10\\
-28.785953125	-3.97484028048115e-10\\
-28.765958984375	-5.09201157999692e-10\\
-28.74596484375	-4.95497263098307e-10\\
-28.725970703125	-4.97378060711345e-10\\
-28.7059765625	-4.67639056339027e-10\\
-28.685982421875	-3.86629614737548e-10\\
-28.66598828125	-3.30128169336888e-10\\
-28.645994140625	-4.31094222415043e-10\\
-28.626	-3.35077916283737e-10\\
-28.606005859375	-3.60268564181836e-10\\
-28.58601171875	-3.25117766672526e-10\\
-28.566017578125	-4.10278395787661e-10\\
-28.5460234375	-3.18634155673816e-10\\
-28.526029296875	-3.27586634989775e-10\\
-28.50603515625	-3.84526278253743e-10\\
-28.486041015625	-3.0657662291806e-10\\
-28.466046875	-3.65415657522421e-10\\
-28.446052734375	-2.32167857515672e-10\\
-28.42605859375	-2.72052001308946e-10\\
-28.406064453125	-2.15451349796327e-10\\
-28.3860703125	-2.37245919818081e-10\\
-28.366076171875	-1.55305169157071e-10\\
-28.34608203125	-1.89019531222627e-10\\
-28.326087890625	-2.48965740512932e-10\\
-28.30609375	-1.58785556490839e-10\\
-28.286099609375	-1.90343683086112e-10\\
-28.26610546875	-1.9426893054753e-10\\
-28.246111328125	-2.61146008462869e-10\\
-28.2261171875	-3.88566654892468e-10\\
-28.206123046875	-4.11847242636812e-10\\
-28.18612890625	-4.90410960570357e-10\\
-28.166134765625	-5.32349087629261e-10\\
-28.146140625	-3.86891801714742e-10\\
-28.126146484375	-3.91757957528933e-10\\
-28.10615234375	-5.26346596082102e-10\\
-28.086158203125	-4.22315400546392e-10\\
-28.0661640625	-4.14549508928473e-10\\
-28.046169921875	-5.09525783738643e-10\\
-28.02617578125	-5.63900763456738e-10\\
-28.006181640625	-5.89644739274269e-10\\
-27.9861875	-7.3780806630875e-10\\
-27.966193359375	-7.18061437528153e-10\\
-27.94619921875	-7.81358903084567e-10\\
-27.926205078125	-7.43412443690272e-10\\
-27.9062109375	-8.39788028093868e-10\\
-27.886216796875	-8.95926697461706e-10\\
-27.86622265625	-7.98029302983188e-10\\
-27.846228515625	-8.3184354396612e-10\\
-27.826234375	-8.39378567644044e-10\\
-27.806240234375	-8.99845632257358e-10\\
-27.78624609375	-9.13464517346092e-10\\
-27.766251953125	-9.17773218719524e-10\\
-27.7462578125	-9.53663211873259e-10\\
-27.726263671875	-9.183735820439e-10\\
-27.70626953125	-9.14976961374895e-10\\
-27.686275390625	-9.01130024334176e-10\\
-27.66628125	-8.65320816257362e-10\\
-27.646287109375	-8.60099776217985e-10\\
-27.62629296875	-7.819330392814e-10\\
-27.606298828125	-7.73120536260312e-10\\
-27.5863046875	-7.30602326508031e-10\\
-27.566310546875	-8.38177451625558e-10\\
-27.54631640625	-9.28727091458676e-10\\
-27.526322265625	-1.01808549675775e-09\\
-27.506328125	-1.04542917385156e-09\\
-27.486333984375	-1.00586385533737e-09\\
-27.46633984375	-1.06377002865417e-09\\
-27.446345703125	-1.00911876491814e-09\\
-27.4263515625	-9.23356163497834e-10\\
-27.406357421875	-9.72414750934701e-10\\
-27.38636328125	-9.66228485025489e-10\\
-27.366369140625	-8.90522641768178e-10\\
-27.346375	-9.45703200181337e-10\\
-27.326380859375	-1.01200841512774e-09\\
-27.30638671875	-9.98298954180648e-10\\
-27.286392578125	-1.06726389940821e-09\\
-27.2663984375	-1.06671199651719e-09\\
-27.246404296875	-1.17240989919249e-09\\
-27.22641015625	-1.08490158225317e-09\\
-27.206416015625	-1.13009855035612e-09\\
-27.186421875	-1.20116166363115e-09\\
-27.166427734375	-1.2312172021874e-09\\
-27.14643359375	-1.27307140509938e-09\\
-27.126439453125	-1.28817116442968e-09\\
-27.1064453125	-1.36550153385371e-09\\
-27.086451171875	-1.33339955104659e-09\\
-27.06645703125	-1.31620059176322e-09\\
-27.046462890625	-1.4146841569565e-09\\
-27.02646875	-1.33012573786123e-09\\
-27.006474609375	-1.44422229176449e-09\\
-26.98648046875	-1.51442756745819e-09\\
-26.966486328125	-1.66723014778546e-09\\
-26.9464921875	-1.67182169845041e-09\\
-26.926498046875	-1.80240384692109e-09\\
-26.90650390625	-1.8410496203435e-09\\
-26.886509765625	-1.83604044881349e-09\\
-26.866515625	-1.8220205327206e-09\\
-26.846521484375	-1.78959674855654e-09\\
-26.82652734375	-1.90041625996539e-09\\
-26.806533203125	-1.85050160978563e-09\\
-26.7865390625	-1.89464380102877e-09\\
-26.766544921875	-1.85151824434683e-09\\
-26.74655078125	-1.90300711885074e-09\\
-26.726556640625	-1.80453684873924e-09\\
-26.7065625	-1.82914491318695e-09\\
-26.686568359375	-1.75057907753913e-09\\
-26.66657421875	-1.71673597026607e-09\\
-26.646580078125	-1.69235853447465e-09\\
-26.6265859375	-1.72904146259416e-09\\
-26.606591796875	-1.67015045768621e-09\\
-26.58659765625	-1.53785848454453e-09\\
-26.566603515625	-1.57428636279066e-09\\
-26.546609375	-1.57447601258113e-09\\
-26.526615234375	-1.4913307803037e-09\\
-26.50662109375	-1.47286995891707e-09\\
-26.486626953125	-1.56419070512536e-09\\
-26.4666328125	-1.4587555833727e-09\\
-26.446638671875	-1.43338621901002e-09\\
-26.42664453125	-1.52498651453971e-09\\
-26.406650390625	-1.4091375961839e-09\\
-26.38665625	-1.35355785359008e-09\\
-26.366662109375	-1.30651509429087e-09\\
-26.34666796875	-1.22905244585285e-09\\
-26.326673828125	-1.24240433945182e-09\\
-26.3066796875	-1.16721266947404e-09\\
-26.286685546875	-1.22117697475612e-09\\
-26.26669140625	-1.27664592002642e-09\\
-26.246697265625	-1.24834175274934e-09\\
-26.226703125	-1.29346706952852e-09\\
-26.206708984375	-1.36526590337921e-09\\
-26.18671484375	-1.34816798462085e-09\\
-26.166720703125	-1.42195249576921e-09\\
-26.1467265625	-1.37964104306054e-09\\
-26.126732421875	-1.46813924189022e-09\\
-26.10673828125	-1.52175148419276e-09\\
-26.086744140625	-1.44175422047441e-09\\
-26.06675	-1.40697868472708e-09\\
-26.046755859375	-1.38245802922152e-09\\
-26.02676171875	-1.31623031292422e-09\\
-26.006767578125	-1.22080575617605e-09\\
-25.9867734375	-1.32503503415385e-09\\
-25.966779296875	-1.24741251842106e-09\\
-25.94678515625	-1.33376967178627e-09\\
-25.926791015625	-1.24031232143463e-09\\
-25.906796875	-1.32517412037101e-09\\
-25.886802734375	-1.24311230216897e-09\\
-25.86680859375	-1.24626402892301e-09\\
-25.846814453125	-1.144160367668e-09\\
-25.8268203125	-1.06897618576516e-09\\
-25.806826171875	-1.06075313141266e-09\\
-25.78683203125	-9.71843078229393e-10\\
-25.766837890625	-1.05403730084735e-09\\
-25.74684375	-1.05460032471326e-09\\
-25.726849609375	-1.01545085733452e-09\\
-25.70685546875	-1.04567998681539e-09\\
-25.686861328125	-9.52424050811929e-10\\
-25.6668671875	-9.07135975067843e-10\\
-25.646873046875	-8.2520559009081e-10\\
-25.62687890625	-5.27936909030867e-10\\
-25.606884765625	-3.26363685178652e-10\\
-25.586890625	-3.72953935341414e-10\\
-25.566896484375	-2.1534153475318e-10\\
-25.54690234375	-2.32825997493264e-10\\
-25.526908203125	-2.21071469041236e-10\\
-25.5069140625	-1.70309137400427e-10\\
-25.486919921875	-1.19667153396572e-10\\
-25.46692578125	-2.14232319561384e-11\\
-25.446931640625	1.05613118153312e-10\\
-25.4269375	2.74628717593594e-10\\
-25.406943359375	3.7936660379236e-10\\
-25.38694921875	5.19895599936688e-10\\
-25.366955078125	5.72459138151403e-10\\
-25.3469609375	6.16847388385539e-10\\
-25.326966796875	5.6279111799668e-10\\
-25.30697265625	6.18969874443865e-10\\
-25.286978515625	5.36820303777757e-10\\
-25.266984375	5.40943979780211e-10\\
-25.246990234375	5.52685547162075e-10\\
-25.22699609375	4.68248280143961e-10\\
-25.207001953125	5.99989010148338e-10\\
-25.1870078125	6.06523221394137e-10\\
-25.167013671875	6.09097402853687e-10\\
-25.14701953125	7.39058278591184e-10\\
-25.127025390625	6.24318078877684e-10\\
-25.10703125	5.80716903465069e-10\\
-25.087037109375	6.39363397953698e-10\\
-25.06704296875	7.01473937160628e-10\\
-25.047048828125	7.76918159888615e-10\\
-25.0270546875	7.2391543125017e-10\\
-25.007060546875	8.39631536574247e-10\\
-24.98706640625	8.55438207635576e-10\\
-24.967072265625	1.00655366265253e-09\\
-24.947078125	9.51429824415873e-10\\
-24.927083984375	9.81325193690232e-10\\
-24.90708984375	9.37636736403679e-10\\
-24.887095703125	9.33055386132151e-10\\
-24.8671015625	7.97587848006513e-10\\
-24.847107421875	8.74084978700524e-10\\
-24.82711328125	8.74750052285004e-10\\
-24.807119140625	9.30095663664793e-10\\
-24.787125	1.06796304707965e-09\\
-24.767130859375	1.08147504381182e-09\\
-24.74713671875	1.1190546625238e-09\\
-24.727142578125	1.11738957872577e-09\\
-24.7071484375	1.18408694338405e-09\\
-24.687154296875	1.14970842101209e-09\\
-24.66716015625	1.16150538174096e-09\\
-24.647166015625	1.12247291363372e-09\\
-24.627171875	1.24128716049672e-09\\
-24.607177734375	1.25883665632957e-09\\
-24.58718359375	1.24915869817479e-09\\
-24.567189453125	1.29535875546352e-09\\
-24.5471953125	1.28213138003375e-09\\
-24.527201171875	1.33224827138524e-09\\
-24.50720703125	1.26909087051464e-09\\
-24.487212890625	1.32938585826039e-09\\
-24.46721875	1.32236542671673e-09\\
-24.447224609375	1.32569773437603e-09\\
-24.42723046875	1.47107970871504e-09\\
-24.407236328125	1.35437643308068e-09\\
-24.3872421875	1.60466995089428e-09\\
-24.367248046875	1.56636559856566e-09\\
-24.34725390625	1.58775392742963e-09\\
-24.327259765625	1.57113134826446e-09\\
-24.307265625	1.50426551494047e-09\\
-24.287271484375	1.56709469949167e-09\\
-24.26727734375	1.49246247710104e-09\\
-24.247283203125	1.42807096388072e-09\\
-24.2272890625	1.28862643334915e-09\\
-24.207294921875	1.37372710602665e-09\\
-24.18730078125	1.16124815922166e-09\\
-24.167306640625	1.19219606707195e-09\\
-24.1473125	1.27103272957274e-09\\
-24.127318359375	1.28285244575792e-09\\
-24.10732421875	1.17905853265614e-09\\
-24.087330078125	1.1151261463228e-09\\
-24.0673359375	1.08356593847266e-09\\
-24.047341796875	1.01174823497156e-09\\
-24.02734765625	9.39712707019912e-10\\
-24.007353515625	9.82922336017776e-10\\
-23.987359375	9.90924204339681e-10\\
-23.967365234375	8.92348980827855e-10\\
-23.94737109375	8.62716269225874e-10\\
-23.927376953125	8.33609930516313e-10\\
-23.9073828125	7.92737953975226e-10\\
-23.887388671875	7.39543293166196e-10\\
-23.86739453125	6.94349713244603e-10\\
-23.847400390625	7.00182848614142e-10\\
-23.82740625	6.06508099439782e-10\\
-23.807412109375	4.75146560195677e-10\\
-23.78741796875	4.58475183187204e-10\\
-23.767423828125	4.99616806500103e-10\\
-23.7474296875	3.70582016076038e-10\\
-23.727435546875	4.91813790393692e-10\\
-23.70744140625	5.55046010239738e-10\\
-23.687447265625	5.25633578423703e-10\\
-23.667453125	6.04503709896669e-10\\
-23.647458984375	7.42006461088898e-10\\
-23.62746484375	7.84208682069523e-10\\
-23.607470703125	8.61559242328815e-10\\
-23.5874765625	8.45371673728079e-10\\
-23.567482421875	6.98816915688287e-10\\
-23.54748828125	7.57161913730376e-10\\
-23.527494140625	5.75275740634772e-10\\
-23.5075	5.18004285337518e-10\\
-23.487505859375	5.3070399304208e-10\\
-23.46751171875	3.9654609730833e-10\\
-23.447517578125	5.26030191074934e-10\\
-23.4275234375	4.70638459156439e-10\\
-23.407529296875	6.14234703361086e-10\\
-23.38753515625	5.44180247099742e-10\\
-23.367541015625	5.20622644340699e-10\\
-23.347546875	4.33037242633333e-10\\
-23.327552734375	3.30810911192546e-10\\
-23.30755859375	3.24013006212012e-10\\
-23.287564453125	1.36064122117556e-10\\
-23.2675703125	8.68698733636112e-11\\
-23.247576171875	6.81785369095397e-11\\
-23.22758203125	-2.39218108567829e-11\\
-23.207587890625	6.50600435086587e-12\\
-23.18759375	-3.25735240614424e-11\\
-23.167599609375	-7.67105266898549e-11\\
-23.14760546875	-2.54927047816392e-10\\
-23.127611328125	-3.99896260720256e-10\\
-23.1076171875	-4.11011954655687e-10\\
-23.087623046875	-5.03232983788245e-10\\
-23.06762890625	-7.74816157574177e-10\\
-23.047634765625	-8.59145482064458e-10\\
-23.027640625	-9.10149539480305e-10\\
-23.007646484375	-9.65727542108078e-10\\
-22.98765234375	-1.03535099711557e-09\\
-22.967658203125	-1.05660329996714e-09\\
-22.9476640625	-1.0555560137177e-09\\
-22.927669921875	-1.17152564769918e-09\\
-22.90767578125	-1.31195377361466e-09\\
-22.887681640625	-1.30632736906425e-09\\
-22.8676875	-1.60026308390132e-09\\
-22.847693359375	-1.56064341616405e-09\\
-22.82769921875	-1.64950762197737e-09\\
-22.807705078125	-1.67832485411641e-09\\
-22.7877109375	-1.69260870530205e-09\\
-22.767716796875	-1.75255353724131e-09\\
-22.74772265625	-1.71457640763254e-09\\
-22.727728515625	-1.6629647502472e-09\\
-22.707734375	-1.71097721494442e-09\\
-22.687740234375	-1.66506836239431e-09\\
-22.66774609375	-1.67719082433282e-09\\
-22.647751953125	-1.67635274187394e-09\\
-22.6277578125	-1.63389566517355e-09\\
-22.607763671875	-1.59918030919823e-09\\
-22.58776953125	-1.63225432760813e-09\\
-22.567775390625	-1.64696686216297e-09\\
-22.54778125	-1.67945015482734e-09\\
-22.527787109375	-1.69716550856149e-09\\
-22.50779296875	-1.7595738719762e-09\\
-22.487798828125	-1.76319960769779e-09\\
-22.4678046875	-1.93910947417648e-09\\
-22.447810546875	-1.9671255453211e-09\\
-22.42781640625	-2.12074092278846e-09\\
-22.407822265625	-2.18228013528015e-09\\
-22.387828125	-2.13819751548037e-09\\
-22.367833984375	-2.04257181790408e-09\\
-22.34783984375	-2.08471932519424e-09\\
-22.327845703125	-1.99625732586351e-09\\
-22.3078515625	-2.06957566431766e-09\\
-22.287857421875	-2.10277447676409e-09\\
-22.26786328125	-2.07081950372856e-09\\
-22.247869140625	-2.1035358813679e-09\\
-22.227875	-2.16965266411459e-09\\
-22.207880859375	-2.26110614054965e-09\\
-22.18788671875	-2.20277633323222e-09\\
-22.167892578125	-2.28589120733368e-09\\
-22.1478984375	-2.20449736204042e-09\\
-22.127904296875	-2.25068336027892e-09\\
-22.10791015625	-2.28655659935448e-09\\
-22.087916015625	-2.45556832153143e-09\\
-22.067921875	-2.54115334858359e-09\\
-22.047927734375	-2.48500649028331e-09\\
-22.02793359375	-2.5020008907563e-09\\
-22.007939453125	-2.58640194971603e-09\\
-21.9879453125	-2.4405063205647e-09\\
-21.967951171875	-2.47608168695271e-09\\
-21.94795703125	-2.48127525512887e-09\\
-21.927962890625	-2.31817325458887e-09\\
-21.90796875	-2.22887014530935e-09\\
-21.887974609375	-2.43301490680579e-09\\
-21.86798046875	-2.49783665384868e-09\\
-21.847986328125	-2.43343639627724e-09\\
-21.8279921875	-2.46850758404835e-09\\
-21.807998046875	-2.46869962797765e-09\\
-21.78800390625	-2.47474927886266e-09\\
-21.768009765625	-2.48851238020723e-09\\
-21.748015625	-2.38138031103013e-09\\
-21.728021484375	-2.24697124305633e-09\\
-21.70802734375	-2.23947940095259e-09\\
-21.688033203125	-2.05858144088558e-09\\
-21.6680390625	-2.05812343419978e-09\\
-21.648044921875	-2.12267586135556e-09\\
-21.62805078125	-2.13790229510301e-09\\
-21.608056640625	-2.08551731076989e-09\\
-21.5880625	-2.13777790868887e-09\\
-21.568068359375	-1.98181632176789e-09\\
-21.54807421875	-2.04792174468987e-09\\
-21.528080078125	-1.96675863614308e-09\\
-21.5080859375	-1.9420000908632e-09\\
-21.488091796875	-1.90627008840413e-09\\
-21.46809765625	-1.90718311289926e-09\\
-21.448103515625	-1.74888584493519e-09\\
-21.428109375	-1.86062199897838e-09\\
-21.408115234375	-1.86442850696283e-09\\
-21.38812109375	-1.8009287531006e-09\\
-21.368126953125	-1.76140644735215e-09\\
-21.3481328125	-1.70508930874351e-09\\
-21.328138671875	-1.67239942199127e-09\\
-21.30814453125	-1.5193158035741e-09\\
-21.288150390625	-1.52900790531506e-09\\
-21.26815625	-1.40062424384638e-09\\
-21.248162109375	-1.38275410116213e-09\\
-21.22816796875	-1.41521768997573e-09\\
-21.208173828125	-1.45656760102205e-09\\
-21.1881796875	-1.44998655065178e-09\\
-21.168185546875	-1.45731326090233e-09\\
-21.14819140625	-1.59808214131953e-09\\
-21.128197265625	-1.57568948947253e-09\\
-21.108203125	-1.42381290080382e-09\\
-21.088208984375	-1.3757363651157e-09\\
-21.06821484375	-1.08349486113756e-09\\
-21.048220703125	-9.63744328876422e-10\\
-21.0282265625	-8.61672966726009e-10\\
-21.008232421875	-6.96454585814736e-10\\
-20.98823828125	-7.89534621308956e-10\\
-20.968244140625	-6.25683446114634e-10\\
-20.94825	-8.34288502845767e-10\\
-20.928255859375	-8.10165385853645e-10\\
-20.90826171875	-7.51248024995403e-10\\
-20.888267578125	-6.66795862374888e-10\\
-20.8682734375	-6.65293820722337e-10\\
-20.848279296875	-3.63343484946403e-10\\
-20.82828515625	-3.65216090565399e-10\\
-20.808291015625	-1.17898821838639e-10\\
-20.788296875	4.24345588018592e-11\\
-20.768302734375	1.47879131766649e-10\\
-20.74830859375	1.58959962613329e-10\\
-20.728314453125	1.10780683370334e-10\\
-20.7083203125	9.72277569531434e-11\\
-20.688326171875	1.21048619782296e-10\\
-20.66833203125	2.37448167319704e-10\\
-20.648337890625	3.63931946236405e-10\\
-20.62834375	4.97413488443996e-10\\
-20.608349609375	7.25902261687892e-10\\
-20.58835546875	8.10674216167929e-10\\
-20.568361328125	1.11734788321996e-09\\
-20.5483671875	1.20770870233496e-09\\
-20.528373046875	1.30665875118757e-09\\
-20.50837890625	1.33506074974061e-09\\
-20.488384765625	1.4833120412871e-09\\
-20.468390625	1.54954434238471e-09\\
-20.448396484375	1.51114633298191e-09\\
-20.42840234375	1.79174877490567e-09\\
-20.408408203125	1.96757712125931e-09\\
-20.3884140625	2.05416245857176e-09\\
-20.368419921875	2.37047667202952e-09\\
-20.34842578125	2.29404220121742e-09\\
-20.328431640625	2.43347930330988e-09\\
-20.3084375	2.40968629407773e-09\\
-20.288443359375	2.412358835281e-09\\
-20.26844921875	2.5190106121062e-09\\
-20.248455078125	2.37876093397431e-09\\
-20.2284609375	2.3773126391429e-09\\
-20.208466796875	2.48256714485596e-09\\
-20.18847265625	2.51759919896621e-09\\
-20.168478515625	2.59517502249014e-09\\
-20.148484375	2.67653277900999e-09\\
-20.128490234375	2.6456752371618e-09\\
-20.10849609375	2.66233396724105e-09\\
-20.088501953125	2.62092733892006e-09\\
-20.0685078125	2.664955572789e-09\\
-20.048513671875	2.71419485898188e-09\\
-20.02851953125	2.75708852104296e-09\\
-20.008525390625	2.81225457496767e-09\\
-19.98853125	2.74593334910025e-09\\
-19.968537109375	3.02924253282569e-09\\
-19.94854296875	3.02580174055755e-09\\
-19.928548828125	3.1851927471991e-09\\
-19.9085546875	3.24918277618442e-09\\
-19.888560546875	3.24522094090591e-09\\
-19.86856640625	3.28731570255094e-09\\
-19.848572265625	3.29053708891414e-09\\
-19.828578125	3.41313167092666e-09\\
-19.808583984375	3.41688740096796e-09\\
-19.78858984375	3.40483782741716e-09\\
-19.768595703125	3.58143233049011e-09\\
-19.7486015625	3.54104238616916e-09\\
-19.728607421875	3.62276660139628e-09\\
-19.70861328125	3.70178174241429e-09\\
-19.688619140625	3.56324391036146e-09\\
-19.668625	3.60716871784238e-09\\
-19.648630859375	3.44340540366013e-09\\
-19.62863671875	3.56920034959173e-09\\
-19.608642578125	3.58164232136195e-09\\
-19.5886484375	3.63617610327014e-09\\
-19.568654296875	3.66438450593532e-09\\
-19.54866015625	3.85934564211149e-09\\
-19.528666015625	3.94043284395349e-09\\
-19.508671875	4.0386394516595e-09\\
-19.488677734375	4.03245503103454e-09\\
-19.46868359375	4.00601040956566e-09\\
-19.448689453125	3.8613828726607e-09\\
-19.4286953125	3.91171218109699e-09\\
-19.408701171875	3.85072191350645e-09\\
-19.38870703125	3.83855610147892e-09\\
-19.368712890625	3.79989016938687e-09\\
-19.34871875	3.78025100415877e-09\\
-19.328724609375	3.96170583519443e-09\\
-19.30873046875	3.94556562260368e-09\\
-19.288736328125	3.92699277991324e-09\\
-19.2687421875	4.10437071313978e-09\\
-19.248748046875	3.97422813937297e-09\\
-19.22875390625	3.92347570229744e-09\\
-19.208759765625	3.9678820748993e-09\\
-19.188765625	3.69596197687525e-09\\
-19.168771484375	3.63976305594401e-09\\
-19.14877734375	3.61756652565804e-09\\
-19.128783203125	3.46754323819987e-09\\
-19.1087890625	3.55453842974756e-09\\
-19.088794921875	3.52602089912802e-09\\
-19.06880078125	3.35588963803254e-09\\
-19.048806640625	3.52991210103881e-09\\
-19.0288125	3.39041784279623e-09\\
-19.008818359375	3.51831719603013e-09\\
-18.98882421875	3.42280356396993e-09\\
-18.968830078125	3.38014402024511e-09\\
-18.9488359375	3.28423687520934e-09\\
-18.928841796875	3.27864809340126e-09\\
-18.90884765625	3.27487811748128e-09\\
-18.888853515625	3.27690427542031e-09\\
-18.868859375	3.29969877150787e-09\\
-18.848865234375	3.3620105869131e-09\\
-18.82887109375	3.25633315186531e-09\\
-18.808876953125	3.2326111028278e-09\\
-18.7888828125	3.31502948281929e-09\\
-18.768888671875	3.24391322614441e-09\\
-18.74889453125	3.3177499793828e-09\\
-18.728900390625	3.16091418058907e-09\\
-18.70890625	3.15191497776751e-09\\
-18.688912109375	3.25355921871478e-09\\
-18.66891796875	3.15667284634127e-09\\
-18.648923828125	3.266635053976e-09\\
-18.6289296875	3.33647355516109e-09\\
-18.608935546875	3.26324779048175e-09\\
-18.58894140625	3.29051814117197e-09\\
-18.568947265625	3.25125520070428e-09\\
-18.548953125	3.22051268723068e-09\\
-18.528958984375	3.23046128381792e-09\\
-18.50896484375	3.08320122579008e-09\\
-18.488970703125	3.08800000889763e-09\\
-18.4689765625	2.96668097595598e-09\\
-18.448982421875	2.83768996108022e-09\\
-18.42898828125	2.80753615116401e-09\\
-18.408994140625	2.73961175745083e-09\\
-18.389	2.65542289827426e-09\\
-18.369005859375	2.68621285037818e-09\\
-18.34901171875	2.46472688641567e-09\\
-18.329017578125	2.40548487453456e-09\\
-18.3090234375	2.30672714693584e-09\\
-18.289029296875	2.17772774082309e-09\\
-18.26903515625	2.0966059360087e-09\\
-18.249041015625	1.98235889429563e-09\\
-18.229046875	1.84217960568286e-09\\
-18.209052734375	1.76226234720678e-09\\
-18.18905859375	1.67310673971131e-09\\
-18.169064453125	1.51097534622666e-09\\
-18.1490703125	1.47022204806205e-09\\
-18.129076171875	1.37226038025075e-09\\
-18.10908203125	1.1513339690094e-09\\
-18.089087890625	1.22557805729278e-09\\
-18.06909375	9.31998045002514e-10\\
-18.049099609375	9.05544326581114e-10\\
-18.02910546875	8.29922740781285e-10\\
-18.009111328125	8.23677758038201e-10\\
-17.9891171875	8.83479644494127e-10\\
-17.969123046875	9.79489764080432e-10\\
-17.94912890625	9.76833357145317e-10\\
-17.929134765625	7.68288728907874e-10\\
-17.909140625	5.14011494580212e-10\\
-17.889146484375	3.83681315335744e-10\\
-17.86915234375	6.19893558086396e-11\\
-17.849158203125	1.48273996502167e-10\\
-17.8291640625	-6.55780412872273e-11\\
-17.809169921875	-4.30926610617141e-11\\
-17.78917578125	-5.2361351863979e-11\\
-17.769181640625	-3.60976603607936e-11\\
-17.7491875	1.41442307818771e-10\\
-17.729193359375	1.56998968775663e-10\\
-17.70919921875	8.97050591504522e-11\\
-17.689205078125	1.3135856852758e-10\\
-17.6692109375	-6.23801210107759e-11\\
-17.649216796875	-1.32131179836399e-10\\
-17.62922265625	-1.92123407212217e-10\\
-17.609228515625	-3.85328779007712e-10\\
-17.589234375	-4.17474400898581e-10\\
-17.569240234375	-5.32376831419557e-10\\
-17.54924609375	-4.54453849945912e-10\\
-17.529251953125	-4.50017314756806e-10\\
-17.5092578125	-6.19201178113022e-10\\
-17.489263671875	-4.48254301530029e-10\\
-17.46926953125	-6.3623445528256e-10\\
-17.449275390625	-6.27565874312954e-10\\
-17.42928125	-6.48978289892506e-10\\
-17.409287109375	-8.45029366928833e-10\\
-17.38929296875	-9.09831943647049e-10\\
-17.369298828125	-1.04491686811075e-09\\
-17.3493046875	-9.63036246308734e-10\\
-17.329310546875	-1.20500561457058e-09\\
-17.30931640625	-1.13644664840511e-09\\
-17.289322265625	-1.4925813067997e-09\\
-17.269328125	-1.55052321983388e-09\\
-17.249333984375	-1.63630221947979e-09\\
-17.22933984375	-1.6902363405336e-09\\
-17.209345703125	-1.79135188806849e-09\\
-17.1893515625	-1.68870424226112e-09\\
-17.169357421875	-1.82685888857241e-09\\
-17.14936328125	-1.76151027041843e-09\\
-17.129369140625	-1.68104566939462e-09\\
-17.109375	-1.7214368351417e-09\\
-17.089380859375	-1.8687600709947e-09\\
-17.06938671875	-1.97626326805701e-09\\
-17.049392578125	-2.17878914202135e-09\\
-17.0293984375	-2.24665969976691e-09\\
-17.009404296875	-2.29995364759513e-09\\
-16.98941015625	-2.48687983402354e-09\\
-16.969416015625	-2.41267337262053e-09\\
-16.949421875	-2.49283504583741e-09\\
-16.929427734375	-2.54271641151325e-09\\
-16.90943359375	-2.55789315157226e-09\\
-16.889439453125	-2.61258071360486e-09\\
-16.8694453125	-2.82165963748671e-09\\
-16.849451171875	-2.9509602520004e-09\\
-16.82945703125	-2.96555809611469e-09\\
-16.809462890625	-3.01305965055236e-09\\
-16.78946875	-2.99501484390461e-09\\
-16.769474609375	-3.01318139043214e-09\\
-16.74948046875	-2.93675767639355e-09\\
-16.729486328125	-2.78740604049624e-09\\
-16.7094921875	-2.78508905869639e-09\\
-16.689498046875	-2.75828499526949e-09\\
-16.66950390625	-2.8245442408607e-09\\
-16.649509765625	-2.93230959184765e-09\\
-16.629515625	-2.86720764041023e-09\\
-16.609521484375	-2.9881685341527e-09\\
-16.58952734375	-2.95393429508409e-09\\
-16.569533203125	-2.76456555416184e-09\\
-16.5495390625	-2.77122886814086e-09\\
-16.529544921875	-2.5545799072642e-09\\
-16.50955078125	-2.61011429367642e-09\\
-16.489556640625	-2.52490594857042e-09\\
-16.4695625	-2.56376151090866e-09\\
-16.449568359375	-2.57419675913504e-09\\
-16.42957421875	-2.60685162845298e-09\\
-16.409580078125	-2.629428742205e-09\\
-16.3895859375	-2.64091865564466e-09\\
-16.369591796875	-2.79462151919162e-09\\
-16.34959765625	-2.76099533300518e-09\\
-16.329603515625	-2.8327388502839e-09\\
-16.309609375	-2.72949377257971e-09\\
-16.289615234375	-2.75691895870997e-09\\
-16.26962109375	-2.78834481441871e-09\\
-16.249626953125	-2.77774091346472e-09\\
-16.2296328125	-2.6957737568611e-09\\
-16.209638671875	-2.88317383567535e-09\\
-16.18964453125	-2.91773444809253e-09\\
-16.169650390625	-2.96040703977022e-09\\
-16.14965625	-3.11340211557313e-09\\
-16.129662109375	-3.09518984141926e-09\\
-16.10966796875	-3.09541844187315e-09\\
-16.089673828125	-3.15319530536207e-09\\
-16.0696796875	-3.20649241253009e-09\\
-16.049685546875	-3.19585027585334e-09\\
-16.02969140625	-3.25238302729426e-09\\
-16.009697265625	-3.10279581536865e-09\\
-15.989703125	-3.10229290818353e-09\\
-15.969708984375	-3.08054283432654e-09\\
-15.94971484375	-2.97065872139744e-09\\
-15.929720703125	-2.8783792459904e-09\\
-15.9097265625	-2.80122799025175e-09\\
-15.889732421875	-2.76218734205923e-09\\
-15.86973828125	-2.77547855106542e-09\\
-15.849744140625	-2.59672281464524e-09\\
-15.82975	-2.64397494039232e-09\\
-15.809755859375	-2.49119222254365e-09\\
-15.78976171875	-2.33131951229153e-09\\
-15.769767578125	-2.16439982520636e-09\\
-15.7497734375	-2.21693692450056e-09\\
-15.729779296875	-1.99037967527665e-09\\
-15.70978515625	-2.04279560838283e-09\\
-15.689791015625	-1.92806827234399e-09\\
-15.669796875	-1.87253064556271e-09\\
-15.649802734375	-1.73784421019239e-09\\
-15.62980859375	-1.68316837964164e-09\\
-15.609814453125	-1.56395033779816e-09\\
-15.5898203125	-1.5143057263873e-09\\
-15.569826171875	-1.35635312731873e-09\\
-15.54983203125	-1.38591653951875e-09\\
-15.529837890625	-1.28438086717522e-09\\
-15.50984375	-1.20860357352229e-09\\
-15.489849609375	-1.23718748797897e-09\\
-15.46985546875	-1.30835108287705e-09\\
-15.449861328125	-1.07343047843081e-09\\
-15.4298671875	-1.17144845237536e-09\\
-15.409873046875	-1.046278048458e-09\\
-15.38987890625	-1.05081124423518e-09\\
-15.369884765625	-8.08630805522706e-10\\
-15.349890625	-8.59053014755483e-10\\
-15.329896484375	-7.96252827288303e-10\\
-15.30990234375	-7.60679999379408e-10\\
-15.289908203125	-7.06386526692073e-10\\
-15.2699140625	-7.69958832258609e-10\\
-15.249919921875	-7.88684339879069e-10\\
-15.22992578125	-7.19260616680976e-10\\
-15.209931640625	-6.92998853871459e-10\\
-15.1899375	-7.36105155284738e-10\\
-15.169943359375	-5.802733224147e-10\\
-15.14994921875	-6.13670637814103e-10\\
-15.129955078125	-6.4666266099011e-10\\
-15.1099609375	-5.07573521472895e-10\\
-15.089966796875	-5.31110077914233e-10\\
-15.06997265625	-5.39758838132082e-10\\
-15.049978515625	-5.00222172692514e-10\\
-15.029984375	-5.68780754998931e-10\\
-15.009990234375	-4.21147423290633e-10\\
-14.98999609375	-4.63321593461232e-10\\
-14.970001953125	-1.54790006342087e-10\\
-14.9500078125	-8.77001344106807e-11\\
-14.930013671875	-3.13139466923289e-11\\
-14.91001953125	2.23660011963727e-10\\
-14.890025390625	2.49374003283653e-10\\
-14.87003125	3.73603072480197e-10\\
-14.850037109375	2.81760917822329e-10\\
-14.83004296875	3.44656460494962e-10\\
-14.810048828125	2.794590148917e-10\\
-14.7900546875	5.08371635232165e-10\\
-14.770060546875	4.05127603106808e-10\\
-14.75006640625	7.70515969677217e-10\\
-14.730072265625	9.34417748373338e-10\\
-14.710078125	1.10492081586418e-09\\
-14.690083984375	1.17527948095143e-09\\
-14.67008984375	1.35804792702684e-09\\
-14.650095703125	1.33901580679515e-09\\
-14.6301015625	1.2086176757111e-09\\
-14.610107421875	1.14974668222384e-09\\
-14.59011328125	1.26919240668354e-09\\
-14.570119140625	1.13415419141139e-09\\
-14.550125	1.21130651744195e-09\\
-14.530130859375	1.3231804079116e-09\\
-14.51013671875	1.4401163512405e-09\\
-14.490142578125	1.62904799031031e-09\\
-14.4701484375	1.58679614128869e-09\\
-14.450154296875	1.82991199156218e-09\\
-14.43016015625	1.79006348792424e-09\\
-14.410166015625	1.84387604596825e-09\\
-14.390171875	1.91194663964994e-09\\
-14.370177734375	1.9453423768631e-09\\
-14.35018359375	1.98473854483374e-09\\
-14.330189453125	2.02251359696552e-09\\
-14.3101953125	2.07934946502586e-09\\
-14.290201171875	2.03629543728397e-09\\
-14.27020703125	2.1275178432154e-09\\
-14.250212890625	2.0389492889197e-09\\
-14.23021875	2.05681776459938e-09\\
-14.210224609375	2.02686668158317e-09\\
-14.19023046875	2.04629731014573e-09\\
-14.170236328125	2.00108724279893e-09\\
-14.1502421875	1.90036550471551e-09\\
-14.130248046875	2.05809373133012e-09\\
-14.11025390625	2.05720862557693e-09\\
-14.090259765625	1.98849907515029e-09\\
-14.070265625	2.05003025133436e-09\\
-14.050271484375	2.13542342496882e-09\\
-14.03027734375	1.98022494868172e-09\\
-14.010283203125	2.1222366678769e-09\\
-13.9902890625	1.88063051361344e-09\\
-13.970294921875	1.9629634111483e-09\\
-13.95030078125	1.84046890728535e-09\\
-13.930306640625	1.87703330902428e-09\\
-13.9103125	1.91611142022429e-09\\
-13.890318359375	1.92803142175268e-09\\
-13.87032421875	1.7963752482818e-09\\
-13.850330078125	1.73332866508099e-09\\
-13.8303359375	1.90218819869672e-09\\
-13.810341796875	1.83714933860912e-09\\
-13.79034765625	1.92676474978284e-09\\
-13.770353515625	1.95628633076841e-09\\
-13.750359375	2.00918589114603e-09\\
-13.730365234375	2.08212749477353e-09\\
-13.71037109375	2.17610215911838e-09\\
-13.690376953125	2.1930287659093e-09\\
-13.6703828125	2.18560583297288e-09\\
-13.650388671875	2.18626580793489e-09\\
-13.63039453125	2.0731711077326e-09\\
-13.610400390625	2.1908922929032e-09\\
-13.59040625	2.1566399055075e-09\\
-13.570412109375	2.31128166684209e-09\\
-13.55041796875	2.31351484188806e-09\\
-13.530423828125	2.35877280555723e-09\\
-13.5104296875	2.36664851723729e-09\\
-13.490435546875	2.41524795356066e-09\\
-13.47044140625	2.44559412691162e-09\\
-13.450447265625	2.33402222007372e-09\\
-13.430453125	2.27639158981931e-09\\
-13.410458984375	2.26961482961606e-09\\
-13.39046484375	2.13810758201458e-09\\
-13.370470703125	2.1006825708691e-09\\
-13.3504765625	2.07530599551916e-09\\
-13.330482421875	2.02437806227161e-09\\
-13.31048828125	1.96181165405695e-09\\
-13.290494140625	1.8829354046623e-09\\
-13.2705	1.68454322645051e-09\\
-13.250505859375	1.63946380600257e-09\\
-13.23051171875	1.44412255649537e-09\\
-13.210517578125	1.45307103685188e-09\\
-13.1905234375	1.41374620992929e-09\\
-13.170529296875	1.38631722040165e-09\\
-13.15053515625	1.35584659160598e-09\\
-13.130541015625	1.33410186788806e-09\\
-13.110546875	1.28517794805605e-09\\
-13.090552734375	1.30314658250945e-09\\
-13.07055859375	1.16041289282456e-09\\
-13.050564453125	1.13556930087085e-09\\
-13.0305703125	8.4961512899674e-10\\
-13.010576171875	8.97733241222425e-10\\
-12.99058203125	9.80397553621197e-10\\
-12.970587890625	8.77113960837589e-10\\
-12.95059375	8.34170335501216e-10\\
-12.930599609375	9.39735871910536e-10\\
-12.91060546875	8.47654997058793e-10\\
-12.890611328125	8.4151063346522e-10\\
-12.8706171875	7.43926927743731e-10\\
-12.850623046875	8.71195918430746e-10\\
-12.83062890625	5.89539463347713e-10\\
-12.810634765625	7.11775652458989e-10\\
-12.790640625	6.03055430842106e-10\\
-12.770646484375	7.03077199950726e-10\\
-12.75065234375	5.30917170869934e-10\\
-12.730658203125	6.54498441515539e-10\\
-12.7106640625	6.04508799478252e-10\\
-12.690669921875	5.28471109926207e-10\\
-12.67067578125	5.69355314207509e-10\\
-12.650681640625	5.7365774962195e-10\\
-12.6306875	4.55538340806426e-10\\
-12.610693359375	4.97604167141277e-10\\
-12.59069921875	4.00164251064382e-10\\
-12.570705078125	3.91009227625486e-10\\
-12.5507109375	4.48706555806057e-10\\
-12.530716796875	4.03244757398756e-10\\
-12.51072265625	4.22837974938764e-10\\
-12.490728515625	4.1100080327623e-10\\
-12.470734375	3.99714091296442e-10\\
-12.450740234375	4.03556714922751e-10\\
-12.43074609375	2.86073539754458e-10\\
-12.410751953125	1.90484178291388e-10\\
-12.3907578125	1.7949014448335e-10\\
-12.370763671875	-2.44739182462166e-12\\
-12.35076953125	5.21028381137941e-11\\
-12.330775390625	-8.63543422757048e-11\\
-12.31078125	4.75280241695442e-11\\
-12.290787109375	-8.74228575100185e-12\\
-12.27079296875	1.75416001820753e-10\\
-12.250798828125	3.05354297773214e-11\\
-12.2308046875	1.14673673191819e-10\\
-12.210810546875	-8.05380412231654e-11\\
-12.19081640625	-8.07227653083369e-11\\
-12.170822265625	-1.35562207332109e-10\\
-12.150828125	-2.25273808751406e-10\\
-12.130833984375	-1.37586933444555e-10\\
-12.11083984375	-2.86755725472097e-10\\
-12.090845703125	-3.09588090003175e-10\\
-12.0708515625	-3.50605084901365e-10\\
-12.050857421875	-2.84406671795079e-10\\
-12.03086328125	-3.10555128486233e-10\\
-12.010869140625	-3.70657358338745e-10\\
-11.990875	-3.08158756970502e-10\\
-11.970880859375	-3.3701480066583e-10\\
-11.95088671875	-4.42675936521615e-10\\
-11.930892578125	-3.62632852770126e-10\\
-11.9108984375	-4.96443369201886e-10\\
-11.890904296875	-5.22589699954748e-10\\
-11.87091015625	-6.23395216277196e-10\\
-11.850916015625	-7.04227085008257e-10\\
-11.830921875	-8.22528142728799e-10\\
-11.810927734375	-9.18818134908206e-10\\
-11.79093359375	-9.82766448995222e-10\\
-11.770939453125	-9.83853399212876e-10\\
-11.7509453125	-8.69894282376886e-10\\
-11.730951171875	-8.91525692269653e-10\\
-11.71095703125	-8.59310528007107e-10\\
-11.690962890625	-7.08886950382436e-10\\
-11.67096875	-7.65256206826781e-10\\
-11.650974609375	-6.81766971918882e-10\\
-11.63098046875	-9.12065368738514e-10\\
-11.610986328125	-7.38551551695288e-10\\
-11.5909921875	-8.1810879781857e-10\\
-11.570998046875	-7.54507206624244e-10\\
-11.55100390625	-7.4111802916128e-10\\
-11.531009765625	-6.57579656416306e-10\\
-11.511015625	-7.89048344077717e-10\\
-11.491021484375	-5.89464210686039e-10\\
-11.47102734375	-7.34540956426283e-10\\
-11.451033203125	-6.8631978039855e-10\\
-11.4310390625	-7.24100063594498e-10\\
-11.411044921875	-6.79087497791468e-10\\
-11.39105078125	-6.31664200849294e-10\\
-11.371056640625	-6.65622312763327e-10\\
-11.3510625	-5.78544749775048e-10\\
-11.331068359375	-6.18346420849484e-10\\
-11.31107421875	-5.43411461989433e-10\\
-11.291080078125	-5.85942915553383e-10\\
-11.2710859375	-5.55780963627091e-10\\
-11.251091796875	-6.78559547120911e-10\\
-11.23109765625	-6.97881967276574e-10\\
-11.211103515625	-7.32654679886691e-10\\
-11.191109375	-7.62246321127996e-10\\
-11.171115234375	-8.98060275685784e-10\\
-11.15112109375	-9.01808609576543e-10\\
-11.131126953125	-8.5775275255489e-10\\
-11.1111328125	-9.38308078477138e-10\\
-11.091138671875	-8.63542750423399e-10\\
-11.07114453125	-9.73651393190118e-10\\
-11.051150390625	-8.74695525104865e-10\\
-11.03115625	-8.8810633465152e-10\\
-11.011162109375	-8.56511504601447e-10\\
-10.99116796875	-8.62697247096231e-10\\
-10.971173828125	-8.07346310801134e-10\\
-10.9511796875	-8.81405736156105e-10\\
-10.931185546875	-9.02528655889489e-10\\
-10.91119140625	-8.46259006137629e-10\\
-10.891197265625	-8.16478533840177e-10\\
-10.871203125	-7.94111485934393e-10\\
-10.851208984375	-8.07185340037711e-10\\
-10.83121484375	-7.50277719494682e-10\\
-10.811220703125	-6.46716999370545e-10\\
-10.7912265625	-6.93668915410102e-10\\
-10.771232421875	-5.29336818693187e-10\\
-10.75123828125	-4.86371792658288e-10\\
-10.731244140625	-2.72235307408222e-10\\
-10.71125	-2.17788418630274e-10\\
-10.691255859375	-5.66970851435383e-13\\
-10.67126171875	4.85895441182225e-11\\
-10.651267578125	6.93490055132711e-11\\
-10.6312734375	1.53783325131657e-10\\
-10.611279296875	2.39206016724091e-10\\
-10.59128515625	2.68743120001299e-10\\
-10.571291015625	2.58873498005625e-10\\
-10.551296875	1.77155621755458e-10\\
-10.531302734375	3.86165479338147e-10\\
-10.51130859375	2.98945509273318e-10\\
-10.491314453125	3.80940076844078e-10\\
-10.4713203125	5.65968571377836e-10\\
-10.451326171875	4.92960723367573e-10\\
-10.43133203125	6.02999740847149e-10\\
-10.411337890625	7.33822507072657e-10\\
-10.39134375	5.88776844812021e-10\\
-10.371349609375	7.815503371436e-10\\
-10.35135546875	6.52647898215948e-10\\
-10.331361328125	8.94953129225957e-10\\
-10.3113671875	6.25532704530998e-10\\
-10.291373046875	8.4053866923119e-10\\
-10.27137890625	7.63473280990501e-10\\
-10.251384765625	8.40087276087184e-10\\
-10.231390625	7.02008970588736e-10\\
-10.211396484375	7.69727834570808e-10\\
-10.19140234375	7.6645606932401e-10\\
-10.171408203125	7.83635133159951e-10\\
-10.1514140625	7.42687418683664e-10\\
-10.131419921875	9.36453179513273e-10\\
-10.11142578125	8.96438117403655e-10\\
-10.091431640625	9.95180968564978e-10\\
-10.0714375	1.06368452869541e-09\\
-10.051443359375	1.09081689495425e-09\\
-10.03144921875	1.23235325981323e-09\\
-10.011455078125	1.18643376341208e-09\\
-9.9914609375	1.21832268966128e-09\\
-9.971466796875	1.25008925291074e-09\\
-9.95147265625	1.24314908095912e-09\\
-9.931478515625	1.40556150389173e-09\\
-9.911484375	1.43093639263761e-09\\
-9.891490234375	1.42556483807301e-09\\
-9.87149609375	1.51311500758281e-09\\
-9.851501953125	1.5409288178438e-09\\
-9.8315078125	1.54179629393498e-09\\
-9.81151367187501	1.51245822669731e-09\\
-9.79151953125	1.68221748439682e-09\\
-9.771525390625	1.55327707655274e-09\\
-9.75153125	1.77380509900831e-09\\
-9.731537109375	1.53919112172524e-09\\
-9.71154296875	1.78568129233843e-09\\
-9.691548828125	1.8439710111613e-09\\
-9.6715546875	1.74486950338666e-09\\
-9.651560546875	1.78748957400236e-09\\
-9.63156640625	1.78620182443287e-09\\
-9.611572265625	1.69899567428459e-09\\
-9.591578125	1.67742987791908e-09\\
-9.571583984375	1.59990362101076e-09\\
-9.55158984375	1.60780916895588e-09\\
-9.531595703125	1.64540049731859e-09\\
-9.5116015625	1.66124879351631e-09\\
-9.491607421875	1.67034858086462e-09\\
-9.47161328125	1.83020696692356e-09\\
-9.451619140625	1.9070803847715e-09\\
-9.431625	1.82304360464053e-09\\
-9.411630859375	2.01953365309392e-09\\
-9.39163671875	1.90804262604076e-09\\
-9.371642578125	1.8892475990896e-09\\
-9.3516484375	1.87714420183594e-09\\
-9.331654296875	1.84125606617625e-09\\
-9.31166015625	1.89959922156222e-09\\
-9.291666015625	1.88413904694208e-09\\
-9.271671875	1.82399739412262e-09\\
-9.251677734375	1.88933889306387e-09\\
-9.23168359375	2.04808838215423e-09\\
-9.211689453125	1.97216386306027e-09\\
-9.1916953125	1.89087666086951e-09\\
-9.171701171875	2.08011613149141e-09\\
-9.15170703125	1.84638861332361e-09\\
-9.131712890625	1.778081327585e-09\\
-9.11171875	1.52683731240732e-09\\
-9.091724609375	1.66650625792405e-09\\
-9.07173046875	1.52705383583135e-09\\
-9.051736328125	1.49424741899654e-09\\
-9.0317421875	1.40454666920866e-09\\
-9.011748046875	1.48935134687763e-09\\
-8.99175390625	1.18956769652566e-09\\
-8.971759765625	1.24625305443259e-09\\
-8.951765625	1.05948645322618e-09\\
-8.931771484375	1.12429635525464e-09\\
-8.91177734375	9.72991166752004e-10\\
-8.891783203125	1.07117031853999e-09\\
-8.8717890625	1.01842874134396e-09\\
-8.851794921875	9.33413161145846e-10\\
-8.83180078125	9.06551205920604e-10\\
-8.811806640625	8.51007296941664e-10\\
-8.7918125	8.99073809918429e-10\\
-8.771818359375	7.90509897731167e-10\\
-8.75182421875	7.45689810273446e-10\\
-8.731830078125	7.63408704942743e-10\\
-8.7118359375	8.02641258802085e-10\\
-8.691841796875	8.08852268800093e-10\\
-8.67184765625	8.70051162585617e-10\\
-8.651853515625	8.58814335462263e-10\\
-8.631859375	9.42359461509826e-10\\
-8.611865234375	9.09066420994493e-10\\
-8.59187109375	8.07852007187029e-10\\
-8.571876953125	8.57003508926015e-10\\
-8.5518828125	8.21230279559409e-10\\
-8.531888671875	7.60839746281066e-10\\
-8.51189453125	6.99012041754413e-10\\
-8.491900390625	6.76343913038905e-10\\
-8.47190625	6.12518585385233e-10\\
-8.451912109375	6.72659088738004e-10\\
-8.43191796875	6.33264248123472e-10\\
-8.411923828125	6.68050731271259e-10\\
-8.3919296875	5.60372151474973e-10\\
-8.371935546875	4.65062431751053e-10\\
-8.35194140625	4.22088719611794e-10\\
-8.331947265625	3.21207385858034e-10\\
-8.311953125	2.29745997332396e-10\\
-8.291958984375	1.2682574551392e-10\\
-8.27196484375	3.32357378311725e-11\\
-8.251970703125	-7.98147861791574e-11\\
-8.2319765625	-2.26048983369435e-10\\
-8.211982421875	-2.55044374583463e-10\\
-8.19198828125	-4.9203172229799e-10\\
-8.171994140625	-5.42797538529652e-10\\
-8.152	-6.66474831815071e-10\\
-8.132005859375	-6.63500832330942e-10\\
-8.11201171875	-8.65081304785116e-10\\
-8.092017578125	-9.69474283603819e-10\\
-8.0720234375	-9.11754475807914e-10\\
-8.052029296875	-1.03068272291276e-09\\
-8.03203515625	-1.06242248176379e-09\\
-8.012041015625	-1.04235061473286e-09\\
-7.992046875	-9.54438623587131e-10\\
-7.972052734375	-1.00167052395691e-09\\
-7.95205859375	-9.32262592083746e-10\\
-7.932064453125	-9.81619195194859e-10\\
-7.9120703125	-1.00565009205182e-09\\
-7.892076171875	-9.17638535209953e-10\\
-7.87208203125	-1.18821400568657e-09\\
-7.852087890625	-1.0008418085231e-09\\
-7.83209375	-1.32309132415747e-09\\
-7.812099609375	-1.10267036676411e-09\\
-7.79210546875	-1.33025910414437e-09\\
-7.772111328125	-1.27427313795959e-09\\
-7.75211718750001	-1.31140164201943e-09\\
-7.732123046875	-1.14392052135076e-09\\
-7.71212890625	-1.17848331201645e-09\\
-7.692134765625	-1.10333036276272e-09\\
-7.672140625	-1.28270872717659e-09\\
-7.652146484375	-1.14704613720914e-09\\
-7.63215234375	-1.40076749901202e-09\\
-7.612158203125	-1.42282396948402e-09\\
-7.5921640625	-1.59212163242398e-09\\
-7.572169921875	-1.66339134044718e-09\\
-7.55217578125	-1.74059740132663e-09\\
-7.53218164062501	-1.8106847186988e-09\\
-7.5121875	-1.72472591355727e-09\\
-7.492193359375	-1.83276363905811e-09\\
-7.47219921875	-1.76953384364726e-09\\
-7.452205078125	-1.82591015630582e-09\\
-7.4322109375	-1.85132407844278e-09\\
-7.412216796875	-1.91125607888281e-09\\
-7.39222265625	-2.01615595443549e-09\\
-7.372228515625	-2.17215518478259e-09\\
-7.352234375	-2.32604287333138e-09\\
-7.332240234375	-2.35505785328992e-09\\
-7.31224609375001	-2.34462678981461e-09\\
-7.292251953125	-2.26441562311876e-09\\
-7.2722578125	-2.21395453287585e-09\\
-7.252263671875	-2.23707646755669e-09\\
-7.23226953125	-2.03654595010596e-09\\
-7.212275390625	-2.30722273641718e-09\\
-7.19228125	-2.27038960088735e-09\\
-7.172287109375	-2.31640535602218e-09\\
-7.15229296875	-2.44190674002099e-09\\
-7.132298828125	-2.50735454034356e-09\\
-7.1123046875	-2.44240116116779e-09\\
-7.092310546875	-2.48145919446274e-09\\
-7.07231640625	-2.46203998783967e-09\\
-7.052322265625	-2.42336474618969e-09\\
-7.032328125	-2.37226067533827e-09\\
-7.012333984375	-2.34181845655738e-09\\
-6.99233984375	-2.20525723505565e-09\\
-6.972345703125	-2.35912603076274e-09\\
-6.9523515625	-2.37504777035543e-09\\
-6.932357421875	-2.26295957257476e-09\\
-6.91236328125	-2.45719082360718e-09\\
-6.892369140625	-2.37708234296371e-09\\
-6.872375	-2.41705365696827e-09\\
-6.852380859375	-2.43437192912325e-09\\
-6.83238671875	-2.40831992681638e-09\\
-6.812392578125	-2.40200773375451e-09\\
-6.7923984375	-2.37700611843204e-09\\
-6.772404296875	-2.31647489012394e-09\\
-6.75241015625	-2.38143033425408e-09\\
-6.732416015625	-2.47028580876311e-09\\
-6.712421875	-2.47498877344622e-09\\
-6.692427734375	-2.37754998629578e-09\\
-6.67243359375	-2.5772278032016e-09\\
-6.652439453125	-2.44059187092413e-09\\
-6.6324453125	-2.44318697638197e-09\\
-6.612451171875	-2.32581307807036e-09\\
-6.59245703125	-2.46015216514229e-09\\
-6.572462890625	-2.26922702324954e-09\\
-6.55246875	-2.30330739399985e-09\\
-6.532474609375	-2.13364956414915e-09\\
-6.51248046875	-2.20392970843039e-09\\
-6.492486328125	-2.03650020400484e-09\\
-6.4724921875	-1.99782872411752e-09\\
-6.452498046875	-1.95936133949713e-09\\
-6.43250390625	-2.06647757792668e-09\\
-6.412509765625	-1.91119859884883e-09\\
-6.392515625	-1.98748360661927e-09\\
-6.372521484375	-1.920343095456e-09\\
-6.35252734375	-1.86498776446459e-09\\
-6.332533203125	-1.88529372980305e-09\\
-6.3125390625	-1.71386121824473e-09\\
-6.292544921875	-1.72098934935217e-09\\
-6.27255078125	-1.50880789809152e-09\\
-6.252556640625	-1.51315416284168e-09\\
-6.2325625	-1.53850094049563e-09\\
-6.212568359375	-1.63985712568791e-09\\
-6.19257421875	-1.71235270018202e-09\\
-6.172580078125	-1.75534617577898e-09\\
-6.1525859375	-1.85202310930233e-09\\
-6.132591796875	-2.01102849655328e-09\\
-6.11259765625	-1.86256613779884e-09\\
-6.092603515625	-1.78209940296285e-09\\
-6.072609375	-1.77374010053444e-09\\
-6.052615234375	-1.67280765554009e-09\\
-6.03262109375	-1.63877300919494e-09\\
-6.012626953125	-1.50237905178733e-09\\
-5.9926328125	-1.65589528610496e-09\\
-5.972638671875	-1.68924892286237e-09\\
-5.95264453125	-1.80309121720548e-09\\
-5.932650390625	-1.68426643622714e-09\\
-5.91265625	-1.69514528817794e-09\\
-5.892662109375	-1.57046086304586e-09\\
-5.87266796875	-1.51306056181143e-09\\
-5.852673828125	-1.35148168205268e-09\\
-5.8326796875	-1.17144138596614e-09\\
-5.812685546875	-1.0721029566084e-09\\
-5.79269140625	-1.04137090021204e-09\\
-5.772697265625	-9.20369650360886e-10\\
-5.752703125	-8.29851544682563e-10\\
-5.732708984375	-7.90894531057839e-10\\
-5.71271484375	-7.29928237581571e-10\\
-5.692720703125	-5.24212013144936e-10\\
-5.6727265625	-4.36999434761845e-10\\
-5.652732421875	-3.52390673085975e-10\\
-5.63273828125	-2.86099883279153e-10\\
-5.612744140625	-1.59246259845907e-10\\
-5.59275	-1.13185798151966e-11\\
-5.572755859375	1.13979725627661e-11\\
-5.55276171875	1.94103365983707e-10\\
-5.532767578125	2.1384881306326e-10\\
-5.5127734375	3.41633514694243e-10\\
-5.492779296875	2.92022575724125e-10\\
-5.47278515625	4.58538619512067e-10\\
-5.452791015625	3.01392909285612e-10\\
-5.432796875	3.7794552243502e-10\\
-5.412802734375	4.52107438043135e-10\\
-5.39280859375	6.49789222521531e-10\\
-5.372814453125	8.66053346161909e-10\\
-5.3528203125	8.46254135581249e-10\\
-5.332826171875	1.27414444551474e-09\\
-5.31283203125	1.19850801521719e-09\\
-5.292837890625	1.51093022594823e-09\\
-5.27284375	1.37201853512653e-09\\
-5.25284960937501	1.56061878295155e-09\\
-5.23285546875	1.27943913983899e-09\\
-5.212861328125	1.29996773328256e-09\\
-5.1928671875	1.28960256773031e-09\\
-5.172873046875	1.34613470091983e-09\\
-5.15287890625	1.40900017058099e-09\\
-5.132884765625	1.72879933178566e-09\\
-5.112890625	1.82347856143954e-09\\
-5.092896484375	2.08735917247099e-09\\
-5.07290234375	2.20784864415368e-09\\
-5.052908203125	2.36487739691896e-09\\
-5.03291406250001	2.35662344466904e-09\\
-5.012919921875	2.35183227623078e-09\\
-4.99292578125	2.46227808671215e-09\\
-4.972931640625	2.2729192706697e-09\\
-4.9529375	2.40197731194952e-09\\
-4.932943359375	2.49836701343963e-09\\
-4.91294921875	2.6132555205453e-09\\
-4.892955078125	2.88513163596387e-09\\
-4.8729609375	2.94679541035709e-09\\
-4.852966796875	3.1564397638466e-09\\
-4.83297265625	3.06987983914951e-09\\
-4.81297851562501	3.15670305560401e-09\\
-4.792984375	3.10927873069521e-09\\
-4.772990234375	3.01396647263866e-09\\
-4.75299609375	2.95209752810954e-09\\
-4.733001953125	3.08130371797518e-09\\
-4.7130078125	3.17206811784837e-09\\
-4.693013671875	3.23256521994844e-09\\
-4.67301953125	3.42845979368117e-09\\
-4.653025390625	3.48110884650081e-09\\
-4.63303125	3.52232115382226e-09\\
-4.613037109375	3.32110654921428e-09\\
-4.59304296875	3.2729614480571e-09\\
-4.573048828125	3.25407394273058e-09\\
-4.5530546875	3.03049430010556e-09\\
-4.533060546875	2.91458775165666e-09\\
-4.51306640625	2.99192533248384e-09\\
-4.493072265625	2.91533339684382e-09\\
-4.473078125	3.11267893572802e-09\\
-4.453083984375	3.23593556451105e-09\\
-4.43308984375	3.24642681357876e-09\\
-4.413095703125	3.32583278428702e-09\\
-4.3931015625	3.15477824248811e-09\\
-4.373107421875	3.28734831239726e-09\\
-4.35311328125	3.20689123597545e-09\\
-4.333119140625	3.09630597692205e-09\\
-4.313125	3.05222713730452e-09\\
-4.293130859375	3.02886546502651e-09\\
-4.27313671875	2.87342980884461e-09\\
-4.253142578125	2.92946139509461e-09\\
-4.2331484375	2.90173274464202e-09\\
-4.213154296875	2.91400420217341e-09\\
-4.19316015625	2.86149725066731e-09\\
-4.173166015625	2.94478163677779e-09\\
-4.153171875	2.83266489201981e-09\\
-4.133177734375	2.82251747676098e-09\\
-4.11318359375	2.71290408352738e-09\\
-4.093189453125	2.787762854041e-09\\
-4.0731953125	2.65910729463146e-09\\
-4.053201171875	2.75619220767317e-09\\
-4.03320703125	2.58327088256554e-09\\
-4.013212890625	2.49263706291719e-09\\
-3.99321875	2.37252174668563e-09\\
-3.973224609375	2.31954528280159e-09\\
-3.95323046875	2.3160743135134e-09\\
-3.933236328125	2.26176700640764e-09\\
-3.9132421875	2.13546061355088e-09\\
-3.893248046875	2.21714032550857e-09\\
-3.87325390625	2.14289432042901e-09\\
-3.853259765625	2.17979519695051e-09\\
-3.833265625	2.21179821830173e-09\\
-3.813271484375	2.12258113473675e-09\\
-3.79327734375	1.97417623303533e-09\\
-3.773283203125	1.77699978046381e-09\\
-3.7532890625	1.84740403967111e-09\\
-3.733294921875	1.75511523827255e-09\\
-3.71330078125	1.73715695951154e-09\\
-3.693306640625	1.86047623236563e-09\\
-3.6733125	1.81127983964593e-09\\
-3.653318359375	1.84318437472825e-09\\
-3.63332421875	1.84930998177028e-09\\
-3.613330078125	1.86619518813006e-09\\
-3.5933359375	1.7598881573967e-09\\
-3.573341796875	1.58430073178165e-09\\
-3.55334765625	1.46869183566642e-09\\
-3.533353515625	1.48043413308367e-09\\
-3.513359375	1.32976409617452e-09\\
-3.493365234375	1.36337549297839e-09\\
-3.47337109375	1.39341207045409e-09\\
-3.453376953125	1.41688598514133e-09\\
-3.4333828125	1.3758555099551e-09\\
-3.413388671875	1.50580446454766e-09\\
-3.39339453125	1.52842826453586e-09\\
-3.373400390625	1.46437846053814e-09\\
-3.35340625	1.31784810991767e-09\\
-3.333412109375	1.23265632609417e-09\\
-3.31341796875	1.08163322891167e-09\\
-3.293423828125	9.53876104089893e-10\\
-3.2734296875	8.01038456714794e-10\\
-3.253435546875	7.37981599660848e-10\\
-3.23344140625	6.33215585376223e-10\\
-3.213447265625	5.77078004529711e-10\\
-3.193453125	4.17600231023627e-10\\
-3.173458984375	4.77761308400579e-10\\
-3.15346484375	4.56477663672484e-10\\
-3.133470703125	4.44346954360297e-10\\
-3.1134765625	4.38254259069945e-10\\
-3.093482421875	5.11868549700338e-10\\
-3.07348828125	5.29921575280523e-10\\
-3.053494140625	3.90700823782495e-10\\
-3.0335	4.62446073652264e-10\\
-3.013505859375	2.79971286379457e-10\\
-2.99351171875	2.22708475167728e-10\\
-2.973517578125	1.39898931645043e-10\\
-2.9535234375	2.29446155324716e-10\\
-2.933529296875	1.77655604444381e-10\\
-2.91353515625	1.75638617008882e-10\\
-2.893541015625	1.0017871059171e-10\\
-2.873546875	9.73431906765097e-11\\
-2.853552734375	-1.16577119782135e-13\\
-2.83355859375	-2.59809342302112e-10\\
-2.813564453125	-1.99676924039327e-11\\
-2.7935703125	-5.15823020198296e-10\\
-2.773576171875	-1.99250542540037e-10\\
-2.75358203125001	-4.71737860049301e-10\\
-2.733587890625	-3.57823871249691e-10\\
-2.71359375	-4.36523937728074e-10\\
-2.693599609375	-3.02588341269143e-10\\
-2.67360546875	-3.7307975455175e-10\\
-2.653611328125	-4.10978081126457e-10\\
-2.6336171875	-5.4200614047103e-10\\
-2.613623046875	-6.29583969804047e-10\\
-2.59362890625	-8.94858606958847e-10\\
-2.573634765625	-1.13379132423565e-09\\
-2.553640625	-1.33545584632971e-09\\
-2.53364648437501	-1.31001798513565e-09\\
-2.51365234375	-1.38929949837817e-09\\
-2.493658203125	-1.35169054406381e-09\\
-2.4736640625	-1.09435542755634e-09\\
-2.453669921875	-1.24349947495773e-09\\
-2.43367578125	-1.32155658544903e-09\\
-2.413681640625	-1.29339916742946e-09\\
-2.3936875	-1.48447663223503e-09\\
-2.373693359375	-1.5763822953682e-09\\
-2.35369921875	-1.78903276366068e-09\\
-2.333705078125	-1.77615729809415e-09\\
-2.3137109375	-1.85450355461305e-09\\
-2.293716796875	-1.72415827215561e-09\\
-2.27372265625	-1.62525850580749e-09\\
-2.253728515625	-1.61125742508234e-09\\
-2.233734375	-1.65168878548325e-09\\
-2.213740234375	-1.61589861990969e-09\\
-2.19374609375	-1.62490630521815e-09\\
-2.173751953125	-1.84655300323615e-09\\
-2.1537578125	-1.78339385110769e-09\\
-2.133763671875	-1.77638364894894e-09\\
-2.11376953125	-1.68829517431467e-09\\
-2.093775390625	-1.67970223793051e-09\\
-2.07378125	-1.66420145660538e-09\\
-2.053787109375	-1.58760908581915e-09\\
-2.03379296875	-1.46995838937878e-09\\
-2.013798828125	-1.71215574806623e-09\\
-1.9938046875	-1.65605345852421e-09\\
-1.973810546875	-1.61645490043383e-09\\
-1.95381640625	-1.68715873358743e-09\\
-1.933822265625	-1.51483963202268e-09\\
-1.913828125	-1.56860821429036e-09\\
-1.893833984375	-1.49514488875404e-09\\
-1.87383984375	-1.83633239019995e-09\\
-1.853845703125	-1.74340620065505e-09\\
-1.8338515625	-1.77002679587719e-09\\
-1.813857421875	-1.85462528039695e-09\\
-1.79386328125	-1.87385154912742e-09\\
-1.773869140625	-1.99688787643132e-09\\
-1.753875	-2.06825980107713e-09\\
-1.733880859375	-1.92565651800773e-09\\
-1.71388671875	-1.94509671674861e-09\\
-1.693892578125	-1.91780465393362e-09\\
-1.6738984375	-1.90071381547304e-09\\
-1.653904296875	-1.82047325614138e-09\\
-1.63391015625	-1.85670736725681e-09\\
-1.613916015625	-1.78410379163241e-09\\
-1.593921875	-1.82568483699151e-09\\
-1.573927734375	-1.71661704793459e-09\\
-1.55393359375	-1.89334172246569e-09\\
-1.533939453125	-1.93139504027013e-09\\
-1.5139453125	-1.79267476357905e-09\\
-1.493951171875	-1.90301264841112e-09\\
-1.47395703125	-1.82563391910799e-09\\
-1.453962890625	-1.84716555661532e-09\\
-1.43396875	-1.67561272790702e-09\\
-1.413974609375	-1.71164538500061e-09\\
-1.39398046875	-1.66617819405685e-09\\
-1.373986328125	-1.5279806060108e-09\\
-1.3539921875	-1.57786818819491e-09\\
-1.333998046875	-1.5295390609162e-09\\
-1.31400390625	-1.45142710833763e-09\\
-1.294009765625	-1.37688311332406e-09\\
-1.274015625	-1.3889692580479e-09\\
-1.254021484375	-1.51819010250428e-09\\
-1.23402734375	-1.5366291812932e-09\\
-1.214033203125	-1.44814864605419e-09\\
-1.1940390625	-1.6719892208418e-09\\
-1.174044921875	-1.58240786466707e-09\\
-1.15405078125	-1.58838350607396e-09\\
-1.134056640625	-1.48136461991714e-09\\
-1.1140625	-1.60421845769247e-09\\
-1.094068359375	-1.39035354991747e-09\\
-1.07407421875	-1.29787102478688e-09\\
-1.054080078125	-1.30934303022681e-09\\
-1.0340859375	-1.22325843177002e-09\\
-1.014091796875	-1.2114978068711e-09\\
-0.994097656249998	-1.18920802673929e-09\\
-0.974103515625004	-1.14849960480663e-09\\
-0.954109375000002	-1.29689432700449e-09\\
-0.934115234375	-1.2525700230769e-09\\
-0.914121093750005	-1.43530045526589e-09\\
-0.894126953125003	-1.3580739330184e-09\\
-0.874132812500001	-1.33037029216357e-09\\
-0.854138671874999	-1.24908420239413e-09\\
-0.834144531250004	-1.18565289140418e-09\\
-0.814150390625002	-9.65307571865525e-10\\
-0.79415625	-7.41544440257986e-10\\
-0.774162109374998	-6.36496882567609e-10\\
-0.754167968750004	-5.56699220078241e-10\\
-0.734173828125002	-3.18553259709848e-10\\
-0.7141796875	-2.29070681670669e-10\\
-0.694185546875005	-1.28855632201547e-10\\
-0.674191406250003	-7.11845666260172e-11\\
-0.654197265625001	9.09325916633877e-11\\
-0.634203124999999	4.97365467785861e-12\\
-0.614208984375004	-1.647435315345e-10\\
-0.594214843750002	-1.71995277137956e-10\\
-0.574220703125	-9.71044258561382e-11\\
-0.554226562499998	-1.33973870038834e-10\\
-0.534232421875004	-2.56420909340857e-11\\
-0.514238281250002	1.97673070623936e-10\\
-0.494244140625	3.36955006228685e-10\\
-0.474250000000005	5.23054070240691e-10\\
-0.454255859375003	5.65934124198206e-10\\
-0.434261718750001	7.04825951478714e-10\\
-0.414267578124999	6.8525831017746e-10\\
-0.394273437500004	6.44611489076195e-10\\
-0.374279296875002	8.20040154932022e-10\\
-0.35428515625	8.52503961846803e-10\\
-0.334291015624999	1.08951178969312e-09\\
-0.314296875000004	1.02475818305233e-09\\
-0.294302734375002	1.4771009712588e-09\\
-0.27430859375	1.31063967517842e-09\\
-0.254314453125005	1.61400439224291e-09\\
-0.234320312500003	1.43331464219694e-09\\
-0.214326171875001	1.55797338187249e-09\\
-0.194332031249999	1.42282988319397e-09\\
-0.174337890625004	1.63516753356147e-09\\
-0.154343750000002	1.78419688482473e-09\\
-0.134349609375001	1.96395891452989e-09\\
-0.114355468749999	2.1353505292129e-09\\
-0.0943613281250038	2.38336280203191e-09\\
-0.0743671875000018	2.63626630685588e-09\\
-0.0543730468749999	2.8572672806023e-09\\
-0.0343789062500051	3.00833497180488e-09\\
-0.0143847656250031	2.97817974380631e-09\\
0.00560937499999881	3.01176635776514e-09\\
0.0256035156250007	2.96064965288453e-09\\
0.0455976562499956	3.05424712210127e-09\\
0.0655917968749975	3.07841641923736e-09\\
0.0855859374999994	3.00575838318436e-09\\
0.105580078125001	3.32523566307239e-09\\
0.125574218749996	3.32694945770131e-09\\
0.145568359374998	3.52001615685182e-09\\
0.1655625	3.53477912574644e-09\\
0.185556640625002	3.97406414593265e-09\\
0.205550781249997	3.93273375728753e-09\\
0.225544921874999	4.03486182525824e-09\\
0.245539062500001	4.08559779286234e-09\\
0.265533203124996	4.11591959282296e-09\\
0.285527343749997	4.18217862718127e-09\\
0.305521484374999	4.18341679471968e-09\\
0.325515625000001	4.33908092143104e-09\\
0.345509765624996	4.2809187928919e-09\\
0.365503906249998	4.36246980481095e-09\\
0.385498046875	4.33238497124844e-09\\
0.405492187500002	4.5345255319201e-09\\
0.425486328124997	4.50501351278548e-09\\
0.445480468749999	4.73401853529727e-09\\
0.465474609375001	4.72471024841569e-09\\
0.485468749999995	4.85143401702039e-09\\
0.505462890624997	4.79618214839651e-09\\
0.525457031249999	4.73229501445863e-09\\
0.545451171875001	4.7848579631603e-09\\
0.565445312499996	4.66695250491337e-09\\
0.585439453124998	4.86431681507119e-09\\
0.60543359375	4.89097248624585e-09\\
0.625427734375002	5.23589036415217e-09\\
0.645421874999997	5.21969455470335e-09\\
0.665416015624999	5.63592095771524e-09\\
0.685410156250001	5.65842195895401e-09\\
0.705404296874995	5.70738957490503e-09\\
0.725398437499997	5.70655377027375e-09\\
0.745392578124999	5.81879385480999e-09\\
0.765386718750001	5.79219919200107e-09\\
0.785380859374996	5.85981100831048e-09\\
0.805374999999998	5.99826027608931e-09\\
0.825369140625	6.05178219603478e-09\\
0.845363281250002	6.28173901548362e-09\\
0.865357421874997	6.35305794661821e-09\\
0.885351562499999	6.48997334605177e-09\\
0.905345703125001	6.60966368395828e-09\\
0.925339843749995	6.44609111044729e-09\\
0.945333984374997	6.57592486367467e-09\\
0.965328124999999	6.55799819124777e-09\\
0.985322265625001	6.50481930833035e-09\\
1.00531640625	6.83133336390645e-09\\
1.025310546875	6.78058703387437e-09\\
1.0453046875	6.91484729359168e-09\\
1.065298828125	6.99855765834363e-09\\
1.08529296875	7.18579985112338e-09\\
1.105287109375	7.32987793523488e-09\\
1.12528125	7.31016609361748e-09\\
1.145275390625	7.42270975524962e-09\\
1.16526953125	7.33230490097392e-09\\
1.185263671875	7.33890170897943e-09\\
1.2052578125	7.27516804406004e-09\\
1.225251953125	7.26771543388481e-09\\
1.24524609375	7.36951280958578e-09\\
1.265240234375	7.39804267221338e-09\\
1.285234375	7.5652747072916e-09\\
1.305228515625	7.77405921297863e-09\\
1.32522265625	7.83890765650069e-09\\
1.345216796875	7.97218212592892e-09\\
1.3652109375	7.93523379070835e-09\\
1.385205078125	8.17431020376346e-09\\
1.40519921875	8.11586606025821e-09\\
1.425193359375	8.18567977236227e-09\\
1.4451875	8.28382560876331e-09\\
1.465181640625	8.08314862334759e-09\\
1.48517578125	8.27423536877177e-09\\
1.505169921875	8.26464365710352e-09\\
1.5251640625	8.30541864796059e-09\\
1.545158203125	8.47061814987612e-09\\
1.56515234375	8.50790626014209e-09\\
1.585146484375	8.65696554160399e-09\\
1.605140625	8.56936774323031e-09\\
1.625134765625	8.50304665186815e-09\\
1.64512890625	8.54326071669137e-09\\
1.665123046875	8.67412051157909e-09\\
1.6851171875	8.56890301708509e-09\\
1.705111328125	8.50199109540827e-09\\
1.72510546875	8.66800379063613e-09\\
1.745099609375	8.60816524889882e-09\\
1.76509375	8.51421016771724e-09\\
1.785087890625	8.49435056356986e-09\\
1.80508203125	8.5156285611924e-09\\
1.825076171875	8.39290998371503e-09\\
1.8450703125	8.36654530441618e-09\\
1.865064453125	8.47753173811206e-09\\
1.88505859375	8.38455650822145e-09\\
1.905052734375	8.48352520222414e-09\\
1.925046875	8.51299222560088e-09\\
1.945041015625	8.6215298717579e-09\\
1.96503515625	8.63467848936375e-09\\
1.985029296875	8.53578162578191e-09\\
2.0050234375	8.58188371426648e-09\\
2.025017578125	8.56239118788708e-09\\
2.04501171875	8.59588661068036e-09\\
2.065005859375	8.63863344989269e-09\\
2.085	8.78214743124395e-09\\
2.104994140625	8.68570798820437e-09\\
2.12498828125	8.54956986330731e-09\\
2.144982421875	8.66958949948049e-09\\
2.1649765625	8.50515514823006e-09\\
2.184970703125	8.508578010063e-09\\
2.20496484375	8.29520221656864e-09\\
2.224958984375	8.40915550798723e-09\\
2.244953125	8.31784017754673e-09\\
2.264947265625	8.47063766623719e-09\\
2.28494140625	8.25392386546394e-09\\
2.304935546875	8.4985857665626e-09\\
2.3249296875	8.10592915190369e-09\\
2.344923828125	8.11063141995379e-09\\
2.36491796875	7.92233037623639e-09\\
2.384912109375	7.69067365361849e-09\\
2.40490625	7.46910992509747e-09\\
2.424900390625	7.28450457337918e-09\\
2.44489453125	7.31927647472161e-09\\
2.464888671875	7.15503977140049e-09\\
2.4848828125	7.25632133726234e-09\\
2.504876953125	7.21819570336726e-09\\
2.52487109375	7.4369659617645e-09\\
2.544865234375	7.25697573723964e-09\\
2.564859375	7.29516308420513e-09\\
2.584853515625	7.27858128619905e-09\\
2.60484765625	7.1463739064364e-09\\
2.624841796875	6.99063973799166e-09\\
2.6448359375	6.98469668501253e-09\\
2.664830078125	6.84250111833315e-09\\
2.68482421875	6.68497300343687e-09\\
2.704818359375	6.77825810418337e-09\\
2.7248125	6.6261511465752e-09\\
2.744806640625	6.7658804497881e-09\\
2.76480078125	6.75369561232081e-09\\
2.784794921875	6.51551241178478e-09\\
2.8047890625	6.52677380329349e-09\\
2.824783203125	6.4148350410217e-09\\
2.84477734375	6.42786171530394e-09\\
2.864771484375	6.36908374848745e-09\\
2.884765625	6.30778524508441e-09\\
2.904759765625	6.21738912648365e-09\\
2.92475390625	6.33308604993886e-09\\
2.944748046875	6.08425261424321e-09\\
2.9647421875	6.29527575602524e-09\\
2.984736328125	6.26942103080538e-09\\
3.00473046875	6.28714848466713e-09\\
3.024724609375	6.27025558063031e-09\\
3.04471875	6.17300696247894e-09\\
3.064712890625	6.16066892235164e-09\\
3.08470703125	5.95976636920968e-09\\
3.104701171875	5.80757178762798e-09\\
3.1246953125	5.54287999705932e-09\\
3.144689453125	5.59521255011879e-09\\
3.16468359375	5.29786455403574e-09\\
3.184677734375	5.49993873697606e-09\\
3.204671875	5.61625715715007e-09\\
3.224666015625	5.64610090778554e-09\\
3.24466015625	5.68278350352445e-09\\
3.264654296875	5.77307429487314e-09\\
3.2846484375	5.55343619662995e-09\\
3.304642578125	5.32474357018525e-09\\
3.32463671875	5.05646988626171e-09\\
3.344630859375	4.74790001354579e-09\\
3.364625	4.63934095544614e-09\\
3.384619140625	4.41477467166602e-09\\
3.40461328125	4.61093813016468e-09\\
3.424607421875	4.73110041568583e-09\\
3.4446015625	4.75128350218794e-09\\
3.464595703125	4.88015739670886e-09\\
3.48458984375	4.89261153114672e-09\\
3.504583984375	4.58918818571584e-09\\
3.524578125	4.54316124294014e-09\\
3.544572265625	4.06197186167692e-09\\
3.56456640625	3.78145370592615e-09\\
3.584560546875	3.61327938758611e-09\\
3.6045546875	3.43172655861527e-09\\
3.624548828125	3.42761060268624e-09\\
3.64454296875	3.37332216663216e-09\\
3.664537109375	3.42369037142862e-09\\
3.68453125	3.52276619553821e-09\\
3.704525390625	3.50877756530048e-09\\
3.72451953125	3.42797573773184e-09\\
3.744513671875	3.36976030807575e-09\\
3.7645078125	3.28880433261172e-09\\
3.784501953125	3.05377754790947e-09\\
3.80449609375	2.76465187667835e-09\\
3.824490234375	2.62597241221388e-09\\
3.844484375	2.42652643589479e-09\\
3.864478515625	2.38058674306705e-09\\
3.88447265625	2.31403722293358e-09\\
3.904466796875	2.40663843576712e-09\\
3.9244609375	2.26163007861647e-09\\
3.944455078125	2.3745689274662e-09\\
3.96444921875	2.46522515321317e-09\\
3.984443359375	2.36526905976122e-09\\
4.0044375	2.3394560750006e-09\\
4.024431640625	2.31237832204446e-09\\
4.04442578125	2.04696174473751e-09\\
4.064419921875	1.92751074526605e-09\\
4.0844140625	1.74990608632764e-09\\
4.104408203125	1.73647110459799e-09\\
4.12440234375	1.83482828382412e-09\\
4.144396484375	1.76425331470654e-09\\
4.164390625	1.72269375779863e-09\\
4.184384765625	1.97328828620755e-09\\
4.20437890625	1.79246472515617e-09\\
4.224373046875	1.83324454947564e-09\\
4.2443671875	1.83177401127222e-09\\
4.264361328125	1.78439477938324e-09\\
4.28435546875	1.59321298412466e-09\\
4.304349609375	1.54852587452787e-09\\
4.32434375	1.47791849978817e-09\\
4.344337890625	1.33297566095001e-09\\
4.36433203125	1.30662054711552e-09\\
4.384326171875	1.3159634251121e-09\\
4.4043203125	1.3320859452508e-09\\
4.424314453125	1.25057552460578e-09\\
4.44430859375	1.32570436993336e-09\\
4.464302734375	1.34867606240299e-09\\
4.484296875	1.19968219521869e-09\\
4.504291015625	1.08409835666976e-09\\
4.52428515625	9.72313524362141e-10\\
4.544279296875	8.83348823870526e-10\\
4.5642734375	7.06785971014597e-10\\
4.584267578125	5.31480117415024e-10\\
4.60426171875	5.1696347232801e-10\\
4.624255859375	6.45702254702439e-10\\
4.64425	5.16354836901069e-10\\
4.664244140625	6.13954474091524e-10\\
4.68423828125	6.22361013695737e-10\\
4.704232421875	7.57908875817557e-10\\
4.7242265625	5.07401283900847e-10\\
4.744220703125	6.31942109145735e-10\\
4.76421484375	3.36667388509363e-10\\
4.784208984375	4.65213339250033e-10\\
4.804203125	1.06850207437479e-10\\
4.824197265625	2.22181955979874e-10\\
4.84419140625	2.16472161859661e-10\\
4.864185546875	3.20953902353743e-10\\
4.8841796875	3.2221019251199e-10\\
4.904173828125	2.97303571770928e-10\\
4.92416796875	3.60423980132312e-10\\
4.944162109375	3.18132494833249e-10\\
4.96415625	5.593052751789e-10\\
4.984150390625	4.09994646270021e-10\\
5.00414453125	5.89681099627005e-10\\
5.024138671875	4.89180133099639e-10\\
5.0441328125	4.16417382342124e-10\\
5.064126953125	2.65725597602058e-10\\
5.08412109375	1.56120403404205e-10\\
5.104115234375	1.27144065830257e-11\\
5.124109375	-6.78123710159194e-11\\
5.144103515625	-8.32347585463697e-11\\
5.16409765625	-1.34335357301134e-11\\
5.184091796875	2.12394978388306e-11\\
5.2040859375	2.17940358842981e-11\\
5.224080078125	1.08915844933366e-10\\
5.24407421875	1.13617459092708e-10\\
5.264068359375	-2.94057230793085e-11\\
5.2840625	-9.68253355372961e-12\\
5.304056640625	-9.96470757470002e-11\\
5.32405078125	2.83791223562129e-11\\
5.344044921875	-2.24368190016802e-10\\
5.3640390625	-1.8595775219559e-11\\
5.384033203125	-4.6181164735369e-11\\
5.40402734375	-8.74180475754726e-11\\
5.424021484375	-1.16290293964589e-10\\
5.444015625	9.34424711882407e-11\\
5.464009765625	-1.75703714138018e-10\\
5.48400390625	-7.44437193920735e-11\\
5.503998046875	-3.56822026474074e-10\\
5.5239921875	-3.05948579408841e-10\\
5.543986328125	-4.87224654559782e-10\\
5.56398046875	-5.49272490057422e-10\\
5.583974609375	-4.58620790602738e-10\\
5.60396875	-5.07794741591959e-10\\
5.623962890625	-3.88791154312776e-10\\
5.64395703125	-4.02919166157353e-10\\
5.663951171875	-3.4758761479741e-10\\
5.6839453125	-4.09102181213038e-10\\
5.703939453125	-5.70300760418054e-10\\
5.72393359375	-6.8241045993445e-10\\
5.743927734375	-8.25711406265661e-10\\
5.763921875	-8.72700097939087e-10\\
5.783916015625	-9.06102225279948e-10\\
5.80391015625	-9.6767070640168e-10\\
5.823904296875	-7.95932026334716e-10\\
5.8438984375	-6.95907975785173e-10\\
5.863892578125	-6.75920413027742e-10\\
5.88388671875	-5.87322893857931e-10\\
5.903880859375	-6.48892795517988e-10\\
5.923875	-6.72334914090539e-10\\
5.943869140625	-7.45198562200654e-10\\
5.96386328125	-7.95699811798504e-10\\
5.983857421875	-8.24478898893438e-10\\
6.0038515625	-8.91452734631571e-10\\
6.023845703125	-1.01473895071676e-09\\
6.04383984375	-7.02062372675025e-10\\
6.063833984375	-8.19843473607953e-10\\
6.083828125	-6.26543045820623e-10\\
6.103822265625	-5.8234297644784e-10\\
6.12381640625	-5.26185030549528e-10\\
6.143810546875	-5.96418595761481e-10\\
6.1638046875	-6.33324560062684e-10\\
6.183798828125	-7.61227452447464e-10\\
6.20379296875	-8.7474549692074e-10\\
6.223787109375	-8.92177441870418e-10\\
6.24378125	-8.82141438105909e-10\\
6.263775390625	-7.68397898862214e-10\\
6.28376953125	-7.50197222212509e-10\\
6.303763671875	-5.10174831698928e-10\\
6.3237578125	-4.12319253639305e-10\\
6.343751953125	-3.38031945751209e-10\\
6.36374609375	-2.7804652299903e-10\\
6.383740234375	-2.78710444334868e-10\\
6.403734375	-3.04285780284165e-10\\
6.423728515625	-3.98192972177507e-10\\
6.44372265625	-5.13547121355646e-10\\
6.463716796875	-6.58898099562478e-10\\
6.4837109375	-6.15770749161537e-10\\
6.503705078125	-7.21908284772734e-10\\
6.52369921875	-6.71084382351606e-10\\
6.543693359375	-5.53907740009293e-10\\
6.5636875	-6.18349286795557e-10\\
6.583681640625	-4.43772874319929e-10\\
6.60367578125	-4.66503840044168e-10\\
6.623669921875	-5.36176352250234e-10\\
6.6436640625	-5.50871915011876e-10\\
6.663658203125	-7.31034502535368e-10\\
6.68365234375	-9.81205195983074e-10\\
6.703646484375	-1.00305964620863e-09\\
6.723640625	-1.16800678311318e-09\\
6.743634765625	-1.2077432590235e-09\\
6.76362890625	-1.10414148725165e-09\\
6.783623046875	-1.04035241184062e-09\\
6.8036171875	-1.10066464891266e-09\\
6.823611328125	-9.33528816832026e-10\\
6.84360546875	-1.03925879291684e-09\\
6.863599609375	-1.00664513918389e-09\\
6.88359375	-1.1535746835732e-09\\
6.903587890625	-1.13188072291328e-09\\
6.92358203125	-1.23885222708475e-09\\
6.943576171875	-1.25634036876971e-09\\
6.9635703125	-1.29555387287696e-09\\
6.983564453125	-1.16419476641561e-09\\
7.00355859375	-1.16584704258243e-09\\
7.023552734375	-1.11732514786215e-09\\
7.043546875	-1.05079263023908e-09\\
7.063541015625	-9.34659469572533e-10\\
7.08353515625	-8.24746437293904e-10\\
7.103529296875	-6.86984553054923e-10\\
7.1235234375	-7.1633890172681e-10\\
7.143517578125	-7.622119937066e-10\\
7.16351171875	-8.45473625523174e-10\\
7.183505859375	-8.48494789519175e-10\\
7.2035	-9.13204747529994e-10\\
7.223494140625	-7.26063239456401e-10\\
7.24348828125	-8.77396797870207e-10\\
7.263482421875	-7.2074480698679e-10\\
7.2834765625	-8.37532052717011e-10\\
7.303470703125	-6.27444992480933e-10\\
7.32346484375	-5.82521386458821e-10\\
7.343458984375	-5.50504640406064e-10\\
7.363453125	-5.19032231217021e-10\\
7.383447265625	-5.17741205247511e-10\\
7.40344140625	-5.7508007868948e-10\\
7.423435546875	-7.22860794480742e-10\\
7.4434296875	-6.99409567044089e-10\\
7.463423828125	-7.96107396958832e-10\\
7.48341796875	-6.15355682733045e-10\\
7.503412109375	-7.82523527665709e-10\\
7.52340625	-6.91902076518022e-10\\
7.543400390625	-4.73364055925901e-10\\
7.56339453125	-5.56024754945483e-10\\
7.583388671875	-4.46830473451181e-10\\
7.6033828125	-4.62399562309305e-10\\
7.623376953125	-3.30092337362134e-10\\
7.64337109375	-5.36558761781944e-10\\
7.663365234375	-5.32176283019393e-10\\
7.683359375	-4.34107502834457e-10\\
7.703353515625	-6.10020202000709e-10\\
7.72334765625	-6.33180355055849e-10\\
7.743341796875	-5.64437942308094e-10\\
7.7633359375	-4.83645437326984e-10\\
7.783330078125	-3.50458992982641e-10\\
7.80332421875	-3.05994864636508e-10\\
7.823318359375	-3.40205935712525e-10\\
7.8433125	-2.10271756062105e-10\\
7.863306640625	-4.20907023190027e-10\\
7.88330078125	-3.60619985025104e-10\\
7.903294921875	-4.80416087857523e-10\\
7.9232890625	-5.52021427626046e-10\\
7.943283203125	-6.42381053060868e-10\\
7.96327734375	-4.46244078371652e-10\\
7.983271484375	-4.94841840695982e-10\\
8.003265625	-1.94897370006027e-10\\
8.023259765625	-1.29188109466082e-10\\
8.04325390625	1.43658260174776e-10\\
8.063248046875	8.40556618364793e-11\\
8.0832421875	1.144159746938e-10\\
8.103236328125	1.26658527422139e-10\\
8.12323046875	-3.02003026525159e-11\\
8.143224609375	-1.75644969228999e-10\\
8.16321875	-9.79988799188573e-11\\
8.183212890625	-2.01469831046615e-10\\
8.20320703125	-1.07673307541535e-10\\
8.223201171875	-5.48195829735559e-11\\
8.2431953125	1.03971752246478e-10\\
8.263189453125	2.0186792093754e-10\\
8.28318359375	3.03074996839642e-10\\
8.303177734375	3.57694790291402e-10\\
8.323171875	3.62573095810162e-10\\
8.343166015625	3.51707331124852e-10\\
8.36316015625	3.1546938637029e-10\\
8.383154296875	2.45654315616866e-10\\
8.4031484375	2.4666997513848e-10\\
8.423142578125	1.59102112361795e-10\\
8.44313671875	2.49784214936258e-10\\
8.463130859375	2.22196106990305e-10\\
8.483125	1.66073962862514e-10\\
8.503119140625	2.2356765196584e-10\\
8.52311328125	3.31120892888219e-10\\
8.543107421875	1.51876795084487e-10\\
8.5631015625	3.00476303169375e-10\\
8.583095703125	1.28736780433111e-10\\
8.60308984375	1.47744919897621e-10\\
8.623083984375	4.55073772994653e-11\\
8.643078125	2.21919666831874e-12\\
8.663072265625	-5.75602684473136e-11\\
8.68306640625	-2.1936726916882e-11\\
8.703060546875	-3.78582584227098e-11\\
8.7230546875	5.58571488782169e-11\\
8.743048828125	1.17694536522715e-10\\
8.76304296875	1.31502913022457e-10\\
8.783037109375	1.09569081137236e-10\\
8.80303125	1.02909898046372e-10\\
8.823025390625	-5.12337897978566e-12\\
8.84301953125	-1.38251566632156e-10\\
8.863013671875	-2.15615956627366e-10\\
8.8830078125	-3.13307580148502e-10\\
8.903001953125	-2.53718126036474e-10\\
8.92299609375	-3.10852060838014e-10\\
8.942990234375	-2.32679751396131e-10\\
8.962984375	-1.89811356338684e-10\\
8.982978515625	-1.65648854402088e-10\\
9.00297265625	1.55069480514295e-12\\
9.022966796875	3.09849021128154e-12\\
9.0429609375	-1.00043172132014e-10\\
9.062955078125	1.85001725870808e-11\\
9.08294921875	-1.67559930859412e-10\\
9.102943359375	-8.57558913788397e-11\\
9.1229375	-1.92857153376975e-10\\
9.142931640625	2.14221219173703e-11\\
9.16292578125	5.70840744197617e-11\\
9.182919921875	6.12051297036757e-11\\
9.2029140625	2.49751308382656e-10\\
9.222908203125	2.03711787386318e-10\\
9.24290234375	1.77212262073581e-10\\
9.262896484375	1.22095487072989e-10\\
9.282890625	2.74720616539743e-10\\
9.302884765625	1.89990426003893e-10\\
9.32287890625	9.79010384197572e-11\\
9.342873046875	2.16982352839938e-10\\
9.3628671875	2.49753561873153e-10\\
9.382861328125	3.42546553525567e-10\\
9.40285546875	3.97469983669155e-10\\
9.422849609375	5.4430488641229e-10\\
9.44284375	6.05122093782794e-10\\
9.462837890625	6.27773844597033e-10\\
9.48283203125	5.06294955027526e-10\\
9.502826171875	4.19193741479462e-10\\
9.5228203125	3.18140032120496e-10\\
9.542814453125	2.10306107823676e-10\\
9.56280859375	1.6878560459674e-10\\
9.582802734375	2.03776968595529e-10\\
9.602796875	5.92034920366801e-11\\
9.622791015625	1.55277477364115e-10\\
9.64278515625	2.23741094427599e-10\\
9.662779296875	2.99259304546546e-10\\
9.6827734375	3.44395854636005e-10\\
9.702767578125	4.17902734362123e-10\\
9.72276171875	2.41675307133908e-10\\
9.742755859375	3.78270646329319e-10\\
9.76275	1.46115713552879e-10\\
9.782744140625	1.50255549568009e-10\\
9.80273828125	9.21649583803322e-11\\
9.822732421875	9.55267244581948e-11\\
9.8427265625	7.95886665684753e-11\\
9.862720703125	1.36933235949478e-10\\
9.88271484375	2.28633546975316e-10\\
9.902708984375	3.37719477810193e-10\\
9.922703125	3.55721312679043e-10\\
9.942697265625	2.95451091130058e-10\\
9.96269140625	3.78639188186261e-10\\
9.982685546875	3.18813421726041e-10\\
10.0026796875	4.05171387597476e-10\\
10.022673828125	4.00538100064734e-10\\
10.04266796875	4.69168731655885e-10\\
10.062662109375	4.5189748442547e-10\\
10.08265625	3.37253318055188e-10\\
10.102650390625	3.03468250058037e-10\\
10.12264453125	1.83540686578862e-10\\
10.142638671875	3.1286248431921e-10\\
10.1626328125	2.10885061567163e-10\\
10.182626953125	3.58216812576962e-10\\
10.20262109375	4.05389186444468e-10\\
10.222615234375	3.48429615574268e-10\\
10.242609375	3.61735392650644e-10\\
10.262603515625	4.57043364208171e-10\\
10.28259765625	3.5952107858295e-10\\
10.302591796875	3.34756231126814e-10\\
10.3225859375	3.38389100449998e-10\\
10.342580078125	3.60856784040335e-10\\
10.36257421875	3.66222907750026e-10\\
10.382568359375	3.26857526871318e-10\\
10.4025625	4.73157851936561e-10\\
10.422556640625	4.08944126486727e-10\\
10.44255078125	4.10499970392692e-10\\
10.462544921875	2.94353835904058e-10\\
10.4825390625	3.61473392391902e-10\\
10.502533203125	1.9864899812827e-10\\
10.52252734375	2.61838981315026e-10\\
10.542521484375	-2.55723390135975e-11\\
10.562515625	3.90718331822436e-11\\
10.582509765625	-8.92423627821253e-12\\
10.60250390625	-4.56811325799024e-11\\
10.622498046875	2.5034831097899e-11\\
10.6424921875	1.73110144313347e-10\\
10.662486328125	-1.80765060756552e-11\\
10.68248046875	2.0085681965887e-10\\
10.702474609375	1.61970906798907e-10\\
10.72246875	1.40847798889355e-10\\
10.742462890625	4.36643240971082e-11\\
10.76245703125	-1.58214578569261e-11\\
10.782451171875	-6.39820264785781e-11\\
10.8024453125	-1.57016530853663e-10\\
10.822439453125	-2.99136643821768e-10\\
10.84243359375	-1.41615561480442e-10\\
10.862427734375	-1.32970959604997e-10\\
10.882421875	-2.19934324693995e-10\\
10.902416015625	-1.25866519580564e-10\\
10.92241015625	-1.22081031088955e-10\\
10.942404296875	-2.24483321521234e-10\\
10.9623984375	-1.65310905033009e-10\\
10.982392578125	-8.58342848223501e-11\\
11.00238671875	-2.31420594409035e-10\\
11.022380859375	-2.45683446202332e-10\\
11.042375	-1.89073465809829e-10\\
11.062369140625	-3.18123112132924e-10\\
11.08236328125	-2.13949202863383e-10\\
11.102357421875	-2.83780941032272e-10\\
11.1223515625	-2.36957271574236e-10\\
11.142345703125	-2.40118987678699e-10\\
11.16233984375	-4.00742305491725e-10\\
11.182333984375	-3.43791537339429e-10\\
11.202328125	-4.07406092777266e-10\\
11.222322265625	-4.80560001455744e-10\\
11.24231640625	-4.54732394565579e-10\\
11.262310546875	-4.91621353347881e-10\\
11.2823046875	-5.47535245981806e-10\\
11.302298828125	-6.03177198498129e-10\\
11.32229296875	-5.30029648902872e-10\\
11.342287109375	-4.85137726162804e-10\\
11.36228125	-5.7845173277839e-10\\
11.382275390625	-4.95346853264279e-10\\
11.40226953125	-4.43538226715106e-10\\
11.422263671875	-4.95545676787647e-10\\
11.4422578125	-4.95771500832292e-10\\
11.462251953125	-5.51129950022212e-10\\
11.48224609375	-5.63836349932615e-10\\
11.502240234375	-7.18065312442335e-10\\
11.522234375	-6.40816678145751e-10\\
11.542228515625	-6.58642767391368e-10\\
11.56222265625	-7.89024114803985e-10\\
11.582216796875	-7.9681617637098e-10\\
11.6022109375	-8.84930644918549e-10\\
11.622205078125	-8.79597520798274e-10\\
11.64219921875	-9.26904849594964e-10\\
11.662193359375	-9.02671000988506e-10\\
11.6821875	-9.97572836001381e-10\\
11.702181640625	-1.10588661470949e-09\\
11.72217578125	-1.00278262405358e-09\\
11.742169921875	-1.19270957953616e-09\\
11.7621640625	-1.18445639398157e-09\\
11.782158203125	-1.34160178996194e-09\\
11.80215234375	-1.25691085516667e-09\\
11.822146484375	-1.33012601128346e-09\\
11.842140625	-1.37087746560596e-09\\
11.862134765625	-1.3612821585237e-09\\
11.88212890625	-1.38966603482307e-09\\
11.902123046875	-1.31719051322325e-09\\
11.9221171875	-1.38321617194247e-09\\
11.942111328125	-1.38261200449706e-09\\
11.96210546875	-1.44479577374202e-09\\
11.982099609375	-1.45566289820096e-09\\
12.00209375	-1.4934292930291e-09\\
12.022087890625	-1.46786055824763e-09\\
12.04208203125	-1.33192761375721e-09\\
12.062076171875	-1.21968216815296e-09\\
12.0820703125	-1.24319741619231e-09\\
12.102064453125	-1.05058346773158e-09\\
12.12205859375	-1.03648475774172e-09\\
12.142052734375	-1.10164826864087e-09\\
12.162046875	-1.12182041774374e-09\\
12.182041015625	-1.22017701422529e-09\\
12.20203515625	-1.2188071039856e-09\\
12.222029296875	-1.16495052814512e-09\\
12.2420234375	-1.15562191704002e-09\\
12.262017578125	-9.75553538609286e-10\\
12.28201171875	-9.21420368496751e-10\\
12.302005859375	-8.26737261027369e-10\\
12.322	-8.49494436689467e-10\\
12.341994140625	-6.67733080129369e-10\\
12.36198828125	-7.39005410180583e-10\\
12.381982421875	-7.99687154551096e-10\\
12.4019765625	-8.66625294833234e-10\\
12.421970703125	-8.76655947553402e-10\\
12.44196484375	-8.36831244668606e-10\\
12.461958984375	-1.02270566865145e-09\\
12.481953125	-9.37095084490056e-10\\
12.501947265625	-8.97981792794244e-10\\
12.52194140625	-9.5904686973636e-10\\
12.541935546875	-1.04979382487875e-09\\
12.5619296875	-9.76178938465755e-10\\
12.581923828125	-9.61598842673286e-10\\
12.60191796875	-8.39940936822352e-10\\
12.621912109375	-7.59553024756895e-10\\
12.64190625	-6.8182853211615e-10\\
12.661900390625	-6.18332201044528e-10\\
12.68189453125	-7.44181245507366e-10\\
12.701888671875	-7.58012198223324e-10\\
12.7218828125	-7.60616185712228e-10\\
12.741876953125	-7.48844922660996e-10\\
12.76187109375	-9.54220234458437e-10\\
12.781865234375	-8.54935077125031e-10\\
12.801859375	-7.90667866664766e-10\\
12.821853515625	-7.85668083174849e-10\\
12.84184765625	-7.60893353051773e-10\\
12.861841796875	-6.14429225592418e-10\\
12.8818359375	-5.34658102670847e-10\\
12.901830078125	-6.15006241877357e-10\\
12.92182421875	-5.08365769738872e-10\\
12.941818359375	-6.24011375757567e-10\\
12.9618125	-4.66522163016834e-10\\
12.981806640625	-5.16684349258175e-10\\
13.00180078125	-4.83491934422502e-10\\
13.021794921875	-4.24458233316452e-10\\
13.0417890625	-1.92123373534637e-10\\
13.061783203125	-1.99648576828866e-10\\
13.08177734375	-1.01887042206476e-10\\
13.101771484375	-1.63990160949557e-10\\
13.121765625	-1.89279129283979e-10\\
13.141759765625	-2.12288195043888e-10\\
13.16175390625	-2.54809972711715e-10\\
13.181748046875	-3.69558083459425e-10\\
13.2017421875	-3.29785691453146e-10\\
13.221736328125	-3.31389113471897e-10\\
13.24173046875	-2.86976543712987e-10\\
13.261724609375	-2.70568783273863e-10\\
13.28171875	-9.08150987590826e-11\\
13.301712890625	-4.43282876980236e-11\\
13.32170703125	-5.01487709733156e-11\\
13.341701171875	1.17066503301564e-11\\
13.3616953125	9.21531969825708e-11\\
13.381689453125	-4.39582183153762e-11\\
13.40168359375	-3.66867417466399e-11\\
13.421677734375	-2.00150849654324e-10\\
13.441671875	-2.57475041587605e-10\\
13.461666015625	-3.97756452260596e-10\\
13.48166015625	-3.10813675886119e-10\\
13.501654296875	-3.22678457264066e-10\\
13.5216484375	-3.18833932609442e-10\\
13.541642578125	-2.04920820455658e-10\\
13.56163671875	-5.74833044882269e-11\\
13.581630859375	-2.48944005535228e-11\\
13.601625	5.01831942105855e-11\\
13.621619140625	-6.68045611799597e-11\\
13.64161328125	1.76406136717198e-12\\
13.661607421875	-1.27902824711034e-10\\
13.6816015625	-1.96521276870514e-10\\
13.701595703125	-2.50780691922517e-10\\
13.72158984375	-3.0155623746091e-10\\
13.741583984375	-3.53935062745147e-10\\
13.761578125	-2.40614696911449e-10\\
13.781572265625	-2.65633734087938e-10\\
13.80156640625	-1.82158519115668e-10\\
13.821560546875	-2.11288376602966e-10\\
13.8415546875	-1.59841244489022e-10\\
13.861548828125	-1.56652320077334e-10\\
13.88154296875	-2.35769848711412e-10\\
13.901537109375	-1.22082204381845e-10\\
13.92153125	-1.43969820142178e-10\\
13.941525390625	-2.46965592330983e-10\\
13.96151953125	-6.5968307889037e-11\\
13.981513671875	-7.04195484487428e-11\\
14.0015078125	-8.93601119208443e-11\\
14.021501953125	-4.24272533689112e-11\\
14.04149609375	-7.42536040353956e-11\\
14.061490234375	2.69191735041818e-11\\
14.081484375	1.27061020495145e-10\\
14.101478515625	8.98320076344851e-11\\
14.12147265625	7.9874708133426e-11\\
14.141466796875	7.50596095182756e-11\\
14.1614609375	4.30595553539204e-11\\
14.181455078125	1.65492341999892e-10\\
14.20144921875	1.80433349159795e-10\\
14.221443359375	7.25988891919714e-11\\
14.2414375	2.73134966339316e-10\\
14.261431640625	2.33651606817574e-10\\
14.28142578125	3.49460294510515e-10\\
14.301419921875	2.55680551593289e-10\\
14.3214140625	4.77971881199457e-10\\
14.341408203125	3.92506407987569e-10\\
14.36140234375	3.48212411365719e-10\\
14.381396484375	3.9049702510879e-10\\
14.401390625	4.15017940001783e-10\\
14.421384765625	5.29637644640042e-10\\
14.44137890625	4.96238744777706e-10\\
14.461373046875	6.56531612264884e-10\\
14.4813671875	8.21361915359368e-10\\
14.501361328125	8.37824328458142e-10\\
14.52135546875	8.48719434219759e-10\\
14.541349609375	7.65960461170205e-10\\
14.56134375	6.70596789062847e-10\\
14.581337890625	6.29033235585805e-10\\
14.60133203125	5.30617643021481e-10\\
14.621326171875	5.65046951153412e-10\\
14.6413203125	5.34817984718375e-10\\
14.661314453125	4.75155928204267e-10\\
14.68130859375	5.4886942812274e-10\\
14.701302734375	6.39339066588278e-10\\
14.721296875	6.22589790832237e-10\\
14.741291015625	6.38529733543349e-10\\
14.76128515625	5.8426085458218e-10\\
14.781279296875	7.09263728850483e-10\\
14.8012734375	5.07142588820702e-10\\
14.821267578125	5.02835021595284e-10\\
14.84126171875	4.81886420221912e-10\\
14.861255859375	4.99726383358661e-10\\
14.88125	6.10431011555269e-10\\
14.901244140625	6.63040307981046e-10\\
14.92123828125	6.99616998660864e-10\\
14.941232421875	7.38903181341506e-10\\
14.9612265625	8.75500334663732e-10\\
14.981220703125	8.13736405714969e-10\\
15.00121484375	7.98579110249969e-10\\
15.021208984375	9.22215421626415e-10\\
15.041203125	8.63356761787903e-10\\
15.061197265625	9.80959417234143e-10\\
15.08119140625	9.94927880422755e-10\\
15.101185546875	1.01510587232491e-09\\
15.1211796875	8.99008959167106e-10\\
15.141173828125	9.66340984284226e-10\\
15.16116796875	9.83811648538048e-10\\
15.181162109375	9.13323369220309e-10\\
15.20115625	9.70614990336358e-10\\
15.221150390625	9.71359713497697e-10\\
15.24114453125	1.11640907065591e-09\\
15.261138671875	1.15778627885633e-09\\
15.2811328125	1.18563609378028e-09\\
15.301126953125	1.18927981381262e-09\\
15.32112109375	1.18741196135795e-09\\
15.341115234375	1.11607430097788e-09\\
15.361109375	1.01057525084216e-09\\
15.381103515625	1.11930519182015e-09\\
15.40109765625	1.09168709708819e-09\\
15.421091796875	1.1001762093717e-09\\
15.4410859375	1.31255320542856e-09\\
15.461080078125	1.1273291146392e-09\\
15.48107421875	1.2160926199012e-09\\
15.501068359375	1.16793904433698e-09\\
15.5210625	1.11477410893997e-09\\
15.541056640625	9.7532684866775e-10\\
15.56105078125	1.02736997894273e-09\\
15.581044921875	9.79003238791992e-10\\
15.6010390625	1.07888291493962e-09\\
15.621033203125	1.06689590583567e-09\\
15.64102734375	1.01889330834011e-09\\
15.661021484375	1.12098938158714e-09\\
15.681015625	1.16435794969369e-09\\
15.701009765625	1.05284598222689e-09\\
15.72100390625	1.100223203047e-09\\
15.740998046875	1.1205830341205e-09\\
15.7609921875	1.05006056181738e-09\\
15.780986328125	1.01251188253168e-09\\
15.80098046875	1.13208825127568e-09\\
15.820974609375	1.11234257447704e-09\\
15.84096875	1.06463117653943e-09\\
15.860962890625	1.12492597877824e-09\\
15.88095703125	1.29849318396673e-09\\
15.900951171875	1.31993556524939e-09\\
15.9209453125	1.38441578007585e-09\\
15.940939453125	1.42724991592273e-09\\
15.96093359375	1.47937879301331e-09\\
15.980927734375	1.44615860516656e-09\\
16.000921875	1.43674436826363e-09\\
16.020916015625	1.36557349292152e-09\\
16.04091015625	1.3934933452796e-09\\
16.060904296875	1.15453544292523e-09\\
16.0808984375	1.16672398446503e-09\\
16.100892578125	1.20403168484481e-09\\
16.12088671875	1.11715719332982e-09\\
16.140880859375	1.06532491124055e-09\\
16.160875	1.26263876026598e-09\\
16.180869140625	1.20396057938116e-09\\
16.20086328125	1.31999329175509e-09\\
16.220857421875	1.31202028049851e-09\\
16.2408515625	1.36328255293621e-09\\
16.260845703125	1.29259113834729e-09\\
16.28083984375	1.34654329436693e-09\\
16.300833984375	1.31178478727946e-09\\
16.320828125	1.25474937202652e-09\\
16.340822265625	1.08086272441805e-09\\
16.36081640625	1.16446794299765e-09\\
16.380810546875	1.18006523549465e-09\\
16.4008046875	1.01687284009402e-09\\
16.420798828125	1.07368713320662e-09\\
16.44079296875	1.17920879809117e-09\\
16.460787109375	1.16786603799338e-09\\
16.48078125	1.19765611640102e-09\\
16.500775390625	1.2404661381153e-09\\
16.52076953125	1.29300005715405e-09\\
16.540763671875	1.25631367062783e-09\\
16.5607578125	1.31540556625025e-09\\
16.580751953125	1.223025951781e-09\\
16.60074609375	1.27035721258018e-09\\
16.620740234375	1.23840516339852e-09\\
16.640734375	1.23432438536944e-09\\
16.660728515625	1.17762424195592e-09\\
16.68072265625	1.03238919349447e-09\\
16.700716796875	9.62970467985531e-10\\
16.7207109375	1.01640915116563e-09\\
16.740705078125	8.4404659814402e-10\\
16.76069921875	9.23890285540814e-10\\
16.780693359375	8.07175016595679e-10\\
16.8006875	1.00779518921576e-09\\
16.820681640625	8.30603897565982e-10\\
16.84067578125	9.44678567901399e-10\\
16.860669921875	8.96902997084825e-10\\
16.8806640625	7.3889333380963e-10\\
16.900658203125	6.23798938108588e-10\\
16.92065234375	4.47475410106721e-10\\
16.940646484375	4.84768997988611e-10\\
16.960640625	1.8735640490425e-10\\
16.980634765625	1.87880298949709e-10\\
17.00062890625	1.20854454978e-10\\
17.020623046875	2.10834149556666e-10\\
17.0406171875	2.15415818921634e-10\\
17.060611328125	3.20317926291025e-10\\
17.08060546875	4.00918392305313e-10\\
17.100599609375	4.92657219886169e-10\\
17.12059375	4.8256073194393e-10\\
17.140587890625	5.21045430075721e-10\\
17.16058203125	5.14261755911565e-10\\
17.180576171875	4.04206934793047e-10\\
17.2005703125	3.46431218787839e-10\\
17.220564453125	3.44997012990813e-10\\
17.24055859375	2.27114047563832e-10\\
17.260552734375	3.28902698554785e-10\\
17.280546875	1.78422154921583e-10\\
17.300541015625	3.44692415918816e-10\\
17.32053515625	3.23088472362069e-10\\
17.340529296875	2.89603559931668e-10\\
17.3605234375	3.18307684438513e-10\\
17.380517578125	3.19697688240067e-10\\
17.40051171875	3.63486782377418e-10\\
17.420505859375	3.24579966431406e-10\\
17.4405	3.2647502043999e-10\\
17.460494140625	3.55059559384675e-10\\
17.48048828125	2.33605256235906e-10\\
17.500482421875	1.85349220095127e-10\\
17.5204765625	7.34496637016473e-11\\
17.540470703125	9.45796105631848e-11\\
17.56046484375	3.18722154440534e-11\\
17.580458984375	-3.67991548931008e-11\\
17.600453125	-9.40635477628119e-12\\
17.620447265625	1.35610729248455e-10\\
17.64044140625	1.27281291916715e-10\\
17.660435546875	2.869450836489e-10\\
17.6804296875	3.45696313981409e-10\\
17.700423828125	2.53106631445468e-10\\
17.72041796875	2.67141291943972e-10\\
17.740412109375	2.14698860004276e-10\\
17.76040625	1.45652937916882e-10\\
17.780400390625	1.16269651593632e-10\\
17.80039453125	1.72731464886008e-11\\
17.820388671875	6.97594836710307e-12\\
17.8403828125	9.4638938855307e-11\\
17.860376953125	1.90298086673551e-12\\
17.88037109375	-4.5140911567918e-11\\
17.900365234375	-1.39766633819336e-11\\
17.920359375	-4.37012547706957e-11\\
17.940353515625	-1.31094169875142e-10\\
17.96034765625	1.44712413649167e-11\\
17.980341796875	-1.61626521076008e-10\\
18.0003359375	-8.84281781949395e-11\\
18.020330078125	-1.02215734137975e-10\\
18.04032421875	-2.52453601454931e-10\\
18.060318359375	-2.42649814899251e-10\\
18.0803125	-2.07484431422773e-10\\
18.100306640625	-3.42243299370803e-10\\
18.12030078125	-2.12595405557109e-10\\
18.140294921875	-2.83950598428059e-10\\
18.1602890625	-1.17488324681489e-10\\
18.180283203125	-2.77522281913266e-10\\
18.20027734375	-1.01507282859411e-10\\
18.220271484375	-3.32471225458192e-10\\
18.240265625	-3.19314399476243e-10\\
18.260259765625	-4.28089097157427e-10\\
18.28025390625	-5.43526338079493e-10\\
18.300248046875	-5.48315161685996e-10\\
18.3202421875	-6.80525700168362e-10\\
18.340236328125	-6.43019812974448e-10\\
18.36023046875	-6.86806842440934e-10\\
18.380224609375	-7.39114786945627e-10\\
18.40021875	-6.77517463568384e-10\\
18.420212890625	-6.70394585896595e-10\\
18.44020703125	-7.5721806196238e-10\\
18.460201171875	-6.53742108632262e-10\\
18.4801953125	-7.68629629912845e-10\\
18.500189453125	-8.72309362236773e-10\\
18.52018359375	-8.54021163563619e-10\\
18.540177734375	-9.41731592591504e-10\\
18.560171875	-9.23108917251682e-10\\
18.580166015625	-8.52724646687819e-10\\
18.60016015625	-8.97088594881123e-10\\
18.620154296875	-7.9154671148575e-10\\
18.6401484375	-7.45118229757546e-10\\
18.660142578125	-8.84337729587638e-10\\
18.68013671875	-8.04381734440829e-10\\
18.700130859375	-9.33856688002189e-10\\
18.720125	-1.06536191243512e-09\\
18.740119140625	-1.15946325151634e-09\\
18.76011328125	-1.16043134254582e-09\\
18.780107421875	-1.22703430803295e-09\\
18.8001015625	-1.06485761039003e-09\\
18.820095703125	-1.0680907178357e-09\\
18.84008984375	-1.02998841567966e-09\\
18.860083984375	-1.13587988249132e-09\\
18.880078125	-9.75811954271847e-10\\
18.900072265625	-9.27704152054956e-10\\
18.92006640625	-8.28277656310839e-10\\
18.940060546875	-8.72029337199825e-10\\
18.9600546875	-8.83163368406011e-10\\
18.980048828125	-9.41124748353772e-10\\
19.00004296875	-1.08870763761299e-09\\
19.020037109375	-1.12105851172731e-09\\
19.04003125	-1.28059502177536e-09\\
19.060025390625	-1.32286484315129e-09\\
19.08001953125	-1.2645390876408e-09\\
19.100013671875	-1.32202077389767e-09\\
19.1200078125	-1.2560491468056e-09\\
19.140001953125	-1.30179809027985e-09\\
19.15999609375	-1.13460242672224e-09\\
19.179990234375	-1.04162608931307e-09\\
19.199984375	-9.88232900929917e-10\\
19.219978515625	-1.06705767243956e-09\\
19.23997265625	-1.05063136342827e-09\\
19.259966796875	-1.19150290998246e-09\\
19.2799609375	-1.22862864861699e-09\\
19.299955078125	-1.36361517325836e-09\\
19.31994921875	-1.31749095685464e-09\\
19.339943359375	-1.41973539978503e-09\\
19.3599375	-1.39386268177576e-09\\
19.379931640625	-1.26958197582597e-09\\
19.39992578125	-1.21626729604101e-09\\
19.419919921875	-1.07226050275907e-09\\
19.4399140625	-1.12962205202145e-09\\
19.459908203125	-9.90385778668079e-10\\
19.47990234375	-1.07459958358757e-09\\
19.499896484375	-1.0165262096088e-09\\
19.519890625	-1.12264894952743e-09\\
19.539884765625	-1.0030772504972e-09\\
19.55987890625	-1.12408617007506e-09\\
19.579873046875	-1.09893323967694e-09\\
19.5998671875	-1.08659233352714e-09\\
19.619861328125	-1.08231832744086e-09\\
19.63985546875	-1.05283250739424e-09\\
19.659849609375	-1.12692710009612e-09\\
19.67984375	-1.25653879121559e-09\\
19.699837890625	-1.21892388985378e-09\\
19.71983203125	-1.20675182855313e-09\\
19.739826171875	-1.20940250600216e-09\\
19.7598203125	-1.17375846733822e-09\\
19.779814453125	-1.10478764596623e-09\\
19.79980859375	-1.15862013629876e-09\\
19.819802734375	-9.99209742436806e-10\\
19.839796875	-1.05606933024475e-09\\
19.859791015625	-1.01581784282351e-09\\
19.87978515625	-9.67575433264284e-10\\
19.899779296875	-1.06558686730674e-09\\
19.9197734375	-9.94359556062614e-10\\
19.939767578125	-8.58376824458686e-10\\
19.95976171875	-8.9054871560617e-10\\
19.979755859375	-6.92821686674822e-10\\
19.99975	-6.39306651646957e-10\\
20.019744140625	-4.83384802690938e-10\\
20.03973828125	-4.70484588378852e-10\\
20.059732421875	-3.427840507848e-10\\
20.0797265625	-4.05410923052593e-10\\
20.099720703125	-3.71383554941807e-10\\
20.11971484375	-4.85649309026861e-10\\
20.139708984375	-3.47226435293208e-10\\
20.159703125	-5.19409420807652e-10\\
20.179697265625	-5.12121512891246e-10\\
20.19969140625	-5.01783465127975e-10\\
20.219685546875	-5.40569963073591e-10\\
20.2396796875	-3.95478522036682e-10\\
20.259673828125	-3.48871150759126e-10\\
20.27966796875	-3.16331750172953e-10\\
20.299662109375	-3.8948849412642e-10\\
20.31965625	-3.36563362601415e-10\\
20.339650390625	-4.01626096686647e-10\\
20.35964453125	-3.38278248222122e-10\\
20.379638671875	-5.10529800961264e-10\\
20.3996328125	-5.06896885049811e-10\\
20.419626953125	-4.68209465653363e-10\\
20.43962109375	-5.42271877925134e-10\\
20.459615234375	-5.87880202759731e-10\\
20.479609375	-3.7180122010385e-10\\
20.499603515625	-5.13626573003999e-10\\
20.51959765625	-3.99913065302272e-10\\
20.539591796875	-3.20265045069752e-10\\
20.5595859375	-3.39169165615705e-10\\
20.579580078125	-3.22841048519104e-10\\
20.59957421875	-4.13031376332507e-10\\
20.619568359375	-4.10308193844153e-10\\
20.6395625	-3.33912972531646e-10\\
20.659556640625	-5.63667610119076e-10\\
20.67955078125	-3.61707861135168e-10\\
20.699544921875	-5.29729293432269e-10\\
20.7195390625	-3.8766215721196e-10\\
20.739533203125	-4.74653413794566e-10\\
20.75952734375	-3.61012008550797e-10\\
20.779521484375	-3.04440537350445e-10\\
20.799515625	-3.5863214066348e-10\\
20.819509765625	-3.00306202743825e-10\\
20.83950390625	-2.24588627467198e-10\\
20.859498046875	-1.39109756308239e-10\\
20.8794921875	-1.96618016498989e-10\\
20.899486328125	-1.49106030408237e-10\\
20.91948046875	-2.57374587701317e-10\\
20.939474609375	-2.20356874505176e-10\\
20.95946875	-2.81064030051827e-10\\
20.979462890625	-2.33331883250619e-10\\
20.99945703125	-2.54145202561457e-10\\
21.019451171875	-3.7902737597315e-10\\
21.0394453125	-2.76254410771825e-10\\
21.059439453125	-3.03589893097223e-10\\
21.07943359375	-2.45657655040132e-10\\
21.099427734375	-2.98315450170916e-10\\
21.119421875	-2.29468711092843e-10\\
21.139416015625	-3.01283812944253e-10\\
21.15941015625	-2.702271114768e-10\\
21.179404296875	-3.1100289601944e-10\\
21.1993984375	-2.53903476214352e-10\\
21.219392578125	-2.47292671227627e-10\\
21.23938671875	-2.3953377719941e-10\\
21.259380859375	-1.81938234647571e-10\\
21.279375	-1.42665972274005e-10\\
21.299369140625	-1.69832320608752e-10\\
21.31936328125	-2.15277598276919e-10\\
21.339357421875	-4.20974872318921e-10\\
21.3593515625	-3.8893493283462e-10\\
21.379345703125	-5.20056387793931e-10\\
21.39933984375	-3.87191629297329e-10\\
21.419333984375	-4.72363895211973e-10\\
21.439328125	-4.30829085168635e-10\\
21.459322265625	-4.5057432724703e-10\\
21.47931640625	-4.56827096361578e-10\\
21.499310546875	-3.67866368693878e-10\\
21.5193046875	-3.29030982627396e-10\\
21.539298828125	-4.00614334836268e-10\\
21.55929296875	-3.80332247666174e-10\\
21.579287109375	-3.10149064517699e-10\\
21.59928125	-3.03197802508813e-10\\
21.619275390625	-3.67578652525859e-10\\
21.63926953125	-2.83390598864781e-10\\
21.659263671875	-4.8536239259656e-10\\
21.6792578125	-4.01547645171007e-10\\
21.699251953125	-4.92245484777495e-10\\
21.71924609375	-4.79445588413992e-10\\
21.739240234375	-5.1273754484584e-10\\
21.759234375	-5.24790673289194e-10\\
21.779228515625	-5.66886827995347e-10\\
21.79922265625	-5.2926143245608e-10\\
21.819216796875	-4.10366427309446e-10\\
21.8392109375	-4.65957285750428e-10\\
21.859205078125	-5.36559468003075e-10\\
21.87919921875	-5.92067567210349e-10\\
21.899193359375	-5.05921015821247e-10\\
21.9191875	-6.82813790310144e-10\\
21.939181640625	-6.20517196709961e-10\\
21.95917578125	-8.87959728204264e-10\\
21.979169921875	-7.02191054418812e-10\\
21.9991640625	-9.46579957405063e-10\\
22.019158203125	-8.92267650377762e-10\\
22.03915234375	-8.68204232459207e-10\\
22.059146484375	-7.66891861157643e-10\\
22.079140625	-8.81024122284085e-10\\
22.099134765625	-7.69712697376731e-10\\
22.11912890625	-7.15938420571902e-10\\
22.139123046875	-7.09806102122158e-10\\
22.1591171875	-6.55029466555829e-10\\
22.179111328125	-4.76878352616201e-10\\
22.19910546875	-5.60109014322595e-10\\
22.219099609375	-5.25207683941932e-10\\
22.23909375	-6.53635854232017e-10\\
22.259087890625	-6.15470378421893e-10\\
22.27908203125	-6.56085604398785e-10\\
22.299076171875	-6.28568845094099e-10\\
22.3190703125	-6.70635447701532e-10\\
22.339064453125	-6.67024913538162e-10\\
22.35905859375	-5.20528354248725e-10\\
22.379052734375	-5.70603026013016e-10\\
22.399046875	-4.39782413660179e-10\\
22.419041015625	-5.79163754522176e-10\\
22.43903515625	-6.35560497509286e-10\\
22.459029296875	-6.18321099582244e-10\\
22.4790234375	-7.99478253231476e-10\\
22.499017578125	-8.10546545962895e-10\\
22.51901171875	-1.02998570607954e-09\\
22.539005859375	-9.27979855473836e-10\\
22.559	-1.00835344325007e-09\\
22.578994140625	-9.61523807491841e-10\\
22.59898828125	-1.04011098435291e-09\\
22.618982421875	-8.12998794565535e-10\\
22.6389765625	-9.41200196545925e-10\\
22.658970703125	-7.42058413774944e-10\\
22.67896484375	-8.30115775465355e-10\\
22.698958984375	-7.6617210211059e-10\\
22.718953125	-6.79147231252073e-10\\
22.738947265625	-7.99536602131699e-10\\
22.75894140625	-7.25899336999127e-10\\
22.778935546875	-9.66717632929952e-10\\
22.7989296875	-8.99967332678167e-10\\
22.818923828125	-9.90527770798557e-10\\
22.83891796875	-8.1350998276748e-10\\
22.858912109375	-8.38435615109137e-10\\
22.87890625	-7.2668905543069e-10\\
22.898900390625	-6.48202453319869e-10\\
22.91889453125	-5.91771865057594e-10\\
22.938888671875	-4.78660030267812e-10\\
22.9588828125	-4.26652165362445e-10\\
22.978876953125	-4.85131922105149e-10\\
22.99887109375	-3.43233626011768e-10\\
23.018865234375	-5.32447972960625e-10\\
23.038859375	-4.35885733195729e-10\\
23.058853515625	-3.40084945898064e-10\\
23.07884765625	-3.53335687042172e-10\\
23.098841796875	-3.01976298949162e-10\\
23.1188359375	-3.91905653056296e-10\\
23.138830078125	-4.63413315947969e-10\\
23.15882421875	-2.90992827721549e-10\\
23.178818359375	-4.24573232068173e-10\\
23.1988125	-1.7324510304125e-10\\
23.218806640625	-2.70458693589613e-10\\
23.23880078125	-3.2261500757286e-11\\
23.258794921875	-1.38908492905677e-10\\
23.2787890625	6.57672460516708e-12\\
23.298783203125	-3.26419915271994e-11\\
23.31877734375	-1.004206699228e-10\\
23.338771484375	-1.24106180317776e-10\\
23.358765625	-2.16823024776273e-10\\
23.378759765625	-2.04350164331911e-10\\
23.39875390625	-2.56345765061125e-10\\
23.418748046875	-2.71212444266365e-10\\
23.4387421875	-2.73129664578179e-10\\
23.458736328125	-2.40488306208577e-10\\
23.47873046875	-9.73203220854479e-11\\
23.498724609375	-8.7093586730446e-11\\
23.51871875	6.87196165142079e-11\\
23.538712890625	1.57980065467431e-11\\
23.55870703125	1.14146457725935e-10\\
23.578701171875	-8.82446771246146e-11\\
23.5986953125	1.03051126590128e-10\\
23.618689453125	-1.24205619122156e-11\\
23.63868359375	-2.45585647573236e-11\\
23.658677734375	-7.92691176067832e-11\\
23.678671875	-1.14213357682479e-10\\
23.698666015625	-6.43047871730264e-11\\
23.71866015625	-8.24614460889469e-11\\
23.738654296875	4.90325570117e-12\\
23.7586484375	1.437411523714e-10\\
23.778642578125	1.19428428984157e-10\\
23.79863671875	1.59896632230572e-10\\
23.818630859375	9.51533041620045e-11\\
23.838625	2.28869796656557e-10\\
23.858619140625	1.5891791240521e-10\\
23.87861328125	7.42055168319244e-11\\
23.898607421875	4.71213411385189e-11\\
23.9186015625	2.21952584040431e-10\\
23.938595703125	1.54320187907336e-10\\
23.95858984375	1.57345133966979e-10\\
23.978583984375	1.35066724540396e-10\\
23.998578125	1.27272396526816e-10\\
24.018572265625	-8.61968221211412e-11\\
24.03856640625	8.8386839955211e-11\\
24.058560546875	3.61671807838852e-11\\
24.0785546875	1.60611210691425e-10\\
24.098548828125	1.30021238288996e-10\\
24.11854296875	2.76433088526917e-10\\
24.138537109375	2.70447827178052e-10\\
24.15853125	3.10281878779669e-10\\
24.178525390625	2.9931449830886e-10\\
24.19851953125	4.09542848751652e-10\\
24.218513671875	4.98726258024532e-10\\
24.2385078125	4.03920265697752e-10\\
24.258501953125	3.91587782219448e-10\\
24.27849609375	4.45654853572814e-10\\
24.298490234375	4.65572762224563e-10\\
24.318484375	3.81953129938019e-10\\
24.338478515625	3.88986069921045e-10\\
24.35847265625	4.27337166524363e-10\\
24.378466796875	3.5172009368443e-10\\
24.3984609375	3.54212134900099e-10\\
24.418455078125	4.41793385378201e-10\\
24.43844921875	4.10200658921009e-10\\
24.458443359375	5.82849831096568e-10\\
24.4784375	4.33592660066658e-10\\
24.498431640625	6.82165009895919e-10\\
24.51842578125	6.81696143776194e-10\\
24.538419921875	7.73588110051653e-10\\
24.5584140625	7.72072996931004e-10\\
24.578408203125	9.07138212890484e-10\\
24.59840234375	7.28950165984756e-10\\
24.618396484375	7.38299634492522e-10\\
24.638390625	6.27678393069293e-10\\
24.658384765625	5.55862468933323e-10\\
24.67837890625	4.4299446017186e-10\\
24.698373046875	5.27533632758032e-10\\
24.7183671875	4.72128364297875e-10\\
24.738361328125	5.99326561328247e-10\\
24.75835546875	6.10133929123164e-10\\
24.778349609375	7.48050791699666e-10\\
24.79834375	6.6907358203228e-10\\
24.818337890625	7.79212640691362e-10\\
24.83833203125	6.93889619657074e-10\\
24.858326171875	6.24751797660929e-10\\
24.8783203125	6.16159478337871e-10\\
24.898314453125	5.20114227035562e-10\\
24.91830859375	5.40073114959958e-10\\
24.938302734375	6.28502434503452e-10\\
24.958296875	5.12552959552369e-10\\
24.978291015625	7.09043522486933e-10\\
24.99828515625	6.48426745142355e-10\\
25.018279296875	8.78359279879358e-10\\
25.0382734375	8.98251329268818e-10\\
25.058267578125	9.16009287491889e-10\\
25.07826171875	9.98046057623773e-10\\
25.098255859375	9.41082432015294e-10\\
25.11825	9.70602727243865e-10\\
25.138244140625	1.01886189325595e-09\\
25.15823828125	8.1751023382574e-10\\
25.178232421875	9.27221493917671e-10\\
25.1982265625	7.54641294642113e-10\\
25.218220703125	8.90827003274191e-10\\
25.23821484375	9.32497352157896e-10\\
25.258208984375	8.68022291688128e-10\\
25.278203125	1.07449367501467e-09\\
25.298197265625	9.50371230476271e-10\\
25.31819140625	9.81750665079898e-10\\
25.338185546875	9.99770888317948e-10\\
25.3581796875	1.00486550204973e-09\\
25.378173828125	9.73549700242785e-10\\
25.39816796875	9.48885725203681e-10\\
25.418162109375	7.63269708941352e-10\\
25.43815625	7.99952496660451e-10\\
25.458150390625	7.70383382426389e-10\\
25.47814453125	7.44628278418915e-10\\
25.498138671875	7.02962840456273e-10\\
25.5181328125	7.0486029824547e-10\\
25.538126953125	5.60699284115683e-10\\
25.55812109375	5.54606691419233e-10\\
25.578115234375	4.79780584813743e-10\\
25.598109375	5.17948350751424e-10\\
25.618103515625	4.99275874735369e-10\\
25.63809765625	5.63848869539662e-10\\
25.658091796875	5.803851254791e-10\\
25.6780859375	7.20835930993131e-10\\
25.698080078125	6.49776696247632e-10\\
25.71807421875	7.96171442936276e-10\\
25.738068359375	6.51655310429186e-10\\
25.7580625	7.00478595638525e-10\\
25.778056640625	5.06075652305496e-10\\
25.79805078125	5.0006816497902e-10\\
25.818044921875	4.10254544352487e-10\\
25.8380390625	2.49185461461042e-10\\
25.858033203125	4.31890835078325e-10\\
25.87802734375	2.62637336747619e-10\\
25.898021484375	4.63116999212872e-10\\
25.918015625	5.23948820647727e-10\\
25.938009765625	6.1053548781379e-10\\
25.95800390625	6.42190065483467e-10\\
25.977998046875	6.21147118866095e-10\\
25.9979921875	7.1677846558566e-10\\
26.017986328125	4.78303253174483e-10\\
26.03798046875	5.14585243669614e-10\\
26.057974609375	3.59786977394961e-10\\
26.07796875	3.22342293125052e-10\\
26.097962890625	2.18980121547961e-10\\
26.11795703125	3.39366433774223e-10\\
26.137951171875	3.69901444042743e-10\\
26.1579453125	3.74521774478389e-10\\
26.177939453125	4.55481456308168e-10\\
26.19793359375	4.4675908539037e-10\\
26.217927734375	4.79464937472314e-10\\
26.237921875	2.97651914174058e-10\\
26.257916015625	2.18240221094846e-10\\
26.27791015625	2.48343614847961e-10\\
26.297904296875	2.05796685206365e-10\\
26.3178984375	1.07051727931286e-10\\
26.337892578125	7.09360372756395e-11\\
26.35788671875	8.07481472924698e-11\\
26.377880859375	8.04764868415074e-11\\
26.397875	9.93560110063952e-11\\
26.417869140625	-3.43652882523258e-13\\
26.43786328125	9.90318003167218e-11\\
26.457857421875	1.75792466448786e-11\\
26.4778515625	9.35947125473231e-11\\
26.497845703125	7.4323104727477e-11\\
26.51783984375	8.27763299967878e-11\\
26.537833984375	2.87226561547878e-11\\
26.557828125	1.51218160920994e-10\\
26.577822265625	6.04227328436206e-11\\
26.59781640625	6.55442124480165e-11\\
26.617810546875	-4.44195076793769e-11\\
26.6378046875	9.48776607286549e-12\\
26.657798828125	-1.45189412603524e-10\\
26.67779296875	-8.85491183871268e-11\\
26.697787109375	-2.85866999100018e-10\\
26.71778125	-3.27108416865812e-10\\
26.737775390625	-2.29492213909291e-10\\
26.75776953125	-3.29147187082741e-10\\
26.777763671875	-2.76278638589733e-10\\
26.7977578125	-2.2537758900907e-10\\
26.817751953125	-2.0796613622935e-10\\
26.83774609375	-2.43204893641226e-10\\
26.857740234375	-2.26224206373298e-10\\
26.877734375	-1.65103195487607e-10\\
26.897728515625	-2.14436454770955e-10\\
26.91772265625	-1.81306564163981e-10\\
26.937716796875	-8.75212396232773e-11\\
26.9577109375	-2.09184872961265e-10\\
26.977705078125	-1.16579404416822e-10\\
26.99769921875	-3.13139243369167e-10\\
27.017693359375	-2.52392370153854e-10\\
27.0376875	-5.10263053653323e-10\\
27.057681640625	-4.76584285817957e-10\\
27.07767578125	-6.02357196141648e-10\\
27.097669921875	-4.97463329621295e-10\\
27.1176640625	-6.99621891244722e-10\\
27.137658203125	-5.07469264099055e-10\\
27.15765234375	-5.38317510464248e-10\\
27.177646484375	-4.23112868687542e-10\\
27.197640625	-5.01765091485712e-10\\
27.217634765625	-4.40305940602069e-10\\
27.23762890625	-5.52836252722973e-10\\
27.257623046875	-5.37514968031525e-10\\
27.2776171875	-6.06644997492401e-10\\
27.297611328125	-6.10677593945071e-10\\
27.31760546875	-7.01760731597566e-10\\
27.337599609375	-6.4473263411707e-10\\
27.35759375	-7.23690480040552e-10\\
27.377587890625	-7.55399688584103e-10\\
27.39758203125	-6.81602509144078e-10\\
27.417576171875	-8.02214067999349e-10\\
27.4375703125	-7.50325474978224e-10\\
27.457564453125	-7.68741080985662e-10\\
27.47755859375	-7.82635617725695e-10\\
27.497552734375	-7.1130546437988e-10\\
27.517546875	-8.17599052647081e-10\\
27.537541015625	-8.21255534925993e-10\\
27.55753515625	-8.38056605511416e-10\\
27.577529296875	-9.73322831796618e-10\\
27.5975234375	-9.55661275179954e-10\\
27.617517578125	-1.06086696106847e-09\\
27.63751171875	-9.89105586792294e-10\\
27.657505859375	-8.88442137398689e-10\\
27.6775	-8.13004059675903e-10\\
27.697494140625	-6.09736630457303e-10\\
27.71748828125	-5.78435989706138e-10\\
27.737482421875	-5.99026453039678e-10\\
27.7574765625	-5.83242253675926e-10\\
27.777470703125	-7.15776736726882e-10\\
27.79746484375	-6.38761809488092e-10\\
27.817458984375	-7.99143559976995e-10\\
27.837453125	-7.43090949317892e-10\\
27.857447265625	-7.32758188744625e-10\\
27.87744140625	-6.98473271525188e-10\\
27.897435546875	-6.75498219725115e-10\\
27.9174296875	-5.34747980165704e-10\\
27.937423828125	-4.40960272909243e-10\\
27.95741796875	-3.55381043615075e-10\\
27.977412109375	-1.95560030421674e-10\\
27.99740625	-1.51968380671047e-10\\
28.017400390625	-1.43832529343067e-10\\
28.03739453125	-1.6042533400462e-10\\
28.057388671875	-1.68659413155911e-10\\
28.0773828125	-1.59291457626186e-10\\
28.097376953125	-1.91338682414468e-10\\
28.11737109375	-1.59395718538944e-10\\
28.137365234375	-2.35364478117103e-10\\
28.157359375	-2.35266117495649e-10\\
28.177353515625	-2.35894891366521e-10\\
28.19734765625	-7.67918452876749e-11\\
28.217341796875	-1.82637771286904e-10\\
28.2373359375	-9.38142918107954e-11\\
28.257330078125	-1.25812485855036e-10\\
28.27732421875	3.14101790423283e-11\\
28.297318359375	-9.70086792847505e-11\\
28.3173125	-5.07853260400566e-11\\
28.337306640625	5.16800285550871e-11\\
28.35730078125	6.07044817960388e-11\\
28.377294921875	1.12396807509249e-10\\
28.3972890625	-3.38544792180753e-12\\
28.417283203125	-9.18035627946383e-12\\
28.43727734375	-5.62484727399921e-11\\
28.457271484375	-2.28422196629479e-10\\
28.477265625	-2.68223858273085e-10\\
28.497259765625	-2.61347736461727e-10\\
28.51725390625	-1.55463778110868e-10\\
28.537248046875	-2.18589955704088e-10\\
28.5572421875	-9.55792213409165e-11\\
28.577236328125	-1.32173705726693e-10\\
28.59723046875	1.43017133710798e-11\\
28.617224609375	-4.58664138733427e-12\\
28.63721875	3.32484004779447e-11\\
28.657212890625	9.36938609063408e-11\\
28.67720703125	-3.68942064613973e-11\\
28.697201171875	-5.48508505029093e-11\\
28.7171953125	-1.89949105313603e-10\\
28.737189453125	-5.84336235844704e-11\\
28.75718359375	-1.82341010779265e-10\\
28.777177734375	-1.21680059701666e-11\\
28.797171875	-1.33393160746482e-11\\
28.817166015625	8.03135681091843e-11\\
28.83716015625	1.2782265307939e-10\\
28.857154296875	9.80624828180544e-11\\
28.8771484375	1.87698075078719e-10\\
28.897142578125	2.21409689037192e-10\\
28.91713671875	1.51584745650801e-10\\
28.937130859375	1.92346300853387e-10\\
28.957125	1.78180914476127e-10\\
28.977119140625	5.96708491114881e-11\\
28.99711328125	1.03717354363745e-10\\
29.017107421875	7.39775688323184e-11\\
29.0371015625	1.32752728436889e-10\\
29.057095703125	3.43974828309223e-11\\
29.07708984375	1.15028812992414e-10\\
29.097083984375	3.60446994954592e-11\\
29.117078125	1.05095965951189e-10\\
29.137072265625	1.4539459767889e-10\\
29.15706640625	1.91351047791226e-10\\
29.177060546875	1.38138219741455e-10\\
29.1970546875	3.03480421591164e-10\\
29.217048828125	2.52412331381548e-10\\
29.23704296875	2.40887503632007e-10\\
29.257037109375	2.18304374378211e-10\\
29.27703125	2.14356212987657e-10\\
29.297025390625	2.40813780379277e-10\\
29.31701953125	2.69359789677706e-10\\
29.337013671875	3.22106541547974e-10\\
29.3570078125	1.4323654844147e-10\\
29.377001953125	1.27198302795945e-10\\
29.39699609375	1.71471652449733e-10\\
29.416990234375	6.65631620769532e-12\\
29.436984375	6.14899814355186e-12\\
29.456978515625	6.88044989540108e-11\\
29.47697265625	-1.03343742789396e-10\\
29.496966796875	5.59554013538669e-11\\
29.5169609375	-1.06046026732464e-10\\
29.536955078125	1.24325983739005e-10\\
29.55694921875	1.14319651241375e-10\\
29.576943359375	2.24061717843147e-10\\
29.5969375	1.77146566949285e-10\\
29.616931640625	3.12157944652156e-10\\
29.63692578125	1.81297564566719e-10\\
29.656919921875	2.11311988241177e-10\\
29.6769140625	8.92198158546902e-11\\
29.696908203125	1.44841687842522e-10\\
29.71690234375	9.72976183658775e-11\\
29.736896484375	1.93416543155414e-10\\
29.756890625	1.36119250958366e-10\\
29.776884765625	2.14497328394295e-10\\
29.79687890625	1.64125302016532e-10\\
29.816873046875	3.67100851556749e-10\\
29.8368671875	2.95934188819521e-10\\
29.856861328125	3.14849378122616e-10\\
29.87685546875	2.11261725638913e-10\\
29.896849609375	3.79779713485862e-10\\
29.91684375	3.28870255139945e-10\\
29.936837890625	3.42486680694641e-10\\
29.95683203125	4.76411226230979e-10\\
29.976826171875	4.52211429878727e-10\\
29.9968203125	5.12904846277457e-10\\
30.016814453125	5.28099164831332e-10\\
30.03680859375	6.00858403503292e-10\\
30.056802734375	6.33459663574539e-10\\
30.076796875	6.08486079833291e-10\\
30.096791015625	6.07906704570451e-10\\
30.11678515625	7.5685780389997e-10\\
30.136779296875	6.4183493798245e-10\\
30.1567734375	6.7397721704144e-10\\
30.176767578125	6.84735865957622e-10\\
30.19676171875	5.76288308147239e-10\\
30.216755859375	6.1240133733735e-10\\
30.23675	5.66158307732893e-10\\
30.256744140625	5.09667070347982e-10\\
30.27673828125	6.43532567393089e-10\\
30.296732421875	5.51349824790717e-10\\
30.3167265625	7.47362052003757e-10\\
30.336720703125	6.59457800547797e-10\\
30.35671484375	7.58333616861881e-10\\
30.376708984375	7.37135491132274e-10\\
30.396703125	7.43386442596139e-10\\
30.416697265625	6.35234360656767e-10\\
30.43669140625	6.03286700676837e-10\\
30.456685546875	4.68661718412103e-10\\
30.4766796875	3.72269760501767e-10\\
30.496673828125	2.50920166582984e-10\\
30.51666796875	1.932739635156e-10\\
30.536662109375	9.39443724325469e-11\\
30.55665625	1.51277711917074e-10\\
30.576650390625	1.52628429650137e-10\\
30.59664453125	2.53476502224167e-10\\
30.616638671875	2.46059296204722e-10\\
30.6366328125	3.5801542536809e-10\\
30.656626953125	4.15002934498809e-10\\
30.67662109375	4.75207659314321e-10\\
30.696615234375	3.20298094455617e-10\\
30.716609375	3.85023816405238e-10\\
30.736603515625	2.65398807231643e-10\\
30.75659765625	2.4133421441385e-10\\
30.776591796875	1.83374340750042e-10\\
30.7965859375	2.27773226281723e-10\\
30.816580078125	1.89111849553952e-10\\
30.83657421875	2.02680880802556e-10\\
30.856568359375	2.28155014984945e-10\\
30.8765625	2.79346644295696e-10\\
30.896556640625	2.86677483687053e-10\\
30.91655078125	1.85707184806979e-10\\
30.936544921875	2.84561282544839e-10\\
30.9565390625	3.39360678272469e-10\\
30.976533203125	3.47965745647999e-10\\
30.99652734375	3.14276370525991e-10\\
31.016521484375	3.24120452606796e-10\\
31.036515625	3.48130778249414e-10\\
31.056509765625	1.45427760332436e-10\\
31.07650390625	2.64236306002265e-10\\
31.096498046875	1.4866322958671e-10\\
31.1164921875	1.29491137718803e-10\\
31.136486328125	1.08056925206719e-10\\
31.15648046875	1.99840447947953e-11\\
31.176474609375	1.03698102930657e-10\\
31.19646875	1.46006975960957e-10\\
31.216462890625	1.91830505913358e-10\\
31.23645703125	1.37355126876442e-10\\
31.256451171875	2.81934415662497e-10\\
31.2764453125	-1.76534394674162e-11\\
31.296439453125	9.77803408464365e-11\\
31.31643359375	2.58751270093038e-11\\
31.336427734375	-2.44847135069112e-11\\
31.356421875	-1.784992652201e-10\\
31.376416015625	-1.8078539499891e-10\\
31.39641015625	-2.55531336520638e-10\\
31.416404296875	-1.75116526823419e-10\\
31.4363984375	-3.23295433882965e-10\\
31.456392578125	-2.73140049955274e-10\\
31.47638671875	-2.81509789273907e-10\\
31.496380859375	-3.0427539616015e-10\\
31.516375	-1.80526097254254e-10\\
31.536369140625	-2.19474611272323e-10\\
31.55636328125	-1.75269637320158e-10\\
31.576357421875	-2.41667775919412e-10\\
31.5963515625	-1.88094569038621e-10\\
31.616345703125	-3.02656049079697e-10\\
31.63633984375	-4.54882341951595e-10\\
31.656333984375	-4.90575298573186e-10\\
31.676328125	-5.24672544919099e-10\\
31.696322265625	-5.93399061686609e-10\\
31.71631640625	-5.48777366043497e-10\\
31.736310546875	-5.85314923295517e-10\\
31.7563046875	-6.22112761899321e-10\\
31.776298828125	-6.77803463701027e-10\\
31.79629296875	-6.7775123174659e-10\\
31.816287109375	-7.01387104414863e-10\\
31.83628125	-7.7253193050741e-10\\
31.856275390625	-6.43371160684022e-10\\
31.87626953125	-6.86159889623791e-10\\
31.896263671875	-7.18445515113151e-10\\
31.9162578125	-5.63120291225131e-10\\
31.936251953125	-6.97652931842589e-10\\
31.95624609375	-6.81509208358252e-10\\
31.976240234375	-6.48641958537532e-10\\
31.996234375	-7.94656646742451e-10\\
32.016228515625	-6.60231475721671e-10\\
32.03622265625	-8.23043673625094e-10\\
32.056216796875	-8.33152470303552e-10\\
32.0762109375	-8.76636751715155e-10\\
32.096205078125	-9.54253272544057e-10\\
32.11619921875	-1.00127839929619e-09\\
32.136193359375	-9.0684469934704e-10\\
32.1561875	-9.95946169245626e-10\\
32.176181640625	-8.68103827818797e-10\\
32.19617578125	-9.06055674592667e-10\\
32.216169921875	-8.13196556670209e-10\\
32.2361640625	-9.10455705668877e-10\\
32.256158203125	-7.7648331716285e-10\\
32.27615234375	-9.21836830474385e-10\\
32.296146484375	-9.18205827308789e-10\\
32.316140625	-1.04072596010167e-09\\
32.336134765625	-9.87462861965641e-10\\
32.35612890625	-1.08169181457559e-09\\
32.376123046875	-1.14766402571951e-09\\
32.3961171875	-1.20246980177866e-09\\
32.416111328125	-1.19967354978607e-09\\
32.43610546875	-1.2151657374735e-09\\
32.456099609375	-1.25872306325666e-09\\
32.47609375	-1.27560210306003e-09\\
32.496087890625	-1.29473807948984e-09\\
32.51608203125	-1.30922263343215e-09\\
32.536076171875	-1.29565741474546e-09\\
32.5560703125	-1.25416051795482e-09\\
32.576064453125	-1.41494668023893e-09\\
32.59605859375	-1.2976422254944e-09\\
32.616052734375	-1.53969484852656e-09\\
32.636046875	-1.46211258981997e-09\\
32.656041015625	-1.55023993284591e-09\\
32.67603515625	-1.58038258957466e-09\\
32.696029296875	-1.42714138825123e-09\\
32.7160234375	-1.57734438815015e-09\\
32.736017578125	-1.45253048465239e-09\\
32.75601171875	-1.45143360819286e-09\\
32.776005859375	-1.52797990046428e-09\\
32.796	-1.48317822240108e-09\\
32.815994140625	-1.49548565128776e-09\\
32.83598828125	-1.35374824942572e-09\\
32.855982421875	-1.43847284058072e-09\\
32.8759765625	-1.30855778446848e-09\\
32.895970703125	-1.44801837349452e-09\\
32.91596484375	-1.34005429585315e-09\\
32.935958984375	-1.32573740037562e-09\\
32.955953125	-1.22782598023572e-09\\
32.975947265625	-1.19624049338619e-09\\
32.99594140625	-1.0483772931143e-09\\
33.015935546875	-1.10668006779337e-09\\
33.0359296875	-9.68791712201792e-10\\
33.055923828125	-9.63989969806638e-10\\
33.07591796875	-1.08773151018227e-09\\
33.095912109375	-1.10433831627779e-09\\
33.11590625	-1.07132570268079e-09\\
33.135900390625	-1.18801891942966e-09\\
33.15589453125	-1.21464169020357e-09\\
33.175888671875	-1.15055904573915e-09\\
33.1958828125	-1.08384063789143e-09\\
33.215876953125	-1.08884070975277e-09\\
33.23587109375	-1.01095615765861e-09\\
33.255865234375	-1.04773505230227e-09\\
33.275859375	-9.85338950950446e-10\\
33.295853515625	-1.02782651752895e-09\\
33.31584765625	-9.97524804176345e-10\\
33.335841796875	-1.03105399413366e-09\\
33.3558359375	-1.01213284077665e-09\\
33.375830078125	-9.07443411072095e-10\\
33.39582421875	-8.25709742332402e-10\\
33.415818359375	-7.72825687566723e-10\\
33.4358125	-7.94036792491226e-10\\
33.455806640625	-8.15012114109209e-10\\
33.47580078125	-7.56454241568101e-10\\
33.495794921875	-8.06913240128769e-10\\
33.5157890625	-8.82819690428883e-10\\
33.535783203125	-9.18325255647335e-10\\
33.55577734375	-8.16025414811107e-10\\
33.575771484375	-8.81285635774488e-10\\
33.595765625	-7.82265913196221e-10\\
33.615759765625	-7.44862680119384e-10\\
33.63575390625	-7.01964435787469e-10\\
33.655748046875	-6.9678995137575e-10\\
33.6757421875	-6.8143790298305e-10\\
33.695736328125	-7.4330765313635e-10\\
33.71573046875	-7.01487094151921e-10\\
33.735724609375	-7.35686318990587e-10\\
33.75571875	-8.04409350918933e-10\\
33.775712890625	-6.51046687017876e-10\\
33.79570703125	-7.68693858181767e-10\\
33.815701171875	-5.91263869707581e-10\\
33.8356953125	-5.80462795688783e-10\\
33.855689453125	-5.3906726340355e-10\\
33.87568359375	-5.16227420618721e-10\\
33.895677734375	-4.59882998285983e-10\\
33.915671875	-4.47984040693247e-10\\
33.935666015625	-3.19041215242997e-10\\
33.95566015625	-2.84736740779882e-10\\
33.975654296875	-3.07276385047731e-10\\
33.9956484375	-3.36015007335061e-10\\
34.015642578125	-4.16671084955365e-10\\
34.03563671875	-4.55062022143912e-10\\
34.055630859375	-4.74437910889308e-10\\
34.075625	-4.72881663489086e-10\\
34.095619140625	-5.25174113191407e-10\\
34.11561328125	-4.22834049936188e-10\\
34.135607421875	-4.38705718411642e-10\\
34.1556015625	-3.24522062079796e-10\\
34.175595703125	-3.23486361214767e-10\\
34.19558984375	-2.51801119721769e-10\\
34.215583984375	-2.57760098319818e-10\\
34.235578125	-8.76345015553601e-11\\
34.255572265625	-2.48273028211821e-10\\
34.27556640625	-1.80669621673187e-10\\
34.295560546875	-1.0688165370869e-10\\
34.3155546875	-1.00389029502412e-10\\
34.335548828125	-1.57054899661928e-10\\
34.35554296875	-2.11830387977522e-11\\
34.375537109375	2.52564354297164e-11\\
34.39553125	3.99747827987882e-11\\
34.415525390625	-4.72470747888902e-11\\
34.43551953125	2.46327682447726e-11\\
34.455513671875	-4.6637594579076e-11\\
34.4755078125	-1.05888916834554e-10\\
34.495501953125	3.93741845264751e-11\\
34.51549609375	1.35074058062633e-11\\
34.535490234375	2.5305482504571e-10\\
34.555484375	2.06160506273724e-10\\
34.575478515625	4.62859306592873e-10\\
34.59547265625	5.04515583080736e-10\\
34.615466796875	4.98975619021007e-10\\
34.6354609375	4.75531962171913e-10\\
34.655455078125	5.26170518024587e-10\\
34.67544921875	3.33599639490636e-10\\
34.695443359375	3.55010569583206e-10\\
34.7154375	2.96372650198382e-10\\
34.735431640625	4.22161866172782e-10\\
34.75542578125	3.73228365910319e-10\\
34.775419921875	5.23119908481178e-10\\
34.7954140625	6.01440646826627e-10\\
34.815408203125	6.99395981217322e-10\\
34.83540234375	7.50541861816896e-10\\
34.855396484375	8.70549924672156e-10\\
34.875390625	8.61570587590715e-10\\
34.895384765625	8.99237224716615e-10\\
34.91537890625	8.70485775054131e-10\\
34.935373046875	8.55305575260817e-10\\
34.9553671875	8.41701567889755e-10\\
34.975361328125	8.22410583397828e-10\\
34.99535546875	8.96563482577408e-10\\
35.015349609375	9.79862490005962e-10\\
35.03534375	1.03924506337505e-09\\
35.055337890625	1.03455636043966e-09\\
35.07533203125	1.18821371888254e-09\\
35.095326171875	1.11411369452132e-09\\
35.1153203125	1.21369826853015e-09\\
35.135314453125	1.16618339704892e-09\\
35.15530859375	1.21594076711517e-09\\
35.175302734375	1.1950081207614e-09\\
35.195296875	1.2000874064842e-09\\
35.215291015625	1.3256646217334e-09\\
35.23528515625	1.25267334671656e-09\\
35.255279296875	1.28386774506772e-09\\
35.2752734375	1.37501988658335e-09\\
35.295267578125	1.37608155152057e-09\\
35.31526171875	1.40262312711026e-09\\
35.335255859375	1.38351152966665e-09\\
35.35525	1.44270979771153e-09\\
35.375244140625	1.34753083618956e-09\\
35.39523828125	1.34006711018949e-09\\
35.415232421875	1.2682007159916e-09\\
35.4352265625	1.32385967229302e-09\\
35.455220703125	1.21627620993238e-09\\
35.47521484375	1.27212859299541e-09\\
35.495208984375	1.09457306124448e-09\\
35.515203125	1.26032950691858e-09\\
35.535197265625	1.14713686915846e-09\\
35.55519140625	1.1844375242203e-09\\
35.575185546875	1.22719030006041e-09\\
35.5951796875	1.27165364372544e-09\\
35.615173828125	1.22853597139437e-09\\
35.63516796875	1.31988841178582e-09\\
35.655162109375	1.28111397865397e-09\\
35.67515625	1.2848525239469e-09\\
35.695150390625	1.25107162267814e-09\\
35.71514453125	1.35727402488706e-09\\
35.735138671875	1.28245085801454e-09\\
35.7551328125	1.3725591484502e-09\\
35.775126953125	1.37707202011807e-09\\
35.79512109375	1.43753317401242e-09\\
35.815115234375	1.36230050935543e-09\\
35.835109375	1.48515234542669e-09\\
35.855103515625	1.34300807375037e-09\\
35.87509765625	1.25796903350719e-09\\
35.895091796875	1.30032994435909e-09\\
35.9150859375	1.24672644052049e-09\\
35.935080078125	1.30895286420906e-09\\
35.95507421875	1.23526326892028e-09\\
35.975068359375	1.29467085795275e-09\\
35.9950625	1.40030383944594e-09\\
36.015056640625	1.42813790463004e-09\\
36.03505078125	1.4689243697953e-09\\
36.055044921875	1.43774066488124e-09\\
36.0750390625	1.47996616618273e-09\\
36.095033203125	1.37065184173781e-09\\
36.11502734375	1.31848372157918e-09\\
36.135021484375	1.2675969522058e-09\\
36.155015625	1.25665160222102e-09\\
36.175009765625	1.19656043095116e-09\\
36.19500390625	1.26625167932771e-09\\
36.214998046875	1.25638559617474e-09\\
36.2349921875	1.30698981167812e-09\\
36.254986328125	1.34652790959788e-09\\
36.27498046875	1.29248102618537e-09\\
36.294974609375	1.39395737579335e-09\\
36.31496875	1.2559282567896e-09\\
36.334962890625	1.28361124347871e-09\\
36.35495703125	1.19089946628336e-09\\
36.374951171875	1.18796927634973e-09\\
36.3949453125	1.06174574036629e-09\\
36.414939453125	1.01958712127715e-09\\
36.43493359375	8.77189225569824e-10\\
36.454927734375	8.06225394545389e-10\\
36.474921875	7.01591125974952e-10\\
36.494916015625	6.61890686264909e-10\\
36.51491015625	8.39514114858511e-10\\
36.534904296875	7.94746727127439e-10\\
36.5548984375	8.9493022197951e-10\\
36.574892578125	9.19448131795338e-10\\
36.59488671875	1.04745356271712e-09\\
36.614880859375	9.4338707017105e-10\\
36.634875	9.21414785406438e-10\\
36.654869140625	7.47583051318383e-10\\
36.67486328125	6.74942622182456e-10\\
36.694857421875	5.55786288846661e-10\\
36.7148515625	6.3380645813657e-10\\
36.734845703125	5.10990186000358e-10\\
36.75483984375	5.89894031199811e-10\\
36.774833984375	5.34863526308534e-10\\
36.794828125	4.99125752136567e-10\\
36.814822265625	5.50621228375683e-10\\
36.83481640625	4.71293229137519e-10\\
36.854810546875	4.42028085412743e-10\\
36.8748046875	3.93684565822416e-10\\
36.894798828125	3.83394725347744e-10\\
36.91479296875	5.0867099684663e-10\\
36.934787109375	4.89972723304907e-10\\
36.95478125	5.14191320306781e-10\\
36.974775390625	5.33082751195668e-10\\
36.99476953125	4.72651479986296e-10\\
37.014763671875	3.7693131766247e-10\\
37.0347578125	1.54621868319265e-10\\
37.054751953125	2.71541389678658e-10\\
37.07474609375	8.78332026889973e-11\\
37.094740234375	1.48907379940865e-10\\
37.114734375	3.16881010571933e-11\\
37.134728515625	1.09919630039666e-10\\
37.15472265625	5.51664264714024e-11\\
37.174716796875	1.21552256396454e-10\\
37.1947109375	1.55898033994015e-10\\
37.214705078125	2.46735333945155e-10\\
37.23469921875	2.14659082339881e-10\\
37.254693359375	2.77342670579367e-10\\
37.2746875	1.80340803904316e-10\\
37.294681640625	2.21536150994604e-10\\
37.31467578125	1.07348271476807e-10\\
37.334669921875	1.20984416571347e-10\\
37.3546640625	3.41000444133905e-11\\
37.374658203125	-3.93082201583128e-11\\
37.39465234375	-3.59326848715511e-11\\
37.414646484375	3.38451794075811e-11\\
37.434640625	2.19391729422397e-11\\
37.454634765625	1.58295722045201e-11\\
37.47462890625	-1.00155700630814e-13\\
37.494623046875	-1.08519656646119e-12\\
37.5146171875	-3.49947914146004e-11\\
37.534611328125	-6.71027678618253e-11\\
37.55460546875	-4.99958297167133e-11\\
37.574599609375	-2.21940430429563e-10\\
37.59459375	-4.68634631466133e-11\\
37.614587890625	-1.58877476432236e-10\\
37.63458203125	-1.64949763088231e-10\\
37.654576171875	-1.49322385497588e-10\\
37.6745703125	-1.80526099581921e-10\\
37.694564453125	-2.94481608576409e-10\\
37.71455859375	-3.85864398352934e-10\\
37.734552734375	-3.54012852254023e-10\\
37.754546875	-3.93837243328435e-10\\
37.774541015625	-4.19948537594456e-10\\
37.79453515625	-3.97481163458345e-10\\
37.814529296875	-4.04886688545674e-10\\
37.8345234375	-3.14525011897033e-10\\
37.854517578125	-3.48682204284248e-10\\
};
\addplot [color=mycolor3,solid]
  table[row sep=crcr]{%
37.854517578125	-3.48682204284248e-10\\
37.87451171875	-1.52228302109132e-10\\
37.894505859375	-2.66432995661225e-10\\
37.9145	-2.63250399327523e-10\\
37.934494140625	-1.85672648257793e-10\\
37.95448828125	-1.3643845619084e-10\\
37.974482421875	-1.52435184712923e-10\\
37.9944765625	-6.95695547283366e-11\\
38.014470703125	-1.1803189671861e-10\\
38.03446484375	-5.68376669130866e-13\\
38.054458984375	-7.83982475017505e-11\\
38.074453125	-7.49005074838422e-11\\
38.094447265625	-2.36487308329265e-11\\
38.11444140625	-1.76311605696616e-10\\
38.134435546875	-1.21312763031284e-10\\
38.1544296875	-6.03332610003422e-11\\
38.174423828125	-1.75510417619406e-10\\
38.19441796875	-1.37466696852242e-10\\
38.214412109375	-1.47088122647725e-10\\
38.23440625	-1.46113663190398e-10\\
38.254400390625	-2.15010539769531e-10\\
38.27439453125	-1.17172432331863e-10\\
38.294388671875	-2.03166564690931e-10\\
38.3143828125	-1.44655021163029e-10\\
38.334376953125	-2.76605160529541e-10\\
38.35437109375	-7.52893262836563e-11\\
38.374365234375	-5.52381401257428e-12\\
38.394359375	9.59529236992806e-11\\
38.414353515625	9.83159240150961e-11\\
38.43434765625	1.67374808556225e-10\\
38.454341796875	1.2004194789963e-10\\
38.4743359375	7.6974420150146e-11\\
38.494330078125	8.32807089032188e-13\\
38.51432421875	-8.97535862628537e-11\\
38.534318359375	-2.05010213846639e-10\\
38.5543125	-2.6516904883061e-10\\
38.574306640625	-2.11704578413167e-10\\
38.59430078125	-1.89152590864453e-10\\
38.614294921875	-1.37915132801793e-10\\
38.6342890625	-7.49863946471219e-11\\
38.654283203125	-1.63932688594337e-12\\
38.67427734375	1.217361225685e-11\\
38.694271484375	1.45066350488148e-11\\
38.714265625	-6.94019867402271e-11\\
38.734259765625	-3.04344019664899e-11\\
38.75425390625	-1.61291071122131e-10\\
38.774248046875	-1.91218919118155e-10\\
38.7942421875	-1.43009908407417e-10\\
38.814236328125	-4.28005948497378e-11\\
38.83423046875	-9.95161207376076e-11\\
38.854224609375	-5.73915707182656e-11\\
38.87421875	-3.03195653327358e-11\\
38.894212890625	1.09538375228639e-10\\
38.91420703125	2.07325311933031e-10\\
38.934201171875	2.70394019999711e-10\\
38.9541953125	3.52758850091536e-10\\
38.974189453125	3.97254292457195e-10\\
38.99418359375	4.67930983545809e-10\\
39.014177734375	2.59435261796583e-10\\
39.034171875	2.57565023184724e-10\\
39.054166015625	5.32121341047746e-11\\
39.07416015625	5.41461759315112e-11\\
39.094154296875	-1.97769943013338e-10\\
39.1141484375	-1.39825804373422e-11\\
39.134142578125	2.06756382271976e-11\\
39.15413671875	1.79384373677745e-10\\
39.174130859375	2.18327434121021e-10\\
39.194125	3.22376769771793e-10\\
39.214119140625	2.88419529597463e-10\\
39.23411328125	2.85354956268737e-10\\
39.254107421875	2.26491227933473e-10\\
39.2741015625	2.35725588972615e-10\\
39.294095703125	2.04400923828435e-10\\
39.31408984375	1.67960669734905e-10\\
39.334083984375	2.40074542901772e-10\\
39.354078125	2.35872056089041e-10\\
39.374072265625	3.22176049805513e-10\\
39.39406640625	2.87144834146957e-10\\
39.414060546875	2.83635815565199e-10\\
39.4340546875	2.71341183151412e-10\\
39.454048828125	2.35298851150092e-10\\
39.47404296875	2.44599702007127e-10\\
39.494037109375	2.41815577926603e-10\\
39.51403125	2.32345143239948e-10\\
39.534025390625	3.52692915805401e-10\\
39.55401953125	1.91842625865413e-10\\
39.574013671875	4.35443782450453e-10\\
39.5940078125	2.89548291681391e-10\\
39.614001953125	4.96306488854303e-10\\
39.63399609375	3.66843541170368e-10\\
39.653990234375	4.22012119677438e-10\\
39.673984375	3.09302049587237e-10\\
39.693978515625	3.64342915107336e-10\\
39.71397265625	1.35918748780131e-10\\
39.733966796875	1.51528899646148e-10\\
39.7539609375	6.69311808951309e-11\\
39.773955078125	1.24345859390562e-10\\
39.79394921875	6.51487305466155e-11\\
39.813943359375	2.07348083687481e-10\\
39.8339375	1.96326341086916e-10\\
39.853931640625	2.59928187691905e-10\\
39.87392578125	2.86428681095556e-10\\
39.893919921875	2.23174049876152e-10\\
39.9139140625	2.3607395458152e-10\\
39.933908203125	1.81125181051536e-10\\
39.95390234375	2.02386725162235e-10\\
39.973896484375	1.57069652653376e-10\\
39.993890625	1.70377580013409e-10\\
40.013884765625	2.09893429770395e-10\\
40.03387890625	1.69957365266015e-10\\
40.053873046875	3.28370502662351e-10\\
40.0738671875	2.46877288059832e-10\\
40.093861328125	2.85539483178211e-10\\
40.11385546875	3.10573501575859e-10\\
40.133849609375	2.69658781731143e-10\\
40.15384375	2.99944472527759e-10\\
40.173837890625	2.98027520041182e-10\\
40.19383203125	3.76075373312989e-10\\
40.213826171875	4.59914086497198e-10\\
40.2338203125	4.67867054093542e-10\\
40.253814453125	5.09307658777699e-10\\
40.27380859375	5.21993578247966e-10\\
40.293802734375	5.09415930791181e-10\\
40.313796875	4.96679557351266e-10\\
40.333791015625	3.81795252214807e-10\\
40.35378515625	4.74917139519638e-10\\
40.373779296875	3.65433657985639e-10\\
40.3937734375	4.66993676069594e-10\\
40.413767578125	4.14508721339844e-10\\
40.43376171875	4.19606688089662e-10\\
40.453755859375	4.02800227043819e-10\\
40.47375	3.87430579679791e-10\\
40.493744140625	3.49086552571869e-10\\
40.51373828125	2.9382443515059e-10\\
40.533732421875	2.10587145062205e-10\\
40.5537265625	1.98809188636532e-10\\
40.573720703125	1.34782700013641e-10\\
40.59371484375	7.99732298234148e-11\\
40.613708984375	1.79789443416773e-10\\
40.633703125	7.25202791786481e-11\\
40.653697265625	1.71665778732291e-10\\
40.67369140625	1.80022982980499e-10\\
40.693685546875	2.20777410552486e-10\\
40.7136796875	1.71926671540119e-10\\
40.733673828125	1.99800954811949e-10\\
40.75366796875	2.0894403037598e-10\\
40.773662109375	1.47542487593801e-10\\
40.79365625	2.00716514087049e-10\\
40.813650390625	1.1705703350894e-10\\
40.83364453125	1.66595018163291e-10\\
40.853638671875	6.64016373553358e-11\\
40.8736328125	1.02501970503982e-10\\
40.893626953125	2.46917168415991e-11\\
40.91362109375	-5.93189143596085e-11\\
40.933615234375	-7.97752149845483e-11\\
40.953609375	-5.61904806066818e-11\\
40.973603515625	-2.0445461218511e-12\\
40.99359765625	1.5848404602103e-11\\
41.013591796875	1.01238611332288e-10\\
41.0335859375	1.75084447931828e-10\\
41.053580078125	1.43141396833462e-10\\
41.07357421875	1.67955368404742e-10\\
41.093568359375	1.96862659678205e-10\\
41.1135625	1.25280898936069e-10\\
41.133556640625	1.14273589669803e-11\\
41.15355078125	7.06165419484456e-12\\
41.173544921875	-3.84889896469618e-11\\
41.1935390625	-2.89204032292564e-11\\
41.213533203125	2.57369685447639e-11\\
41.23352734375	-2.73269096525757e-11\\
41.253521484375	9.91023916446599e-11\\
41.273515625	5.70340967639169e-11\\
41.293509765625	1.36479996784628e-11\\
41.31350390625	4.3626209782693e-12\\
41.333498046875	3.79567197774365e-11\\
41.3534921875	6.43827394351837e-12\\
41.373486328125	-2.19242444695874e-11\\
41.39348046875	-2.17164661037081e-11\\
41.413474609375	-1.66413005336441e-10\\
41.43346875	-1.46520322755524e-10\\
41.453462890625	-2.0821813770161e-10\\
41.47345703125	-2.19824506777776e-10\\
41.493451171875	-3.48708091987202e-10\\
41.5134453125	-1.88717096145184e-10\\
41.533439453125	-2.39284240699213e-10\\
41.55343359375	-1.72726980259325e-10\\
41.573427734375	-1.43934560643001e-10\\
41.593421875	-2.78279129498272e-11\\
41.613416015625	-5.99138147510461e-11\\
41.63341015625	-5.4301456799118e-11\\
41.653404296875	-7.13937953205144e-11\\
41.6733984375	-1.04000987596129e-10\\
41.693392578125	-2.31837332860606e-10\\
41.71338671875	-2.02926129592719e-10\\
41.733380859375	-2.49499794162717e-10\\
41.753375	-2.33716680887462e-10\\
41.773369140625	-3.20293137232371e-10\\
41.79336328125	-2.74517334475945e-10\\
41.813357421875	-2.84253559617128e-10\\
41.8333515625	-3.10356988695799e-10\\
41.853345703125	-3.33777935908801e-10\\
41.87333984375	-3.54202167563467e-10\\
41.893333984375	-3.17368673698298e-10\\
41.913328125	-3.56619831437921e-10\\
41.933322265625	-3.03385566627605e-10\\
41.95331640625	-3.1258088004247e-10\\
41.973310546875	-3.6768994798193e-10\\
41.9933046875	-3.83003774980813e-10\\
42.013298828125	-3.11163607104599e-10\\
42.03329296875	-3.52265558311371e-10\\
42.053287109375	-3.05530124574991e-10\\
42.07328125	-4.76965767719823e-10\\
42.093275390625	-3.61787940730868e-10\\
42.11326953125	-5.40341942286891e-10\\
42.133263671875	-4.39515961808364e-10\\
42.1532578125	-4.81176357237277e-10\\
42.173251953125	-5.2555287056404e-10\\
42.19324609375	-5.10939235890979e-10\\
42.213240234375	-3.95912605030658e-10\\
42.233234375	-3.98868374852091e-10\\
42.253228515625	-2.77737378546977e-10\\
42.27322265625	-2.30399768067463e-10\\
42.293216796875	-2.36586529108973e-10\\
42.3132109375	-3.06584999867553e-10\\
42.333205078125	-2.41655736908132e-10\\
42.35319921875	-2.83829384945311e-10\\
42.373193359375	-3.58426928774281e-10\\
42.3931875	-3.09351469682723e-10\\
42.413181640625	-3.5351107516001e-10\\
42.43317578125	-2.63077824323532e-10\\
42.453169921875	-2.79535057411275e-10\\
42.4731640625	-2.83582075774813e-10\\
42.493158203125	-3.60465598455588e-10\\
42.51315234375	-3.21867918781681e-10\\
42.533146484375	-3.72725856899899e-10\\
42.553140625	-3.63877416468441e-10\\
42.573134765625	-3.37657991862322e-10\\
42.59312890625	-3.99596486522768e-10\\
42.613123046875	-4.61473013540979e-10\\
42.6331171875	-4.03967323658651e-10\\
42.653111328125	-4.47065252230526e-10\\
42.67310546875	-4.40006939085256e-10\\
42.693099609375	-4.18497580392724e-10\\
42.71309375	-4.81262851985064e-10\\
42.733087890625	-4.94965166349014e-10\\
42.75308203125	-4.92175474399382e-10\\
42.773076171875	-5.77604420436395e-10\\
42.7930703125	-5.60764427886112e-10\\
42.813064453125	-5.80074453318437e-10\\
42.83305859375	-5.74575359065755e-10\\
42.853052734375	-5.87069176449354e-10\\
42.873046875	-4.72085435732246e-10\\
42.893041015625	-5.51864914113963e-10\\
42.91303515625	-5.25070517655892e-10\\
42.933029296875	-6.03883255468131e-10\\
42.9530234375	-6.08934071540719e-10\\
42.973017578125	-5.48482596808179e-10\\
42.99301171875	-4.53405556520439e-10\\
43.013005859375	-3.86894326659168e-10\\
43.033	-3.30701970523128e-10\\
43.052994140625	-2.59854889690764e-10\\
43.07298828125	-2.44723708889956e-10\\
43.092982421875	-2.53823644613521e-10\\
43.1129765625	-3.324839787526e-10\\
43.132970703125	-2.76467651722659e-10\\
43.15296484375	-3.75532949457023e-10\\
43.172958984375	-4.38360269206743e-10\\
43.192953125	-3.7579538665403e-10\\
43.212947265625	-3.89725788930294e-10\\
43.23294140625	-4.09291121451557e-10\\
43.252935546875	-3.06838817896644e-10\\
43.2729296875	-2.58937200965645e-10\\
43.292923828125	-3.55876653907471e-10\\
43.31291796875	-2.40437913282101e-10\\
43.332912109375	-2.1423466151683e-10\\
43.35290625	-2.19973358576112e-10\\
43.372900390625	-1.65774974625695e-10\\
43.39289453125	-5.56914711850641e-11\\
43.412888671875	-9.6617911423194e-11\\
43.4328828125	-1.58362986326299e-11\\
43.452876953125	-6.614023918357e-11\\
43.47287109375	-7.18618899699147e-11\\
43.492865234375	7.84166736542153e-13\\
43.512859375	-7.25201041044493e-11\\
43.532853515625	-1.53243588347667e-10\\
43.55284765625	-1.16434762048652e-10\\
43.572841796875	-2.20723661898546e-10\\
43.5928359375	-1.46084711954286e-10\\
43.612830078125	-1.86682472458416e-10\\
43.63282421875	-1.03638025672949e-10\\
43.652818359375	-1.3278689820118e-10\\
43.6728125	-1.60513256587215e-11\\
43.692806640625	-1.04527287997595e-10\\
43.71280078125	-7.3628676547644e-12\\
43.732794921875	2.17240400460627e-11\\
43.7527890625	-1.40743008890469e-11\\
43.772783203125	4.70138654236e-11\\
43.79277734375	2.95431538389755e-11\\
43.812771484375	3.88850895299968e-11\\
43.832765625	3.01888274235239e-12\\
43.852759765625	8.92206742146631e-11\\
43.87275390625	-7.0745651929423e-11\\
43.892748046875	4.02936323170382e-11\\
43.9127421875	7.2762122817248e-11\\
43.932736328125	6.82527708228258e-11\\
43.95273046875	1.35098159651337e-10\\
43.972724609375	2.43233142140759e-10\\
43.99271875	2.75740105275989e-10\\
44.012712890625	2.48499017189839e-10\\
44.03270703125	2.56257175548467e-10\\
44.052701171875	2.37683951983172e-10\\
44.0726953125	2.08327129976959e-10\\
44.092689453125	4.49593153064855e-11\\
44.11268359375	8.36467115730823e-11\\
44.132677734375	-2.97491347344792e-11\\
44.152671875	1.04734800238855e-11\\
44.172666015625	2.17826677075048e-11\\
44.19266015625	5.56829668262493e-12\\
44.212654296875	9.555262860783e-11\\
44.2326484375	1.58421694758689e-10\\
44.252642578125	1.38197802541832e-10\\
44.27263671875	2.04811525939641e-10\\
44.292630859375	1.36299481737266e-10\\
44.312625	1.87584787619575e-10\\
44.332619140625	1.60238381158504e-10\\
44.35261328125	2.00912882208202e-10\\
44.372607421875	1.49778102650031e-10\\
44.3926015625	7.88263059408758e-11\\
44.412595703125	5.91659844307327e-11\\
44.43258984375	1.58567661478583e-10\\
44.452583984375	1.01670171281441e-10\\
44.472578125	1.60798196623745e-10\\
44.492572265625	1.18955450234765e-10\\
44.51256640625	2.25715818523818e-10\\
44.532560546875	2.08605902610753e-10\\
44.5525546875	1.89266905422709e-10\\
44.572548828125	1.87968031096963e-10\\
44.59254296875	1.31245833668322e-10\\
44.612537109375	2.3785776594999e-10\\
44.63253125	1.6709281428982e-10\\
44.652525390625	3.13874260722569e-10\\
44.67251953125	2.91989297212206e-10\\
44.692513671875	2.35652949614758e-10\\
44.7125078125	2.088987592346e-10\\
44.732501953125	1.83581004180126e-10\\
44.75249609375	9.02878186479531e-11\\
44.772490234375	6.91563769184239e-11\\
44.792484375	1.19585237914934e-10\\
44.812478515625	1.19471716738671e-10\\
44.83247265625	4.52291994628149e-11\\
44.852466796875	1.44421633026944e-10\\
44.8724609375	1.1957739428749e-10\\
44.892455078125	2.14338927373981e-10\\
44.91244921875	1.99501626150649e-10\\
44.932443359375	1.1148139980523e-10\\
44.9524375	1.23902748585235e-10\\
44.972431640625	1.61469944208392e-10\\
44.99242578125	1.34732028113367e-10\\
45.012419921875	8.68810286279905e-11\\
45.0324140625	2.15215678394476e-10\\
45.052408203125	2.35249981288116e-10\\
45.07240234375	1.88056643959522e-10\\
45.092396484375	3.05523648478939e-10\\
45.112390625	3.10053759151632e-10\\
45.132384765625	3.03503039870434e-10\\
45.15237890625	3.77772054506843e-10\\
45.172373046875	2.69685119616672e-10\\
45.1923671875	3.35688661527044e-10\\
45.212361328125	3.78248222584199e-10\\
45.23235546875	3.49002601511836e-10\\
45.252349609375	3.7585391909582e-10\\
45.27234375	3.95553346912366e-10\\
45.292337890625	3.4236989005996e-10\\
45.31233203125	4.27977625418128e-10\\
45.332326171875	4.30458779394572e-10\\
45.3523203125	4.88076702340266e-10\\
45.372314453125	4.37869295887711e-10\\
45.39230859375	4.63510398271938e-10\\
45.412302734375	4.34908427371248e-10\\
45.432296875	5.2411807803667e-10\\
45.452291015625	4.36675972035666e-10\\
45.47228515625	4.48291507548809e-10\\
45.492279296875	3.51956185333652e-10\\
45.5122734375	2.45705110722655e-10\\
45.532267578125	2.634883759985e-10\\
45.55226171875	2.02419498344616e-10\\
45.572255859375	1.21222856357872e-10\\
45.59225	1.35779883200947e-10\\
45.612244140625	2.27580998225544e-10\\
45.63223828125	2.13152742146759e-10\\
45.652232421875	2.54451301949255e-10\\
45.6722265625	2.96222455623468e-10\\
45.692220703125	2.42326945549468e-10\\
45.71221484375	3.89185824377986e-10\\
45.732208984375	3.9134096736034e-10\\
45.752203125	3.25700244950895e-10\\
45.772197265625	3.40292073479802e-10\\
45.79219140625	3.04617967407146e-10\\
45.812185546875	2.28232610089214e-10\\
45.8321796875	2.97833943360271e-10\\
45.852173828125	2.26607644634777e-10\\
45.87216796875	2.80900132612581e-10\\
45.892162109375	2.06880370717678e-10\\
45.91215625	2.22223324885418e-10\\
45.932150390625	1.86510721886328e-10\\
45.95214453125	1.66593259498134e-10\\
45.972138671875	1.59856516680341e-10\\
45.9921328125	1.59549794174452e-10\\
46.012126953125	1.95586886512844e-10\\
46.03212109375	1.94688968579877e-10\\
46.052115234375	2.82589260094777e-10\\
46.072109375	2.89670441622017e-10\\
46.092103515625	2.26979275365397e-10\\
46.11209765625	3.77569495933243e-10\\
46.132091796875	2.57191709202289e-10\\
46.1520859375	2.75495351998896e-10\\
46.172080078125	1.80164784903452e-10\\
46.19207421875	2.92106195717884e-10\\
46.212068359375	2.31083195573037e-10\\
46.2320625	1.81028164788386e-10\\
46.252056640625	2.03851211665773e-10\\
46.27205078125	1.83445660734569e-10\\
46.292044921875	2.0813490168263e-10\\
46.3120390625	2.04134422178694e-10\\
46.332033203125	2.00011866042975e-10\\
46.35202734375	1.86800748205817e-10\\
46.372021484375	2.49879986703122e-10\\
46.392015625	2.41321133974786e-10\\
46.412009765625	1.44836931161822e-10\\
46.43200390625	1.60945454226011e-10\\
46.451998046875	1.39977591842718e-10\\
46.4719921875	1.81832709074762e-11\\
46.491986328125	8.07128647221482e-11\\
46.51198046875	-5.59275717667582e-11\\
46.531974609375	-1.83132111063923e-11\\
46.55196875	-5.78427009761172e-11\\
46.571962890625	-4.21344848788954e-11\\
46.59195703125	1.80188875726724e-11\\
46.611951171875	6.76631954742611e-11\\
46.6319453125	1.52100346601022e-10\\
46.651939453125	2.48287865854591e-10\\
46.67193359375	1.48358443251892e-10\\
46.691927734375	1.94496948137671e-10\\
46.711921875	8.20791614052468e-11\\
46.731916015625	5.02973549473215e-11\\
46.75191015625	5.98580503986627e-11\\
46.771904296875	-6.48276700577976e-11\\
46.7918984375	-4.84726781335438e-11\\
46.811892578125	-9.41654787116466e-11\\
46.83188671875	-4.51895023811583e-11\\
46.851880859375	-4.69681571951756e-11\\
46.871875	4.96322720395787e-12\\
46.891869140625	-6.72614447808908e-11\\
46.91186328125	-1.46031719713494e-11\\
46.931857421875	-8.81281012450825e-12\\
46.9518515625	-1.23599280303943e-11\\
46.971845703125	-5.88452211613927e-11\\
46.99183984375	-4.35005178506619e-11\\
47.011833984375	-1.10019899445654e-11\\
47.031828125	-7.44561493462616e-11\\
47.051822265625	3.22346111244778e-12\\
47.07181640625	-3.11977110689977e-11\\
47.091810546875	6.7436712636911e-14\\
47.1118046875	-6.12375662185415e-11\\
47.131798828125	-2.96285995318399e-11\\
47.15179296875	-1.80678175203067e-10\\
47.171787109375	-1.28775806001332e-10\\
47.19178125	-1.36911290387161e-10\\
47.211775390625	-1.28966728105077e-10\\
47.23176953125	-4.62598291216285e-11\\
47.251763671875	-3.66984096393413e-11\\
47.2717578125	8.0660391398332e-11\\
47.291751953125	-2.61785551545542e-11\\
47.31174609375	5.03926291334175e-11\\
47.331740234375	5.40056787886274e-11\\
47.351734375	-1.0945419067672e-10\\
47.371728515625	-5.62488528140436e-11\\
47.39172265625	-1.12871718231749e-10\\
47.411716796875	-8.03095675689923e-11\\
47.4317109375	-1.15421495285507e-10\\
47.451705078125	-4.13094118777238e-11\\
47.47169921875	-4.13989023553326e-11\\
47.491693359375	4.61008044456946e-11\\
47.5116875	1.42281676103253e-11\\
47.531681640625	2.09253201936647e-11\\
47.55167578125	-2.0711421936463e-11\\
47.571669921875	5.18651522091432e-11\\
47.5916640625	-8.57461251021925e-11\\
47.611658203125	-1.31128686496197e-10\\
47.63165234375	-1.83225355100771e-10\\
47.651646484375	-2.12933041028451e-10\\
47.671640625	-1.80835665759734e-10\\
47.691634765625	-2.02520747573767e-10\\
47.71162890625	-2.23451138053055e-10\\
47.731623046875	-1.89409247143636e-10\\
47.7516171875	-1.09475458948308e-10\\
47.771611328125	-2.48427339392265e-10\\
47.79160546875	-9.94012533955513e-11\\
47.811599609375	-1.62520078016661e-10\\
47.83159375	-1.23628393224054e-10\\
47.851587890625	-2.29788429437355e-10\\
47.87158203125	-1.91715891032605e-10\\
47.891576171875	-2.86056227934322e-10\\
47.9115703125	-2.72829400816188e-10\\
47.931564453125	-3.23849946894365e-10\\
47.95155859375	-1.88853769520992e-10\\
47.971552734375	-1.68300285288701e-10\\
47.991546875	-1.3686880556587e-10\\
48.011541015625	-3.12843885898815e-11\\
48.03153515625	-9.38518792314764e-12\\
48.051529296875	-8.95795894187546e-12\\
48.0715234375	2.32012784085559e-11\\
48.091517578125	4.84598894026426e-11\\
48.11151171875	3.9020553335143e-11\\
48.131505859375	-1.25059814496459e-11\\
48.1515	5.72346550117857e-11\\
48.171494140625	9.35471452848812e-11\\
48.19148828125	7.8053894968812e-11\\
48.211482421875	1.74964070638308e-11\\
48.2314765625	1.95641934969665e-11\\
48.251470703125	1.77693254537474e-11\\
48.27146484375	-4.24498511700818e-11\\
48.291458984375	-1.19062349182944e-11\\
48.311453125	1.01819559197531e-10\\
48.331447265625	-7.87556162847898e-12\\
48.35144140625	7.31203572016946e-11\\
48.371435546875	1.18856446184432e-10\\
48.3914296875	1.0085189945461e-10\\
48.411423828125	1.10351874614145e-10\\
48.43141796875	1.85173339529232e-10\\
48.451412109375	1.69134183927514e-10\\
48.47140625	1.55179150357193e-10\\
48.491400390625	1.88257128350252e-10\\
48.51139453125	1.70877043151897e-10\\
48.531388671875	2.45671661429806e-10\\
48.5513828125	1.29765998043579e-10\\
48.571376953125	1.5621251513562e-10\\
48.59137109375	1.46536066950491e-10\\
48.611365234375	1.23805017815716e-10\\
48.631359375	1.81917031203301e-10\\
48.651353515625	1.87590420134323e-10\\
48.67134765625	2.03365697491688e-10\\
48.691341796875	2.2342170909186e-10\\
48.7113359375	2.14176244510544e-10\\
48.731330078125	1.19747833278071e-10\\
48.75132421875	1.80522822244243e-10\\
48.771318359375	1.58549669688037e-10\\
48.7913125	2.04588605512029e-10\\
48.811306640625	2.49813550107701e-10\\
48.83130078125	2.03241091698001e-10\\
48.851294921875	2.16748278165412e-10\\
48.8712890625	2.66064224491563e-10\\
48.891283203125	2.85002851417565e-10\\
48.91127734375	2.82975428195033e-10\\
48.931271484375	3.25906251816544e-10\\
48.951265625	3.89384405606442e-10\\
48.971259765625	3.23413684949649e-10\\
48.99125390625	2.65869070979554e-10\\
49.011248046875	3.94101876801085e-10\\
49.0312421875	3.55411640263033e-10\\
49.051236328125	3.41097163453227e-10\\
49.07123046875	3.17754554294811e-10\\
49.091224609375	3.51337186591293e-10\\
49.11121875	3.04945760992563e-10\\
49.131212890625	3.56907092530895e-10\\
49.15120703125	3.88769050541584e-10\\
49.171201171875	3.29778780280127e-10\\
49.1911953125	3.00430451779725e-10\\
49.211189453125	3.33092199200385e-10\\
49.23118359375	3.6045518310647e-10\\
49.251177734375	3.43919357729711e-10\\
49.271171875	3.74514005279016e-10\\
49.291166015625	3.6747595574055e-10\\
49.31116015625	3.92381082373128e-10\\
49.331154296875	3.61495577505018e-10\\
49.3511484375	4.10981409808433e-10\\
49.371142578125	3.48655551997654e-10\\
49.39113671875	3.6219581375033e-10\\
49.411130859375	3.49153066378379e-10\\
49.431125	3.7771485025465e-10\\
49.451119140625	3.19786867244592e-10\\
49.47111328125	3.47024614465538e-10\\
49.491107421875	3.67712293164607e-10\\
49.5111015625	3.43637425401673e-10\\
49.531095703125	4.23557005030221e-10\\
49.55108984375	3.57987614351249e-10\\
49.571083984375	4.19419108407184e-10\\
49.591078125	4.4536679514922e-10\\
49.611072265625	4.55055698690296e-10\\
49.63106640625	4.69999305968722e-10\\
49.651060546875	4.03102869227385e-10\\
49.6710546875	3.77374533722974e-10\\
49.691048828125	4.14052087678891e-10\\
49.71104296875	3.760659190804e-10\\
49.731037109375	3.5821667387442e-10\\
49.75103125	3.57944788726702e-10\\
49.771025390625	3.89658991884625e-10\\
49.79101953125	3.97045807632574e-10\\
49.811013671875	3.27768045092233e-10\\
49.8310078125	4.16760843733014e-10\\
49.851001953125	4.07940825348848e-10\\
49.87099609375	3.55213714174121e-10\\
49.890990234375	4.04587946562126e-10\\
49.910984375	3.08640347955894e-10\\
49.930978515625	3.47094424470562e-10\\
49.95097265625	2.52864371256758e-10\\
49.970966796875	2.98432516742807e-10\\
49.9909609375	2.08944529624567e-10\\
50.010955078125	2.3235218504928e-10\\
50.03094921875	2.30409091199086e-10\\
50.050943359375	2.19008654730987e-10\\
50.0709375	2.42668515955488e-10\\
50.090931640625	2.87134474378701e-10\\
50.11092578125	2.75510089959477e-10\\
50.130919921875	3.07286443843935e-10\\
50.1509140625	3.82511767860229e-10\\
50.170908203125	4.52087710028853e-10\\
50.19090234375	3.20068274674374e-10\\
50.210896484375	4.35166467793256e-10\\
50.230890625	4.12709373815384e-10\\
50.250884765625	3.67162694535411e-10\\
50.27087890625	3.74858575375178e-10\\
50.290873046875	3.07494368306292e-10\\
50.3108671875	3.44268457072532e-10\\
50.330861328125	2.94865662641845e-10\\
50.35085546875	3.19778170366801e-10\\
50.370849609375	3.60311403731699e-10\\
50.39084375	4.71095277638123e-10\\
50.410837890625	4.27355288640001e-10\\
50.43083203125	5.23686913502206e-10\\
50.450826171875	4.2639308455474e-10\\
50.4708203125	4.79117096032324e-10\\
50.490814453125	3.81960587245164e-10\\
50.51080859375	3.4021520787416e-10\\
50.530802734375	2.83683711269426e-10\\
50.550796875	2.22248258604375e-10\\
50.570791015625	2.44873784930605e-10\\
50.59078515625	2.628093624543e-10\\
50.610779296875	3.0617743101428e-10\\
50.6307734375	3.25650883980055e-10\\
50.650767578125	3.45357439533242e-10\\
50.67076171875	3.57016062851651e-10\\
50.690755859375	3.17558840725608e-10\\
50.71075	3.11018366100905e-10\\
50.730744140625	2.36071050140896e-10\\
50.75073828125	2.79045883149263e-10\\
50.770732421875	3.18253969111127e-10\\
50.7907265625	2.46630395926449e-10\\
50.810720703125	3.09812821936055e-10\\
50.83071484375	2.52866442720596e-10\\
50.850708984375	2.19855380404944e-10\\
50.870703125	2.35648674511309e-10\\
50.890697265625	1.80101709264477e-10\\
50.91069140625	2.15276165540124e-10\\
50.930685546875	1.35607344951586e-10\\
50.9506796875	1.52819373054934e-10\\
50.970673828125	1.56854235798608e-10\\
50.99066796875	1.73085655605305e-10\\
51.010662109375	1.33880614256135e-10\\
51.03065625	4.26813607296275e-11\\
51.050650390625	1.021674364405e-10\\
51.07064453125	2.78606609968045e-11\\
51.090638671875	-1.12015840527641e-11\\
51.1106328125	-7.91230076496463e-11\\
51.130626953125	-5.64407671148981e-11\\
51.15062109375	-1.10819181774938e-11\\
51.170615234375	-7.91142311671032e-11\\
51.190609375	-6.2216925958255e-11\\
51.210603515625	-8.22754188489818e-11\\
51.23059765625	-1.13101196019508e-10\\
51.250591796875	-1.2591176042863e-10\\
51.2705859375	-2.39557031196336e-11\\
51.290580078125	-1.16396692163179e-10\\
51.31057421875	-1.06777006141555e-10\\
51.330568359375	-1.59927230756713e-10\\
51.3505625	-1.80858574741722e-10\\
51.370556640625	-2.15979398053405e-10\\
51.39055078125	-1.63810816905799e-10\\
51.410544921875	-1.35590480387997e-10\\
51.4305390625	-1.88353258220867e-10\\
51.450533203125	-2.08690580031139e-10\\
51.47052734375	-1.36405566011649e-10\\
51.490521484375	-1.41624963593369e-10\\
51.510515625	-1.86563021193912e-10\\
51.530509765625	-1.53893186293653e-10\\
51.55050390625	-2.00977125884514e-10\\
51.570498046875	-2.33186865395061e-10\\
51.5904921875	-2.81596890994332e-10\\
51.610486328125	-2.27813956742382e-10\\
51.63048046875	-2.3356066202157e-10\\
51.650474609375	-1.85815217103905e-10\\
51.67046875	-2.22048237772169e-10\\
51.690462890625	-1.67584521627358e-10\\
51.71045703125	-1.98873460626086e-10\\
51.730451171875	-2.13016868638834e-10\\
51.7504453125	-2.39398078224775e-10\\
51.770439453125	-2.20615699607075e-10\\
51.79043359375	-2.84230892833876e-10\\
51.810427734375	-2.6949681087925e-10\\
51.830421875	-3.06462452669624e-10\\
51.850416015625	-3.52584454957873e-10\\
51.87041015625	-3.04425931482167e-10\\
51.890404296875	-2.72870345165396e-10\\
51.9103984375	-2.92674842792614e-10\\
51.930392578125	-2.82093788371439e-10\\
51.95038671875	-2.68541380954925e-10\\
51.970380859375	-2.800460036926e-10\\
51.990375	-3.15898357629108e-10\\
52.010369140625	-3.48294102052811e-10\\
52.03036328125	-3.21979413302034e-10\\
52.050357421875	-4.27761495662838e-10\\
52.0703515625	-4.54076104657318e-10\\
52.090345703125	-3.51432049662886e-10\\
52.11033984375	-4.4597530205213e-10\\
52.130333984375	-4.29143611368915e-10\\
52.150328125	-3.50807910036608e-10\\
52.170322265625	-3.96580781205074e-10\\
52.19031640625	-4.47552604112901e-10\\
52.210310546875	-3.86339444567134e-10\\
52.2303046875	-4.30331586243332e-10\\
52.250298828125	-4.01940959631437e-10\\
52.27029296875	-3.98527562013302e-10\\
52.290287109375	-4.96640211203404e-10\\
52.31028125	-4.31944803008063e-10\\
52.330275390625	-4.74827872385497e-10\\
52.35026953125	-4.38449510722294e-10\\
52.370263671875	-5.12994737075112e-10\\
52.3902578125	-4.87194430359879e-10\\
52.410251953125	-4.99178134773111e-10\\
52.43024609375	-5.30618589106456e-10\\
52.450240234375	-4.64773924686653e-10\\
52.470234375	-5.21992605825855e-10\\
52.490228515625	-4.24388949317174e-10\\
52.51022265625	-4.65465446350298e-10\\
52.530216796875	-4.24979253160958e-10\\
52.5502109375	-4.17977771115464e-10\\
52.570205078125	-4.33468726283921e-10\\
52.59019921875	-4.56315032042127e-10\\
52.610193359375	-4.77730077383514e-10\\
52.6301875	-4.98781671023613e-10\\
52.650181640625	-4.66532256224649e-10\\
52.67017578125	-5.24901417921296e-10\\
52.690169921875	-4.69227097459303e-10\\
52.7101640625	-5.30427491748174e-10\\
52.730158203125	-4.81276384040058e-10\\
52.75015234375	-4.96888258740086e-10\\
52.770146484375	-4.80552274732601e-10\\
52.790140625	-5.11524478671445e-10\\
52.810134765625	-5.05920603911818e-10\\
52.83012890625	-5.07143719314894e-10\\
52.850123046875	-5.24469191347812e-10\\
52.8701171875	-4.77395529700253e-10\\
52.890111328125	-5.84297560439913e-10\\
52.91010546875	-5.08925429520052e-10\\
52.930099609375	-5.65033035217148e-10\\
52.95009375	-4.61476300782432e-10\\
52.970087890625	-5.20616753864027e-10\\
52.99008203125	-4.76381445259141e-10\\
53.010076171875	-4.62929481930626e-10\\
53.0300703125	-3.84516369823874e-10\\
53.050064453125	-4.17694355064877e-10\\
53.07005859375	-3.57014626161362e-10\\
53.090052734375	-3.66031932903765e-10\\
53.110046875	-3.64911984930008e-10\\
53.130041015625	-3.83162081534514e-10\\
53.15003515625	-3.70486066402882e-10\\
53.170029296875	-3.76195592168523e-10\\
53.1900234375	-3.71611104258618e-10\\
53.210017578125	-3.32760605419202e-10\\
53.23001171875	-3.41289329343084e-10\\
53.250005859375	-3.15358251389318e-10\\
53.27	-2.51688124181781e-10\\
53.289994140625	-3.06003472878986e-10\\
53.30998828125	-3.2908808883905e-10\\
53.329982421875	-2.24098680406805e-10\\
53.3499765625	-3.12348357667503e-10\\
53.369970703125	-2.50533964512557e-10\\
53.38996484375	-2.52778670467824e-10\\
53.409958984375	-2.45596418245063e-10\\
53.429953125	-2.19203793268919e-10\\
53.449947265625	-2.67679130654856e-10\\
53.46994140625	-2.53729820348221e-10\\
53.489935546875	-2.34706552019738e-10\\
53.5099296875	-2.1121122102702e-10\\
53.529923828125	-1.01369639321321e-10\\
53.54991796875	-1.15192392306449e-10\\
53.569912109375	-1.38549511815835e-10\\
53.58990625	-1.45472515708392e-10\\
53.609900390625	-1.15105021003534e-10\\
53.62989453125	-7.7498801390491e-11\\
};
\addlegendentry{$\text{train 5 -\textgreater{} Heimdal}$};

\addplot [color=mycolor4,solid,forget plot]
  table[row sep=crcr]{%
-45.1653759765625	8.45071704492093e-11\\
-45.1452265625	1.47186402046743e-10\\
-45.1250771484375	1.66537649863325e-10\\
-45.104927734375	1.89168744488152e-10\\
-45.0847783203125	2.51825909432672e-10\\
-45.06462890625	1.85140700118071e-10\\
-45.0444794921875	1.60974487947595e-10\\
-45.024330078125	1.96776212132971e-10\\
-45.0041806640625	1.07660351893289e-10\\
-44.98403125	1.04467819996769e-10\\
-44.9638818359375	6.45725450081922e-11\\
-44.943732421875	4.05755900340877e-11\\
-44.9235830078125	-4.72656635694992e-12\\
-44.90343359375	6.25518182974636e-11\\
-44.8832841796875	1.26766336467024e-10\\
-44.863134765625	2.34843363093219e-10\\
-44.8429853515625	2.26711671938427e-10\\
-44.8228359375	3.53291595053458e-10\\
-44.8026865234375	3.20931226924431e-10\\
-44.782537109375	3.51496266578038e-10\\
-44.7623876953125	3.34590382489976e-10\\
-44.74223828125	1.10858527702809e-10\\
-44.7220888671875	1.88966267643341e-10\\
-44.701939453125	1.17333568570331e-10\\
-44.6817900390625	1.24254817895901e-10\\
-44.661640625	1.120500433092e-10\\
-44.6414912109375	1.33403976391458e-10\\
-44.621341796875	1.75097188285274e-10\\
-44.6011923828125	2.25601881605018e-10\\
-44.58104296875	2.66689427133213e-10\\
-44.5608935546875	3.09691606054273e-10\\
-44.540744140625	2.61335101040893e-10\\
-44.5205947265625	2.64434175091679e-10\\
-44.5004453125	1.99922961692525e-10\\
-44.4802958984375	1.84576385776907e-10\\
-44.460146484375	1.89862142437024e-10\\
-44.4399970703125	1.49824669366176e-10\\
-44.41984765625	2.16388628248071e-10\\
-44.3996982421875	1.92810996885641e-10\\
-44.379548828125	3.01344889499769e-10\\
-44.3593994140625	2.79694625761555e-10\\
-44.33925	2.56869307121692e-10\\
-44.3191005859375	3.3681499559406e-10\\
-44.298951171875	3.33285724890725e-10\\
-44.2788017578125	3.73417248672543e-10\\
-44.25865234375	3.98023612073178e-10\\
-44.2385029296875	3.39531182462425e-10\\
-44.218353515625	4.14087514950173e-10\\
-44.1982041015625	4.10562980772119e-10\\
-44.1780546875	3.92123265837225e-10\\
-44.1579052734375	3.84016754961744e-10\\
-44.137755859375	4.05451430056021e-10\\
-44.1176064453125	3.74549302974662e-10\\
-44.09745703125	3.28666754250349e-10\\
-44.0773076171875	3.24667805742834e-10\\
-44.057158203125	2.96102964526397e-10\\
-44.0370087890625	3.06166926992747e-10\\
-44.016859375	2.37722427008296e-10\\
-43.9967099609375	2.47227237896423e-10\\
-43.976560546875	2.87873189527022e-10\\
-43.9564111328125	3.63731566510949e-10\\
-43.93626171875	3.70302606907852e-10\\
-43.9161123046875	3.07924070925648e-10\\
-43.895962890625	2.70713732140356e-10\\
-43.8758134765625	2.4564644160087e-10\\
-43.8556640625	1.78124907526322e-10\\
-43.8355146484375	1.49452014138433e-10\\
-43.815365234375	1.12530586963397e-10\\
-43.7952158203125	1.11295888407545e-10\\
-43.77506640625	9.86461774800864e-11\\
-43.7549169921875	1.68386464469345e-10\\
-43.734767578125	2.4942932926659e-10\\
-43.7146181640625	3.70869528264839e-10\\
-43.69446875	3.85159281944497e-10\\
-43.6743193359375	2.57886080464655e-10\\
-43.654169921875	2.9191411598409e-10\\
-43.6340205078125	2.54179869151856e-10\\
-43.61387109375	1.80977788398939e-10\\
-43.5937216796875	1.13516269106395e-10\\
-43.573572265625	1.26467709892426e-10\\
-43.5534228515625	1.82686228943838e-10\\
-43.5332734375	2.22860394517438e-10\\
-43.5131240234375	3.66385417422612e-10\\
-43.492974609375	4.34295626885672e-10\\
-43.4728251953125	5.15558954869194e-10\\
-43.45267578125	5.52055707911911e-10\\
-43.4325263671875	5.03581272386895e-10\\
-43.412376953125	5.16541230590738e-10\\
-43.3922275390625	4.53525220440843e-10\\
-43.372078125	3.72409591934362e-10\\
-43.3519287109375	3.5689514469571e-10\\
-43.331779296875	3.05825497601182e-10\\
-43.3116298828125	3.03244983197268e-10\\
-43.29148046875	3.75300675076262e-10\\
-43.2713310546875	4.96129589596963e-10\\
-43.251181640625	4.77648559795403e-10\\
-43.2310322265625	5.47211486284802e-10\\
-43.2108828125	5.329399761301e-10\\
-43.1907333984375	4.99307190755159e-10\\
-43.170583984375	5.01740048970684e-10\\
-43.1504345703125	4.00304593491811e-10\\
-43.13028515625	3.31766472313022e-10\\
-43.1101357421875	3.028239910383e-10\\
-43.089986328125	2.45927514525047e-10\\
-43.0698369140625	3.17538944272444e-10\\
-43.0496875	3.00506931834751e-10\\
-43.0295380859375	2.18062294376546e-10\\
-43.009388671875	3.27746233289089e-10\\
-42.9892392578125	3.58008669030622e-10\\
-42.96908984375	3.56320374151984e-10\\
-42.9489404296875	3.49896839028505e-10\\
-42.928791015625	3.19947755930646e-10\\
-42.9086416015625	3.30748031748884e-10\\
-42.8884921875	2.48746029881314e-10\\
-42.8683427734375	2.84291593741262e-10\\
-42.848193359375	2.74472334507741e-10\\
-42.8280439453125	3.28615614044435e-10\\
-42.80789453125	3.14554926487969e-10\\
-42.7877451171875	3.52706575851935e-10\\
-42.767595703125	2.80215753022282e-10\\
-42.7474462890625	2.42832960419557e-10\\
-42.727296875	1.82939478855698e-10\\
-42.7071474609375	1.29810047902454e-10\\
-42.686998046875	7.35418427088892e-11\\
-42.6668486328125	1.37000639274501e-11\\
-42.64669921875	4.74950083467495e-11\\
-42.6265498046875	1.04061557464502e-10\\
-42.606400390625	1.39615702503384e-10\\
-42.5862509765625	8.72575979367904e-11\\
-42.5661015625	1.13986901682991e-10\\
-42.5459521484375	1.66912105717008e-10\\
-42.525802734375	1.27405316275178e-10\\
-42.5056533203125	5.11708308894323e-11\\
-42.48550390625	5.18762546547704e-11\\
-42.4653544921875	-3.08236903313104e-11\\
-42.445205078125	-3.25988479936393e-11\\
-42.4250556640625	-7.91700641654964e-11\\
-42.40490625	-4.40813165819488e-11\\
-42.3847568359375	-1.77853792709322e-11\\
-42.364607421875	-3.15291105511583e-11\\
-42.3444580078125	1.27473021494965e-12\\
-42.32430859375	1.24430585703477e-11\\
-42.3041591796875	6.23388089846842e-12\\
-42.284009765625	3.83086455196708e-11\\
-42.2638603515625	6.23732620629207e-12\\
-42.2437109375	-4.1733587097987e-11\\
-42.2235615234375	-6.82278278663436e-11\\
-42.203412109375	-1.43940528166327e-10\\
-42.1832626953125	-1.11079541582209e-10\\
-42.16311328125	-8.58980878129485e-11\\
-42.1429638671875	-1.06199496531501e-10\\
-42.122814453125	-5.12576502868089e-11\\
-42.1026650390625	-5.9035404435504e-11\\
-42.082515625	-8.41759418032227e-11\\
-42.0623662109375	-7.15507713948961e-11\\
-42.042216796875	1.6963411021588e-11\\
-42.0220673828125	-3.50171813998166e-11\\
-42.00191796875	-1.20097564510386e-10\\
-41.9817685546875	-1.00369633711301e-10\\
-41.961619140625	-1.00672467013208e-10\\
-41.9414697265625	-1.30306351173618e-10\\
-41.9213203125	-1.1149445259239e-10\\
-41.9011708984375	-9.08683711606058e-11\\
-41.881021484375	-8.22787639769717e-11\\
-41.8608720703125	-1.2021823063252e-10\\
-41.84072265625	-6.84178223979833e-11\\
-41.8205732421875	-1.46145052843903e-10\\
-41.800423828125	-1.25108789909495e-10\\
-41.7802744140625	-5.4521572528833e-11\\
-41.760125	-9.79746596629107e-11\\
-41.7399755859375	-5.71895223899277e-11\\
-41.719826171875	-1.60689919819299e-10\\
-41.6996767578125	-2.11583337962421e-10\\
-41.67952734375	-1.45118446444819e-10\\
-41.6593779296875	-2.60817455742576e-10\\
-41.639228515625	-2.39987604030519e-10\\
-41.6190791015625	-3.31621680366851e-10\\
-41.5989296875	-2.8809082227865e-10\\
-41.5787802734375	-3.03962329154317e-10\\
-41.558630859375	-2.8960501712943e-10\\
-41.5384814453125	-2.53678137439546e-10\\
-41.51833203125	-2.72755629400946e-10\\
-41.4981826171875	-1.99071776675999e-10\\
-41.478033203125	-2.66077858505019e-10\\
-41.4578837890625	-2.92454135728307e-10\\
-41.437734375	-2.26776445485712e-10\\
-41.4175849609375	-1.71996510456505e-10\\
-41.397435546875	-2.38883457098826e-10\\
-41.3772861328125	-3.15872412618553e-10\\
-41.35713671875	-2.78732158229254e-10\\
-41.3369873046875	-3.46880198498753e-10\\
-41.316837890625	-3.09969360265575e-10\\
-41.2966884765625	-2.8241588208718e-10\\
-41.2765390625	-2.83783194261095e-10\\
-41.2563896484375	-2.27813248084786e-10\\
-41.236240234375	-2.43028386979129e-10\\
-41.2160908203125	-3.0645204552716e-10\\
-41.19594140625	-2.27341515020874e-10\\
-41.1757919921875	-3.5687281657363e-10\\
-41.155642578125	-2.7738280847184e-10\\
-41.1354931640625	-3.10382772352612e-10\\
-41.11534375	-3.42105989934424e-10\\
-41.0951943359375	-2.93391243140955e-10\\
-41.075044921875	-3.05574798786699e-10\\
-41.0548955078125	-3.75981931614062e-10\\
-41.03474609375	-2.53005262219133e-10\\
-41.0145966796875	-2.64289205022848e-10\\
-40.994447265625	-3.12822052973208e-10\\
-40.9742978515625	-2.49673425365842e-10\\
-40.9541484375	-2.91927478875861e-10\\
-40.9339990234375	-3.20971717867444e-10\\
-40.913849609375	-3.61159014213956e-10\\
-40.8937001953125	-4.24865327049043e-10\\
-40.87355078125	-4.40893371399578e-10\\
-40.8534013671875	-4.25137553904161e-10\\
-40.833251953125	-4.39388776476773e-10\\
-40.8131025390625	-3.54846751266613e-10\\
-40.792953125	-3.57052456077831e-10\\
-40.7728037109375	-3.11749598989538e-10\\
-40.752654296875	-2.96019323645466e-10\\
-40.7325048828125	-2.8295560459537e-10\\
-40.71235546875	-3.24462992123191e-10\\
-40.6922060546875	-3.539137646497e-10\\
-40.672056640625	-3.71895473644621e-10\\
-40.6519072265625	-3.66517611827136e-10\\
-40.6317578125	-3.77807248889677e-10\\
-40.6116083984375	-4.23156319336844e-10\\
-40.591458984375	-3.33200007256496e-10\\
-40.5713095703125	-3.04201765253259e-10\\
-40.55116015625	-3.47362521292898e-10\\
-40.5310107421875	-2.13218462555475e-10\\
-40.510861328125	-2.47799265853614e-10\\
-40.4907119140625	-3.49267300928087e-10\\
-40.4705625	-2.37048040500289e-10\\
-40.4504130859375	-2.84160495359225e-10\\
-40.430263671875	-3.54668890547381e-10\\
-40.4101142578125	-3.86054454104878e-10\\
-40.38996484375	-3.35518865728542e-10\\
-40.3698154296875	-4.30467528545092e-10\\
-40.349666015625	-3.53348145239417e-10\\
-40.3295166015625	-3.20049591911552e-10\\
-40.3093671875	-1.91799451487675e-10\\
-40.2892177734375	-2.07584312434791e-10\\
-40.269068359375	-2.96278619644099e-10\\
-40.2489189453125	-1.95990157765772e-10\\
-40.22876953125	-2.75436147898873e-10\\
-40.2086201171875	-2.61643387373477e-10\\
-40.188470703125	-3.44389786016329e-10\\
-40.1683212890625	-2.05916983114685e-10\\
-40.148171875	-2.4507132318369e-10\\
-40.1280224609375	-2.38216080192966e-10\\
-40.107873046875	-1.27938418580172e-10\\
-40.0877236328125	-7.43888055562643e-11\\
-40.06757421875	-1.09776901360383e-10\\
-40.0474248046875	-9.13680514744758e-11\\
-40.027275390625	-2.06745345958897e-11\\
-40.0071259765625	2.35290414302332e-12\\
-39.9869765625	-3.12855655079708e-11\\
-39.9668271484375	-1.88580387741714e-11\\
-39.946677734375	3.34897320973204e-11\\
-39.9265283203125	3.10874046942056e-11\\
-39.90637890625	-2.19263597150404e-11\\
-39.8862294921875	-2.37118492952963e-11\\
-39.866080078125	-1.21817514968979e-11\\
-39.8459306640625	1.69536808572227e-11\\
-39.82578125	6.96510362456965e-11\\
-39.8056318359375	1.48120042071378e-10\\
-39.785482421875	1.70267124431307e-10\\
-39.7653330078125	1.86836717439551e-10\\
-39.74518359375	1.03000912194164e-10\\
-39.7250341796875	7.8934776499808e-11\\
-39.704884765625	8.71912686437777e-11\\
-39.6847353515625	5.09577686275623e-11\\
-39.6645859375	1.19449809734938e-10\\
-39.6444365234375	1.0282845168955e-10\\
-39.624287109375	1.13165843420284e-10\\
-39.6041376953125	8.40113239150128e-11\\
-39.58398828125	1.46310466113093e-10\\
-39.5638388671875	1.06518400206331e-10\\
-39.543689453125	1.2417566415594e-10\\
-39.5235400390625	1.23680633907791e-10\\
-39.503390625	1.89193830311941e-10\\
-39.4832412109375	1.20356027565979e-10\\
-39.463091796875	1.52200237114705e-10\\
-39.4429423828125	1.4462676252214e-10\\
-39.42279296875	1.20310247510465e-10\\
-39.4026435546875	1.16553357448759e-10\\
-39.382494140625	1.25724803102885e-10\\
-39.3623447265625	2.01570414786787e-10\\
-39.3421953125	1.49614288495762e-10\\
-39.3220458984375	1.31509806671003e-10\\
-39.301896484375	1.51001465918215e-10\\
-39.2817470703125	2.10224130652202e-10\\
-39.26159765625	2.13231774474774e-10\\
-39.2414482421875	2.22718430658176e-10\\
-39.221298828125	2.12938319404898e-10\\
-39.2011494140625	2.55204968997117e-10\\
-39.181	1.42981114229063e-10\\
-39.1608505859375	2.66119596656742e-10\\
-39.140701171875	2.84158237367409e-10\\
-39.1205517578125	3.23177659545226e-10\\
-39.10040234375	2.96981716957021e-10\\
-39.0802529296875	3.54347392768925e-10\\
-39.060103515625	3.3788638034364e-10\\
-39.0399541015625	3.14397449940228e-10\\
-39.0198046875	4.08166664164557e-10\\
-38.9996552734375	3.68323815716061e-10\\
-38.979505859375	3.62233890395841e-10\\
-38.9593564453125	3.43234128746621e-10\\
-38.93920703125	4.12778452286625e-10\\
-38.9190576171875	4.10434836427291e-10\\
-38.898908203125	5.06268110291126e-10\\
-38.8787587890625	4.56361904611914e-10\\
-38.858609375	5.28217105965641e-10\\
-38.8384599609375	4.07202602704635e-10\\
-38.818310546875	4.33448970801117e-10\\
-38.7981611328125	3.33579017652303e-10\\
-38.77801171875	3.99654831267416e-10\\
-38.7578623046875	3.37970955942681e-10\\
-38.737712890625	3.8692720010138e-10\\
-38.7175634765625	3.69910847491894e-10\\
-38.6974140625	4.42622275628115e-10\\
-38.6772646484375	4.97497891986403e-10\\
-38.657115234375	4.61917714023834e-10\\
-38.6369658203125	4.07542865770421e-10\\
-38.61681640625	4.17347666726089e-10\\
-38.5966669921875	3.34251805250722e-10\\
-38.576517578125	3.33267746557604e-10\\
-38.5563681640625	3.64492810361604e-10\\
-38.53621875	3.72141328209962e-10\\
-38.5160693359375	3.43147685784677e-10\\
-38.495919921875	4.86866479701063e-10\\
-38.4757705078125	4.49107207575412e-10\\
-38.45562109375	5.34896538643991e-10\\
-38.4354716796875	6.0664056406698e-10\\
-38.415322265625	4.56788141102212e-10\\
-38.3951728515625	3.99084460035095e-10\\
-38.3750234375	3.88966835749001e-10\\
-38.3548740234375	3.58276647808498e-10\\
-38.334724609375	3.84498836419529e-10\\
-38.3145751953125	3.40023037498424e-10\\
-38.29442578125	3.69357549490894e-10\\
-38.2742763671875	4.49563067558909e-10\\
-38.254126953125	4.44822235097409e-10\\
-38.2339775390625	4.92722433622869e-10\\
-38.213828125	5.20305942602462e-10\\
-38.1936787109375	4.6120243167384e-10\\
-38.173529296875	3.6915425160187e-10\\
-38.1533798828125	3.3906328858504e-10\\
-38.13323046875	2.8831947308937e-10\\
-38.1130810546875	3.53453467501004e-10\\
-38.092931640625	2.38147758826311e-10\\
-38.0727822265625	3.13455500848482e-10\\
-38.0526328125	3.57709514551976e-10\\
-38.0324833984375	3.92986389181146e-10\\
-38.012333984375	4.16832732961038e-10\\
-37.9921845703125	4.18035947083047e-10\\
-37.97203515625	1.85513628946487e-10\\
-37.9518857421875	2.75668175296938e-10\\
-37.931736328125	2.50368147259035e-10\\
-37.9115869140625	2.65019497849782e-10\\
-37.8914375	2.44337320972701e-10\\
-37.8712880859375	2.5611669803033e-10\\
-37.851138671875	2.50143330261398e-10\\
-37.8309892578125	3.7020112330243e-10\\
-37.81083984375	3.85744268227185e-10\\
-37.7906904296875	3.41554125329755e-10\\
-37.770541015625	3.53974457624128e-10\\
-37.7503916015625	2.12242065265684e-10\\
-37.7302421875	1.74221489950698e-10\\
-37.7100927734375	1.62173633189577e-10\\
-37.689943359375	1.21238374767504e-10\\
-37.6697939453125	6.17459880865736e-11\\
-37.64964453125	1.24845371399906e-10\\
-37.6294951171875	1.60753513382174e-10\\
-37.609345703125	2.11312769212862e-10\\
-37.5891962890625	2.53760163524158e-10\\
-37.569046875	2.31869307056868e-10\\
-37.5488974609375	2.95066034982466e-10\\
-37.528748046875	1.52143697751197e-10\\
-37.5085986328125	9.60445523698819e-11\\
-37.48844921875	7.95323031954592e-11\\
-37.4682998046875	2.58955794179104e-11\\
-37.448150390625	-9.57615507741424e-11\\
-37.4280009765625	-1.23556017154178e-10\\
-37.4078515625	-1.16989445950445e-10\\
-37.3877021484375	-7.74225304991874e-11\\
-37.367552734375	-2.52840660436308e-11\\
-37.3474033203125	-7.36195680458553e-11\\
-37.32725390625	-4.58634145224955e-12\\
-37.3071044921875	-7.0287300264927e-11\\
-37.286955078125	-5.83922836708437e-11\\
-37.2668056640625	-1.16762947403622e-10\\
-37.24665625	-1.76390570979185e-10\\
-37.2265068359375	-1.91018634755896e-10\\
-37.206357421875	-1.47781230858945e-10\\
-37.1862080078125	-2.05566119956639e-10\\
-37.16605859375	-1.3813970970571e-10\\
-37.1459091796875	-2.05887204738469e-10\\
-37.125759765625	-1.19910722113508e-10\\
-37.1056103515625	-9.09357280293361e-11\\
-37.0854609375	-1.26910823278236e-10\\
-37.0653115234375	-1.23930178708643e-10\\
-37.045162109375	-1.60610433652702e-10\\
-37.0250126953125	-1.2539120499826e-10\\
-37.00486328125	-2.0953696091343e-10\\
-36.9847138671875	-2.14267967618891e-10\\
-36.964564453125	-2.29890634847864e-10\\
-36.9444150390625	-2.83173482565027e-10\\
-36.924265625	-3.73174710652255e-10\\
-36.9041162109375	-2.45257518116384e-10\\
-36.883966796875	-3.1844001381099e-10\\
-36.8638173828125	-3.64823258097997e-10\\
-36.84366796875	-2.31604326347603e-10\\
-36.8235185546875	-2.63003019601731e-10\\
-36.803369140625	-3.32766459857455e-10\\
-36.7832197265625	-3.05557924680684e-10\\
-36.7630703125	-2.72059567778909e-10\\
-36.7429208984375	-3.60825193938205e-10\\
-36.722771484375	-3.4507892728853e-10\\
-36.7026220703125	-3.68670784423006e-10\\
-36.68247265625	-3.85179294659965e-10\\
-36.6623232421875	-4.2075992612794e-10\\
-36.642173828125	-3.91763593034071e-10\\
-36.6220244140625	-3.11190635898247e-10\\
-36.601875	-4.14352918101326e-10\\
-36.5817255859375	-4.27310086915522e-10\\
-36.561576171875	-4.23359039590705e-10\\
-36.5414267578125	-4.52641539883995e-10\\
-36.52127734375	-4.88658011317105e-10\\
-36.5011279296875	-4.72584222433935e-10\\
-36.480978515625	-4.64123270973071e-10\\
-36.4608291015625	-4.66009776674485e-10\\
-36.4406796875	-5.0248351515004e-10\\
-36.4205302734375	-4.21168062211283e-10\\
-36.400380859375	-4.54112243074265e-10\\
-36.3802314453125	-5.35822288125432e-10\\
-36.36008203125	-6.02813410721471e-10\\
-36.3399326171875	-5.71459548362459e-10\\
-36.319783203125	-6.27297988411162e-10\\
-36.2996337890625	-6.09578382800366e-10\\
-36.279484375	-5.32802037592354e-10\\
-36.2593349609375	-4.95261050675149e-10\\
-36.239185546875	-4.95785482414829e-10\\
-36.2190361328125	-3.87865799660882e-10\\
-36.19888671875	-4.4828398223117e-10\\
-36.1787373046875	-4.2784674273163e-10\\
-36.158587890625	-5.03160294648753e-10\\
-36.1384384765625	-6.25433856785078e-10\\
-36.1182890625	-6.62818393116765e-10\\
-36.0981396484375	-6.07425021049149e-10\\
-36.077990234375	-6.46503112725529e-10\\
-36.0578408203125	-5.48106651359939e-10\\
-36.03769140625	-5.61172235509216e-10\\
-36.0175419921875	-5.22931174006698e-10\\
-35.997392578125	-4.8791646794321e-10\\
-35.9772431640625	-4.03045956238178e-10\\
-35.95709375	-4.91157891086813e-10\\
-35.9369443359375	-5.11570567235526e-10\\
-35.916794921875	-5.58646628240315e-10\\
-35.8966455078125	-6.56373555161862e-10\\
-35.87649609375	-7.11415154699367e-10\\
-35.8563466796875	-8.01348306718201e-10\\
-35.836197265625	-7.17125879029735e-10\\
-35.8160478515625	-6.17283935775429e-10\\
-35.7958984375	-6.20002642822935e-10\\
-35.7757490234375	-5.3891783012323e-10\\
-35.755599609375	-4.39400502813209e-10\\
-35.7354501953125	-4.48115698722041e-10\\
-35.71530078125	-4.92512359914353e-10\\
-35.6951513671875	-5.26624945810048e-10\\
-35.675001953125	-5.50965984048777e-10\\
-35.6548525390625	-5.46083368666888e-10\\
-35.634703125	-6.72064782084104e-10\\
-35.6145537109375	-4.81007823875804e-10\\
-35.594404296875	-4.97228714143445e-10\\
-35.5742548828125	-5.15862335898206e-10\\
-35.55410546875	-4.51056108573085e-10\\
-35.5339560546875	-4.01687977352936e-10\\
-35.513806640625	-4.22226474889249e-10\\
-35.4936572265625	-4.92749813678858e-10\\
-35.4735078125	-4.2550810337838e-10\\
-35.4533583984375	-4.95546308161018e-10\\
-35.433208984375	-4.57595907787351e-10\\
-35.4130595703125	-3.55117793760708e-10\\
-35.39291015625	-3.74874306248711e-10\\
-35.3727607421875	-3.32240723228091e-10\\
-35.352611328125	-2.17159744881258e-10\\
-35.3324619140625	-2.79822165273226e-10\\
-35.3123125	-2.47268480726679e-10\\
-35.2921630859375	-1.47404604589299e-10\\
-35.272013671875	-3.09945571603633e-10\\
-35.2518642578125	-3.26045390474208e-10\\
-35.23171484375	-3.24762342039988e-10\\
-35.2115654296875	-2.66172919086767e-10\\
-35.191416015625	-1.65368580025121e-10\\
-35.1712666015625	-1.49424221633605e-10\\
-35.1511171875	-1.16524654548054e-10\\
-35.1309677734375	-2.31527348404749e-11\\
-35.110818359375	4.4928335039214e-11\\
-35.0906689453125	-4.33615421569865e-11\\
-35.07051953125	3.32818812699984e-11\\
-35.0503701171875	-1.25341381113758e-10\\
-35.030220703125	-9.94789969890685e-11\\
-35.0100712890625	-5.34831212398412e-11\\
-34.989921875	-1.3978166246004e-10\\
-34.9697724609375	-6.76122048146978e-11\\
-34.949623046875	-8.62889017343632e-11\\
-34.9294736328125	-2.41806200069764e-11\\
-34.90932421875	4.9687113686835e-11\\
-34.8891748046875	1.51144167069511e-10\\
-34.869025390625	1.22017532796676e-10\\
-34.8488759765625	1.65078240257204e-10\\
-34.8287265625	1.84036269113898e-10\\
-34.8085771484375	7.46715989681544e-11\\
-34.788427734375	1.57138744709713e-10\\
-34.7682783203125	1.21515038087188e-10\\
-34.74812890625	2.4104791016929e-10\\
-34.7279794921875	1.4414166967577e-10\\
-34.707830078125	1.3814143272814e-10\\
-34.6876806640625	1.83858874280822e-10\\
-34.66753125	2.41779437314929e-10\\
-34.6473818359375	2.51259814708783e-10\\
-34.627232421875	3.40270163799107e-10\\
-34.6070830078125	2.90069105405359e-10\\
-34.58693359375	2.78498854600137e-10\\
-34.5667841796875	2.66918527875214e-10\\
-34.546634765625	2.64263316081027e-10\\
-34.5264853515625	2.9080033080136e-10\\
-34.5063359375	1.92037190157532e-10\\
-34.4861865234375	2.76028787467887e-10\\
-34.466037109375	2.03082848705794e-10\\
-34.4458876953125	2.49780827828785e-10\\
-34.42573828125	3.57619792564682e-10\\
-34.4055888671875	3.36853438523902e-10\\
-34.385439453125	3.23239681962815e-10\\
-34.3652900390625	4.65036586860955e-10\\
-34.345140625	3.94644238440784e-10\\
-34.3249912109375	4.55918345223526e-10\\
-34.304841796875	4.69205403822081e-10\\
-34.2846923828125	4.3445904684478e-10\\
-34.26454296875	4.27548944578574e-10\\
-34.2443935546875	4.15112335578268e-10\\
-34.224244140625	4.95958363313845e-10\\
-34.2040947265625	4.73646817328286e-10\\
-34.1839453125	4.43388550429409e-10\\
-34.1637958984375	5.1855678897591e-10\\
-34.143646484375	6.41990657585918e-10\\
-34.1234970703125	5.06732015563703e-10\\
-34.10334765625	6.01238914930195e-10\\
-34.0831982421875	5.60760844420515e-10\\
-34.063048828125	5.57198130951925e-10\\
-34.0428994140625	5.87878849998064e-10\\
-34.02275	5.5129321256992e-10\\
-34.0026005859375	5.30525333322101e-10\\
-33.982451171875	4.99915453414278e-10\\
-33.9623017578125	5.48444222467402e-10\\
-33.94215234375	5.69483897346804e-10\\
-33.9220029296875	5.65429009031269e-10\\
-33.901853515625	5.74661892125626e-10\\
-33.8817041015625	6.52052860891272e-10\\
-33.8615546875	6.92162788503604e-10\\
-33.8414052734375	7.5447147542435e-10\\
-33.821255859375	7.51206358468145e-10\\
-33.8011064453125	7.49774776035306e-10\\
-33.78095703125	7.29662475705791e-10\\
-33.7608076171875	6.97519276974136e-10\\
-33.740658203125	6.67863593966969e-10\\
-33.7205087890625	6.17228800600587e-10\\
-33.700359375	5.88621602620691e-10\\
-33.6802099609375	6.06984866406849e-10\\
-33.660060546875	6.46101696851198e-10\\
-33.6399111328125	6.95373958173113e-10\\
-33.61976171875	7.12409231836329e-10\\
-33.5996123046875	7.77015164368383e-10\\
-33.579462890625	7.78760938959249e-10\\
-33.5593134765625	7.41210167769652e-10\\
-33.5391640625	7.05829499189214e-10\\
-33.5190146484375	6.87066076207728e-10\\
-33.498865234375	6.45228139644093e-10\\
-33.4787158203125	5.56083134104143e-10\\
-33.45856640625	6.21058312736806e-10\\
-33.4384169921875	6.10593640858765e-10\\
-33.418267578125	6.09938449119371e-10\\
-33.3981181640625	7.27255652957677e-10\\
-33.37796875	6.90127291604422e-10\\
-33.3578193359375	7.37076954226847e-10\\
-33.337669921875	7.76918087195517e-10\\
-33.3175205078125	7.09967531483223e-10\\
-33.29737109375	6.48321340589756e-10\\
-33.2772216796875	6.73765801499977e-10\\
-33.257072265625	5.68927381424858e-10\\
-33.2369228515625	5.83694507224617e-10\\
-33.2167734375	5.16168898181936e-10\\
-33.1966240234375	5.64959534415381e-10\\
-33.176474609375	5.28911430115763e-10\\
-33.1563251953125	5.55898140595314e-10\\
-33.13617578125	6.43039741949583e-10\\
-33.1160263671875	5.62203214234417e-10\\
-33.095876953125	5.2497744925814e-10\\
-33.0757275390625	6.18885049901935e-10\\
-33.055578125	4.96768904565229e-10\\
-33.0354287109375	4.23411086168401e-10\\
-33.015279296875	4.33943086472785e-10\\
-32.9951298828125	4.11764897710419e-10\\
-32.97498046875	4.51929151448416e-10\\
-32.9548310546875	4.51639294962729e-10\\
-32.934681640625	5.57504586309584e-10\\
-32.9145322265625	5.06240936278716e-10\\
-32.8943828125	5.14006737855009e-10\\
-32.8742333984375	4.96651355573715e-10\\
-32.854083984375	3.9696616903296e-10\\
-32.8339345703125	3.45709537090029e-10\\
-32.81378515625	3.53805224981308e-10\\
-32.7936357421875	1.89530032711433e-10\\
-32.773486328125	2.18637680400501e-10\\
-32.7533369140625	1.61007048595768e-10\\
-32.7331875	1.44799916231359e-10\\
-32.7130380859375	2.23946432351925e-10\\
-32.692888671875	2.07866092700248e-10\\
-32.6727392578125	2.75616278443632e-10\\
-32.65258984375	3.09399439558013e-10\\
-32.6324404296875	2.14702194638836e-10\\
-32.612291015625	1.86322132825259e-10\\
-32.5921416015625	1.36059164938141e-10\\
-32.5719921875	2.76779539846367e-11\\
-32.5518427734375	-3.88838259319611e-11\\
-32.531693359375	-3.08840170791109e-11\\
-32.5115439453125	-1.60711650538188e-10\\
-32.49139453125	-1.13105560063989e-10\\
-32.4712451171875	-4.86857355056706e-11\\
-32.451095703125	-7.84671164119504e-11\\
-32.4309462890625	-5.71100084929907e-11\\
-32.410796875	-9.7931062413366e-11\\
-32.3906474609375	-1.18936484107165e-10\\
-32.370498046875	-1.46035981532495e-10\\
-32.3503486328125	-1.85327845716646e-10\\
-32.33019921875	-1.86433193022185e-10\\
-32.3100498046875	-2.23520686705789e-10\\
-32.289900390625	-2.54166332844111e-10\\
-32.2697509765625	-3.17660906129197e-10\\
-32.2496015625	-3.52652597466332e-10\\
-32.2294521484375	-3.56343051578905e-10\\
-32.209302734375	-3.30774979067609e-10\\
-32.1891533203125	-4.07380536801674e-10\\
-32.16900390625	-3.43443269402258e-10\\
-32.1488544921875	-4.29196574957825e-10\\
-32.128705078125	-4.29649993020825e-10\\
-32.1085556640625	-4.33430600171442e-10\\
-32.08840625	-5.38424318814769e-10\\
-32.0682568359375	-4.91637243334262e-10\\
-32.048107421875	-5.16098043127941e-10\\
-32.0279580078125	-5.08347264229301e-10\\
-32.00780859375	-4.40335734330213e-10\\
-31.9876591796875	-5.02114335900297e-10\\
-31.967509765625	-4.92794493257257e-10\\
-31.9473603515625	-4.75920739447978e-10\\
-31.9272109375	-5.20720950463764e-10\\
-31.9070615234375	-6.03969825988775e-10\\
-31.886912109375	-5.6559890662825e-10\\
-31.8667626953125	-6.30846534933967e-10\\
-31.84661328125	-6.41784763338837e-10\\
-31.8264638671875	-5.40209031404193e-10\\
-31.806314453125	-6.0765269968728e-10\\
-31.7861650390625	-6.1465350805007e-10\\
-31.766015625	-6.47231373328406e-10\\
-31.7458662109375	-6.40313227937552e-10\\
-31.725716796875	-6.37210822297159e-10\\
-31.7055673828125	-6.95424689872047e-10\\
-31.68541796875	-7.00759125680395e-10\\
-31.6652685546875	-7.96294372829391e-10\\
-31.645119140625	-7.4003546731401e-10\\
-31.6249697265625	-6.96666514260915e-10\\
-31.6048203125	-7.27579278982012e-10\\
-31.5846708984375	-7.25748065359744e-10\\
-31.564521484375	-7.16236741082908e-10\\
-31.5443720703125	-7.91328151797169e-10\\
-31.52422265625	-8.22522116199243e-10\\
-31.5040732421875	-7.82119564083096e-10\\
-31.483923828125	-7.94628011161043e-10\\
-31.4637744140625	-7.65111652093523e-10\\
-31.443625	-6.92139921841939e-10\\
-31.4234755859375	-7.24079706002772e-10\\
-31.403326171875	-6.21063752024602e-10\\
-31.3831767578125	-7.37838833704541e-10\\
-31.36302734375	-6.89765231414774e-10\\
-31.3428779296875	-7.19007330410991e-10\\
-31.322728515625	-7.78576396095887e-10\\
-31.3025791015625	-7.9513365303004e-10\\
-31.2824296875	-8.80686407238901e-10\\
-31.2622802734375	-8.75309869854885e-10\\
-31.242130859375	-8.73745022710186e-10\\
-31.2219814453125	-8.26135458718705e-10\\
-31.20183203125	-7.10629100858646e-10\\
-31.1816826171875	-7.69304313294411e-10\\
-31.161533203125	-7.37615860725114e-10\\
-31.1413837890625	-7.34440736195477e-10\\
-31.121234375	-8.26067061733188e-10\\
-31.1010849609375	-9.09181184211258e-10\\
-31.080935546875	-8.98914643076416e-10\\
-31.0607861328125	-9.79238514956434e-10\\
-31.04063671875	-9.47533854975833e-10\\
-31.0204873046875	-9.22307984189087e-10\\
-31.000337890625	-8.32929947472171e-10\\
-30.9801884765625	-8.52649349141438e-10\\
-30.9600390625	-8.29958538043629e-10\\
-30.9398896484375	-8.08869778188192e-10\\
-30.919740234375	-8.20715594438939e-10\\
-30.8995908203125	-8.94589035220822e-10\\
-30.87944140625	-9.39227705796861e-10\\
-30.8592919921875	-1.08258341399366e-09\\
-30.839142578125	-1.04845425675398e-09\\
-30.8189931640625	-1.10576651363653e-09\\
-30.79884375	-1.09719892670101e-09\\
-30.7786943359375	-1.04664994190342e-09\\
-30.758544921875	-1.01532591373769e-09\\
-30.7383955078125	-8.7559123113101e-10\\
-30.71824609375	-9.33502005267488e-10\\
-30.6980966796875	-8.53570288778153e-10\\
-30.677947265625	-8.71736836294679e-10\\
-30.6577978515625	-8.37954944125762e-10\\
-30.6376484375	-9.43481480018474e-10\\
-30.6174990234375	-9.29965367770359e-10\\
-30.597349609375	-9.1729409351016e-10\\
-30.5772001953125	-8.84223968943065e-10\\
-30.55705078125	-8.39946942679565e-10\\
-30.5369013671875	-7.05230461076858e-10\\
-30.516751953125	-6.88275220375361e-10\\
-30.4966025390625	-7.22312548942399e-10\\
-30.476453125	-6.78724214632007e-10\\
-30.4563037109375	-5.96876173389396e-10\\
-30.436154296875	-6.02773403965726e-10\\
-30.4160048828125	-6.09934640532276e-10\\
-30.39585546875	-6.2301004181665e-10\\
-30.3757060546875	-6.51563326608471e-10\\
-30.355556640625	-5.74479684181941e-10\\
-30.3354072265625	-5.31449180421039e-10\\
-30.3152578125	-5.71646291034615e-10\\
-30.2951083984375	-4.33355909058951e-10\\
-30.274958984375	-4.27937022078791e-10\\
-30.2548095703125	-4.23842215793787e-10\\
-30.23466015625	-3.08453422730445e-10\\
-30.2145107421875	-4.46905885598805e-10\\
-30.194361328125	-4.13686906434258e-10\\
-30.1742119140625	-3.62782089766111e-10\\
-30.1540625	-3.26916399702423e-10\\
-30.1339130859375	-3.40576896799072e-10\\
-30.113763671875	-2.21975831980668e-10\\
-30.0936142578125	-2.38120649853214e-10\\
-30.07346484375	-2.08743879068482e-10\\
-30.0533154296875	-1.31505534934951e-10\\
-30.033166015625	-7.13120633154084e-11\\
-30.0130166015625	-5.14245081903101e-11\\
-29.9928671875	-8.75273005217232e-11\\
-29.9727177734375	-8.77041598534916e-11\\
-29.952568359375	-8.020399863212e-11\\
-29.9324189453125	-8.6326921513141e-11\\
-29.91226953125	-8.25925758710123e-11\\
-29.8921201171875	-7.82295959407536e-11\\
-29.871970703125	-5.24583220218841e-11\\
-29.8518212890625	1.286553499326e-12\\
-29.831671875	2.30424559605468e-11\\
-29.8115224609375	5.09998527224752e-11\\
-29.791373046875	5.03538237458951e-11\\
-29.7712236328125	1.70846062718908e-11\\
-29.75107421875	7.52317317491147e-11\\
-29.7309248046875	8.24049501541745e-11\\
-29.710775390625	5.22926385744651e-11\\
-29.6906259765625	1.44510184542837e-10\\
-29.6704765625	2.01195553353335e-10\\
-29.6503271484375	2.61636808462241e-10\\
-29.630177734375	3.0104589230239e-10\\
-29.6100283203125	2.80741131348938e-10\\
-29.58987890625	3.49385750817531e-10\\
-29.5697294921875	3.11921122652723e-10\\
-29.549580078125	2.46352963429888e-10\\
-29.5294306640625	2.46503724839975e-10\\
-29.50928125	1.5690832626037e-10\\
-29.4891318359375	2.47636446643861e-10\\
-29.468982421875	2.86910675076836e-10\\
-29.4488330078125	2.81044567342142e-10\\
-29.42868359375	3.35291572243942e-10\\
-29.4085341796875	4.15311021797601e-10\\
-29.388384765625	4.43472875314317e-10\\
-29.3682353515625	4.82979663674578e-10\\
-29.3480859375	5.53032313679154e-10\\
-29.3279365234375	4.85019741593348e-10\\
-29.307787109375	4.78074315512698e-10\\
-29.2876376953125	4.63835629552618e-10\\
-29.26748828125	5.30204528214694e-10\\
-29.2473388671875	5.47323076607027e-10\\
-29.227189453125	5.35820869218366e-10\\
-29.2070400390625	5.1406916305995e-10\\
-29.186890625	5.14760352608826e-10\\
-29.1667412109375	5.05745310170751e-10\\
-29.146591796875	6.28050593373857e-10\\
-29.1264423828125	6.49758565561987e-10\\
-29.10629296875	6.61357394289303e-10\\
-29.0861435546875	6.25832587611115e-10\\
-29.065994140625	6.80935303488384e-10\\
-29.0458447265625	6.15795831999358e-10\\
-29.0256953125	6.32026877168491e-10\\
-29.0055458984375	6.40213630356829e-10\\
-28.985396484375	6.10395810790866e-10\\
-28.9652470703125	6.46028866473367e-10\\
-28.94509765625	5.80932584283426e-10\\
-28.9249482421875	6.0735713281298e-10\\
-28.904798828125	6.31632099780142e-10\\
-28.8846494140625	6.18126155451346e-10\\
-28.8645	6.50594867742967e-10\\
-28.8443505859375	6.48666691443725e-10\\
-28.824201171875	6.70994731817419e-10\\
-28.8040517578125	7.0281314607472e-10\\
-28.78390234375	7.45002513027948e-10\\
-28.7637529296875	7.04358190125405e-10\\
-28.743603515625	6.60183538693489e-10\\
-28.7234541015625	6.27000642237445e-10\\
-28.7033046875	6.25775349961348e-10\\
-28.6831552734375	5.13889658242967e-10\\
-28.663005859375	5.89264848907228e-10\\
-28.6428564453125	4.52726597922866e-10\\
-28.62270703125	4.48318639499798e-10\\
-28.6025576171875	5.94095115826446e-10\\
-28.582408203125	5.48107207969749e-10\\
-28.5622587890625	6.63404415310843e-10\\
-28.542109375	7.06137730574603e-10\\
-28.5219599609375	6.56758038873424e-10\\
-28.501810546875	6.54597946242742e-10\\
-28.4816611328125	6.51126966664166e-10\\
-28.46151171875	7.17690158536946e-10\\
-28.4413623046875	7.03062084296037e-10\\
-28.421212890625	6.25492038235308e-10\\
-28.4010634765625	6.32425001545337e-10\\
-28.3809140625	5.63575523028053e-10\\
-28.3607646484375	5.97275252876149e-10\\
-28.340615234375	5.04501538811551e-10\\
-28.3204658203125	4.8548080271018e-10\\
-28.30031640625	4.56265729925698e-10\\
-28.2801669921875	4.46908797930112e-10\\
-28.260017578125	4.44803749659648e-10\\
-28.2398681640625	5.32740386547229e-10\\
-28.21971875	5.95705432763588e-10\\
-28.1995693359375	5.45921141738671e-10\\
-28.179419921875	5.27888879596925e-10\\
-28.1592705078125	4.66119496099315e-10\\
-28.13912109375	5.19104847840288e-10\\
-28.1189716796875	4.66825882545978e-10\\
-28.098822265625	4.12898515674715e-10\\
-28.0786728515625	3.0429280416688e-10\\
-28.0585234375	3.90444601414727e-10\\
-28.0383740234375	3.63625400354666e-10\\
-28.018224609375	3.56925563910419e-10\\
-27.9980751953125	3.49911192755136e-10\\
-27.97792578125	3.33021602053508e-10\\
-27.9577763671875	3.719858958451e-10\\
-27.937626953125	3.4695455136108e-10\\
-27.9174775390625	3.01082558542468e-10\\
-27.897328125	2.99736900532082e-10\\
-27.8771787109375	2.3312774208249e-10\\
-27.857029296875	1.72218198184387e-10\\
-27.8368798828125	2.26566463749944e-10\\
-27.81673046875	1.93854645209371e-10\\
-27.7965810546875	1.97697530551903e-10\\
-27.776431640625	1.05387554266354e-10\\
-27.7562822265625	1.69276999076046e-10\\
-27.7361328125	8.47971288484425e-11\\
-27.7159833984375	5.31083892281575e-11\\
-27.695833984375	9.6100826089725e-12\\
-27.6756845703125	-5.98199352094232e-11\\
-27.65553515625	-7.91286215171928e-11\\
-27.6353857421875	4.64629797085267e-12\\
-27.615236328125	-3.01726796139037e-11\\
-27.5950869140625	-1.57177896173676e-10\\
-27.5749375	-5.21671074368004e-11\\
-27.5547880859375	-7.13710350849734e-11\\
-27.534638671875	-4.86176619255877e-11\\
-27.5144892578125	-9.51060696331599e-11\\
-27.49433984375	-9.72849963669179e-11\\
-27.4741904296875	-2.33373935510791e-10\\
-27.454041015625	-3.23058018683507e-10\\
-27.4338916015625	-4.17292733165494e-10\\
-27.4137421875	-4.16756661077823e-10\\
-27.3935927734375	-3.76793929891587e-10\\
-27.373443359375	-3.59231018832744e-10\\
-27.3532939453125	-3.70505762169908e-10\\
-27.33314453125	-3.0021848305402e-10\\
-27.3129951171875	-2.62322571753922e-10\\
-27.292845703125	-2.88239620071753e-10\\
-27.2726962890625	-3.42459120939096e-10\\
-27.252546875	-4.06361203129097e-10\\
-27.2323974609375	-6.13938757094917e-10\\
-27.212248046875	-6.81745153657533e-10\\
-27.1920986328125	-7.80231141302627e-10\\
-27.17194921875	-7.53347021726505e-10\\
-27.1517998046875	-6.98905638296935e-10\\
-27.131650390625	-6.84160785202786e-10\\
-27.1115009765625	-5.26439796178945e-10\\
-27.0913515625	-4.59718946871801e-10\\
-27.0712021484375	-5.51550740160077e-10\\
-27.051052734375	-5.29163312656365e-10\\
-27.0309033203125	-6.17333894308348e-10\\
-27.01075390625	-7.21820378224868e-10\\
-26.9906044921875	-8.33624273814058e-10\\
-26.970455078125	-8.65297421064993e-10\\
-26.9503056640625	-8.96618536915733e-10\\
-26.93015625	-9.25801775836673e-10\\
-26.9100068359375	-9.26980124426054e-10\\
-26.889857421875	-8.18193553958544e-10\\
-26.8697080078125	-8.14530201985943e-10\\
-26.84955859375	-9.17813183461643e-10\\
-26.8294091796875	-7.9323024618637e-10\\
-26.809259765625	-8.61061692016806e-10\\
-26.7891103515625	-8.43864823794651e-10\\
-26.7689609375	-8.99359297343186e-10\\
-26.7488115234375	-9.55290686384182e-10\\
-26.728662109375	-1.00045290689921e-09\\
-26.7085126953125	-9.88461695763871e-10\\
-26.68836328125	-9.68809158289675e-10\\
-26.6682138671875	-9.8860688821792e-10\\
-26.648064453125	-9.87592340191046e-10\\
-26.6279150390625	-8.77435265518271e-10\\
-26.607765625	-8.87367887283068e-10\\
-26.5876162109375	-9.4938090323076e-10\\
-26.567466796875	-9.07221468514353e-10\\
-26.5473173828125	-9.81296443706973e-10\\
-26.52716796875	-9.42652410750864e-10\\
-26.5070185546875	-9.60049191278727e-10\\
-26.486869140625	-1.00320267419179e-09\\
-26.4667197265625	-9.5898440554817e-10\\
-26.4465703125	-9.86140935371648e-10\\
-26.4264208984375	-9.02484611604511e-10\\
-26.406271484375	-8.57935225389224e-10\\
-26.3861220703125	-8.34047481738166e-10\\
-26.36597265625	-7.64873066107819e-10\\
-26.3458232421875	-9.14153440912429e-10\\
-26.325673828125	-8.75252192596651e-10\\
-26.3055244140625	-9.37609362425539e-10\\
-26.285375	-1.02913672684707e-09\\
-26.2652255859375	-9.78549488530355e-10\\
-26.245076171875	-1.12161193670813e-09\\
-26.2249267578125	-1.03856492132381e-09\\
-26.20477734375	-9.02592648522906e-10\\
-26.1846279296875	-7.73809876637848e-10\\
-26.164478515625	-6.57514802324396e-10\\
-26.1443291015625	-6.18759056967378e-10\\
-26.1241796875	-5.78197759878716e-10\\
-26.1040302734375	-7.41850144872791e-10\\
-26.083880859375	-7.88801180583631e-10\\
-26.0637314453125	-7.99236274684805e-10\\
-26.04358203125	-9.8349768957892e-10\\
-26.0234326171875	-1.02263968871247e-09\\
-26.003283203125	-9.81558540384951e-10\\
-25.9831337890625	-9.30757012239513e-10\\
-25.962984375	-8.29023705478457e-10\\
-25.9428349609375	-6.35285325464256e-10\\
-25.922685546875	-5.8222248581312e-10\\
-25.9025361328125	-4.56144932808255e-10\\
-25.88238671875	-5.46952636423724e-10\\
-25.8622373046875	-4.75483859744923e-10\\
-25.842087890625	-5.7874856032367e-10\\
-25.8219384765625	-6.70487167054997e-10\\
-25.8017890625	-7.28414889517997e-10\\
-25.7816396484375	-8.58342890693521e-10\\
-25.761490234375	-8.46914415298277e-10\\
-25.7413408203125	-7.20976372038954e-10\\
-25.72119140625	-6.88230609079533e-10\\
-25.7010419921875	-5.10124372854822e-10\\
-25.680892578125	-4.49576800656374e-10\\
-25.6607431640625	-3.78862272715589e-10\\
-25.64059375	-4.04506620995385e-10\\
-25.6204443359375	-4.11703326585681e-10\\
-25.600294921875	-4.11228137532308e-10\\
-25.5801455078125	-4.85235396097136e-10\\
-25.55999609375	-5.72499646983868e-10\\
-25.5398466796875	-5.39697138178846e-10\\
-25.519697265625	-5.14296008330187e-10\\
-25.4995478515625	-4.30591206330103e-10\\
-25.4793984375	-3.4899294535569e-10\\
-25.4592490234375	-2.53897525919507e-10\\
-25.439099609375	-2.15640850072392e-10\\
-25.4189501953125	-2.42339296233497e-10\\
-25.39880078125	-2.55115719995551e-10\\
-25.3786513671875	-2.61820385414309e-10\\
-25.358501953125	-1.91601112573658e-10\\
-25.3383525390625	-3.27884595467914e-10\\
-25.318203125	-2.04612523239736e-10\\
-25.2980537109375	-2.01113754511404e-10\\
-25.277904296875	-1.16542296043218e-10\\
-25.2577548828125	-8.31909551350722e-11\\
-25.23760546875	-2.3316018055161e-11\\
-25.2174560546875	1.77867410602872e-11\\
-25.197306640625	9.6730440794428e-11\\
-25.1771572265625	1.30946704751215e-10\\
-25.1570078125	1.24116564900479e-10\\
-25.1368583984375	1.49528768344428e-10\\
-25.116708984375	2.00817850350528e-10\\
-25.0965595703125	2.41371829352218e-10\\
-25.07641015625	2.50586043364858e-10\\
-25.0562607421875	2.94481196799793e-10\\
-25.036111328125	3.81711190482501e-10\\
-25.0159619140625	4.54516460282757e-10\\
-24.9958125	4.3518767284731e-10\\
-24.9756630859375	5.16696427236782e-10\\
-24.955513671875	5.90400478563624e-10\\
-24.9353642578125	5.50539440723632e-10\\
-24.91521484375	6.87010432384422e-10\\
-24.8950654296875	6.25737491537863e-10\\
-24.874916015625	7.89635436409015e-10\\
-24.8547666015625	8.25715740160503e-10\\
-24.8346171875	8.14434253415454e-10\\
-24.8144677734375	8.10379136584346e-10\\
-24.794318359375	8.12840224845506e-10\\
-24.7741689453125	8.01221902580002e-10\\
-24.75401953125	6.98898460736056e-10\\
-24.7338701171875	7.87279415130256e-10\\
-24.713720703125	8.34890310853267e-10\\
-24.6935712890625	8.21733774666995e-10\\
-24.673421875	8.87558733014377e-10\\
-24.6532724609375	9.45336295387691e-10\\
-24.633123046875	9.90109597811151e-10\\
-24.6129736328125	9.81568445478875e-10\\
-24.59282421875	9.91073566019442e-10\\
-24.5726748046875	9.56818462844565e-10\\
-24.552525390625	9.98064138931443e-10\\
-24.5323759765625	1.00661451761898e-09\\
-24.5122265625	9.83980508080294e-10\\
-24.4920771484375	1.01794021062828e-09\\
-24.471927734375	9.67632280585002e-10\\
-24.4517783203125	1.06956247052656e-09\\
-24.43162890625	1.07880946120524e-09\\
-24.4114794921875	1.15876004384863e-09\\
-24.391330078125	1.17296539209481e-09\\
-24.3711806640625	1.23535266044511e-09\\
-24.35103125	1.37508892849322e-09\\
-24.3308818359375	1.31516092807291e-09\\
-24.310732421875	1.3247870366044e-09\\
-24.2905830078125	1.39441047843164e-09\\
-24.27043359375	1.35944209442327e-09\\
-24.2502841796875	1.38240152844611e-09\\
-24.230134765625	1.33603879469921e-09\\
-24.2099853515625	1.32314122303789e-09\\
-24.1898359375	1.30280297181578e-09\\
-24.1696865234375	1.3311537096882e-09\\
-24.149537109375	1.41069945992454e-09\\
-24.1293876953125	1.34942590363562e-09\\
-24.10923828125	1.46952983337411e-09\\
-24.0890888671875	1.57409793895687e-09\\
-24.068939453125	1.60953233593913e-09\\
-24.0487900390625	1.76285003508289e-09\\
-24.028640625	1.6739196274369e-09\\
-24.0084912109375	1.5864646859213e-09\\
-23.988341796875	1.59787329873927e-09\\
-23.9681923828125	1.45114819200424e-09\\
-23.94804296875	1.36987162443244e-09\\
-23.9278935546875	1.3503007316713e-09\\
-23.907744140625	1.31173322718765e-09\\
-23.8875947265625	1.2242035826842e-09\\
-23.8674453125	1.33849234295465e-09\\
-23.8472958984375	1.43958382851836e-09\\
-23.827146484375	1.41329394592586e-09\\
-23.8069970703125	1.46262646586612e-09\\
-23.78684765625	1.4654508598051e-09\\
-23.7666982421875	1.43724401831738e-09\\
-23.746548828125	1.42914459372655e-09\\
-23.7263994140625	1.47756068917275e-09\\
-23.70625	1.26364124763868e-09\\
-23.6861005859375	1.27440754234419e-09\\
-23.665951171875	1.09604735126109e-09\\
-23.6458017578125	1.05251837008782e-09\\
-23.62565234375	1.08852553517727e-09\\
-23.6055029296875	1.10069932522581e-09\\
-23.585353515625	1.12568781714916e-09\\
-23.5652041015625	1.24550858102218e-09\\
-23.5450546875	1.28936010107195e-09\\
-23.5249052734375	1.39316205813058e-09\\
-23.504755859375	1.31858954542594e-09\\
-23.4846064453125	1.1692070471065e-09\\
-23.46445703125	1.11566287594915e-09\\
-23.4443076171875	9.55012426421099e-10\\
-23.424158203125	1.0590307549243e-09\\
-23.4040087890625	9.00649698046819e-10\\
-23.383859375	1.07326581790567e-09\\
-23.3637099609375	1.16261504687099e-09\\
-23.343560546875	1.18623242436563e-09\\
-23.3234111328125	1.20149366378611e-09\\
-23.30326171875	1.28087840853967e-09\\
-23.2831123046875	1.20545055664576e-09\\
-23.262962890625	1.16294664706244e-09\\
-23.2428134765625	1.06315010715315e-09\\
-23.2226640625	9.1961509720711e-10\\
-23.2025146484375	9.22045916836768e-10\\
-23.182365234375	9.97461997316799e-10\\
-23.1622158203125	8.65527685088514e-10\\
-23.14206640625	1.02854855927996e-09\\
-23.1219169921875	1.04794882226588e-09\\
-23.101767578125	1.19147444911992e-09\\
-23.0816181640625	1.17891792131645e-09\\
-23.06146875	1.30973686522142e-09\\
-23.0413193359375	1.25616050021316e-09\\
-23.021169921875	1.23711161037825e-09\\
-23.0010205078125	1.0733606028326e-09\\
-22.98087109375	9.13486376222322e-10\\
-22.9607216796875	9.19174582282614e-10\\
-22.940572265625	7.75990151827987e-10\\
-22.9204228515625	8.63740407450987e-10\\
-22.9002734375	9.10711340720391e-10\\
-22.8801240234375	9.75901111417992e-10\\
-22.859974609375	1.05551646636686e-09\\
-22.8398251953125	9.54148392959217e-10\\
-22.81967578125	1.00012137616421e-09\\
-22.7995263671875	8.25102827967987e-10\\
-22.779376953125	7.98659862842245e-10\\
-22.7592275390625	5.92252857862991e-10\\
-22.739078125	5.3655375672352e-10\\
-22.7189287109375	5.04074848632875e-10\\
-22.698779296875	5.0578439996194e-10\\
-22.6786298828125	5.07598989177772e-10\\
-22.65848046875	5.43475986156089e-10\\
-22.6383310546875	4.53264493403909e-10\\
-22.618181640625	4.28801675915433e-10\\
-22.5980322265625	4.3092066523451e-10\\
-22.5778828125	2.65853972491097e-10\\
-22.5577333984375	1.50880549384367e-10\\
-22.537583984375	1.23125064236861e-10\\
-22.5174345703125	1.9453484508218e-11\\
-22.49728515625	3.46063271594175e-11\\
-22.4771357421875	9.63288405868209e-11\\
-22.456986328125	-2.7547852919308e-11\\
-22.4368369140625	1.23634084924692e-11\\
-22.4166875	-2.10892545626346e-11\\
-22.3965380859375	-3.42008619837429e-11\\
-22.376388671875	-5.39040670938846e-11\\
-22.3562392578125	-6.36355455766861e-11\\
-22.33608984375	-3.65683824161063e-11\\
-22.3159404296875	-7.96449651902297e-11\\
-22.295791015625	3.70842385929304e-12\\
-22.2756416015625	-9.21378232536823e-11\\
-22.2554921875	7.19424399551061e-11\\
-22.2353427734375	-4.19428518577039e-11\\
-22.215193359375	-4.16208378603984e-11\\
-22.1950439453125	-1.16690270007926e-10\\
-22.17489453125	-9.80931784094891e-11\\
-22.1547451171875	-1.50843449802301e-10\\
-22.134595703125	-1.70380606300871e-10\\
-22.1144462890625	-1.62283224831423e-10\\
-22.094296875	-3.13577837085103e-10\\
-22.0741474609375	-3.31370509153555e-10\\
-22.053998046875	-3.11179722391452e-10\\
-22.0338486328125	-4.07174976219238e-10\\
-22.01369921875	-3.15943237783098e-10\\
-21.9935498046875	-3.33303360945344e-10\\
-21.973400390625	-3.05144084729215e-10\\
-21.9532509765625	-2.99799270503006e-10\\
-21.9331015625	-3.46571822259933e-10\\
-21.9129521484375	-4.91688898816696e-10\\
-21.892802734375	-5.41499733820913e-10\\
-21.8726533203125	-6.84913499017016e-10\\
-21.85250390625	-8.01260585083849e-10\\
-21.8323544921875	-9.08690264642314e-10\\
-21.812205078125	-9.43738263020803e-10\\
-21.7920556640625	-1.00523163400898e-09\\
-21.77190625	-9.27899329528342e-10\\
-21.7517568359375	-8.58196485109931e-10\\
-21.731607421875	-7.87080887722236e-10\\
-21.7114580078125	-7.54409397573002e-10\\
-21.69130859375	-7.5542993801648e-10\\
-21.6711591796875	-7.62936303005442e-10\\
-21.651009765625	-9.82075710991384e-10\\
-21.6308603515625	-1.03545576761357e-09\\
-21.6107109375	-1.15382402547728e-09\\
-21.5905615234375	-1.19459326056356e-09\\
-21.570412109375	-1.18548193553969e-09\\
-21.5502626953125	-1.12147727533728e-09\\
-21.53011328125	-1.10450806440728e-09\\
-21.5099638671875	-9.92879730664281e-10\\
-21.489814453125	-9.48247983682542e-10\\
-21.4696650390625	-7.06533527177314e-10\\
-21.449515625	-7.13111333416811e-10\\
-21.4293662109375	-7.49948293563093e-10\\
-21.409216796875	-8.53953225966978e-10\\
-21.3890673828125	-9.24802223395835e-10\\
-21.36891796875	-9.80475423452244e-10\\
-21.3487685546875	-1.05048870362017e-09\\
-21.328619140625	-1.01728277383251e-09\\
-21.3084697265625	-1.19672935393595e-09\\
-21.2883203125	-1.08859182691526e-09\\
-21.2681708984375	-1.04615668043e-09\\
-21.248021484375	-1.05843467010326e-09\\
-21.2278720703125	-1.06140697875054e-09\\
-21.20772265625	-9.46078954196293e-10\\
-21.1875732421875	-1.02784305399485e-09\\
-21.167423828125	-9.55415331502625e-10\\
-21.1472744140625	-7.52528799931744e-10\\
-21.127125	-6.72435433762146e-10\\
-21.1069755859375	-7.83207618078484e-10\\
-21.086826171875	-7.52379456615419e-10\\
-21.0666767578125	-8.43854567716554e-10\\
-21.04652734375	-9.10692881564997e-10\\
-21.0263779296875	-1.00798988962674e-09\\
-21.006228515625	-1.11900823462262e-09\\
-20.9860791015625	-1.18252011954058e-09\\
-20.9659296875	-1.13974144166038e-09\\
-20.9457802734375	-1.03429949554205e-09\\
-20.925630859375	-9.66686247731513e-10\\
-20.9054814453125	-8.50565428326684e-10\\
-20.88533203125	-7.83745310627193e-10\\
-20.8651826171875	-7.18274780704071e-10\\
-20.845033203125	-5.30868225109189e-10\\
-20.8248837890625	-6.91592907853428e-10\\
-20.804734375	-5.76252262447607e-10\\
-20.7845849609375	-7.06109703192784e-10\\
-20.764435546875	-8.21827157792482e-10\\
-20.7442861328125	-9.10770350710301e-10\\
-20.72413671875	-9.66704393228107e-10\\
-20.7039873046875	-1.04707190047964e-09\\
-20.683837890625	-1.05915045730476e-09\\
-20.6636884765625	-1.00182726616007e-09\\
-20.6435390625	-9.73108001324406e-10\\
-20.6233896484375	-8.46128348892079e-10\\
-20.603240234375	-9.47488316615665e-10\\
-20.5830908203125	-7.53645268926811e-10\\
-20.56294140625	-6.86106301232083e-10\\
-20.5427919921875	-7.99605937707696e-10\\
-20.522642578125	-7.36086293163375e-10\\
-20.5024931640625	-7.27717496633869e-10\\
-20.48234375	-6.42407779290096e-10\\
-20.4621943359375	-7.36884027416652e-10\\
-20.442044921875	-6.92519620822899e-10\\
-20.4218955078125	-7.4324305142353e-10\\
-20.40174609375	-7.35317756433465e-10\\
-20.3815966796875	-6.17997559698559e-10\\
-20.361447265625	-6.92816962729517e-10\\
-20.3412978515625	-6.86230825722959e-10\\
-20.3211484375	-6.65708373990626e-10\\
-20.3009990234375	-4.8434214627379e-10\\
-20.280849609375	-5.17509326216498e-10\\
-20.2607001953125	-5.42465170025438e-10\\
-20.24055078125	-5.45097683690809e-10\\
-20.2204013671875	-4.90686180311539e-10\\
-20.200251953125	-5.81137104135074e-10\\
-20.1801025390625	-4.96373314933251e-10\\
-20.159953125	-6.1558337745802e-10\\
-20.1398037109375	-6.53655899622509e-10\\
-20.119654296875	-6.37700446905398e-10\\
-20.0995048828125	-5.75848901021198e-10\\
-20.07935546875	-4.66118317147112e-10\\
-20.0592060546875	-4.32216142446853e-10\\
-20.039056640625	-3.6152737241298e-10\\
-20.0189072265625	-2.88774028370514e-10\\
-19.9987578125	-2.10821663618144e-10\\
-19.9786083984375	-7.82042975138185e-11\\
-19.958458984375	-1.14904611443791e-11\\
-19.9383095703125	1.68592363395308e-12\\
-19.91816015625	2.67978969699912e-11\\
-19.8980107421875	1.2107141065005e-10\\
-19.877861328125	1.99819091065352e-10\\
-19.8577119140625	2.36770072657713e-10\\
-19.8375625	2.15593994629169e-10\\
-19.8174130859375	1.75741659999621e-10\\
-19.797263671875	1.29034071422785e-10\\
-19.7771142578125	1.29827791087039e-10\\
-19.75696484375	1.75046411445219e-10\\
-19.7368154296875	3.34421870696345e-10\\
-19.716666015625	2.99387584550757e-10\\
-19.6965166015625	4.14855451095434e-10\\
-19.6763671875	5.70376598456175e-10\\
-19.6562177734375	6.34043620228336e-10\\
-19.636068359375	7.69999927448405e-10\\
-19.6159189453125	6.84105799823144e-10\\
-19.59576953125	6.47637068275776e-10\\
-19.5756201171875	5.74427977813305e-10\\
-19.555470703125	4.91431954140971e-10\\
-19.5353212890625	4.83654384691234e-10\\
-19.515171875	5.00489063058726e-10\\
-19.4950224609375	3.87436008110955e-10\\
-19.474873046875	5.15167559633504e-10\\
-19.4547236328125	5.28525623677158e-10\\
-19.43457421875	6.16645942631987e-10\\
-19.4144248046875	7.97011516153748e-10\\
-19.394275390625	7.6716746785012e-10\\
-19.3741259765625	7.96004956110153e-10\\
-19.3539765625	7.52637349414876e-10\\
-19.3338271484375	4.75249389787891e-10\\
-19.313677734375	4.65189629886961e-10\\
-19.2935283203125	4.34061668085775e-10\\
-19.27337890625	2.74105623920711e-10\\
-19.2532294921875	4.07390566495882e-10\\
-19.233080078125	3.93889396006692e-10\\
-19.2129306640625	5.87551453225165e-10\\
-19.19278125	7.1881271913055e-10\\
-19.1726318359375	7.05917383152413e-10\\
-19.152482421875	7.25176525448925e-10\\
-19.1323330078125	7.69651957378599e-10\\
-19.11218359375	5.81146678656855e-10\\
-19.0920341796875	5.53213266906521e-10\\
-19.071884765625	5.34076183396614e-10\\
-19.0517353515625	5.86161531546212e-10\\
-19.0315859375	6.31862021533591e-10\\
-19.0114365234375	7.48592639252816e-10\\
-18.991287109375	9.65659198698888e-10\\
-18.9711376953125	9.14790229361026e-10\\
-18.95098828125	1.01755983288967e-09\\
-18.9308388671875	9.50682174732246e-10\\
-18.910689453125	9.31515344666561e-10\\
-18.8905400390625	8.17545921764882e-10\\
-18.870390625	7.36396887847227e-10\\
-18.8502412109375	7.19415444872813e-10\\
-18.830091796875	7.91041777244755e-10\\
-18.8099423828125	7.81375805977618e-10\\
-18.78979296875	9.61515153488358e-10\\
-18.7696435546875	9.722084312165e-10\\
-18.749494140625	1.11867070607195e-09\\
-18.7293447265625	1.1312888591405e-09\\
-18.7091953125	1.00836115537274e-09\\
-18.6890458984375	8.48444082639874e-10\\
-18.668896484375	7.91304884084884e-10\\
-18.6487470703125	6.29674188873176e-10\\
-18.62859765625	5.85438132978748e-10\\
-18.6084482421875	6.01221875537274e-10\\
-18.588298828125	6.5203463181818e-10\\
-18.5681494140625	8.4526936470161e-10\\
-18.548	1.01278326506147e-09\\
-18.5278505859375	1.15007831854233e-09\\
-18.507701171875	1.21209138425444e-09\\
-18.4875517578125	1.2871908029793e-09\\
-18.46740234375	1.09742630863867e-09\\
-18.4472529296875	1.09067773784957e-09\\
-18.427103515625	9.07046921933302e-10\\
-18.4069541015625	7.65322921174385e-10\\
-18.3868046875	6.88660334627423e-10\\
-18.3666552734375	6.18329494765135e-10\\
-18.346505859375	7.29426326673959e-10\\
-18.3263564453125	7.25551204483134e-10\\
-18.30620703125	8.38176002620391e-10\\
-18.2860576171875	9.89030600587022e-10\\
-18.265908203125	1.03813617137277e-09\\
-18.2457587890625	1.09170557354302e-09\\
-18.225609375	1.07760397609775e-09\\
-18.2054599609375	9.15306662448801e-10\\
-18.185310546875	8.50458017874562e-10\\
-18.1651611328125	6.92561105229468e-10\\
-18.14501171875	6.6744969869322e-10\\
-18.1248623046875	6.26352602332428e-10\\
-18.104712890625	7.12725932778312e-10\\
-18.0845634765625	8.27583705627044e-10\\
-18.0644140625	8.88364491817135e-10\\
-18.0442646484375	7.61385434672759e-10\\
-18.024115234375	9.17310090176172e-10\\
-18.0039658203125	8.21890877228509e-10\\
-17.98381640625	8.57629338043397e-10\\
-17.9636669921875	8.42437488532071e-10\\
-17.943517578125	9.58436153862441e-10\\
-17.9233681640625	7.75688551163589e-10\\
-17.90321875	7.52126460308735e-10\\
-17.8830693359375	7.38276962349764e-10\\
-17.862919921875	5.56913967897467e-10\\
-17.8427705078125	5.99018096160858e-10\\
-17.82262109375	6.57987325496031e-10\\
-17.8024716796875	5.21441093848825e-10\\
-17.782322265625	6.00047325936706e-10\\
-17.7621728515625	5.99475312965157e-10\\
-17.7420234375	5.78729147326128e-10\\
-17.7218740234375	6.26708174782653e-10\\
-17.701724609375	5.61697646218499e-10\\
-17.6815751953125	5.65800121090067e-10\\
-17.66142578125	4.94883776499911e-10\\
-17.6412763671875	4.09746807438063e-10\\
-17.621126953125	3.66994494084762e-10\\
-17.6009775390625	3.86843427739059e-10\\
-17.580828125	3.78016467509036e-10\\
-17.5606787109375	3.91069996017961e-10\\
-17.540529296875	3.40948429574364e-10\\
-17.5203798828125	4.36394954770892e-10\\
-17.50023046875	3.76810466162761e-10\\
-17.4800810546875	3.3600308642791e-10\\
-17.459931640625	2.1885261762312e-10\\
-17.4397822265625	1.52851506993026e-10\\
-17.4196328125	7.25961418086041e-11\\
-17.3994833984375	8.72925358657768e-11\\
-17.379333984375	4.29052014194948e-11\\
-17.3591845703125	-1.13799714538757e-10\\
-17.33903515625	-3.83274200150091e-12\\
-17.3188857421875	-1.05873851042815e-10\\
-17.298736328125	-6.9687943480993e-11\\
-17.2785869140625	-1.45902562598771e-10\\
-17.2584375	-1.60888271442502e-10\\
-17.2382880859375	-2.89324421150407e-10\\
-17.218138671875	-2.11115410077386e-10\\
-17.1979892578125	-2.24111702218596e-10\\
-17.17783984375	-1.54703996691278e-10\\
-17.1576904296875	-2.21547163745723e-10\\
-17.137541015625	-2.57275189256513e-10\\
-17.1173916015625	-2.26816405564126e-10\\
-17.0972421875	-1.85043584811662e-10\\
-17.0770927734375	-3.13491866287683e-10\\
-17.056943359375	-2.81176404702819e-10\\
-17.0367939453125	-4.13410387937241e-10\\
-17.01664453125	-3.03256912265721e-10\\
-16.9964951171875	-2.6379067364639e-10\\
-16.976345703125	-2.800361529706e-10\\
-16.9561962890625	-2.77620909375217e-10\\
-16.936046875	-2.24949159125089e-10\\
-16.9158974609375	-3.07963600626119e-10\\
-16.895748046875	-3.4133854351256e-10\\
-16.8755986328125	-3.80046925679888e-10\\
-16.85544921875	-4.47042447342692e-10\\
-16.8352998046875	-4.82017766286861e-10\\
-16.815150390625	-5.12842326497603e-10\\
-16.7950009765625	-4.63999069965389e-10\\
-16.7748515625	-4.36485357968663e-10\\
-16.7547021484375	-3.82935664250822e-10\\
-16.734552734375	-4.61902499536123e-10\\
-16.7144033203125	-3.5456441813964e-10\\
-16.69425390625	-3.96889952665076e-10\\
-16.6741044921875	-3.61723833697815e-10\\
-16.653955078125	-5.48777867651687e-10\\
-16.6338056640625	-5.28893661944472e-10\\
-16.61365625	-6.59730311186978e-10\\
-16.5935068359375	-6.00810938120683e-10\\
-16.573357421875	-6.65477021066714e-10\\
-16.5532080078125	-5.84136030080054e-10\\
-16.53305859375	-5.714142551447e-10\\
-16.5129091796875	-5.42720523478931e-10\\
-16.492759765625	-6.55454794639933e-10\\
-16.4726103515625	-6.18819448601885e-10\\
-16.4524609375	-7.3210519567489e-10\\
-16.4323115234375	-7.43902364822919e-10\\
-16.412162109375	-9.15402163756514e-10\\
-16.3920126953125	-9.06205774327059e-10\\
-16.37186328125	-7.22186986630333e-10\\
-16.3517138671875	-6.20867709078956e-10\\
-16.331564453125	-6.73457062243025e-10\\
-16.3114150390625	-5.05787027672192e-10\\
-16.291265625	-5.24629059682862e-10\\
-16.2711162109375	-5.53707552064656e-10\\
-16.250966796875	-4.90809219624623e-10\\
-16.2308173828125	-5.49839194545115e-10\\
-16.21066796875	-6.38846436651802e-10\\
-16.1905185546875	-7.09365056677997e-10\\
-16.170369140625	-7.12886400404884e-10\\
-16.1502197265625	-7.12220641990121e-10\\
-16.1300703125	-6.14713018529116e-10\\
-16.1099208984375	-6.01512537279672e-10\\
-16.089771484375	-3.44506296542195e-10\\
-16.0696220703125	-2.75296627909443e-10\\
-16.04947265625	-2.93296465369593e-10\\
-16.0293232421875	-2.9995265425655e-10\\
-16.009173828125	-3.65252350599234e-10\\
-15.9890244140625	-4.34744804214561e-10\\
-15.968875	-6.88414040351566e-10\\
-15.9487255859375	-7.03986549499021e-10\\
-15.928576171875	-6.98011279267381e-10\\
-15.9084267578125	-5.54240732010963e-10\\
-15.88827734375	-6.448152453399e-10\\
-15.8681279296875	-5.41011999963936e-10\\
-15.847978515625	-5.34899580184689e-10\\
-15.8278291015625	-5.40055726429779e-10\\
-15.8076796875	-4.69812263667946e-10\\
-15.7875302734375	-5.54888205735416e-10\\
-15.767380859375	-5.03889597261952e-10\\
-15.7472314453125	-5.86860064570087e-10\\
-15.72708203125	-5.35213972410584e-10\\
-15.7069326171875	-5.91880726179422e-10\\
-15.686783203125	-5.13339853232414e-10\\
-15.6666337890625	-6.18986349216106e-10\\
-15.646484375	-5.60733207147842e-10\\
-15.6263349609375	-5.50096923159647e-10\\
-15.606185546875	-5.15431370441876e-10\\
-15.5860361328125	-5.2174513032015e-10\\
-15.56588671875	-4.44707871977572e-10\\
-15.5457373046875	-3.2839517521744e-10\\
-15.525587890625	-3.85726505454452e-10\\
-15.5054384765625	-3.65374599432106e-10\\
-15.4852890625	-3.43210381153522e-10\\
-15.4651396484375	-3.78080022567299e-10\\
-15.444990234375	-4.09208726172903e-10\\
-15.4248408203125	-3.15971987808508e-10\\
-15.40469140625	-3.19093059351204e-10\\
-15.3845419921875	-3.18580466661762e-10\\
-15.364392578125	-3.92497472808438e-10\\
-15.3442431640625	-2.81225683158226e-10\\
-15.32409375	-4.11416421416792e-10\\
-15.3039443359375	-2.27418557389332e-10\\
-15.283794921875	-2.33153470291281e-10\\
-15.2636455078125	-1.78513308985213e-10\\
-15.24349609375	-4.77689980397013e-11\\
-15.2233466796875	-8.66891278412442e-12\\
-15.203197265625	-7.76499228314291e-12\\
-15.1830478515625	-5.21111368290403e-12\\
-15.1628984375	-1.60990295386848e-11\\
-15.1427490234375	-1.3812078151613e-11\\
-15.122599609375	-6.31975264528482e-13\\
-15.1024501953125	-1.54172184665485e-10\\
-15.08230078125	-1.08733166686743e-11\\
-15.0621513671875	-4.66628681263867e-12\\
-15.042001953125	5.11599185450558e-11\\
-15.0218525390625	2.67938094337589e-11\\
-15.001703125	1.38633686895177e-10\\
-14.9815537109375	1.61148620462202e-10\\
-14.961404296875	1.58398090324937e-10\\
-14.9412548828125	1.8575680441025e-10\\
-14.92110546875	1.41442022715862e-10\\
-14.9009560546875	1.40865594253763e-10\\
-14.880806640625	1.5952891175747e-10\\
-14.8606572265625	1.11783668077903e-10\\
-14.8405078125	1.25445173715065e-10\\
-14.8203583984375	4.73124399632025e-11\\
-14.800208984375	1.03586284718181e-10\\
-14.7800595703125	1.92889221750891e-10\\
-14.75991015625	2.27162925588057e-10\\
-14.7397607421875	3.33990932458079e-10\\
-14.719611328125	4.23059081527282e-10\\
-14.6994619140625	3.85187781159673e-10\\
-14.6793125	4.5810825743376e-10\\
-14.6591630859375	4.60358903237752e-10\\
-14.639013671875	4.42875129046472e-10\\
-14.6188642578125	5.21767507966805e-10\\
-14.59871484375	4.85869406128447e-10\\
-14.5785654296875	4.60873213263577e-10\\
-14.558416015625	5.55901228220738e-10\\
-14.5382666015625	5.68111733433527e-10\\
-14.5181171875	6.74678632627018e-10\\
-14.4979677734375	6.77992341915501e-10\\
-14.477818359375	7.66281321818208e-10\\
-14.4576689453125	7.61532867242579e-10\\
-14.43751953125	6.98406287625422e-10\\
-14.4173701171875	7.71607158423309e-10\\
-14.397220703125	7.56580788255177e-10\\
-14.3770712890625	7.20322664954625e-10\\
-14.356921875	6.81207841570948e-10\\
-14.3367724609375	8.72500215187082e-10\\
-14.316623046875	9.10245866034927e-10\\
-14.2964736328125	1.11463679290826e-09\\
-14.27632421875	1.0702608278375e-09\\
-14.2561748046875	1.23326071433992e-09\\
-14.236025390625	1.24749063604681e-09\\
-14.2158759765625	1.26373189345477e-09\\
-14.1957265625	1.20175536501298e-09\\
-14.1755771484375	1.06884728440036e-09\\
-14.155427734375	1.07720738855107e-09\\
-14.1352783203125	9.79577145033131e-10\\
-14.11512890625	1.06315317173588e-09\\
-14.0949794921875	1.01997407425041e-09\\
-14.074830078125	1.14317863399494e-09\\
-14.0546806640625	1.20070472046916e-09\\
-14.03453125	1.06293706888909e-09\\
-14.0143818359375	1.17688708133971e-09\\
-13.994232421875	1.22870542819658e-09\\
-13.9740830078125	1.14115537752392e-09\\
-13.95393359375	1.18616478927808e-09\\
-13.9337841796875	1.05219695177189e-09\\
-13.913634765625	1.17790356491472e-09\\
-13.8934853515625	1.14099343992265e-09\\
-13.8733359375	1.12171759506363e-09\\
-13.8531865234375	1.18938728676722e-09\\
-13.833037109375	1.12507677099183e-09\\
-13.8128876953125	1.10025403714044e-09\\
-13.79273828125	1.12475815682923e-09\\
-13.7725888671875	1.07860690504649e-09\\
-13.752439453125	1.16677791293742e-09\\
-13.7322900390625	1.17976104766477e-09\\
-13.712140625	1.11992888948251e-09\\
-13.6919912109375	1.23478145524671e-09\\
-13.671841796875	1.11676096824683e-09\\
-13.6516923828125	1.0842773506647e-09\\
-13.63154296875	1.04445826136727e-09\\
-13.6113935546875	9.82224775317274e-10\\
-13.591244140625	1.04566391696837e-09\\
-13.5710947265625	1.04917026515801e-09\\
-13.5509453125	1.12749022945847e-09\\
-13.5307958984375	9.63355408224271e-10\\
-13.510646484375	9.08500210482478e-10\\
-13.4904970703125	9.27321578272646e-10\\
-13.47034765625	9.58929398759467e-10\\
-13.4501982421875	1.06037243894235e-09\\
-13.430048828125	9.96445920069245e-10\\
-13.4098994140625	1.08314802275566e-09\\
-13.38975	1.153355177002e-09\\
-13.3696005859375	1.07677265693955e-09\\
-13.349451171875	1.05013167362605e-09\\
-13.3293017578125	1.11266910655212e-09\\
-13.30915234375	1.07487813977971e-09\\
-13.2890029296875	1.01993680058186e-09\\
-13.268853515625	1.09150293469238e-09\\
-13.2487041015625	1.05519506698989e-09\\
-13.2285546875	1.2828764577506e-09\\
-13.2084052734375	1.19084976992117e-09\\
-13.188255859375	1.31281520839583e-09\\
-13.1681064453125	1.14996825597176e-09\\
-13.14795703125	1.07605732931184e-09\\
-13.1278076171875	9.51410112327252e-10\\
-13.107658203125	9.73055479837677e-10\\
-13.0875087890625	8.4916185823506e-10\\
-13.067359375	9.27408296484715e-10\\
-13.0472099609375	9.02470931324225e-10\\
-13.027060546875	9.52718305066237e-10\\
-13.0069111328125	9.62973161218931e-10\\
-12.98676171875	8.7883180152196e-10\\
-12.9666123046875	9.05462410178405e-10\\
-12.946462890625	8.73781370181132e-10\\
-12.9263134765625	8.82413984967221e-10\\
-12.9061640625	8.92312940882179e-10\\
-12.8860146484375	8.82051175270995e-10\\
-12.865865234375	8.83949757934311e-10\\
-12.8457158203125	7.79978141621218e-10\\
-12.82556640625	7.79394718585604e-10\\
-12.8054169921875	8.2478327026993e-10\\
-12.785267578125	6.47089163942566e-10\\
-12.7651181640625	7.47366161667601e-10\\
-12.74496875	7.11444523727777e-10\\
-12.7248193359375	5.29051010653091e-10\\
-12.704669921875	6.75784426681767e-10\\
-12.6845205078125	6.36403661740751e-10\\
-12.66437109375	6.32805843521247e-10\\
-12.6442216796875	5.69556569319569e-10\\
-12.624072265625	5.01404365072065e-10\\
-12.6039228515625	5.59981287301586e-10\\
-12.5837734375	3.95704241147941e-10\\
-12.5636240234375	3.82143136121511e-10\\
-12.543474609375	4.89768923934767e-10\\
-12.5233251953125	3.71600593617065e-10\\
-12.50317578125	4.34733658653983e-10\\
-12.4830263671875	4.49207517057734e-10\\
-12.462876953125	4.46507590369098e-10\\
-12.4427275390625	3.7904752439527e-10\\
-12.422578125	2.83634656975625e-10\\
-12.4024287109375	2.35126851533883e-10\\
-12.382279296875	2.93454894205371e-10\\
-12.3621298828125	2.45037805674523e-10\\
-12.34198046875	3.39000352220311e-10\\
-12.3218310546875	3.34112123655344e-10\\
-12.301681640625	2.95334374395844e-10\\
-12.2815322265625	2.86268997686516e-10\\
-12.2613828125	2.92944295440549e-10\\
-12.2412333984375	2.91173257881554e-10\\
-12.221083984375	7.64662733770234e-11\\
-12.2009345703125	1.12441690904708e-10\\
-12.18078515625	-2.61020462700394e-12\\
-12.1606357421875	-6.80491280008393e-11\\
-12.140486328125	6.33700625627646e-11\\
-12.1203369140625	1.86530430856289e-10\\
-12.1001875	2.03484575728467e-11\\
-12.0800380859375	1.3383049770034e-10\\
-12.059888671875	4.17885556674694e-11\\
-12.0397392578125	2.81498653159361e-11\\
-12.01958984375	-1.34130885232248e-10\\
-11.9994404296875	-1.99052988619967e-10\\
-11.979291015625	-2.61356801203416e-10\\
-11.9591416015625	-3.41334214424187e-10\\
-11.9389921875	-4.47594408834142e-10\\
-11.9188427734375	-2.18023154967928e-10\\
-11.898693359375	-2.09566070040972e-10\\
-11.8785439453125	-1.37148905799281e-10\\
-11.85839453125	1.45464783921906e-11\\
-11.8382451171875	-8.85001450448442e-12\\
-11.818095703125	9.67529249795233e-12\\
-11.7979462890625	-2.92852141583826e-11\\
-11.777796875	-1.76863029380192e-10\\
-11.7576474609375	-3.51139175030918e-10\\
-11.737498046875	-4.95304965625076e-10\\
-11.7173486328125	-4.97265255217547e-10\\
-11.69719921875	-6.65578578387949e-10\\
-11.6770498046875	-4.60982903647499e-10\\
-11.656900390625	-4.862638076167e-10\\
-11.6367509765625	-4.11428284267804e-10\\
-11.6166015625	-3.46459176146202e-10\\
-11.5964521484375	-3.46190157516604e-10\\
-11.576302734375	-3.80055766952206e-10\\
-11.5561533203125	-4.77203109045716e-10\\
-11.53600390625	-6.03790610637178e-10\\
-11.5158544921875	-6.53098482236709e-10\\
-11.495705078125	-6.81564706440048e-10\\
-11.4755556640625	-6.43205004432352e-10\\
-11.45540625	-6.72980167955282e-10\\
-11.4352568359375	-6.92922028665436e-10\\
-11.415107421875	-5.89660389175218e-10\\
-11.3949580078125	-5.71597850136477e-10\\
-11.37480859375	-4.46338530427105e-10\\
-11.3546591796875	-5.66061081872931e-10\\
-11.334509765625	-5.29317760345403e-10\\
-11.3143603515625	-6.85639796502379e-10\\
-11.2942109375	-6.94358266661006e-10\\
-11.2740615234375	-6.76137908922367e-10\\
-11.253912109375	-7.15840538169536e-10\\
-11.2337626953125	-8.24184677159887e-10\\
-11.21361328125	-5.99836072257856e-10\\
-11.1934638671875	-6.76044967153043e-10\\
-11.173314453125	-5.86378270666073e-10\\
-11.1531650390625	-4.89247271629298e-10\\
-11.133015625	-6.04918063160689e-10\\
-11.1128662109375	-6.56986293646306e-10\\
-11.092716796875	-6.45001773641269e-10\\
-11.0725673828125	-7.26538783657787e-10\\
-11.05241796875	-7.42112146725898e-10\\
-11.0322685546875	-8.0398384217232e-10\\
-11.012119140625	-7.96354674589061e-10\\
-10.9919697265625	-7.86830969790149e-10\\
-10.9718203125	-6.97272160822857e-10\\
-10.9516708984375	-5.9437141469284e-10\\
-10.931521484375	-4.43015187045578e-10\\
-10.9113720703125	-5.15122045740802e-10\\
-10.89122265625	-5.57306318563766e-10\\
-10.8710732421875	-5.516879969021e-10\\
-10.850923828125	-5.94027513317723e-10\\
-10.8307744140625	-6.86394885426429e-10\\
-10.810625	-7.22622693163223e-10\\
-10.7904755859375	-7.31160716427614e-10\\
-10.770326171875	-6.74615338660852e-10\\
-10.7501767578125	-6.59196823480168e-10\\
-10.73002734375	-4.56652521211461e-10\\
-10.7098779296875	-3.92526869266509e-10\\
-10.689728515625	-3.33119092811792e-10\\
-10.6695791015625	-5.20770310966973e-10\\
-10.6494296875	-4.74848742580801e-10\\
-10.6292802734375	-5.81605871246938e-10\\
-10.609130859375	-5.54135993012654e-10\\
-10.5889814453125	-6.8090750229731e-10\\
-10.56883203125	-5.83184230751104e-10\\
-10.5486826171875	-6.31511139540817e-10\\
-10.528533203125	-4.99775098684614e-10\\
-10.5083837890625	-3.13472476581788e-10\\
-10.488234375	-3.17181329700021e-10\\
-10.4680849609375	-1.6943311306286e-10\\
-10.447935546875	-3.26226962779017e-10\\
-10.4277861328125	-2.30877597125394e-10\\
-10.40763671875	-3.23724271820756e-10\\
-10.3874873046875	-3.9800018304559e-10\\
-10.367337890625	-4.53939640432401e-10\\
-10.3471884765625	-4.23955777551463e-10\\
-10.3270390625	-4.61754733662952e-10\\
-10.3068896484375	-4.74060828236369e-10\\
-10.286740234375	-3.70013476198259e-10\\
-10.2665908203125	-2.73898132295191e-10\\
-10.24644140625	-3.15382136941566e-10\\
-10.2262919921875	-2.20711415250284e-10\\
-10.206142578125	-1.8387079595673e-10\\
-10.1859931640625	-1.54363479830808e-10\\
-10.16584375	-1.17401299414987e-10\\
-10.1456943359375	-1.55828327671514e-10\\
-10.125544921875	-1.9708464196356e-10\\
-10.1053955078125	-2.94007488223453e-10\\
-10.08524609375	-2.54421949400691e-10\\
-10.0650966796875	-1.1104084685014e-10\\
-10.044947265625	-1.80089870489337e-10\\
-10.0247978515625	-1.24707743722275e-10\\
-10.0046484375	-9.83437596177126e-11\\
-9.98449902343749	-8.22794291446418e-11\\
-9.964349609375	-9.86001434252439e-11\\
-9.9442001953125	-8.48896027064034e-11\\
-9.92405078125	-3.48294719168156e-11\\
-9.90390136718749	-3.9976372071443e-11\\
-9.88375195312499	1.12154841952906e-10\\
-9.8636025390625	9.99732103743118e-11\\
-9.843453125	8.93147181480957e-11\\
-9.82330371093749	3.4772254926957e-11\\
-9.80315429687499	4.98138596189295e-11\\
-9.7830048828125	6.72501433794137e-11\\
-9.76285546875	7.5795754109002e-12\\
-9.7427060546875	2.18128728388151e-12\\
-9.72255664062499	4.80736111083313e-11\\
-9.70240722656249	3.02669583166979e-11\\
-9.6822578125	1.19704589961742e-10\\
-9.6621083984375	1.8119062759371e-10\\
-9.64195898437499	3.14130262601067e-10\\
-9.62180957031249	2.44436871551798e-10\\
-9.60166015625	3.11141655427589e-10\\
-9.5815107421875	3.26672760410382e-10\\
-9.561361328125	1.35807988931693e-10\\
-9.54121191406249	3.14512878050203e-10\\
-9.52106249999999	3.95884529323539e-10\\
-9.5009130859375	3.60031833799195e-10\\
-9.480763671875	4.78447172970301e-10\\
-9.46061425781249	4.26898780789514e-10\\
-9.44046484374999	4.04391923586848e-10\\
-9.4203154296875	3.27134026292767e-10\\
-9.400166015625	2.64589550530608e-10\\
-9.3800166015625	2.9113572684085e-10\\
-9.35986718749999	2.44008148990884e-10\\
-9.33971777343749	3.0764242494189e-10\\
-9.319568359375	2.86951074649468e-10\\
-9.2994189453125	3.56356170941486e-10\\
-9.27926953124999	5.24673833927329e-10\\
-9.25912011718749	5.89439797820934e-10\\
-9.238970703125	6.41010741294612e-10\\
-9.2188212890625	6.67546232528137e-10\\
-9.198671875	6.92944623760749e-10\\
-9.17852246093749	7.26524672915758e-10\\
-9.158373046875	7.79912543982399e-10\\
-9.1382236328125	7.0294537513355e-10\\
-9.11807421875	7.42607625343884e-10\\
-9.09792480468749	6.88707943881208e-10\\
-9.07777539062499	8.5473423626485e-10\\
-9.0576259765625	9.25412980126355e-10\\
-9.0374765625	9.99391227257909e-10\\
-9.0173271484375	9.62790064711417e-10\\
-8.99717773437499	9.48738514538543e-10\\
-8.9770283203125	9.50490344165498e-10\\
-8.95687890625	8.94183487875994e-10\\
-8.9367294921875	8.99145484738123e-10\\
-8.91658007812499	9.66340667772964e-10\\
-8.89643066406249	8.59091653743201e-10\\
-8.87628125	8.5520320771012e-10\\
-8.8561318359375	8.72512573046723e-10\\
-8.83598242187499	9.00489748047973e-10\\
-8.81583300781249	8.72182891826032e-10\\
-8.79568359375	9.34787449080135e-10\\
-8.7755341796875	9.46183012455672e-10\\
-8.755384765625	9.01792649444217e-10\\
-8.73523535156249	1.07799213298324e-09\\
-8.71508593749999	1.04916037136524e-09\\
-8.6949365234375	1.05692371745196e-09\\
-8.674787109375	1.10575334717128e-09\\
-8.65463769531249	9.85790139754341e-10\\
-8.63448828124999	1.02013171371162e-09\\
-8.6143388671875	9.73006834816585e-10\\
-8.594189453125	9.33696350297527e-10\\
-8.5740400390625	9.61301835469911e-10\\
-8.55389062499999	9.21787674980536e-10\\
-8.53374121093749	8.2188512614158e-10\\
-8.513591796875	8.93824818476319e-10\\
-8.4934423828125	8.50273773727962e-10\\
-8.47329296874999	8.35943025959801e-10\\
-8.45314355468749	7.68095993065882e-10\\
-8.432994140625	7.69697355691217e-10\\
-8.4128447265625	8.19713366300925e-10\\
-8.3926953125	8.32158599443722e-10\\
-8.37254589843749	8.18371269066642e-10\\
-8.35239648437499	8.07665126871417e-10\\
-8.3322470703125	7.64280415101491e-10\\
-8.31209765625	7.96451992999045e-10\\
-8.29194824218749	8.39286284795262e-10\\
-8.27179882812499	9.66617023041718e-10\\
-8.2516494140625	1.0332041638171e-09\\
-8.2315	1.01238908185286e-09\\
-8.2113505859375	8.50682056020502e-10\\
-8.19120117187499	9.32159240993766e-10\\
-8.17105175781249	8.76441736725922e-10\\
-8.15090234375	8.66912920347509e-10\\
-8.1307529296875	8.80940617860734e-10\\
-8.11060351562499	8.17692094464867e-10\\
-8.09045410156249	8.52795121325842e-10\\
-8.0703046875	7.44898262465349e-10\\
-8.0501552734375	7.68571131711658e-10\\
-8.030005859375	8.61494346620343e-10\\
-8.00985644531249	7.60171676582582e-10\\
-7.98970703125	8.10981940150797e-10\\
-7.9695576171875	8.03684089956262e-10\\
-7.949408203125	6.23989875413703e-10\\
-7.92925878906249	6.38682040774337e-10\\
-7.90910937499999	6.6442672190532e-10\\
-7.8889599609375	6.41340331093249e-10\\
-7.868810546875	5.94448688776608e-10\\
-7.8486611328125	5.37449463270686e-10\\
-7.82851171874999	6.01962469990305e-10\\
-7.8083623046875	6.30813729078789e-10\\
-7.788212890625	5.80688853292416e-10\\
-7.7680634765625	5.28015651441822e-10\\
-7.74791406249999	4.85956990118226e-10\\
-7.72776464843749	5.32555095327966e-10\\
-7.707615234375	4.45198250162567e-10\\
-7.6874658203125	5.39974684158549e-10\\
-7.66731640625	5.43249771326547e-10\\
-7.64716699218749	4.22027806392934e-10\\
-7.627017578125	4.07399769932351e-10\\
-7.6068681640625	3.67186532016555e-10\\
-7.58671875	3.39556444040001e-10\\
-7.56656933593749	1.65390069804197e-10\\
-7.54641992187499	9.66913845426165e-11\\
-7.5262705078125	2.0226621838681e-10\\
-7.50612109375	1.00138428586771e-10\\
-7.48597167968749	2.04078277775314e-10\\
-7.46582226562499	1.40910288383059e-10\\
-7.4456728515625	2.58678616974195e-10\\
-7.4255234375	2.02383190138728e-10\\
-7.4053740234375	3.12016764347642e-10\\
-7.38522460937499	5.46553150754694e-11\\
-7.36507519531249	5.93141893923854e-11\\
-7.34492578125	-1.14658016482016e-11\\
-7.3247763671875	-7.0818419797189e-11\\
-7.30462695312499	-6.17395772591515e-11\\
-7.28447753906249	-9.97113756789089e-12\\
-7.264328125	-4.35745870574009e-11\\
-7.2441787109375	9.57632943592492e-11\\
-7.224029296875	-6.53874307979932e-11\\
-7.20387988281249	-2.67445338359765e-12\\
-7.18373046874999	-2.40949570328146e-11\\
-7.1635810546875	-2.19453638599421e-10\\
-7.143431640625	-2.52343696589552e-10\\
-7.12328222656249	-3.32011396325573e-10\\
-7.10313281249999	-2.53729912089102e-10\\
-7.0829833984375	-3.03287511443893e-10\\
-7.062833984375	-1.48205572263451e-10\\
-7.0426845703125	-4.2872457391014e-11\\
-7.02253515624999	-1.07245988622148e-10\\
-7.00238574218749	-5.2799748046187e-11\\
-6.982236328125	-2.66527568662095e-11\\
-6.9620869140625	-1.08275045929261e-10\\
-6.94193749999999	-1.45096684161521e-10\\
-6.92178808593749	-2.85842839288645e-10\\
-6.901638671875	-3.10777147808849e-10\\
-6.8814892578125	-3.73665964342089e-10\\
-6.86133984375	-3.38059475207244e-10\\
-6.84119042968749	-3.58954900986898e-10\\
-6.821041015625	-2.07613148717395e-10\\
-6.8008916015625	-3.10528012810539e-10\\
-6.7807421875	-1.02014638972414e-10\\
-6.76059277343749	-1.87971795539786e-10\\
-6.74044335937499	-5.09709702533492e-11\\
-6.7202939453125	-1.7155735133313e-10\\
-6.70014453125	-2.30195527880699e-10\\
-6.6799951171875	-3.57417598596642e-10\\
-6.65984570312499	-5.39092032790802e-10\\
-6.6396962890625	-6.33234548909977e-10\\
-6.619546875	-7.70273569611764e-10\\
-6.5993974609375	-6.69468323293479e-10\\
-6.57924804687499	-7.94153053864854e-10\\
-6.55909863281249	-8.88209980283928e-10\\
-6.53894921875	-9.28794949859935e-10\\
-6.5187998046875	-8.18913679174485e-10\\
-6.498650390625	-8.06538071442788e-10\\
-6.47850097656249	-7.89689837301307e-10\\
-6.4583515625	-8.18104174413747e-10\\
-6.4382021484375	-7.72713360498415e-10\\
-6.418052734375	-8.44015521697987e-10\\
-6.39790332031249	-8.41862644241417e-10\\
-6.37775390624999	-9.33250712729678e-10\\
-6.3576044921875	-9.65814642541236e-10\\
-6.337455078125	-9.80901542883246e-10\\
-6.31730566406249	-9.87009539302234e-10\\
-6.29715624999999	-1.00658173536148e-09\\
-6.2770068359375	-1.05349604370716e-09\\
-6.256857421875	-9.83564719788021e-10\\
-6.2367080078125	-9.5977591249665e-10\\
-6.21655859374999	-1.03661248625405e-09\\
-6.19640917968749	-1.03928458728135e-09\\
-6.176259765625	-1.07926060219288e-09\\
-6.1561103515625	-1.14930294117664e-09\\
-6.13596093749999	-1.14552272257672e-09\\
-6.11581152343749	-1.02735562114098e-09\\
-6.095662109375	-9.98953452612487e-10\\
-6.0755126953125	-9.33362324806767e-10\\
-6.05536328125	-8.69610413041136e-10\\
-6.03521386718749	-8.07156796492285e-10\\
-6.01506445312499	-7.95767729334196e-10\\
-5.9949150390625	-8.99391180437681e-10\\
-5.974765625	-9.35023561771882e-10\\
-5.95461621093749	-9.202273862261e-10\\
-5.93446679687499	-9.20629667988135e-10\\
-5.9143173828125	-1.02041525958454e-09\\
-5.89416796875	-9.32470228464872e-10\\
-5.8740185546875	-1.02794371345429e-09\\
-5.85386914062499	-1.01611070769491e-09\\
-5.83371972656249	-8.52069436519839e-10\\
-5.8135703125	-9.10431907761471e-10\\
-5.7934208984375	-8.61828014812652e-10\\
-5.77327148437499	-9.19108673046319e-10\\
-5.75312207031249	-9.30935606416427e-10\\
-5.73297265625	-9.79851623714549e-10\\
-5.7128232421875	-9.66091013377635e-10\\
-5.692673828125	-9.93235257471209e-10\\
-5.67252441406249	-1.09469272819813e-09\\
-5.652375	-8.8348862523174e-10\\
-5.6322255859375	-9.6491721157765e-10\\
-5.612076171875	-9.12284685215441e-10\\
-5.59192675781249	-7.30566246177052e-10\\
-5.57177734374999	-7.32321634667777e-10\\
-5.5516279296875	-8.03533682034421e-10\\
-5.531478515625	-7.36703804548195e-10\\
-5.5113291015625	-8.46093175712411e-10\\
-5.49117968749999	-8.79921949029147e-10\\
-5.4710302734375	-9.53554791534036e-10\\
-5.450880859375	-9.18058913258659e-10\\
-5.4307314453125	-7.31541011025866e-10\\
-5.41058203124999	-7.28041487753499e-10\\
-5.39043261718749	-6.53664923094349e-10\\
-5.370283203125	-4.57193489115792e-10\\
-5.3501337890625	-4.51114373599329e-10\\
-5.329984375	-4.42635873324765e-10\\
-5.30983496093749	-5.03100060965839e-10\\
-5.289685546875	-4.60802319721603e-10\\
-5.2695361328125	-5.98285660029964e-10\\
-5.24938671875	-6.75744283859183e-10\\
-5.22923730468749	-6.76051448097738e-10\\
-5.20908789062499	-6.98641412217464e-10\\
-5.1889384765625	-5.66894085960953e-10\\
-5.1687890625	-5.20156518102605e-10\\
-5.1486396484375	-4.10167863917266e-10\\
-5.12849023437499	-3.07734616081671e-10\\
-5.1083408203125	-3.35830948558105e-10\\
-5.08819140625	-1.97327394352936e-10\\
-5.0680419921875	-2.25754653106421e-10\\
-5.04789257812499	-1.96560484621126e-10\\
-5.02774316406249	-2.34729737223383e-10\\
-5.00759375	-2.92898234143119e-10\\
-4.9874443359375	-2.78898894605213e-10\\
-4.96729492187499	-2.35836236168028e-10\\
-4.94714550781249	-3.54165844196569e-10\\
-4.92699609375	-1.86765668311038e-10\\
-4.9068466796875	-1.75545620541577e-10\\
-4.886697265625	1.35931478035213e-10\\
-4.86654785156249	1.65443042679606e-10\\
-4.84639843749999	2.21888697400807e-10\\
-4.8262490234375	1.28730063025832e-10\\
-4.806099609375	1.81860769373058e-10\\
-4.78595019531249	1.54283124940878e-11\\
-4.76580078124999	-5.3634662049444e-11\\
-4.7456513671875	-1.1907433496072e-10\\
-4.725501953125	-1.27980271081094e-10\\
-4.7053525390625	-1.57670165265951e-11\\
-4.68520312499999	6.114272918188e-11\\
-4.66505371093749	2.23802469597627e-10\\
-4.644904296875	4.40569002781853e-10\\
-4.6247548828125	5.46962052298115e-10\\
-4.60460546874999	5.51303046137914e-10\\
-4.58445605468749	5.14995971490003e-10\\
-4.564306640625	4.04796841879693e-10\\
-4.5441572265625	2.52786988631869e-10\\
-4.5240078125	2.271845379308e-10\\
-4.50385839843749	2.75661130768756e-10\\
-4.483708984375	2.76624010504251e-10\\
-4.4635595703125	4.12863016048456e-10\\
-4.44341015625	5.99137408857439e-10\\
-4.42326074218749	6.98327778995142e-10\\
-4.40311132812499	8.66700050205352e-10\\
-4.3829619140625	7.17600150476285e-10\\
-4.3628125	6.02828820758954e-10\\
-4.3426630859375	4.40000244761401e-10\\
-4.32251367187499	4.2622328353419e-10\\
-4.3023642578125	3.67575402831319e-10\\
-4.28221484375	3.57562321979995e-10\\
-4.2620654296875	5.60681926621042e-10\\
-4.24191601562499	6.20666828597697e-10\\
-4.22176660156249	7.26939527712664e-10\\
-4.2016171875	1.03972454625854e-09\\
-4.1814677734375	1.23608953961149e-09\\
-4.161318359375	1.328878296411e-09\\
-4.14116894531249	1.30740013843167e-09\\
-4.12101953125	1.29026457875691e-09\\
-4.1008701171875	1.27961821990359e-09\\
-4.080720703125	1.26979164996534e-09\\
-4.06057128906249	1.14389656312128e-09\\
-4.04042187499999	1.2732893234271e-09\\
-4.0202724609375	1.20875500138227e-09\\
-4.000123046875	1.23081434502829e-09\\
-3.9799736328125	1.33709059834412e-09\\
-3.95982421874999	1.42624382377303e-09\\
-3.9396748046875	1.44729368743038e-09\\
-3.919525390625	1.51471827233407e-09\\
-3.8993759765625	1.59148500644956e-09\\
-3.87922656249999	1.51830240249494e-09\\
-3.85907714843749	1.61413663143843e-09\\
-3.838927734375	1.54408433174977e-09\\
-3.8187783203125	1.47893925106714e-09\\
-3.79862890624999	1.54114633653067e-09\\
-3.77847949218749	1.41468347126905e-09\\
-3.758330078125	1.5524317944053e-09\\
-3.7381806640625	1.58051263541375e-09\\
-3.71803125	1.70423262108342e-09\\
-3.69788183593749	1.77850840207081e-09\\
-3.67773242187499	1.76822308069628e-09\\
-3.6575830078125	1.82228790203124e-09\\
-3.63743359375	1.85063034392027e-09\\
-3.61728417968749	1.68003316861579e-09\\
-3.59713476562499	1.5885487146884e-09\\
-3.5769853515625	1.52792377764072e-09\\
-3.5568359375	1.43684411980211e-09\\
-3.5366865234375	1.50419704164605e-09\\
-3.51653710937499	1.55353302392513e-09\\
-3.49638769531249	1.56218471614333e-09\\
-3.47623828125	1.63766021317948e-09\\
-3.4560888671875	1.54470250242169e-09\\
-3.43593945312499	1.57135050186338e-09\\
-3.41579003906249	1.62114215439037e-09\\
-3.395640625	1.45750987414169e-09\\
-3.3754912109375	1.49810721721607e-09\\
-3.355341796875	1.59155452971967e-09\\
-3.33519238281249	1.43906941512399e-09\\
-3.31504296874999	1.61224534827414e-09\\
-3.2948935546875	1.67037592638933e-09\\
-3.274744140625	1.64075073850775e-09\\
-3.25459472656249	1.68142093031504e-09\\
-3.23444531249999	1.75828557869223e-09\\
-3.2142958984375	1.59316843028999e-09\\
-3.194146484375	1.57055449998493e-09\\
-3.1739970703125	1.57037041273201e-09\\
-3.15384765624999	1.52234738955348e-09\\
-3.1336982421875	1.41383457793648e-09\\
-3.113548828125	1.71972773523917e-09\\
-3.0933994140625	1.62611053699799e-09\\
-3.07324999999999	1.78949558711203e-09\\
-3.05310058593749	1.82236537558251e-09\\
-3.032951171875	1.81183162016405e-09\\
-3.0128017578125	1.81571693114034e-09\\
-2.99265234375	1.81291644904206e-09\\
-2.97250292968749	1.7464750482019e-09\\
-2.952353515625	1.68695054652406e-09\\
-2.9322041015625	1.4855404830249e-09\\
-2.9120546875	1.45602578140766e-09\\
-2.89190527343749	1.44247923131897e-09\\
-2.87175585937499	1.46489156158878e-09\\
-2.8516064453125	1.67373949071883e-09\\
-2.83145703125	1.78346534989267e-09\\
-2.8113076171875	1.87775944105148e-09\\
-2.79115820312499	1.83290704301935e-09\\
-2.7710087890625	1.8813022540238e-09\\
-2.750859375	1.93629546284617e-09\\
-2.7307099609375	1.73274449824517e-09\\
-2.71056054687499	1.61397523078855e-09\\
-2.69041113281249	1.57839366939063e-09\\
-2.67026171875	1.38288579503195e-09\\
-2.6501123046875	1.56864609334645e-09\\
-2.629962890625	1.4470917088442e-09\\
-2.60981347656249	1.56366817551656e-09\\
-2.5896640625	1.54552420609669e-09\\
-2.5695146484375	1.46060356254422e-09\\
-2.549365234375	1.51282419008678e-09\\
-2.52921582031249	1.4851843749604e-09\\
-2.50906640624999	1.35003705738024e-09\\
-2.4889169921875	1.33312357443657e-09\\
-2.468767578125	1.19726058470768e-09\\
-2.44861816406249	1.2829841085597e-09\\
-2.42846874999999	1.21707336524295e-09\\
-2.4083193359375	1.13032355913226e-09\\
-2.388169921875	1.25092397099253e-09\\
-2.3680205078125	1.07182424939499e-09\\
-2.34787109374999	1.06777838860879e-09\\
-2.32772167968749	1.03359970195164e-09\\
-2.307572265625	1.05177955005612e-09\\
-2.2874228515625	9.72698746416079e-10\\
-2.26727343749999	1.08700518223507e-09\\
-2.24712402343749	1.11387526784685e-09\\
-2.226974609375	1.06945585861821e-09\\
-2.2068251953125	1.07374214247232e-09\\
-2.18667578125	1.09783833059703e-09\\
-2.16652636718749	1.02100285795678e-09\\
-2.14637695312499	8.17671932793303e-10\\
-2.1262275390625	7.66931071545141e-10\\
-2.106078125	7.02825323643878e-10\\
-2.08592871093749	6.28282879658703e-10\\
-2.06577929687499	6.44999756216189e-10\\
-2.0456298828125	7.00479775432975e-10\\
-2.02548046875	7.82022194246294e-10\\
-2.0053310546875	7.11127624355587e-10\\
-1.98518164062499	8.20492352962122e-10\\
-1.9650322265625	7.59065102785725e-10\\
-1.9448828125	6.62934978226197e-10\\
-1.9247333984375	6.08328279535488e-10\\
-1.90458398437499	4.60640202297988e-10\\
-1.88443457031249	5.51927730945958e-10\\
-1.86428515625	4.81661869109162e-10\\
-1.8441357421875	6.31316513640664e-10\\
-1.823986328125	6.61902933734606e-10\\
-1.80383691406249	5.27204197140194e-10\\
-1.7836875	5.38549164339305e-10\\
-1.7635380859375	4.63074509785951e-10\\
-1.743388671875	3.16471436617507e-10\\
-1.72323925781249	4.18405612591521e-10\\
-1.70308984374999	3.7220501338156e-10\\
-1.6829404296875	2.19701453894569e-10\\
-1.662791015625	1.55585238797822e-10\\
-1.6426416015625	2.10002031617482e-10\\
-1.62249218749999	4.02200784346034e-11\\
-1.6023427734375	1.29877178973864e-11\\
-1.582193359375	-4.50423971175513e-11\\
-1.5620439453125	-1.18668600549861e-11\\
-1.54189453124999	-2.35014100399208e-10\\
-1.52174511718749	-1.16027063746185e-10\\
-1.501595703125	-2.03637499085472e-10\\
-1.4814462890625	-2.66592741707504e-10\\
-1.461296875	-3.12885404367769e-10\\
-1.44114746093749	-3.86691880622314e-10\\
-1.420998046875	-2.77959922416677e-10\\
-1.4008486328125	-2.95281930950679e-10\\
-1.38069921875	-2.46897311720711e-10\\
-1.36054980468749	-8.79710286885871e-11\\
-1.34040039062499	-1.58702879541427e-10\\
-1.3202509765625	-6.6125055934191e-11\\
-1.3001015625	-1.99360889914488e-10\\
-1.27995214843749	-2.59707375299919e-10\\
-1.25980273437499	-2.3361716577045e-10\\
-1.2396533203125	-4.2230132702971e-10\\
-1.21950390625	-3.9146964212669e-10\\
-1.1993544921875	-5.44963092013456e-10\\
-1.17920507812499	-4.75368069967095e-10\\
-1.15905566406249	-5.06185875182135e-10\\
-1.13890625	-4.58068162246615e-10\\
-1.1187568359375	-3.64357585475997e-10\\
-1.09860742187499	-3.24177603318743e-10\\
-1.07845800781249	-2.94982942648331e-10\\
-1.05830859375	-1.91442013437085e-10\\
-1.0381591796875	-3.90041367840596e-10\\
-1.018009765625	-3.05936056247912e-10\\
-0.997860351562494	-3.39611449022176e-10\\
-0.977710937499992	-2.74632386227888e-10\\
-0.957561523437498	-1.65676425417933e-10\\
-0.937412109374996	1.40709776105871e-11\\
-0.917262695312495	4.26867958526909e-11\\
-0.897113281249993	6.35173124431793e-11\\
-0.876963867187499	-2.14121641614234e-11\\
-0.856814453124997	-1.2547026063333e-10\\
-0.836665039062495	-6.75405851887087e-11\\
-0.816515624999994	-2.13482602205779e-10\\
-0.796366210937499	-2.8682087150117e-10\\
-0.776216796874998	-2.00807413111698e-10\\
-0.756067382812496	-1.95723708154093e-10\\
-0.735917968749995	-1.23664198947574e-10\\
-0.715768554687493	-2.96728631934473e-11\\
-0.695619140624999	-5.79796787402065e-11\\
-0.675469726562497	-4.40834248087479e-11\\
-0.655320312499995	-9.57476415024677e-11\\
-0.635170898437494	-1.10909215165506e-10\\
-0.615021484374999	-1.85827548542168e-11\\
-0.594872070312498	-1.92756381686429e-10\\
-0.574722656249996	-1.49521326060482e-10\\
-0.554573242187494	-2.57009285149451e-10\\
-0.534423828124993	-1.63408777941241e-10\\
-0.514274414062498	-1.45899689781626e-10\\
-0.494124999999997	1.33605189844405e-11\\
-0.473975585937495	-1.04396069185672e-11\\
-0.453826171874994	1.19498276235254e-10\\
-0.433676757812499	9.58835898322727e-11\\
-0.413527343749998	2.33313328472888e-10\\
-0.393377929687496	2.18909157471499e-10\\
-0.373228515624994	1.90923654735109e-10\\
-0.353079101562493	2.65182604196714e-10\\
-0.332929687499998	1.95544022084891e-10\\
-0.312780273437497	2.28604276752021e-10\\
-0.292630859374995	2.06274687357738e-10\\
-0.272481445312494	3.42814281534392e-10\\
-0.252332031249999	3.28913277617362e-10\\
-0.232182617187497	5.46653305584173e-10\\
-0.212033203124996	5.39122683543538e-10\\
-0.191883789062494	6.42633288094734e-10\\
-0.171734374999993	7.53082457081632e-10\\
-0.151584960937498	7.23918006696213e-10\\
-0.131435546874997	8.5017120936188e-10\\
-0.111286132812495	1.06226332591699e-09\\
-0.0911367187499934	9.86221648408654e-10\\
-0.0709873046874989	1.22848854082204e-09\\
-0.0508378906249973	1.33148870881326e-09\\
-0.0306884765624957	1.35578036310216e-09\\
-0.0105390624999941	1.54200385179823e-09\\
0.00961035156250745	1.56000051830543e-09\\
0.0297597656250019	1.67414723639338e-09\\
0.0499091796875035	1.5511281723354e-09\\
0.0700585937500051	1.68050034329208e-09\\
0.0902080078125067	1.75029500455436e-09\\
0.110357421875001	1.78166306985253e-09\\
0.130506835937503	1.83101932351234e-09\\
0.150656250000004	2.21110549084786e-09\\
0.170805664062506	2.156661231232e-09\\
0.190955078125008	2.38322343905032e-09\\
0.211104492187502	2.47446561797509e-09\\
0.231253906250004	2.49151900846188e-09\\
0.251403320312505	2.39241982808285e-09\\
0.271552734375007	2.32754186273854e-09\\
0.291702148437501	2.22295465129632e-09\\
0.311851562500003	2.27375197047017e-09\\
0.332000976562504	2.34687196359386e-09\\
0.352150390625006	2.59776894575209e-09\\
0.372299804687501	2.8623952705491e-09\\
0.392449218750002	3.01654458566355e-09\\
0.412598632812504	3.30820267637101e-09\\
0.432748046875005	3.4110339109575e-09\\
0.452897460937507	3.550783531058e-09\\
0.473046875000001	3.4384925366409e-09\\
0.493196289062503	3.31202511356265e-09\\
0.513345703125005	3.3626169355237e-09\\
0.533495117187506	3.21634457469051e-09\\
0.553644531250001	3.26778773503061e-09\\
0.573793945312502	3.32754517489435e-09\\
0.593943359375004	3.57225976661555e-09\\
0.614092773437505	3.9021058685081e-09\\
0.634242187500007	4.19487003635436e-09\\
0.654391601562502	4.4364699326425e-09\\
0.674541015625003	4.55813628305397e-09\\
0.694690429687505	4.59288668937402e-09\\
0.714839843750006	4.43491670235142e-09\\
0.734989257812501	4.27546855979165e-09\\
0.755138671875002	4.29960249212859e-09\\
0.775288085937504	4.2344897424106e-09\\
0.795437500000006	4.12204366433163e-09\\
0.815586914062507	4.44013765178621e-09\\
0.835736328125002	4.67621826350142e-09\\
0.855885742187503	5.01149620815272e-09\\
0.876035156250005	5.40340097007332e-09\\
0.896184570312506	5.78765891605383e-09\\
0.916333984375001	5.98720065906477e-09\\
0.936483398437503	5.97717188788629e-09\\
0.956632812500004	6.02611962893976e-09\\
0.976782226562506	5.90412250337475e-09\\
0.996931640625007	5.81539108618653e-09\\
1.0170810546875	5.70826820436324e-09\\
1.03723046875	5.79808743080016e-09\\
1.0573798828125	5.85397810357418e-09\\
1.07752929687501	5.93021294591822e-09\\
1.0976787109375	6.33322361433322e-09\\
1.117828125	6.58215360491828e-09\\
1.1379775390625	6.77870253135203e-09\\
1.15812695312501	6.99292022550427e-09\\
1.17827636718751	7.18548709802176e-09\\
1.19842578125	7.04413049100717e-09\\
1.2185751953125	7.11761946806205e-09\\
1.23872460937501	7.0906113459861e-09\\
1.25887402343751	6.98609269952518e-09\\
1.2790234375	7.16189796142393e-09\\
1.2991728515625	7.20019843746348e-09\\
1.319322265625	7.37848194705596e-09\\
1.33947167968751	7.71776231077976e-09\\
1.35962109375001	7.89350238550395e-09\\
1.3797705078125	8.04002306110861e-09\\
1.399919921875	8.23249897550073e-09\\
1.42006933593751	8.27233377893329e-09\\
1.44021875000001	8.15536828420209e-09\\
1.4603681640625	8.19401064681429e-09\\
1.480517578125	8.21505643994722e-09\\
1.5006669921875	8.11225307028831e-09\\
1.52081640625001	8.21742183456577e-09\\
1.5409658203125	8.32607351291082e-09\\
1.561115234375	8.4637909244545e-09\\
1.5812646484375	8.61635409053648e-09\\
1.60141406250001	8.83903619942022e-09\\
1.62156347656251	8.93796043454151e-09\\
1.641712890625	8.99211908932398e-09\\
1.6618623046875	9.03117405008076e-09\\
1.68201171875	9.0419327026283e-09\\
1.70216113281251	8.99568123960409e-09\\
1.722310546875	9.07180351179288e-09\\
1.7424599609375	8.9953929862216e-09\\
1.762609375	9.11596368034871e-09\\
1.78275878906251	9.22494552889802e-09\\
1.80290820312501	9.31075737360143e-09\\
1.8230576171875	9.63405401855023e-09\\
1.84320703125	9.78933680893489e-09\\
1.8633564453125	9.94749093034798e-09\\
1.88350585937501	1.00014200092094e-08\\
1.9036552734375	1.00414987997147e-08\\
1.9238046875	1.00430214562393e-08\\
1.9439541015625	1.01720458264704e-08\\
1.96410351562501	1.01379221525682e-08\\
1.98425292968751	1.03074924941822e-08\\
2.00440234375	1.03063781965528e-08\\
2.0245517578125	1.04578388932108e-08\\
2.044701171875	1.05616559859584e-08\\
2.06485058593751	1.06943987191332e-08\\
2.085	1.0848599343748e-08\\
2.1051494140625	1.07385378359329e-08\\
2.125298828125	1.07415744994033e-08\\
2.14544824218751	1.06088042474197e-08\\
2.16559765625001	1.05091982590077e-08\\
2.1857470703125	1.04475246210945e-08\\
2.205896484375	1.04063699834371e-08\\
2.2260458984375	1.03804127946742e-08\\
2.24619531250001	1.04458100920759e-08\\
2.2663447265625	1.0479151128519e-08\\
2.286494140625	1.05521304744716e-08\\
2.3066435546875	1.07177351352969e-08\\
2.32679296875001	1.05298331946567e-08\\
2.34694238281251	1.0481958137593e-08\\
2.367091796875	1.01266293232368e-08\\
2.3872412109375	1.00282005771348e-08\\
2.40739062500001	9.78453545396906e-09\\
2.42754003906251	9.66285800335848e-09\\
2.447689453125	9.46708420666629e-09\\
2.4678388671875	9.4401057090082e-09\\
2.48798828125	9.43398499027808e-09\\
2.50813769531251	9.46445175865456e-09\\
2.52828710937501	9.52560307862594e-09\\
2.5484365234375	9.5024299457279e-09\\
2.5685859375	9.44840380774678e-09\\
2.58873535156251	8.97049997176723e-09\\
2.60888476562501	8.8961718287187e-09\\
2.6290341796875	8.46834272284425e-09\\
2.64918359375	8.27657669940332e-09\\
2.6693330078125	7.90925336851795e-09\\
2.68948242187501	7.82876451807146e-09\\
2.7096318359375	7.81218299279192e-09\\
2.72978125	7.76360971775548e-09\\
2.7499306640625	8.02846882774384e-09\\
2.77008007812501	8.36496708300924e-09\\
2.79022949218751	8.32402415088705e-09\\
2.81037890625	8.47738711453094e-09\\
2.8305283203125	8.36492502119965e-09\\
2.850677734375	8.09693628124677e-09\\
2.87082714843751	7.84386659759845e-09\\
2.8909765625	7.6157756356447e-09\\
2.9111259765625	7.34607426037784e-09\\
2.931275390625	7.17778013357298e-09\\
2.95142480468751	7.15480294792186e-09\\
2.97157421875001	7.11401314819584e-09\\
2.9917236328125	7.44299455584962e-09\\
3.011873046875	7.44567855840661e-09\\
3.0320224609375	7.60801136034911e-09\\
3.05217187500001	7.70930674297667e-09\\
3.0723212890625	7.59474004877991e-09\\
3.092470703125	7.3872456768514e-09\\
3.1126201171875	7.20264509307946e-09\\
3.13276953125001	6.89087415186804e-09\\
3.15291894531251	6.67532458045803e-09\\
3.173068359375	6.37293530516409e-09\\
3.1932177734375	6.28690964180283e-09\\
3.2133671875	6.50896745541196e-09\\
3.23351660156251	6.46951707370189e-09\\
3.253666015625	6.63706108019447e-09\\
3.2738154296875	6.72440911434006e-09\\
3.29396484375	6.66021755060615e-09\\
3.31411425781251	6.63709272007297e-09\\
3.33426367187501	6.43640836907281e-09\\
3.3544130859375	6.29342609849076e-09\\
3.3745625	6.17715052314638e-09\\
3.3947119140625	5.90219290221853e-09\\
3.41486132812501	5.86484903288669e-09\\
3.4350107421875	5.92173563426419e-09\\
3.45516015625	5.78035992507519e-09\\
3.4753095703125	5.85767143517118e-09\\
3.49545898437501	5.78171558887907e-09\\
3.51560839843751	5.70798603035018e-09\\
3.5357578125	5.63355233847071e-09\\
3.5559072265625	5.5169886692835e-09\\
3.576056640625	5.28902281473891e-09\\
3.59620605468751	5.22373927758921e-09\\
3.61635546875	4.93932521205499e-09\\
3.6365048828125	4.96557056729371e-09\\
3.656654296875	4.90335681951696e-09\\
3.67680371093751	4.86006248197808e-09\\
3.69695312500001	4.64194156016894e-09\\
3.7171025390625	4.73078508249443e-09\\
3.737251953125	4.58188892835089e-09\\
3.75740136718751	4.43299780804553e-09\\
3.77755078125001	4.36292907928103e-09\\
3.7977001953125	4.20363792383944e-09\\
3.817849609375	4.02535446956928e-09\\
3.8379990234375	3.98959202257011e-09\\
3.85814843750001	4.01123983159533e-09\\
3.8782978515625	4.04518211991099e-09\\
3.898447265625	4.01274702109404e-09\\
3.9185966796875	4.01931919447776e-09\\
3.93874609375001	3.92919510604204e-09\\
3.95889550781251	3.84421649604763e-09\\
3.979044921875	3.77827392390878e-09\\
3.9991943359375	3.76853531120059e-09\\
4.01934375	3.69578839268768e-09\\
4.03949316406251	3.51775878447939e-09\\
4.059642578125	3.57510960402306e-09\\
4.0797919921875	3.49364780628298e-09\\
4.09994140625	3.33269513223524e-09\\
4.12009082031251	3.43192066807862e-09\\
4.14024023437501	3.28223257605143e-09\\
4.1603896484375	3.14368896775458e-09\\
4.1805390625	3.0174527780217e-09\\
4.2006884765625	3.02970780976985e-09\\
4.22083789062501	2.95050651233033e-09\\
4.2409873046875	2.73602776483383e-09\\
4.26113671875	2.83203581575082e-09\\
4.2812861328125	2.77412562946647e-09\\
4.30143554687501	2.6282046174569e-09\\
4.32158496093751	2.6484300929967e-09\\
4.341734375	2.7419124491441e-09\\
4.3618837890625	2.49493421537927e-09\\
4.382033203125	2.45121440983658e-09\\
4.40218261718751	2.41696522213943e-09\\
4.42233203125	2.25746299418944e-09\\
4.4424814453125	1.95877735166305e-09\\
4.462630859375	2.03295706348706e-09\\
4.48278027343751	1.81079709661706e-09\\
4.50292968750001	1.71561989842445e-09\\
4.5230791015625	1.62152906181042e-09\\
4.543228515625	1.64807475157757e-09\\
4.5633779296875	1.57334582944191e-09\\
4.58352734375001	1.58875333656605e-09\\
4.6036767578125	1.49462104748235e-09\\
4.623826171875	1.43094958017104e-09\\
4.6439755859375	1.19674989316264e-09\\
4.66412500000001	1.19348356410342e-09\\
4.68427441406251	1.08695457260779e-09\\
4.704423828125	1.0964593047153e-09\\
4.7245732421875	9.9866066899357e-10\\
4.74472265625	1.00645255405174e-09\\
4.76487207031251	1.01703245483283e-09\\
4.785021484375	1.07907903718498e-09\\
4.8051708984375	1.12621772566202e-09\\
4.8253203125	1.15465968929692e-09\\
4.84546972656251	1.20159777879335e-09\\
4.86561914062501	1.22554610362799e-09\\
4.8857685546875	1.14352815306665e-09\\
4.90591796875	1.03260409655679e-09\\
4.92606738281251	9.3098267767322e-10\\
4.94621679687501	8.66402505276789e-10\\
4.9663662109375	8.24505142021257e-10\\
4.986515625	7.66108240578084e-10\\
5.0066650390625	8.22428083874049e-10\\
5.02681445312501	8.00569319504638e-10\\
5.0469638671875	7.23650195968981e-10\\
5.06711328125	7.43616997342851e-10\\
5.0872626953125	7.16904634413937e-10\\
5.10741210937501	6.91350708816647e-10\\
5.12756152343751	5.75188314001121e-10\\
5.1477109375	5.60600889728445e-10\\
5.1678603515625	5.7637909788143e-10\\
5.188009765625	4.73629438404683e-10\\
5.20815917968751	4.46325516495453e-10\\
5.22830859375	3.36389487409343e-10\\
5.2484580078125	3.83772471806453e-10\\
5.268607421875	2.58938750676534e-10\\
5.28875683593751	2.91933754529544e-10\\
5.30890625000001	2.67590207939276e-10\\
5.3290556640625	1.10401098118512e-10\\
5.349205078125	2.05138353403815e-10\\
5.3693544921875	1.12544745013367e-10\\
5.38950390625001	1.54909819507509e-10\\
5.4096533203125	3.52703082673063e-11\\
5.429802734375	6.76868293922431e-11\\
5.4499521484375	-6.70510497838294e-11\\
5.47010156250001	-1.40652328743099e-10\\
5.49025097656251	-1.81992697021646e-10\\
5.510400390625	-2.38525752337551e-10\\
5.5305498046875	-3.50765827598333e-10\\
5.55069921875	-4.85085103953607e-10\\
5.57084863281251	-4.87852050232594e-10\\
5.590998046875	-5.18197667632187e-10\\
5.6111474609375	-5.06551331308841e-10\\
5.631296875	-5.61387461509877e-10\\
5.65144628906251	-5.19879316152007e-10\\
5.67159570312501	-5.08953657629934e-10\\
5.6917451171875	-6.64033997786495e-10\\
5.71189453125	-7.97657326687803e-10\\
5.7320439453125	-8.64568330420595e-10\\
5.75219335937501	-7.9824521976666e-10\\
5.7723427734375	-9.6623866379358e-10\\
5.7924921875	-9.41318358799248e-10\\
5.8126416015625	-1.03739466709185e-09\\
5.83279101562501	-8.92468666504786e-10\\
5.85294042968751	-8.2017053528151e-10\\
5.87308984375	-7.78970617161475e-10\\
5.8932392578125	-7.20309034488646e-10\\
5.913388671875	-7.32961012710654e-10\\
5.93353808593751	-7.80957305451123e-10\\
5.9536875	-7.02076687164033e-10\\
5.9738369140625	-8.78249071513595e-10\\
5.993986328125	-1.02067071508687e-09\\
6.01413574218751	-1.02004772146257e-09\\
6.03428515625001	-1.18401652199418e-09\\
6.0544345703125	-1.09253770704596e-09\\
6.074583984375	-1.08246583847477e-09\\
6.0947333984375	-1.08933288773148e-09\\
6.11488281250001	-1.00650473426698e-09\\
6.1350322265625	-9.69653317717401e-10\\
6.155181640625	-1.01920860586711e-09\\
6.1753310546875	-8.89170903833844e-10\\
6.19548046875001	-9.95242832222079e-10\\
6.2156298828125	-1.00465714045752e-09\\
6.235779296875	-1.12499062078219e-09\\
6.2559287109375	-1.02327694484113e-09\\
6.27607812500001	-1.13149633257791e-09\\
6.29622753906251	-1.11924333912403e-09\\
6.316376953125	-1.02095102485594e-09\\
6.3365263671875	-1.04821858824297e-09\\
6.35667578125	-1.06356923082205e-09\\
6.37682519531251	-9.72729972752365e-10\\
6.396974609375	-1.06434516833066e-09\\
6.4171240234375	-1.0074365822772e-09\\
6.4372734375	-1.21196492721549e-09\\
6.45742285156251	-1.17313038191276e-09\\
6.47757226562501	-1.16818054089085e-09\\
6.4977216796875	-1.21603887193848e-09\\
6.51787109375	-1.09861526617397e-09\\
6.5380205078125	-1.13546248125858e-09\\
6.55816992187501	-1.14444178346273e-09\\
6.5783193359375	-1.18741993364816e-09\\
6.59846875	-1.26151059446033e-09\\
6.6186181640625	-1.28174803623827e-09\\
6.63876757812501	-1.35908604863785e-09\\
6.65891699218751	-1.37817682037096e-09\\
6.67906640625	-1.42133237222744e-09\\
6.6992158203125	-1.38776145369706e-09\\
6.719365234375	-1.40766631311704e-09\\
6.73951464843751	-1.35467489125969e-09\\
6.7596640625	-1.2385520371547e-09\\
6.7798134765625	-1.28763161594668e-09\\
6.799962890625	-1.2112561927121e-09\\
6.82011230468751	-1.11174730821593e-09\\
6.84026171875001	-1.25639710155138e-09\\
6.8604111328125	-1.12540064797978e-09\\
6.880560546875	-1.1522282484205e-09\\
6.9007099609375	-1.29831446472327e-09\\
6.92085937500001	-1.30734899509832e-09\\
6.9410087890625	-1.1543268983683e-09\\
6.961158203125	-1.21450840467793e-09\\
6.9813076171875	-1.20695633284813e-09\\
7.00145703125001	-1.22690400089825e-09\\
7.02160644531251	-1.19000302030757e-09\\
7.041755859375	-9.87882344923473e-10\\
7.0619052734375	-1.04181568083581e-09\\
7.0820546875	-9.00303132524545e-10\\
7.10220410156251	-9.90605808179939e-10\\
7.122353515625	-1.02551117633645e-09\\
7.1425029296875	-1.05524265890962e-09\\
7.16265234375	-1.03310465022587e-09\\
7.18280175781251	-1.06604099673689e-09\\
7.20295117187501	-1.11512143416373e-09\\
7.2231005859375	-1.16607227785057e-09\\
7.24325	-1.04145677767455e-09\\
7.2633994140625	-9.7115031896659e-10\\
7.28354882812501	-8.79086675571935e-10\\
7.3036982421875	-8.75277441365796e-10\\
7.32384765625	-9.68778273934532e-10\\
7.3439970703125	-9.05960398299049e-10\\
7.36414648437501	-1.10495596331995e-09\\
7.3842958984375	-1.11188002752382e-09\\
7.4044453125	-1.18560459374991e-09\\
7.4245947265625	-1.09730706908409e-09\\
7.44474414062501	-1.24878932695219e-09\\
7.46489355468751	-1.19518254230086e-09\\
7.48504296875	-1.0892429508738e-09\\
7.5051923828125	-1.12157906044996e-09\\
7.525341796875	-1.04276679486158e-09\\
7.54549121093751	-1.08681863072568e-09\\
7.565640625	-1.10031519473315e-09\\
7.5857900390625	-1.10416154574144e-09\\
7.605939453125	-1.11704382262996e-09\\
7.62608886718751	-1.15019475110762e-09\\
7.64623828125001	-1.11930200621125e-09\\
7.6663876953125	-1.1680984386846e-09\\
7.686537109375	-1.03287618582683e-09\\
7.7066865234375	-1.09622013188494e-09\\
7.72683593750001	-9.375906160101e-10\\
7.7469853515625	-1.10396700906959e-09\\
7.767134765625	-9.87524280573618e-10\\
7.7872841796875	-1.07430175024708e-09\\
7.80743359375001	-9.98751559368171e-10\\
7.82758300781251	-1.12172510813596e-09\\
7.847732421875	-1.08943396008254e-09\\
7.8678818359375	-1.0968043703922e-09\\
7.88803125	-1.08734561341074e-09\\
7.90818066406251	-9.83885535340862e-10\\
7.928330078125	-1.01437586354188e-09\\
7.9484794921875	-9.85465154800382e-10\\
7.96862890625	-8.04276066721668e-10\\
7.98877832031251	-8.30040293417784e-10\\
8.00892773437501	-8.38269291468735e-10\\
8.0290771484375	-7.57840253696713e-10\\
8.0492265625	-7.79219676215784e-10\\
8.0693759765625	-8.35231203189068e-10\\
8.08952539062501	-8.04532769624994e-10\\
8.1096748046875	-7.3802558186643e-10\\
8.12982421875	-7.69393628742061e-10\\
8.1499736328125	-6.8287097547345e-10\\
8.17012304687501	-6.19364345342397e-10\\
8.19027246093751	-6.52137114957733e-10\\
8.210421875	-5.63785812597005e-10\\
8.2305712890625	-6.36177799317873e-10\\
8.250720703125	-6.30463821575563e-10\\
8.27087011718751	-5.83074657620475e-10\\
8.29101953125	-5.47876519414119e-10\\
8.3111689453125	-5.93442247102709e-10\\
8.331318359375	-4.97944272734412e-10\\
8.35146777343751	-5.99487484587365e-10\\
8.37161718750001	-5.47104417546601e-10\\
8.3917666015625	-5.79322244909808e-10\\
8.411916015625	-5.68703353180045e-10\\
8.4320654296875	-5.36974041599036e-10\\
8.45221484375001	-5.95333999830253e-10\\
8.4723642578125	-5.50061409816423e-10\\
8.492513671875	-5.05773247553592e-10\\
8.5126630859375	-4.07409474414524e-10\\
8.53281250000001	-4.67540214371947e-10\\
8.5529619140625	-3.25769498122597e-10\\
8.573111328125	-3.47706887138264e-10\\
8.5932607421875	-2.55733197678741e-10\\
8.61341015625	-2.81591952361076e-10\\
8.63355957031251	-2.05766565575329e-10\\
8.653708984375	-1.68524216258924e-10\\
8.6738583984375	-1.89764948560578e-10\\
8.6940078125	-2.01645917955787e-10\\
8.71415722656251	-1.47652073535791e-10\\
8.734306640625	-1.6766319193121e-10\\
8.7544560546875	-5.8251918969402e-11\\
8.77460546875	-2.18135032856518e-10\\
8.79475488281251	-6.40059204924393e-11\\
8.81490429687501	-1.03897991457977e-10\\
8.8350537109375	-2.23250052611152e-10\\
8.855203125	-1.36756122386354e-10\\
8.8753525390625	-1.31707003466414e-10\\
8.89550195312501	-1.17474820882244e-10\\
8.9156513671875	-8.59870147830064e-11\\
8.93580078125	-7.38731358574507e-13\\
8.9559501953125	1.49477245232236e-11\\
8.97609960937501	-2.17080152585615e-11\\
8.99624902343751	-3.74727014679845e-11\\
9.0163984375	-4.83956063130116e-11\\
9.0365478515625	-1.40115683478132e-10\\
9.056697265625	-2.20383848685721e-10\\
9.07684667968751	-3.66256697412479e-10\\
9.09699609375	-2.04585509448135e-10\\
9.1171455078125	-3.06277672356856e-10\\
9.137294921875	-1.72135019801541e-10\\
9.15744433593751	-1.48497970704281e-10\\
9.17759375000001	-2.3153463978671e-11\\
9.1977431640625	-1.25876142088276e-11\\
9.217892578125	4.0341136880463e-11\\
9.2380419921875	-5.97900280182659e-11\\
9.25819140625001	-7.40070392252687e-11\\
9.2783408203125	-1.11007840610866e-10\\
9.298490234375	-7.95445033474825e-11\\
9.3186396484375	-1.09713017175421e-10\\
9.33878906250001	-7.4647357107795e-11\\
9.35893847656251	5.55149719213577e-11\\
9.379087890625	-4.41296216735574e-11\\
9.3992373046875	-3.54652865838437e-11\\
9.41938671875	1.23222494625995e-10\\
9.43953613281251	4.34291132520653e-12\\
9.459685546875	-2.96999425800648e-11\\
9.4798349609375	3.7232553483532e-11\\
9.499984375	-4.23758377146125e-11\\
9.52013378906251	-3.07577923093189e-11\\
9.54028320312501	-3.62115688636809e-11\\
9.5604326171875	-1.24248670737011e-10\\
9.58058203125	-4.88258273513245e-11\\
9.6007314453125	-1.70413779777661e-11\\
9.62088085937501	7.55741983036035e-11\\
9.6410302734375	8.67848102033139e-11\\
9.6611796875	-5.50434804131447e-11\\
9.6813291015625	4.94917846337955e-11\\
9.70147851562501	-1.03797285007752e-10\\
9.7216279296875	-7.03419468816357e-11\\
9.74177734375	-1.24505004407596e-10\\
9.7619267578125	-2.32630332378154e-10\\
9.782076171875	-3.63829530098836e-10\\
9.80222558593751	-3.66271539823765e-10\\
9.822375	-3.70852769270093e-10\\
9.8425244140625	-1.72310477223712e-10\\
9.862673828125	-1.07899435225538e-10\\
9.88282324218751	-4.91311079254905e-11\\
9.90297265625	4.79392041391778e-11\\
9.9231220703125	6.32773906567707e-11\\
9.943271484375	2.43886112318637e-10\\
9.96342089843751	2.24445194163162e-10\\
9.98357031250001	1.64065012805467e-10\\
10.0037197265625	-3.3104161961917e-11\\
10.023869140625	4.13412280701553e-11\\
10.0440185546875	-1.26324182651422e-10\\
10.06416796875	-4.1845499964601e-11\\
10.0843173828125	-8.34514639390356e-11\\
10.104466796875	-1.02306370653239e-10\\
10.1246162109375	-1.2922463668323e-10\\
10.144765625	-1.82759472135552e-12\\
10.1649150390625	-2.9103684066064e-11\\
10.185064453125	4.53540594857951e-13\\
10.2052138671875	-9.60438670054248e-11\\
10.22536328125	-3.15517084089716e-11\\
10.2455126953125	-9.10203426950919e-11\\
10.265662109375	-5.94576826923778e-12\\
10.2858115234375	-3.75443245558971e-11\\
10.3059609375	-3.44946684374074e-11\\
10.3261103515625	-1.92168310159402e-11\\
10.346259765625	1.16611212621735e-10\\
10.3664091796875	6.18603519214318e-11\\
10.38655859375	1.12542446717473e-10\\
10.4067080078125	8.27616697779349e-11\\
10.426857421875	-6.08368506251656e-11\\
10.4470068359375	4.76315764416944e-12\\
10.46715625	-6.82949879944901e-11\\
10.4873056640625	-1.19253250220753e-10\\
10.507455078125	1.57794893726244e-11\\
10.5276044921875	-5.45149123164114e-12\\
10.54775390625	3.19125577922758e-11\\
10.5679033203125	8.69021247855355e-11\\
10.588052734375	6.38076059246e-11\\
10.6082021484375	8.42876279059793e-11\\
10.6283515625	5.39690076353179e-11\\
10.6485009765625	-2.24427514810294e-11\\
10.668650390625	-4.21863778263924e-11\\
10.6887998046875	-1.60860994862911e-10\\
10.70894921875	-1.26308183271076e-10\\
10.7290986328125	-1.47162453336992e-10\\
10.749248046875	-1.82604047967817e-10\\
10.7693974609375	-2.13567180685563e-10\\
10.789546875	-1.8564325526842e-10\\
10.8096962890625	-1.1078483296908e-10\\
10.829845703125	-7.26521057322517e-11\\
10.8499951171875	-1.91313436530159e-10\\
10.87014453125	-1.34497593183432e-10\\
10.8902939453125	-2.42211893944298e-10\\
10.910443359375	-2.3238831584232e-10\\
10.9305927734375	-3.3707564317252e-10\\
10.9507421875	-2.32207645703108e-10\\
10.9708916015625	-1.41170339693255e-10\\
10.991041015625	-2.1112441741327e-10\\
11.0111904296875	-8.19941414910284e-11\\
11.03133984375	9.49550790655514e-12\\
11.0514892578125	2.18804726251717e-11\\
11.071638671875	-5.58997702177591e-11\\
11.0917880859375	-2.7173960561967e-11\\
11.1119375	-2.87124411547578e-10\\
11.1320869140625	-3.23516385261365e-10\\
11.152236328125	-4.96068203184205e-10\\
11.1723857421875	-4.89789319342284e-10\\
11.19253515625	-5.06900193526091e-10\\
11.2126845703125	-5.17691108628892e-10\\
11.232833984375	-4.31235467528804e-10\\
11.2529833984375	-3.69902657571375e-10\\
11.2731328125	-2.78038938822226e-10\\
11.2932822265625	-1.44773845779976e-10\\
11.313431640625	-1.98979601978543e-10\\
11.3335810546875	-7.28444840200849e-11\\
11.35373046875	-1.77410761413662e-10\\
11.3738798828125	-1.61016971078352e-10\\
11.394029296875	-1.19746430081465e-10\\
11.4141787109375	-2.82683349033038e-10\\
11.434328125	-2.67040791722135e-10\\
11.4544775390625	-2.62675458695115e-10\\
11.474626953125	-1.66810841328507e-10\\
11.4947763671875	-1.04400750418024e-10\\
11.51492578125	-1.56921067498967e-10\\
11.5350751953125	-4.43465191147332e-12\\
11.555224609375	-3.4079302397716e-11\\
11.5753740234375	-1.88225786964982e-11\\
11.5955234375	-7.27213580081108e-11\\
11.6156728515625	-1.16034599393812e-10\\
11.635822265625	-2.42927153754246e-10\\
11.6559716796875	-2.44093725171666e-10\\
11.67612109375	-2.84258709977497e-10\\
11.6962705078125	-3.98483192823779e-10\\
11.716419921875	-3.26562042377816e-10\\
11.7365693359375	-3.6476081749316e-10\\
11.75671875	-3.48507331112475e-10\\
11.7768681640625	-3.05294365599199e-10\\
11.797017578125	-4.36161447038697e-10\\
11.8171669921875	-3.06692070075182e-10\\
11.83731640625	-3.82485188777108e-10\\
11.8574658203125	-4.37862291779621e-10\\
11.877615234375	-4.11917263507305e-10\\
11.8977646484375	-3.97757488782628e-10\\
11.9179140625	-3.55624127995473e-10\\
11.9380634765625	-4.39653128969375e-10\\
11.958212890625	-3.63132368583319e-10\\
11.9783623046875	-4.24795856460991e-10\\
11.99851171875	-4.9509654978805e-10\\
12.0186611328125	-3.7973632347296e-10\\
12.038810546875	-4.76541519893743e-10\\
12.0589599609375	-5.42100905428001e-10\\
12.079109375	-4.01886717135638e-10\\
12.0992587890625	-4.50139623516829e-10\\
12.119408203125	-3.37130383756413e-10\\
12.1395576171875	-2.57698072246021e-10\\
12.15970703125	-2.7739210494547e-10\\
12.1798564453125	-2.34256545858236e-10\\
12.200005859375	-1.9196087495968e-10\\
12.2201552734375	-2.61134090357343e-10\\
12.2403046875	-3.38878545998837e-10\\
12.2604541015625	-4.14571426212128e-10\\
12.280603515625	-3.95770133096383e-10\\
12.3007529296875	-3.97524100965402e-10\\
12.32090234375	-3.60114468822069e-10\\
12.3410517578125	-3.26268709864348e-10\\
12.361201171875	-3.11497284004418e-10\\
12.3813505859375	-2.95831357325157e-10\\
12.4015	-3.43740415971473e-10\\
12.4216494140625	-4.09082322280232e-10\\
12.441798828125	-4.54180239895885e-10\\
12.4619482421875	-4.08008036364044e-10\\
12.48209765625	-3.89400211688852e-10\\
12.5022470703125	-3.86969696984219e-10\\
12.522396484375	-3.62534490642142e-10\\
12.5425458984375	-3.14540187587971e-10\\
12.5626953125	-3.47622716266327e-10\\
12.5828447265625	-4.04426665799873e-10\\
12.602994140625	-4.50923852035705e-10\\
12.6231435546875	-4.13129170937459e-10\\
12.64329296875	-4.48497601832813e-10\\
12.6634423828125	-4.09568557759013e-10\\
12.683591796875	-4.39521228272104e-10\\
12.7037412109375	-4.36836793115631e-10\\
12.723890625	-3.73655405462209e-10\\
12.7440400390625	-3.72210753149829e-10\\
12.764189453125	-3.46837729166476e-10\\
12.7843388671875	-4.60870882287728e-10\\
12.80448828125	-4.50013951911467e-10\\
12.8246376953125	-5.17583226219494e-10\\
12.844787109375	-5.20066160317178e-10\\
12.8649365234375	-4.69000449529926e-10\\
12.8850859375	-4.81067236051691e-10\\
12.9052353515625	-5.01668028654386e-10\\
12.925384765625	-4.57996007563446e-10\\
12.9455341796875	-4.17069670874844e-10\\
12.96568359375	-4.03271890034314e-10\\
12.9858330078125	-4.14815411083268e-10\\
13.005982421875	-4.62426863545993e-10\\
13.0261318359375	-4.15116461000048e-10\\
13.04628125	-3.60824560270422e-10\\
13.0664306640625	-3.41225954238166e-10\\
13.086580078125	-3.71443731382264e-10\\
13.1067294921875	-4.5342091523859e-10\\
13.12687890625	-3.63531519679192e-10\\
13.1470283203125	-2.84620496652187e-10\\
13.167177734375	-3.00931776697367e-10\\
13.1873271484375	-2.62602911219724e-10\\
13.2074765625	-3.38334252617099e-10\\
13.2276259765625	-3.82041858588835e-10\\
13.247775390625	-3.24338052735744e-10\\
13.2679248046875	-3.58825432157763e-10\\
13.28807421875	-3.14931807699526e-10\\
13.3082236328125	-2.48964667486196e-10\\
13.328373046875	-1.9948149865888e-10\\
13.3485224609375	-2.47837332711746e-10\\
13.368671875	-1.60029068895596e-10\\
13.3888212890625	-1.762013652465e-10\\
13.408970703125	-1.49385324944624e-10\\
13.4291201171875	-2.62234272392094e-10\\
13.44926953125	-2.54459776735258e-10\\
13.4694189453125	-4.09659085393415e-10\\
13.489568359375	-4.00362455465899e-10\\
13.5097177734375	-4.72301960394494e-10\\
13.5298671875	-5.12004931627666e-10\\
13.5500166015625	-5.58667388172608e-10\\
13.570166015625	-4.11519605758797e-10\\
13.5903154296875	-3.50703204903249e-10\\
13.61046484375	-3.7931224004855e-10\\
13.6306142578125	-3.14474318023003e-10\\
13.650763671875	-3.10974132219149e-10\\
13.6709130859375	-2.51365386569674e-10\\
13.6910625	-2.6699563803782e-10\\
13.7112119140625	-2.95407033470052e-10\\
13.731361328125	-2.48195960231374e-10\\
13.7515107421875	-3.38651562314688e-10\\
13.77166015625	-3.36782128288128e-10\\
13.7918095703125	-2.87050629826728e-10\\
13.811958984375	-3.27174911605502e-10\\
13.8321083984375	-2.7421309996662e-10\\
13.8522578125	-3.51367015769315e-10\\
13.8724072265625	-3.46766521405508e-10\\
13.892556640625	-3.18058063848855e-10\\
13.9127060546875	-3.71692872447913e-10\\
13.93285546875	-3.19018211972922e-10\\
13.9530048828125	-3.61431554756374e-10\\
13.973154296875	-3.20763294275433e-10\\
13.9933037109375	-3.17260457161651e-10\\
14.013453125	-2.7856751499737e-10\\
14.0336025390625	-3.48168443909056e-10\\
14.053751953125	-2.82089864110635e-10\\
14.0739013671875	-2.78475421328141e-10\\
14.09405078125	-1.63847163183394e-10\\
14.1142001953125	-2.63380465080139e-10\\
14.134349609375	-1.81805290258593e-10\\
14.1544990234375	-2.10714307764331e-10\\
14.1746484375	-2.3836705693179e-10\\
14.1947978515625	-2.18682480178587e-10\\
14.214947265625	-1.77151264773819e-10\\
14.2350966796875	-3.02028504625371e-10\\
14.25524609375	-2.06102468852601e-10\\
14.2753955078125	-2.581379454869e-10\\
14.295544921875	-2.5042985826528e-10\\
14.3156943359375	-1.57880523828518e-10\\
14.33584375	-2.26719460936872e-10\\
14.3559931640625	-6.59543921669226e-11\\
14.376142578125	-1.12080827343738e-10\\
14.3962919921875	-1.95209617459629e-10\\
14.41644140625	-1.09682264662197e-10\\
14.4365908203125	-1.45174735508371e-10\\
14.456740234375	-1.52635164786309e-10\\
14.4768896484375	-2.25940967330633e-10\\
14.4970390625	-9.98437273829368e-11\\
14.5171884765625	-1.86368044746705e-10\\
14.537337890625	-1.6284398493916e-10\\
14.5574873046875	-2.2314844876568e-10\\
14.57763671875	-1.07314046104888e-10\\
14.5977861328125	-1.52394508621006e-10\\
14.617935546875	-1.80001207091841e-10\\
14.6380849609375	-1.95092018303935e-10\\
14.658234375	-1.70216547594692e-10\\
14.6783837890625	-9.39168455355079e-11\\
14.698533203125	5.94892030901254e-11\\
14.7186826171875	9.41665490617768e-11\\
14.73883203125	1.64344014704499e-10\\
14.7589814453125	1.28441616659939e-10\\
14.779130859375	6.10308216842403e-11\\
14.7992802734375	1.18151288390066e-11\\
14.8194296875	6.49491446736342e-12\\
14.8395791015625	-1.74528015830592e-10\\
14.859728515625	-3.35158144939929e-10\\
14.8798779296875	-2.57074762725456e-10\\
14.90002734375	-1.56503892676411e-10\\
14.9201767578125	-4.53779550943492e-12\\
14.940326171875	1.53722369726265e-10\\
14.9604755859375	5.04534431550087e-11\\
14.980625	2.66562305897387e-10\\
15.0007744140625	1.01232056608616e-10\\
15.020923828125	1.14943166257856e-10\\
15.0410732421875	7.32237492328962e-11\\
15.06122265625	1.57919937390605e-11\\
15.0813720703125	-2.52020528950256e-13\\
15.101521484375	-5.21557341486415e-12\\
15.1216708984375	-4.63630268066207e-11\\
15.1418203125	-2.78306044897146e-11\\
15.1619697265625	-2.64897553656847e-11\\
15.182119140625	-9.81677282304933e-12\\
15.2022685546875	5.62100192048825e-11\\
15.22241796875	4.83022483703135e-11\\
15.2425673828125	3.86579217148814e-11\\
15.262716796875	8.20760933513432e-11\\
15.2828662109375	7.69089429249167e-11\\
15.303015625	2.32154688686757e-11\\
15.3231650390625	7.74099556392106e-11\\
15.343314453125	-4.09142343489049e-11\\
15.3634638671875	-6.69560062158953e-11\\
15.38361328125	-1.73632239479082e-10\\
15.4037626953125	-1.30306496683057e-10\\
15.423912109375	-5.70446182673168e-11\\
15.4440615234375	-3.15318076812649e-11\\
15.4642109375	1.02074540939936e-10\\
15.4843603515625	5.17560231330152e-11\\
15.504509765625	8.8317343576122e-11\\
15.5246591796875	2.21799063141863e-11\\
15.54480859375	7.06763024402591e-12\\
15.5649580078125	6.60865725449422e-12\\
15.585107421875	-7.17871850251546e-12\\
15.6052568359375	-6.61752200447132e-11\\
15.62540625	1.03688094236991e-10\\
15.6455556640625	1.39136842322815e-10\\
15.665705078125	2.526375150329e-10\\
15.6858544921875	2.51896929059074e-10\\
15.70600390625	2.95863954824026e-10\\
15.7261533203125	1.80606233265079e-10\\
15.746302734375	1.073725493056e-10\\
15.7664521484375	4.10612292109626e-11\\
15.7866015625	-2.05340093075686e-11\\
15.8067509765625	-6.82511179978091e-11\\
15.826900390625	8.68755511827226e-11\\
15.8470498046875	1.18016190755988e-10\\
15.86719921875	2.34446482706198e-10\\
15.8873486328125	2.2613510812919e-10\\
15.907498046875	3.88361994393842e-10\\
15.9276474609375	2.68903729298085e-10\\
15.947796875	3.48261999723803e-10\\
15.9679462890625	2.4204665249822e-10\\
15.988095703125	2.64564552726533e-10\\
16.0082451171875	2.20353267357525e-10\\
16.02839453125	2.09814336470111e-10\\
16.0485439453125	2.63945101205479e-10\\
16.068693359375	2.76340765939918e-10\\
16.0888427734375	2.63967986353759e-10\\
16.1089921875	2.69969012171155e-10\\
16.1291416015625	2.77815643328758e-10\\
16.149291015625	1.64096946152456e-10\\
16.1694404296875	1.94531434971215e-10\\
16.18958984375	-4.38998059096395e-11\\
16.2097392578125	-7.15950837408676e-11\\
16.229888671875	-4.48629446553213e-13\\
16.2500380859375	-1.86083573596977e-10\\
16.2701875	-8.33680559128766e-11\\
16.2903369140625	-1.33409331348815e-10\\
16.310486328125	-3.87586263382376e-11\\
16.3306357421875	7.26580878589156e-11\\
16.35078515625	8.05761214461792e-11\\
16.3709345703125	1.77485239874107e-10\\
16.391083984375	9.4080268669805e-12\\
16.4112333984375	8.6048557120382e-11\\
16.4313828125	-6.58166864253769e-11\\
16.4515322265625	-2.24535698273066e-11\\
16.471681640625	-1.2609309186964e-10\\
16.4918310546875	-2.01121792214945e-10\\
16.51198046875	-1.68879276508245e-10\\
16.5321298828125	-8.05667970890558e-11\\
16.552279296875	-1.51911783443657e-11\\
16.5724287109375	6.00403664277016e-11\\
16.592578125	-1.45331449054945e-10\\
16.6127275390625	-8.95027803573227e-11\\
16.632876953125	-3.25219405284298e-11\\
16.6530263671875	-1.61926762943576e-10\\
16.67317578125	-6.10561796807414e-11\\
16.6933251953125	5.48899776824492e-11\\
16.713474609375	-8.30003928713363e-11\\
16.7336240234375	-3.08705650267899e-11\\
16.7537734375	-5.57165636582876e-11\\
16.7739228515625	-5.25647385025322e-11\\
16.794072265625	-1.13307021714721e-10\\
16.8142216796875	-9.42684712370743e-11\\
16.83437109375	-1.42573129078544e-10\\
16.8545205078125	-1.49991615226398e-10\\
16.874669921875	-1.51145784068041e-10\\
16.8948193359375	-1.43934488268179e-10\\
16.91496875	-4.82767983731566e-11\\
16.9351181640625	-1.55822609211313e-10\\
16.955267578125	-1.21210745565194e-10\\
16.9754169921875	-1.20933218081369e-10\\
16.99556640625	-1.80983201841617e-10\\
17.0157158203125	-1.90035909379637e-10\\
17.035865234375	-1.48925711799037e-10\\
17.0560146484375	-1.53800178586581e-10\\
17.0761640625	-1.82941187960373e-10\\
17.0963134765625	-1.89230329005855e-10\\
17.116462890625	-1.89894378750569e-10\\
17.1366123046875	-2.61192013245106e-10\\
17.15676171875	-2.95630346849904e-10\\
17.1769111328125	-3.58081723579687e-10\\
17.197060546875	-4.46404890265308e-10\\
17.2172099609375	-4.67907948783701e-10\\
17.237359375	-3.95545207940932e-10\\
17.2575087890625	-3.9216109105208e-10\\
17.277658203125	-2.96469533150766e-10\\
17.2978076171875	-3.82096075944299e-10\\
17.31795703125	-3.47841179355061e-10\\
17.3381064453125	-3.20616155761441e-10\\
17.358255859375	-3.01166949641362e-10\\
17.3784052734375	-2.81988709603358e-10\\
17.3985546875	-2.83182356233003e-10\\
17.4187041015625	-3.34807235133185e-10\\
17.438853515625	-5.14341768040927e-10\\
17.4590029296875	-4.1480972390469e-10\\
17.47915234375	-3.1488004314238e-10\\
17.4993017578125	-2.91057490888138e-10\\
17.519451171875	-2.60773237578938e-10\\
17.5396005859375	-2.08369524055745e-10\\
17.55975	-1.784869184275e-10\\
17.5798994140625	-1.87683845758016e-10\\
17.600048828125	-2.28097259302735e-10\\
17.6201982421875	-2.16009947750269e-10\\
17.64034765625	-2.791943951744e-10\\
17.6604970703125	-3.46652014443424e-10\\
17.680646484375	-3.14838956968106e-10\\
17.7007958984375	-3.73549210643314e-10\\
17.7209453125	-3.50287576841799e-10\\
17.7410947265625	-3.23776397247912e-10\\
17.761244140625	-1.38428421090136e-10\\
17.7813935546875	-1.51334544768673e-10\\
17.80154296875	-3.43138608810748e-11\\
17.8216923828125	-2.14687220764563e-11\\
17.841841796875	-1.19439059921671e-11\\
17.8619912109375	2.81096371292375e-11\\
17.882140625	-2.75571482096492e-11\\
17.9022900390625	-5.95671627610989e-11\\
17.922439453125	-9.32437002716517e-11\\
17.9425888671875	-1.39462095544056e-10\\
17.96273828125	-1.97936434099781e-10\\
17.9828876953125	-2.44032359147744e-10\\
18.003037109375	-1.8652720183621e-10\\
18.0231865234375	-1.95961579704787e-10\\
18.0433359375	-1.61801506487061e-10\\
18.0634853515625	-3.68149255817654e-10\\
18.083634765625	-4.25296849805458e-10\\
18.1037841796875	-4.68808502623404e-10\\
18.12393359375	-5.90466204021941e-10\\
18.1440830078125	-5.95295163307574e-10\\
18.164232421875	-4.99224971434662e-10\\
18.1843818359375	-4.97462912970573e-10\\
18.20453125	-4.61806722105399e-10\\
18.2246806640625	-3.98560119263079e-10\\
18.244830078125	-2.85437764770288e-10\\
18.2649794921875	-3.52106859844404e-10\\
18.28512890625	-3.79310598468621e-10\\
18.3052783203125	-3.98707858760148e-10\\
18.325427734375	-4.6598785971213e-10\\
18.3455771484375	-6.26433464647677e-10\\
18.3657265625	-5.19510142782995e-10\\
18.3858759765625	-5.6095811350169e-10\\
18.406025390625	-5.47018011271318e-10\\
18.4261748046875	-5.07123912313299e-10\\
18.44632421875	-3.51555436134089e-10\\
18.4664736328125	-3.81034523578246e-10\\
18.486623046875	-1.80123856864789e-10\\
18.5067724609375	-2.84882375043962e-10\\
18.526921875	-3.58252193368058e-10\\
18.5470712890625	-2.83513190601461e-10\\
18.567220703125	-5.05302898695391e-10\\
18.5873701171875	-4.57158909466318e-10\\
18.60751953125	-4.93951609036748e-10\\
18.6276689453125	-4.62998005972349e-10\\
18.647818359375	-4.04180950496136e-10\\
18.6679677734375	-3.1360806434857e-10\\
18.6881171875	-2.92399255677931e-10\\
18.7082666015625	-2.64461633991763e-10\\
18.728416015625	-1.67120680868724e-10\\
18.7485654296875	-9.01495486643727e-11\\
18.76871484375	-1.05985994902684e-10\\
18.7888642578125	-2.47155753680238e-10\\
18.809013671875	-3.2108503247571e-10\\
18.8291630859375	-3.64690846545067e-10\\
18.8493125	-4.8000744752957e-10\\
18.8694619140625	-4.9941213502039e-10\\
18.889611328125	-5.12618836102719e-10\\
18.9097607421875	-4.12268655441004e-10\\
18.92991015625	-3.57469786103374e-10\\
18.9500595703125	-2.4120279395753e-10\\
18.970208984375	-3.07137959194923e-10\\
18.9903583984375	-2.43119865437976e-10\\
19.0105078125	-2.54222704860214e-10\\
19.0306572265625	-1.87428671521405e-10\\
19.050806640625	-2.61796873721964e-10\\
19.0709560546875	-2.66465297666151e-10\\
19.09110546875	-1.72862777968567e-10\\
19.1112548828125	-1.50028017414765e-10\\
19.131404296875	-9.01373836102737e-11\\
19.1515537109375	-1.4873387908534e-11\\
19.171703125	1.14349852579383e-10\\
19.1918525390625	1.24193425731173e-10\\
19.212001953125	2.25116845344207e-10\\
19.2321513671875	2.83300344682304e-10\\
19.25230078125	1.0427737705754e-10\\
19.2724501953125	-5.58022665638457e-11\\
19.292599609375	-2.18819591885747e-10\\
19.3127490234375	-2.51572204976915e-10\\
19.3328984375	-3.0292639222117e-10\\
19.3530478515625	-1.21573170152292e-10\\
19.373197265625	-5.40643369093341e-11\\
19.3933466796875	5.59384186431625e-11\\
19.41349609375	2.44887442003839e-10\\
19.4336455078125	2.69977858892281e-10\\
19.453794921875	3.08388121303437e-10\\
19.4739443359375	1.82313598458233e-10\\
19.49409375	8.90934696883197e-11\\
19.5142431640625	4.89378998394869e-11\\
19.534392578125	-7.7975951331869e-11\\
19.5545419921875	-7.0646024565099e-11\\
19.57469140625	-4.44944731811127e-11\\
19.5948408203125	-1.12125926633754e-10\\
19.614990234375	-1.08301396537053e-11\\
19.6351396484375	4.2229973743176e-11\\
19.6552890625	9.86868923183584e-11\\
19.6754384765625	1.45751517452331e-10\\
19.695587890625	5.55471108979621e-11\\
19.7157373046875	-3.09925256657386e-11\\
19.73588671875	-1.56972616205227e-11\\
19.7560361328125	-1.92430097158854e-10\\
19.776185546875	-1.36632057953253e-10\\
19.7963349609375	-1.33290202870518e-10\\
19.816484375	-1.3671674873267e-10\\
19.8366337890625	-1.31828782033711e-10\\
19.856783203125	-5.99515386025188e-11\\
19.8769326171875	-3.21945839238307e-11\\
19.89708203125	-1.04028295222426e-10\\
19.9172314453125	-1.77238482720819e-10\\
19.937380859375	-1.8650460988747e-10\\
19.9575302734375	-1.84718918075158e-10\\
19.9776796875	-3.31974389758551e-10\\
19.9978291015625	-2.94847134973904e-10\\
20.017978515625	-2.39132228377315e-10\\
20.0381279296875	-3.45272189790096e-10\\
20.05827734375	-4.01043849234567e-10\\
20.0784267578125	-2.59465997293011e-10\\
20.098576171875	-2.12511797579562e-10\\
20.1187255859375	-2.19169256254837e-10\\
20.138875	-2.64524653219248e-10\\
20.1590244140625	-2.64642253313586e-10\\
20.179173828125	-2.77007308488959e-10\\
20.1993232421875	-2.55214853076757e-10\\
20.21947265625	-3.38117609504679e-10\\
20.2396220703125	-2.58496188334086e-10\\
20.259771484375	-4.14912544690689e-10\\
20.2799208984375	-4.96981469943073e-10\\
20.3000703125	-3.9111320409616e-10\\
20.3202197265625	-4.62749610249792e-10\\
20.340369140625	-4.35171763561487e-10\\
20.3605185546875	-3.61922422206399e-10\\
20.38066796875	-3.82333577903343e-10\\
20.4008173828125	-2.42371688425666e-10\\
20.420966796875	-3.82845076301057e-10\\
20.4411162109375	-2.753473691074e-10\\
20.461265625	-2.9847628741294e-10\\
20.4814150390625	-3.65032005921316e-10\\
20.501564453125	-4.0629763265435e-10\\
20.5217138671875	-3.8481567012133e-10\\
20.54186328125	-4.58918429286373e-10\\
20.5620126953125	-4.02125065711882e-10\\
20.582162109375	-3.55041522787695e-10\\
20.6023115234375	-2.07947468189335e-10\\
20.6224609375	-1.79794772559522e-10\\
20.6426103515625	-9.39042004277396e-11\\
20.662759765625	-4.18040945238317e-11\\
20.6829091796875	-8.7873843079231e-11\\
20.70305859375	-1.42532277658577e-10\\
20.7232080078125	-1.8329640285624e-10\\
20.743357421875	-2.93930883248825e-10\\
20.7635068359375	-3.93093543396166e-10\\
20.78365625	-3.43552216653934e-10\\
20.8038056640625	-3.61467499103037e-10\\
20.823955078125	-4.48796208370225e-10\\
20.8441044921875	-2.9628353267302e-10\\
20.86425390625	-3.19311485803685e-10\\
20.8844033203125	-2.5730305759026e-10\\
20.904552734375	-2.22938666869053e-10\\
20.9247021484375	-5.94711263047202e-11\\
20.9448515625	5.95906536893821e-11\\
20.9650009765625	7.74750073638537e-11\\
20.985150390625	8.6836998417703e-11\\
21.0052998046875	-5.25224192890075e-11\\
21.02544921875	-7.10329074325645e-11\\
21.0455986328125	-1.06079553056289e-10\\
21.065748046875	-2.12110867637414e-10\\
21.0858974609375	-1.78079917385417e-10\\
21.106046875	-2.45078773187437e-10\\
21.1261962890625	-2.20952163250404e-10\\
21.146345703125	-1.89173092911422e-10\\
21.1664951171875	-1.60310853596134e-10\\
21.18664453125	-1.35119421880199e-10\\
21.2067939453125	-6.94064398111674e-11\\
21.226943359375	-2.10795949066605e-10\\
21.2470927734375	-2.31612076153104e-10\\
21.2672421875	-4.23302289893039e-10\\
21.2873916015625	-5.19693394287178e-10\\
21.307541015625	-5.25821907082597e-10\\
21.3276904296875	-5.14263258086313e-10\\
21.34783984375	-4.70989223537359e-10\\
21.3679892578125	-4.22274546084666e-10\\
21.388138671875	-3.53101869979343e-10\\
21.4082880859375	-2.9653301825304e-10\\
21.4284375	-2.56634706555516e-10\\
21.4485869140625	-3.07170013770494e-10\\
21.468736328125	-1.71263860862319e-10\\
21.4888857421875	-2.66702992343303e-10\\
21.50903515625	-3.0240730079214e-10\\
21.5291845703125	-3.27988364687637e-10\\
21.549333984375	-2.91224805982983e-10\\
21.5694833984375	-3.84309758065917e-10\\
21.5896328125	-2.95221134004932e-10\\
21.6097822265625	-3.83646922412838e-10\\
21.629931640625	-3.38772837345586e-10\\
21.6500810546875	-4.18577277618074e-10\\
21.67023046875	-5.60500589767233e-10\\
21.6903798828125	-5.21452142562836e-10\\
21.710529296875	-4.28057674299189e-10\\
21.7306787109375	-4.77473853241608e-10\\
21.750828125	-3.64984532567556e-10\\
21.7709775390625	-5.01689920594525e-10\\
21.791126953125	-4.56726142997867e-10\\
21.8112763671875	-4.56116897768888e-10\\
21.83142578125	-4.63891539258489e-10\\
21.8515751953125	-4.52011728998057e-10\\
21.871724609375	-5.22634584976946e-10\\
21.8918740234375	-4.13460744207989e-10\\
21.9120234375	-3.41305301801301e-10\\
21.9321728515625	-3.03933931906878e-10\\
21.952322265625	-2.14545550540821e-10\\
21.9724716796875	-1.85205942553331e-10\\
21.99262109375	-2.69304934944497e-10\\
22.0127705078125	-3.05127969893539e-10\\
22.032919921875	-4.41741833772907e-10\\
22.0530693359375	-5.06061714129559e-10\\
22.07321875	-6.70301785466327e-10\\
22.0933681640625	-6.78655878458971e-10\\
22.113517578125	-5.94933073460861e-10\\
22.1336669921875	-5.19251924180135e-10\\
22.15381640625	-3.63120747756132e-10\\
22.1739658203125	-2.74574185451407e-10\\
22.194115234375	-7.89420136457316e-11\\
22.2142646484375	9.2876624797105e-11\\
22.2344140625	5.89629463739528e-11\\
22.2545634765625	-3.72536191617624e-11\\
22.274712890625	-1.34253823774676e-10\\
22.2948623046875	-3.45152533593926e-10\\
22.31501171875	-4.83466538003188e-10\\
22.3351611328125	-4.19936123117132e-10\\
22.355310546875	-4.37502826015173e-10\\
22.3754599609375	-2.1735157412922e-10\\
22.395609375	-1.72143005557542e-10\\
22.4157587890625	6.84594184623828e-11\\
22.435908203125	1.88010109106657e-10\\
22.4560576171875	2.04049617330049e-10\\
22.47620703125	2.27687403229491e-10\\
22.4963564453125	1.21675212031493e-10\\
22.516505859375	1.21098334871432e-11\\
22.5366552734375	-3.06400737365247e-11\\
22.5568046875	-1.60545091702904e-11\\
22.5769541015625	-7.26320473246592e-11\\
22.597103515625	-3.56577829847218e-11\\
22.6172529296875	4.21895967800568e-11\\
22.63740234375	1.42637905129425e-10\\
22.6575517578125	1.68149187464812e-10\\
22.677701171875	1.37625957917721e-10\\
22.6978505859375	1.38673942967301e-10\\
22.718	1.30935751500604e-10\\
22.7381494140625	1.19120143693249e-10\\
22.758298828125	6.95337567416919e-11\\
22.7784482421875	1.63360138332754e-10\\
22.79859765625	1.38382587221706e-10\\
22.8187470703125	6.02172650075542e-11\\
22.838896484375	2.67386398184551e-10\\
22.8590458984375	2.06386638554576e-10\\
22.8791953125	3.2072140082762e-10\\
22.8993447265625	2.331562602714e-10\\
22.919494140625	1.82952095649557e-10\\
22.9396435546875	2.41436806617873e-10\\
22.95979296875	1.68906553207116e-10\\
22.9799423828125	2.35271172952567e-10\\
23.000091796875	2.0495940087515e-10\\
23.0202412109375	1.38109852831445e-10\\
23.040390625	9.13161923082709e-11\\
23.0605400390625	8.9130586618614e-11\\
23.080689453125	9.50915353671866e-11\\
23.1008388671875	1.55454799042646e-10\\
23.12098828125	7.74725087574087e-11\\
23.1411376953125	1.34461942229782e-10\\
23.161287109375	4.75072987533216e-11\\
23.1814365234375	1.55007262530758e-10\\
23.2015859375	2.20966886356195e-10\\
23.2217353515625	2.46995025405267e-10\\
23.241884765625	9.35224365423303e-11\\
23.2620341796875	1.86416656119242e-10\\
23.28218359375	2.68403489535197e-10\\
23.3023330078125	1.55860343369339e-10\\
23.322482421875	2.68582378567515e-10\\
23.3426318359375	2.55351202704643e-10\\
23.36278125	2.59161699327703e-10\\
23.3829306640625	2.30418470203174e-10\\
23.403080078125	2.36015795371779e-10\\
23.4232294921875	1.31086169340017e-10\\
23.44337890625	1.34932901325285e-10\\
23.4635283203125	1.12057067403978e-10\\
23.483677734375	2.2091403799346e-10\\
23.5038271484375	2.10053749285672e-10\\
23.5239765625	2.6159658664382e-10\\
23.5441259765625	2.98560623533273e-10\\
23.564275390625	3.32717641515253e-10\\
23.5844248046875	3.09946455731221e-10\\
23.60457421875	2.84537031600745e-10\\
23.6247236328125	2.26039339319448e-10\\
23.644873046875	1.97589989232712e-10\\
23.6650224609375	1.92294358608345e-10\\
23.685171875	2.28342456996681e-10\\
23.7053212890625	1.9360051196696e-10\\
23.725470703125	3.18030970738134e-10\\
23.7456201171875	3.81262820324736e-10\\
23.76576953125	4.04781616405973e-10\\
23.7859189453125	4.96684915780497e-10\\
23.806068359375	4.16429802071018e-10\\
23.8262177734375	3.84668573991701e-10\\
23.8463671875	2.83877784972181e-10\\
23.8665166015625	2.48246742964142e-10\\
23.886666015625	2.75654329338989e-10\\
23.9068154296875	1.48902115576404e-10\\
23.92696484375	1.93872286090894e-10\\
23.9471142578125	2.72373585234092e-10\\
23.967263671875	2.97321368700767e-10\\
23.9874130859375	4.28196980957806e-10\\
24.0075625	4.34277703442418e-10\\
24.0277119140625	3.28001197679856e-10\\
24.047861328125	2.7558825503254e-10\\
24.0680107421875	2.76611804015434e-10\\
24.08816015625	2.26120452119502e-10\\
24.1083095703125	2.40161842878314e-10\\
24.128458984375	1.69450988926326e-10\\
24.1486083984375	1.08133243202842e-10\\
24.1687578125	1.64850176592101e-10\\
24.1889072265625	2.9872932125971e-10\\
24.209056640625	2.70185685188364e-10\\
24.2292060546875	3.06667778893843e-10\\
24.24935546875	3.19505706708359e-10\\
24.2695048828125	3.15825063990731e-10\\
24.289654296875	2.30184791469642e-10\\
24.3098037109375	2.01274600913986e-10\\
24.329953125	1.6038030717018e-10\\
24.3501025390625	2.24404918166502e-10\\
24.370251953125	2.25796497612182e-10\\
24.3904013671875	2.02320575877332e-10\\
24.41055078125	2.61246151680984e-10\\
24.4307001953125	2.78554881097923e-10\\
24.450849609375	2.45460410209769e-10\\
24.4709990234375	2.43179655909415e-10\\
24.4911484375	8.49696899104869e-11\\
24.5112978515625	6.59770341687149e-11\\
24.531447265625	5.31122316619713e-11\\
24.5515966796875	6.87441644465412e-12\\
24.57174609375	1.0814155185305e-10\\
24.5918955078125	1.24738781745059e-10\\
24.612044921875	2.07948642440131e-10\\
24.6321943359375	1.76369159077076e-10\\
24.65234375	2.82338363799029e-10\\
24.6724931640625	2.0460466077932e-10\\
24.692642578125	1.17419742217932e-10\\
24.7127919921875	2.46861279662984e-11\\
24.73294140625	-4.8807358317399e-11\\
24.7530908203125	-1.33278408608761e-10\\
24.773240234375	-1.12878398512967e-10\\
24.7933896484375	-1.69345451132346e-10\\
24.8135390625	1.33201812881564e-11\\
24.8336884765625	5.05616450884778e-12\\
24.853837890625	1.27922266839e-10\\
24.8739873046875	1.28914762194709e-10\\
24.89413671875	1.40248684327618e-10\\
24.9142861328125	-7.09874147172022e-12\\
24.934435546875	-1.20292098790365e-10\\
24.9545849609375	-2.35775265284645e-10\\
24.974734375	-3.71160687262343e-10\\
24.9948837890625	-5.62853245751289e-10\\
25.015033203125	-4.46897512545796e-10\\
25.0351826171875	-3.21080054760104e-10\\
25.05533203125	-2.27114356486241e-10\\
25.0754814453125	-1.94468488885091e-10\\
25.095630859375	-1.14452875387632e-10\\
25.1157802734375	-3.41167466052115e-11\\
25.1359296875	-7.25280379235509e-11\\
25.1560791015625	-1.51293549581312e-10\\
25.176228515625	-1.22634578168494e-10\\
25.1963779296875	-2.55774892076165e-10\\
25.21652734375	-2.70162039103706e-10\\
25.2366767578125	-3.35956642656619e-10\\
25.256826171875	-3.38894352314892e-10\\
25.2769755859375	-2.10312483479689e-10\\
25.297125	-3.01085405249448e-10\\
25.3172744140625	-1.11760462986446e-10\\
25.337423828125	-9.57866715675104e-11\\
25.3575732421875	-2.10376315884341e-10\\
25.37772265625	-1.11960885044217e-10\\
25.3978720703125	-3.02847579321563e-10\\
25.418021484375	-2.98308291435267e-10\\
25.4381708984375	-3.16219276562098e-10\\
25.4583203125	-3.72987618287543e-10\\
25.4784697265625	-2.93591871556601e-10\\
25.498619140625	-2.95946851567571e-10\\
25.5187685546875	-2.73760048662989e-10\\
25.53891796875	-1.86554198719498e-10\\
25.5590673828125	-2.24132716334608e-10\\
25.579216796875	-1.92525661244787e-10\\
25.5993662109375	-1.84319730244792e-10\\
25.619515625	-1.49218760666811e-10\\
25.6396650390625	-1.56631022973247e-10\\
25.659814453125	-1.59180494692287e-10\\
25.6799638671875	-1.74268056740461e-10\\
25.70011328125	-1.54915454702913e-10\\
25.7202626953125	-1.05014932066663e-10\\
25.740412109375	-1.38842766374243e-10\\
25.7605615234375	-8.35901564317299e-11\\
25.7807109375	-1.96217051988353e-10\\
25.8008603515625	-1.44473843139582e-10\\
25.821009765625	-1.79281839280997e-10\\
25.8411591796875	-2.16052454805941e-10\\
25.86130859375	-2.94961875443831e-10\\
25.8814580078125	-4.29974683746028e-10\\
25.901607421875	-3.58724109883043e-10\\
25.9217568359375	-3.14293110496184e-10\\
25.94190625	-2.11504490799316e-10\\
25.9620556640625	-1.74820588117899e-10\\
25.982205078125	-1.80583998859805e-10\\
26.0023544921875	-2.07756773778413e-10\\
26.02250390625	-9.79383745535507e-11\\
26.0426533203125	-1.86471321753716e-10\\
26.062802734375	-2.42933529852648e-10\\
26.0829521484375	-3.55585320303197e-10\\
26.1031015625	-3.85709759700943e-10\\
26.1232509765625	-4.30513706584067e-10\\
26.143400390625	-2.74122852693033e-10\\
26.1635498046875	-2.56189163786877e-10\\
26.18369921875	-1.93864482376911e-10\\
26.2038486328125	-1.11520862250917e-10\\
26.223998046875	-1.42232491365418e-10\\
26.2441474609375	-7.27032676503768e-11\\
26.264296875	-5.07956123626472e-11\\
26.2844462890625	-2.02176797930009e-10\\
26.304595703125	-2.95588819970538e-10\\
26.3247451171875	-3.38005431074895e-10\\
26.34489453125	-4.44536315750838e-10\\
26.3650439453125	-4.01801187446405e-10\\
26.385193359375	-3.62792683474493e-10\\
26.4053427734375	-3.23493344608127e-10\\
26.4254921875	-2.88961278000097e-10\\
26.4456416015625	-2.4855788946404e-10\\
26.465791015625	-2.41890824118724e-10\\
26.4859404296875	-1.81915057716032e-10\\
26.50608984375	-2.53923879821699e-10\\
26.5262392578125	-2.70284213393392e-10\\
26.546388671875	-3.58207338310938e-10\\
26.5665380859375	-2.44563919739719e-10\\
26.5866875	-3.17560396352845e-10\\
26.6068369140625	-3.33156495638213e-10\\
26.626986328125	-2.9754653807045e-10\\
26.6471357421875	-2.62636271637281e-10\\
26.66728515625	-2.4018410708685e-10\\
26.6874345703125	-1.44813618644804e-10\\
26.707583984375	-1.02251384668396e-10\\
26.7277333984375	-2.48627133960963e-10\\
26.7478828125	-2.05416139434719e-10\\
26.7680322265625	-2.33039804857336e-10\\
26.788181640625	-2.40795537801914e-10\\
26.8083310546875	-2.16905367707575e-10\\
26.82848046875	-2.72667985222734e-10\\
26.8486298828125	-2.29851066369729e-10\\
26.868779296875	-1.90293368389179e-10\\
26.8889287109375	-2.11907324306256e-10\\
26.909078125	-5.98129334649867e-11\\
26.9292275390625	-2.60099092952369e-10\\
26.949376953125	-1.77976359668072e-10\\
26.9695263671875	-1.3376531760693e-10\\
26.98967578125	-2.37657670003302e-10\\
27.0098251953125	-1.67528984807783e-10\\
27.029974609375	-1.78270596459595e-10\\
27.0501240234375	-1.10771093479726e-10\\
27.0702734375	-9.17209064764692e-11\\
27.0904228515625	-9.11147058099252e-11\\
27.110572265625	-7.20756456410763e-11\\
27.1307216796875	-1.04244283356529e-11\\
27.15087109375	-3.28567024760406e-11\\
27.1710205078125	-9.91078540736815e-11\\
27.191169921875	-2.06304811430742e-11\\
27.2113193359375	-1.78927033667317e-10\\
27.23146875	1.87999630328541e-11\\
27.2516181640625	-1.35655164513126e-11\\
27.271767578125	-2.67520096265891e-12\\
27.2919169921875	4.66677930717582e-11\\
27.31206640625	1.10191156599843e-10\\
27.3322158203125	7.96673825834398e-11\\
27.352365234375	1.67303180360894e-10\\
27.3725146484375	1.14792698862538e-10\\
27.3926640625	5.62692311243301e-11\\
27.4128134765625	2.55744269845763e-11\\
27.432962890625	3.72230514766449e-11\\
27.4531123046875	-9.43724238362841e-12\\
27.47326171875	-1.16405743117456e-11\\
27.4934111328125	4.99351482152291e-11\\
27.513560546875	1.47917520673822e-10\\
27.5337099609375	6.02994355662401e-11\\
27.553859375	1.42634772042514e-10\\
27.5740087890625	1.43914294862124e-10\\
27.594158203125	2.05674061024699e-10\\
27.6143076171875	2.24626693808576e-10\\
27.63445703125	1.06796994402189e-10\\
27.6546064453125	1.5230569609271e-10\\
27.674755859375	3.62649831628478e-11\\
27.6949052734375	6.70557007277367e-11\\
27.7150546875	2.45602092099475e-11\\
27.7352041015625	1.39780805489773e-10\\
27.755353515625	9.01437361528582e-11\\
27.7755029296875	1.40628960719573e-10\\
27.79565234375	1.01987588957186e-10\\
27.8158017578125	1.92940875620974e-10\\
27.835951171875	9.01180784555819e-11\\
27.8561005859375	5.39938974733799e-11\\
27.87625	1.28645762900556e-10\\
27.8963994140625	8.07583588394994e-11\\
27.916548828125	1.52458586517128e-10\\
27.9366982421875	2.08563619743104e-10\\
27.95684765625	1.57449053499591e-10\\
27.9769970703125	1.68703445614127e-10\\
27.997146484375	2.17844570834394e-10\\
28.0172958984375	2.01898120867739e-10\\
28.0374453125	1.51735066028779e-10\\
28.0575947265625	1.00827268560041e-10\\
28.077744140625	1.47794991749108e-10\\
28.0978935546875	1.78258305753072e-10\\
28.11804296875	3.21838866996009e-10\\
28.1381923828125	2.02259275070307e-10\\
28.158341796875	3.06585143135513e-10\\
28.1784912109375	2.00078427427243e-10\\
28.198640625	2.56507163324774e-10\\
28.2187900390625	2.17418554425354e-10\\
28.238939453125	2.52489354341867e-10\\
28.2590888671875	1.56603284759759e-10\\
28.27923828125	1.34363887359646e-10\\
28.2993876953125	1.87427767943946e-10\\
28.319537109375	1.40534743920737e-10\\
28.3396865234375	2.20934038745161e-10\\
28.3598359375	3.16332429873982e-10\\
28.3799853515625	2.56597495951374e-10\\
28.400134765625	2.53483934151073e-10\\
28.4202841796875	2.70889622560502e-10\\
28.44043359375	3.23405156610034e-10\\
28.4605830078125	2.79175406611201e-10\\
28.480732421875	2.64536094981627e-10\\
28.5008818359375	1.85532562396093e-10\\
28.52103125	1.88868841418613e-10\\
28.5411806640625	2.24483641262292e-10\\
28.561330078125	2.54821361820004e-10\\
28.5814794921875	2.84860757210435e-10\\
28.60162890625	3.29859164551382e-10\\
28.6217783203125	3.32713932659276e-10\\
28.641927734375	3.29727054681978e-10\\
28.6620771484375	2.00493490353572e-10\\
28.6822265625	2.54230564294158e-10\\
28.7023759765625	2.63162420013579e-10\\
28.722525390625	2.28197081140841e-10\\
28.7426748046875	3.16560329302525e-10\\
28.76282421875	2.90645635003567e-10\\
28.7829736328125	2.76193556225296e-10\\
28.803123046875	3.41861681090282e-10\\
28.8232724609375	5.10580267225357e-10\\
28.843421875	4.20898648308319e-10\\
28.8635712890625	5.04442572315423e-10\\
28.883720703125	3.92430195921554e-10\\
28.9038701171875	4.44497407725032e-10\\
28.92401953125	4.6463630131833e-10\\
28.9441689453125	4.74190682693021e-10\\
28.964318359375	6.05014869331657e-10\\
28.9844677734375	4.66223014653682e-10\\
29.0046171875	4.34110927547164e-10\\
29.0247666015625	3.83718248111037e-10\\
29.044916015625	4.3313212841259e-10\\
29.0650654296875	3.76677099040403e-10\\
29.08521484375	4.38640973422112e-10\\
29.1053642578125	4.21506764114391e-10\\
29.125513671875	4.41197941659035e-10\\
29.1456630859375	5.09053013519328e-10\\
29.1658125	5.61715463713062e-10\\
29.1859619140625	5.51034306634493e-10\\
29.206111328125	5.12222210978508e-10\\
29.2262607421875	4.25546410312161e-10\\
29.24641015625	4.47769780481975e-10\\
29.2665595703125	3.93175521793446e-10\\
29.286708984375	4.2135830894663e-10\\
29.3068583984375	3.62027660059199e-10\\
29.3270078125	4.35473289789673e-10\\
29.3471572265625	4.47057870011416e-10\\
29.367306640625	4.69172641903475e-10\\
29.3874560546875	4.99689658705962e-10\\
29.40760546875	5.14765910296206e-10\\
29.4277548828125	3.46303763853954e-10\\
29.447904296875	4.22535402192609e-10\\
29.4680537109375	3.00126755548228e-10\\
29.488203125	2.88762051802038e-10\\
29.5083525390625	2.61714285494751e-10\\
29.528501953125	2.84831876622616e-10\\
29.5486513671875	3.54218276482002e-10\\
29.56880078125	3.24791355040295e-10\\
29.5889501953125	3.2235438541625e-10\\
29.609099609375	2.87163884946451e-10\\
29.6292490234375	2.31437519963714e-10\\
29.6493984375	1.32305062889363e-10\\
29.6695478515625	9.51340019342269e-11\\
29.689697265625	1.46645809628163e-10\\
29.7098466796875	1.32668375299752e-10\\
29.72999609375	9.6486298931482e-11\\
29.7501455078125	1.95873916927354e-10\\
29.770294921875	2.11730214914494e-10\\
29.7904443359375	1.5947470657669e-10\\
29.81059375	1.63884439004922e-10\\
29.8307431640625	1.31080423168888e-10\\
29.850892578125	1.50565091358725e-10\\
29.8710419921875	2.62495485155743e-11\\
29.89119140625	3.46903014894796e-11\\
29.9113408203125	-1.21335450518068e-12\\
29.931490234375	1.13132865813786e-10\\
29.9516396484375	1.51439394005448e-10\\
29.9717890625	1.79286993777659e-10\\
29.9919384765625	2.88158975135627e-10\\
30.012087890625	2.21615637036004e-10\\
30.0322373046875	2.91123834022165e-10\\
30.05238671875	1.41891309025647e-10\\
30.0725361328125	1.85755517287648e-10\\
30.092685546875	1.42204772847938e-10\\
30.1128349609375	6.6379534976707e-11\\
30.132984375	-4.85329753448352e-12\\
30.1531337890625	8.21585732032713e-11\\
30.173283203125	9.4245563139862e-11\\
30.1934326171875	1.75579773890091e-10\\
30.21358203125	2.75625268023603e-10\\
30.2337314453125	2.54762393876996e-10\\
30.253880859375	3.30163926165762e-10\\
30.2740302734375	2.98168214228373e-10\\
30.2941796875	2.47120595933946e-10\\
30.3143291015625	1.37241066476833e-10\\
30.334478515625	7.48458840509639e-11\\
30.3546279296875	1.54287123654911e-10\\
30.37477734375	1.85812654690688e-10\\
30.3949267578125	2.03573837131396e-10\\
30.415076171875	2.80465385174919e-10\\
30.4352255859375	3.28978137322108e-10\\
30.455375	2.19429988882181e-10\\
30.4755244140625	2.56363414149956e-10\\
30.495673828125	1.88915804162294e-10\\
30.5158232421875	9.47293988679164e-11\\
30.53597265625	1.26943719126219e-11\\
30.5561220703125	6.44331920988049e-11\\
30.576271484375	8.04581744563063e-12\\
30.5964208984375	6.12994258382357e-11\\
30.6165703125	2.51575043913861e-11\\
30.6367197265625	5.00290442286259e-11\\
30.656869140625	-2.90331039156016e-11\\
30.6770185546875	4.80474956200391e-11\\
30.69716796875	1.35108570191697e-11\\
30.7173173828125	-1.93321896810504e-11\\
30.737466796875	1.94328688172961e-11\\
30.7576162109375	-2.1807498747619e-11\\
30.777765625	1.44077512096587e-13\\
30.7979150390625	8.10743885060685e-11\\
30.818064453125	9.61418357671459e-11\\
30.8382138671875	1.50365619725992e-10\\
30.85836328125	5.30288190444397e-11\\
30.8785126953125	5.50430440083554e-11\\
30.898662109375	3.18521377369382e-11\\
30.9188115234375	-3.27504931220863e-11\\
30.9389609375	-3.40436840639609e-12\\
30.9591103515625	-6.54871271348663e-11\\
30.979259765625	-2.36432144497632e-11\\
30.9994091796875	-2.20233017407717e-11\\
31.01955859375	-9.01407699856202e-12\\
31.0397080078125	8.93211342363362e-11\\
31.059857421875	-7.02119719856068e-12\\
31.0800068359375	7.93684511905989e-11\\
31.10015625	5.80676651338372e-11\\
31.1203056640625	-2.40071267152858e-11\\
31.140455078125	5.07944123228441e-11\\
31.1606044921875	-3.00309700699463e-11\\
31.18075390625	-2.28404109517619e-11\\
31.2009033203125	-4.02441324716802e-11\\
31.221052734375	-1.24362769072907e-10\\
31.2412021484375	-1.74077962600032e-10\\
31.2613515625	-1.25283810868337e-10\\
31.2815009765625	-1.37683200773168e-10\\
31.301650390625	-3.20572050442401e-11\\
31.3217998046875	-1.209748272338e-10\\
31.34194921875	-2.34210209743468e-12\\
31.3620986328125	-2.76808113187095e-11\\
31.382248046875	-7.6815543587896e-11\\
31.4023974609375	-3.0798777057205e-11\\
31.422546875	-1.41574639002163e-10\\
31.4426962890625	-1.20414198589477e-10\\
31.462845703125	-1.66469936103916e-10\\
31.4829951171875	-2.68207433538738e-10\\
31.50314453125	-3.18863276316063e-10\\
31.5232939453125	-2.73721799616417e-10\\
31.543443359375	-2.35557522143662e-10\\
31.5635927734375	-1.83906805523547e-10\\
31.5837421875	-2.29463611660537e-10\\
31.6038916015625	-8.98452565820821e-11\\
31.624041015625	-9.06268886015925e-11\\
31.6441904296875	-4.83370735562749e-11\\
31.66433984375	-1.6590550339535e-10\\
31.6844892578125	-2.05372434448956e-10\\
31.704638671875	-3.21230501549091e-10\\
31.7247880859375	-4.0143512371825e-10\\
31.7449375	-4.30577216480025e-10\\
31.7650869140625	-3.98039075036507e-10\\
31.785236328125	-3.54773330433725e-10\\
31.8053857421875	-3.20698947805144e-10\\
31.82553515625	-2.29468663241011e-10\\
31.8456845703125	-1.75232502940151e-10\\
31.865833984375	-5.46358094040444e-11\\
31.8859833984375	-9.39217656030894e-12\\
31.9061328125	-9.34541244986921e-11\\
31.9262822265625	-9.20978323707699e-11\\
31.946431640625	-2.3579015431542e-10\\
31.9665810546875	-2.65886903782596e-10\\
31.98673046875	-2.10829031417385e-10\\
32.0068798828125	-2.09325855054474e-10\\
32.027029296875	-1.39324194622299e-10\\
32.0471787109375	-3.77279521011255e-11\\
32.067328125	-2.91345775711388e-11\\
32.0874775390625	-4.24190783463293e-11\\
32.107626953125	-2.96592827895472e-11\\
32.1277763671875	-1.11417134367628e-10\\
32.14792578125	-2.29276259077286e-10\\
32.1680751953125	-2.75743657476948e-10\\
32.188224609375	-3.42711267166408e-10\\
32.2083740234375	-2.32433241237609e-10\\
32.2285234375	-1.18567714559297e-10\\
32.2486728515625	-1.00138380582576e-10\\
32.268822265625	-1.63516262776898e-11\\
32.2889716796875	6.00289442818402e-11\\
32.30912109375	6.23008854861239e-11\\
32.3292705078125	-3.50109129313473e-11\\
32.349419921875	-8.53532208745183e-11\\
32.3695693359375	-1.05746970331857e-10\\
32.38971875	-1.11627440060134e-10\\
32.4098681640625	-1.46098785094782e-10\\
32.430017578125	-1.14917870240613e-10\\
32.4501669921875	-1.06950366632449e-10\\
32.47031640625	-3.66573826474381e-11\\
32.4904658203125	-6.84529957301121e-11\\
32.510615234375	-1.43913074353412e-12\\
32.5307646484375	1.599634934146e-11\\
32.5509140625	-5.90255466360576e-11\\
32.5710634765625	2.77218288470586e-11\\
32.591212890625	-5.87710842429826e-11\\
32.6113623046875	-1.63605816663526e-11\\
32.63151171875	-3.15808943115428e-11\\
32.6516611328125	-6.45315852012945e-11\\
32.671810546875	-4.36507618267979e-11\\
32.6919599609375	5.64163716352208e-11\\
32.712109375	-1.01876187218333e-11\\
32.7322587890625	1.86694580727228e-12\\
32.752408203125	4.86810245231583e-11\\
32.7725576171875	4.17112928549581e-11\\
32.79270703125	4.73172748782518e-11\\
32.8128564453125	2.44029625928584e-12\\
32.833005859375	9.83818633799037e-12\\
32.8531552734375	-6.60677868169969e-11\\
32.8733046875	-9.28209269306619e-12\\
32.8934541015625	2.31031700585136e-11\\
32.913603515625	2.68995030888176e-11\\
32.9337529296875	5.9290167521572e-11\\
32.95390234375	1.08996580232053e-10\\
32.9740517578125	6.97219745424587e-12\\
32.994201171875	4.18489683325575e-11\\
33.0143505859375	5.44437270755331e-11\\
33.0345	4.54242885356033e-11\\
33.0546494140625	-7.612478042936e-12\\
33.074798828125	8.42835354322478e-11\\
33.0949482421875	1.10122860152522e-10\\
33.11509765625	8.92408434532358e-11\\
33.1352470703125	1.4659943976865e-10\\
33.155396484375	1.58518767811209e-10\\
33.1755458984375	1.36921970811381e-10\\
33.1956953125	1.63429515351898e-10\\
33.2158447265625	1.48189828196871e-10\\
33.235994140625	1.40980144006618e-10\\
33.2561435546875	1.93226948863449e-10\\
33.27629296875	2.28384108944143e-10\\
33.2964423828125	2.21050346756588e-10\\
33.316591796875	2.01075258899118e-10\\
33.3367412109375	1.9065913967222e-10\\
33.356890625	1.00365631100455e-10\\
33.3770400390625	1.74135692221534e-10\\
33.397189453125	1.39739715078102e-10\\
33.4173388671875	2.69760662153686e-10\\
33.43748828125	3.12966224273874e-10\\
33.4576376953125	2.51797493673882e-10\\
33.477787109375	3.65365630172001e-10\\
33.4979365234375	2.76966649326811e-10\\
33.5180859375	2.77425576186454e-10\\
33.5382353515625	1.58619137574406e-10\\
33.558384765625	1.85043230668834e-10\\
33.5785341796875	1.5242520636884e-10\\
33.59868359375	1.55378972402952e-10\\
33.6188330078125	1.6642594065372e-10\\
33.638982421875	2.66242728335374e-10\\
33.6591318359375	2.17574778706529e-10\\
33.67928125	3.69615938971725e-10\\
33.6994306640625	3.39010811659109e-10\\
33.719580078125	3.95452633522133e-10\\
33.7397294921875	2.99097251202877e-10\\
33.75987890625	2.190360254805e-10\\
33.7800283203125	2.733942927824e-10\\
33.800177734375	2.49259594404826e-10\\
33.8203271484375	1.57659915582523e-10\\
33.8404765625	2.33904185465835e-10\\
33.8606259765625	2.31385397943631e-10\\
33.880775390625	3.44238511281164e-10\\
33.9009248046875	4.15681307768612e-10\\
33.92107421875	4.37661956631997e-10\\
33.9412236328125	5.09742237931746e-10\\
33.961373046875	4.30666833218192e-10\\
33.9815224609375	3.64228587771062e-10\\
34.001671875	4.31288558230484e-10\\
34.0218212890625	3.48478626857116e-10\\
34.041970703125	3.44796297666355e-10\\
34.0621201171875	2.97305284532803e-10\\
34.08226953125	3.38807472843788e-10\\
34.1024189453125	3.29457432397852e-10\\
34.122568359375	4.13287101882673e-10\\
34.1427177734375	4.21479517872409e-10\\
34.1628671875	4.39104621409019e-10\\
34.1830166015625	4.9251311202815e-10\\
34.203166015625	4.86472406060594e-10\\
34.2233154296875	6.11437452009477e-10\\
34.24346484375	5.09378335781277e-10\\
34.2636142578125	4.46702039845395e-10\\
34.283763671875	4.117399778681e-10\\
34.3039130859375	3.35486474757982e-10\\
34.3240625	3.46749514003515e-10\\
34.3442119140625	3.21212716034827e-10\\
34.364361328125	2.55868703413473e-10\\
34.3845107421875	2.52598948231416e-10\\
34.40466015625	2.84074461773121e-10\\
34.4248095703125	3.78662074587134e-10\\
34.444958984375	3.51220913690622e-10\\
34.4651083984375	3.71608007526534e-10\\
34.4852578125	3.22796803472536e-10\\
34.5054072265625	3.00462802704556e-10\\
34.525556640625	2.02858486173228e-10\\
34.5457060546875	2.06851297853444e-10\\
34.56585546875	2.22925547113281e-10\\
34.5860048828125	2.45051017683381e-10\\
34.606154296875	3.39141753667483e-10\\
34.6263037109375	3.41929210693473e-10\\
34.646453125	3.76091474361166e-10\\
34.6666025390625	3.49622763794791e-10\\
34.686751953125	2.95753092289797e-10\\
34.7069013671875	2.61570439686382e-10\\
34.72705078125	2.5236306396587e-10\\
34.7472001953125	2.08012835122044e-10\\
34.767349609375	1.51885651343325e-10\\
34.7874990234375	1.63473481816362e-10\\
34.8076484375	1.71857120340408e-10\\
34.8277978515625	2.01957802889184e-10\\
34.847947265625	2.3804893765568e-10\\
34.8680966796875	3.06789179416815e-10\\
34.88824609375	3.17795290205647e-10\\
34.9083955078125	3.0017481382014e-10\\
34.928544921875	2.75896786251668e-10\\
34.9486943359375	2.13721572822749e-10\\
34.96884375	2.78610624642016e-10\\
34.9889931640625	1.73514690065703e-10\\
35.009142578125	2.72024835521762e-10\\
35.0292919921875	1.9068918567607e-10\\
35.04944140625	2.57175703939712e-10\\
35.0695908203125	2.27774264656773e-10\\
35.089740234375	3.00888865203074e-10\\
35.1098896484375	2.95530217187366e-10\\
35.1300390625	2.49119051104535e-10\\
35.1501884765625	2.639728264949e-10\\
35.170337890625	2.58452583201069e-10\\
35.1904873046875	2.76794446141917e-10\\
35.21063671875	1.43368175839597e-10\\
35.2307861328125	2.41989374318265e-10\\
35.250935546875	2.46347493750077e-10\\
35.2710849609375	2.82394453793899e-10\\
35.291234375	2.57906086803839e-10\\
35.3113837890625	2.97317389235872e-10\\
35.331533203125	3.12803035993477e-10\\
35.3516826171875	2.77738466552639e-10\\
35.37183203125	2.52218620025149e-10\\
35.3919814453125	2.5270519277037e-10\\
35.412130859375	2.13210377377655e-10\\
35.4322802734375	2.08548455128835e-10\\
};
\addplot [color=mycolor4,solid]
  table[row sep=crcr]{%
35.4322802734375	2.08548455128835e-10\\
35.4524296875	1.57626523016495e-10\\
35.4725791015625	1.79340721740246e-10\\
35.492728515625	1.21424515627835e-10\\
35.5128779296875	3.38458336966072e-11\\
35.53302734375	8.87858594076494e-11\\
35.5531767578125	2.95578479036608e-11\\
35.573326171875	7.25758593452401e-11\\
35.5934755859375	-6.51533084134126e-11\\
35.613625	-1.28861729312484e-10\\
35.6337744140625	-7.309099683147e-11\\
35.653923828125	-9.53687450383912e-11\\
35.6740732421875	-1.21953380393282e-10\\
35.69422265625	-1.26863852705396e-10\\
35.7143720703125	-2.05262032581983e-10\\
35.734521484375	-2.22404866735985e-10\\
35.7546708984375	-1.97638042462664e-10\\
35.7748203125	-6.33425869633105e-11\\
35.7949697265625	-1.05025466480816e-10\\
35.815119140625	-3.55572765935132e-11\\
35.8352685546875	-1.05846099919997e-10\\
35.85541796875	-1.05867630693016e-10\\
35.8755673828125	-5.53297191662778e-11\\
35.895716796875	-1.00984927368753e-10\\
35.9158662109375	-1.8016923816064e-10\\
35.936015625	-2.58428406222563e-10\\
35.9561650390625	-2.57349640617604e-10\\
35.976314453125	-2.33439141168524e-10\\
35.9964638671875	-1.60381159183842e-10\\
36.01661328125	-9.19114066000881e-11\\
36.0367626953125	-3.17950081143995e-11\\
36.056912109375	5.52857253051987e-11\\
36.0770615234375	8.80625254494245e-11\\
36.0972109375	3.80475457330424e-11\\
36.1173603515625	4.5436096679311e-11\\
36.137509765625	-3.38105301379352e-11\\
36.1576591796875	-1.10407548897508e-10\\
36.17780859375	-5.95901989942686e-11\\
36.1979580078125	-7.9355655582966e-11\\
36.218107421875	-2.64957342528049e-11\\
36.2382568359375	1.23154282315941e-10\\
36.25840625	1.59523448137755e-10\\
36.2785556640625	2.89590269167901e-10\\
36.298705078125	2.08817891482227e-10\\
36.3188544921875	1.58059491991885e-10\\
36.33900390625	1.09677742762126e-10\\
36.3591533203125	3.37401577702965e-11\\
36.379302734375	-3.92207121503601e-11\\
36.3994521484375	-3.45026768953598e-11\\
36.4196015625	-1.56306326941698e-10\\
36.4397509765625	-7.19904676813321e-11\\
36.459900390625	8.86900440222398e-12\\
36.4800498046875	1.42964833121323e-10\\
36.50019921875	1.06170958268567e-10\\
36.5203486328125	1.01663096036362e-10\\
36.540498046875	1.37332145904716e-10\\
36.5606474609375	8.43554079157311e-11\\
36.580796875	-1.37264450984043e-11\\
36.6009462890625	-1.22356616466301e-10\\
36.621095703125	-1.86134687085689e-10\\
36.6412451171875	-1.55281427855921e-10\\
36.66139453125	-1.26207100568936e-10\\
36.6815439453125	-1.33971368677509e-10\\
36.701693359375	-4.22233409660961e-12\\
36.7218427734375	2.70374560331873e-11\\
36.7419921875	-5.39913569115318e-12\\
36.7621416015625	-1.24104001263093e-11\\
36.782291015625	9.43958178116577e-11\\
36.8024404296875	-6.93333598756161e-11\\
36.82258984375	-4.06554299828148e-11\\
36.8427392578125	-6.16849284769561e-11\\
36.862888671875	-3.22092636793863e-12\\
36.8830380859375	-2.37529096645044e-11\\
36.9031875	1.66140955878475e-11\\
36.9233369140625	-5.47144621419897e-11\\
36.943486328125	4.93241159969188e-11\\
36.9636357421875	-3.3889500482967e-11\\
36.98378515625	3.70055156652263e-11\\
37.0039345703125	9.52616391298915e-11\\
37.024083984375	1.23925167608637e-10\\
37.0442333984375	1.56947557151352e-10\\
37.0643828125	1.88477440452849e-10\\
37.0845322265625	1.50743143902211e-10\\
37.104681640625	1.75285997125467e-10\\
37.1248310546875	1.57856417181465e-10\\
37.14498046875	9.24202442589652e-11\\
37.1651298828125	1.16364106803353e-10\\
37.185279296875	8.19200037609305e-11\\
37.2054287109375	2.00488070691767e-10\\
37.225578125	1.58512641870517e-10\\
37.2457275390625	2.1096218323835e-10\\
37.265876953125	2.70728300219638e-10\\
37.2860263671875	1.89801929139616e-10\\
37.30617578125	1.6552627623122e-10\\
37.3263251953125	2.19108816956396e-10\\
37.346474609375	2.11467182739736e-10\\
37.3666240234375	1.43991199342973e-10\\
37.3867734375	1.4591056213527e-10\\
37.4069228515625	1.15448627104602e-10\\
37.427072265625	1.05575457983473e-10\\
37.4472216796875	9.90702834958915e-11\\
37.46737109375	4.46230527390174e-11\\
37.4875205078125	6.37635972269817e-11\\
37.507669921875	6.09552003191262e-11\\
37.5278193359375	-3.99945441874871e-12\\
37.54796875	4.88300874964092e-11\\
37.5681181640625	4.449821855569e-11\\
37.588267578125	1.36689103626292e-10\\
37.6084169921875	7.18002124787436e-11\\
37.62856640625	9.82209742482586e-11\\
37.6487158203125	1.35619834314104e-10\\
37.668865234375	1.93108579154261e-10\\
37.6890146484375	1.26067146416366e-10\\
37.7091640625	1.36197470389372e-10\\
37.7293134765625	1.49610066622513e-10\\
37.749462890625	1.47390780776686e-10\\
37.7696123046875	2.22488901028707e-10\\
37.78976171875	1.39366364968048e-10\\
37.8099111328125	1.52097324910034e-10\\
37.830060546875	1.11859116542119e-10\\
37.8502099609375	1.46588167743713e-10\\
37.870359375	1.13676813417376e-10\\
37.8905087890625	1.71786034617043e-10\\
37.910658203125	2.14886427113065e-10\\
37.9308076171875	2.7845768068359e-10\\
37.95095703125	2.3588168403765e-10\\
37.9711064453125	2.42906718121501e-10\\
37.991255859375	2.76058917489603e-10\\
38.0114052734375	2.85672545064574e-10\\
38.0315546875	2.64414673361573e-10\\
38.0517041015625	2.3132017311694e-10\\
38.071853515625	2.44868506745926e-10\\
38.0920029296875	2.83058151361495e-10\\
38.11215234375	3.95733791978521e-10\\
38.1323017578125	4.27319651094385e-10\\
38.152451171875	4.72896426208647e-10\\
38.1726005859375	4.75068753988018e-10\\
38.19275	4.38839637599961e-10\\
38.2128994140625	4.06292834762675e-10\\
38.233048828125	4.34954705436354e-10\\
38.2531982421875	3.8006246560302e-10\\
38.27334765625	3.59806374939931e-10\\
38.2934970703125	3.98345829942918e-10\\
38.313646484375	4.01627527776401e-10\\
38.3337958984375	4.34883054312965e-10\\
38.3539453125	4.37227614413369e-10\\
38.3740947265625	4.86823216468381e-10\\
38.394244140625	4.37944812817453e-10\\
38.4143935546875	3.94936825731129e-10\\
38.43454296875	3.73826935129205e-10\\
38.4546923828125	3.06066428552616e-10\\
38.474841796875	2.83257090049238e-10\\
38.4949912109375	3.34106800744996e-10\\
38.515140625	3.37980915355997e-10\\
38.5352900390625	3.90837750110545e-10\\
38.555439453125	3.85467128216475e-10\\
38.5755888671875	3.94941601774606e-10\\
38.59573828125	4.17645098391514e-10\\
38.6158876953125	3.40441203904723e-10\\
38.636037109375	3.25772796374956e-10\\
38.6561865234375	2.26400060780186e-10\\
38.6763359375	2.39090779691946e-10\\
38.6964853515625	1.26912462873068e-10\\
38.716634765625	1.48369500453191e-10\\
38.7367841796875	1.62727780417095e-10\\
38.75693359375	1.80517948642423e-10\\
38.7770830078125	2.04093093234875e-10\\
38.797232421875	2.05172815339756e-10\\
38.8173818359375	2.31088495363127e-10\\
38.83753125	1.84102613007503e-10\\
38.8576806640625	1.26747812321966e-10\\
38.877830078125	1.51985575160895e-10\\
38.8979794921875	9.26322738132272e-11\\
38.91812890625	5.38742532687608e-11\\
38.9382783203125	1.13293093920417e-10\\
38.958427734375	1.23653199199255e-10\\
38.9785771484375	7.21360415871528e-11\\
38.9987265625	9.95026474520862e-11\\
39.0188759765625	1.37351286865272e-10\\
39.039025390625	1.9908272105609e-10\\
39.0591748046875	1.47245373263786e-10\\
39.07932421875	1.72166519254305e-10\\
39.0994736328125	2.73019979702749e-10\\
39.119623046875	2.22782101818495e-10\\
39.1397724609375	1.46501105423469e-10\\
39.159921875	2.36962425668871e-10\\
39.1800712890625	2.03956557283453e-10\\
39.200220703125	1.68740639284762e-10\\
39.2203701171875	1.69245437460216e-10\\
39.24051953125	1.33931249116785e-10\\
39.2606689453125	1.5432777929573e-10\\
39.280818359375	1.41870170519976e-10\\
39.3009677734375	1.70404026880409e-10\\
39.3211171875	2.07002975683104e-10\\
39.3412666015625	1.88345070823906e-10\\
39.361416015625	2.09138867632138e-10\\
39.3815654296875	1.50390866904942e-10\\
39.40171484375	9.910186310284e-11\\
39.4218642578125	1.64114892624607e-10\\
39.442013671875	8.06173729184155e-11\\
39.4621630859375	1.08705876269486e-10\\
39.4823125	9.01879887470997e-11\\
39.5024619140625	7.64056308739846e-12\\
39.522611328125	3.97978750731111e-11\\
39.5427607421875	4.09714697013056e-11\\
39.56291015625	3.87858933854741e-11\\
39.5830595703125	1.01400475044977e-11\\
39.603208984375	8.25020749384986e-12\\
39.6233583984375	-2.12268110052814e-11\\
39.6435078125	-3.82238746460024e-12\\
39.6636572265625	-1.1395781964179e-12\\
39.683806640625	-2.1519252361858e-11\\
39.7039560546875	-5.61938161105317e-11\\
39.72410546875	-8.08472056507464e-11\\
39.7442548828125	-9.10104471378313e-11\\
39.764404296875	-5.31181824860112e-11\\
39.7845537109375	-6.48117488310118e-11\\
39.804703125	-1.4385435392723e-10\\
39.8248525390625	-1.12070195516336e-10\\
39.845001953125	-1.41736244927888e-10\\
39.8651513671875	-1.0483836650549e-10\\
39.88530078125	-1.04805798107109e-10\\
39.9054501953125	-8.97218134789538e-11\\
39.925599609375	-1.08571010799637e-10\\
39.9457490234375	-1.35409012471024e-11\\
39.9658984375	-6.72660833474468e-12\\
39.9860478515625	-1.07810650770457e-10\\
40.006197265625	-9.67191407374938e-11\\
40.0263466796875	-1.03172506672646e-10\\
40.04649609375	-6.73475441272765e-11\\
40.0666455078125	-1.08874436303357e-10\\
40.086794921875	-1.01366227278909e-10\\
40.1069443359375	-1.67496847119756e-10\\
40.12709375	-1.78147310918662e-10\\
40.1472431640625	-1.08898008185523e-10\\
40.167392578125	-1.94851984646859e-10\\
40.1875419921875	-1.40920361823852e-10\\
40.20769140625	-1.60260450604076e-10\\
40.2278408203125	-1.19939586929248e-10\\
40.247990234375	-1.57510077248839e-10\\
40.2681396484375	-1.29583734213759e-10\\
40.2882890625	-1.75444122269765e-10\\
40.3084384765625	-1.24446813843412e-10\\
40.328587890625	-9.40460369451458e-11\\
40.3487373046875	-1.01785098734416e-10\\
40.36888671875	-8.28287136114835e-11\\
40.3890361328125	-9.60310376864314e-11\\
40.409185546875	-1.05870016700327e-10\\
40.4293349609375	-1.17582078852799e-10\\
40.449484375	-1.93827459839826e-10\\
40.4696337890625	-2.0097923435504e-10\\
40.489783203125	-2.20654899542062e-10\\
40.5099326171875	-2.23859678502846e-10\\
40.53008203125	-1.90574242980513e-10\\
40.5502314453125	-1.66581912145914e-10\\
40.570380859375	-1.71941841286094e-10\\
40.5905302734375	-1.51744803176086e-10\\
40.6106796875	-1.28764390199351e-10\\
40.6308291015625	-2.29999186500405e-10\\
40.650978515625	-1.80740254760885e-10\\
40.6711279296875	-2.44983493432655e-10\\
40.69127734375	-2.61020556858752e-10\\
40.7114267578125	-2.75241574269905e-10\\
40.731576171875	-2.08894634743258e-10\\
40.7517255859375	-2.98324183737178e-10\\
40.771875	-2.07967158865701e-10\\
40.7920244140625	-1.96648727971686e-10\\
40.812173828125	-1.6381710196449e-10\\
40.8323232421875	-1.91380841526816e-10\\
40.85247265625	-2.74865538358329e-10\\
40.8726220703125	-2.67988169153278e-10\\
40.892771484375	-3.04670231331589e-10\\
40.9129208984375	-2.62296924506142e-10\\
40.9330703125	-2.5105562506054e-10\\
40.9532197265625	-1.85763482095423e-10\\
40.973369140625	-1.49658627407388e-10\\
40.9935185546875	-1.99120967295021e-10\\
41.01366796875	-1.72785762704415e-10\\
41.0338173828125	-1.63243541706374e-10\\
41.053966796875	-1.93581451532929e-10\\
41.0741162109375	-2.40829137002706e-10\\
41.094265625	-1.86665110042834e-10\\
41.1144150390625	-2.19543766451504e-10\\
41.134564453125	-8.9479933798566e-11\\
41.1547138671875	-1.04038837831482e-10\\
41.17486328125	-5.17716448376594e-11\\
41.1950126953125	-1.20720170386787e-10\\
41.215162109375	-1.49297127932416e-11\\
41.2353115234375	-4.64781404033107e-12\\
41.2554609375	-1.14728480246081e-10\\
41.2756103515625	-6.14563580893189e-11\\
41.295759765625	-7.41300320820258e-11\\
41.3159091796875	-9.70583757486375e-11\\
41.33605859375	-5.6812758309404e-11\\
41.3562080078125	-9.65020249014508e-13\\
41.376357421875	-4.08097273008921e-11\\
41.3965068359375	-3.62015233795369e-11\\
41.41665625	-3.12104214819438e-11\\
41.4368056640625	3.40157206186405e-11\\
41.456955078125	-4.19477329162248e-11\\
41.4771044921875	3.75525725796756e-11\\
41.49725390625	2.70938534603047e-11\\
41.5174033203125	-5.83318204645277e-11\\
41.537552734375	-4.83988857543613e-11\\
41.5577021484375	-1.87218092722918e-11\\
41.5778515625	-7.62433159499659e-11\\
41.5980009765625	-6.99347071600484e-11\\
41.618150390625	-9.61469388746961e-11\\
41.6382998046875	-1.13714537161963e-10\\
41.65844921875	-1.27185048423457e-10\\
41.6785986328125	-1.51450879572409e-10\\
41.698748046875	-1.69965807050959e-10\\
41.7188974609375	-9.85646816644362e-11\\
41.739046875	-1.0479309873506e-10\\
41.7591962890625	-1.18550555672708e-10\\
41.779345703125	-5.53845339647093e-11\\
41.7994951171875	-1.45906193025374e-10\\
41.81964453125	-1.14007788186897e-10\\
41.8397939453125	-1.29686499795883e-10\\
41.859943359375	-1.38388499667249e-10\\
41.8800927734375	-1.74712236441916e-10\\
41.9002421875	-1.47816788055988e-10\\
41.9203916015625	-9.69159687156636e-11\\
41.940541015625	-1.42062731350629e-10\\
41.9606904296875	-8.78671387271628e-11\\
41.98083984375	-1.08852270703775e-10\\
42.0009892578125	-9.29395367151288e-11\\
42.021138671875	-1.31539293934229e-10\\
42.0412880859375	-9.59052461913093e-11\\
42.0614375	-5.73788393483803e-11\\
42.0815869140625	-9.67340544247148e-11\\
42.101736328125	-4.78925203642373e-11\\
42.1218857421875	-3.09517448700265e-11\\
42.14203515625	-1.16450223984637e-11\\
42.1621845703125	3.39456612472001e-13\\
42.182333984375	1.9685547401862e-11\\
42.2024833984375	5.57091818922012e-11\\
42.2226328125	5.51025636059585e-12\\
42.2427822265625	3.80331626137542e-11\\
42.262931640625	6.09713895731417e-11\\
42.2830810546875	7.24006444698854e-11\\
42.30323046875	5.19219614688455e-11\\
42.3233798828125	4.73137317260688e-11\\
42.343529296875	9.32292107835959e-11\\
42.3636787109375	1.240803779471e-10\\
42.383828125	9.47719251131267e-11\\
42.4039775390625	9.63591967178028e-11\\
42.424126953125	6.55110502350947e-11\\
42.4442763671875	9.02591388974784e-11\\
42.46442578125	1.07559924081936e-10\\
42.4845751953125	8.90337271230411e-11\\
42.504724609375	1.14263921341755e-10\\
42.5248740234375	5.89590835519849e-11\\
42.5450234375	1.02397432397758e-10\\
42.5651728515625	8.70083663291663e-11\\
42.585322265625	1.01277812638113e-10\\
42.6054716796875	1.19254900223693e-10\\
42.62562109375	1.48011797635912e-10\\
42.6457705078125	9.80861746047868e-11\\
42.665919921875	6.86287268543734e-11\\
42.6860693359375	9.24432916661401e-11\\
42.70621875	2.98532339343913e-11\\
42.7263681640625	6.09064538344779e-11\\
42.746517578125	1.15446096401105e-10\\
42.7666669921875	7.30602494233367e-11\\
42.78681640625	-2.43478133164141e-11\\
42.8069658203125	8.77308996780184e-11\\
42.827115234375	7.27799783768693e-11\\
42.8472646484375	6.16875156175629e-11\\
42.8674140625	1.09914975808249e-10\\
42.8875634765625	1.28307406419174e-10\\
42.907712890625	7.28807472474551e-11\\
42.9278623046875	2.86709606056836e-11\\
42.94801171875	6.14110455430421e-11\\
42.9681611328125	1.58818758898246e-10\\
42.988310546875	1.42940983273251e-10\\
43.0084599609375	1.86690413564733e-10\\
43.028609375	1.876665511357e-10\\
43.0487587890625	9.69472022968509e-11\\
43.068908203125	1.86524205356598e-10\\
43.0890576171875	1.57388963941436e-10\\
43.10920703125	1.09468187620841e-10\\
43.1293564453125	1.40175896377552e-10\\
43.149505859375	9.60393474479818e-11\\
43.1696552734375	9.48243493702467e-11\\
43.1898046875	1.64716905383186e-10\\
43.2099541015625	2.03456193213241e-10\\
43.230103515625	1.35186997713433e-10\\
43.2502529296875	2.03637775492343e-10\\
43.27040234375	2.15142107559072e-10\\
43.2905517578125	1.82230820281812e-10\\
43.310701171875	1.77388788041696e-10\\
43.3308505859375	1.563366975399e-10\\
43.351	8.63549812451828e-11\\
43.3711494140625	1.42047286751699e-10\\
43.391298828125	8.76583229253374e-11\\
43.4114482421875	2.35665339860954e-10\\
43.43159765625	1.5470810296236e-10\\
43.4517470703125	2.32390213744178e-10\\
43.471896484375	2.17080300562857e-10\\
43.4920458984375	1.75512518075458e-10\\
43.5121953125	1.8394184334776e-10\\
43.5323447265625	1.71059362089561e-10\\
43.552494140625	1.31075076725084e-10\\
43.5726435546875	1.25750307918657e-10\\
43.59279296875	1.28299931732649e-10\\
43.6129423828125	1.42326377931866e-10\\
43.633091796875	1.57262030705785e-10\\
43.6532412109375	7.7919687013171e-11\\
43.673390625	1.08600382893933e-10\\
43.6935400390625	6.78731284387436e-11\\
43.713689453125	7.26973222020942e-11\\
43.7338388671875	3.93959016582461e-11\\
43.75398828125	1.7071791754651e-11\\
43.7741376953125	9.44064112622782e-11\\
43.794287109375	4.09354185672134e-11\\
43.8144365234375	5.29699333077133e-11\\
43.8345859375	6.8742121610524e-11\\
43.8547353515625	-1.42752649708306e-11\\
43.874884765625	3.2722363949255e-11\\
43.8950341796875	-2.63877241098074e-11\\
43.91518359375	7.78779352098728e-12\\
43.9353330078125	-1.02716231083323e-11\\
43.955482421875	-5.1778808136632e-11\\
43.9756318359375	3.03201480036128e-11\\
43.99578125	2.47789046345349e-11\\
44.0159306640625	-3.76565269850937e-12\\
44.036080078125	1.01899703478542e-10\\
44.0562294921875	1.4001678209012e-10\\
44.07637890625	7.12506807327289e-11\\
44.0965283203125	1.22261576740247e-10\\
44.116677734375	5.17260657274605e-11\\
44.1368271484375	2.40954989084663e-11\\
44.1569765625	5.21048405536816e-11\\
44.1771259765625	3.71767312272628e-11\\
44.197275390625	1.11135355183577e-10\\
44.2174248046875	8.03349467718022e-11\\
44.23757421875	1.57992167857972e-10\\
44.2577236328125	1.61931176955509e-10\\
44.277873046875	2.40384548653091e-10\\
44.2980224609375	1.83709815892321e-10\\
44.318171875	1.28789370523169e-10\\
44.3383212890625	1.03584736585996e-10\\
44.358470703125	1.19273268015522e-10\\
44.3786201171875	5.0800568463856e-11\\
44.39876953125	2.84442892886461e-11\\
44.4189189453125	4.16163449726584e-11\\
44.439068359375	9.51094271986383e-11\\
44.4592177734375	8.40388511630424e-11\\
44.4793671875	1.40622697350757e-10\\
44.4995166015625	1.4404804267532e-10\\
44.519666015625	1.12272245134998e-10\\
44.5398154296875	1.36391046905791e-10\\
44.55996484375	6.32675651380757e-11\\
44.5801142578125	4.02602085684361e-11\\
44.600263671875	5.48641253907737e-11\\
44.6204130859375	-3.05126255471163e-11\\
44.6405625	-3.18094817232965e-11\\
44.6607119140625	2.20573556103073e-11\\
44.680861328125	-5.4632072390153e-12\\
44.7010107421875	-7.55052954921873e-12\\
44.72116015625	-3.53934944503331e-11\\
44.7413095703125	-2.77842667396711e-11\\
44.761458984375	-4.28204037108894e-11\\
44.7816083984375	-1.22969231706758e-10\\
44.8017578125	-1.25956904792055e-10\\
44.8219072265625	-5.3330445802355e-11\\
44.842056640625	-1.60917787426349e-10\\
44.8622060546875	-1.06090263338647e-10\\
44.88235546875	-5.66030510227641e-11\\
44.9025048828125	-6.53483011404463e-11\\
44.922654296875	-2.34378996349518e-11\\
44.9428037109375	-9.66737696637002e-11\\
44.962953125	-3.38488066318802e-11\\
44.9831025390625	-9.4925855225897e-11\\
45.003251953125	-1.22027443203997e-10\\
45.0234013671875	-1.30918410905025e-10\\
45.04355078125	-1.08568081550449e-10\\
45.0637001953125	-1.68605826867737e-10\\
45.083849609375	-1.14429676666146e-10\\
45.1039990234375	-8.52699557293677e-11\\
45.1241484375	-1.43560059086623e-11\\
45.1442978515625	-6.16214038718477e-11\\
45.164447265625	-6.17286135108552e-11\\
45.1845966796875	-4.32300778400579e-11\\
45.20474609375	-7.34494964292674e-11\\
45.2248955078125	-1.02097093000351e-10\\
45.245044921875	-9.7248816587423e-11\\
45.2651943359375	-1.97927863407068e-10\\
45.28534375	-1.62928187231978e-10\\
45.3054931640625	-1.62455700731737e-10\\
45.325642578125	-1.25357126829312e-10\\
45.3457919921875	-1.01343915144415e-10\\
45.36594140625	-6.94845647118238e-11\\
45.3860908203125	-4.61788764324554e-11\\
45.406240234375	-1.08016725208711e-10\\
45.4263896484375	-1.22861649989057e-10\\
45.4465390625	-1.67403389891262e-10\\
45.4666884765625	-1.52760654074667e-10\\
45.486837890625	-2.12252594176725e-10\\
45.5069873046875	-1.69303735940115e-10\\
45.52713671875	-2.18323310805463e-10\\
45.5472861328125	-1.44676398299587e-10\\
45.567435546875	-2.01157459331292e-10\\
45.5875849609375	-1.91951215040081e-10\\
45.607734375	-1.5955266517759e-10\\
45.6278837890625	-1.78744558712466e-10\\
45.648033203125	-1.92841349879459e-10\\
45.6681826171875	-1.52447360868157e-10\\
45.68833203125	-1.94096171524033e-10\\
45.7084814453125	-1.50601208394011e-10\\
45.728630859375	-2.17306076004735e-10\\
45.7487802734375	-2.29936758889897e-10\\
45.7689296875	-2.19480916226825e-10\\
45.7890791015625	-2.43277792675812e-10\\
45.809228515625	-2.6747428790563e-10\\
45.8293779296875	-2.5411147956669e-10\\
45.84952734375	-2.81543851004052e-10\\
45.8696767578125	-2.31634134076085e-10\\
45.889826171875	-2.80329532910264e-10\\
45.9099755859375	-1.77811957019947e-10\\
45.930125	-1.62444987259236e-10\\
45.9502744140625	-2.03842391319128e-10\\
45.970423828125	-1.65472469439667e-10\\
45.9905732421875	-1.90504628026224e-10\\
46.01072265625	-2.46721388714246e-10\\
46.0308720703125	-1.66288300667244e-10\\
46.051021484375	-2.22054809352955e-10\\
46.0711708984375	-1.75475911183861e-10\\
46.0913203125	-1.94775390279682e-10\\
46.1114697265625	-1.96687189700983e-10\\
46.131619140625	-1.18186330521701e-10\\
46.1517685546875	-1.016648117135e-10\\
46.17191796875	-1.5124896676621e-10\\
46.1920673828125	-1.08890610010774e-10\\
46.212216796875	-1.06437841886178e-10\\
46.2323662109375	-8.79336807448482e-11\\
46.252515625	-6.67594525418517e-11\\
46.2726650390625	-1.04435321709162e-10\\
46.292814453125	-1.04706881780073e-10\\
46.3129638671875	-7.84436737053522e-11\\
46.33311328125	-1.20093917744825e-10\\
46.3532626953125	-9.45632307382584e-11\\
46.373412109375	-3.44182383601953e-11\\
46.3935615234375	-7.58322207183939e-11\\
46.4137109375	-1.10097551693826e-11\\
46.4338603515625	-4.37984342343962e-11\\
46.454009765625	-2.77106631709055e-11\\
46.4741591796875	-6.02149523886753e-11\\
46.49430859375	-1.03166492992926e-10\\
46.5144580078125	-1.06933836111689e-10\\
46.534607421875	-6.18438274169824e-11\\
46.5547568359375	-1.85018289062091e-10\\
46.57490625	-1.44542798712725e-10\\
46.5950556640625	-1.66276356134325e-10\\
46.615205078125	-1.50273369843042e-10\\
46.6353544921875	-1.02600472658128e-10\\
46.65550390625	-1.43506303932556e-10\\
46.6756533203125	-1.31365330900711e-10\\
46.695802734375	-1.61579522080825e-10\\
46.7159521484375	-1.45249691428204e-10\\
46.7361015625	-2.03840356074285e-10\\
46.7562509765625	-2.69977871320071e-10\\
46.776400390625	-2.33931976536564e-10\\
46.7965498046875	-1.92231094405866e-10\\
46.81669921875	-1.75555361904471e-10\\
46.8368486328125	-2.5977597662615e-10\\
46.856998046875	-2.57973193958762e-10\\
46.8771474609375	-1.94896406711151e-10\\
46.897296875	-2.52964952051112e-10\\
46.9174462890625	-2.15675278481359e-10\\
46.937595703125	-1.58656752012925e-10\\
46.9577451171875	-2.51352080884651e-10\\
46.97789453125	-1.99490564930579e-10\\
46.9980439453125	-2.1336261585843e-10\\
47.018193359375	-2.38027082135093e-10\\
47.0383427734375	-2.01051368383032e-10\\
47.0584921875	-1.74499677339937e-10\\
47.0786416015625	-1.62949309935923e-10\\
47.098791015625	-1.03797101520995e-10\\
47.1189404296875	-1.59515318166277e-10\\
47.13908984375	-8.96648389413047e-11\\
47.1592392578125	-1.60428146903123e-10\\
47.179388671875	-4.65786107651292e-11\\
47.1995380859375	-7.63367012705624e-11\\
47.2196875	-4.75667069141439e-11\\
47.2398369140625	-4.78884238810894e-11\\
47.259986328125	-7.59553725144309e-11\\
47.2801357421875	-9.23731674337151e-11\\
47.30028515625	-3.34102178568611e-11\\
47.3204345703125	-5.82311986833249e-11\\
47.340583984375	-8.34370601247581e-11\\
47.3607333984375	-7.49196510869111e-12\\
47.3808828125	-1.17832097336615e-10\\
47.4010322265625	-1.63407173769872e-11\\
47.421181640625	1.95013443261289e-11\\
47.4413310546875	-3.57984378151427e-11\\
47.46148046875	-9.54812395786059e-12\\
47.4816298828125	-4.10193920629034e-11\\
47.501779296875	-1.68559635228405e-11\\
47.5219287109375	-9.45040445514424e-12\\
47.542078125	3.22421735306172e-11\\
47.5622275390625	5.07274638562573e-12\\
47.582376953125	-1.54462114509735e-11\\
47.6025263671875	2.34194830494265e-11\\
47.62267578125	-2.13880211474028e-11\\
47.6428251953125	1.36529375554271e-11\\
47.662974609375	-2.46071295478278e-11\\
47.6831240234375	-3.79463908417331e-11\\
47.7032734375	-1.87727162703869e-11\\
47.7234228515625	2.55646313160448e-11\\
47.743572265625	2.26942902662446e-11\\
47.7637216796875	3.33622913226549e-11\\
47.78387109375	6.2926372345274e-11\\
47.8040205078125	6.32754686750439e-11\\
47.824169921875	3.32984232217448e-11\\
47.8443193359375	7.60910737193999e-11\\
47.86446875	-4.97168725843029e-11\\
47.8846181640625	-6.15268827183307e-11\\
47.904767578125	-8.13201515933233e-11\\
47.9249169921875	-3.91757107584316e-11\\
47.94506640625	-3.17957606485383e-11\\
47.9652158203125	2.14171348551855e-11\\
47.985365234375	4.77204486083e-11\\
48.0055146484375	6.15821512461406e-11\\
48.0256640625	7.45009081529063e-11\\
48.0458134765625	1.09938384486081e-10\\
48.065962890625	1.11338241793036e-10\\
48.0861123046875	3.39062354109499e-11\\
48.10626171875	3.2842766800999e-11\\
48.1264111328125	-2.47338283721243e-11\\
48.146560546875	2.47771577725021e-11\\
48.1667099609375	3.9274846577808e-11\\
48.186859375	1.65343732361119e-11\\
48.2070087890625	4.21837197247046e-11\\
48.227158203125	8.58465186272964e-11\\
48.2473076171875	1.6867927913735e-10\\
48.26745703125	1.9915275793967e-10\\
48.2876064453125	1.89694908360581e-10\\
48.307755859375	1.82874188366891e-10\\
48.3279052734375	1.77783578930127e-10\\
48.3480546875	9.45722982612662e-11\\
48.3682041015625	6.85086263609739e-11\\
48.388353515625	7.29030074405203e-11\\
48.4085029296875	3.36515688937264e-11\\
48.42865234375	8.74791349304765e-11\\
48.4488017578125	1.08295317861214e-10\\
48.468951171875	1.62252488089301e-10\\
48.4891005859375	1.31148103499403e-10\\
48.50925	1.7165355622638e-10\\
48.5293994140625	1.60980139876796e-10\\
48.549548828125	9.00104375150376e-11\\
48.5696982421875	1.3145488638499e-10\\
48.58984765625	6.37082210355501e-11\\
48.6099970703125	8.28470627278599e-11\\
48.630146484375	8.12644855544378e-11\\
48.6502958984375	1.71002933121275e-10\\
48.6704453125	1.72056589351305e-10\\
48.6905947265625	2.15152277909438e-10\\
48.710744140625	1.95928683149203e-10\\
48.7308935546875	1.53299125200182e-10\\
48.75104296875	9.91282170555363e-11\\
48.7711923828125	3.26059937627526e-11\\
48.791341796875	2.51377597923369e-11\\
48.8114912109375	3.15274260854771e-11\\
48.831640625	4.29364229760951e-11\\
48.8517900390625	5.5583542034034e-11\\
48.871939453125	8.93451819488178e-11\\
48.8920888671875	1.03150067315483e-10\\
48.91223828125	1.46594583306168e-10\\
48.9323876953125	1.03160255245675e-10\\
48.952537109375	9.96042522040127e-11\\
48.9726865234375	7.54863965214571e-11\\
48.9928359375	5.8147235952189e-11\\
49.0129853515625	-1.03621518736006e-12\\
49.033134765625	-2.43114863073546e-11\\
49.0532841796875	-5.06247673666221e-11\\
49.07343359375	4.18491619413363e-11\\
49.0935830078125	4.13542140381326e-11\\
49.113732421875	1.34733023474172e-10\\
49.1338818359375	1.37757129363058e-10\\
49.15403125	1.68888306841643e-10\\
49.1741806640625	1.86025137346421e-10\\
49.194330078125	1.62527801243254e-10\\
49.2144794921875	8.27869146781047e-11\\
49.23462890625	1.32114600253853e-11\\
49.2547783203125	2.79894218573364e-11\\
49.274927734375	6.14030116296402e-12\\
49.2950771484375	8.457970435816e-12\\
49.3152265625	1.08419882457388e-10\\
49.3353759765625	1.05147917500732e-10\\
49.355525390625	1.53489380305483e-10\\
49.3756748046875	1.73210519015814e-10\\
49.39582421875	1.30613516628868e-10\\
49.4159736328125	2.26806918676774e-10\\
49.436123046875	8.46843978059755e-11\\
49.4562724609375	4.60669120557042e-11\\
49.476421875	1.17569868075338e-10\\
49.4965712890625	8.58603993402807e-11\\
49.516720703125	1.1075729620003e-10\\
49.5368701171875	8.89973924802146e-11\\
49.55701953125	1.56296280285362e-10\\
49.5771689453125	1.80930742955768e-10\\
49.597318359375	1.07456015798053e-10\\
49.6174677734375	1.28129895260447e-10\\
49.6376171875	6.98789907216101e-11\\
49.6577666015625	4.52909588884015e-11\\
49.677916015625	9.73926347685741e-11\\
49.6980654296875	8.33865821840876e-11\\
49.71821484375	1.15877639887173e-10\\
49.7383642578125	2.85931763788223e-11\\
49.758513671875	8.9385617127001e-11\\
49.7786630859375	1.54911346939441e-11\\
49.7988125	-1.48995899371993e-11\\
49.8189619140625	1.19490466670128e-11\\
49.839111328125	2.10522820626413e-11\\
49.8592607421875	-3.13007471430396e-11\\
49.87941015625	-1.17670523703764e-10\\
49.8995595703125	1.86926248768618e-11\\
49.919708984375	-3.70301285284047e-11\\
49.9398583984375	-2.45949846079226e-11\\
49.9600078125	-6.01394965171175e-11\\
49.9801572265625	-2.0298618742516e-11\\
50.000306640625	-7.11022021685221e-11\\
50.0204560546875	-3.4070470418432e-11\\
50.04060546875	-2.06962926542657e-11\\
50.0607548828125	-4.79003911180118e-12\\
50.080904296875	-5.01350432127255e-11\\
50.1010537109375	-4.2838486778643e-11\\
50.121203125	-8.02132472835215e-11\\
50.1413525390625	-4.77659274918513e-11\\
50.161501953125	-6.96760075392383e-11\\
50.1816513671875	-9.28981896374473e-11\\
50.20180078125	-8.11664362903704e-11\\
50.2219501953125	-1.13183305112339e-10\\
50.242099609375	-3.63980424134625e-11\\
50.2622490234375	-3.06438628585e-11\\
50.2823984375	-1.72312322573223e-11\\
50.3025478515625	5.34520392232698e-12\\
50.322697265625	-4.09816978554036e-11\\
50.3428466796875	-9.83725637036585e-11\\
50.36299609375	-3.56377500201986e-11\\
50.3831455078125	-1.00782503518212e-10\\
50.403294921875	-1.209534464855e-10\\
50.4234443359375	-1.43686167869833e-10\\
50.44359375	-1.37674775678574e-10\\
50.4637431640625	-8.05124445434447e-11\\
50.483892578125	-4.60273517085354e-11\\
50.5040419921875	-6.26836523249122e-11\\
50.52419140625	-6.64569481807522e-12\\
50.5443408203125	-1.17654407558972e-11\\
50.564490234375	-2.05659372574996e-11\\
50.5846396484375	-4.81352404137243e-11\\
50.6047890625	-1.98276376141764e-10\\
50.6249384765625	-1.53538354035273e-10\\
50.645087890625	-2.24887879639564e-10\\
50.6652373046875	-2.15510861148416e-10\\
50.68538671875	-1.94023052926998e-10\\
50.7055361328125	-1.86516501063604e-10\\
50.725685546875	-1.71409952618667e-10\\
50.7458349609375	-1.59669631852321e-10\\
50.765984375	-7.21786670651077e-11\\
50.7861337890625	-9.68386527917995e-11\\
50.806283203125	-1.17860522831959e-10\\
50.8264326171875	-1.73407422007028e-10\\
50.84658203125	-1.78110351660904e-10\\
50.8667314453125	-2.14348559969612e-10\\
50.886880859375	-1.82144261485547e-10\\
50.9070302734375	-2.05874478460344e-10\\
50.9271796875	-1.97224439133791e-10\\
50.9473291015625	-2.05703841126632e-10\\
50.967478515625	-9.79913004538909e-11\\
50.9876279296875	-1.57847798096694e-10\\
51.00777734375	-1.16792359720576e-10\\
51.0279267578125	-1.68636848173774e-10\\
51.048076171875	-1.77288566980353e-10\\
51.0682255859375	-1.7985137857083e-10\\
51.088375	-2.31912979205184e-10\\
51.1085244140625	-1.46270748199351e-10\\
51.128673828125	-2.01118806370061e-10\\
51.1488232421875	-1.88289751695562e-10\\
51.16897265625	-1.68705179590638e-10\\
51.1891220703125	-1.06576414715245e-10\\
51.209271484375	-2.08754156296833e-10\\
51.2294208984375	-1.70048718824457e-10\\
51.2495703125	-1.81151334792578e-10\\
51.2697197265625	-2.01788961621852e-10\\
51.289869140625	-1.65446488029716e-10\\
51.3100185546875	-1.79537091562061e-10\\
51.33016796875	-1.77637604875476e-10\\
51.3503173828125	-1.58885232611753e-10\\
51.370466796875	-1.24732019052005e-10\\
51.3906162109375	-1.3668883262377e-10\\
51.410765625	-1.15526325561755e-10\\
51.4309150390625	-1.20337928361094e-10\\
51.451064453125	-1.5866961970241e-10\\
51.4712138671875	-1.91215279235188e-10\\
51.49136328125	-1.96479151418165e-10\\
51.5115126953125	-1.73635061345664e-10\\
51.531662109375	-1.72173230277557e-10\\
51.5518115234375	-1.61093661563657e-10\\
51.5719609375	-1.30866510420798e-10\\
51.5921103515625	-1.07800992691227e-10\\
51.612259765625	-1.15740964983748e-10\\
51.6324091796875	-1.00931170142041e-10\\
51.65255859375	-1.38200529236918e-10\\
51.6727080078125	-1.87224033247745e-10\\
51.692857421875	-1.86237165721728e-10\\
51.7130068359375	-2.53163830709638e-10\\
51.73315625	-2.62403806063535e-10\\
51.7533056640625	-2.48020159344146e-10\\
51.773455078125	-2.12165944202361e-10\\
51.7936044921875	-1.82960942884525e-10\\
51.81375390625	-1.47511151745792e-10\\
51.8339033203125	-1.37055132901354e-10\\
51.854052734375	-1.2512286592591e-10\\
51.8742021484375	-2.13282969745873e-10\\
51.8943515625	-1.65935909490303e-10\\
51.9145009765625	-2.52783756613158e-10\\
51.934650390625	-2.33173816261663e-10\\
51.9547998046875	-1.9272884058452e-10\\
51.97494921875	-1.82882024319188e-10\\
51.9950986328125	-1.99320257410424e-10\\
52.015248046875	-6.96943260648816e-11\\
52.0353974609375	-1.07117734622216e-10\\
52.055546875	-1.46273038641953e-10\\
52.0756962890625	-9.56492201353996e-11\\
52.095845703125	-8.20906662474326e-11\\
52.1159951171875	-1.28891257268031e-10\\
52.13614453125	-1.35260953933929e-10\\
52.1562939453125	-9.43887307761542e-11\\
52.176443359375	-6.39006120793813e-11\\
52.1965927734375	-4.76124591218444e-11\\
52.2167421875	-1.12933155885818e-11\\
52.2368916015625	-6.60144745933914e-11\\
52.257041015625	-1.29165439787861e-10\\
52.2771904296875	-1.00027766442833e-10\\
52.29733984375	-1.452375845647e-10\\
52.3174892578125	-1.37309103200683e-10\\
52.337638671875	-1.23884341046217e-10\\
52.3577880859375	-8.38540344828025e-11\\
52.3779375	-5.39547117374363e-11\\
52.3980869140625	2.83683532667851e-11\\
52.418236328125	-6.31548103308951e-12\\
52.4383857421875	8.39314112076659e-11\\
52.45853515625	4.33491818694137e-11\\
52.4786845703125	-3.14836448095099e-11\\
52.498833984375	5.44703415141017e-12\\
52.5189833984375	-5.44628694611153e-11\\
52.5391328125	-3.78171722568164e-11\\
52.5592822265625	-4.82868940848192e-11\\
52.579431640625	-7.31213012484196e-11\\
52.5995810546875	-4.50775798402044e-11\\
52.61973046875	-6.47682914878453e-12\\
52.6398798828125	-4.31250048891074e-11\\
52.660029296875	2.19025458762684e-11\\
52.6801787109375	-2.15216341557773e-11\\
52.700328125	2.05681210014738e-11\\
52.7204775390625	-7.26514441017868e-11\\
52.740626953125	-9.12743629764658e-11\\
52.7607763671875	-1.70299710532246e-10\\
52.78092578125	-1.83069552821292e-10\\
52.8010751953125	-2.23799450295295e-10\\
52.821224609375	-1.65288265807757e-10\\
52.8413740234375	-1.60434186588593e-10\\
52.8615234375	-8.14985772129379e-11\\
52.8816728515625	-5.00481885141541e-11\\
52.901822265625	1.55188987743677e-11\\
52.9219716796875	-4.13227875489846e-11\\
52.94212109375	-2.81781611099223e-11\\
52.9622705078125	-4.70658746098127e-11\\
52.982419921875	-1.37763365557695e-10\\
53.0025693359375	-1.76743888225594e-10\\
53.02271875	-1.73791131505771e-10\\
53.0428681640625	-1.95040290105108e-10\\
53.063017578125	-2.12397574493212e-10\\
53.0831669921875	-2.15544126576098e-10\\
53.10331640625	-7.78745703157329e-11\\
53.1234658203125	-7.20143616836411e-11\\
53.143615234375	-7.53960192840946e-11\\
53.1637646484375	8.44757385710117e-12\\
53.1839140625	-7.17075620917974e-11\\
53.2040634765625	-1.38757646806171e-11\\
53.224212890625	-8.27398330678658e-11\\
53.2443623046875	-7.00555570296481e-11\\
53.26451171875	-1.32773695200544e-10\\
53.2846611328125	-1.70666257823576e-10\\
53.304810546875	-1.40927366058127e-10\\
53.3249599609375	-1.20467698310684e-10\\
53.345109375	-1.00102962064649e-10\\
53.3652587890625	-8.91320896726376e-11\\
53.385408203125	-8.72508257054589e-11\\
53.4055576171875	1.11843748810608e-11\\
53.42570703125	1.22154182825827e-12\\
53.4458564453125	-5.40238769210968e-12\\
53.466005859375	-3.17352191346575e-11\\
53.4861552734375	-1.79047036961672e-11\\
53.5063046875	-9.4105906485479e-11\\
53.5264541015625	-1.41179074600441e-10\\
53.546603515625	-1.58110613157817e-10\\
53.5667529296875	-1.44456591144783e-10\\
53.58690234375	-8.71412519693034e-11\\
53.6070517578125	-5.81752304455928e-11\\
53.627201171875	-3.32493413396788e-11\\
53.6473505859375	-5.53871960547668e-11\\
53.6675	-4.06965260004178e-11\\
53.6876494140625	-9.43630425511341e-11\\
53.707798828125	-4.88972333666884e-11\\
53.7279482421875	-1.29611834876833e-10\\
53.74809765625	-1.16711433018509e-10\\
53.7682470703125	-3.41621736827235e-11\\
53.788396484375	-3.07247517445734e-11\\
53.8085458984375	-4.97353546139106e-12\\
53.8286953125	-7.44473104803869e-12\\
53.8488447265625	3.37595141284236e-11\\
53.868994140625	3.811850161172e-12\\
53.8891435546875	-1.76111012566555e-11\\
53.90929296875	-5.89861027429436e-11\\
53.9294423828125	-6.94868385897195e-11\\
53.949591796875	-1.66951891185218e-11\\
53.9697412109375	-6.68642217077232e-11\\
53.989890625	2.53774601107085e-11\\
54.0100400390625	4.63716127449092e-11\\
54.030189453125	7.09872306767125e-11\\
54.0503388671875	5.09863936325505e-11\\
54.07048828125	7.29014048170082e-11\\
54.0906376953125	-1.58309129830203e-11\\
54.110787109375	-5.16073040683167e-13\\
54.1309365234375	1.49978451482109e-11\\
54.1510859375	-5.22491281929422e-11\\
54.1712353515625	-5.49390538950068e-11\\
54.191384765625	-5.68739633244454e-11\\
54.2115341796875	-2.77342837759312e-11\\
54.23168359375	-2.12039241575495e-11\\
54.2518330078125	-2.6628477715076e-12\\
54.271982421875	-3.06168401528676e-11\\
54.2921318359375	-1.56958258021227e-11\\
54.31228125	-7.96988757668425e-12\\
54.3324306640625	-1.4344854278763e-11\\
54.352580078125	-3.52800023073854e-11\\
54.3727294921875	-6.26781307930107e-11\\
54.39287890625	-7.83883728661531e-11\\
54.4130283203125	-7.02379011805134e-11\\
54.433177734375	-8.99348959027506e-11\\
54.4533271484375	-9.27493080022483e-11\\
54.4734765625	-8.47795517400158e-11\\
54.4936259765625	-1.00799268786761e-10\\
54.513775390625	-8.11855745262166e-11\\
54.5339248046875	-1.39861172593003e-10\\
54.55407421875	-9.28371063416029e-11\\
54.5742236328125	-9.26241787338701e-11\\
54.594373046875	-1.10279506090335e-10\\
54.6145224609375	-1.4195191509924e-10\\
54.634671875	-1.04728312379698e-10\\
54.6548212890625	-1.15709217691609e-10\\
54.674970703125	-1.06965740339314e-10\\
54.6951201171875	-1.30893075192649e-10\\
54.71526953125	-1.10817064023257e-10\\
54.7354189453125	-1.05130584587961e-10\\
54.755568359375	-6.05862455212137e-11\\
54.7757177734375	-1.35038854264852e-10\\
54.7958671875	-4.53119220537702e-11\\
54.8160166015625	-1.05744931418675e-10\\
54.836166015625	-1.10590608760721e-10\\
54.8563154296875	-9.02171497987019e-11\\
54.87646484375	-8.52346362220016e-11\\
54.8966142578125	-8.6776205718469e-11\\
54.916763671875	-1.31879516521125e-10\\
54.9369130859375	-9.97815389330837e-11\\
54.9570625	-1.1520395749655e-10\\
54.9772119140625	-8.64514059991474e-11\\
54.997361328125	-8.67308190783729e-11\\
55.0175107421875	-1.07748220425506e-10\\
55.03766015625	-6.2034431353196e-11\\
55.0578095703125	-8.25725630237247e-11\\
55.077958984375	-1.15171247433926e-10\\
55.0981083984375	-1.66905233087362e-10\\
55.1182578125	-1.32530698951587e-10\\
55.1384072265625	-1.49422809529877e-10\\
55.158556640625	-7.00260841705445e-11\\
55.1787060546875	-7.19964313081837e-11\\
};
\addlegendentry{$\text{train 8 -> Trondheim}$};

\end{axis}
\end{tikzpicture}%

\caption{Influence lines from figure:\ref{fig:infl_trains} on top of each other}\label{fig:infl_all_trains}
\end{figure}
