Oppsummering av masteroppgave

Bridge Weigh in Motion, BWIM, er et system for å veie kjøretøyer mens de kjører over bruer. To, eller flere, sensorer plassert på undersiden av brua, og registrerer tøyningen ved posisjonen av sensorene. For å finne vekten av akslingene som passerer, trenger man å vite influenslinjene til brua ved sensorene. I tillegg må man finne hastigheten og avstanden mellom akslene til kjøretøyet.

I tilegg til å veie kjøretøyer kan BWIM systemer potensielt brukes til å overvåke bruas helse over tid, og til oppdage svikt av broas elementer.

Influenslinje, virkningslinje, diagram som viser hvordan lastvirkningen på en konstruksjonsdel avhenger av lastens plassering. Når belastningen flyttes langsetter en bærende konstruksjon, f.eks. en bil kjører over en bro, vil spenninger og deformasjoner i konstruksjonens forskjellige deler avhenge av hvor lasten virker. 

Influenslinjene kan finnes gjennom å lage en god 3D modell av brua og simulere at et tog kjører  over. Eller man kan bruke sensorene allerede installert. Hvis man kjører et kalibreringstog over brua, der man kjenner hastigheten, akslingsavstanden og akslingstyngden, kan man gjennom ulike metoder finne influenslinjene. 

BWIM har stort sett tidligere blitt tatt i bruk på veibruer, men forskning og testing av systemet på jernbanebruer er også tatt i bruk.

Min masteroppgave fokuserer på å finne influenslinjene til brua gjennom målt tøyning fra sensorene. For å gjøre dette har jeg brukt programmeringsspråket matlab. Først for å simulere en tøyningshistorie fra et tog over bru, og deretter til å utforme metoder for å beregne influenslinjer for denne tøyningshistorien. Disse metodene har senere blitt tilpasset til å kunne brukes på faktiske data fra togpasseringer over Leirelva jernbanebru. Masteroppgaven min består dermed i hovedsak av å finne optimale inlfluenslinjer for jernbanebruer og BWIM systemer.

Kunnskapene jeg innehar er dermed evnen til å kombinere konstruksjonteknikk og programmering.  